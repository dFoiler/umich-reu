% LTeX: enabled=false

\documentclass{article}
\usepackage[utf8]{inputenc}

\newcommand{\nirpdftitle}{Abelian Extensions}
\usepackage{import}
\inputfrom{.}{nir}
\numberwithin{equation}{section}
\usepackage[backend=biber,
    style=alphabetic,
    sorting=ynt
]{biblatex}
\addbibresource{bib.bib}

\pagestyle{contentpage}

\title{Encoding Cohomology, Classifying Extensions, and Explicit Galois Gerbs}
\author{Nir Elber}
\date{\today}
\usepackage{graphicx}
\setlength{\headheight}{12.0pt}
\lhead{}
\rhead{\textit{ABELIAN EXTENSIONS}}

\begin{document}

\maketitle

\begin{abstract}
	\noindent We use group cohomology to provide some general theory to classify all group extensions of a $ G$-module $A$ in the case of an abelian group $ G$. The main idea is to use a group presentation of $G$ provide a group presentation of the extension using specially chosen elements of $A$. It turns out that this ``encoding'' of the extension into elements of $A$ enjoys a number of good homological properties, which are of separate interest. This machinery is then used to provide explicit group presentations for the various Kottwitz gerbs \cite{kottwitz}, in special cases.
\end{abstract}

\setcounter{tocdepth}{4}
\tableofcontents

\section{Introduction} \label{sec:intro}
% !TEX root = ../abeliangerbs.tex

Given a Galois extension of fields of $L/K$ with Galois group $G$ and an algebraic torus $\mathbb T$, a Galois gerb $\mc E$ of $L/K$ bound by $\mathbb T$ is a group extension
\[0\to\mathbb T(L)\to\mc E\to G\to0.\]
Roughly speaking, the goal of the paper is to provide explicit descriptions of these Galois gerbs---and group extensions in general---by giving $\mc E$ a group presentation.

More specifically, let $L/K$ be a finite Galois extension of global fields with Galois group $G$. In \cite{kottwitz}, Kottwitz defined three global gerbs $\mc E_1$, $\mc E_2$, and $\mc E_3$. The overall goal of this paper is to be able to provide a somewhat explicit description of the group law for $\mc E_3$ in the toy case of $L=\QQ(\zeta_q)$ and $K=\QQ$ for $q$ a prime-power. Along the way, we will develop various tools which suggest that the methods can be feasibly extended beyond this toy case.

To describe the approach, we quickly recall the definitions of $\mc E_1$, $\mc E_2$, and $\mc E_3$; more details are provided in \autoref{sec:globalsetup}. Let $V_F$ denote the set of places of a global field $F$. We begin with the following short exact sequences.
\begin{align*}
	0 \to \ZZ[V_L]_0 \to \ZZ[V_L] \to \ZZ \to 0 \tag{X} \label{eq:X} \\
	0 \to L^\times \to \AA_L^\times \to \AA_L^\times/L^\times \to 0 \tag{A} \label{eq:A}
\end{align*}
Selecting the global fundamental class $\alpha_1(L/K)=u_{L/K}\in\widehat H^2(G,\AA_L^\times/L^\times)$, we may construct the Galois gerb
\[0\to\op{Hom}_\ZZ(\ZZ,\AA_L^\times/L^\times)\to\mc E_1\to G\to0.\]
Similarly, by gluing together local fundamental classes, we may construct a class $\alpha_2(L/K)\in\widehat H^2(G,\AA_L^\times)$ defining the Galois gerb
\[0\to\op{Hom}_\ZZ(\ZZ[V_L],\AA_L^\times)\to\mc E_2\to G\to0.\]
Constructing $\mc E_3$ is more difficult. Denote the set of morphisms of short exact sequences from \autoref{eq:X} to \autoref{eq:A} by $\op{Hom}_\ZZ(X,A)$. It turns out that
% https://q.uiver.app/?q=WzAsNCxbMCwwLCJIXjIoRyxcXG9we0hvbX0oWCxBKSkiXSxbMCwxLCJIXjIoRyxcXG9we0hvbX1fXFxaWihcXFpaLFxcQUFfTF5cXHRpbWVzL0xeXFx0aW1lcykpIl0sWzEsMCwiSF4yKEcsXFxvcHtIb219X1xcWlooXFxaWltWX0xdLFxcQUFfTF5cXHRpbWVzKSkiXSxbMSwxLCJIXjIoRyxcXG9we0hvbX1fXFxaWihcXFpaW1ZfTF0sXFxBQV9MXlxcdGltZXMvTF5cXHRpbWVzKSkiXSxbMCwxXSxbMSwzXSxbMCwyXSxbMiwzXV0=&macro_url=https%3A%2F%2Fraw.githubusercontent.com%2FdFoiler%2Fnotes%2Fmaster%2Fnir.tex
\begin{equation}
	\begin{tikzcd}
		{H^2(G,\op{Hom}_\ZZ(X,A))} & {H^2(G,\op{Hom}_\ZZ(\ZZ[V_L],\AA_L^\times))} \\
		{H^2(G,\op{Hom}_\ZZ(\ZZ,\AA_L^\times/L^\times))} & {H^2(G,\op{Hom}_\ZZ(\ZZ[V_L],\AA_L^\times/L^\times))}
		\arrow[from=1-1, to=2-1]
		\arrow[from=2-1, to=2-2]
		\arrow[from=1-1, to=1-2]
		\arrow[from=1-2, to=2-2]
	\end{tikzcd} \label{eq:magicalpullback}
\end{equation}
is a pull-back square, so we can check that we can construct a unique element $\alpha(L/K)\in H^2(G,\op{Hom}(X,A))$ to project down to $\alpha_1(L/K)\in H^2(G,\op{Hom}_\ZZ(\ZZ,\AA_L^\times/L^\times))$ and $\alpha_2(L/K)\in H^2(G,\op{Hom}_\ZZ(\ZZ[V_L],\AA_L^\times))$. Projecting $\alpha(L/K)$ to $H^2(G,\op{Hom}_\ZZ(\ZZ[V_L]_0,L^\times))$ yields $\alpha_3(L/K)$ and hence the Galois gerb
\[0\to\op{Hom}_\ZZ(\ZZ[V_L]_0,L^\times)\to\mc E_3\to G\to 0.\]

Most of this definition can be turned directly into a computation. For example, a $2$-cocycle representing $\alpha_1(L/K)=u_{L/K}$ is not too hard to construct, especially in our toy case of $\QQ(\zeta_q)/\QQ$. Continuing, finding a representative for $\alpha_2(L/K)$ is simply a matter of constructing local fundamental classes and then gluing them together appropriately. However, it is harder to make the pull-back square of \autoref{eq:magicalpullback}. In short, this requires choosing a representative for $\alpha_2(L/K)$ in such a way to appropriately cohere with our choice of representative for $\alpha_1(L/K)$. This is by far the hardest part of this approach.

\subsection{Overview}
The layout of the paper is as follows. We review some background in \autoref{sec:background}. To write down the group law of a group extension of a group $G$ by a $G$-module $A$ requires being able to easily carry around $2$-cocycles in $Z^2(G,A)$. As such, \autoref{sec:crackpot} is interested in studying how one can, in general, encode cocycles. This section is rather pure homological algebra and is largely of separate interest.

The rest of the paper is interested in abelian groups $G$. In \autoref{sec:general}, we describe a natural way to give a group extension of $G$ by a $G$-module $A$ a group law and use this to provide a classification of group extensions. In \autoref{sec:tuplestudy}, we recast this theory in the more abstract machinery of \autoref{sec:crackpot}.

Having established enough algebra, we turn to executing the above computation. In \autoref{sec:local}, we use the framework provided by \autoref{sec:general} to write down local fundamental classes of abelian extensions. Lastly, \autoref{sec:global} finishes the computation by gluing the local fundamental classes together appropriately to represent $\alpha_2(\QQ(\zeta_p)/\QQ)$ and so also $\alpha_3(\QQ(\zeta_p)/\QQ)$, in our toy case.

\subsection{Acknowledgements}
This research was conducted at the University of Michigan REU during the summer of 2022. The author would especially like to thank his advisors Alexander Bertoloni Meli, Patrick Daniels, and Peter Dillery for their eternal patience and guidance. Without their advice, this project would have been impossible. The author would also like to thank Maxwell Ye for a number of helpful conversations and consistent companionship. Without him, the author would have been left floating adrift and soulless.

\section{Background} \label{sec:background}
% !TEX root = ../abeliangerbs.tex

In this section, we familiarize ourselves with various tools used throughout the paper.

\subsection{Group Cohomology}

\subsubsection{Definitions}
Fix $G$ to be a group. We take a moment to review some of the group cohomology will be used in the paper. There is a unique sequence of functors $H^i(G,-)\colon\mathrm{Mod}_G\to\mathrm{Ab}$ for $i\in\NN$ satisfying the following set of properties.
\begin{itemize}
	\item $H^0(G,-)=\op{Hom}_{\ZZ[G]}(\ZZ,-)=(-)^G$.
	\item $H^i(G,I)=0$ for all $i>1$ and injective modules $I$.
	\item There is a functor taking short exact sequences
	\[0\to A\to B\to C\to 0\]
	of $G$-modules to long exact sequences
	\[0\to H^0(G,A)\to H^0(G,B)\to H^0(G,C)\to H^1(G,A)\to H^1(G,B)\to H^1(G,C)\to H^2(G,A)\to\cdots.\]
\end{itemize}
The functors $H^i(G,-)$ are the cohomology functors. Analogously, there is a unique sequence of functors $H_i(G,-)\colon\mathrm{Mod}_G\to\mathrm{Ab}$ for $i\in\NN$ satisfying the following set of properties.
\begin{itemize}
	\item $H_0(G,-)=\ZZ\otimes_{\ZZ[G]}-$.
	\item $H_i(G,P)=0$ for all $i>1$ and projective modules $P$.
	\item There is a functor taking short exact sequences
	\[0\to A\to B\to C\to 0\]
	of $G$-modules to long exact sequences
	\[\cdots\to H_2(G,C)\to H_1(G,A)\to H_1(G,B)\to H_1(G,C)\to H_0(G,A)\to H_0(G,B)\to H_0(G,C)\to0.\]
\end{itemize}
When $G$ is a finite group, it turns out that we can tie these together by defining Tate cohomology: define for a $G$-module $A$, define
\[\widehat H^i(G,A)\coloneqq\begin{cases}
	H^i(G,A) & i\ge1, \\
	A^G/\im N_G & i=0, \\
	\ker N_G/I_GA & i=-1, \\
	H_{-i-1}(G,A) & i\le-2,
\end{cases}\]
where $N_G\colon A\to A$ is the norm map, and $I_G$ is the kernel of the augmentation map $\varepsilon\colon\ZZ[G]\to G$ sending $\varepsilon\colon g\mapsto1$ for each $g\in G$. Then we have the following.
\begin{theorem}[{\cite[Theorem~3]{atiyah-wall}}]
	Let $TG$ be a finite group. There is a functor taking short exact sequences
	\[0\to A\to B\to C\to 0\]
	of $G$-modules to (very) long exact sequences
	\[\cdots\to\widehat H^{-1}(G,A)\to\widehat H^{-1}(G,B)\to\widehat H^{-1}(G,C)\to\widehat H^0(G,A)\to\widehat H^0(G,B)\to\widehat H^0(G,C)\to\cdots.\]
\end{theorem}
Throughout the paper, we will essentially exclusively assume that $G$ is finite and will thus use Tate cohomology unless explicitly stated otherwise.

\subsubsection{The Bar Resolution} \label{sec:barres}
% To actually compute $H^i(G,A)$ for $i\ge0$, we can use the bar resolution \cite[p.~96]{atiyah-wall}: set $P_i\coloneqq\ZZ[G^{i+1}]$ and define the free resolution of $\ZZ$
% \begin{equation}
% 	\cdots\to P_3\stackrel {d^3}\to P_2\stackrel {d^2}\to P_1\stackrel {d^1}\to P_0\stackrel\varepsilon\to\ZZ\to0 \label{eq:barresolution}
% \end{equation}
% by letting $\varepsilon\colon\ZZ[G]\to\ZZ$ be the augmentation map and $d^{i+1}\colon P^{i+1}\to P_i$ be defined by
% \[d(g_0,\ldots,g_{i+1})\coloneqq\sum_{j=0}^{i+1}(-1)^j(g_0,\ldots,g_{j-1},g_{j+1},\ldots,g_{i+1}).\]
% One can check that \autoref{eq:barresolution} is in fact a free resolution of $\ZZ$, and then we define the complex
% \[0\to\op{Hom}_\ZZ(\ZZ,A)\stackrel{\varepsilon^*}\to\op{Hom}_\ZZ(P_0,A)\stackrel{(d^1)^*}\to\op{Hom}_\ZZ(P_1,A)\stackrel{(d^2)^*}\op{Hom}_\ZZ(P_2,A)\to\cdots.\]
To actually compute group cohomology, one can use the bar resolution. We will not need the full bar resolution for Tate cohomology except in a few circumstances, so we will content ourselves with describing $H^1(G,A)$ and $H^2(G,A)$ for a $G$-module $A$, and we will say explicitly when we need to refer to the full standard complex from \cite{atiyah-wall}.

We have the following definitions.
\begin{definition}
	Fix a group $G$ and $G$-module $A$. Then a \textit{$1$-cocycle} is a function $f\colon G\to A$ satisfying the relation
	\[f(gg')=f(g)+g\cdot f(g')\]
	for each $g,g'\in G$. The set of $1$-cocycles is denoted $Z^1(G,A)$.
\end{definition}
\begin{example}
	Let $G$ act on an abelian group $A$ trivially. Then $Z^1(G,A)=\op{Hom}_\ZZ(G,A)$.
\end{example}
\begin{definition}
	Fix a group $G$ and $G$-module $A$. Then a \textit{$1$-coboundary} is a function $f\colon G\to A$ such that there exists $a\in A$ such that
	\[f(g)=(g-1)\cdot a\]
	for each $g\in G$. The set of $1$-coboundaries is denoted $B^1(G,A)$.
\end{definition}
One can check that $B^1(G,A)\subseteq Z^1(G,A)$, and it is a fact that
\[H^1(G,A)\cong Z^1(G,A)/B^1(G,A).\]
There is a similar description for $H^2(G,A)$.
\begin{definition}
	Fix a group $G$ and $G$-module $A$. Then a \textit{$2$-cocycle} is a function $f\colon G^2\to A$ satisfying the relation
	\[g\cdot f(g',g'')+f(g,g'g'')=f(g,g')+f(gg',g'')\]
	for each $g,g',g''\in G$. The set of $2$-coboundaries is denoted $Z^2(G,A)$.
\end{definition}
\begin{example}
	If $G$ is cyclic of order $n$ generated by $\sigma\in G$, then any $a\in A$ can define the $2$-cocycle
	\[f(\sigma^i,\sigma^j)\coloneqq\floor{\frac{i+j}n}\]
	where $0\le i,j<n$.
\end{example}
\begin{definition}
	Fix a group $G$ and $G$-module $A$. Then a \textit{$2$-coboundary} is a function $f\colon G^2\to A$ such that there exists $b\colon G\to A$ with
	\[f(g,g')=gb(g')-b(gg')+b(g')\]
	for each $g,g'\in G$. The set of $2$-cobounaries is denoted $B^2(G,A)$.
\end{definition}
Again, one can check that $B^2(G,A)\subseteq Z^2(G,A)$, and it is again a fact that
\[H^2(G,A)\cong Z^2(G,A)/B^2(G,A).\]
These descriptions will suffice for us, for the most part.

\subsubsection{Change of Group}
Let $G$ be a finite group and $A$ a $G$-module. Given a morphism $f\colon A\to B$ of $G$-modules, we know that we induce a morphism
\[\widehat H^i(G,f)\colon\widehat H^i(G,A)\to\widehat H^i(G,B)\]
because $\widehat H^i(G,-)$ is a functor. It will benefit us somewhat to be able to change the group here as well.

For most of the paper, we will only need two change-of-group morphisms. Observe that, given a subgroup $H\subseteq G$, we can take a $1$-cocycle $f\colon G\to A$ and restrict it to $f|_H$. Additionally, $1$-coboundaries $G$ restrict to $1$-cobounaries of $H$, so we have induced a morphism
\[\op{Res}\colon H^1(G,A)\to H^1(H,A).\]
A similar story works for defining the map $\op{Res}\colon H^2(G,A)\to H^2(H,A)$, and in fact one can define for all $i\in\ZZ$ a morphism
\[\op{Res}\colon\widehat H^i(G,A)\to\widehat H^i(H,A)\]
by extending the same approach.

Next, fix a normal subgroup $H\subseteq G$. Then given a $G$-module $A$, we see that $A^H$ is a $G/H$-module. Now, we can take a $1$-cocycle $f\colon G/H\to A^H$ and then define the composite
\[G\onto G/H\stackrel f\to A^H\into A\]
to be a $1$-cocycle in $Z^1(G,A)$. As before, this will induce a map
\[\op{Inf}\colon H^1\left(G/H,A^H\right)\to H^1(G,A).\]
And we also get to extend this morphism to all indices $i\in\ZZ$ as
\[\op{Inf}\colon\widehat H^i\left(G/H,A^H\right)\to\widehat H^i(G,A)\]
by extending this construction.

\subsubsection{Cup Products}
Let $G$ be a finite group. Given two $G$-modules $A$ and $B$, we can make $A\otimes_\ZZ B$ a $G$-module by letting $G$ acting diagonally. Now, by \cite[Theorem~4]{atiyah-wall} there is a unique family of cup-product morphisms
\[\cup\colon\widehat H^i(G,A)\otimes_\ZZ\widehat H^j(G,B)\to\widehat H^{i+j}(G,A\otimes_\ZZ B)\]
for all $G$-modules $A,B$ and $i,j\in\ZZ$ satisfying the following.
\begin{enumerate}
	\item The cup products $\cup$ are natural in $A$ and $B$.
	\item The cup product
	\[\cup\colon\widehat H^0(G,A)\otimes_\ZZ\widehat H^0(G,B)\to\widehat H^0(G,A\otimes_\ZZ B)\]
	is the natural one, induced by $A^G\otimes_\ZZ B^G\to(A\otimes_\ZZ B)^G$.
	\item Given an exact sequence
	\[0\to A\to B\to C\to 0\]
	of $G$-modules and a $G$-module $M$ such that
	\[0\to A\otimes_\ZZ M\to B\otimes_\ZZ M\to C\otimes_\ZZ M\to 0\]
	is also exact, our cup product commutes with $\delta$ morphisms in that
	\[(\delta c)\cup m=\delta(c\cup m)\in\widehat H^{i+j+1}(G,A\otimes_\ZZ M)\]
	for $c\in\widehat H^i(G,C)$ and $m\in\widehat H^j(G,M)$.
	\item Given an exact sequence
	\[0\to A\to B\to C\to 0\]
	of $G$-modules and a $G$-module $M$ such that
	\[0\to M\otimes_\ZZ A\to M\otimes_\ZZ B\to M\otimes_\ZZ C\to 0\]
	is also exact, our cup product commutes with $\delta$ morphisms in that
	\[m\cup(\delta c)=\delta(m\cup c)\in\widehat H^{i+j+1}(G,M\otimes_\ZZ A\otimes_\ZZ)\]
	for $m\in\widehat H^i(G,M)$ and $c\in\widehat H^j(G,C)$.
\end{enumerate}
There are explicit formulae for these cup products in terms of the standard resolution in \cite[p.~107]{atiyah-wall}, which we will occasionally reference.

Here are a few properties of cup products which we will use without citation.
\begin{proposition}[{\cite[Proposition~9]{atiyah-wall}}]
	Let $G$ be a finite group and $A$, $B$, and $C$ all $G$-modules. Then, for $a\in\widehat H^i(G,A)$ and $b\in\widehat H^j(G,B)$ and $c\in\widehat H^k(G,C)$, the following are true.
	\begin{itemize}
		\item $(a\cup b)\cup c=a\cup(b\cup c)$, where we have identified $(A\otimes_\ZZ B)\otimes_\ZZ C$ and $A\otimes_\ZZ(B\otimes_\ZZ C)$ in the obvious way.
		\item $a\cup b=(-1)^{ij}(b\cup a)$, where we have identified $A\otimes_\ZZ B$ and $B\otimes_\ZZ A$ in the obvious way.
		\item For a subgroup $H\subseteq G$, we have $\op{Res}(a\cup b)=(\op{Res}a)\cup(\op{Res}b)$.
	\end{itemize}
\end{proposition}

\subsubsection{Periodic Cohomology}
Some results from \autoref{sec:crackpot} will mirror the theory of periodic cohomology, so we take a moment to state the main theorem here. We have the following definition.
\begin{definition}
	A finite group $G$ has \textit{periodic cohomology} if and only if there is a natural isomorphism between the functors $\widehat H^i(G,-)$ and $\widehat H^{i+d}(G,-)$ for some $d\in\ZZ$ for each $i\in\ZZ$.
\end{definition}
Then one can show the following.
\begin{theorem}[{\cite[Theorems~VI.9.1, VI.9.5]{brown-cohomology}}]
	Let $G$ be a finite group. Then the following are equivalent.
	\begin{listalph}
		\item $G$ has periodic cohomology.
		\item There is a nonzero $i\in\ZZ$ such that $\widehat H^0(G,\ZZ)\cong\ZZ/\#G\ZZ$.
		\item There is a nonzero $i\in\ZZ$ and $x\in\widehat H^i(G,\ZZ)$ and $x^\lor\in\widehat H^{-i}(G,\ZZ)$ with
		\[x\cup x^\lor=x^\lor\cup x=[1]\in\widehat H^0(G,\ZZ).\]
		\item For some nonzero $i,d\in\ZZ$ and $x\in\widehat H^i(G,\ZZ)$, we have a natural isomorphism
		\[(x\cup-)\colon\widehat H^i(G,-)\Rightarrow\widehat H^{i+d}(G,\ZZ).\]
		\item All Sylow $p$-subgroups of $G$ are cyclic.
	\end{listalph}
\end{theorem}
\begin{example}
	If $G$ is cyclic of order $n$ generated by $\sigma$, then one can show that $G$ has $2$-periodic cohomology, in that
	\[(\chi\cup-)\colon\widehat H^i(G,-)\Rightarrow\widehat H^{i+2}(G,-)\]
	defines a natural isomorphism, where $\chi\in\widehat H^2(G,\ZZ)$ is represented by the $2$-cocycle
	\[\left(\sigma^i,\sigma^j\right)\mapsto\floor{\frac{i+j}n},\]
	where $0\le i,j<n$.
\end{example}
We will not prove this, but it is useful to note that these periodic cohomology theories all come from cup products and that they can be witnessed by an ``invertible'' element in the cohomology ring $\widehat H^\bullet(G,\ZZ)$. These themes will reoccur.
% state tate cohomology
% cup products
% a little on change of group
% explicit cocycles
% give shapiro's lemma
% induced modules?

\subsection{Group Extensions}
We continue with $G$ as a group and $A$ as a $G$-module. We have the following definition.
\begin{definition}
	Let $G$ be a group and $A$ a $G$-module. A \textit{group extension $\mc E$ of $G$ by $A$} is a short exact sequence
	\[0\to A\stackrel\iota\to\mc E\stackrel\pi\to G\to0\]
	such that any $a\in A$ and $w\in\mc E$ have
	\[\pi(w)\cdot\iota(a)=\iota\left(waw^{-1}\right).\]
\end{definition}
For example, Galois gerbs are group extensions.

An isomorphism of group extensions $\mc E_1\to\mc E_2$ is a morphism of the corresponding short exact sequences, as follows.
% https://q.uiver.app/?q=WzAsMTAsWzAsMCwiMCJdLFsxLDAsIkEiXSxbMiwwLCJcXG1jIEVfMSJdLFszLDAsIkciXSxbNCwwLCIwIl0sWzAsMSwiMCJdLFsxLDEsIkEiXSxbMiwxLCJcXG1jIEVfMiJdLFszLDEsIkciXSxbNCwxLCIwIl0sWzAsMV0sWzEsMl0sWzIsM10sWzMsNF0sWzUsNl0sWzYsN10sWzcsOF0sWzgsOV0sWzIsN10sWzEsNiwiIiwxLHsibGV2ZWwiOjIsInN0eWxlIjp7ImhlYWQiOnsibmFtZSI6Im5vbmUifX19XSxbMyw4LCIiLDEseyJsZXZlbCI6Miwic3R5bGUiOnsiaGVhZCI6eyJuYW1lIjoibm9uZSJ9fX1dXQ==&macro_url=https%3A%2F%2Fraw.githubusercontent.com%2FdFoiler%2Fnotes%2Fmaster%2Fnir.tex
\[\begin{tikzcd}
	0 & A & {\mc E_1} & G & 0 \\
	0 & A & {\mc E_2} & G & 0
	\arrow[from=1-1, to=1-2]
	\arrow[from=1-2, to=1-3]
	\arrow[from=1-3, to=1-4]
	\arrow[from=1-4, to=1-5]
	\arrow[from=2-1, to=2-2]
	\arrow[from=2-2, to=2-3]
	\arrow[from=2-3, to=2-4]
	\arrow[from=2-4, to=2-5]
	\arrow[from=1-3, to=2-3]
	\arrow[Rightarrow, no head, from=1-2, to=2-2]
	\arrow[Rightarrow, no head, from=1-4, to=2-4]
\end{tikzcd}\]
By the Five lemma, all such morphisms must be isomorphisms of short exact sequences, which justify why these are isomorphisms of group extensions.

We have the following classification result.
\begin{theorem}[{\cite[Theorem~IV.3.12]{brown-cohomology}}] \label{thm:classifyextensionscohom}
	Let $G$ be a group and $A$ a $G$-module. Then isomorphism classes of group extensions $\mc E$ of $G$ by $A$ are in bijection with cohomology classes in $H^2(G,A)$.
\end{theorem}
\begin{proof}[Sketch]
	We will describe the maps from $2$-cocycles to group extensions and vice versa; that the maps are well-defined and provided the needed isomorphism are a matter of computation. In one direction, fix a group extension
	\[0\to A\stackrel\iota\to\mc E\stackrel\pi\to G\to0.\]
	Now, choose a set-theoretic lift $s\colon G\to\mc E$ of $\pi$, and it turns out that the function $c\colon G^2\to A$ given by
	\[c(g,h)\coloneqq s(g)s(h)s(gh)^{-1}\]
	defines a $2$-cocycle $c\in Z^2(G,A)$.

	In the other direction, fix a $2$-cocycle $c\in Z^2(G,A)$. Then we build the extension
	\[0\to A\stackrel\iota\to \mc E_c\stackrel\pi\to G\to 0\]
	as follows. As a set, $\mc E_c=A\times G$, with group law defined by
	\[(a,g)(a',g')\coloneqq\big(a+g\cdot a'+c(g,g'),gg'\big).\]
	The identity is $(-c(1,1),1)$. To finish, we define $\pi\colon\mc E_c\to G$ by projection and $\iota\colon A\to\mc E_c$ by $a\mapsto(a-c(1,1),1)$.
\end{proof}
The isomorphism of \autoref{thm:classifyextensionscohom} also behaves well with the functoriality of our cohomology groups. For example, a group homomorphism $\varphi\colon G\to H$ and $G$-module $A$ induces a map $H^2(H,A)\to H^2(G,A)$. On the side of group extensions, given a class $u\in H^2(H,A)$ corresponding to the group extension $\mc E$, we can construct $\mc E'$ corresponding to $\varphi(u)$ by pulling back as follows.
% https://q.uiver.app/?q=WzAsMTAsWzAsMCwiMCJdLFsxLDAsIkEiXSxbMiwwLCJcXG1jIEUiXSxbMSwxLCJBIl0sWzAsMSwiMCJdLFsyLDEsIlxcbWMgRSciXSxbMywwLCJIIl0sWzMsMSwiRyJdLFs0LDAsIjAiXSxbNCwxLCIwIl0sWzcsNiwiXFx2YXJwaGkiLDJdLFsxLDMsIiIsMCx7ImxldmVsIjoyLCJzdHlsZSI6eyJoZWFkIjp7Im5hbWUiOiJub25lIn19fV0sWzUsMiwiIiwyLHsic3R5bGUiOnsiYm9keSI6eyJuYW1lIjoiZGFzaGVkIn19fV0sWzAsMV0sWzEsMl0sWzIsNl0sWzYsOF0sWzQsM10sWzMsNV0sWzUsN10sWzcsOV0sWzUsNiwiIiwyLHsic3R5bGUiOnsibmFtZSI6ImNvcm5lciJ9fV1d&macro_url=https%3A%2F%2Fraw.githubusercontent.com%2FdFoiler%2Fnotes%2Fmaster%2Fnir.tex
\[\begin{tikzcd}
	0 & A & {\mc E} & H & 0 \\
	0 & A & {\mc E'} & G & 0
	\arrow["\varphi"', from=2-4, to=1-4]
	\arrow[Rightarrow, no head, from=1-2, to=2-2]
	\arrow[dashed, from=2-3, to=1-3]
	\arrow[from=1-1, to=1-2]
	\arrow[from=1-2, to=1-3]
	\arrow[from=1-3, to=1-4]
	\arrow[from=1-4, to=1-5]
	\arrow[from=2-1, to=2-2]
	\arrow[from=2-2, to=2-3]
	\arrow[from=2-3, to=2-4]
	\arrow[from=2-4, to=2-5]
	\arrow["\lrcorner"{anchor=center, pos=0.125, rotate=90}, draw=none, from=2-3, to=1-4]
\end{tikzcd}\]
Similarly, a $G$-module homomorphism $f\colon A\to B$ induces a map $H^2(G,A)\to H^2(G,B)$. On the side of group extensions, given a class $u\in H^2(G,A)$ corresponding to the group extension $\mc E$, we can construct $\mc E'$ corresponding to $f(u)$ by pushing out as follows.
% https://q.uiver.app/?q=WzAsMTAsWzAsMCwiMCJdLFsxLDAsIkEiXSxbMiwwLCJcXG1jIEUiXSxbMywwLCJHIl0sWzQsMCwiMCJdLFszLDEsIkciXSxbNCwxLCIwIl0sWzAsMSwiMCJdLFsxLDEsIkIiXSxbMiwxLCJcXG1jIEUnIl0sWzAsMV0sWzEsOCwiZiJdLFsxLDJdLFs4LDldLFsyLDksIiIsMSx7InN0eWxlIjp7ImJvZHkiOnsibmFtZSI6ImRhc2hlZCJ9fX1dLFs5LDEsIiIsMSx7InN0eWxlIjp7Im5hbWUiOiJjb3JuZXIifX1dLFs3LDhdLFsyLDNdLFs5LDVdLFs1LDZdLFszLDRdLFszLDUsIiIsMSx7ImxldmVsIjoyLCJzdHlsZSI6eyJoZWFkIjp7Im5hbWUiOiJub25lIn19fV1d&macro_url=https%3A%2F%2Fraw.githubusercontent.com%2FdFoiler%2Fnotes%2Fmaster%2Fnir.tex
\[\begin{tikzcd}
	0 & A & {\mc E} & G & 0 \\
	0 & B & {\mc E'} & G & 0
	\arrow[from=1-1, to=1-2]
	\arrow["f", from=1-2, to=2-2]
	\arrow[from=1-2, to=1-3]
	\arrow[from=2-2, to=2-3]
	\arrow[dashed, from=1-3, to=2-3]
	\arrow["\lrcorner"{anchor=center, pos=0.125, rotate=180}, draw=none, from=2-3, to=1-2]
	\arrow[from=2-1, to=2-2]
	\arrow[from=1-3, to=1-4]
	\arrow[from=2-3, to=2-4]
	\arrow[from=2-4, to=2-5]
	\arrow[from=1-4, to=1-5]
	\arrow[Rightarrow, no head, from=1-4, to=2-4]
\end{tikzcd}\]

\subsection{Class Field Theory}
For our purposes, class field theory will be used to be able to describe certain cohomology groups associated to local and global fields.

\subsubsection{Local Class Field Theory}
We begin with the local story. Let $L/K$ be a finite Galois extension of  degree $n$ and Galois group $G\coloneqq\op{Gal}(L/K)$. Because we are interested in extensions, we begin with what $H^2\left(G,L^\times\right)$ looks like.
\begin{theorem}[{\cite[Lemma~III.2.2]{milne-cft}}]
	Let $L/K$ be a Galois extension of local fields of degree $n$ and Galois group $G\coloneqq\op{Gal}(L/K)$. Then there is a canonical isomorphism
	\[\op{inv}\colon H^2\left(G,L^\times\right)\to{\textstyle\frac1n}\ZZ/\ZZ.\]
\end{theorem}
The element of $H^2(G,L^\times)$ corresponding to $\frac1n$ deserves a name.
\begin{definition}
	Let $L/K$ be a Galois extension of local fields of degree $n$ and Galois group $G\coloneqq\op{Gal}(L/K)$. Then the \textit{local fundamental class} $u_{L/K}$ is the class in $H^2\left(G,L^\times\right)$ with
	\[\op{inv}u_{L/K}=1/n.\]
\end{definition}
The local fundamental class satisfies a number of good functoriality properties.
\begin{proposition}[{\cite[Lemma~III.2.7]{milne-cft}}] \label{prop:functorialfundclass}
	Let $M/L/K$ be a tower of finite local field extensions where $M/K$ is Galois. Then
	\[\op{Res}u_{M/K}=u_{M/L}.\]
	If $L/K$ is also Galois, then
	\[\op{Inf}u_{L/K}=[M:L]u_{M/K}.\]
\end{proposition}
With the machinery in place, we might as well mention the local Artin reciprocity map.
\begin{theorem}[{\cite[Theorem~III.3.1]{milne-cft}}]
	Let $L/K$ be a finite Galois extension of local fields with Galois group $G$. Then the map
	\[(u_{L/K}\cup-)\colon\widehat H^i(G,\ZZ)\to\widehat H^{i+2}\left(G,L^\times\right)\]
	is an isomorphism for all $i\in\ZZ$.
\end{theorem}
\begin{remark}
	More generally, if $T$ is an algebraic $K$-torus which splits over $L$, then the map
	\[(u_{L/K}\cup-)\colon\widehat H^i(G,X_*(T))\to\widehat H^{i+2}\left(G,L^\times\right)\]
	is an isomorphism for all $i\in\ZZ$; see \cite[Theorem~6.2]{alg-tori}.
\end{remark}

\subsubsection{Global Class Field Theory}
We now turn to the global story. Given a global field $K$, we let $V_K$ denote its set of places.

Let $L/K$ be a finite Galois extension of global fields of degree $n$ and Galois group $G\coloneqq\op{Gal}(L/K)$. To be able to make class field theory, we need to fix the correct objects.
\begin{definition}
	Given a global field $K$, we define the \textit{ring of adel\'es} to be the restricted direct product
	\[\AA_K\coloneqq\prod_{v\in V_K}(K_v,\mathcal O_v).\]
	Namely, we are considering infinite tuples $(a_v)_{v\in V_K}$, where $a_v\in K_v$ for each $v\in V_K$ but $v\in\mathcal O_v$ for all but finitely many $v\in V_K$.
\end{definition}
Observe that there is a natural embedding $K\into\AA_K$ by
\[a\mapsto(a)_{v\in V_K}.\]
This embedding descends to an embedding $K^\times\into\AA_K^\times/K^\times$, which lets us consider the quotient $\AA_K^\times/K^\times$.

It turns out that $\AA_K^\times$ and $\AA_K^\times/K^\times$ are the right objects to study. For example, we have the following result.
\begin{theorem}[{\cite[Proposition~2.5]{milne-cft}}] \label{thm:idelecohom}
	Let $L/K$ be a finite Galois extension of global fields with Galois group $G\coloneqq\op{Gal}(L/K)$. Then the various restrictions define an isomorphism
	\[\widehat H^i(G,\AA_L^\times)\simeq\bigoplus_{u\in V_K}\widehat H^i(G,L_v^\times),\]
	for $i\ge0$, where the $v\in V_L$ is a chosen prime over each $u\in V_K$.
\end{theorem}
We also have a global invariant map.
\begin{theorem}[{\cite[p.~194]{global-cft}}]
	Let $L/K$ be a finite Galois extension of global fields of degree $n$ with Galois group $G\coloneqq\op{Gal}(L/K)$. Then there is a canonical map
	\[\op{inv}=\sum_{v\in V_L}\op{inv}_v\colon H^2(G,\AA_L^\times)\to\QQ/\ZZ\]
	induced by the local invariant maps and \autoref{thm:idelecohom}. This map induces an isomorphism
	\[\op{inv}\colon H^2(G,\AA_L^\times/L^\times)\to{\textstyle\frac1n}\ZZ/\ZZ.\]
\end{theorem}
As before, the canonical generator we chose will be of special interest.
\begin{definition}
	Let $L/K$ be a Galois extension of global fields of degree $n$ with Galois group $G$. Then the \textit{global fundamental class} $u_{L/K}$ is the class in $H^2\left(G,\AA_L^\times/L^\times\right)$ with
	\[\op{inv}u_{L/K}=1/n.\]
\end{definition}
And, for fun, here is our global Artin reciprocity map.
\begin{theorem}[{\cite[p.~197]{global-cft}}]
	Let $L/K$ be a Galois extension of global fields of degree $n$ with Galois group $G$. Then the map
	\[(u_{L/K}\cup-)\colon\widehat H^i(G,\ZZ)\to\widehat H^{i+2}\left(G,\AA_L^\times/L^\times\right)\]
	is an isomorphism for all $i\in\ZZ$.
\end{theorem}

\subsection{The Kottwitz Gerbs} \label{sec:globalsetup}
We quickly recall the construction of the Kottwitz gerbs $\mathcal E_1$, $\mathcal E_2$, and $\mathcal E_3$. Given a global field $K$, let $V_K$ denote the set of places of $K$. We follow \cite{kottwitz} and \cite{tate-torus}.

Fix a finite Galois extension of global fields $L/K$ with Galois group $G\coloneqq\op{Gal}(L/K)$. For later use, we will also let $G_v\subseteq G$ denote the decomposition group of a place $v\in V_L$. Now, we build two short exact sequences, as described in \autoref{sec:intro}. To begin, we note that the augmentation map $\ZZ[V_K]\onto\ZZ$ induces the short exact sequence
\[0\to\ZZ[V_L]_0\to\ZZ[V_L]\to\ZZ\to0\label{eq:sesx}\tag{$X$}\]
where $\ZZ[V_L]$ is the kernel of $\ZZ[V_L]\onto\ZZ$. We also have the short exact sequence
\[0\to L^\times\to\AA_L^\times\to\AA_L^\times/L^\times\to0\tag{$A$}\label{eq:sesa}\]
where the inclusion $L^\times\into\AA_L^\times$ is the diagonal one.

\subsubsection{Construction of \texorpdfstring{$\mc E_1$ and $\mc E_2$}{ E1 and E2}}
We now construct the Kottwitz gerbs one at a time. For $\mc E_1$, we let $\alpha_1(L/K)\in\widehat H^2\left(G,\op{Hom}_\ZZ(\ZZ,\AA_L^\times/L^\times)\right)$ denote the global fundamental class. Then we use the recipe from \autoref{thm:classifyextensionscohom} to construct the group extension
\[0\to\AA_L^\times/L^\times\to\mc E_1(L/K)\to G\to0.\]
This completes the construction of $\mc E_1(L/K)$, so we see that constructing $\mc E_1(L/K)$ is exactly as hard as constructing the global fundamental class.

To construct $\mc E_2$, let $\mathbb D_2\coloneqq\op{Hom}_\ZZ(\ZZ[V_L],-)$ denote the protorus with character group $\ZZ[V_L]$. Then $\mathcal E_2(L/K)$ is the Galois gerb associated to a particular class $\alpha_2\in\widehat H^2\left(G,\mathbb D(\mathbb A_L)\right)$. To construct this class, we need the following lemma.
\begin{lemma}[{\cite[p.~714]{tate-torus}}] \label{lem:magicaltate}
	Let $L/K$ be an extension of global fields with Galois group $G$, and let $V_L$ and $V_K$ denote the set of places of $L$ and $K$ respectively. Given a place $v\in V_L$, let $G_v\subseteq G$ denote its decomposition group. Then, for any $i\in\ZZ$,
	\[\widehat H^i(G,\op{Hom}_\ZZ(\ZZ[V_L],M))\simeq\prod_{u\in V_K}\widehat H^i(G_{v(u)},M),\]
	where the product is over places $u\in V_K$ taking a fixed place $v(u)\in V_L$ above $u$.
\end{lemma}
\begin{proof}
	We give the proof for later use. This is essentially a matter of separating our places and then applying Shapiro's lemma. For each $u\in V_K$, let $V_{u}\subseteq V_L$ denote the set of places in $L$ above $u$. Then we see
	\[\ZZ[V_L]\simeq\bigoplus_{u\in V_K}\ZZ[V_u]\]
	as $G$-modules because the $G$-orbit of a place $v\in V_L$ lying over a place $u\in V_K$ is exactly $V_u$. Thus, we have the isomorphisms
	\begin{align*}
		\widehat H^i(G,\op{Hom}_\ZZ(\ZZ[V_L],M)) &\simeq \widehat H^i\left(G,\op{Hom}_\ZZ\Bigg(\bigoplus_{u\in V_L}\ZZ[V_u],M\Bigg)\right) \\
		&\simeq \widehat H^i\left(G,\prod_{u\in V_K}\op{Hom}_\ZZ(\ZZ[V_u],M)\right) \\
		&\simeq \prod_{u\in V_K}\widehat H^i\left(G,\op{Hom}_\ZZ(\ZZ[V_u],M)\right).
	\end{align*}
	It remains to show that
	\[\widehat H^i\left(G,\op{Hom}_\ZZ(\ZZ[V_u],M)\right)\stackrel?\simeq\widehat H^i(G_{v(u)},M).\]
	Well, for each place $u\in V_K$, find a place $v(u)\in V_L$ above it. As discussed above, $V_u$ is a transitive $G$-set, and the stabilizer of $v(u)$ is $G_{v(u)}$. Thus, $V_u\simeq G_{v(u)}\backslash G$ as $G$-sets (note the distinction between left and right $G$-sets is somewhat irrelevant because $gG_v=G_vg$ for each $g\in G_v$), so $\ZZ[V_u]\simeq\ZZ[G_{v(u)}\backslash G]$ as $G$-modules. Thus, we may write
	\begin{align*}
		\widehat H^i\left(G,\op{Hom}_\ZZ(\ZZ[V_u],M)\right) &\simeq \widehat H^i\left(G,\op{Hom}_\ZZ(\ZZ[G_{v(u)}\backslash G],M)\right) \\
		&\simeq \widehat H^i\left(G,\op{Mor}_{\mathrm{Set}}(G_{v(u)}\backslash G,M)\right) \\
		&\simeq \widehat H^i\big(G,\op{CoInd}_{G_{v(u)}}^G(M)\big),
	\end{align*}
	where the last isomorphism is because $\op{Mor}_{\mathrm{Set}}(G_{v(u)}\backslash G,M)\simeq\op{CoInd}_H^G(M)$ by taking $f\colon G_{v(u)}\backslash G\to M$ to the function $g\mapsto gf\left(G_vg^{-1}\right)$. Now, this last cohomology group is isomorphic to $\widehat H^i(G_{v(u)},M)$ by Shapiro's lemma, thus finishing.
\end{proof}
\begin{remark} \label{rem:forwardshapiro}
	Tracking through the application of Shapiro's lemma above, we can see that the isomorphism behaves as
	\[\widehat H^i(G,\op{Hom}_\ZZ(\ZZ[V_L],M))\stackrel{\op{Res}}\to\widehat H^i(G_{v},\op{Hom}_\ZZ(\ZZ[V_L],M))\stackrel{\op{eval}_v}\to\widehat H^i(G_{v},M)\]
	on components; here $\op{eval}_v$ is induced by the evaluation-at-$v$ map $\op{Hom}_\ZZ(\ZZ[V_L],M)\to M$.
\end{remark}
Thus, to specify $\alpha_2\in\widehat H^2(G,\mathbb D_2(\AA_L))$, it is enough to specify a set of classes
\[\alpha_2(u)\in\widehat H^2\left(G_{v(u)},\AA_L^\times\right)\]
for each $u\in V_K$. To do so, we note that $G_{v(u)}=\op{Gal}(L_{v(u)}/K_u)$, so we use the natural embedding $i_v\colon L_v\into\AA_L^\times$ (for $u\in V_L$) to set
\[\alpha_2(u)\coloneqq i_{v(u)}\big(\alpha(L_{v(u)}/K_u)\big),\]
where $\alpha(L_{v(u)}/K_u)\in\widehat H^2\left(G_{v(u)},L_{v(u)}^\times\right)$ is the local fundamental class. Now, from $\alpha_2$, we construct $\mc E_2(L/K)$ again from \autoref{thm:classifyextensionscohom} as the extension
\[0\to\mathbb D_2(\AA_L^\times)\to\mc E_2(L/K)\to G\to0.\]
This completes the construction of $\mc E_2$.

\subsubsection{Constructing \texorpdfstring{$\mc E_3$}{ E3}}
Lastly, we construct $\mc E_3$. Roughly speaking, we note that the morphism of short exact sequences from \autoref{eq:sesx} and \autoref{eq:sesa} can be specified by commuting morphisms $\ZZ[V_L]\to\AA_L^\times$ and $\ZZ\to\AA_L^\times/L^\times$, inducing the last arrow as follows.
% https://q.uiver.app/?q=WzAsMTAsWzAsMCwiMCJdLFsxLDAsIlxcWlpbVl9MXV8wIl0sWzIsMCwiXFxaWltWX0xdIl0sWzMsMCwiXFxaWiJdLFs0LDAsIjAiXSxbMCwxLCIwIl0sWzEsMSwiTF5cXHRpbWVzIl0sWzIsMSwiXFxBQV9MXlxcdGltZXMiXSxbMywxLCJcXEFBX0xeXFx0aW1lcy9MXlxcdGltZXMiXSxbNCwxLCIwIl0sWzAsMV0sWzEsMl0sWzIsM10sWzMsNF0sWzUsNl0sWzYsN10sWzcsOF0sWzgsOV0sWzEsNiwiIiwxLHsic3R5bGUiOnsiYm9keSI6eyJuYW1lIjoiZGFzaGVkIn19fV0sWzIsN10sWzMsOF1d&macro_url=https%3A%2F%2Fraw.githubusercontent.com%2FdFoiler%2Fnotes%2Fmaster%2Fnir.tex
\[\begin{tikzcd}
	0 & {\ZZ[V_L]_0} & {\ZZ[V_L]} & \ZZ & 0 \\
	0 & {L^\times} & {\AA_L^\times} & {\AA_L^\times/L^\times} & 0
	\arrow[from=1-1, to=1-2]
	\arrow[from=1-2, to=1-3]
	\arrow[from=1-3, to=1-4]
	\arrow[from=1-4, to=1-5]
	\arrow[from=2-1, to=2-2]
	\arrow[from=2-2, to=2-3]
	\arrow[from=2-3, to=2-4]
	\arrow[from=2-4, to=2-5]
	\arrow[dashed, from=1-2, to=2-2]
	\arrow[from=1-3, to=2-3]
	\arrow[from=1-4, to=2-4]
\end{tikzcd}\]
Intuitively, this should let us specify a cohomology class $\alpha_3\in\widehat H^2\left(G,\op{Hom}_\ZZ(\ZZ[V_L]_0,L^\times)\right)$ from melding together $\alpha_1$ and $\alpha_2$.

To rigorize this, we let $\op{Hom}_\ZZ(X,A)$ denote the group of morphisms of short exact sequences from \autoref{eq:sesx} to \autoref{eq:sesa}; we let $\pi_1,\pi_2,\pi_3$ denote the projections from $\op{Hom}_\ZZ(X,A)$ to $\op{Hom}_\ZZ(\ZZ[V_L]_0,L^\times)$, to $\op{Hom}_\ZZ(\ZZ[V_L],\AA_L^\times)$, and to $\op{Hom}_\ZZ(\ZZ,\AA_L^\times/L^\times)$ respectively. Then the above argument tells us that
% https://q.uiver.app/?q=WzAsNCxbMCwwLCJcXG9we0hvbX1fXFxaWihYLEEpIl0sWzEsMCwiXFxvcHtIb219X1xcWlooXFxaWltWX0xdXzAsXFxBQV9MXlxcdGltZXMpIl0sWzAsMSwiXFxvcHtIb219X1xcWlooXFxaWixMXlxcdGltZXMpIl0sWzEsMSwiXFxvcHtIb219X1xcWlooXFxaWltWX0xdLFxcQUFfTF5cXHRpbWVzKSJdLFswLDEsIlxccGlfMiJdLFswLDIsIlxccGlfMyIsMl0sWzEsM10sWzIsM10sWzAsMywiIiwxLHsic3R5bGUiOnsibmFtZSI6ImNvcm5lciJ9fV1d&macro_url=https%3A%2F%2Fraw.githubusercontent.com%2FdFoiler%2Fnotes%2Fmaster%2Fnir.tex
\[\begin{tikzcd}
	{\op{Hom}_\ZZ(X,A)} & {\op{Hom}_\ZZ(\ZZ[V_L]_0,\AA_L^\times)} \\
	{\op{Hom}_\ZZ(\ZZ,L^\times)} & {\op{Hom}_\ZZ(\ZZ[V_L],\AA_L^\times)}
	\arrow["{\pi_2}", from=1-1, to=1-2]
	\arrow["{\pi_3}"', from=1-1, to=2-1]
	\arrow[from=1-2, to=2-2]
	\arrow[from=2-1, to=2-2]
	\arrow["\lrcorner"{anchor=center, pos=0.125}, draw=none, from=1-1, to=2-2]
\end{tikzcd}\]
is a pull-back square. Then we can check via \autoref{lem:magicaltate} that $\widehat H^1(G,\op{Hom}_\ZZ(\ZZ[V_L],\AA_L^\times))=0$, which gives the following result.
\begin{lemma}[{\cite[p.~716]{tate-torus}, \cite[Lemma~6.3]{kottwitz}}] \label{lem:constructtatealpha}
	Fix everything as above. Then
	% https://q.uiver.app/?q=WzAsNCxbMCwwLCJcXHdpZGVoYXQgSF4yKEcsXFxvcHtIb219X1xcWlooWCxBKSkiXSxbMSwwLCJcXHdpZGVoYXQgSF4yKEcsXFxvcHtIb219X1xcWlooXFxaWltWX0xdXzAsXFxBQV9MXlxcdGltZXMpKSJdLFswLDEsIlxcd2lkZWhhdCBIXjIoRyxcXG9we0hvbX1fXFxaWihcXFpaLExeXFx0aW1lcykpIl0sWzEsMSwiXFx3aWRlaGF0IEheMihHLFxcb3B7SG9tfV9cXFpaKFxcWlpbVl9MXSxcXEFBX0xeXFx0aW1lcykpIl0sWzAsMSwiXFxwaV8yIl0sWzAsMiwiXFxwaV8zIiwyXSxbMSwzXSxbMiwzXSxbMCwzLCIiLDEseyJzdHlsZSI6eyJuYW1lIjoiY29ybmVyIn19XV0=&macro_url=https%3A%2F%2Fraw.githubusercontent.com%2FdFoiler%2Fnotes%2Fmaster%2Fnir.tex
	\[\begin{tikzcd}
		{\widehat H^2(G,\op{Hom}_\ZZ(X,A))} & {\widehat H^2(G,\op{Hom}_\ZZ(\ZZ[V_L]_0,\AA_L^\times))} \\
		{\widehat H^2(G,\op{Hom}_\ZZ(\ZZ,L^\times))} & {\widehat H^2(G,\op{Hom}_\ZZ(\ZZ[V_L],\AA_L^\times))}
		\arrow["{\pi_2}", from=1-1, to=1-2]
		\arrow["{\pi_3}"', from=1-1, to=2-1]
		\arrow[from=1-2, to=2-2]
		\arrow[from=2-1, to=2-2]
		\arrow["\lrcorner"{anchor=center, pos=0.125, rotate=45}, draw=none, from=1-1, to=2-2]
	\end{tikzcd}\]
	is a pull-back square.
\end{lemma} 
To finish, we merely have to check that $\alpha_2$ and $\alpha_1$ have the same image in $\widehat H^2(G,\op{Hom}_\ZZ(\ZZ[V_L],\AA_L^\times))$ so that \autoref{lem:constructtatealpha} promises us $\alpha\in\widehat H^2(G,\op{Hom}_\ZZ(X,A))$ such that $\pi_2\alpha=\alpha_2$ and $\pi_1\alpha=\alpha_1$. These together let us construct $\alpha_3\coloneqq\pi_3\alpha$ and hence $\mc E_3(L/K)$ from \autoref{thm:classifyextensionscohom} as the extension
\[0\to\mathbb D_3(L^\times)\to\mc E_3(L/K)\to G\to0,\]
where $\mathbb D_3\coloneqq\op{Hom}_\ZZ(\ZZ[V_L]_0,-)$.
\begin{remark}
	It is true that certain morphisms $f_2\colon\ZZ[V_L]\to\AA_L^\times$ induce a morphism $f_3\colon\ZZ[V_L]_0\to L^\times$ making
	% https://q.uiver.app/?q=WzAsNixbMCwwLCIwIl0sWzEsMCwiXFxaWltWX0xdXzAiXSxbMiwwLCJcXFpaW1ZfTF0iXSxbMCwxLCIwIl0sWzEsMSwiTF5cXHRpbWVzIl0sWzIsMSwiXFxBQV9MXlxcdGltZXMiXSxbMCwxXSxbMSwyLCJiJyJdLFszLDRdLFs0LDUsImEnIl0sWzEsNCwiZl8zIiwwLHsic3R5bGUiOnsiYm9keSI6eyJuYW1lIjoiZGFzaGVkIn19fV0sWzIsNSwiZl8yIl1d&macro_url=https%3A%2F%2Fraw.githubusercontent.com%2FdFoiler%2Fnotes%2Fmaster%2Fnir.tex
	\[\begin{tikzcd}
		0 & {\ZZ[V_L]_0} & {\ZZ[V_L]} \\
		0 & {L^\times} & {\AA_L^\times}
		\arrow[from=1-1, to=1-2]
		\arrow["{b'}", from=1-2, to=1-3]
		\arrow[from=2-1, to=2-2]
		\arrow["{a'}", from=2-2, to=2-3]
		\arrow["{f_3}", dashed, from=1-2, to=2-2]
		\arrow["{f_2}", from=1-3, to=2-3]
	\end{tikzcd}\]
	commute by solving $a'f_3=f_2b'$ (when possible). However, it is not true that $\alpha_2$ uniquely determines $\alpha_3$ like this because the induced map
	\[a'\colon\widehat H^2(G,\op{Hom}_\ZZ(\ZZ[V_L]_0,L^\times))\to\widehat H^2(G,\op{Hom}_\ZZ(\ZZ[V_L]_0,\AA_L^\times))\]
	need not be injective in general, so knowing $a'\alpha_3=b'\alpha_2$ does not specify $\alpha_3$ in general.
\end{remark}

\section{Generalized Periodic Cohomology} \label{sec:crackpot}
% !TEX root = ../abeliangerbs.tex

The goal of this section is to separate out what we can, a priori, expect from our cohomology-encoding modules from what is a special property of the specific cohomology-encoding module we study in the rest of the paper.

Throughout this section, $G$ will be a finite group. To motivate where we are going, we will go ahead and say that a $p$-encoding $G$-module $X$ is a $G$-module equipped with a natural isomorphism
\[\widehat H^i(G,\op{Hom}_\ZZ(X,-))\Rightarrow\widehat H^{i+p}(G,-).\]
The idea is that, in the case of $i=0$ for a specific $G$-modulee $A$, we are taking cohomology of $\widehat H^p(G,A)$ and encoding this data as
\[\widehat H^0(G,\op{Hom}_\ZZ(X,A))=\frac{\op{Hom}_{\ZZ[G]}(X,A)}{N_G\op{Hom}_\ZZ(X,A)}.\]
If $X$ is finitely generated, we can write $X=\ZZ[G]^m/M$ for some $m\ge0$ and $G$-module $M$, so this object essentially picks out $m$ elements of $A$ and encodes some relations among them. In other words, an $m$-tuple of elements in $A$ (satisfying some special relations) is able to encode cohomology.

When we may take $X=\ZZ$, we are essentially studying groups with periodic cohomology, so some results in this section will mimic these results. However, periodic cohomology requires somewhat stringent conditions on the group itself, and allowing this ``free parameter'' $X$ will permit general groups at the cost of a perhaps more complex $X$. For example, when $p\ge0$, we can take $X=I_G^{\otimes p}$, though this $G$-module is quite rough to handle.

\subsection{Shiftable Functors}
The main point of this section is to set up some theory around what we call shiftable functors.
\begin{definition}
	Let $G$ be a finite group. Then a functor $F\colon\op{Mod}_G\to\op{Mod}_G$ is a \textit{shiftable functor} if and only if $F$ is both additive and sends induced modules to induced modules.
\end{definition}
The main point to shiftable functors $F$ is that the dimension-shifting short exact sequences
\[\arraycolsep=1.4pt\begin{array}{ccccccccc}
	0 &\to& I_G\otimes_\ZZ A &\to& \ZZ[G]\otimes_\ZZ A &\to& A &\to& 0 \\
	0 &\to& A &\to& \op{Hom}_\ZZ(\ZZ[G],A) &\to& \op{Hom}_\ZZ(I_G,A) &\to& 0
\end{array}\]
will remain exact upon applying $F$ (because $F$ is additive, and these short exact sequences are $\ZZ$-split), and the middle term will remain induced.

Our key example of a shiftable functor will be $\op{Hom}_\ZZ(X,-)$ for $G$-modules $X$.
\begin{lemma} \label{lem:hompreservesinduced}
	Let $G$ be a finite group and $X$ a $G$-module. Then $\op{Hom}_\ZZ(X,-)$ is a shiftable functor.
\end{lemma}
\begin{proof}
	It is known that $\op{Hom}_\ZZ(X,-)$ is an additive functor, so we just need to check that it sends induced modules to induced modules. Let $M$ be an induced module, and we want to show that $\op{Hom}_\ZZ(X,M)$ is also induced. By definition, we can write $M\coloneqq\op{Hom}_\ZZ(\ZZ[G],A)$ for some $G$-module $A$, where $A$ has perhaps trivial $G$-action. Now, we claim that
	\[\arraycolsep=1.4pt\begin{array}{cccc}
		\varphi\colon& \op{Hom}_\ZZ(X,\op{Hom}_\ZZ(\ZZ[G],A)) &\simeq& \op{Hom}_\ZZ(\ZZ[G],\op{Hom}_\ZZ(X,A)) \\
		\varphi\colon& f &\mapsto& \big(z\mapsto(x\mapsto f(x)(z))\big)
	\end{array}\]
	is an isomorphism of $G$-modules. This will finish because the right-hand $G$-module is induced.
	
	Now, $\varphi$ s a homomorphism of abelian groups because
	\[\varphi(f+f')(z)(x)=(f+g)(x)(z)=\varphi(f)(z)(x)+\varphi(f')(z)(x)\]
	for any $x$ and $z$ and $f,f'\in\op{Hom}_\ZZ(X,\op{Hom}_\ZZ(\ZZ[G],A))$. This is a $G$-module homomorphism because any $g\in G$ and $f\in\op{Hom}_\ZZ(X,\op{Hom}_\ZZ(\ZZ[G],A))$ has
	\begin{align*}
		\varphi(gf)(z)(x) &= \big(g\cdot\varphi(f)(g^{-1}z)\big)(x) \\
		&= g\cdot\varphi(f)(g^{-1}z)(g^{-1}x) \\
		&= g\cdot f(g^{-1}x)(g^{-1}z) \\
		&= \big(g\cdot f(g^{-1}x)\big)(z) \\
		&= (gf)(x)(z) \\
		&= \varphi(gf)(x)(z)
	\end{align*}
	for each $x$ and $z$.

	Now, we define
	\[\arraycolsep=1.4pt\begin{array}{cccc}
		\psi\colon& \op{Hom}_\ZZ(\ZZ[G],\op{Hom}_\ZZ(X,A)) &\simeq& \op{Hom}_\ZZ(X,\op{Hom}_\ZZ(\ZZ[G],A)) \\
		\psi\colon& f &\mapsto& \big(x\mapsto(z\mapsto f(z)(x))\big)
	\end{array}\]
	to be the inverse morphism. The exact same checks show that this is a $G$-module homomorphism, and it is not hard to see that
	\[\varphi\psi(f)(z)(x)=\psi(f)(z)(x)=f(x)(z),\]
	so $\varphi\circ\psi$ is the identity; similarly, $\psi\circ\varphi$ is the identity.
\end{proof}
With that said, we also remark that shifting functors are rather expansive, and we will need a little more freedom in applications.
\begin{lemma}
	Let $G$ be a finite group and $X$ a $G$-module. Then $X\otimes_\ZZ-$ is a shiftable functor.
\end{lemma}
\begin{proof}
	Again, $X\otimes_\ZZ-$ is additive, so we just need to check that it sends induced modules to induced modules. Well, suppose $M\coloneqq\ZZ[G]\otimes_\ZZ A$ is an induced module. Then we note the isomorphisms
	\[X\otimes_\ZZ M=X\otimes_\ZZ\ZZ[G]\otimes_\ZZ A\simeq\ZZ[G]\otimes_\ZZ(X\otimes_\ZZ A)\]
	are all also isomorphisms of $G$-modules. Because $\ZZ[G]\otimes_\ZZ(X\otimes_\ZZ A)$ is induced, we are done.
\end{proof}
\begin{lemma}
	Let $G$ be a finite group. If $F$ and $F'$ are shiftable functors, then $F\circ F'$ is a shiftable functor.
\end{lemma}
\begin{proof}
	This follows directly from the definition.
\end{proof}
\begin{example}
	The functor
	\[A\mapsto\op{Hom}_\ZZ(I_G,I_G\otimes_\ZZ A)\]
	is a shiftable functor.
\end{example}

\subsection{Shifting by Cup Products}
A key property of shiftable functors is how we will be able to relate them to each other via cup products. With this in mind, we have the following definition.
\begin{definition}
	Let $G$ be a finite group. Then we define a \textit{shifting pair} $(F,F',X,\eta)$ to be a pair of shiftable functors $F$ and $F'$ equipped with a natural transformation
	\[\eta_\bullet\colon X\otimes_\ZZ F\Rightarrow F'.\]
\end{definition}
\begin{example} \label{ex:shiftingpair}
	Given $G$-modules $X$ and $X'$, there is a canonical pre-composition map
	\[\eta_\bullet\colon\op{Hom}_\ZZ(X',X)\otimes_\ZZ\op{Hom}_\ZZ(X,-)\to\op{Hom}_\ZZ(X',-),\]
	so $(\op{Hom}_\ZZ(X,-),\op{Hom}_\ZZ(X',-),\op{Hom}_\ZZ(X',X),\eta_\bullet)$ is a shifting pair.
\end{example}
\begin{lemma} \label{lem:cuppingisnatural}
	Let $G$ be a finite group, and let $(F,F',X,\eta)$ be a shifting pair. Then, given indices $p,q\in\ZZ$ and $c\in\widehat H^p(G,X)$, the cup-product maps
	\[(c\cup-)\colon\widehat H^q(G,F-)\to\widehat H^{p+q}(G,F'-)\]
	make a natural transformation of cohomology functors.
\end{lemma}
\begin{proof}
	Given a $G$-module $A$, we note that our cup-product map is defined by
	\[\widehat H^q(G,FA)\stackrel{c\cup-}\to\widehat H^{p+q}(G,X\otimes_\ZZ FA)\stackrel{\eta_A}\to\widehat H^{p+q}(G,F'A).\]
	So, to check naturality, we pick up a $G$-module homomorphism $\varphi\colon A\to B$ and draw the following diagram.
	% https://q.uiver.app/?q=WzAsNixbMCwwLCJcXHdpZGVoYXQgSF5xKEcsRkEpIl0sWzEsMCwiXFx3aWRlaGF0IEhee3ArcX0oRyxYXFxvdGltZXNfXFxaWiBGQSkiXSxbMiwwLCJcXHdpZGVoYXQgSF57cCtxfShHLEYnQSkiXSxbMCwxLCJcXHdpZGVoYXQgSF5xKEcsRkIpIl0sWzEsMSwiXFx3aWRlaGF0IEhee3ArcX0oRyxYXFxvdGltZXNfXFxaWiBGQikiXSxbMiwxLCJcXHdpZGVoYXQgSF57cCtxfShHLEYnQikiXSxbMCwxLCJjXFxjdXAtIl0sWzEsMiwiXFxldGFfQSJdLFs0LDUsIlxcZXRhX0IiXSxbMSw0LCJmIiwyXSxbMiw1LCJmIiwyXSxbMCwzLCJmIiwyXSxbMyw0LCJjXFxjdXAtIl1d&macro_url=https%3A%2F%2Fraw.githubusercontent.com%2FdFoiler%2Fnotes%2Fmaster%2Fnir.tex
	\[\begin{tikzcd}
		{\widehat H^q(G,FA)} & {\widehat H^{p+q}(G,X\otimes_\ZZ FA)} & {\widehat H^{p+q}(G,F'A)} \\
		{\widehat H^q(G,FB)} & {\widehat H^{p+q}(G,X\otimes_\ZZ FB)} & {\widehat H^{p+q}(G,F'B)}
		\arrow["{c\cup-}", from=1-1, to=1-2]
		\arrow["{\eta_A}", from=1-2, to=1-3]
		\arrow["{\eta_B}", from=2-2, to=2-3]
		\arrow["f"', from=1-2, to=2-2]
		\arrow["f"', from=1-3, to=2-3]
		\arrow["f"', from=1-1, to=2-1]
		\arrow["{c\cup-}", from=2-1, to=2-2]
	\end{tikzcd}\]
	The left square commutes by functoriality of cup products (see \cite{bonn-lectures}, Proposition~I.5.3), and the right square commutes by the naturality of $\eta$ and functoriality of $\widehat H^{p+q}(G,-)$.
\end{proof}
Let's start with a key result on shiftable functors, which gives a taste for why our hypotheses are so specially chosen.
\begin{proposition} \label{prop:dimshiftcupisos}
	Let $G$ be a finite group, and let $(F,F',X,\eta)$ be a shifting pair. If we have indices $p,q\in\ZZ$ and $c\in H^p(G,X)$ such that the cup-product map
	\[c\cup-\colon\widehat H^q(G,F-)\to\widehat H^{p+q}(G,F'-)\]
	is a natural isomorphism, then the cup-product map
	\[c\cup-\colon\widehat H^j(G,F-)\to\widehat H^{p+j}(G,F'-)\]
	is a natural isomorphism and indices $j\in\ZZ$.
\end{proposition}
\begin{proof}
	This proof is by dimension-shifting on $q$. Note that it suffices by \autoref{lem:cuppingisnatural} to only worry about the component morphisms being isomorphisms.
	
	To shift downwards, we suppose that the cup-product map is always an isomorphism for $j$, and we show that it is always an isomorphism $j-1$. Namely, fix a $G$-module $A$, and we are interested in showing that the cup-product map
	\[c\cup-\colon\widehat H^{j-1}(G,FA)\to\widehat H^{p+j-1}(G,F'A)\]
	is an isomorphism. To do so, we note the short exact sequence
	\begin{equation}
		0\to I_G\to\ZZ[G]\to\ZZ\to0 \label{eq:shiftingses}
	\end{equation}
	which splits over $\ZZ$ and thus gives us the short exact sequences
	% https://q.uiver.app/?q=WzAsMTUsWzAsMCwiMCJdLFsxLDAsIlxcb3B7SG9tfV9cXFpaKFgsSV9HXFxvdGltZXNfXFxaWiBBKSJdLFsyLDAsIlxcb3B7SG9tfV9cXFpaKFgsXFxaWltHXVxcb3RpbWVzX1xcWlogQSkiXSxbMywwLCJcXG9we0hvbX1fXFxaWihYLEEpIl0sWzQsMCwiMCJdLFswLDEsIjAiXSxbNCwxLCIwIl0sWzEsMSwiWFxcb3RpbWVzX1xcWlpcXG9we0hvbX1fXFxaWihYLElfR1xcb3RpbWVzX1xcWlogQSkiXSxbMiwxLCJYXFxvdGltZXNfXFxaWlxcb3B7SG9tfV9cXFpaKFgsXFxaWltHXVxcb3RpbWVzX1xcWlogQSkiXSxbMywxLCJYXFxvdGltZXNfXFxaWlxcb3B7SG9tfV9cXFpaKFgsQSkiXSxbMCwyLCIwIl0sWzEsMiwiSV9HXFxvdGltZXNfXFxaWiBBIl0sWzIsMiwiXFxaWltHXVxcb3RpbWVzX1xcWlogQSJdLFszLDIsIkEiXSxbNCwyLCIwIl0sWzAsMV0sWzEsMl0sWzIsM10sWzMsNF0sWzUsN10sWzcsOF0sWzgsOV0sWzksNl0sWzEwLDExXSxbMTEsMTJdLFsxMiwxM10sWzEzLDE0XSxbNywxMSwiXFxldGFfe0lfR30iLDJdLFs4LDEyLCJcXGV0YV97XFxaWltHXX0iLDJdLFs5LDEzLCJcXGV0YV9BIiwyXV0=&macro_url=https%3A%2F%2Fraw.githubusercontent.com%2FdFoiler%2Fnotes%2Fmaster%2Fnir.tex
	\[\begin{tikzcd}
		0 & {F(I_G\otimes_\ZZ A)} & {F(\ZZ[G]\otimes_\ZZ A)} & {FA} & 0 \\
		0 & {X\otimes_\ZZ F(I_G\otimes_\ZZ A)} & {X\otimes_\ZZ F(\ZZ[G]\otimes_\ZZ A)} & {X\otimes_\ZZ FA} & 0 \\
		0 & {F'(I_G\otimes_\ZZ A)} & {F'(\ZZ[G]\otimes_\ZZ A)} & {F'A} & 0
		\arrow[from=1-1, to=1-2]
		\arrow[from=1-2, to=1-3]
		\arrow[from=1-3, to=1-4]
		\arrow[from=1-4, to=1-5]
		\arrow[from=2-1, to=2-2]
		\arrow[from=2-2, to=2-3]
		\arrow[from=2-3, to=2-4]
		\arrow[from=2-4, to=2-5]
		\arrow[from=3-1, to=3-2]
		\arrow[from=3-2, to=3-3]
		\arrow[from=3-3, to=3-4]
		\arrow[from=3-4, to=3-5]
		\arrow["{\eta_{I_G}}"', from=2-2, to=3-2]
		\arrow["{\eta_{\ZZ[G]}}"', from=2-3, to=3-3]
		\arrow["{\eta_A}"', from=2-4, to=3-4]
	\end{tikzcd}\]
	where the bottom two rows commute by definition of $\eta$ and thus give a morphism of short exact sequences. These short exact sequences give us boundary morphisms
	\[\arraycolsep=1.4pt\begin{array}{rlcl}
		\delta\colon& \widehat H^{p+j-1}(G,F'A) &\to& \widehat H^{p+j}(G,F'(I_G\otimes_\ZZ A)) \\
		\delta_h\colon& \widehat H^{j-1}(G,FA) &\to& \widehat H^j(G,F(I_G\otimes_\ZZ A)) \\
		\delta_t\colon& \widehat H^{p+j-1}(G,X\otimes_\ZZ FA) &\to& \widehat H^{p+j}(G,X\otimes_\ZZ F(I_G\otimes_\ZZ A)).
	\end{array}\]
	Notably, all these $\delta$ morphisms because their short exact sequences have induced middle terms: in particular, $F$, $X\otimes_\ZZ F$, and $F'$ are all shiftable functors.
	
	Now, the key to this dimension-shifting is claiming that the diagram
	% https://q.uiver.app/?q=WzAsNCxbMCwwLCJcXHdpZGVoYXQgSF57ai0xfShHLFxcb3B7SG9tfV9cXFpaKFgsQSkpIl0sWzEsMCwiXFx3aWRlaGF0IEhee3Arai0xfShHLEEpIl0sWzAsMSwiXFx3aWRlaGF0IEhee2p9KEcsXFxvcHtIb219X1xcWlooWCxJX0dcXG90aW1lc19cXFpaIEEpKSJdLFsxLDEsIlxcd2lkZWhhdCBIXntwK2p9KEcsSV9HXFxvdGltZXNfXFxaWiBBKSJdLFswLDEsImNcXGN1cC0iXSxbMiwzLCJjXFxjdXAtIl0sWzAsMiwiXFxkZWx0YV9oIiwyXSxbMSwzLCJcXGRlbHRhIiwyXV0=&macro_url=https%3A%2F%2Fraw.githubusercontent.com%2FdFoiler%2Fnotes%2Fmaster%2Fnir.tex
	\[\begin{tikzcd}
		{\widehat H^{j-1}(G,FA)} & {\widehat H^{p+j-1}(G,F'A)} \\
		{\widehat H^{j}(G,F(I_G\otimes_\ZZ A))} & {\widehat H^{p+j}(G,F'(I_G\otimes_\ZZ A))}
		\arrow["{c\cup-}", from=1-1, to=1-2]
		\arrow["{c\cup-}", from=2-1, to=2-2]
		\arrow["{\delta_h}"', from=1-1, to=2-1]
		\arrow["(-1)^p\delta"', from=1-2, to=2-2]
	\end{tikzcd}\]
	commutes. Indeed, this will be enough because the bottom row is an isomorphism by the inductive hypothesis, and the left and morphisms are isomorphisms as discussed above, which makes the top row into an isomorphism. Well, to see that the diagram commutes, we expand the diagram as follows.
	% https://q.uiver.app/?q=WzAsNixbMCwwLCJcXHdpZGVoYXQgSF57ai0xfShHLFxcb3B7SG9tfV9cXFpaKFgsQSkpIl0sWzEsMCwiXFx3aWRlaGF0IEhee3Aran0oRyxYXFxvdGltZXNfXFxaWlxcb3B7SG9tfV9cXFpaKEEpKSJdLFswLDEsIlxcd2lkZWhhdCBIXntqfShHLFxcb3B7SG9tfV9cXFpaKFgsSV9HXFxvdGltZXNfXFxaWiBBKSkiXSxbMSwxLCJcXHdpZGVoYXQgSF57cCtqfShHLFhcXG90aW1lc19cXFpaXFxvcHtIb219X1xcWlooSV9HXFxvdGltZXNfXFxaWiBBKSkiXSxbMiwwLCJcXHdpZGVoYXQgSF57cCtqLTF9KEcsQSkiXSxbMiwxLCJcXHdpZGVoYXQgSF57cCtqfShHLElfR1xcb3RpbWVzX1xcWlogQSkiXSxbMCwxLCJjXFxjdXAtIl0sWzIsMywiY1xcY3VwLSJdLFswLDIsIlxcZGVsdGFfaCIsMl0sWzEsMywiXFxkZWx0YV90IiwyXSxbMSw0LCJcXGV0YV9BIl0sWzMsNSwiXFxldGFfe0lfR30iXSxbNCw1LCJcXGRlbHRhIiwyXV0=&macro_url=https%3A%2F%2Fraw.githubusercontent.com%2FdFoiler%2Fnotes%2Fmaster%2Fnir.tex
	\[\begin{tikzcd}
		{\widehat H^{j-1}(G,FA)} & {\widehat H^{p+j-1}(G,X\otimes_\ZZ FA)} & {\widehat H^{p+j-1}(G,F'A)} \\
		{\widehat H^{j}(G,F(I_G\otimes_\ZZ A))} & {\widehat H^{p+j}(G,X\otimes_\ZZ F(I_G\otimes_\ZZ A))} & {\widehat H^{p+j}(G,F'(I_G\otimes_\ZZ A))}
		\arrow["{c\cup-}", from=1-1, to=1-2]
		\arrow["{c\cup-}", from=2-1, to=2-2]
		\arrow["{\delta_h}"', from=1-1, to=2-1]
		\arrow["{(-1)^p\delta_t}"', from=1-2, to=2-2]
		\arrow["{\eta_A}", from=1-2, to=1-3]
		\arrow["{\eta_{I_G}}", from=2-2, to=2-3]
		\arrow["(-1)^p\delta"', from=1-3, to=2-3]
	\end{tikzcd}\]
	The left square commutes because cup products commute with boundary morphisms; the right square commutes by functoriality of boundary morphisms.

	Shifting upwards is similar. Suppose that the cup-product in question is always an isomorphism for $j$, and we show that it is always an isomorphism for $j+1$. Namely, fix a $G$-module $A$, and we are interested in showing that the cup-product map
	\[c\cup-\colon\widehat H^{j+1}(G,FA)\to\widehat H^{p+j+1}(G,F'A)\]
	is an isomorphism. As before, we use \autoref{eq:shiftingses} to induce the short exact sequences
	% https://q.uiver.app/?q=WzAsMTUsWzAsMCwiMCJdLFsxLDAsIlxcb3B7SG9tfV9cXFpaKFgsQSkiXSxbMiwwLCJcXG9we0hvbX1fXFxaWihYLFxcb3B7SG9tfV9cXFpaKFxcWlpbR10sQSkpIl0sWzMsMCwiXFxvcHtIb219X1xcWlooWCxcXG9we0hvbX1fXFxaWihJX0csQSkpIl0sWzQsMCwiMCJdLFswLDEsIjAiXSxbNCwxLCIwIl0sWzEsMSwiWFxcb3RpbWVzX1xcWlpcXG9we0hvbX1fXFxaWihYLEEpIl0sWzIsMSwiWFxcb3RpbWVzX1xcWlpcXG9we0hvbX1fXFxaWihYLFxcb3B7SG9tfV9cXFpaKFxcWlpbR10sQSkpIl0sWzMsMSwiWFxcb3RpbWVzX1xcWlpcXG9we0hvbX1fXFxaWihYLFxcb3B7SG9tfV9cXFpaKElfRyxBKSkiXSxbMCwyLCIwIl0sWzEsMiwiQSJdLFsyLDIsIlxcb3B7SG9tfV9cXFpaKFxcWlpbR10sQSkiXSxbMywyLCJcXG9we0hvbX1fXFxaWihJX0csQSkiXSxbNCwyLCIwIl0sWzAsMV0sWzEsMl0sWzIsM10sWzMsNF0sWzUsN10sWzcsOF0sWzgsOV0sWzksNl0sWzEwLDExXSxbMTEsMTJdLFsxMiwxM10sWzEzLDE0XSxbNywxMSwiXFxldGFfQSIsMl0sWzgsMTIsIlxcZXRhX3tcXFpaW0ddfSIsMl0sWzksMTMsIlxcZXRhX3tJX0d9IiwyXV0=&macro_url=https%3A%2F%2Fraw.githubusercontent.com%2FdFoiler%2Fnotes%2Fmaster%2Fnir.tex
	\[\begin{tikzcd}
		0 & {FA} & {F(\op{Hom}_\ZZ(\ZZ[G],A))} & {F(\op{Hom}_\ZZ(I_G,A))} & 0 \\
		0 & {X\otimes_\ZZ FA} & {X\otimes_\ZZ F(\op{Hom}_\ZZ(\ZZ[G],A))} & {X\otimes_\ZZ F(\op{Hom}_\ZZ(I_G,A))} & 0 \\
		0 & F'A & {F'(\op{Hom}_\ZZ(\ZZ[G],A))} & {F'(\op{Hom}_\ZZ(I_G,A))} & 0
		\arrow[from=1-1, to=1-2]
		\arrow[from=1-2, to=1-3]
		\arrow[from=1-3, to=1-4]
		\arrow[from=1-4, to=1-5]
		\arrow[from=2-1, to=2-2]
		\arrow[from=2-2, to=2-3]
		\arrow[from=2-3, to=2-4]
		\arrow[from=2-4, to=2-5]
		\arrow[from=3-1, to=3-2]
		\arrow[from=3-2, to=3-3]
		\arrow[from=3-3, to=3-4]
		\arrow[from=3-4, to=3-5]
		\arrow["{\eta_A}"', from=2-2, to=3-2]
		\arrow["{\eta_{\ZZ[G]}}"', from=2-3, to=3-3]
		\arrow["{\eta_{I_G}}"', from=2-4, to=3-4]
	\end{tikzcd}\]
	where again the bottom rows commute by definition of $\eta$. As before, we have the boundary morphisms
	\[\arraycolsep=1.4pt\begin{array}{rlcl}
		\delta\colon& \widehat H^{p+j}(G,F'(\op{Hom}_\ZZ(I_G,A))) &\to& \widehat H^{p+j+1}(G,F'A) \\
		\delta_h\colon& \widehat H^{j}(G,F(\op{Hom}_\ZZ(I_G,A))) &\to& \widehat H^{j+1}(G,FA) \\
		\delta_t\colon& \widehat H^{p+j}(G,X\otimes_\ZZ F(\op{Hom}_\ZZ(I_G,A))) &\to& \widehat H^{p+j+1}(G,X\otimes_\ZZ FA).
	\end{array}\]
	We again note that all $\delta$ are isomorphisms because the middle terms of our short exact sequences are induced: all of $F$ and $X\otimes_\ZZ F$ and $F'$ are shiftable functors.

	Once more, the key to the dimension-shifting will be the claim that the diagram
	% https://q.uiver.app/?q=WzAsNCxbMCwwLCJcXHdpZGVoYXQgSF57an0oRyxcXG9we0hvbX1fXFxaWihYLFxcb3B7SG9tfV9cXFpaKElfRyxBKSkpIl0sWzAsMSwiXFx3aWRlaGF0IEhee2orMX0oRyxcXG9we0hvbX1fXFxaWihYLEEpKSJdLFsxLDAsIlxcd2lkZWhhdCBIXntwK2p9KEcsXFxvcHtIb219X1xcWlooSV9HLEEpKSJdLFsxLDEsIlxcd2lkZWhhdCBIXntwK2orMX0oRyxBKSJdLFswLDEsIlxcZGVsdGFfaCIsMl0sWzIsMywiXFxkZWx0YSIsMl0sWzAsMiwiY1xcY3VwLSJdLFsxLDMsImNcXGN1cC0iXV0=&macro_url=https%3A%2F%2Fraw.githubusercontent.com%2FdFoiler%2Fnotes%2Fmaster%2Fnir.tex
	\[\begin{tikzcd}
		{\widehat H^{j}(G, F(\op{Hom}_\ZZ(I_G,A)))} & {\widehat H^{p+j}(G,F'(\op{Hom}_\ZZ(I_G,A)))} \\
		{\widehat H^{j+1}(G,FA)} & {\widehat H^{p+j+1}(G,F'A)}
		\arrow["{\delta_h}"', from=1-1, to=2-1]
		\arrow["(-1)^p\delta"', from=1-2, to=2-2]
		\arrow["{c\cup-}", from=1-1, to=1-2]
		\arrow["{c\cup-}", from=2-1, to=2-2]
	\end{tikzcd}\]
	commutes. This will be enough because the top arrow is an isomorphism by the inductive hypothesis, and the left and right arrows are isomorphisms as discussed above, thus making the bottom arrow also an isomorphism. Now, to see that the diagram commutes, we expand out our cup products as follows.
	% https://q.uiver.app/?q=WzAsNixbMCwwLCJcXHdpZGVoYXQgSF57an0oRyxcXG9we0hvbX1fXFxaWihYLFxcb3B7SG9tfV9cXFpaKElfRyxBKSkpIl0sWzAsMSwiXFx3aWRlaGF0IEhee2orMX0oRyxcXG9we0hvbX1fXFxaWihYLEEpKSJdLFsxLDAsIlxcd2lkZWhhdCBIXntwK2p9KEcsWFxcb3RpbWVzX1xcWlpcXG9we0hvbX1fXFxaWihJX0csQSkpIl0sWzEsMSwiXFx3aWRlaGF0IEhee3AraisxfShHLFhcXG90aW1lc19cXFpaXFxvcHtIb219X1xcWlooWCxBKSkiXSxbMiwxLCJcXHdpZGVoYXQgSF57cCtqKzF9KEcsQSkiXSxbMiwwLCJcXHdpZGVoYXQgSF57cCtqfShHLFxcb3B7SG9tfV9cXFpaKElfRyxBKSkiXSxbMCwxLCJcXGRlbHRhX2giLDJdLFsyLDMsIigtMSlecFxcZGVsdGFfdCIsMl0sWzAsMiwiY1xcY3VwLSJdLFsxLDMsImNcXGN1cC0iXSxbMiw1LCJcXGV0YV97SV9HfSJdLFszLDQsIlxcZXRhX0EiXSxbNSw0LCIoLTEpXnBcXGRlbHRhIiwyXV0=&macro_url=https%3A%2F%2Fraw.githubusercontent.com%2FdFoiler%2Fnotes%2Fmaster%2Fnir.tex
	\[\begin{tikzcd}
		{\widehat H^{j}(G,F(\op{Hom}_\ZZ(I_G,A)))} & {\widehat H^{p+j}(G,X\otimes_\ZZ F(\op{Hom}_\ZZ(I_G,A)))} & {\widehat H^{p+j}(G,F'(\op{Hom}_\ZZ(I_G,A)))} \\
		{\widehat H^{j+1}(G,FA)} & {\widehat H^{p+j+1}(G,X\otimes_\ZZ FA)} & {\widehat H^{p+j+1}(G,F'A)}
		\arrow["{\delta_h}"', from=1-1, to=2-1]
		\arrow["{(-1)^p\delta_t}"', from=1-2, to=2-2]
		\arrow["{c\cup-}", from=1-1, to=1-2]
		\arrow["{c\cup-}", from=2-1, to=2-2]
		\arrow["{\eta_{I_G}}", from=1-2, to=1-3]
		\arrow["{\eta_A}", from=2-2, to=2-3]
		\arrow["{(-1)^p\delta}"', from=1-3, to=2-3]
	\end{tikzcd}\]
	The left square commutes because cup products commute with boundary morphisms, and the right square commutes by functoriality of boundary morphisms. This finishes.
\end{proof}
Here are some applications.
\begin{cor} \label{cor:cupup}
	Let $G$ be a finite group. There exists $c\in\widehat H^1(G,I_G)$ such that, for any $G$-module $X$,
	\[c\cup-\colon\widehat H^i(G,\op{Hom}_\ZZ(X,-))\Rightarrow\widehat H^{i+1}(G,\op{Hom}_\ZZ(X,I_G\otimes_\ZZ-))\]
	is a natural isomorphism for any $i\in\ZZ$.
\end{cor}
\begin{proof}
	Here, we are using the shifting pair $(\op{Hom}_\ZZ(X,-),\op{Hom}_\ZZ(X,I_G\otimes_\ZZ-),I_G,\eta)$, where
	\[\eta_A\colon I_G\otimes_\ZZ\op{Hom}_\ZZ(X,A)\to\op{Hom}_\ZZ(X,I_G\otimes_\ZZ A)\]
	is the canonical map sending $z\otimes f$ to $x\mapsto z\otimes f(x)$.

	Now, in light of \autoref{prop:dimshiftcupisos}, we merely have to find $c\in\widehat H^1(G,I_G)$ and show that we have a natural isomorphism at $i=0$. Because we already have a natural transformation by \autoref{lem:cuppingisnatural}, we are only worried about making the component morphisms
	\[\widehat H^0(G,\op{Hom}_\ZZ(X,A))\to\widehat H^1(G,\op{Hom}_\ZZ(X,I_G\otimes_\ZZ A))\]
	isomorphisms for all $G$-modules $A$. Well, we note that we have the ($\ZZ$-split) short exact sequence
	\[0\to\op{Hom}_\ZZ(X,I_G\otimes_\ZZ A)\to\op{Hom}_\ZZ(X,\ZZ[G]\otimes_\ZZ A)\to\op{Hom}_\ZZ(X,I_G\otimes_\ZZ A)\to0\]
	which will induce a $\delta$ morphism between the correct modules. In fact, because $\op{Hom}_\ZZ(X,-)$ is a shiftable functor, the middle term here is induced, so the $\delta$ morphism
	\[\delta\colon\widehat H^0(G,\op{Hom}_\ZZ(X,A))\to\widehat H^1(G,\op{Hom}_\ZZ(X,I_G\otimes_\ZZ A))\]
	is an isomorphism.

	To finish, we claim that this $\delta$ morphism arises as a cup product. We simply show this by hand by tracking through the $\delta$ morphism. Given $[f]\in\widehat H^0(G,\op{Hom}_\ZZ(X,A))$ where $f\colon X\to A$ is a $G$-module homomorphism, we can pull this back to the $0$-chain $\widetilde f\colon X\to\ZZ[G]\otimes_\ZZ A$ by
	\[\widetilde f\colon x\mapsto 1\otimes f(x).\]
	Applying the differential, we get the $1$-cocycle $d\widetilde f\in B^1(G,\op{Hom}_\ZZ(X,\ZZ[G]\otimes_\ZZ A))$ defined by
	\begin{align*}
		(d\widetilde f)(g)(x) &= (g\widetilde f)(x)-\widetilde f(x) \\
		&= g\left(1\otimes f(g^{-1}x)\right)-(1\otimes f(x)) \\
		&= (g-1)\otimes f(x),
	\end{align*}
	which we know must be the $1$-cocycle $\delta([f])\in C^1(G,\op{Hom}_\ZZ(X,I_G\otimes_\ZZ A))$.

	Thus, we see that we should set $c\in\widehat H^1(G,I_G)$ to be represented by $g\mapsto(g-1)$. This will work as long as $g\mapsto(g-1)$ is a $1$-cocycle in $\widehat H^1(G,I_G)$. Well, take $X=A=\ZZ$ and $f=\id_\ZZ$ in the above argument so that $\delta(f)$ is exactly $g\mapsto(x\mapsto(g-1)\otimes x)$, which is $g\mapsto(g-1)$ after applying $\op{Hom}_\ZZ(\ZZ,I_G)\simeq I_G$.
\end{proof}
\begin{remark}
	Essentially the same proof will work when $\op{Hom}_\ZZ(X,-)$ is replaced by $X\otimes_\ZZ-$, or any composite of these. There isn't an analogue for arbitrary shiftable functors because, for example, there is no way obvious way to construct $\eta$ in general. Regardless, we will not need to work in these levels of generality.
\end{remark}
\begin{cor} \label{cor:cupdown}
	Let $G$ be a finite group. There exists $c\in\widehat H^1(G,I_G)$ such that, for any $G$-module $X$,
	\[c\cup-\colon\widehat H^i(G,\op{Hom}_\ZZ(X,\op{Hom}_\ZZ(I_G,-)))\Rightarrow\widehat H^{i+1}(G,\op{Hom}_\ZZ(X,-))\]
	is a natural isomorphism for any $i\in\ZZ$.
\end{cor}
\begin{proof}
	Similar to before, we are using the shifting pair $(\op{Hom}_\ZZ(X,\op{Hom}_\ZZ(I_G,-)),\op{Hom}_\ZZ(X,-),I_G,\eta)$, where
	\[\eta_A\colon I_G\otimes_\ZZ\op{Hom}_\ZZ(X,\op{Hom}_\ZZ(I_G,A))\Rightarrow\op{Hom}_\ZZ(X,-)\]
	is the canonical map sending $z\otimes f$ to $x\mapsto f(x)(z)$.

	Using \autoref{prop:dimshiftcupisos} and \autoref{lem:cuppingisnatural} again, it will suffice to find $c\in\widehat H^1(G,I_G)$ such that we have isomorphisms
	\[c\cup-\colon\widehat H^0(G,\op{Hom}_\ZZ(X,\op{Hom}_\ZZ(I_G,A)))\to\widehat H^1(G,\op{Hom}_\ZZ(X,A))\]
	for all $G$-modules $A$. This time around we use the ($\ZZ$-split) short exact sequence
	\[0\to\op{Hom}_\ZZ(X,A)\to\op{Hom}_\ZZ(X,\op{Hom}_\ZZ(\ZZ[G],A))\to\op{Hom}_\ZZ(X,\op{Hom}_\ZZ(I_G,A))\to0\]
	which will induce a boundary morphism
	\[\delta\colon\widehat H^0(G,\op{Hom}_\ZZ(X,\op{Hom}_\ZZ(I_G,A)))\to\widehat H^1(G,\op{Hom}_\ZZ(X,A)).\]
	In fact, this is an isomorphism because our middle term $\op{Hom}_\ZZ(X,\op{Hom}_\ZZ(\ZZ[G],A))$ is induced.

	We now show that $\delta$ is a cup product by hand. We start with some $[f]\in\widehat H^0(G,\op{Hom}_\ZZ(X,\op{Hom}_\ZZ(I_G,A)))$ where $f\colon X\to\op{Hom}_\ZZ(I_G,A)$ is a $G$-module homomorphism. This pulls back to the $0$-cochain
	\[\widetilde f\colon x\mapsto\big(z\mapsto f(x)(z-\varepsilon(z))\big).\]
	Applying the differential, we compute
	\begin{align*}
		(d\widetilde f)(g)(x)(z) &= (g\widetilde f-\widetilde f)(x)(z) \\
		&= (g\widetilde f)(x)(z)-\widetilde f(x)(z) \\
		&= \left(g\cdot\widetilde f\left(g^{-1}x\right)\right)(z)-\widetilde f(x)(z) \\
		&= g\cdot\widetilde f\left(g^{-1}x\right)\left(g^{-1}z\right)-\widetilde f(x)(z) \\
		&= g\cdot f\left(g^{-1}x\right)\left(g^{-1}z-\varepsilon(z)\right)-f(x)(z-\varepsilon(z)) \\
		&= g\cdot \left(g^{-1}f\left(x\right)\right)\left(g^{-1}z-\varepsilon(z)\right)-f(x)(z-\varepsilon(z)) \\
		&= f(x)\left(z-g\varepsilon(z)\right)-f(x)(z-\varepsilon(z)) \\
		&= \varepsilon(z)f(x)\left(1-g\right).
	\end{align*}
	Thus, this pulls back to the $1$-cocycle $g\mapsto(x\mapsto f(x)(1-g))$ in $\widehat H^1(G,\op{Hom}_\ZZ(X,A))$.
	
	In particular, we see that we should take $c$ represented by $g\mapsto(1-g)$, which will work as soon as we know that $g\mapsto(1-g)$ is a $1$-cocycle. Well, this is the negation of $g\mapsto(g-1)$ from the previous corollary. We close by remarking that we can actually take $c$ represented by $g\mapsto(g-1)$ because negating $c$ does not change the fact that the cup product gives an isomorphism.
\end{proof}
The point of the above two results is that have a somewhat general version of dimension-shifting granted by cup products. In fact, we see that we can use the same $c\in\widehat H^1(G,I_G)$ represented by $g\mapsto(g-1)$ for both shifting isomorphisms.

\subsection{Shifting Natural Transformations}
Observe that a natural transformation $F\Rightarrow F'$ of shiftable functors will induce natural transformations in cohomology
\[\widehat H^i(G,F-)\Rightarrow\widehat H^i(G,F'-)\]
It will turn out that, when $F=\op{Hom}_\ZZ(X,-)$ and $F'=\op{Hom}_\ZZ(X',-)$, we will be able to force all natural transformations in cohomology will come from natural transformations $F\Rightarrow F'$.

To begin, we show this result for $i=0$.
\begin{lemma} \label{lem:naturaltransiscupping}
	Let $G$ be a finite group, and let $X$ and $X'$ be $G$-modules. Suppose that, for given index $p\in\ZZ$, there is a natural transformation
	\[\Phi_\bullet\colon\widehat H^0(G,\op{Hom}_\ZZ(X,-))\Rightarrow\widehat H^p(G,\op{Hom}_\ZZ(X',-)).\]
	Then there exists $x\in\widehat H^p(G,\op{Hom}_\ZZ(X',X))$ such that $\Phi_\bullet=(x\cup-)$, where the cup product is induced by the shifting pair of \autoref{ex:shiftingpair}.
	% Then there exists a $G$-module homomorphism $\varphi\colon X'\to X$ such that $\Phi_A([f])=(-\circ\varphi)([f])$ for any $f\in\op{Hom}_{\ZZ[G]}(X,A)$.
\end{lemma}
\begin{proof}
	This is essentially the Yoneda lemma. As such, set $[x]\coloneqq\Phi_X([\id_X])$. The point is to fix some $G$-module $A$ and $[\overline f]\in\widehat H^0(G,\op{Hom}_\ZZ(X,A))$ in order to track through the commutativity of the following diagram.
	% https://q.uiver.app/?q=WzAsNCxbMCwwLCJcXHdpZGVoYXQgSF4wKEcsXFxvcHtIb219X1xcWlooWCxYKSkiXSxbMSwwLCJcXHdpZGVoYXQgSF4wKEcsXFxvcHtIb219X1xcWlooWCcsWCkpIl0sWzAsMSwiXFx3aWRlaGF0IEheMChHLFxcb3B7SG9tfV9cXFpaKFgsQSkpIl0sWzEsMSwiXFx3aWRlaGF0IEheMChHLFxcb3B7SG9tfV9cXFpaKFgnLEEpKSJdLFswLDEsIlxcUGhpX1giXSxbMCwyLCJmIiwyXSxbMSwzLCJmIiwyXSxbMiwzLCJcXFBoaV9BIl1d&macro_url=https%3A%2F%2Fraw.githubusercontent.com%2FdFoiler%2Fnotes%2Fmaster%2Fnir.tex
	\begin{equation}
		\begin{tikzcd}
			{\widehat H^0(G,\op{Hom}_\ZZ(X,X))} & {\widehat H^p(G,\op{Hom}_\ZZ(X',X))} \\
			{\widehat H^0(G,\op{Hom}_\ZZ(X,A))} & {\widehat H^p(G,\op{Hom}_\ZZ(X',A))}
			\arrow["{\Phi_X}", from=1-1, to=1-2]
			\arrow["\overline f"', from=1-1, to=2-1]
			\arrow["\overline f"', from=1-2, to=2-2]
			\arrow["{\Phi_A}", from=2-1, to=2-2]
		\end{tikzcd} \label{eq:cohomologicalyoneda}
	\end{equation}
	Because we will need to deal with the cup products with negative indices, we will use the standard resolution of \cite{cassels-frolich}. For example, we interpret $f\in[\overline f]\in\widehat H^0(G,\op{Hom}_\ZZ(X,A))$ as a constant function $f\in\op{Hom}_G(\ZZ[G],\op{Hom}_\ZZ(X,A))$ outputting $\overline f$, which means that $f(z)$ is the same $G$-module homomorphism for each $z\in\ZZ[G]$.

	As such, we can track the left arrow of \autoref{eq:cohomologicalyoneda} as
	\[\arraycolsep=1.4pt\begin{array}{cccc}
		\overline f\colon& \widehat H^0(G,\op{Hom}_\ZZ(X,X)) &\to& \widehat H^0(G,\op{Hom}_\ZZ(X,A)) \\
		& {[z\mapsto\id_X]} &\to& {[z\mapsto f(z)\circ\id_X]=[\overline f]}.
	\end{array}\]
	So, along the bottom of \autoref{eq:cohomologicalyoneda}, we are evaluating $\Phi_A([\overline f])$.

	Along the top of \autoref{eq:cohomologicalyoneda}, we immediately send $[z\mapsto\id_X]$ to $\Phi_X([z\mapsto\id_X])=[x]$, so to finish the proof, we need to show that
	\[\overline f([x])\stackrel?=[x]\cup[\overline f],\]
	which will be enough by the commutativity of \autoref{eq:cohomologicalyoneda}. We have two similar cases to appropriately deal with the cup product.
	\begin{itemize}
		\item Suppose that $p\ge0$ so that we can interpret $x$ as an element of $\op{Hom}_{\ZZ[G]}\left(\ZZ[G^{p+1}],X\right)$, using the standard resolution. As such, we compute
		\[(x\cup f)(g_0,\ldots,g_p) = x(g_0,\ldots,g_p) \otimes f(g_p),\]
		where our output is in $\op{Hom}_\ZZ(X',X)\otimes_\ZZ\op{Hom}_\ZZ(X,A)$. Applying evaluation, the cup product is outputting
		\[(g_0,\ldots,g_p)\mapsto(f(g_p)\circ x)(g_0,\ldots,g_p)\]
		as our element of $\op{Hom}_{\ZZ[G]}(\ZZ[G^{p+1}],A)$. Indeed, this morphism represents $\overline f([x])$.
		\item Analogously, suppose that $p<0$ so that we interpret $x$ as an element of $\op{Hom}_{\ZZ[G]}\left(\op{Hom}_\ZZ(\ZZ[G]^p,\ZZ),X\right)$. To decrease headaches, we let $g^*\colon\ZZ[G]\to\ZZ$ denote the $G$-module homomorphism sending $g\mapsto1$ and other group elements to $0$. Then $p$-tuples $(g_1^*,\ldots,g_p^*)$ form a $\ZZ$-basis of $\op{Hom}_\ZZ(\ZZ[G]^p,\ZZ)$, so it's enough to specify
		\[(x\cup f)(g_1^*,\ldots,g_p^*) = x(g_1^*,\ldots,g_p^*)\otimes f(g_p),\]
		where the output is in $\op{Hom}_\ZZ(X',X)\otimes_\ZZ\op{Hom}_\ZZ(X,A)$. Applying evaluation, the cup product is outputting
		\[(g_1^*,\ldots,g_p^*)\mapsto (f(g_p)\circ x)(g_1^*,\ldots,g_p^*)\]
		as an element of $\op{Hom}_{\ZZ[G]}\left(\op{Hom}_\ZZ(\ZZ[G]^p,\ZZ),A\right)$. Indeed, this represents $\overline f([x])$.
	\end{itemize}
	The above cases finish tracking through \autoref{eq:cohomologicalyoneda} and hence finish the proof.
	% This is essentially the Yoneda lemma. Choose $\varphi\colon X'\to X$ to represent $[\varphi]=\Phi_X([\id_X])$. Then tracking through the commutativity of
	% % https://q.uiver.app/?q=WzAsNCxbMCwwLCJcXHdpZGVoYXQgSF4wKEcsXFxvcHtIb219X1xcWlooWCxYKSkiXSxbMSwwLCJcXHdpZGVoYXQgSF4wKEcsXFxvcHtIb219X1xcWlooWCcsWCkpIl0sWzAsMSwiXFx3aWRlaGF0IEheMChHLFxcb3B7SG9tfV9cXFpaKFgsQSkpIl0sWzEsMSwiXFx3aWRlaGF0IEheMChHLFxcb3B7SG9tfV9cXFpaKFgnLEEpKSJdLFswLDEsIlxcUGhpX1giXSxbMCwyLCJmIiwyXSxbMSwzLCJmIiwyXSxbMiwzLCJcXFBoaV9BIl1d&macro_url=https%3A%2F%2Fraw.githubusercontent.com%2FdFoiler%2Fnotes%2Fmaster%2Fnir.tex
	% \[\begin{tikzcd}
	% 	{\widehat H^0(G,\op{Hom}_\ZZ(X,X))} & {\widehat H^0(G,\op{Hom}_\ZZ(X',X))} \\
	% 	{\widehat H^0(G,\op{Hom}_\ZZ(X,A))} & {\widehat H^0(G,\op{Hom}_\ZZ(X',A))}
	% 	\arrow["{\Phi_X}", from=1-1, to=1-2]
	% 	\arrow["f"', from=1-1, to=2-1]
	% 	\arrow["f"', from=1-2, to=2-2]
	% 	\arrow["{\Phi_A}", from=2-1, to=2-2]
	% \end{tikzcd}\]
	% reveals that
	% % https://q.uiver.app/?q=WzAsNCxbMCwwLCJbXFxpZF9YXSJdLFsxLDAsIltcXHZhcnBoaV0iXSxbMCwxLCJbZl0iXSxbMSwxLCJcXFBoaV9BKFtmXSk9W2ZcXGNpcmNcXHZhcnBoaV0iXSxbMCwxLCJcXFBoaV9YIiwwLHsic3R5bGUiOnsidGFpbCI6eyJuYW1lIjoibWFwcyB0byJ9fX1dLFswLDIsImYiLDIseyJzdHlsZSI6eyJ0YWlsIjp7Im5hbWUiOiJtYXBzIHRvIn19fV0sWzEsMywiZiIsMix7InN0eWxlIjp7InRhaWwiOnsibmFtZSI6Im1hcHMgdG8ifX19XSxbMiwzLCJcXFBoaV9BIiwwLHsic3R5bGUiOnsidGFpbCI6eyJuYW1lIjoibWFwcyB0byJ9fX1dXQ==&macro_url=https%3A%2F%2Fraw.githubusercontent.com%2FdFoiler%2Fnotes%2Fmaster%2Fnir.tex
	% \[\begin{tikzcd}
	% 	{[\id_X]} & {[\varphi]} \\
	% 	{[f]} & {\Phi_A([f])=[f\circ\varphi]}
	% 	\arrow["{\Phi_X}", maps to, from=1-1, to=1-2]
	% 	\arrow["f"', maps to, from=1-1, to=2-1]
	% 	\arrow["f"', maps to, from=1-2, to=2-2]
	% 	\arrow["{\Phi_A}", maps to, from=2-1, to=2-2]
	% \end{tikzcd}\]
	% after plugging in. To finish, we note that $\varphi([f])=[f\circ\varphi]$ under the induced map $\varphi\colon\op{Hom}_\ZZ(X,A)\to\op{Hom}_\ZZ(X',A)$. This finishes.
\end{proof}
The case of $p=0$ will be particularly interesting to us, so we note that the above proof gives it a more concrete description.
\begin{cor} \label{cor:cupiscomp}
	Let $G$ be a finite group, and let $X$ and $X'$ be $G$-modules. Then, given a $G$-module morphism $\varphi\colon X'\to X$, the maps $(-\circ\varphi)$ and $[\varphi]\cup-$ on
	\[\widehat H^i(G,\op{Hom}_\ZZ(X,-))\Rightarrow\widehat H^i(G,\op{Hom}_\ZZ(X',-))\]
	assemble into the same natural transformation.
\end{cor}
\begin{proof}
	This follows from unpacking the definitions.
	
	We already know that $[\varphi]\cup-$ is a natural transformation by \autoref{lem:cuppingisnatural}, so it suffices to show that the two maps agree on components. (Namely, naturality of $(-\circ\varphi)$ will immediately follow.) To see this, we note that the proof of \autoref{lem:naturaltransiscupping} above immediately computed for us that, given a $G$-module $A$, $[f]\in\widehat H^0(G,\op{Hom}_\ZZ(X,A))$ got sent to
	\[[f\circ\varphi]=f([\varphi])=[\varphi]\cup[f],\]
	which is what we wanted.
\end{proof}
We now get the main result by dimension-shifting.
\begin{prop} \label{prop:allnaturaltransarecups}
	Let $G$ be a finite group, and let $X$ and $X'$ be $G$-modules. Then, given indices $p,q\in\ZZ$, any natural transformation
	\[\Phi_\bullet^{(q)}\colon\widehat H^q(G,\op{Hom}_\ZZ(X,-))\Rightarrow\widehat H^{p+q}(G,\op{Hom}_\ZZ(X',-)),\]
	is $\Phi_\bullet^{(q)}=x\cup-$ for some $x\in\widehat H^p(G,\op{Hom}_\ZZ(X',X))$.
	% are natural transformations (respectively, isomorphisms)
	% \[\Phi_\bullet^{(i)}\colon\widehat H^i(G,F-)\Rightarrow\widehat H^i(G,F'-),\]
	% for all $i\in\ZZ$.
\end{prop}
% \begin{proof}
% 	This argument is by dimension-shifting the $p$ upwards and downwards. Namely, we show the conclusion of the statement by induction on $i$; for $i=p$, there is nothing to say. We will show how to induct downwards to $i\le p$ in detail, and inducting upwards is similar.

% 	To induct downwards, suppose the statement is true for $i+1$, and we show $i$, so fix a natural transformation
% 	\[\Phi_\bullet^{(i+1)}\colon\widehat H^i(G,\op{Hom}_\ZZ(X,-))\Rightarrow\widehat H^i(G,\op{Hom}_\ZZ(X',-)),\]
% 	which we would like to know arises as a cup product. The main idea is to use $\Phi_\bullet^{(i+1)}$ in order to construct $\Phi_\bullet^{(i)}$. Well, for any $G$-module $A$, we note that the two $\ZZ$-split short exact sequences
% 	\begin{equation}
% 		\arraycolsep=1.4pt\begin{array}{ccccccccc}
% 			1 &\to& F(I_G\otimes_\ZZ A) &\to& F(\ZZ[G]\otimes_\ZZ A) &\to& FA &\to& 1 \\
% 			1 &\to& F'(I_G\otimes_\ZZ A) &\to& F'(\ZZ[G]\otimes_\ZZ A) &\to& F'A &\to& 1
% 		\end{array} \label{eq:usualhomshiftingses}
% 	\end{equation}
% 	induce $\delta$ morphisms
% 	\[\arraycolsep=1.4pt\begin{array}{ccccccccc}
% 		\delta_{tA}\colon& \widehat H^i(G,FA) &\to& \widehat H^{i+1}(G,F'(I_G\otimes_\ZZ A)) \\
% 		\delta_{tA}'\colon& \widehat H^i(G,F'A) &\to& \widehat H^{i+1}(G,F'(I_G\otimes_\ZZ A))
% 	\end{array}\]
% 	which are in fact isomorphisms because the modules $\op{Hom}_\ZZ(X,\ZZ[G]\otimes_\ZZ A)$ and $\op{Hom}_\ZZ(X,\ZZ[G]\otimes_\ZZ A)$ are induced by \autoref{lem:hompreservesinduced}. As such, we have the diagram
% 	% https://q.uiver.app/?q=WzAsNCxbMCwwLCJcXHdpZGVoYXQgSF5pKEcsXFxvcHtIb219X1xcWlooWCxBKSkiXSxbMSwwLCJcXHdpZGVoYXQgSF57aSsxfShHLFxcb3B7SG9tfV9cXFpaKFgsSV9HXFxvdGltZXNfXFxaWiBBKSkiXSxbMCwxLCJcXHdpZGVoYXQgSF5pKEcsXFxvcHtIb219X1xcWlooWCcsQSkpIl0sWzEsMSwiXFx3aWRlaGF0IEhee2krMX0oRyxcXG9we0hvbX1fXFxaWihYJyxJX0dcXG90aW1lc19cXFpaIEEpKSJdLFswLDEsIlxcZGVsdGFfWCJdLFsxLDMsIlxcUGhpXnsoaSsxKX1fe0lfR1xcb3RpbWVzX1xcWlogQX0iXSxbMiwzLCJcXGRlbHRhX3tYJ30iXSxbMCwyLCIiLDAseyJzdHlsZSI6eyJib2R5Ijp7Im5hbWUiOiJkYXNoZWQifX19XV0=&macro_url=https%3A%2F%2Fraw.githubusercontent.com%2FdFoiler%2Fnotes%2Fmaster%2Fnir.tex
% 	\[\begin{tikzcd}
% 		{\widehat H^i(G,FA)} & {\widehat H^{i+1}(G,F(I_G\otimes_\ZZ A))} \\
% 		{\widehat H^i(G,F'A)} & {\widehat H^{i+1}(G,F'(I_G\otimes_\ZZ A))}
% 		\arrow["{\delta_{tA}}", from=1-1, to=1-2]
% 		\arrow["{\Phi^{(i+1)}_{I_G\otimes_\ZZ A}}", from=1-2, to=2-2]
% 		\arrow["{\delta_{tA}'}", from=2-1, to=2-2]
% 		\arrow[dashed, from=1-1, to=2-1]
% 	\end{tikzcd}\]
% 	where the horizontal arrows are isomorphisms. Thus, we induce a morphism
% 	\[\Phi_A^{(i)}\coloneqq(\delta_{tA}')^{-1}\circ\Phi^{(i+1)}_{I_G\otimes_\ZZ A}\circ\delta_{tA}.\]
% 	We claim that $\Phi_\bullet^{(i)}$ assembles into a natural transformation $\widehat H^i(G,F-)\Rightarrow\widehat H^i(G,F'-)$. For this, we must check naturality. Suppose that we have a morphism $f\colon A\to B$. This gives rise to the following diagram. 
% 	% https://q.uiver.app/?q=WzAsOCxbMSwwLCJcXHdpZGVoYXQgSF5pKEcsXFxvcHtIb219X1xcWlooWCxBKSkiXSxbMywwLCJcXHdpZGVoYXQgSF57aSsxfShHLFxcb3B7SG9tfV9cXFpaKFgsSV9HXFxvdGltZXNfXFxaWiBBKSkiXSxbMywyLCJcXHdpZGVoYXQgSF57aSsxfShHLFxcb3B7SG9tfV9cXFpaKFgnLElfR1xcb3RpbWVzX1xcWlogQSkpIl0sWzEsMiwiXFx3aWRlaGF0IEheaShHLFxcb3B7SG9tfV9cXFpaKFgnLEEpKSJdLFswLDEsIlxcd2lkZWhhdCBIXmkoRyxcXG9we0hvbX1fXFxaWihYLEIpKSJdLFsyLDEsIlxcd2lkZWhhdCBIXntpKzF9KEcsXFxvcHtIb219X1xcWlooWCxJX0dcXG90aW1lc19cXFpaIEIpKSJdLFswLDMsIlxcd2lkZWhhdCBIXmkoRyxcXG9we0hvbX1fXFxaWihYJyxCKSkiXSxbMiwzLCJcXHdpZGVoYXQgSF57aSsxfShHLFxcb3B7SG9tfV9cXFpaKFgnLElfR1xcb3RpbWVzX1xcWlogQikpIl0sWzAsMSwiXFxkZWx0YV97aEF9IiwwLHsibGFiZWxfcG9zaXRpb24iOjIwfV0sWzMsMiwiXFxkZWx0YV97aEF9JyIsMCx7ImxhYmVsX3Bvc2l0aW9uIjoyMH1dLFs0LDUsIlxcZGVsdGFfe2hCfSIsMSx7ImxhYmVsX3Bvc2l0aW9uIjoyMH1dLFs2LDcsIlxcZGVsdGFfe2hCfSciLDAseyJsYWJlbF9wb3NpdGlvbiI6MjB9XSxbMCwzLCJcXFBoaV9BXnsoaSl9IiwxLHsibGFiZWxfcG9zaXRpb24iOjIwLCJzdHlsZSI6eyJib2R5Ijp7Im5hbWUiOiJkYXNoZWQifX19XSxbNCw2LCJcXFBoaV9CXnsoaSl9IiwxLHsibGFiZWxfcG9zaXRpb24iOjIwLCJzdHlsZSI6eyJib2R5Ijp7Im5hbWUiOiJkYXNoZWQifX19XSxbMCw0LCJmIiwxXSxbMSw1LCJmIiwxXSxbMiw3LCJmIiwxXSxbNSw3LCJcXFBoaV9CXnsoaSsxKX0iLDEseyJsYWJlbF9wb3NpdGlvbiI6MjB9XSxbMSwyLCJcXFBoaV9BXnsoaSsxKX0iLDEseyJsYWJlbF9wb3NpdGlvbiI6MjB9XSxbMyw2LCJmIiwxXV0=&macro_url=https%3A%2F%2Fraw.githubusercontent.com%2FdFoiler%2Fnotes%2Fmaster%2Fnir.tex
% 	\[\begin{tikzcd}[column sep={3cm,between origins}]
% 		& {\widehat H^i(G,FA)} && {\widehat H^{i+1}(G,F(I_G\otimes_\ZZ A))} \\
% 		{\widehat H^i(G,FB)} && {\widehat H^{i+1}(G,F(I_G\otimes_\ZZ B))} \\
% 		& {\widehat H^i(G,F'A)} && {\widehat H^{i+1}(G,F'(I_G\otimes_\ZZ A))} \\
% 		{\widehat H^i(G,F'B)} && {\widehat H^{i+1}(G,F'(I_G\otimes_\ZZ B))}
% 		\arrow["{\delta_{tA}}"{description, pos=0.2}, from=1-2, to=1-4]
% 		\arrow["{\delta_{tA}'}"{description, pos=0.2}, from=3-2, to=3-4]
% 		\arrow["{\delta_{tB}}"{description, pos=0.2}, from=2-1, to=2-3]
% 		\arrow["{\delta_{tB}'}"{description, pos=0.2}, from=4-1, to=4-3]
% 		\arrow["{\Phi_A^{(i)}}"{description, pos=0.2}, dashed, from=1-2, to=3-2]
% 		\arrow["{\Phi_B^{(i)}}"{description, pos=0.2}, dashed, from=2-1, to=4-1]
% 		\arrow["Ff"{description}, from=1-2, to=2-1]
% 		\arrow["Ff"{description}, from=1-4, to=2-3]
% 		\arrow["F'f"{description}, from=3-4, to=4-3]
% 		\arrow["{\Phi_{I_G\otimes_\ZZ B}^{(i+1)}}"{description, pos=0.2}, from=2-3, to=4-3]
% 		\arrow["{\Phi_{I_G\otimes_\ZZ A}^{(i+1)}}"{description, pos=0.2}, from=1-4, to=3-4]
% 		\arrow["F'f"{description}, from=3-2, to=4-1]
% 	\end{tikzcd}\]
% 	We want to show that the left face commutes. For this, we note that all the horizontal arrows are isomorphisms (they're the $\delta$s from before), so it suffices to show that the rest of the cube commutes.
% 	\begin{itemize}
% 		\item The top face commutes by functoriality of $\delta$ morphsims applied to the following morphism of short exact sequences.
% 		% https://q.uiver.app/?q=WzAsMTAsWzAsMCwiMCJdLFsxLDAsIlxcb3B7SG9tfV9cXFpaKFgsSV9HXFxvdGltZXNfXFxaWiBBKSJdLFsyLDAsIlxcb3B7SG9tfV9cXFpaKFgsXFxaWltHXVxcb3RpbWVzX1xcWlogQSkiXSxbMywwLCJcXG9we0hvbX1fXFxaWihYLEEpIl0sWzQsMCwiMCJdLFsxLDEsIlxcb3B7SG9tfV9cXFpaKFgsSV9HXFxvdGltZXNfXFxaWiBCKSJdLFsyLDEsIlxcb3B7SG9tfV9cXFpaKFgsXFxaWltHXVxcb3RpbWVzX1xcWlogQikiXSxbMywxLCJcXG9we0hvbX1fXFxaWihYLEIpIl0sWzQsMSwiMCJdLFswLDEsIjAiXSxbMCwxXSxbOSw1XSxbMSwyXSxbMiwzXSxbMyw0XSxbNSw2XSxbNiw3XSxbNyw4XSxbMSw1LCJmIl0sWzIsNiwiZiJdLFszLDcsImYiXV0=&macro_url=https%3A%2F%2Fraw.githubusercontent.com%2FdFoiler%2Fnotes%2Fmaster%2Fnir.tex
% 		\[\begin{tikzcd}
% 			0 & {F(I_G\otimes_\ZZ A)} & {F(\ZZ[G]\otimes_\ZZ A)} & {FA} & 0 \\
% 			0 & {F(I_G\otimes_\ZZ B)} & {F(\ZZ[G]\otimes_\ZZ B)} & {FB} & 0
% 			\arrow[from=1-1, to=1-2]
% 			\arrow[from=2-1, to=2-2]
% 			\arrow[from=1-2, to=1-3]
% 			\arrow[from=1-3, to=1-4]
% 			\arrow[from=1-4, to=1-5]
% 			\arrow[from=2-2, to=2-3]
% 			\arrow[from=2-3, to=2-4]
% 			\arrow[from=2-4, to=2-5]
% 			\arrow["Ff", from=1-2, to=2-2]
% 			\arrow["Ff", from=1-3, to=2-3]
% 			\arrow["Ff", from=1-4, to=2-4]
% 		\end{tikzcd}\]
% 		The bottom face commutes for the same reason, replacing $F$s with $F'$s in the above morphism of short exact sequences.
% 		\item The front and back faces commute by definition of the morphisms $\Phi_\bullet^{(i)}$.
% 		\item The right face commutes by naturality of $\Phi^{(i+1)}_\bullet$ applied to the induced morphism $f\colon I_G\otimes_\ZZ A\to I_G\otimes_\ZZ B$.
% 	\end{itemize}
% 	The above commutativity checks complete the proof that $\Phi_\bullet^{(i)}$ makes a natural transformation. To finish, we note that, if $\Phi_\bullet^{(i+1)}$ is a natural isomorphism, then $\Phi_\bullet^{(i)}$ is a natural isomorphism as well by its construction. This completes the induction downwards.

% 	We will not give detail for the induction upwards from $i-1$ to $i$, except to say that the short exact sequences \autoref{eq:usualhomshiftingses} are replaced with the following.
% 	\[\arraycolsep=1.4pt\begin{array}{ccccccccc}
% 		1 &\to& FA &\to& F(\op{Hom}_\ZZ(\ZZ[G],A)) &\to& F(\op{Hom}_\ZZ(I_G,A)) &\to& 1 \\
% 		1 &\to& F'A &\to& F'(\op{Hom}_\ZZ(\ZZ[G],A)) &\to& F'(\op{Hom}_\ZZ(I_G,A)) &\to& 1
% 	\end{array}\]
% 	The rest of the approach essentially goes through verbatim, constructing $\Phi_\bullet^{(i)}$ from a given $\Phi_\bullet^{(i-1)}$.
% \end{proof}
\begin{proof}
	This argument is by dimension-shifting the $q$ upwards and downwards. Namely, we show the conclusion of the statement by induction on $i$; for $i=q$, there is nothing to say. We will show how to induct upwards to $i\ge q$ in detail, and inducting downwards is similar. For brevity, we set $F\coloneqq\op{Hom}_\ZZ(X,-)$ and $F'\coloneqq\op{Hom}_\ZZ(X',-)$.

	To induct upwards, suppose the statement is true for $i$, and we show $i+1$, so fix a natural transformation
	\[\Phi_\bullet^{(i+1)}\colon\widehat H^{i+1}(G,F-)\Rightarrow\widehat H^{p+i+1}(G,F'-),\]
	which we would like to know arises as $x\cup-$ for some $x\in\widehat H^p(G,\op{Hom}_\ZZ(X',X))$. The main idea is to use $\Phi_\bullet^{(i+1)}$ in order to construct $\Phi_\bullet^{(i)}$. Well, using \autoref{cor:cupup}, we have some $c\in\widehat H^1(G,I_G)$ given by $g\mapsto(g-1)$ yielding the following isomorphisms for any $G$-module $A$.
	% Well, for any $G$-module $A$, we note that the two $\ZZ$-split short exact sequences
	% \begin{equation}
	% 	\arraycolsep=1.4pt\begin{array}{ccccccccc}
	% 		1 &\to& F(I_G\otimes_\ZZ A) &\to& F(\ZZ[G]\otimes_\ZZ A) &\to& FA &\to& 1 \\
	% 		1 &\to& F'(I_G\otimes_\ZZ A) &\to& F'(\ZZ[G]\otimes_\ZZ A) &\to& F'A &\to& 1
	% 	\end{array} \label{eq:usualhomshiftingses}
	% \end{equation}
	% induce $\delta$ morphisms
	\[\arraycolsep=1.4pt\begin{array}{ccccccccc}
		(c\cup-)_d\colon& \widehat H^i(G,FA) &\to& \widehat H^{i+1}(G,F(I_G\otimes_\ZZ A)) \\
		(c\cup-)'_d\colon& \widehat H^{p+i}(G,F'A) &\to& \widehat H^{p+i+1}(G,F'(I_G\otimes_\ZZ A))
	\end{array}\]
	As such, we have the diagram
	% https://q.uiver.app/?q=WzAsNCxbMCwwLCJcXHdpZGVoYXQgSF5pKEcsXFxvcHtIb219X1xcWlooWCxBKSkiXSxbMSwwLCJcXHdpZGVoYXQgSF57aSsxfShHLFxcb3B7SG9tfV9cXFpaKFgsSV9HXFxvdGltZXNfXFxaWiBBKSkiXSxbMCwxLCJcXHdpZGVoYXQgSF5pKEcsXFxvcHtIb219X1xcWlooWCcsQSkpIl0sWzEsMSwiXFx3aWRlaGF0IEhee2krMX0oRyxcXG9we0hvbX1fXFxaWihYJyxJX0dcXG90aW1lc19cXFpaIEEpKSJdLFswLDEsIlxcZGVsdGFfWCJdLFsxLDMsIlxcUGhpXnsoaSsxKX1fe0lfR1xcb3RpbWVzX1xcWlogQX0iXSxbMiwzLCJcXGRlbHRhX3tYJ30iXSxbMCwyLCIiLDAseyJzdHlsZSI6eyJib2R5Ijp7Im5hbWUiOiJkYXNoZWQifX19XV0=&macro_url=https%3A%2F%2Fraw.githubusercontent.com%2FdFoiler%2Fnotes%2Fmaster%2Fnir.tex
	\[\begin{tikzcd}
		{\widehat H^i(G,FA)} & {\widehat H^{i+1}(G,F(I_G\otimes_\ZZ A))} \\
		{\widehat H^{p+i}(G,F'A)} & {\widehat H^{p+i+1}(G,F'(I_G\otimes_\ZZ A))}
		\arrow["{(c\cup-)_d}", from=1-1, to=1-2]
		\arrow["{\Phi^{(i+1)}_{I_G\otimes_\ZZ A}}", from=1-2, to=2-2]
		\arrow["{(c\cup-)'_d}", from=2-1, to=2-2]
		\arrow[dashed, from=1-1, to=2-1]
	\end{tikzcd}\]
	where the horizontal arrows are isomorphisms. Thus, we induce a morphism
	\[\Phi_A^{(i)}\coloneqq((c\cup-)'_d)^{-1}\circ\Phi^{(i+1)}_{I_G\otimes_\ZZ A}\circ(c\cup-)_d.\]
	Note that $\Phi_\bullet^{(i)}$ is the composition of natural transformations (the cup product is a natural transformation by construction) and therefore is a natural transformation.
	% We claim that $\Phi_\bullet^{(i)}$ assembles into a natural transformation $\widehat H^i(G,F-)\Rightarrow\widehat H^i(G,F'-)$. For this, we must check naturality. Suppose that we have a morphism $f\colon A\to B$. This gives rise to the following diagram.
	% % https://q.uiver.app/?q=WzAsOCxbMSwwLCJcXHdpZGVoYXQgSF5pKEcsXFxvcHtIb219X1xcWlooWCxBKSkiXSxbMywwLCJcXHdpZGVoYXQgSF57aSsxfShHLFxcb3B7SG9tfV9cXFpaKFgsSV9HXFxvdGltZXNfXFxaWiBBKSkiXSxbMywyLCJcXHdpZGVoYXQgSF57aSsxfShHLFxcb3B7SG9tfV9cXFpaKFgnLElfR1xcb3RpbWVzX1xcWlogQSkpIl0sWzEsMiwiXFx3aWRlaGF0IEheaShHLFxcb3B7SG9tfV9cXFpaKFgnLEEpKSJdLFswLDEsIlxcd2lkZWhhdCBIXmkoRyxcXG9we0hvbX1fXFxaWihYLEIpKSJdLFsyLDEsIlxcd2lkZWhhdCBIXntpKzF9KEcsXFxvcHtIb219X1xcWlooWCxJX0dcXG90aW1lc19cXFpaIEIpKSJdLFswLDMsIlxcd2lkZWhhdCBIXmkoRyxcXG9we0hvbX1fXFxaWihYJyxCKSkiXSxbMiwzLCJcXHdpZGVoYXQgSF57aSsxfShHLFxcb3B7SG9tfV9cXFpaKFgnLElfR1xcb3RpbWVzX1xcWlogQikpIl0sWzAsMSwiXFxkZWx0YV97aEF9IiwwLHsibGFiZWxfcG9zaXRpb24iOjIwfV0sWzMsMiwiXFxkZWx0YV97aEF9JyIsMCx7ImxhYmVsX3Bvc2l0aW9uIjoyMH1dLFs0LDUsIlxcZGVsdGFfe2hCfSIsMSx7ImxhYmVsX3Bvc2l0aW9uIjoyMH1dLFs2LDcsIlxcZGVsdGFfe2hCfSciLDAseyJsYWJlbF9wb3NpdGlvbiI6MjB9XSxbMCwzLCJcXFBoaV9BXnsoaSl9IiwxLHsibGFiZWxfcG9zaXRpb24iOjIwLCJzdHlsZSI6eyJib2R5Ijp7Im5hbWUiOiJkYXNoZWQifX19XSxbNCw2LCJcXFBoaV9CXnsoaSl9IiwxLHsibGFiZWxfcG9zaXRpb24iOjIwLCJzdHlsZSI6eyJib2R5Ijp7Im5hbWUiOiJkYXNoZWQifX19XSxbMCw0LCJmIiwxXSxbMSw1LCJmIiwxXSxbMiw3LCJmIiwxXSxbNSw3LCJcXFBoaV9CXnsoaSsxKX0iLDEseyJsYWJlbF9wb3NpdGlvbiI6MjB9XSxbMSwyLCJcXFBoaV9BXnsoaSsxKX0iLDEseyJsYWJlbF9wb3NpdGlvbiI6MjB9XSxbMyw2LCJmIiwxXV0=&macro_url=https%3A%2F%2Fraw.githubusercontent.com%2FdFoiler%2Fnotes%2Fmaster%2Fnir.tex
	% \[\begin{tikzcd}[column sep={3cm,between origins}]
	% 	& {\widehat H^i(G,FA)} && {\widehat H^{i+1}(G,F(I_G\otimes_\ZZ A))} \\
	% 	{\widehat H^i(G,FB)} && {\widehat H^{i+1}(G,F(I_G\otimes_\ZZ B))} \\
	% 	& {\widehat H^i(G,F'A)} && {\widehat H^{i+1}(G,F'(I_G\otimes_\ZZ A))} \\
	% 	{\widehat H^i(G,F'B)} && {\widehat H^{i+1}(G,F'(I_G\otimes_\ZZ B))}
	% 	\arrow["{\delta_{tA}}"{description, pos=0.2}, from=1-2, to=1-4]
	% 	\arrow["{\delta_{tA}'}"{description, pos=0.2}, from=3-2, to=3-4]
	% 	\arrow["{\delta_{tB}}"{description, pos=0.2}, from=2-1, to=2-3]
	% 	\arrow["{\delta_{tB}'}"{description, pos=0.2}, from=4-1, to=4-3]
	% 	\arrow["{\Phi_A^{(i)}}"{description, pos=0.2}, dashed, from=1-2, to=3-2]
	% 	\arrow["{\Phi_B^{(i)}}"{description, pos=0.2}, dashed, from=2-1, to=4-1]
	% 	\arrow["Ff"{description}, from=1-2, to=2-1]
	% 	\arrow["Ff"{description}, from=1-4, to=2-3]
	% 	\arrow["F'f"{description}, from=3-4, to=4-3]
	% 	\arrow["{\Phi_{I_G\otimes_\ZZ B}^{(i+1)}}"{description, pos=0.2}, from=2-3, to=4-3]
	% 	\arrow["{\Phi_{I_G\otimes_\ZZ A}^{(i+1)}}"{description, pos=0.2}, from=1-4, to=3-4]
	% 	\arrow["F'f"{description}, from=3-2, to=4-1]
	% \end{tikzcd}\]
	% We want to show that the left face commutes. For this, we note that all the horizontal arrows are isomorphisms (they're the $\delta$s from before), so it suffices to show that the rest of the cube commutes.
	% \begin{itemize}
	% 	\item The top face commutes by functoriality of $\delta$ morphsims applied to the following morphism of short exact sequences.
	% 	% https://q.uiver.app/?q=WzAsMTAsWzAsMCwiMCJdLFsxLDAsIlxcb3B7SG9tfV9cXFpaKFgsSV9HXFxvdGltZXNfXFxaWiBBKSJdLFsyLDAsIlxcb3B7SG9tfV9cXFpaKFgsXFxaWltHXVxcb3RpbWVzX1xcWlogQSkiXSxbMywwLCJcXG9we0hvbX1fXFxaWihYLEEpIl0sWzQsMCwiMCJdLFsxLDEsIlxcb3B7SG9tfV9cXFpaKFgsSV9HXFxvdGltZXNfXFxaWiBCKSJdLFsyLDEsIlxcb3B7SG9tfV9cXFpaKFgsXFxaWltHXVxcb3RpbWVzX1xcWlogQikiXSxbMywxLCJcXG9we0hvbX1fXFxaWihYLEIpIl0sWzQsMSwiMCJdLFswLDEsIjAiXSxbMCwxXSxbOSw1XSxbMSwyXSxbMiwzXSxbMyw0XSxbNSw2XSxbNiw3XSxbNyw4XSxbMSw1LCJmIl0sWzIsNiwiZiJdLFszLDcsImYiXV0=&macro_url=https%3A%2F%2Fraw.githubusercontent.com%2FdFoiler%2Fnotes%2Fmaster%2Fnir.tex
	% 	\[\begin{tikzcd}
	% 		0 & {F(I_G\otimes_\ZZ A)} & {F(\ZZ[G]\otimes_\ZZ A)} & {FA} & 0 \\
	% 		0 & {F(I_G\otimes_\ZZ B)} & {F(\ZZ[G]\otimes_\ZZ B)} & {FB} & 0
	% 		\arrow[from=1-1, to=1-2]
	% 		\arrow[from=2-1, to=2-2]
	% 		\arrow[from=1-2, to=1-3]
	% 		\arrow[from=1-3, to=1-4]
	% 		\arrow[from=1-4, to=1-5]
	% 		\arrow[from=2-2, to=2-3]
	% 		\arrow[from=2-3, to=2-4]
	% 		\arrow[from=2-4, to=2-5]
	% 		\arrow["Ff", from=1-2, to=2-2]
	% 		\arrow["Ff", from=1-3, to=2-3]
	% 		\arrow["Ff", from=1-4, to=2-4]
	% 	\end{tikzcd}\]
	% 	The bottom face commutes for the same reason, replacing $F$s with $F'$s in the above morphism of short exact sequences.
	% 	\item The front and back faces commute by definition of the morphisms $\Phi_\bullet^{(i)}$.
	% 	\item The right face commutes by naturality of $\Phi^{(i+1)}_\bullet$ applied to the induced morphism $f\colon I_G\otimes_\ZZ A\to I_G\otimes_\ZZ B$.
	% \end{itemize}
	% The above commutativity checks complete the proof that $\Phi_\bullet^{(i)}$ makes a natural transformation. To finish, we note that, if $\Phi_\bullet^{(i+1)}$ is a natural isomorphism, then $\Phi_\bullet^{(i)}$ is a natural isomorphism as well by its construction. This completes the induction downwards.

	Thus, the inductive hypothesis now tells us that $\Phi_\bullet^{(i)}=(x\cup-)$ for some $x\in\widehat H^p(G,\op{Hom}_\ZZ(X',X))$. We now need to turn this around on $\Phi_\bullet^{(i+1)}$, which essentially means we need to shift back in the other direction. As such, we use \autoref{cor:cupdown} to give the following isomorphisms for any $G$-module $A$.
	\[\arraycolsep=1.4pt\begin{array}{ccccccccc}
		(c\cup-)_u\colon&\widehat H^i(G,F(\op{Hom}_\ZZ(I_G,A)))\to\widehat H^{i+1}(G,FA) \\
		(c\cup-)_u'\colon&\widehat H^{p+i}(G,F(\op{Hom}_\ZZ(I_G,A)))\to\widehat H^{p+i+1}(G,FA)
	\end{array}\]
	Now, to deal with $\Phi_\bullet^{(i+1)}$, we note that associativity and commutativity of cup products implies $\left((-1)^px\cup-\right)$ can be used to make the right arrow in the diagram
	% https://q.uiver.app/?q=WzAsNCxbMCwwLCJcXHdpZGVoYXQgSF5pKEcsXFxvcHtIb219X1xcWlooWCxcXG9we0hvbX1fXFxaWihJX0csQSkpKSJdLFsxLDAsIlxcd2lkZWhhdCBIXntpKzF9KEcsXFxvcHtIb219X1xcWlooWCxBKSkiXSxbMCwxLCJcXHdpZGVoYXQgSF5pKEcsXFxvcHtIb219X1xcWlooWCcsXFxvcHtIb219X1xcWlooSV9HLEEpKSkiXSxbMSwxLCJcXHdpZGVoYXQgSF57aSsxfShHLFxcb3B7SG9tfV9cXFpaKFgnLEEpKSJdLFswLDEsIihjXFxjdXAtKV91Il0sWzIsMywiKGNcXGN1cC0pX3UnIl0sWzAsMiwiKFxcdmFycGhpXFxjaXJjLSkiLDJdLFsxLDMsIiIsMSx7InN0eWxlIjp7ImJvZHkiOnsibmFtZSI6ImRhc2hlZCJ9fX1dXQ==&macro_url=https%3A%2F%2Fraw.githubusercontent.com%2FdFoiler%2Fnotes%2Fmaster%2Fnir.tex
	\[\begin{tikzcd}
		{\widehat H^i(G,F(\op{Hom}_\ZZ(I_G,A)))} & {\widehat H^{i+1}(G,FA)} \\
		{\widehat H^{p+i}(G,F'(\op{Hom}_\ZZ(I_G,A)))} & {\widehat H^{p+i+1}(G,F'A)}
		\arrow["{(c\cup-)_u}", from=1-1, to=1-2]
		\arrow["{(c\cup-)_u'}", from=2-1, to=2-2]
		\arrow["{(x\cup-)}"', from=1-1, to=2-1]
		\arrow[dashed, from=1-2, to=2-2]
	\end{tikzcd}\]
	commute; technically, we ought to expand out this diagram to use the associativity and commutativity of the cup product for this diagram to commute, but we won't bother.
	
	Now, this right arrow is unique because the horizontal arrows are isomorphisms, so we will be done if we can show that we can place $\Phi_A^{(i+1)}$ in the right arrow to also make the diagram commute. For this, we draw the following very large diagram.
	% https://q.uiver.app/?q=WzAsNixbMCwwLCJcXHdpZGVoYXQgSF5pKEcsRihcXG9we0hvbX1fXFxaWihJX0csQSkpKSJdLFsxLDEsIlxcd2lkZWhhdCBIXntpKzF9KEcsRihJX0dcXG90aW1lc19cXFpaXFxvcHtIb219X1xcWlooSV9HLEEpKSkiXSxbMiwwLCJcXHdpZGVoYXQgSF57aSsxfShHLEZBKSJdLFswLDMsIlxcd2lkZWhhdCBIXmkoRyxGJyhcXG9we0hvbX1fXFxaWihJX0csQSkpKSJdLFsxLDQsIlxcd2lkZWhhdCBIXntpKzF9KEcsRicoSV9HXFxvdGltZXNfXFxaWlxcb3B7SG9tfV9cXFpaKElfRyxBKSkpIl0sWzIsMywiXFx3aWRlaGF0IEhee2krMX0oRyxGJ0EpIl0sWzAsMiwiKGNcXGN1cC0pX3UiLDEseyJsYWJlbF9wb3NpdGlvbiI6MjB9XSxbMCwxLCIoY1xcY3VwLSlfZCIsMV0sWzEsMiwiZiIsMV0sWzAsMywiKC1cXGNpcmNcXHZhcnBoaSkiLDEseyJsYWJlbF9wb3NpdGlvbiI6MzB9XSxbMSw0LCJcXFBoaV57KGkrMSl9X3tJX0dcXG90aW1lc19cXFpaXFxvcHtIb219X1xcWlooSV9HLEEpfSIsMSx7ImxhYmVsX3Bvc2l0aW9uIjozMH1dLFsyLDUsIlxcUGhpXnsoaSsxKX1fQSIsMSx7ImxhYmVsX3Bvc2l0aW9uIjozMH1dLFszLDUsIihjXFxjdXAtKV91JyIsMSx7ImxhYmVsX3Bvc2l0aW9uIjoyMH1dLFszLDQsIihjXFxjdXAtKV9kJyIsMV0sWzQsNSwiZiIsMV1d&macro_url=https%3A%2F%2Fraw.githubusercontent.com%2FdFoiler%2Fnotes%2Fmaster%2Fnir.tex
	\[\begin{tikzcd}
		{\widehat H^i(G,F(\op{Hom}_\ZZ(I_G,A)))} && {\widehat H^{i+1}(G,FA)} \\
		& {\widehat H^{i+1}(G,F(I_G\otimes_\ZZ\op{Hom}_\ZZ(I_G,A)))} \\
		\\
		{\widehat H^{p+i}(G,F'(\op{Hom}_\ZZ(I_G,A)))} && {\widehat H^{p+i+1}(G,F'A)} \\
		& {\widehat H^{p+i+1}(G,F'(I_G\otimes_\ZZ\op{Hom}_\ZZ(I_G,A)))}
		\arrow["{(c\cup-)_u}"{description, pos=0.2}, from=1-1, to=1-3]
		\arrow["{(c\cup-)_d}"{description}, from=1-1, to=2-2]
		\arrow["f"{description}, from=2-2, to=1-3]
		\arrow["{(x\cup-)}"{description, pos=0.3}, from=1-1, to=4-1]
		\arrow["{\Phi^{(i+1)}_{I_G\otimes_\ZZ\op{Hom}_\ZZ(I_G,A)}}"{description, pos=0.3}, from=2-2, to=5-2]
		\arrow["{\Phi^{(i+1)}_A}"{description, pos=0.3}, from=1-3, to=4-3]
		\arrow["{(c\cup-)_u'}"{description, pos=0.2}, from=4-1, to=4-3]
		\arrow["{(c\cup-)_d'}"{description}, from=4-1, to=5-2]
		\arrow["f"{description}, from=5-2, to=4-3]
	\end{tikzcd}\]
	Here, the $f$ maps are induced by the evaluation map
	\[f\colon I_G\otimes_\ZZ\op{Hom}_\ZZ(I_G,A)\to A.\]
	We want the outer rectangle to commute, for which it suffices to show that each parallelogram and the small top and bottom triangles to commute.
	\begin{itemize}
		\item The left parallelogram commutes by definition of $\Phi_A^{(i)}$.
		\item The right parallelogram commutes by naturality of $\Phi^{(i+1)}_\bullet$.
		\item Showing that the bottom triangle commutes will be analogous to showing that the top triangle commutes, so we will only show the top. Unwinding \autoref{cor:cupup} and \autoref{cor:cupdown}, we see that this triangle is actually induced by the following diagram.
		% https://q.uiver.app/?q=WzAsNCxbMCwwLCJcXHdpZGVoYXQgSF5pKEcsRihcXG9we0hvbX1fXFxaWihJX0csQSkpKSJdLFsxLDAsIlxcd2lkZWhhdCBIXmkoRyxJX0dcXG90aW1lc19cXFpaIEYoXFxvcHtIb219X1xcWlooSV9HLEEpKSkiXSxbMSwxLCJcXHdpZGVoYXQgSF5pKEcsRihJX0dcXG90aW1lc19cXFpaXFxvcHtIb219X1xcWlooSV9HLEEpKSkiXSxbMiwwLCJcXHdpZGVoYXQgSF5pKEcsRkEpIl0sWzAsMSwiY1xcY3VwLSJdLFsyLDMsImYiLDJdLFsxLDIsIlxcZXRhX2QiLDJdLFsxLDMsIlxcZXRhX3UiXV0=&macro_url=https%3A%2F%2Fraw.githubusercontent.com%2FdFoiler%2Fnotes%2Fmaster%2Fnir.tex
		\[\begin{tikzcd}
			{\widehat H^i(G,F(\op{Hom}_\ZZ(I_G,A)))} & {\widehat H^{i+1}(G,I_G\otimes_\ZZ F(\op{Hom}_\ZZ(I_G,A)))} & {\widehat H^{i+1}(G,FA)} \\
			& {\widehat H^{i+1}(G,F(I_G\otimes_\ZZ\op{Hom}_\ZZ(I_G,A)))}
			\arrow["{c\cup-}", from=1-1, to=1-2]
			\arrow["f"', from=2-2, to=1-3]
			\arrow["{\eta_d}"', from=1-2, to=2-2]
			\arrow["{\eta_u}", from=1-2, to=1-3]
		\end{tikzcd}\]
		Here, $\eta_u\colon I_G\otimes_\ZZ\op{Hom}_\ZZ(X,\op{Hom}_\ZZ(I_G,A))\to\op{Hom}_\ZZ(X,A)$ behaves as
		\[\eta_u\colon z\otimes f\mapsto\big(x\mapsto f(z)(x)\big),\]
		and $\eta_d\colon I_G\otimes_\ZZ\op{Hom}_\ZZ(X,\op{Hom}_\ZZ(I_G,A))\to \op{Hom}_\ZZ(X,I_G\otimes_\ZZ\op{Hom}_\ZZ(I_G,A))$ behaves as
		\[\eta_d\colon z\otimes f\mapsto\big(x\mapsto z\otimes f(x)\big).\]
		Now, to check our commutativity, it suffices to show that the triangle
		% https://q.uiver.app/?q=WzAsMyxbMCwwLCJJX0dcXG90aW1lc19cXFpaXFxvcHtIb219X1xcWlooWCxcXG9we0hvbX1fXFxaWihJX0csQSkpIl0sWzAsMSwiXFxvcHtIb219X1xcWlooWCxJX0dcXG90aW1lc19cXFpaXFxvcHtIb219X1xcWlooSV9HLEEpKSJdLFsxLDAsIlxcb3B7SG9tfV9cXFpaKFgsQSkiXSxbMSwyLCJmIiwyXSxbMCwxLCJcXGV0YV9kIiwyXSxbMCwyLCJcXGV0YV91Il1d&macro_url=https%3A%2F%2Fraw.githubusercontent.com%2FdFoiler%2Fnotes%2Fmaster%2Fnir.tex
		\[\begin{tikzcd}
			{I_G\otimes_\ZZ\op{Hom}_\ZZ(X,\op{Hom}_\ZZ(I_G,A))} & {\op{Hom}_\ZZ(X,A)} \\
			{\op{Hom}_\ZZ(X,I_G\otimes_\ZZ\op{Hom}_\ZZ(I_G,A))}
			\arrow["f"', from=2-1, to=1-2]
			\arrow["{\eta_d}"', from=1-1, to=2-1]
			\arrow["{\eta_u}", from=1-1, to=1-2]
		\end{tikzcd}\]
		commutes. Well, we can simply track through the diagram as follows.
		% https://q.uiver.app/?q=WzAsMyxbMCwwLCJ6XFxvdGltZXMgZiJdLFswLDEsIlxcYmlnKHhcXG1hcHN0byB6XFxvdGltZXMgZih4KVxcYmlnKSJdLFsxLDAsIlxcYmlnKHhcXG1hcHN0byBmKHgpKHopXFxiaWcpIl0sWzAsMiwiIiwyLHsic3R5bGUiOnsidGFpbCI6eyJuYW1lIjoibWFwcyB0byJ9fX1dLFswLDEsIiIsMCx7InN0eWxlIjp7InRhaWwiOnsibmFtZSI6Im1hcHMgdG8ifX19XSxbMSwyLCIiLDAseyJzdHlsZSI6eyJ0YWlsIjp7Im5hbWUiOiJtYXBzIHRvIn19fV1d&macro_url=https%3A%2F%2Fraw.githubusercontent.com%2FdFoiler%2Fnotes%2Fmaster%2Fnir.tex
		\[\begin{tikzcd}
			{z\otimes f} & {\big(x\mapsto f(x)(z)\big)} \\
			{\big(x\mapsto z\otimes f(x)\big)}
			\arrow[maps to, from=1-1, to=1-2]
			\arrow[maps to, from=1-1, to=2-1]
			\arrow[maps to, from=2-1, to=1-2]
		\end{tikzcd}\]
	\end{itemize}
	The above commutativity checks finish the induction upwards.

	We will not give detail for the induction downwards from $i-1$ to $i$, except to say that we reverse the applications of \autoref{cor:cupup} and \autoref{cor:cupdown}. The rest of the approach essentially goes through verbatim, constructing $\Phi_\bullet^{(i)}$ from a given $\Phi_\bullet^{(i-1)}$, applying the inducting hypothesis to $\Phi_\bullet^{(i)}$, and then finishing by shifting back to $\Phi_\bullet^{(i-1)}$.
\end{proof}
\begin{remark}
	Essentially the same proof can show that, for any pair of shiftable functors $F,F'\colon\mathrm{Mod}_G\to\mathrm{Mod}_G$, a natural transformation (respectively, isomorphism)
	\[\Phi_\bullet^{(i)}\colon\widehat H^i(G,F-)\Rightarrow\widehat H^i(G,F'-),\]
	at $i=p$ induces natural transformations (respectively, isomorphisms) at all $i\in\ZZ$. Instead of using \autoref{cor:cupup} and \autoref{cor:cupdown}, we must instead dimension-shifting using the usual short exact sequences.
\end{remark}
\begin{cor}
	Let $G$ be a finite group, and let $X$ and $X'$ be $G$-modules. Then, given indices $q\in\ZZ$, any natural transformation
	\[\Phi_\bullet^{(q)}\colon\widehat H^q(G,\op{Hom}_\ZZ(X,-))\Rightarrow\widehat H^{q}(G,\op{Hom}_\ZZ(X',-)),\]
	is $\Phi_\bullet^{(p)}=(-\circ\varphi)$ for some $G$-module morphism $\varphi\colon X'\to X$.
\end{cor}
\begin{proof}
	\autoref{prop:allnaturaltransarecups} tells us that the natural transformation takes the form $[\varphi]\cup-$ for some $G$-module morphism $\varphi\colon X'\to X$. Then $[\varphi]\cup-$ is simply $(-\circ\varphi)$ by \autoref{cor:cupiscomp}.
\end{proof}

\subsection{Cohomological Equivalence}
It might be the case that ``many'' different shiftable functors give the same cohomology groups. Because we are mostly interested in the case of $\op{Hom}_\ZZ(X,-)$, we now have the tools to talk fairly concretely about what this means. We have the following definition.
\begin{definition}
	Let $G$ be a finite group. We say that two $G$-modules $X,X'$ are \textit{cohomologically equivalent} if and only if there exist morphisms $[\varphi]\in\widehat H^0(G,\op{Hom}_\ZZ(X',X))$ and $[\varphi']\in\widehat H^0(G,\op{Hom}_\ZZ(X,X'))$ such that
	\[[\varphi\circ\varphi']=[\id_X]\in\widehat H^0(G,\op{Hom}_\ZZ(X,X))\qquad\text{and}\qquad[\varphi'\circ\varphi]=[\id_{X'}]\in\widehat H^0(G,\op{Hom}_\ZZ(X',X')).\]
\end{definition}
\begin{example}
	All induced modules $X$ are cohomologically equivalent to $0$. To see this, we set $\varphi\colon 0\to X$ and $\varphi'\colon X\to0$ equal to the zero maps (which are our only options). Then note that $\op{Hom}_\ZZ(X,X)$ is induced by \autoref{lem:hompreservesinduced} and $\op{Hom}_\ZZ(0,0)=0$, so
	\[\widehat H^0(G,\op{Hom}_\ZZ(X,X))=\widehat H^0(G,\op{Hom}_\ZZ(X',X'))=0,\]
	making the checks on $\varphi$ and $\varphi'$ both trivial.
\end{example}
More concretely, $X$ and $X'$ are cohomologically equivalent if and only if we have two $G$-module morphisms $\varphi\colon X'\to X$ and $\varphi'\colon X\to X'$ and two $\ZZ$-module morphisms $f\colon X\to X$ and $f'\colon X'\to X'$ such that
\[\varphi\circ\varphi'=\id_X+N_Gf\qquad\text{and}\qquad\varphi'\circ\varphi=\id_{X'}+N_Gf'.\]
As a quick sanity check that this is a reasonable notion of equivalence of modules, we have the following.
\begin{lemma} \label{lem:monoidequiv}
	Let $G$ be a finite group. If the $G$-modules $X$ and $X'$ are equivalent and $Y$ and $Y'$ are equivalent, then $X\oplus X'$ is equivalent to $Y\oplus Y'$.
\end{lemma}
\begin{proof}
	We are promised the morphisms
	\begin{itemize}
		\item $\varphi\colon X'\to X$ and $\varphi'\colon X\to X'$ (as morphisms of $G$-modules),
		\item $f\colon X\to X$ and $f'\colon X'\to X'$ (as morphisms of $\ZZ$-modules),
		\item $\psi\colon Y'\to Y$ and $\psi'\colon Y\to Y'$ (as morphisms of $G$-modules),
		\item $g\colon Y\to Y$ and $g'\colon Y'\to Y'$ (as morphisms of $\ZZ$-modules),
	\end{itemize}
	which are required to satisfy
	\begin{align*}
		\varphi\circ\varphi'={\id_X}+N_Gf\qquad&\text{and}\qquad\varphi'\circ\varphi={\id_{X'}}+N_Gf', \\
		\psi\circ\psi'={\id_Y}+N_Gg\qquad&\text{and}\qquad\psi'\circ\psi={\id_{Y'}}+N_Gg'.
	\end{align*}
	Summing everywhere, we get the $G$-module homomorphisms $\varphi\oplus\psi\colon X\oplus Y\to X'\oplus Y'$ and $\varphi'\oplus\psi'\colon X'\oplus Y'\to X\oplus Y$ satisfying
	\begin{align*}
		(\varphi\oplus\psi)\circ(\varphi'\oplus\psi') &= (\varphi\circ\varphi')\oplus(\psi\circ\psi') \\
		&= ({\id_X}+N_Gf)\oplus({\id_Y}+N_Gg) \\
		&= {\id_X}\oplus{\id_Y}+N_G(f\oplus g).
	\end{align*}
	The other check is analogous, switching primed and unprimed variables.
\end{proof}
We now show that this notion of equivalence correctly translates to shiftable functors.
\begin{proposition} \label{prop:cohomologicaldef}
	Let $G$ be a finite group, and let $X$ and $X'$ be $G$-modules. Then $X$ and $X'$ are cohomologically equivalent if and only if there is a natural isomorphism
	\[\Phi_\bullet\colon\widehat H^0(G,\op{Hom}_\ZZ(X,-))\Rightarrow\widehat H^0(G,\op{Hom}_\ZZ(X',-)).\]
\end{proposition}
\begin{proof}
	In the forward direction, suppose $X$ and $X'$ are cohomologically equivalent so that we have $[\varphi]\in\widehat H^0(G,\op{Hom}_\ZZ(X',X))$ and $[\varphi']\in\widehat H^0(G,\op{Hom}_\ZZ(X,X'))$ such that
	\[[\varphi]\cup[\varphi']=[\varphi\circ\varphi']=[\id_X]\qquad\text{and}\qquad[\varphi']\cup[\varphi]=[\varphi'\circ\varphi]=[\id_{X'}],\]
	where we are using the canonical evaluation maps for the cup products. Now, we note that, for any $G$-module $A$, we have inverse morphisms
	\begin{equation}
		\arraycolsep=1.4pt\begin{array}{ccc}
			\widehat H^0(G,\op{Hom}_\ZZ(X,A)) &\simeq& \widehat H^0(G,\op{Hom}_\ZZ(X,A)) \\
			{[f]} &\mapsto& {[f\circ\varphi]} \\
			{[f'\circ\varphi']} & \mapsfrom & {[f']}.
		\end{array} \label{eq:makenaturaliso}
	\end{equation}
	Indeed, these are mutually inverse because
	\[[f\circ\varphi\circ\varphi']=[f].\]
	To finish, we note that the isomorphisms \autoref{eq:makenaturaliso} assemble into a natural isomorphism by \autoref{lem:cuppingisnatural} and \autoref{cor:cupiscomp}.

	We now show the backwards direction. Suppose we have a natural isomorphism $\Phi_\bullet$. Then \autoref{lem:naturaltransiscupping} promises us $[\varphi]\in\widehat H^0(G,\op{Hom}_\ZZ(X',X))$ and $[\varphi']\in\widehat H^0(G,\op{Hom}_\ZZ(X,X'))$ such that the morphisms
	\[\arraycolsep=1.4pt\begin{array}{cccc}
		\Phi_\bullet\colon& \widehat H^0(G,\op{Hom}_\ZZ(X,-)) &\simeq& \widehat H^0(G,\op{Hom}_\ZZ(X,-)) \\
		&{[f]} &\mapsto& {[f\circ\varphi]} \\
		&{[f'\circ\varphi']} & \mapsfrom & {[f']}
	\end{array}\]
	are mutually inverse. In particular, we see that
	\[[\id_X]=[{\id_X}\circ\varphi\circ\varphi']=[\varphi\circ\varphi'],\]
	so $[\varphi\circ\varphi']=[\id_X]$. Swapping primed and unprimed variables, we see $[\varphi'\circ\varphi]=[\id_{X'}]$ as well.
\end{proof}
\begin{remark}
	The above result makes it fairly clear that cohomological equivalence actually makes an equivalence relation. In particular, we can invert and compose natural isomorphisms, which gives symmetry and transitivity of cohomological equivalence respectively.
\end{remark}
One might hope that we can get more information by using indices away from $0$, but in fact we cannot.
\begin{proposition} \label{prop:betterocohomdef}
	Let $G$ be a finite group, and let $X$ and $X'$ be $G$-modules. Then the following are equivalent.
	\begin{listalph}
		\item $X$ and $X'$ are cohomologically equivalent.
		\item For some $p\in\ZZ$, there is a natural isomorphism
		\[\Phi_\bullet^{(p)}\colon\widehat H^p(G,\op{Hom}_\ZZ(X,-))\Rightarrow\widehat H^p(G,\op{Hom}_\ZZ(X',-)).\]
		\item There is a $G$-module homomorphism $\varphi\colon X'\to X$ such that the induced maps
		\[(-\circ\varphi)\colon\widehat H^i(G,\op{Hom}_\ZZ(X,-))\Rightarrow\widehat H^i(G,\op{Hom}_\ZZ(X',-))\]
		are natural isomorphisms for all $i\in\ZZ$.
	\end{listalph}
\end{proposition}
\begin{proof}
	Note that (a) implies (b) by taking $p=0$ and applying \autoref{prop:cohomologicaldef}. Also, (c) implies (a) by taking $i=0$ and again applying \autoref{prop:cohomologicaldef}. Lastly, to show (b) implies (c), we note that \autoref{prop:allnaturaltransarecups} promises us $\varphi\colon X'\to X$ such that
	\[\Phi_\bullet^{(p)}=(-\circ\varphi).\]
	We would like to use \autoref{prop:dimshiftcupisos}. Let our shifting pair be $(\op{Hom}_\ZZ(X,-),\op{Hom}_\ZZ(X',-),\op{Hom}_\ZZ(X',X),\eta)$, where $\eta_\bullet$ is the canonical pre-composition map
	\[\eta_\bullet\colon\op{Hom}_\ZZ(X',X)\otimes_\ZZ\op{Hom}_\ZZ(X,-)\to\op{Hom}_\ZZ(X',-).\]
	Then we take $p=p$ and $q=0$ and $c=[\varphi]$ as above so that the cup-product natural transformation
	\[[\varphi]\cup-\colon\widehat H^i(G,\op{Hom}_\ZZ(X,-))\Rightarrow\widehat H^i(G,\op{Hom}_\ZZ(X',-))\]
	is simply induced by $(-\circ\varphi)$ for any $i\in\ZZ$ by \autoref{cor:cupiscomp}. So we are given that this is a natural isomorphism at $i=p$, so \autoref{prop:dimshiftcupisos} gives us this isomorphism at all $i\in\ZZ$, which proves (c).
	% Thus, (b) implies (c) is the interesting part. Observe that we already know from \autoref{lem:dimshiftnaturaltrans} that we have a natural isomorphism
	% \[\Phi^{(0)}_\bullet\colon\widehat H^0(G,\op{Hom}_\ZZ(X,-))\Rightarrow\widehat H^0(G,\op{Hom}_\ZZ(X',-)).\]
	% Now, applying \autoref{lem:naturaltransiscupping}, we see that this must be induced by some $\varphi\colon X'\to X$, so we know that we have a natural isomorphism
	% \begin{equation}
	% 	(-\circ\varphi)\colon\widehat H^i(G,\op{Hom}_\ZZ(X,A))\to\widehat H^i(G,\op{Hom}_\ZZ(X',A)) \label{eq:cupshifting}
	% \end{equation}
	% is an isomorphism for all $G$-modules $A$ at $i=0$. To prove (c), we will shift this isomorphism up and down from $0$. To shift downwards, we suppose that we have an isomorphism always at $i$, and we show that we have an isomorphism always at $i-1$. Well, for any $G$-module $A$, we note the morphism of ($\ZZ$-split) short exact sequences
	% % https://q.uiver.app/?q=WzAsMTAsWzAsMSwiMCJdLFsxLDEsIlxcb3B7SG9tfV9cXFpaKFgnLElfR1xcb3RpbWVzX1xcWlogQSkiXSxbMiwxLCJcXG9we0hvbX1fXFxaWihYJyxcXFpaW0ddXFxvdGltZXNfXFxaWiBBKSJdLFszLDEsIlxcb3B7SG9tfV9cXFpaKFgnLEEpIl0sWzQsMSwiMCJdLFsxLDAsIlxcb3B7SG9tfV9cXFpaKFgsSV9HXFxvdGltZXNfXFxaWiBBKSJdLFsyLDAsIlxcb3B7SG9tfV9cXFpaKFgsXFxaWltHXVxcb3RpbWVzX1xcWlogQSkiXSxbMywwLCJcXG9we0hvbX1fXFxaWihYLEEpIl0sWzAsMCwiMCJdLFs0LDAsIjAiXSxbOCw1XSxbNSw2XSxbNiw3XSxbNyw5XSxbMCwxXSxbMSwyXSxbMiwzXSxbMyw0XSxbNSwxLCIoLVxcY2lyY1xcdmFycGhpKSIsMl0sWzYsMiwiKC1cXGNpcmNcXHZhcnBoaSkiLDJdLFs3LDMsIigtXFxjaXJjXFx2YXJwaGkpIiwyXV0=&macro_url=https%3A%2F%2Fraw.githubusercontent.com%2FdFoiler%2Fnotes%2Fmaster%2Fnir.tex
	% \begin{equation}
	% 	\begin{tikzcd}[column sep=10pt]
	% 		0 & {\op{Hom}_\ZZ(X,I_G\otimes_\ZZ A)} & {\op{Hom}_\ZZ(X,\ZZ[G]\otimes_\ZZ A)} & {\op{Hom}_\ZZ(X,A)} & 0 \\
	% 		0 & {\op{Hom}_\ZZ(X',I_G\otimes_\ZZ A)} & {\op{Hom}_\ZZ(X',\ZZ[G]\otimes_\ZZ A)} & {\op{Hom}_\ZZ(X',A)} & 0
	% 		\arrow[from=1-1, to=1-2]
	% 		\arrow[from=1-2, to=1-3]
	% 		\arrow[from=1-3, to=1-4]
	% 		\arrow[from=1-4, to=1-5]
	% 		\arrow[from=2-1, to=2-2]
	% 		\arrow[from=2-2, to=2-3]
	% 		\arrow[from=2-3, to=2-4]
	% 		\arrow[from=2-4, to=2-5]
	% 		\arrow["{(-\circ\varphi)}"', from=1-2, to=2-2]
	% 		\arrow["{(-\circ\varphi)}"', from=1-3, to=2-3]
	% 		\arrow["{(-\circ\varphi)}"', from=1-4, to=2-4]
	% 	\end{tikzcd} \label{eq:somesesgoingdown}
	% \end{equation}
	% whose boundary morphisms induce the following commutative square.
	% % https://q.uiver.app/?q=WzAsNCxbMSwwLCJcXHdpZGVoYXQgSF57aS0xfShHLFxcb3B7SG9tfV9cXFpaKFgnLEEpKSJdLFswLDAsIlxcd2lkZWhhdCBIXntpLTF9KEcsXFxvcHtIb219X1xcWlooWCxBKSkiXSxbMCwxLCJcXHdpZGVoYXQgSF4wKEcsXFxvcHtIb219X1xcWlooWCxJX0dcXG90aW1lc19cXFpaIEEpIl0sWzEsMSwiXFx3aWRlaGF0IEheMChHLFxcb3B7SG9tfV9cXFpaKFgnLElfR1xcb3RpbWVzX1xcWlogQSkiXSxbMSwwLCJcXHZhcnBoaVxcY3VwLSJdLFsyLDMsIlxcdmFycGhpXFxjdXAtIl0sWzEsMiwiXFxkZWx0YSIsMl0sWzAsMywiXFxkZWx0YSIsMl1d&macro_url=https%3A%2F%2Fraw.githubusercontent.com%2FdFoiler%2Fnotes%2Fmaster%2Fnir.tex
	% \[\begin{tikzcd}
	% 	{\widehat H^{i-1}(G,\op{Hom}_\ZZ(X,A))} & {\widehat H^{i-1}(G,\op{Hom}_\ZZ(X',A))} \\
	% 	{\widehat H^i(G,\op{Hom}_\ZZ(X,I_G\otimes_\ZZ A))} & {\widehat H^i(G,\op{Hom}_\ZZ(X',I_G\otimes_\ZZ A))}
	% 	\arrow["{(-\circ\varphi)}", from=1-1, to=1-2]
	% 	\arrow["{(-\circ\varphi)}", from=2-1, to=2-2]
	% 	\arrow["\delta"', from=1-1, to=2-1]
	% 	\arrow["\delta"', from=1-2, to=2-2]
	% \end{tikzcd}\]
	% In particular, the inductive hypothesis tells us that the bottom row is an isomorphism, and the fact that both middle terms of \autoref{eq:somesesgoingdown} are induced by \autoref{lem:hompreservesinduced} implies that the $\delta$s on either side are also isomorphisms. So the top row is an isomorphism, finishing.
	%
	% Similarly, to shift upwards, we suppose that \autoref{eq:cupshifting} is always an isomorphism at $i$, and we show that we have an isomorphism always at $i+1$. Well, for any $G$-module $A$, we note the ($\ZZ$-split) short exact sequences
	% % https://q.uiver.app/?q=WzAsMTAsWzAsMSwiMCJdLFsxLDEsIlxcb3B7SG9tfV9cXFpaKFgnLEEpIl0sWzIsMSwiXFxvcHtIb219X1xcWlooWCcsXFxvcHtIb219X1xcWlooXFxaWltHXSxBKSkiXSxbMywxLCJcXG9we0hvbX1fXFxaWihYJyxcXG9we0hvbX1fXFxaWihJX0csQSkpIl0sWzQsMSwiMCJdLFsxLDAsIlxcb3B7SG9tfV9cXFpaKFgsQSkiXSxbMiwwLCJcXG9we0hvbX1fXFxaWihYLFxcb3B7SG9tfV9cXFpaKFxcWlpbR10sQSkpIl0sWzMsMCwiXFxvcHtIb219X1xcWlooWCxcXG9we0hvbX1fXFxaWihJX0csQSkpIl0sWzAsMCwiMCJdLFs0LDAsIjAiXSxbOCw1XSxbNSw2XSxbNiw3XSxbNyw5XSxbMCwxXSxbMSwyXSxbMiwzXSxbMyw0XSxbNSwxLCIoLVxcY2lyY1xcdmFycGhpKSIsMl0sWzYsMiwiKC1cXGNpcmNcXHZhcnBoaSkiLDJdLFs3LDMsIigtXFxjaXJjXFx2YXJwaGkpIiwyXV0=&macro_url=https%3A%2F%2Fraw.githubusercontent.com%2FdFoiler%2Fnotes%2Fmaster%2Fnir.tex
	% \begin{equation}
	% 	\begin{tikzcd}[column sep=10pt]
	% 		0 & {\op{Hom}_\ZZ(X,A)} & {\op{Hom}_\ZZ(X,\op{Hom}_\ZZ(\ZZ[G],A))} & {\op{Hom}_\ZZ(X,\op{Hom}_\ZZ(I_G,A))} & 0 \\
	% 		0 & {\op{Hom}_\ZZ(X',A)} & {\op{Hom}_\ZZ(X',\op{Hom}_\ZZ(\ZZ[G],A))} & {\op{Hom}_\ZZ(X',\op{Hom}_\ZZ(I_G,A))} & 0
	% 		\arrow[from=1-1, to=1-2]
	% 		\arrow[from=1-2, to=1-3]
	% 		\arrow[from=1-3, to=1-4]
	% 		\arrow[from=1-4, to=1-5]
	% 		\arrow[from=2-1, to=2-2]
	% 		\arrow[from=2-2, to=2-3]
	% 		\arrow[from=2-3, to=2-4]
	% 		\arrow[from=2-4, to=2-5]
	% 		\arrow["{(-\circ\varphi)}"', from=1-2, to=2-2]
	% 		\arrow["{(-\circ\varphi)}"', from=1-3, to=2-3]
	% 		\arrow["{(-\circ\varphi)}"', from=1-4, to=2-4]
	% 	\end{tikzcd} \label{eq:somesesgoingup}
	% \end{equation}
	% whose boundary morphisms induce the following commutative square.
	% % https://q.uiver.app/?q=WzAsNCxbMSwwLCJcXHdpZGVoYXQgSF57aX0oRyxcXG9we0hvbX1fXFxaWihYJyxcXG9we0hvbX1fXFxaWihJX0csQSkpKSJdLFswLDAsIlxcd2lkZWhhdCBIXntpfShHLFxcb3B7SG9tfV9cXFpaKFgsXFxvcHtIb219X1xcWlooSV9HLEEpKSkiXSxbMCwxLCJcXHdpZGVoYXQgSF57aSsxfShHLFxcb3B7SG9tfV9cXFpaKFgsQSkiXSxbMSwxLCJcXHdpZGVoYXQgSF57aSsxfShHLFxcb3B7SG9tfV9cXFpaKFgnLEEpIl0sWzEsMCwiXFx2YXJwaGlcXGN1cC0iXSxbMiwzLCJcXHZhcnBoaVxcY3VwLSJdLFsxLDIsIlxcZGVsdGEiLDJdLFswLDMsIlxcZGVsdGEiLDJdXQ==&macro_url=https%3A%2F%2Fraw.githubusercontent.com%2FdFoiler%2Fnotes%2Fmaster%2Fnir.tex
	% \[\begin{tikzcd}
	% 	{\widehat H^{i}(G,\op{Hom}_\ZZ(X,\op{Hom}_\ZZ(I_G,A)))} & {\widehat H^{i}(G,\op{Hom}_\ZZ(X',\op{Hom}_\ZZ(I_G,A)))} \\
	% 	{\widehat H^{i+1}(G,\op{Hom}_\ZZ(X,A))} & {\widehat H^{i+1}(G,\op{Hom}_\ZZ(X',A))}
	% 	\arrow["{(-\circ\varphi)}", from=1-1, to=1-2]
	% 	\arrow["{(-\circ\varphi)}", from=2-1, to=2-2]
	% 	\arrow["\delta"', from=1-1, to=2-1]
	% 	\arrow["\delta"', from=1-2, to=2-2]
	% \end{tikzcd}\]
	% This time around, the top row is an isomorphism by the inductive hypothesis, and the left and row arrows are isomorphisms because the middle terms of \autoref{eq:somesesgoingup} are induced by \autoref{lem:hompreservesinduced}. So the bottom row is an isomorphism as well, finishing.
\end{proof}

\subsection{Encoding Modules}
Lastly, we arrive at the application we care about: encoding cohomology.
\begin{definition}
	Let $G$ be a finite group and $p\in\ZZ$ be an index. Then a $G$-module $X$ is a \textit{$p$-encoding module} if and only if there is a natural isomorphism
	\[\Phi_\bullet\colon\widehat H^i(G,\op{Hom}_\ZZ(X,-))\Rightarrow\widehat H^{i+p}(G,-)\]
	for some $i\in\ZZ$.
\end{definition}
Cohomological equivalence is exactly what we need to talk about uniqueness.
\begin{cor} \label{cor:encodingmodules}
	Let $G$ be a finite group, and let $p,q\in\ZZ$ be indices. Then the set of $G$-module $X$ with a natural isomorphism
	\[\Phi_\bullet\colon\widehat H^p(G,\op{Hom}_\ZZ(X,-))\Rightarrow\widehat H^q(G,-)\]
	make up exactly one cohomological equivalence class.
\end{cor}
\begin{proof}
	Fix some $G$-module $X$ with such a natural isomorphism
	\[\Psi_\bullet\colon\widehat H^p(G,\op{Hom}_\ZZ(X,-))\Rightarrow\widehat H^q(G,-).\]
	We would like to show that a $G$-module $X$ has a natural isomorphism $\Phi_\bullet$ between the same functors if and only if $X$ and $X'$ are cohomologically equivalent.

	% Noting that the cup product is natural in both arguments (see, for example, \autoref{}), we see that the combination of \autoref{thm:yesitisacocycle} and \autoref{prop:alternativetupleclass} tells us that we have a natural isomorphism
	% \[\Psi_\bullet\colon\widehat H^0(G,\op{Hom}_\ZZ(X,-))\Rightarrow\widehat H^2(G,-).\]
	% We now proceed with the proof.
	If $X$ and $X'$ are cohomologically equivalent, then we can compose the promised natural isomorphism of \autoref{prop:betterocohomdef} (c) with $\Psi_\bullet$, giving a natural isomorphism
	\[\widehat H^p(G,\op{Hom}_\ZZ(X',-))\Rightarrow\widehat H^p(G,\op{Hom}_\ZZ(X,-))\stackrel{\Psi_\bullet}\Rightarrow\widehat H^q(G,-).\]
	In the other direction, if we have a natural isomorphism
	\[\Phi_\bullet\colon\widehat H^p(G,\op{Hom}_\ZZ(X',-))\Rightarrow\widehat H^q(G,-),\]
	then we can compose with $\Psi_\bullet^{-1}$ to build a natural isomorphism
	\[\widehat H^p(G,\op{Hom}_\ZZ(X',-))\stackrel{\Phi_\bullet}\Rightarrow\widehat H^q(G,-)\stackrel{\Psi^{-1}_\bullet}\Rightarrow\widehat H^p(G,\op{Hom}_\ZZ(X,-)),\]
	from which it follows that $X$ and $X'$ are cohomologically equivalent by \autoref{prop:cohomologicaldef} (b).
\end{proof}
% \begin{remark}
% 	In fact, we can see that the natural isomorphism
% 	\[\Phi_\bullet\colon\widehat H^0(G,\op{Hom}_\ZZ(X',-))\Rightarrow\widehat H^2(G,-)\]
% 	must be a cup-product map with an element in $\widehat H^2(G,X')$. Namely, we note that $\Phi_\bullet$ is equal to the composite
% 	\[\widehat H^0(G,\op{Hom}_\ZZ(X',-))\stackrel{\Phi_\bullet}\Rightarrow\widehat H^2(G,-)\stackrel{\Psi^{-1}_\bullet}\Rightarrow\widehat H^0(G,\op{Hom}_\ZZ(X,-))\stackrel{\Psi_\bullet}\Rightarrow\widehat H^2(G,-).\]
% 	However, $\Psi_\bullet$ is a cup-product map with an element in $\widehat H^2(G,X)$ by its construction, and
% 	\[\widehat H^0(G,\op{Hom}_\ZZ(X',-))\stackrel{\Phi_\bullet}\Rightarrow\widehat H^2(G,-)\stackrel{\Psi^{-1}_\bullet}\Rightarrow\widehat H^0(G,\op{Hom}_\ZZ(X,-))\]
% 	is a cup-product map with an element in $\widehat H^0(G,\op{Hom}_\ZZ(X,X'))$ by \autoref{lem:naturaltransiscupping}. Composing our cup-product maps makes a cup-product map with an element in $\widehat H^2(G,X')$.
% \end{remark}
% \begin{example}
% 	Suppose that a $G$-module $X$ has a natural isomorphism
% 	\[\widehat H^0(\]
% 	If $M$ is any induced module, then we know $M$ is cohomologically equivalent to $0$. So \autoref{lem:monoidequiv} reassures us that $X\oplus M$ is cohomologically equivalent to $X\oplus0\simeq X$.
% \end{example}
\begin{example} \label{ex:igisencoding}
	Take $q\ge p$. Dimension-shifting iteratively with the short exact sequence
	\[0\to I_G\otimes_\ZZ A\to\ZZ[G]\otimes_\ZZ A\to A\to0\]
	shows that
	\[\widehat H^q(G,A)\simeq\widehat H^p\left(G,\op{Hom}_\ZZ(I_G^{\otimes (q-p)},A)\right),\]
	and in fact these isomorphisms are natural by the functoriality of boundary morphisms. So the equivalence class of \autoref{cor:encodingmodules} is represented by $I_G^{\otimes(q-p)}$.
\end{example}
% The above two examples should give a feeling for why this uniqueness problem is difficult.
In fact, akin to the classification of natural transformations from \autoref{prop:allnaturaltransarecups}, we can show that these encoding maps must be cup products.
% \begin{lemma}
% 	Let $G$ be a finite group, and let $p\ge0$ be an index. Then there exists some $x_p\in\widehat H^p\left(G,I_G^{\otimes p}\right)$ such that
% 	\[x_p\cup-\colon\widehat H^0\left(G,\op{Hom}_\ZZ(I_G^{\otimes p},-)\right)\Rightarrow H^p(G,-)\]
% 	is a natural isomorphism.
% \end{lemma}
% \begin{proof}
% 	We quickly remark that (akin to \autoref{}), by \autoref{} and functoriality of $\widehat H^p(G,-)$, any $x_p\in\widehat H^p(G,I_G^{\otimes p})$ will at least create a natural transformation. So the main point is to make the cup-product into an isomorphism. For $p=0$, we take $x_0\coloneqq[1]\in\widehat H^p(G,\ZZ)$ so that
% 	\[[1]\cup-\colon\widehat H^0(G,\op{Hom}_\ZZ(\ZZ,-))\Rightarrow\widehat H^0(G,-)\]
% 	is simply the map taking a $0$-cocycle $c$ to $[1]\cup c=c(1)$, which is the isomorphism $\op{Hom}_\ZZ(\ZZ,A)\simeq A$ for each $A$ anyway.

% 	We will also show $p=1$ by hand. The point here is that, for any $G$-module $A$, we already have (natural) isomorphisms
% 	\[\widehat H^0(G,\op{Hom}_\ZZ(I_G,A))\simeq\widehat H^1(G,A)\]
% 	by dimension-shifting, so we just have to track these through. Namely, we have the ($\ZZ$-split) short exact sequence
% 	\[0\to A\to\op{Hom}_\ZZ(\ZZ[G],A)\to\op{Hom}_\ZZ(I_G,A)\to 0,\]
% 	so given $[f]\in\widehat H^0(G,\op{Hom}_\ZZ(I_G,A))$ so that $f\in\op{Hom}_{\ZZ[G]}(I_G,A)$, this gets pulled back to the $0$-cochain $c\in\op{Hom}_\ZZ(\ZZ[G],A)$ defined by
% 	\[c\colon z\mapsto f(z-\varepsilon(z)).\]
% 	Pushing this down to $Z^1(G,\op{Hom}_\ZZ(\ZZ[G],A))$, we compute
% 	\begin{align*}
% 		(dc)(g)(z) &= (gc)(z)-c(z) \\
% 		&= g\cdot c\left(g^{-1}z\right) - c(z) \\
% 		&= g\cdot f\left(g^{-1}z-\varepsilon(g^{-1}z)\right) - f(z-\varepsilon(z)) \\
% 		&= f(z-g\varepsilon(z))-f(z-\varepsilon(z)) \\
% 		&= \varepsilon(z)\cdot f(1-g),
% 	\end{align*}
% 	so we pull back to the $1$-cochain $g\mapsto f(1-g)$ in $H^1(G,A)$. To see this as a cup product, we just note that running $A=I_G$ and $f=\id_{I_G}$ through this argument would reveal that $g\mapsto(1-g)$ is a $1$-cochain in $H^1(G,I_G)$, which we denote by $x_1$. Then returning to a general $G$-module $A$ with $[f]\in\widehat H^1(G,\op{Hom}_\ZZ(I_G,A))$, we see
% 	\[(x_1\cup[f])\colon g\mapsto(f\circ x_1)(g)=f(1-g)\]
% 	by construction of the cup product as evaluation.
% \end{proof}
\begin{cor} \label{cor:encodingsarecups}
	Let $G$ be a finite group, and let $p\in\ZZ$ be an index. Suppose we have a $G$-module $X$ and index $i\in\ZZ$ with a natural transformation
	\[\Phi_\bullet\colon\widehat H^i(G,\op{Hom}_\ZZ(X,-))\Rightarrow\widehat H^{i+p}(G,-).\]
	Then there exists $[x]\in\widehat H^p(G,X)$ such that $\Phi_\bullet$ is the cup-product map $[x]\cup-$.
\end{cor}
\begin{proof}
	The point is to set $X'=\ZZ$ in \autoref{prop:allnaturaltransarecups}. Indeed, $\Phi_\bullet$ will induce a natural transformation
	\[\widehat H^0(G,\op{Hom}_\ZZ(X,-))\stackrel{\Phi_\bullet}\Rightarrow\widehat H^{i+p}(G,-)\Rightarrow\widehat H^p(G,\op{Hom}_\ZZ(\ZZ,-)),\]
	where the last natural transformation is induced by the natural isomorphism $\eta\colon{\id}\simeq\op{Hom}_\ZZ(\ZZ,-)$. By \autoref{prop:allnaturaltransarecups}, we are promised $[x]\in\widehat H^0(G,\op{Hom}_\ZZ(\ZZ,X))$ such that this composite is $[x]\cup-$. Without being too detailed, we'll just say that passing everything through $\eta^{-1}$ shows that $\Phi_\bullet$ is
	\[\left[\eta^{-1}_Xx\right]\cup-\colon\widehat H^i(G,\op{Hom}_\ZZ(X,-))\Rightarrow\widehat H^{i+p}(G,-).\]
	One should check that all the evaluation maps correctly align, but they morally should because we're just doing pre-composition.
\end{proof}
\begin{example}
	For $p\ge0$, standard dimension-shifting arguments give natural isomorphisms
	\[\widehat H^0\left(G,\op{Hom}_\ZZ(I_G^{\otimes p},-)\right)\Rightarrow\widehat H^p(G,-),\]
	so \autoref{cor:encodingsarecups} implies that these isomorphisms are cup products with an element of $\widehat H^p(G,I_G^{\otimes p})$. For example, when $p=0$, we have $[1]\in\widehat H^0(G,\ZZ)$; and when $p=1$, we have $g\mapsto(1-g)$ in $\widehat H^1(G,I_G)$. Observe that we could also see this by inductively dimension-shifting with \autoref{cor:cupdown}.
\end{example}
Because cup products are better-behaved than just general natural transformations, we get the following nice statement.
\begin{cor} \label{cor:betterencodingdef}
	Let $G$ be a finite group, and let $p\in\ZZ$ an index. Then a $p$-encoding module $X$ has $x\in\widehat H^p(G,X)$ such that
	\[x\cup-\colon\widehat H^i(G,\op{Hom}_\ZZ(X,-))\Rightarrow\widehat H^{i+p}(G,-)\]
	is a natural isomorphism for all $i\in\ZZ$.
\end{cor}
\begin{proof}
	By definition of $X$, we know that there is some $i\in\ZZ$ such that we have a natural isomorphism
	\[\Phi_\bullet\colon\widehat H^i(G,\op{Hom}_\ZZ(X,-))\Rightarrow\widehat H^{i+p}(G,-).\]
	Then \autoref{cor:encodingsarecups} tells us that this natural isomorphism arises as $x\cup-$ for some $x\in\widehat H^p(G,X)$.
	
	To finish, we extend $x\cup-$ being a natural isomorphism from a single $i$ to all $i\in\ZZ$ by using \autoref{prop:dimshiftcupisos}. Indeed, take $F=\op{Hom}_\ZZ(X,-)$ and $F'=\mathrm{id}$ and $X=X$ and $\eta\colon X\otimes_\ZZ(X,-)\Rightarrow\mathrm{id}$ to be the canonical evaluation maps. This finishes.
\end{proof}
\begin{remark}
	Taking $X=\ZZ$ above, we are asserting that, if $G$ is a group such that all $G$-modules admit period-$p$ cohomology which is natural in some sense at a single index $i$, then this periodicity extends to all indices and arises from a cup product with an element of $\widehat H^p(G,\ZZ)$.
	
	Observe that the naturality in the isomorphisms is important: letting $G\coloneqq\ZZ/p\ZZ$ act on $A\coloneqq\ZZ/p\ZZ$ trivially,
	\[\widehat H^{-1}(G,A)=\frac{\ZZ/p\ZZ}{0}\simeq\widehat H^0(G,A),\]
	but this does not extend to all $G$-modules. For example,
	\[\widehat H^{-1}(G,\ZZ)=0\not\cong\frac{\ZZ}{p\ZZ}=\widehat H^0(G,\ZZ).\]
\end{remark}

\subsection{Encoding Is Unique}
Fix a $p$-encoding module $X$. As a brief intermission, we will show that there is essentially one way to do the encoding
\[\widehat H^i(G,\op{Hom}_\ZZ(X,-))\Rightarrow\widehat H^{i+p}(G,-).\]
Namely, we know from \autoref{cor:encodingsarecups}, that this natural isomorphism must come from a cup-product with an element $x\in\widehat H^p(G,X)$, so we might wonder how unique this element $x$ is. The answer to this, roughly speaking, will be that $\widehat H^p(G,X)$ is cyclic of order $\#G$ generated by $x$.

Anyway, the main idea will be the following duality result.
\begin{prop}[\cite{cartan-eilenberg}, Corollary~XII.6.5] \label{prop:ceduality}
	Let $G$ be a finite group and $A$ be any $G$-module. Then the cup-product pairing induces an isomorphism
	\[\widehat H^{i-1}(G,\op{Hom}_\ZZ(A,\QQ/\ZZ))\to\op{Hom}_\ZZ\left(\widehat H^{-i}(G,A),\widehat H^{-1}(G,\QQ/\ZZ)\right)\]
	for all $i\in\ZZ$. Indeed, this is a duality upon embedding $\widehat H^{-1}(G,\QQ/\ZZ)$ into $\QQ/\ZZ$.
\end{prop}
And here is our computation.
\begin{cor} \label{cor:h2xcomputation}
	Let $G$ be a finite group and $X$ a $p$-encoding module. Then $\widehat H^p(G,X)\simeq\ZZ/\#G\ZZ$, generated by $x$, where $x\in\widehat H^p(G,X)$ is conjured from \autoref{cor:betterencodingdef}.
\end{cor}
\begin{proof}
	For brevity, set $n\coloneqq\#G$. By \autoref{cor:betterencodingdef}, we have the isomorphism
	\[x\cup-\colon\widehat H^{-p-1}(G,\op{Hom}_\ZZ(X,\QQ/\ZZ))\to\widehat H^0(G,\QQ/\ZZ)=\textstyle\frac1n\ZZ/\ZZ.\]
	In particular, $\widehat H^{-p-1}(G,\op{Hom}_\ZZ(X,\ZZ))\simeq\ZZ/n\ZZ$, generated by some element $x^\lor$ such that $x\cup x^\lor=[1/n]$.

	Now, we apply \autoref{prop:ceduality} to say that the cup-product pairing induces an isomorphism
	\[{\textstyle\frac1n\ZZ/n\ZZ}\simeq\widehat H^{-p-1}(G,\op{Hom}_\ZZ(X,\QQ/\ZZ))\to\op{Hom}_\ZZ\left(\widehat H^p(G,X),\widehat H^{-1}(G,\QQ/\ZZ)\right)\simeq\op{Hom}_\ZZ\left(\widehat H^p(G,X),\textstyle\frac1n\ZZ/\ZZ\right).\]
	Because $\widehat H^p(G,X)$ is $n$-torsion, homomorphisms $\widehat H^2(G,X)\to\QQ/\ZZ$ must have image in $\frac1n\ZZ/\ZZ$, so in fact the rightmost group is the dual of $\widehat H^p(G,X)$. Because an abelian group is isomorphic to its dual, we see that $\widehat H^p(G,X)$ is in fact cyclic of order $n$.

	It remains to show that $x$ is a generator; for this, we show that $x$ has order at least $n$, which will be enough because $H^2(G,X)$ is cyclic of order $n$. Well, if we have $k\in\ZZ$ such that $kx=0$, then
	\[[k/n]=k\big(x\cup x^\lor\big)=kx\cup x^\lor=[0]\cup x^\lor=[0]\]
	in $\widehat H^{-1}(G,\QQ/\ZZ)\simeq\frac1n\ZZ/\ZZ$, so $n\mid k$. This finishes.
\end{proof}
\begin{cor}
	Let $G$ be a finite group, and let $X$ be a $p$-encoding module. Then, given $i\in\ZZ$ and two natural isomorphisms
	\[\Phi_\bullet,\Phi_\bullet'\colon\widehat H^i(G,\op{Hom}_\ZZ(X,-))\Rightarrow\widehat H^{i+p}(G,-),\]
	there exists a unique $k\in(\ZZ/\#G\ZZ)^\times$ such that $\Phi_\bullet'=k\Phi_\bullet$.
\end{cor}
\begin{proof}
	Note that we are allowed to interpret $k\pmod n$ because these cohomology groups are $\#G$-torsion, so $\#G\cdot\Phi_\bullet=0$.

	Anyway, by \autoref{cor:encodingsarecups}, we know that there are $x,x'\in\widehat H^p(G,X)$ such that
	\[\Phi_\bullet=(x\cup-)\qquad\text{and}\qquad\Phi_\bullet'=(x'\cup-).\]
	However, by \autoref{cor:h2xcomputation}, we see that $\widehat H^p(G,X)$ is cyclic generated by $x$ of order $\#G$, so we can write $x'=kx$ for a unique $k\in\ZZ/\#G\ZZ$; because $x'$ must also be a generator, we see that $k\in(\ZZ/\#G\ZZ)^\times$ is forced. Namely, we can find $\ell\in\ZZ/\#G\ZZ$ such that $x=\ell x'$ as well.

	It remains to show that $\Phi_\bullet'=k\Phi_\bullet$. Well, for any $G$-module $A$ and $c\in\widehat H^i(G,\op{Hom}_\ZZ(X,A))$, we observe that
	\[\Phi_A'(c)=x'\cup c=kx\cup c=k(x\cup c)=k\Phi_A(c).\]
	It follows that $\Phi_\bullet'=k\Phi_\bullet$.
\end{proof}

\subsection{Encoding under Restriction}
Let $X$ be a $p$-encoding module, and conjure $x\in\widehat H^p(G,X)$ from \autoref{cor:betterencodingdef}. Then note that our proof of \autoref{cor:h2xcomputation} found $x^\lor\in\widehat H^{-p-1}(G,\op{Hom}_\ZZ(X,\QQ/\ZZ))$ such that
\[x\cup x^\lor=[1/n]\in\widehat H^{-1}(G,\QQ/\ZZ).\]
This is fairly close to saying that the operation of $x\cup-$ can be inverted with the correct $x^\lor\cup-$ operation (and maybe a sign), but cupping with $[1/n]$ would then not necessarily by the identity transformation.

In particular, we would like to actually be in $\widehat H^0(G,\ZZ)$, whose cup products are well-behaved. As such, we have the following.
% However, when $X$ is torsion-free, then it is a projective $\ZZ$-module, and we are safe. Let's codify this.
% \begin{proposition} \label{prop:abstractintegralduality}
% 	Let $G$ be a finite group, and let $X$ be a finitely generated $\ZZ$-free $G$-module. Then the cup-product pairing induces an isomorphism
% 	\[\widehat H^i(G,\op{Hom}_\ZZ(X,\ZZ))\to\op{Hom}_\ZZ\left(\widehat H^{-i}(G,X),\widehat H^0(G,\ZZ)\right)\]
% 	for all $i\in\ZZ$. Indeed, this is a duality upon identifying $\widehat H^0(G,\ZZ)$ with $\frac1{\#G}\ZZ/\ZZ\subseteq\QQ/\ZZ$.
% \end{proposition}
% \begin{proof}
% 	This proof is analogous to \cite{cartan-eilenberg}, Theorem XII.6.6. The key to the proof is the short exact sequence
% 	\begin{equation}
% 		0\to\ZZ\to\QQ\to\QQ/\ZZ\to0. \label{eq:divisibleses}
% 	\end{equation}
% 	The main point is that $X$ being finitely generated and $\ZZ$-free implies that $X$ is projective (as an abelian group), so we can apply $\op{Hom}_\ZZ(X,-)$ to get out the short exact sequence
% 	\begin{equation}
% 		0\to\op{Hom}_\ZZ(X,\ZZ)\to\op{Hom}_\ZZ(X,\QQ)\to\op{Hom}_\ZZ(X,\QQ/\ZZ)\to0. \label{eq:homdivisibleses}
% 	\end{equation}
% 	Now, note that the multiplication-by-$n$ endomorphism on $\op{Hom}_\ZZ(X,\QQ)$ is an isomorphism (namely, $\QQ$ is a divisible abelian group), so the same will be true of $\widehat H^i(G,\op{Hom}_\ZZ(X,\QQ))$ for any $i\in\ZZ$. However, these cohomology groups must be $\#G$-torsion, so in fact $\widehat H^i(G,\op{Hom}_\ZZ(X,\QQ))=0$ for all $i\in\ZZ$.

% 	Similarly, we note that we can hit \autoref{eq:homdivisibleses} with the functor $-\otimes_\ZZ X$ to get another short exact sequence
% 	\begin{equation}
% 		0\to\op{Hom}_\ZZ(X,\ZZ)\otimes_\ZZ X\to\op{Hom}_\ZZ(X,\QQ)\otimes_\ZZ X\to\op{Hom}_\ZZ(X,\QQ/\ZZ)\otimes_\ZZ X\to0. \label{eq:tensorhomdivisibleses}
% 	\end{equation}
% 	Notably, this is exact because $X$ is a finitely generated, torsion-free $\ZZ$-module and hence flat as a $\ZZ$-module. Now, $\op{Hom}_\ZZ(X,\QQ)\otimes_\ZZ X$ is still a divisible abelian group, so again $\widehat H^i(G,\op{Hom}_\ZZ(X,\QQ))=0$ for all $i\in\ZZ$.

% 	The rest of the proof is tracking boundary morphisms around. Fix some $i\in\ZZ$. Note \autoref{eq:divisibleses} and \autoref{eq:homdivisibleses} and \autoref{eq:tensorhomdivisibleses} induce boundary isomorphisms
% 	\[\arraycolsep=1.4pt\begin{array}{rlcl}
% 		\delta \colon& \widehat H^{-1}(G,\QQ/\ZZ) &\to& \widehat H^0(G,\ZZ) \\
% 		\delta_h \colon& \widehat H^{i-1}(G,\op{Hom}_\ZZ(X,\QQ/\ZZ))&\to&\widehat H^i(G,\op{Hom}_\ZZ(X,\ZZ)) \\
% 		\delta_t \colon& \widehat H^{-1}(G,\op{Hom}_\ZZ(\QQ/\ZZ)\otimes_\ZZ X)&\to&\widehat H^0(G,\op{Hom}_\ZZ(X,\ZZ)\otimes_\ZZ X).
% 	\end{array}\]
% 	We also note that we have a morphism of short exact sequences
% 	% https://q.uiver.app/?q=WzAsMTAsWzAsMCwiMCJdLFsxLDAsIlxcb3B7SG9tfV9cXFpaKFgsXFxaWilcXG90aW1lc19cXFpaIFgiXSxbMiwwLCJcXG9we0hvbX1fXFxaWihYLFxcUVEpXFxvdGltZXNfXFxaWiBYIl0sWzMsMCwiXFxvcHtIb219X1xcWlooWCxcXFFRL1xcWlopXFxvdGltZXNfXFxaWiBYIl0sWzAsMSwiMCJdLFs0LDAsIjAiXSxbNCwxLCIwIl0sWzEsMSwiXFxaWiJdLFsyLDEsIlxcUVEiXSxbMywxLCJcXFFRL1xcWloiXSxbMCwxXSxbMSwyXSxbMiwzXSxbMyw1XSxbNCw3XSxbNyw4XSxbOCw5XSxbOSw2XSxbMSw3LCJcXGV0YV9cXFpaIiwyXSxbMiw4LCJcXGV0YV9cXFFRIiwyXSxbMyw5LCJcXGV0YV97XFxRUS9cXFpafSIsMl1d&macro_url=https%3A%2F%2Fraw.githubusercontent.com%2FdFoiler%2Fnotes%2Fmaster%2Fnir.tex
% 	\[\begin{tikzcd}
% 		0 & {\op{Hom}_\ZZ(X,\ZZ)\otimes_\ZZ X} & {\op{Hom}_\ZZ(X,\QQ)\otimes_\ZZ X} & {\op{Hom}_\ZZ(X,\QQ/\ZZ)\otimes_\ZZ X} & 0 \\
% 		0 & \ZZ & \QQ & {\QQ/\ZZ} & 0
% 		\arrow[from=1-1, to=1-2]
% 		\arrow[from=1-2, to=1-3]
% 		\arrow[from=1-3, to=1-4]
% 		\arrow[from=1-4, to=1-5]
% 		\arrow[from=2-1, to=2-2]
% 		\arrow[from=2-2, to=2-3]
% 		\arrow[from=2-3, to=2-4]
% 		\arrow[from=2-4, to=2-5]
% 		\arrow["{\eta_\ZZ}"', from=1-2, to=2-2]
% 		\arrow["{\eta_\QQ}"', from=1-3, to=2-3]
% 		\arrow["{\eta_{\QQ/\ZZ}}"', from=1-4, to=2-4]
% 	\end{tikzcd}\]
% 	where the $\eta_\bullet$ are evaluation maps. For peace of mind, we can check that the squares commute by the following lemma.
% 	\begin{lemma} \label{lem:evcommutes}
% 		Let $G$ be a group and $A,B,C$ be $G$-modules with a $G$-module homomorphism $\varphi\colon B\to C$. Then the diagram
% 		% https://q.uiver.app/?q=WzAsNCxbMCwwLCJBXFxvdGltZXNfXFxaWlxcb3B7SG9tfShBLEIpIl0sWzEsMCwiQVxcb3RpbWVzX1xcWlooQSxDKSJdLFswLDEsIkIiXSxbMSwxLCJDIl0sWzAsMSwiXFx2YXJwaGkiXSxbMiwzLCJcXHZhcnBoaSJdLFswLDJdLFsxLDNdXQ==&macro_url=https%3A%2F%2Fraw.githubusercontent.com%2FdFoiler%2Fnotes%2Fmaster%2Fnir.tex
% 		\[\begin{tikzcd}
% 			{A\otimes_\ZZ\op{Hom}_\ZZ(A,B)} & {A\otimes_\ZZ\op{Hom}_\ZZ(A,C)} \\
% 			B & C
% 			\arrow["\varphi", from=1-1, to=1-2]
% 			\arrow["\varphi", from=2-1, to=2-2]
% 			\arrow[from=1-1, to=2-1]
% 			\arrow[from=1-2, to=2-2]
% 		\end{tikzcd}\]
% 		commutes, where the vertical homomorphisms are evaluation.
% 	\end{lemma}
% 	\begin{proof}
% 		We simply pick up some $a\otimes f\in A\otimes_\ZZ\op{Hom}_\ZZ(A,B)$ and track through
% 		% https://q.uiver.app/?q=WzAsNCxbMCwwLCJhXFxvdGltZXMgZiJdLFsxLDAsImFcXG90aW1lc1xcdmFycGhpXFxjaXJjIGYiXSxbMCwxLCJmKGEpIl0sWzEsMSwiXFx2YXJwaGkoZihhKSkiXSxbMCwxLCJcXHZhcnBoaSIsMCx7InN0eWxlIjp7InRhaWwiOnsibmFtZSI6Im1hcHMgdG8ifX19XSxbMiwzLCJcXHZhcnBoaSIsMCx7InN0eWxlIjp7InRhaWwiOnsibmFtZSI6Im1hcHMgdG8ifX19XSxbMCwyLCIiLDEseyJzdHlsZSI6eyJ0YWlsIjp7Im5hbWUiOiJtYXBzIHRvIn19fV0sWzEsMywiIiwxLHsic3R5bGUiOnsidGFpbCI6eyJuYW1lIjoibWFwcyB0byJ9fX1dXQ==&macro_url=https%3A%2F%2Fraw.githubusercontent.com%2FdFoiler%2Fnotes%2Fmaster%2Fnir.tex
% 		\[\begin{tikzcd}
% 			{a\otimes f} & {a\otimes\varphi\circ f} \\
% 			{f(a)} & {\varphi(f(a))}
% 			\arrow["\varphi", maps to, from=1-1, to=1-2]
% 			\arrow["\varphi", maps to, from=2-1, to=2-2]
% 			\arrow[maps to, from=1-1, to=2-1]
% 			\arrow[maps to, from=1-2, to=2-2]
% 		\end{tikzcd}\]
% 		which finishes the proof.
% 	\end{proof}
% 	Now, \autoref{prop:ceduality} tells us that
% 	\[\arraycolsep=1.4pt\begin{array}{ccc}
% 		\widehat H^{i-1}(G,\op{Hom}_\ZZ(X,\QQ/\ZZ)) &\to& \op{Hom}_\ZZ\left(\widehat H^{-i}(G,X),\widehat H^{-1}(G,\QQ/\ZZ)\right) \\
% 		a &\mapsto& (b\mapsto\eta_{\QQ/\ZZ}(a\cup b))
% 	\end{array}\]
% 	is an isomorphism. Composing this with various other isomorphisms, we can build the isomorphism
% 	\[\arraycolsep=1.4pt\begin{array}{ccccccc}
% 		\widehat H^i(G,X_*) &\to& \widehat H^{i-1}(G,X^*) &\to& \op{Hom}\left(\widehat H^{-i}(G,X),\widehat H^{-1}(G,\QQ/\ZZ)\right) &\to& \op{Hom}\left(\widehat H^{-i}(G,X),\widehat H^0(G,\QQ/\ZZ)\right)  \\
% 		a &\mapsto& \delta_h^{-1}a &\mapsto& \left(b\mapsto\eta_{\QQ/\ZZ}(\delta_h^{-1}a\cup b)\right) &\mapsto& \left(b\mapsto\delta\eta_{\QQ/\ZZ}(\delta_h^{-1}a\cup b)\right)
% 	\end{array}\]
% 	where $X_*\coloneqq\op{Hom}_\ZZ(X,\ZZ)$ and $X^*\coloneqq\op{Hom}_\ZZ(X,\QQ/\ZZ)$, for brevity. This gives an isomorphism between the desired objects, but to prove the result we need to show that the above map is $a\mapsto(b\mapsto\eta_\ZZ(a\cup b))$. Well, given $a\in\widehat H^i(G,\op{Hom}_\ZZ(X,\ZZ))$ and $b\in\widehat H^{-i}(G,X)$, properties of the boundary morphisms tells us
% 	\begin{align*}
% 		\delta\eta_{\QQ/\ZZ}\left(\delta_h^{-1}a\cup b\right) &= \eta_\ZZ\delta_t\left(\delta_h^{-1}a\cup b\right) \\
% 		&= \eta_\ZZ\left(\delta_h\delta_h^{-1}a\cup b\right) \\
% 		&= \eta_\ZZ(a\cup b),
% 	\end{align*}
% 	which is what we wanted.
% \end{proof}
% \begin{remark}
% 	The hypothesis that $X$ be $\ZZ$-free is necessary: the statement is false for $X=\ZZ/\#G\ZZ$ and $i=0$, for example.
% \end{remark}
\begin{lemma} \label{lem:intdualelement}
	Let $G$ be a finite group, and let $X$ be a $p$-encoding module. Constructing $x\in\widehat H^p(G,X)$ from \autoref{cor:betterencodingdef}, there exists a unique $x^\lor\in\widehat H^{-p}(G,\op{Hom}_\ZZ(X,\ZZ))$ such that
	\[x^\lor\cup x=[1]\in\widehat H^0(G,\ZZ).\]
\end{lemma}
\begin{proof}
	Set $n\coloneqq\#G$. By \autoref{cor:betterencodingdef}, we have the isomorphism
	\[x\cup-\colon\widehat H^{-p}(G,\op{Hom}_\ZZ(X,\ZZ))\to\widehat H^0(G,\ZZ)=\ZZ/n\ZZ.\]
	As such, we can find a unique $x^\lor\in\widehat H^{-p}(G,\op{Hom}_\ZZ(X,\ZZ))$ such that $x\cup x^\lor=[1]$.
\end{proof}
Here is an amusing corollary we get from this.
\begin{cor} \label{cor:encodingsubgroups}
	Let $G$ be a finite group, and let $p\in2\ZZ$ be even. Letting $X$ be a $p$-encoding module, and construct $x\in\widehat H^p(G,X)$ from \autoref{cor:betterencodingdef}. Then, for any subgroup $H\subseteq G$ and index $i\in\ZZ$, we have a natural isomorphism
	\[(\op{Res}x)\cup-\colon\widehat H^i(H,\op{Hom}_\ZZ(X,-))\Rightarrow\widehat H^{i+p}(H,-).\]
\end{cor}
\begin{proof}
	The point is that restriction commutes with cup products, so we may use \autoref{lem:intdualelement} to construct the inverse natural transformation.
	
	In particular, we already know that the natural transformations
	\[\arraycolsep=1.4pt\begin{array}{cccc}
		(\op{Res}x)\cup-\colon& \widehat H^i(H,\op{Hom}_\ZZ(X,-)) &\Rightarrow& \widehat H^{i+p}(H,-) \\
		(\op{Res}x^\lor)\cup-\colon& \widehat H^{i+p}(H,-) &\Rightarrow& \widehat H^i(H,\op{Hom}_\ZZ(X,-))
	\end{array}\]
	by \autoref{lem:cuppingisnatural}. To be explicit, the second cup product is induced by the pairing
	\[\arraycolsep=1.4pt\begin{array}{rllcc}
		\op{Hom}_\ZZ(X,\ZZ)&\otimes_\ZZ& A &\to& \op{Hom}_\ZZ(X,A) \\
		f &\otimes& a &\mapsto& \big(x\mapsto f(x)a\big).
	\end{array}\]
	It remains to see that the top natural transformation is a natural isomorphism, which means that we need to check that its component morphisms are isomorphisms at each $G$-module $A$.
	
	Well, we claim that $(\op{Res}x^\lor)\cup-$ is the inverse morphism. Indeed, for any $a^\lor\in\widehat H^i(G,\op{Hom}_\ZZ(X,-))$, we'll be a bit sloppy with our cup products\footnote{Namely, there are evaluation maps flying around everywhere which need to check the commutativity of, but we won't bother.} and compute
	\begin{align*}
		(\op{Res}x^\lor)\cup(\op{Res}x\cup a^\lor) &= (\op{Res}x^\lor\cup\op{Res}x)\cup a^\lor \\
		&= \op{Res}(x^\lor\cup x)\cup a^\lor \\
		&= \op{Res}[1]\cup a^\lor \\
		&= [1]\cup a^\lor \\
		&= a^\lor.
	\end{align*}
	Similarly, for $a\in\widehat H^{i+p}(G,A)$, we have
	\begin{align*}
		(\op{Res}x)\cup(\op{Res}x^\lor\cup a) &= (\op{Res}x\cup\op{Res}x^\lor)\cup a \\
		&= \op{Res}(\op{Res}x\cup\op{Res}x^\lor)\cup a \\
		&= (-1)^p\op{Res}[1]\cup a \\
		&= (-1)^p[1]\cup a \\
		&= (-1)^pa,
	\end{align*}
	which is simply $a$ because $p$ is even. This finishes.
\end{proof}
\begin{remark}
	The constraint that $p$ be even is not too strict. Namely, if $X$ is a $p$-encoding module, then we have natural isomorphisms
	\[\widehat H^i(G,\op{Hom}_\ZZ(X\otimes_\ZZ X,-))\simeq\widehat H^i(G,\op{Hom}_\ZZ(X,\op{Hom}_\ZZ(X,-)))\simeq\widehat H^{i+p}(G,\op{Hom}_\ZZ(X,-))\simeq\widehat H^{i+2p}(G,-),\]
	so $X\otimes_\ZZ X$ is a $2p$-encoding module.
\end{remark}
\begin{remark}
	Essentially the same proof should hold for inflation.
\end{remark}

\subsection{A Perfect Pairing}
We close this section with a hint of Artin reciprocity. The main goal of this subsection is to prove the following result.
\begin{theorem} \label{thm:abstractperfectpairing}
	Let $G$ be a finite group, and let $X$ and $A$ be $G$-modules. Then, if there exists an element $c\in H^p(G,X)$ such that the cup-product maps
	\begin{align*}
		c\cup-&\colon\widehat H^{-p}(G,\op{Hom}_\ZZ(X,\ZZ))\to\widehat H^0(G,\ZZ) \\
		c\cup-&\colon\widehat H^0(G,\op{Hom}_\ZZ(X,A))\to\widehat H^{p}(G,A)
	\end{align*}
	are isomorphisms, then the cup-product pairing induces an isomorphism
	\[\widehat H^p(G,A)\to\op{Hom}_\ZZ\left(\widehat H^{-p}(G,\op{Hom}_\ZZ(X,\ZZ)),\widehat H^0(G,\op{Hom}_\ZZ(X,A))\right).\]
\end{theorem}
The main step in the proof is the following lemma.
\begin{lemma}
	Let $G$ be a finite group, and let $X$ and $A$ be $G$-modules. Pick up another $G$-module $A$. Then, given any $i\in\ZZ$ and $c\in\widehat H^p(G,X)$ and $u\in\widehat H^2(G,A)$, the following diagram commutes, where all arrows are cup-product maps.
	% https://q.uiver.app/?q=WzAsNCxbMCwwLCJcXHdpZGVoYXQgSF57aS0yfShHLFxcb3B7SG9tfV9cXFpaKFgsXFxaWikpIl0sWzEsMCwiXFx3aWRlaGF0IEheaShHLFxcb3B7SG9tfV9cXFpaKFgsQSkpIl0sWzAsMSwiXFx3aWRlaGF0IEheaShHLFxcWlopIl0sWzEsMSwiXFx3aWRlaGF0IEhee2krMn0oRyxBKSJdLFswLDEsIi1cXGN1cCB1Il0sWzAsMiwiY1xcY3VwLSIsMl0sWzIsMywiLVxcY3VwIHUiLDJdLFsxLDMsImNcXGN1cC0iXV0=&macro_url=https%3A%2F%2Fraw.githubusercontent.com%2FdFoiler%2Fnotes%2Fmaster%2Fnir.tex
	\[\begin{tikzcd}
		{\widehat H^{i-p}(G,\op{Hom}_\ZZ(X,\ZZ))} & {\widehat H^i(G,\op{Hom}_\ZZ(X,A))} \\
		{\widehat H^i(G,\ZZ)} & {\widehat H^{i+p}(G,A)}
		\arrow["{-\cup u}", from=1-1, to=1-2]
		\arrow["{c\cup-}"', from=1-1, to=2-1]
		\arrow["{-\cup u}"', from=2-1, to=2-2]
		\arrow["{c\cup-}", from=1-2, to=2-2]
	\end{tikzcd}\]
\end{lemma}
\begin{proof}
	Formally, our cup-product maps are induced by the following ``evaluation morphisms.''
	\begin{itemize}
		\item For the left arrow, we have $\eta_L\colon X\otimes_\ZZ\op{Hom}_\ZZ(X,\ZZ)\to\ZZ$ by evaluation.
		\item For the top arrow, we have $\eta_T\colon\op{Hom}_\ZZ(X,\ZZ)\otimes_\ZZ A\to\op{Hom}_\ZZ(X,A)$ by $f\otimes a\mapsto(x\mapsto f(x)a)$.
		\item For the bottom arrow, we have $\eta_B\colon\ZZ\otimes_\ZZ A\to A$ by $k\otimes a\mapsto ka$.
		\item For the right arrow, we have $\eta_R\colon X\otimes_\ZZ\op{Hom}_\ZZ(X,A)\to A$ by evaluation.
	\end{itemize}
	In particular, these maps are defined so that the following diagram commutes.
	% https://q.uiver.app/?q=WzAsNCxbMCwwLCJYXFxvdGltZXNfXFxaWlxcb3B7SG9tfV9cXFpaKFgsXFxaWilcXG90aW1lc19cXFpaIEEiXSxbMSwwLCJYXFxvdGltZXNfXFxaWlxcb3B7SG9tfV9cXFpaKFgsQSkiXSxbMCwxLCJcXFpaXFxvdGltZXNfXFxaWiBBIl0sWzEsMSwiQSJdLFswLDEsIlxcZXRhX1QiXSxbMCwyLCJcXGV0YV9MIiwyXSxbMSwzLCJcXGV0YV9SIl0sWzIsMywiXFxldGFfQiIsMl1d&macro_url=https%3A%2F%2Fraw.githubusercontent.com%2FdFoiler%2Fnotes%2Fmaster%2Fnir.tex
	\begin{equation}
		\begin{tikzcd}
			{X\otimes_\ZZ\op{Hom}_\ZZ(X,\ZZ)\otimes_\ZZ A} & {X\otimes_\ZZ\op{Hom}_\ZZ(X,A)} \\
			{\ZZ\otimes_\ZZ A} & A
			\arrow["{\eta_T}", from=1-1, to=1-2]
			\arrow["{\eta_L}"', from=1-1, to=2-1]
			\arrow["{\eta_R}", from=1-2, to=2-2]
			\arrow["{\eta_B}"', from=2-1, to=2-2]
		\end{tikzcd} \label{eq:innermorphismcoherence}
	\end{equation}
	Indeed, we can just compute along the following diagram.
	% https://q.uiver.app/?q=WzAsNCxbMCwwLCJ4XFxvdGltZXMgZlxcb3RpbWVzIGEiXSxbMSwwLCJ4XFxvdGltZXMoeCdcXG1hcHN0byBmKHgnKWEpIl0sWzAsMSwiZih4KVxcb3RpbWVzIGEiXSxbMSwxLCJmKHgpYSJdLFswLDEsIlxcZXRhX1QiLDAseyJzdHlsZSI6eyJ0YWlsIjp7Im5hbWUiOiJtYXBzIHRvIn19fV0sWzAsMiwiXFxldGFfTCIsMix7InN0eWxlIjp7InRhaWwiOnsibmFtZSI6Im1hcHMgdG8ifX19XSxbMSwzLCJcXGV0YV9SIiwwLHsic3R5bGUiOnsidGFpbCI6eyJuYW1lIjoibWFwcyB0byJ9fX1dLFsyLDMsIlxcZXRhX0IiLDIseyJzdHlsZSI6eyJ0YWlsIjp7Im5hbWUiOiJtYXBzIHRvIn19fV1d&macro_url=https%3A%2F%2Fraw.githubusercontent.com%2FdFoiler%2Fnotes%2Fmaster%2Fnir.tex
	\[\begin{tikzcd}
		{x\otimes f\otimes a} & {x\otimes(x'\mapsto f(x')a)} \\
		{f(x)\otimes a} & {f(x)a}
		\arrow["{\eta_T}", maps to, from=1-1, to=1-2]
		\arrow["{\eta_L}"', maps to, from=1-1, to=2-1]
		\arrow["{\eta_R}", maps to, from=1-2, to=2-2]
		\arrow["{\eta_B}"', maps to, from=2-1, to=2-2]
	\end{tikzcd}\]
	Now, the core of the proof is in drawing the following very large diagram.
	% https://q.uiver.app/?q=WzAsOSxbMCwwLCJcXHdpZGVoYXQgSF57aS0yfShHLFxcb3B7SG9tfV9cXFpaKFgsXFxaWikpIl0sWzEsMCwiXFx3aWRlaGF0IEheaShHLFxcb3B7SG9tfV9cXFpaKFgsXFxaWilcXG90aW1lc19cXFpaIEEpIl0sWzIsMCwiXFx3aWRlaGF0IEheaShHLFxcb3B7SG9tfV9cXFpaKFgsQSkpIl0sWzAsMSwiXFx3aWRlaGF0IEheaShHLFhcXG90aW1lc19cXFpaXFxvcHtIb219X1xcWlooWCxcXFpaKSkiXSxbMSwxLCJcXHdpZGVoYXQgSF57aSsyfShHLFhcXG90aW1lc19cXFpaXFxvcHtIb219X1xcWlooWCxcXFpaKVxcb3RpbWVzX1xcWlogQSkiXSxbMiwxLCJcXHdpZGVoYXQgSF57aSsyfShHLFhcXG90aW1lc19cXFpaXFxvcHtIb219X1xcWlooWCxBKSkiXSxbMiwyLCJcXHdpZGVoYXQgSF57aSsyfShHLEEpIl0sWzAsMiwiXFx3aWRlaGF0IEheaShHLFxcWlopIl0sWzEsMiwiXFx3aWRlaGF0IEheMihHLFhcXG90aW1lc19cXFpaIEEpIl0sWzAsMSwiLVxcY3VwIHUiXSxbMyw0LCItXFxjdXAgdSJdLFs3LDgsIi1cXGN1cCB1Il0sWzAsMywiY1xcY3VwIC0iLDJdLFsxLDQsImNcXGN1cCAtIiwyXSxbMiw1LCJjXFxjdXAgLSIsMl0sWzEsMiwiXFxldGFfVCJdLFs0LDUsIlxcZXRhX1QiXSxbOCw2LCJcXGV0YV9CIl0sWzMsNywiXFxldGFfTCIsMl0sWzQsOCwiXFxldGFfTCIsMl0sWzUsNiwiXFxldGFfUiIsMl0sWzEyLDEzLCIoMSkiLDMseyJzaG9ydGVuIjp7InNvdXJjZSI6MjAsInRhcmdldCI6MjB9LCJzdHlsZSI6eyJib2R5Ijp7Im5hbWUiOiJub25lIn0sImhlYWQiOnsibmFtZSI6Im5vbmUifX19XSxbMTMsMTQsIigyKSIsMyx7InNob3J0ZW4iOnsic291cmNlIjoyMCwidGFyZ2V0IjoyMH0sInN0eWxlIjp7ImJvZHkiOnsibmFtZSI6Im5vbmUifSwiaGVhZCI6eyJuYW1lIjoibm9uZSJ9fX1dLFsxOCwxOSwiKDMpIiwzLHsic2hvcnRlbiI6eyJzb3VyY2UiOjIwLCJ0YXJnZXQiOjIwfSwic3R5bGUiOnsiYm9keSI6eyJuYW1lIjoibm9uZSJ9LCJoZWFkIjp7Im5hbWUiOiJub25lIn19fV0sWzE5LDIwLCIoNCkiLDMseyJzaG9ydGVuIjp7InNvdXJjZSI6MjAsInRhcmdldCI6MjB9LCJzdHlsZSI6eyJib2R5Ijp7Im5hbWUiOiJub25lIn0sImhlYWQiOnsibmFtZSI6Im5vbmUifX19XV0=&macro_url=https%3A%2F%2Fraw.githubusercontent.com%2FdFoiler%2Fnotes%2Fmaster%2Fnir.tex
	\[\begin{tikzcd}
		{\widehat H^{i-p}(G,\op{Hom}_\ZZ(X,\ZZ))} & {\widehat H^i(G,\op{Hom}_\ZZ(X,\ZZ)\otimes_\ZZ A)} & {\widehat H^i(G,\op{Hom}_\ZZ(X,A))} \\
		{\widehat H^i(G,X\otimes_\ZZ\op{Hom}_\ZZ(X,\ZZ))} & {\widehat H^{i+p}(G,X\otimes_\ZZ\op{Hom}_\ZZ(X,\ZZ)\otimes_\ZZ A)} & {\widehat H^{i+2}(G,X\otimes_\ZZ\op{Hom}_\ZZ(X,A))} \\
		{\widehat H^i(G,\ZZ)} & {\widehat H^{i+2}(G,X\otimes_\ZZ A)} & {\widehat H^{i+2}(G,A)}
		\arrow["{-\cup u}", from=1-1, to=1-2]
		\arrow["{-\cup u}", from=2-1, to=2-2]
		\arrow["{-\cup u}", from=3-1, to=3-2]
		\arrow[""{name=0, anchor=center, inner sep=0}, "{c\cup -}"', from=1-1, to=2-1]
		\arrow[""{name=1, anchor=center, inner sep=0}, "{c\cup -}"', from=1-2, to=2-2]
		\arrow[""{name=2, anchor=center, inner sep=0}, "{c\cup -}"', from=1-3, to=2-3]
		\arrow["{\eta_T}", from=1-2, to=1-3]
		\arrow["{\eta_T}", from=2-2, to=2-3]
		\arrow["{\eta_B}", from=3-2, to=3-3]
		\arrow[""{name=3, anchor=center, inner sep=0}, "{\eta_L}"', from=2-1, to=3-1]
		\arrow[""{name=4, anchor=center, inner sep=0}, "{\eta_L}"', from=2-2, to=3-2]
		\arrow[""{name=5, anchor=center, inner sep=0}, "{\eta_R}"', from=2-3, to=3-3]
		\arrow["{(1)}"{marking}, Rightarrow, draw=none, from=0, to=1]
		\arrow["{(2)}"{marking}, Rightarrow, draw=none, from=1, to=2]
		\arrow["{(3)}"{marking}, Rightarrow, draw=none, from=3, to=4]
		\arrow["{(4)}"{marking}, Rightarrow, draw=none, from=4, to=5]
	\end{tikzcd}\]
	We are being asked to show that the outer square commutes; we will show that each inner square commutes, which will be enough.
	\begin{enumerate}[label=(\arabic*)]
		\item This square commutes by the associativity of the cup product.
		\item This square commutes by functoriality of cup products.
		\item This square commutes by functoriality of cup products.
		\item This square commutes by functoriality of $\widehat H^{i+p}(G,-)$ applied to \autoref{eq:innermorphismcoherence}.
	\end{enumerate}
	The above checks complete the proof.
\end{proof}
We may now proceed directly with \autoref{thm:abstractperfectpairing}.
\begin{proof}[Proof of \autoref{thm:abstractperfectpairing}]
	We use the lemma to assert that, for any $u\in H^2(G,A)$, the diagram
	\[\begin{tikzcd}
		{\widehat H^{-2}(G,\op{Hom}_\ZZ(X,\ZZ))} & {\widehat H^0(G,\op{Hom}_\ZZ(X,A))} \\
		{\widehat H^0(G,\ZZ)} & {\widehat H^{2}(G,A)}
		\arrow["{-\cup u}", from=1-1, to=1-2]
		\arrow["{c\cup-}"', from=1-1, to=2-1]
		\arrow["{-\cup u}"', from=2-1, to=2-2]
		\arrow["{c\cup-}", from=1-2, to=2-2]
	\end{tikzcd}\]
	commutes. By hypothesis, the left and right arrows are isomorphisms, so the commutativity means that showing
	\[\arraycolsep=1.4pt\begin{array}{ccc}
		\widehat H^2(G,A) &\to& \op{Hom}_\ZZ\left(\widehat H^{-2}(G,\op{Hom}_\ZZ(X,\ZZ)),\widehat H^0(G,\op{Hom}_\ZZ(X,A))\right) \\
		u &\mapsto& (a\mapsto (a\cup u))
	\end{array}\]
	is an isomorphism is the same as showing that
	\[\arraycolsep=1.4pt\begin{array}{ccc}
		\widehat H^2(G,A) &\to& \op{Hom}_\ZZ\left(\widehat H^0(G,\ZZ),\widehat H^2(G,A)\right) \\
		u &\mapsto& (k\mapsto (k\cup u))
	\end{array}\]
	is an isomorphism. Setting $n\coloneqq\#G$, we see $\widehat H^0(G,\ZZ)=\ZZ/n\ZZ$, and the cup product we are looking at sends $k\in\ZZ/n\ZZ$ and $u\in\widehat H^2(G,A)$ to $k\cup u=ku$ by how the ``evaluation'' map $\ZZ\otimes_\ZZ A\simeq A$ behaves. Thus, we are showing that
	\[\arraycolsep=1.4pt\begin{array}{ccc}
		\widehat H^2(G,A) &\to& \op{Hom}_\ZZ\left(\ZZ/n\ZZ,\widehat H^2(G,A)\right) \\
		u &\mapsto& (k\mapsto ku)
	\end{array}\]
	is an isomorphism.
	
	However, $\widehat H^2(G,A)$ is $n$-torsion, so in fact maps $\ZZ\to\widehat H^2(G,A)$ automatically have $n\ZZ$ in their kernel and hence reduce to maps $\ZZ/n\ZZ\to\widehat H^2(G,A)$. Conversely, any map $\ZZ/n\ZZ\to\widehat H^2(G,A)$ can be extended by $\ZZ\onto\ZZ/n\ZZ$ to a map $\ZZ\to\widehat H^2(G,A)$, so we have a natural isomorphism
	\[\arraycolsep=1.4pt\begin{array}{ccc}
		\op{Hom}_\ZZ\left(\ZZ/n\ZZ,\widehat H^2(G,A)\right) &\simeq& \op{Hom}_\ZZ\left(\ZZ,\widehat H^2(G,A)\right) \\
		f &\mapsto& (k\mapsto f([k])) \\
		([k]\mapsto f(k)) &\mapsfrom& f.
	\end{array}\]
	In particular, it suffices to show that
	\[\arraycolsep=1.4pt\begin{array}{ccc}
		\widehat H^2(G,A) &\to& \op{Hom}_\ZZ\left(\ZZ,\widehat H^2(G,A)\right) \\
		u &\mapsto& (k\mapsto ku)
	\end{array}\]
	is an isomorphism. But this is a standard fact about the functor $\op{Hom}_\ZZ\colon\mathrm{AbGrp}\to\mathrm{AbGrp}$, so we are done.
\end{proof}
We now synthesize this with the theory we have been building.
\begin{cor}
	Let $G$ be a finite group, and let $X$ be a $p$-encoding module. Then, given a $G$-module $A$, the cup-product pairing induces an isomorphism
	\[\widehat H^p(G,A)\to\op{Hom}_\ZZ\left(\widehat H^{-p}(G,\op{Hom}_\ZZ(X,\ZZ)),\widehat H^0(G,\op{Hom}_\ZZ(X,A))\right).\]
\end{cor}
\begin{proof}
	We apply \autoref{thm:abstractperfectpairing} to our case; we take $c$ to be the $x$ of \autoref{cor:encodingsarecups}. The cup-product maps in question are isomorphisms by \autoref{cor:betterencodingdef}. Thus, \autoref{thm:abstractperfectpairing} kicks in, completing the proof.
\end{proof}
\begin{remark}
	The other side of the pairing
	\[\widehat H^{-2}(G,\op{Hom}_\ZZ(X,\ZZ))\to\op{Hom}_\ZZ\left(\widehat H^2(G,A),\widehat H^0(G,\op{Hom}_\ZZ(X,A))\right)\]
	need not be an isomorphism; for example, take $A=0$.
\end{remark}
\begin{remark} \label{rem:artinreciptaste}
	When $X$ is $\ZZ$-free, we can think about $\op{Hom}_\ZZ(X,-)$ as a torus $T$. For example, if $L/K$ is an extension of local fields, and the torus $T$ splits over $L$, then the above statement says that the Artin reciprocity map
	\[\widehat H^{-2}(L/K,X_*(T))\to\widehat H^0(L/K,L^\times)\]
	uniquely determines $u_{L/K}\in\widehat H^2(L/K,L^\times)$. In theory, a concrete description of this reciprocity map might then be able to describe $u_{L/K}$.
\end{remark}

\section{Group Laws of Group Extensions} \label{sec:general}
% !TEX root = ../abeliangerbs.tex

Having established some background of what we expect from our encoding modules, we will spend the next few sections building a particularly nice example of a $2$-encoding module with ties to classifying group extensions.

Much of the theory in this section will be similar to that built in \cite{abelian-crossed} and \cite{cohom-abelian-crossed}. In particular, providing a group law for the extensions built from our $G$-module $A$ is essentially the same problem as being able to write down a group law for abelian crossed products. Regardless, we will build the theory from the ground.

\subsection{Motivating Results} \label{sec:singlevar}
Throughout this section, $ G$ will be a finite group and $A$ will be a $ G$-module; we will write the group operation of $A$ and the group action of $ G$ on $A$ both multiplicatively.\footnote{We denote the group law on $A$ multiplicatively for two reasons: a key example will be $A=L^\times$ where $L$ is some local field, and we do not want to denote the group law of an extension $\mc E$ of $G$ by $A$ additively because $\mc E$ need not be abelian.} To sketch the idea here, begin with an extension
\[1\to A\to\mc E\stackrel\pi\to G\to1.\]
We know that we can abstractly represent $\mc E$ as the set $A\times G$ with some group law dictated by a $2$-cocycle in $Z^2(G,A)$, so we expect that $\mc E$ can be presented by $A$ and a choice of lifts from $ G$, with some specially chosen relations.

Here are some basic observations realizing this idea. We start by lifting a single element of $ G$.
\begin{lemma} \label{lem:constructalpha}
	Let $A$ be a $ G$-module, and let 
	\[1\to A\to\mc E\stackrel\pi\to G\to1\]
	denote a group extension. Further, fix some $\sigma\in G$ of order $n_\sigma$, and find $F\in\mc E$ such that $\sigma=\pi(F)$. Then
	\[\alpha\coloneqq F^{n_\sigma}\]
	has $\alpha\in A^{\langle\sigma\rangle}$.
\end{lemma}
\begin{proof}
	A priori, we only know that $\alpha\in\mc E$, so we compute
	\[\pi(\alpha)=\pi\left(F^{n_\sigma}\right)=\sigma^{n_\sigma}=1,\]
	so $\alpha\in\ker\pi=A$. Thus, we may say that
	\[\sigma(\alpha)=F\alpha F^{-1}=F^{n_\sigma}=\alpha,\]
	so $\alpha\in A^{\langle\sigma\rangle}$, as desired.
\end{proof}
We can make the above proof more explicit by specifying the group law of $\mc E$.
\begin{lemma} \label{lem:explicitalpha}
	Let $A$ be a $ G$-module. Picking up some $2$-cocycle $c\in Z^2( G,A)$, let
	\[1\to A\to\mc E_c\stackrel\pi\to G\to1\]
	be the corresponding extension. Fixing $\sigma\in G$ of order $n_\sigma$, let $F\coloneqq(m,\sigma)\in\mc E_c$ be a lift. Supposing $c(1,\sigma)=1$, then
	\[F^{n_\sigma}=N_\sigma(m)\prod_{i=0}^{n_\sigma-1}c\left(\sigma^i,\sigma\right),\]
	where $N_\sigma\coloneqq\sum_{i=0}^{n_\sigma-1}\sigma^i$.
\end{lemma}
\begin{proof}
	This is a direct computation. By induction, we can show that
	\[F^k=\left(\prod_{i=0}^{k-1}\sigma^i(m)c\left(\sigma^i,\sigma\right),\sigma^k\right)\]
	for $k\in\NN$. Indeed, there is nothing to say for $k=0$, and the inductive step merely expands out $F^k\cdot F$.

	It follows that
	\[F^{n_\sigma}=\left(\prod_{i=0}^{n_\sigma-1}\sigma^i(m)\cdot\prod_{i=0}^{n_\sigma-1}c\left(\sigma^i,\sigma\right),1\right),\]
	which is what we wanted.
\end{proof}
Having this explicit formula lets us say how $\alpha$ changes as we vary the lift.
\begin{prop} \label{prop:findallalpha}
	Let $A$ be a $ G$-module. Fixing a cohomology class $u\in H^2( G,A)$, let 
	\[1\to A\to\mc E\stackrel\pi\to G\to1\]
	be a group extension whose isomorphism class corresponds to $u$. Further, fix some $\sigma\in G$ of order $n_\sigma$, and let $A_\sigma\coloneqq A^{\langle\sigma\rangle}$ be the fixed submodule. Then the set
	\[S_{\mc E,\sigma}\coloneqq\left\{F^{n_\sigma}:\pi(F)=\sigma\right\}\]
	is an equivalence class in $A_\sigma/N_\sigma(A)$, independent of the choice of $\mc E$; here, $N_\sigma\coloneqq\sum_{i=1}^{n_\sigma-1}\sigma^i$.
\end{prop}
\begin{proof}
	Note that $S_{\mc E,\sigma}\subseteq A_\sigma$ already from \autoref{lem:constructalpha}.
	
	The point is to use \autoref{lem:explicitalpha}. Note the extension $\mc E$ corresponds to the equivalence class $u\in H^2( G,A)$, so let $c\in Z^2( G,A)$ be a representative. Letting $\mc E_c$ be the extension constructed from $c$, we are promised an isomorphism $\varphi\colon\mc E\cong\mc E_c$ making the following diagram commute.
	% https://q.uiver.app/?q=WzAsMTAsWzAsMCwiMSJdLFsxLDAsIkxeXFx0aW1lcyJdLFsyLDAsIlxcbWMgRSJdLFszLDAsIlxcR2FtbWEiXSxbNCwwLCIxIl0sWzAsMSwiMSJdLFsxLDEsIkxeXFx0aW1lcyJdLFsyLDEsIlxcbWMgRV9jIl0sWzMsMSwiXFxHYW1tYSJdLFs0LDEsIjEiXSxbMCwxXSxbMSwyXSxbMiwzLCJcXHBpIl0sWzMsNF0sWzUsNl0sWzYsN10sWzcsOCwiXFxwaV9jIl0sWzgsOV0sWzIsNywiXFx2YXJwaGkiXSxbMSw2LCIiLDEseyJsZXZlbCI6Miwic3R5bGUiOnsiaGVhZCI6eyJuYW1lIjoibm9uZSJ9fX1dLFszLDgsIiIsMSx7ImxldmVsIjoyLCJzdHlsZSI6eyJoZWFkIjp7Im5hbWUiOiJub25lIn19fV1d&macro_url=https%3A%2F%2Fraw.githubusercontent.com%2FdFoiler%2Fnotes%2Fmaster%2Fnir.tex
	\[\begin{tikzcd}
		1 & {A} & {\mc E} &  G & 1 \\
		1 & {A} & {\mc E_c} &  G & 1
		\arrow[from=1-1, to=1-2]
		\arrow[from=1-2, to=1-3]
		\arrow["\pi", from=1-3, to=1-4]
		\arrow[from=1-4, to=1-5]
		\arrow[from=2-1, to=2-2]
		\arrow[from=2-2, to=2-3]
		\arrow["{\pi_c}", from=2-3, to=2-4]
		\arrow[from=2-4, to=2-5]
		\arrow["\varphi", from=1-3, to=2-3]
		\arrow[Rightarrow, no head, from=1-2, to=2-2]
		\arrow[Rightarrow, no head, from=1-4, to=2-4]
	\end{tikzcd}\]
	We start by claiming that $S_{\mc E,\sigma}=S_{\mc E_c,\sigma}$, which will show that $S_{\mc E,\sigma}$ is independent of the choice of representative $\mc E$. To show $S_{\mc E,\sigma}\subseteq S_{\mc E_c,\sigma}$, note that $\alpha\in S_{\mc E,\sigma}$ has $F\in\mc E$ with $\pi(F)=\sigma$ and $\alpha=F^{n_\sigma}$. Pushing this through $\varphi$, we see $\varphi(F)\in\mc E_c$ has
	\[\pi_c(\varphi(F))=\varphi(\pi(F))=\sigma\qquad\text{and}\qquad\varphi(F)^{n_\sigma}=\varphi(F^{n_\sigma})=\alpha,\]
	so $\alpha\in S_{\mc E_c,\sigma}$ follows. An analogous argument with $\varphi^{-1}$ shows the other needed inclusion.

	It thus suffices to show that $S_{\mc E_c,\sigma}$ is an equivalence class in $A_\sigma/N_\sigma(A)$. However, this is exactly what \autoref{lem:explicitalpha} says as we let the possible lifts $F=(m,\sigma)\in\mc E_c$ of $\sigma$ vary over $m\in A$.
\end{proof}
The fact that we are taking elements of $ G$ to equivalence classes in $A_\sigma/N_\sigma\left(A\right)$ is reminiscent of the (inverse) Artin reciprocity map, and indeed that is exactly what is going on.
\begin{cor} \label{cor:alphaiscupproduct}
	Work in the context of \autoref{prop:findallalpha}. Then
	\[S_\sigma\coloneqq S_{\mc E,\sigma}=[\sigma]\cup[\op{Res}c],\]
	where $\cup\colon\widehat H^{-2}(\langle\sigma\rangle,\ZZ)\times\widehat H^2(\langle\sigma\rangle,A)\to\widehat H^0(\langle\sigma\rangle,A)$ is the cup product in Tate cohomology.
\end{cor}
\begin{proof}
	Note that $S_\sigma\in A_\sigma/N_\sigma(A)=\widehat H^0(\langle\sigma\rangle,A)$, so the conclusion at least makes sense.
	
	Now, using notation as in the proof of \autoref{prop:findallalpha}, we recall that $S_\sigma=S_{\mc E_c,\sigma}$, so it suffices to prove the result for $\mc E_c$. Well, by \autoref{lem:explicitalpha}, $S_\sigma\in A_\sigma/N_\sigma(A)$ is represented by
	\[\prod_{i=0}^{n_\sigma-1}c\left(\sigma^i,\sigma\right),\]
	which is exactly the cup product $[\sigma]\cup[c]$.
\end{proof}
\begin{cor}
	Let $L/K$ be a finite Galois extension of local fields with Galois group $ G\coloneqq\op{Gal}(L/K)$. Further, let
	\[1\to L^\times\to\mc E\stackrel\pi\to G\to1\]
	be an $L/K$-gerb bound by $\mathbb G_m$ whose isomorphism class corresponds to the fundamental class $u_{L/K}\in H^2( G,L^\times)$. Further, fix some $\sigma\in G$ of order $n_\sigma$, and let $L_\sigma\coloneqq L^{\langle\sigma\rangle}$ be the fixed field. Then
	\[\theta_{L/L_\sigma}^{-1}(\sigma)=\left\{F^{n_\sigma}:\pi(F)=\sigma\right\}.\]
\end{cor}
\begin{proof}
	Recalling $\theta_{L/L_\sigma}^{-1}$ is a cup product map, note that $\theta_{L/L_\sigma}^{-1}(\sigma)$ is given by $[\sigma]\cup u_{L/K}$. So we are done by \autoref{cor:alphaiscupproduct}.
\end{proof}
The above results are all interested in lifting single elements of $ G$ and studying how they behave on their own. In the discussion that follows, we will need to study how the lifts interact with each other, but for now, we will justify why lifts are adequate to study at all.
\begin{proposition} \label{prop:liftsgenerate}
	Let $A$ be a $ G$-module. Further, let
	\[1\to A\to\mc E\stackrel\pi\to G\to1\]
	be a group extension. Given elements $\Sigma\subseteq G$ which generate $ G$, then $\mc E$ is generated by $A$ and a set of lifts $\{F_\sigma\}_{\sigma\in\Sigma}$ with $\pi(F_\sigma)=\sigma$ for each $\sigma\in\Sigma$.
\end{proposition}
\begin{proof}
	Fix some element $w\in\mc E$, which we need to exhibit as a product of elements in $A$ and $F_\sigma$s. Well, because the $\sigma\in\Sigma$ generate $ G$, we know that $\pi(w)\in G$ can be written as
	\[\pi(w)=\prod_{\sigma\in\Sigma}^m\sigma^{a_\sigma}\]
	for some sequence of integers $\{a_\sigma\}_{\sigma\in\Sigma}\in\NN^{\oplus\Sigma}$. It follows that
	\[\pi\left(\frac w{\prod_{\sigma\in\Sigma}F_\sigma^{a_\sigma}}\right)=1,\]
	so $w/\prod_{\sigma\in\Sigma}F_\sigma^{a_\sigma}\in\ker\pi=A$. Thus, we set $a\in A$ to be the quotient $w/\prod_{\sigma\in\Sigma}F_\sigma^{a_\sigma}$ so that
	\[w=a\cdot\prod_{\sigma\in\Sigma}F_\sigma^{a_\sigma},\]
	which is what we wanted.
\end{proof}

% \section{Abelian Group Extensions} \label{sec:abelian}
% !TEX root = ../abeliangerbs.tex

\subsection{Extensions to Tuples}
The above proofs technically don't even require that the group $ G$ is abelian. If we want to keep track of the fact our group is abelian, we should extract the elements of $A$ which can do so.
\begin{lemma} \label{lem:constructalphabeta}
	Let $A$ be a $ G$-module, and let 
	\[1\to A\to\mc E\stackrel\pi\to G\to1\]
	be a group extension. Further, fix some $F_1,F_2\in\mc E$ and define $\sigma_i\coloneqq\pi(F_i)$ for $i\in\{1,2\}$, and let $\sigma_i\in G$ have order $n_i$. Then, setting
	\[\alpha_i\coloneqq F_i^{n_i}\qquad\text{and}\qquad\beta\coloneqq F_1F_2F_1^{-1}F_2^{-1},\]
	we have the following.
	\begin{listalph}
		\item $\alpha_i\in A^{\langle\sigma_i\rangle}$ for $i\in\{1,2\}$ and $\beta\in A$.
		\item $N_1(\beta)=\alpha_1/\sigma_2(\alpha_1)$ and $N_2(\beta^{-1})=\alpha_2/\sigma_1(\alpha_2)$, where $N_i\coloneqq\sum_{p=0}^{n_i-1}\sigma_i^p$.
	\end{listalph}
\end{lemma}
\begin{proof}
	These checks are a matter of force. For brevity, we set $A_i\coloneqq A^{\langle\sigma_i\rangle}$ for $i\in\{1,2\}$.
	\begin{listalph}
		\item That $\alpha_i\in A_i$ follows from \autoref{lem:constructalpha}. Lastly, $\beta\in A$ follows from noting
		\[\pi(\beta)=\pi(F_1)\pi(F_2)\pi(F_1)^{-1}\pi(F_2)^{-1}=1,\]
		so $\beta\in\ker\pi=A$.
		\item We will check that $\op N_{L/L_1}(\beta)=\alpha_1/\sigma_2(\alpha_1)$; the other equality follows symmetrically after switching $1$s and $2$s because $\beta^{-1}=F_2F_1F_2^{-1}F_1^{-1}$. Well, we compute
		\begin{align*}
			N_1(\beta) &= \sigma_1^{-1}(\beta)\cdot\sigma_1^{-2}(\beta)\cdot\sigma^{-3}\cdot\ldots\cdot\sigma^{-n_1}(\beta) \\
			&= F_1^{-1}\left(F_1F_2F_1^{-1}F_2^{-1}\right)F_1 \\
			&\phantom{{}={}}\cdot F_1^{-2}\left(F_1F_2F_1^{-1}F_2^{-1}\right)F_1^2 \\
			&\phantom{{}={}}\cdot F_1^{-3}\left(F_1F_2F_1^{-1}F_2^{-1}\right)F_1^3\cdot\ldots \\
			&\phantom{{}={}}\cdot F_1^{-n_1}(F_1F_2F_1^{-1}F_2^{-1})F_1^{n_1} \\
			% &= F_2F_1^{-1}F_2^{-1} \\
			% &\phantom{{}={}}\cdot F_2F_1^{-1}F_2^{-1} \\
			% &\phantom{{}={}}\cdot F_2F_1^{-1}F_2^{-1}\cdot\ldots \\
			% &\phantom{{}={}}\cdot F_2F_1^{-1}F_2^{-1}F_1^{n_1} \\
			&= F_2F_1^{-1} \\
			&\phantom{{}={}}\cdot F_1^{-1} \\
			&\phantom{{}={}}\cdot F_1^{-1}\cdot\ldots \\
			&\phantom{{}={}}\cdot F_1^{-1}F_2^{-1}F_1^{n_1} \\
			&= F_2F_1^{-n_1}F_2^{-1}F_1^{n_1} \\
			&= \alpha_1/\sigma_2(\alpha_1).
		\end{align*}
	\end{listalph}
	The above computations finish the proof.
\end{proof}
The proof of (b) above might appear magical, but in fact it comes from a more general idea.
\begin{lemma} \label{lem:switchtwo}
	Fix everything as in \autoref{lem:constructalphabeta}. Then, for $x,y\ge0$, we have
	\[F_1^xF_2^y=\prod_{k=0}^{x-1}\prod_{\ell=0}^{y-1}\sigma_1^k\sigma_2^\ell(\beta)F_2^yF_1^x.\]
\end{lemma}
\begin{proof}
	We induct. We take a moment to write out the case of $x=1$, for which we induct on $y$. To be explicit, we will prove
	\[F_1F_2^y=\prod_{\ell=0}^{y-1}\sigma_2^\ell(\beta)F_2^yF_1.\]
	For $y=0$, there is nothing to say. So suppose the statement for $y$ (and $x=1$), and we show $y+1$ (and $x=1$). Well, we compute
	\begin{align*}
		F_1F_2^{y+1} &= F_1F_2^y\cdot F_2 \\
		&= \prod_{\ell=0}^{y-1}\sigma_2^\ell(\beta)F_2^yF_1\cdot F_2 \\
		&= \prod_{\ell=0}^{y-1}\sigma_2^\ell(\beta)F_2^y\beta F_2F_1 \\
		&= \prod_{\ell=0}^{y-1}\sigma_2^\ell(\beta)\cdot \sigma_2^y(\beta)F_2^y\cdot F_2F_1 \\
		&= \prod_{\ell=0}^{(y+1)-1}\sigma_2^\ell(\beta)\cdot F_2^{y+1}F_1,
	\end{align*}
	which is what we wanted.
	
	We now move on to the general case. We will induct on $y$. Note that $y=0$ makes the product empty, leaving us with $F_1^x=F_1^x$, for any $x$. So suppose that the statement is true for some $y\ge0$, and we will show $y+1$. For this, we now turn to inducting on $x$. For $x=0$, we note that the product is once again empty, so we are left with showing $F_2^{y+1}=F_2^{y+1}$, which is true.
	
	To finish, we suppose the statement for $x$ and show the statement for $x+1$. Well, we compute
	\begin{align*}
		F_1^{x+1}F_2^{y+1} &= F_1\cdot F_1^xF_2^{y+1} \\
		&= F_1\cdot \prod_{k=0}^{x-1}\prod_{\ell=0}^{(y+1)-1}\sigma_1^k\sigma_2^\ell(\beta)\cdot F_2^{y+1}F_1^x \\
		&= \sigma_1\left(\prod_{k=0}^{x-1}\prod_{\ell=0}^{(y+1)-1}\sigma_1^k\sigma_2^\ell(\beta)\right)\cdot F_1F_2^{y+1}F_1^x \\
		&= \prod_{k=1}^{(x+1)-1}\prod_{\ell=0}^{(y+1)-1}\sigma_1^k\sigma_2^\ell(\beta)\cdot F_1F_2^{y+1}F_1^x \\
		&= \prod_{k=1}^{(x+1)-1}\prod_{\ell=0}^{(y+1)-1}\sigma_1^k\sigma_2^\ell(\beta)\cdot \prod_{\ell=0}^{(y+1)-1}\sigma_2^\ell(\beta)\cdot \sigma_2^y(\beta)\cdot F_2^{y+1}F_1\cdot F_1^x \\
		&= \prod_{k=0}^{(x+1)-1}\prod_{\ell=0}^{(y+1)-1}\sigma_1^k\sigma_2^\ell(\beta)F_2^{y+1}F_1^{x+1},
	\end{align*}
	which is what we wanted.
\end{proof}
\begin{remark} \label{rem:alphabetarelation}
	Setting $x=n_1$ and $y=1$ recovers $\op N_{L/L^{\langle\sigma_1\rangle}}(\beta)=\alpha_1/\sigma_2(\alpha_1)$.
\end{remark}
In particular, \autoref{rem:alphabetarelation} tells us that coherence of the group law in $\mc E$ should give rise to relations between our elements of $A$. Here is a more complex example.
\begin{lemma} \label{lem:betarelations}
	Let $A$ be a $ G$-module, and let 
	\[1\to A\to\mc E\stackrel\pi\to G\to1\]
	be a group extension. Further, fix some $F_1,F_2,F_3\in\mc E$ and define $\sigma_i\coloneqq\pi(F_i)$ for $i\in\{1,2,3\}$, and let $\sigma_i\in G$ have order $n_i$. Then, setting
	\[\beta_{ij}\coloneqq F_iF_jF_i^{-1}F_j^{-1}\]
	for each pair of indices $(i,j)$ with $i>j$. Then
	\[\frac{\sigma_2(\beta_{31})}{\beta_{31}}=\frac{\sigma_1(\beta_{32})}{\beta_{32}}\cdot\frac{\sigma_3(\beta_{21})}{\beta_{21}}.\]
\end{lemma}
\begin{proof}
	The point is to turn $F_3F_2F_1$ into $F_1F_2F_3$ in two different ways. On one hand,
	\begin{align*}
		(F_3F_2)F_1 &= \beta_{32}F_2F_3F_1 \\
		&= \beta_{32}F_2\beta_{31}F_1F_3 \\
		&= \beta_{32}\sigma_2(\beta_{31})(F_2F_1)F_3 \\
		&= \beta_{32}\sigma_2(\beta_{31})\beta_{21}F_1F_2F_3.
	\end{align*}
	On the other hand,
	\begin{align*}
		F_3(F_2F_1) &= F_3\beta_{21}F_1F_2 \\
		&= \sigma_3(\beta_{21})(F_3F_1)F_2 \\
		&= \sigma_3(\beta_{21})\beta_{31}F_1(F_3F_2) \\
		&= \sigma_3(\beta_{21})\beta_{31}F_1\beta_{32}F_2F_3 \\
		&= \sigma_3(\beta_{21})\beta_{31}\sigma_1(\beta_{32})F_1F_2F_3.
	\end{align*}
	Thus,
	\[\beta_{32}\sigma_2(\beta_{31})\beta_{21}=\sigma_3(\beta_{21})\beta_{31}\sigma_1(\beta_{32}),\]
	which rearranges into the desired equation.
\end{proof}
\begin{remark}
	The relation from \autoref{lem:betarelations} may look asymmetric in the $\beta_{ij}$, but this is because the definitions of the $\beta_{ij}$s themselves are asymmetric in $F_i$.
\end{remark}

\subsection{Tuples to Cocycles}
\subsubsection{The Set-Up}
The proceeding lemma is intended to give intuition that the element $\beta$ is helping to specify the group law on $\mc E$.

More concretely, we will take the following set-up for the following results: fix a $ G$-module $A$, and let
\[1\to A\to\mc E\to G\to1\]
be a group extension. Once we choose elements $\{\sigma_i\}_{i=1}^m$ generating $ G$, we know by \autoref{prop:liftsgenerate} that we can generate $\mc E$ by $A$ and some arbitrarily chosen lifts $\{F_i\}_{i=1}^m$ of the $\{\sigma_i\}_{i=1}^m$. Then, letting $n_i$ be the order of $\sigma_i$, we set
\[\alpha_i\coloneqq F_i^{n_i}\]
for each index $i$ and
\[\beta_{ij}\coloneqq F_iF_jF_i^{-1}F_j^{-1}\]
for each index $1\le j<i\le m$. Notably, we will not need more $\beta$s: indeed, $\beta_{ii}=1$ and $\beta_{ij}=\beta_{ji}^{-1}$ for any $i$ and $j$. Setting $A_i\coloneqq A^{\langle\sigma_i\rangle}$ and $N_i\coloneqq\sum_{p=0}^{n_i-1}\sigma_i^p$, the story so far is that
\begin{equation}
	\alpha_i\in A_i\text{ for each }i\qquad\text{and}\qquad\beta_{ij}\in A\text{ for each }i>j \label{eq:tuplefields}
\end{equation}
and
\begin{equation}
	N_i(\beta_{ij})=\alpha_i/\sigma_j(\alpha_i)\qquad\text{and}\qquad N_j(\beta_{ij}^{-1})=\alpha_j/\sigma_i(\alpha_j)\qquad\text{ for each }i>j \label{eq:tuplerelations}
\end{equation}
by \autoref{lem:constructalphabeta}, and
\begin{equation}
	\frac{\sigma_j(\beta_{ik})}{\beta_{ik}}=\frac{\sigma_k(\beta_{ij})}{\beta_{ij}}\cdot\frac{\sigma_i(\beta_{jk})}{\beta_{jk}}\qquad\text{ for each }i>j>k \label{eq:betarelations}
\end{equation}
by \autoref{lem:betarelations}. This data is so important that we will give it a name.
\begin{definition}
	In the above set-up, the data of $(\{\alpha_i\},\{\beta_{ij}\})$ satisfying \autoref{eq:tuplefields} and \autoref{eq:tuplerelations} and \autoref{eq:betarelations} will be called a \textit{$\{\sigma_i\}_{i=1}^m$-tuple}. When understood, the $\{\sigma_i\}_{i=1}^m$ will be abbreviated. Once $G$ and $A$ are fixed, we will denote the set of $\{\sigma_i\}_{i=1}^m$-tuples by $\mathcal T(G,A)$.
\end{definition}
Note that this definition is independent of $\mc E$, but a choice of extension $\mc E$ and lifts $F_i$ give a $\{\sigma_i\}_{i=1}^m$-tuple as described above.
\begin{remark}
	The $\mathcal T(G,A)$ form a group under multiplication in $A$. Indeed, the conditions \autoref{eq:tuplefields} and \autoref{eq:tuplerelations} and \autoref{eq:betarelations} are closed under multiplication and inversion.
\end{remark}
We also know from \autoref{lem:switchtwo} that
\[F_i^xF_j^y=\prod_{k=0}^{x-1}\prod_{\ell=0}^{y-1}\sigma_i^k\sigma_j^\ell(\beta_{ij})F_j^yF_i^x\]
for $i>j$ and $x,y\ge0$. It will be helpful to have some notation for the residue term in $A$, so we define
\[\beta_{ij}^{(xy)}\coloneqq\prod_{k=0}^{x-1}\prod_{\ell=0}^{y-1}\sigma_i^k\sigma_j^\ell(\beta_{ij}).\]
Now, combined with the fact that $F_ix=\sigma_i(x)F_i$ for each $F_i$ and $x\in A$, we have been approximately told how the group operation works in $\mc E$. Namely, we could conceivably write any element of $\mc E$ in the form
\[xF_1^{a_1}\cdots F_m^{a_m}\]
for $x\in A$ and $a_i\in\ZZ/n_i\ZZ$ because we know how to make these elements commute and generate $\mc E$. Further, we can multiply out two terms of the form
\[xF_1^{a_1}\cdots F_m^{a_m}\cdot yF_1^{b_1}\cdots F_m^{b_m}\]
into a term of the form $zF_1^{c_1}\cdots F_m^{c_m}$. In fact, it will be helpful for us to see how to do this.
\begin{proposition} \label{prop:multiplytwoelements}
	Fix everything as in the set-up, except drop the assumption that $\{\sigma_i\}_{i=1}^m$ generate $ G$. Then, choosing $a_i,b_i\in\NN$ for each $i$, we have
	\[\left(\prod_{i=1}^mF_i^{a_i}\right)\left(\prod_{i=1}^mF_i^{b_i}\right)=\left[\prod_{1\le j<i\le m}\Bigg(\prod_{1\le k<j}\sigma_k^{a_k+b_k}\Bigg)\Bigg(\prod_{j\le k<i}\sigma_k^{a_k}\Bigg)\beta_{ij}^{(a_ib_j)}\right]\left(\prod_{i=1}^mF_i^{a_i+b_i}\right).\]
\end{proposition}
\begin{proof}
	The reason that we dropped the assumption on $\{\sigma_i\}_{i=1}^m$ is so that we may induct directly on $m$. We start by showing that
	\[\left(\prod_{i=1}^mF_i^{a_i}\right)F_1^{b_1}=\left[\prod_{1<i\le m}\left(\prod_{1\le k<i}\sigma_k^{a_k}\right)\beta_{i1}^{(a_ib_1)}\right]F_1^{a_1+b_1}\prod_{i=2}^mF_i^{a_i}.\]
	We do this by induction on $m$. When $m=0$ and even for $m=1$, there is nothing to say. For the inductive step, we assume
	\[\left(\prod_{i=1}^mF_i^{a_i}\right)F_1^{b_1}=\left[\prod_{1<i\le m}\left(\prod_{1\le k<i}\sigma_k^{a_k}\right)\beta_{i1}^{(a_ib_1)}\right]F_1^{a_1+b_1}\prod_{i=2}^mF_i^{a_i}\]
	and compute
	\begin{align*}
		\left(\prod_{i=1}^{m+1}F_i^{a_i}\right)F_1^{b_1} &= \left(\prod_{i=1}^{m}F_i^{a_i}\right)F_{m+1}^{a_{m+1}}F_1^{b_1} \\
		&= \left(\prod_{i=1}^{m}F_i^{a_i}\right)\beta_{m+1,1}^{(a_{m+1}b_1)}F_1^{b_1}F_{m+1}^{a_{m+1}} \\
		&= \left[\left(\prod_{k=1}^m\sigma_k^{a_k}\right)\beta_{m+1,1}^{(a_{m+1}b_1)}\right]\left[\prod_{1<i\le m}\left(\prod_{1\le k<i}\sigma_k^{a_k}\right)\beta_{i1}^{(a_ib_1)}\right]F_1^{a_1+b_1}\left(\prod_{i=2}^mF_i^{a_i}\right)F_{m+1}^{a_{m+1}} \\
		&= \left[\prod_{1<i\le m+1}\left(\prod_{1\le k<i}\sigma_k^{a_k}\right)\beta_{i1}^{(a_ib_1)}\right]F_1^{a_1+b_1}\left(\prod_{i=2}^{m+1}F_i^{a_i}\right),
	\end{align*}
	which completes our inductive step.

	We now attack the statement of the proposition directly, again inducting on $m$. For $m=0$ and even for $m=1$, there is again nothing to say. For the inductive step, take $m>1$, and we get to assume that
	\[\left(\prod_{i=2}^mF_i^{a_i}\right)\left(\prod_{i=2}^mF_i^{b_i}\right)=\left[\prod_{2\le j<i\le m}\Bigg(\prod_{2\le k<j}\sigma_k^{a_k+b_k}\Bigg)\Bigg(\prod_{j\le k<i}\sigma_k^{a_k}\Bigg)\beta_{ij}^{(a_ib_j)}\right]\left(\prod_{i=2}^mF_i^{a_i+b_i}\right).\]
	From here, we can compute
	\begin{align*}
		\left(\prod_{i=1}^mF_i^{a_i}\right)\left(\prod_{i=1}^mF_i^{b_i}\right) &= \left(\prod_{i=1}^mF_i^{a_i}\right)F_1^{b_1}\left(\prod_{i=2}^mF_i^{b_i}\right) \\
		&= \left[\prod_{1<i\le m}\Bigg(\prod_{1\le k<i}\sigma_k^{a_k}\Bigg)\beta_{i1}^{(a_ib_1)}\right]F_1^{a_1+b_1}\left(\prod_{i=2}^mF_i^{a_i}\right)\left(\prod_{i=2}^mF_i^{b_i}\right) \\
		&= \left[\prod_{1<i\le m}\Bigg(\prod_{1\le k<i}\sigma_k^{a_k}\Bigg)\beta_{i1}^{(a_ib_1)}\right]F_1^{a_1+b_1}\cdot \\
		&\qquad\qquad\left[\prod_{2\le j<i\le m}\Bigg(\prod_{2\le k<j}\sigma_k^{a_k+b_k}\Bigg)\Bigg(\prod_{j\le k<i}\sigma_k^{a_k}\Bigg)\beta_{ij}^{(a_ib_j)}\right]\left(\prod_{i=2}^mF_i^{a_i+b_i}\right) \\
		&= \left[\prod_{1<i\le m}\Bigg(\prod_{1\le k<i}\sigma_k^{a_k}\Bigg)\beta_{i1}^{(a_ib_1)}\right]\cdot \\
		&\qquad\qquad \sigma_1^{a_1+b_1}\left[\prod_{2\le j<i\le m}\Bigg(\prod_{2\le k<j}\sigma_k^{a_k+b_k}\Bigg)\Bigg(\prod_{j\le k<i}\sigma_k^{a_k}\Bigg)\beta_{ij}^{(a_ib_j)}\right]\left(\prod_{i=2}^mF_i^{a_i+b_i}\right).
	\end{align*}
	From here, a little rearrangement finishes the inductive step.
\end{proof}
The reason we exerted this pain upon ourselves is for the following result.
\begin{prop} \label{prop:writedowncocycle}
	Fix everything as in the set-up. Then, if well-defined, we can represent the cohomology class corresponding to $\mc E$ by the cocycle
	\[c(g,h)\coloneqq\left[\prod_{1\le j<i\le m}\Bigg(\prod_{1\le k<j}\sigma_k^{a_k+b_k}\Bigg)\Bigg(\prod_{j\le k<i}\sigma_k^{a_k}\Bigg)\beta_{ij}^{(a_ib_j)}\right]\left[\prod_{i=1}^m\Bigg(\prod_{1\le k<i}\sigma_k^{a_k+b_k}\Bigg)\alpha_i^{\floor{\frac{a_i+b_i}{n_i}}}\right],\]
	where $g=\prod_i\sigma_i^{a_i}$ and $h=\prod_i\sigma_i^{b_i}$.
\end{prop}
Observe that \autoref{prop:writedowncocycle} has a fairly strong hypothesis that $c$ is well-defined; we will return to this later.
\begin{proof}
	Very quickly, we use the division algorithm to define
	\[a_i+b_i=n_iq_i+r_i\]
	where $q_\in\{0,1\}$ and $0\le r_i<n_i$. In particular,
	\[gh=\prod_{i=1}^mF_i^{r_i}.\]
	Now, because the elements $\sigma_i$ generate $ G$, we see that the lifts $\sigma_i\mapsto F_i$ defines a section $s\colon G\to\mc E$. As such, we can compute a representing cocycle for our cohomology class as
	\begin{align*}
		c(g,h) &= s(g)s(h)s(gh)^{-1} \\
		&= \Bigg(\prod_{i=1}^mF_i^{a_i}\Bigg)\Bigg(\prod_{i=1}^mF_i^{b_i}\Bigg)\Bigg(\prod_{i=1}^mF_i^{r_i}\Bigg)^{-1} \\
		&= \left[\prod_{1\le j<i\le m}\Bigg(\prod_{1\le k<j}\sigma_k^{a_k+b_k}\Bigg)\Bigg(\prod_{j\le k<i}\sigma_k^{a_k}\Bigg)\beta_{ij}^{(a_ib_j)}\right]\left(\prod_{i=1}^mF_i^{a_i+b_i}\right)\Bigg(\prod_{i=1}^mF_{m-i+1}^{-r_{m-i+1}}\Bigg).
	\end{align*}
	It remains to deal with the last products; we claim that it is equal to
	\[\left(\prod_{i=1}^mF_i^{a_i+b_i}\right)\Bigg(\prod_{i=1}^mF_{m-i+1}^{-r_{m-i+1}}\Bigg)=\prod_{i=1}^m\Bigg(\prod_{1\le k<i}\sigma_k^{a_k+b_k}\Bigg)\alpha_i^{q_i},\]
	which will finish the proof. We induct on $m$; for $m=0$ and $m=1$, there is nothing to say. For the inductive step, we assume that
	\[\left(\prod_{i=2}^mF_i^{a_i+b_i}\right)\Bigg(\prod_{i=1}^{m-1}F_{m-i+1}^{-r_{m-i+1}}\Bigg)=\prod_{i=2}^m\Bigg(\prod_{2\le k<i}\sigma_k^{a_k+b_k}\Bigg)\alpha_i^{q_i}\]
	and compute
	\begin{align*}
		\left(\prod_{i=1}^mF_i^{a_i+b_i}\right)\Bigg(\prod_{i=1}^mF_{m-i+1}^{-r_{m-i+1}}\Bigg) &= F_1^{a_1+b_1}\left(\prod_{i=2}^mF_i^{a_i+b_i}\right)\Bigg(\prod_{i=1}^{m-1}F_{m-i+1}^{-r_{m-i+1}}\Bigg)F_1^{-a_1-b_1}F_1^{a_1+b_1-r_1} \\
		&= F_1^{a_1+b_1}\left(\prod_{i=2}^m\Bigg(\prod_{2\le k<i}\sigma_k^{a_k+b_k}\Bigg)\alpha_i^{q_i}\right)F_1^{-a_1-b_1}\alpha_1^{q_1} \\
		&= \left(\prod_{i=2}^m\Bigg(\prod_{1\le k<i}\sigma_k^{a_k+b_k}\Bigg)\alpha_i^{q_i}\right)\alpha_1^{q_1} \\
		&= \prod_{i=1}^m\Bigg(\prod_{1\le k<i}\sigma_k^{a_k+b_k}\Bigg)\alpha_i^{q_i},
	\end{align*}
	finishing.
\end{proof}

\subsubsection{The Modified Set-Up}
A priori we have no reason to expect that the $c$ constructed in \autoref{prop:writedowncocycle} is actually a cocycle, especially if the $\sigma_i$ have nontrivial relations.

To account for this, we modify our set-up slightly. By the classification of finitely generated abelian groups, we may write
\[ G\simeq\bigoplus_{k=1}^m G_k,\]
where $ G_k\subseteq G$ with $ G_k\cong\ZZ/n_k\ZZ$ and $n_k>1$ for each $n_k$. As such, we let $\sigma_k$ be a generating element of $ G_k$ so that we still know that the $\sigma_k$ generate $ G$. In this case, we have the following result.
\begin{theorem} \label{thm:getcocycle}
	Fix everything as in the modified set-up, forgetting about the extension $\mc E$. Then a $\{\sigma_i\}_{i=1}^m$-tuple of $\{\alpha_i\}_{i=1}^m$ and $\{\beta_{ij}\}_{i>j}$ makes
	\[c(g,h)\coloneqq\left[\prod_{1\le j<i\le m}\Bigg(\prod_{1\le k<j}\sigma_k^{a_k+b_k}\Bigg)\Bigg(\prod_{j\le k<i}\sigma_k^{a_k}\Bigg)\beta_{ij}^{(a_ib_j)}\right]\left[\prod_{i=1}^m\Bigg(\prod_{1\le k<i}\sigma_k^{a_k+b_k}\Bigg)\alpha_i^{\floor{\frac{a_i+b_i}{n_i}}}\right],\]
	where $g\coloneqq\prod_i\sigma_i^{a_i}$ with $h\coloneqq\prod_i\sigma_j^{a_j}$ and $0\le a_i,b_i<n_i$, into a cocycle in $Z^2( G,A)$.
\end{theorem}
\begin{proof}
	Note that $c$ is now surely well-defined because the elements $g$ and $h$ have unique representations as described. Anyway, we relegate the direct cocycle check to \autoref{sec:verifycocycle} because it is long, annoying, and unenlightening. We will also present an alternative proof in \autoref{sec:tuplestudy}, using more abstract theory.
\end{proof}
Observe that the above construction has now completely forgotten about $\mc E$! Namely, we have managed to go from tuples straight to cocycles; this is theoretically good because it will allow us to go fully in reverse: we will be able to start with a tuple, build the corresponding cocycle, from which the extension arises. However, equivalence classes of cocycles give the ``same'' extension, so we will also need to give equivalence classes for tuples as well.

\subsection{Building Tuples}
We continue in the modified set-up of the previous section. There is already an established way to get from a cocycle to an extension, which means that it should be possible to go straight from the cocycle to a $\{\sigma_i\}_{i=1}^m$-tuple. Again, it will be beneficial to write this out.
\begin{lemma} \label{lem:explicitalphabeta}
	Fix everything as in the modified set-up, but suppose that $\mc E=\mc E_c$ is the extension generated from a cocycle $c\in Z^2( G,A)$. Then, if $F_i=(x_i,\sigma_i)$ are our lifts, we have
	\[\alpha_i=N_i(x_i)\cdot\prod_{k=0}^{n_i-1}c\left(\sigma_i^k,\sigma_i\right)\qquad\text{and}\qquad\beta_{ij}=\frac{x_i}{\sigma_j(x_i)}\cdot\frac{\sigma_i(x_j)}{x_j}\cdot\frac{c(\sigma_i,\sigma_j)}{c(\sigma_j,\sigma_i)}\]
	for each $\alpha_i$ and $\beta_{ij}$.
\end{lemma}
\begin{proof}
	The equality for the $\alpha_i$ follow from \autoref{lem:explicitalpha}. For the equality about $\beta_{ij}$, we simply compute
	by brute force, writing
    \begin{align*}
        F_iF_j &= (x_i\cdot\sigma_ix_j\cdot c(\sigma_i,\sigma_j),\sigma_i\sigma_j) \\
        F_jF_i &= (x_j\cdot\sigma_jx_i\cdot c(\sigma_j,\sigma_i),\sigma_j\sigma_i) \\
        (F_jF_i)^{-1} &= \left((\sigma_j\sigma_i)^{-1}(x_j\cdot\sigma_jx_i\cdot c(\sigma_j,\sigma_i))^{-1},\sigma_i^{-1}\sigma_j^{-1}\right),
    \end{align*}
    which gives
    \begin{align*}
        \beta_{ij} &= (F_iF_j)(F_jF_i)^{-1} \\
        &= \left(\frac{x_i}{\sigma_jx_i}\cdot\frac{\sigma_ix_j}{x_j}\cdot\frac{c(\sigma_i,\sigma_j)}{c(\sigma_j,\sigma_i)},1\right),
    \end{align*}
	finishing.
\end{proof}
Here is a nice sanity check that we are doing things in the right setting: not only can we build tuples from extensions, but we can find an extension corresponding to any tuple.
\begin{cor} \label{cor:alltuplesfromextens}
	Fix everything as in the modified set-up, forgetting about the extension $\mc E$. For any $\{\sigma_i\}_{i=1}^m$-tuple of $\{\alpha_i\}_{i=1}^m$ and $\{\beta_{ij}\}_{i>j}$, there exists an extension $\mc E$ and lifts $F_i$ of the $\sigma_i$ so that
	\[\alpha_i=F_i^{n_i}\qquad\text{and}\qquad\beta_{ij}=F_iF_jF_i^{-1}F_j^{-1}.\]
\end{cor}
\begin{proof}
	From \autoref{thm:getcocycle}, we may build the cocycle $c\in Z^2( G,A)$ defined by
	\begin{equation}
		c(g,h)\coloneqq\left[\prod_{1\le j<i\le m}\Bigg(\prod_{1\le k<j}\sigma_k^{a_k+b_k}\Bigg)\Bigg(\prod_{j\le k<i}\sigma_k^{a_k}\Bigg)\beta_{ij}^{(a_ib_j)}\right]\left[\prod_{i=1}^m\Bigg(\prod_{1\le k<i}\sigma_k^{a_k+b_k}\Bigg)\alpha_i^{\floor{\frac{a_i+b_i}{n_i}}}\right], \label{eq:uglycocycle}
	\end{equation}
	where $g\coloneqq\prod_iF_i^{a_i}$ and $h\coloneqq\prod_iF_j^{a_j}$ and $0\le a_i,b_i<n_i$. As such, we use $\mc E\coloneqq\mc E_c$ to be the corresponding extension and $F_i\coloneqq(1,\sigma_i)$ as our lifts. We have the following checks.
	\begin{itemize}
		\item To show $\alpha_i=F_i^{n_i}$, we use \autoref{lem:explicitalphabeta} to compute $F_i^{n_i}$, which means we want to compute
		\[\prod_{k=0}^{n_i-1}c\left(\sigma_i^k,\sigma_i\right).\]
		Well, plugging $c\left(\sigma_i^k,\sigma_i\right)$ into \autoref{eq:uglycocycle}, we note that all $\beta_{k\ell}^{(a_kb_\ell)}$ terms vanish (either $a_k=0$ or $b_\ell=0$ for each $k\ne\ell$), so the big left product completely vanishes.
		
		As for the right product, the only term we have to worry about is
		\[\Bigg(\prod_{1\le k<i}\sigma_k^{0+0}\Bigg)\alpha_i^{\floor{\frac{k+1}{n_i}}},\]
		which is equal to $1$ when $k\le n_i-1$ and $\alpha_i$ when $k=n_i-1$. As such, we do indeed have $\alpha_i=F_i^{n_i}$.
		\item To show $\beta_{ij}=F_iF_jF_i^{-1}F_j^{-1}$ for $i>j$, we again use \autoref{lem:explicitalphabeta} to compute $F_iF_jF_i^{-1}F_j^{-1}$, which means we want to compute
		\[\frac{c(\sigma_i,\sigma_j)}{c(\sigma_j,\sigma_i)}.\]
		Plugging into \autoref{eq:uglycocycle} once more, there is no way to make $\floor{(a_k+b_k)/n_k}$ nonzero (recall we set $n_k>1$ for each $k$) in either $c(\sigma_i,\sigma_j)$ or $c(\sigma_j,\sigma_i)$. As such, the right-hand product term disappears.

		As for the left product, we note that it still vanishes for $c(\sigma_j,\sigma_i)$ because $i>j$ implies that either $a_k=0$ or $b_\ell=0$ for each $k>\ell$. However, for $c(\sigma_i,\sigma_j)$, we do have $a_i=1$ and $b_j=1$ only, so we have to deal with exactly the term
		\[\Bigg(\prod_{1\le k<j}\sigma_k^{a_k+b_k}\Bigg)\Bigg(\prod_{j\le k<i}\sigma_k^{a_k}\Bigg)\beta_{ij}.\]
		With $i>j$ and $a_k=b_k=0$ for $k\notin\{i,j\}$, we see that the product of all the $\sigma_k$s will disappear, indeed only leaving us with $\beta_{ij}$.
	\end{itemize}
	The above computations complete the proof.
\end{proof}
And here is our first taste of (partial) classification.
\begin{cor} \label{cor:cocycletuplesection}
	Fix everything as in the modified set-up, forgetting about the extension $\mc E$. Then the formula of \autoref{thm:getcocycle} and the formulae of \autoref{lem:explicitalphabeta} (setting $x_i=1$ for each $i$) are homomorphisms of abelian groups between tuples in $\mathcal T(G,A)$ and cocycles in $Z^2( G,A)$. In fact, the formula of \autoref{thm:getcocycle} is a section of the formulae of \autoref{lem:explicitalphabeta}.
\end{cor}
\begin{proof}
	The formulae in \autoref{thm:getcocycle} and \autoref{lem:explicitalphabeta} are both large products in their inputs, so they are multiplicative (i.e., homomorphisms). It remains to check that we have a section. Well, starting with a $\{\sigma_i\}_{i=1}^m$-tuple and building the corresponding cocycle $c$ by \autoref{thm:getcocycle}, the proof of \autoref{cor:alltuplesfromextens} shows that the formulae of \autoref{lem:explicitalphabeta} recovers the correct $\{\sigma_i\}_{i=1}^m$-tuple.
\end{proof}

\subsection{Equivalence Classes of Tuples}
We continue in the modified set-up. We would like to make \autoref{cor:cocycletuplesection} into a proper isomorphism of abelian groups, but this is not feasible; for example, the cocycle $c$ generated by \autoref{thm:getcocycle} will always have $c(\sigma_j,\sigma_i)=1$ for $i>j$, which is not true of all cocycles in $Z^2( G,A)$.

However, we did have a notion that the data of a $\{\sigma_i\}_{i=1}^m$ should be enough to specify the group law of the extension that the tuple comes from, so we do expect to be able to define all extensions---and hence achieve all cohomology classes---from a specially chosen $\{\sigma_i\}_{i=1}^m$-tuple.

To make this precise, we want to define an equivalence relation on tuples which go to the same cohomology class and then show that the map \autoref{thm:getcocycle} is surjective on these equivalence classes. The correct equivalence relation is taken from \autoref{lem:explicitalphabeta}.
\begin{definition}
	Fix everything as in the modified set-up. We say that two $\{\sigma_i\}_{i=1}^m$-tuples $(\{\alpha_i\},\{\beta_{ij}\})$ and $(\{\alpha_i'\},\{\beta_{ij}'\})$ are \textit{equivalent} if and only if there exist elements $x_1,\ldots,x_m\in A$ such that
	\[\alpha_i=N_i(x_i)\cdot\alpha_i'\qquad\text{and}\qquad\beta_{ij}=\frac{x_i}{\sigma_j(x_i)}\cdot\frac{\sigma_i(x_j)}{x_j}\cdot\beta_{ij}'\]
	for each $\alpha_i$ and $\beta_{ij}$. We may notate this by $(\{\alpha_i\},\{\beta_{ij}\})\sim(\{\alpha_i'\},\{\beta_{ij}'\})$.
\end{definition}
\begin{remark}
	It is not too hard to see directly from the definition that this is in fact an equivalence relation. In fact, the set of tuples equivalent to the ``trivial'' tuple of all $1$s is closed under multiplication (and inversion) and hence forms a subgroup of $\mathcal T(G,A)$. As such, the set of equivalence classes forms a quotient group of $\mathcal T(G,A)$. We will denote this quotient group by $\overline{\mathcal T}(G,A)$.
\end{remark}
This notion of equivalence can be seen to be the correct one in the sense that it correctly generalizes \autoref{prop:findallalpha}.
\begin{proposition} \label{prop:extenmakesaclass}
	Fix everything as in the modified set-up with an extension $\mc E$. As the lifts $F_i$ change, the corresponding values of
	\[\alpha_i\coloneqq F_i^{n_i}\qquad\text{and}\qquad\beta_{ij}\coloneqq F_iF_jF_i^{-1}F_j^{-1}\]
	go through a full equivalence class of $\{\sigma_i\}_{i=1}^m$-tuples.
\end{proposition}
\begin{proof}
	We proceed as in \autoref{prop:findallalpha}. Given an extension $\mc E'$, let $S_{\mc E'}$ be the set of $\{\sigma_i\}_{i=1}^m$-tuples generated as the lifts $F_i$ change. We start by showing that an isomorphism $\varphi\colon\mc E\simeq\mc E'$ of extensions implies that $S_{\mc E}=S_{\mc E'}$; by symmetry, it will be enough for $S_{\mc E}\subseteq S_{\mc E'}$. The isomorphism induces the following diagram.
	% https://q.uiver.app/?q=WzAsMTAsWzAsMCwiMSJdLFsxLDAsIkxeXFx0aW1lcyJdLFsyLDAsIlxcbWMgRSJdLFszLDAsIlxcR2FtbWEiXSxbNCwwLCIxIl0sWzAsMSwiMSJdLFsxLDEsIkxeXFx0aW1lcyJdLFsyLDEsIlxcbWMgRSciXSxbMywxLCJcXEdhbW1hIl0sWzQsMSwiMSJdLFswLDFdLFsxLDJdLFsyLDMsIlxccGkiXSxbMyw0XSxbNSw2XSxbNiw3XSxbNyw4LCJcXHBpJyJdLFs4LDldLFsyLDcsIlxcdmFycGhpIl0sWzEsNiwiIiwxLHsibGV2ZWwiOjIsInN0eWxlIjp7ImhlYWQiOnsibmFtZSI6Im5vbmUifX19XSxbMyw4LCIiLDEseyJsZXZlbCI6Miwic3R5bGUiOnsiaGVhZCI6eyJuYW1lIjoibm9uZSJ9fX1dXQ==&macro_url=https%3A%2F%2Fraw.githubusercontent.com%2FdFoiler%2Fnotes%2Fmaster%2Fnir.tex
	\[\begin{tikzcd}
		1 & {A} & {\mc E} &  G & 1 \\
		1 & {A} & {\mc E'} &  G & 1
		\arrow[from=1-1, to=1-2]
		\arrow[from=1-2, to=1-3]
		\arrow["\pi", from=1-3, to=1-4]
		\arrow[from=1-4, to=1-5]
		\arrow[from=2-1, to=2-2]
		\arrow[from=2-2, to=2-3]
		\arrow["{\pi'}", from=2-3, to=2-4]
		\arrow[from=2-4, to=2-5]
		\arrow["\varphi", from=1-3, to=2-3]
		\arrow[Rightarrow, no head, from=1-2, to=2-2]
		\arrow[Rightarrow, no head, from=1-4, to=2-4]
	\end{tikzcd}\]
	To show that $S_{\mc E}\subseteq S_{\mc E'}$, pick up some $\{\sigma_i\}_{i=1}^m$-tuple $(\{\alpha_i\},\{\beta_{ij}\})$ generated from lifts $F_i\in\mc E$ (i.e., $\pi(F_i)=\sigma_i$), where
	\[\alpha_i\coloneqq F_i^{n_i}\qquad\text{and}\qquad\beta_{ij}\coloneqq F_iF_jF_i^{-1}F_j^{-1}.\]
	Now, we note that $F_i'\coloneqq\varphi(F_i)$ will have
	\[\pi(F_i')=\pi(\varphi(F_i))=\varphi(\pi(F_i))=\sigma_i\]
	by the commutativity of the diagram, so the $F_i'$ are lifts of the $\sigma_i$. Further, we see that
	\[(F_i')^{n_i}=\varphi(F_i)^{n_i}=\varphi\left(F_i^{n_i}\right)=\varphi(\alpha_i)=\alpha_i\]
	for each $i$, and
	\[F_i'F_j'(F_i')^{-1}(F_j')^{-1}=\varphi\left(F_iF_jF_i^{-1}F_j^{-1}\right)=\varphi(\beta_{ij})=\beta_{ij}\]
	for each $i>j$. Thus, $(\{\alpha_i\},\{\beta_{ij}\})$ is a $\{\sigma_i\}_{i=1}^m$-tuple generated by lifts from $\mc E'$, implying that $(\{\alpha_i\},\{\beta_{ij}\})\in S_{\mc E'}$.

	It now suffices to show the statement in the proposition for a specific extension isomorphic to $\mc E$. Well, the isomorphism class of $\mc E$ corresponds to some cohomology class in $H^2( G,A)$, for which we let $c$ be a representative; then $\mc E\simeq\mc E_c$, so we may show the statement for $\mc E\coloneqq\mc E_c$. Indeed, as the lifts $F_i=(x_i,\sigma_i)$ change, we know by \autoref{lem:explicitalphabeta} that
	\[\alpha_i=N_i(x_i)\cdot\prod_{k=0}^{n_i-1}c\left(\sigma_i^k,\sigma_i\right)\qquad\text{and}\qquad\beta_{ij}=\frac{x_i}{\sigma_j(x_i)}\cdot\frac{\sigma_i(x_j)}{x_j}\cdot\frac{c(\sigma_i,\sigma_j)}{c(\sigma_j,\sigma_i)}\]
	for each $\alpha_i$ and $\beta_{ij}$. All of these live in the same equivalence class by definition of the equivalence, and as the $x_i$ are allowed to vary over all of $A$, they will fill up that equivalence class fully. This finishes.
\end{proof}
We are now ready to upgrade our section.
\begin{cor} \label{cor:cohomologymakesaclass}
	Fix everything as in the modified set-up, forgetting about the extension $\mc E$. Fixing a cohomology class $[c]\in H^2( G,A)$, the set of $\{\sigma_i\}_{i=1}^m$-tuples which correspond to $[c]$ (via \autoref{thm:getcocycle}) forms exactly one equivalence class.
\end{cor}
\begin{proof}
	We show that two tuples are equivalent if and only if their corresponding cocycles (via \autoref{thm:getcocycle}) to the same cohomology class, which will be enough.
	
	In one direction, suppose $(\{\alpha_i\},\{\beta_{ij}\})\sim(\{\alpha_i'\},\{\beta_{ij}'\})$. By \autoref{cor:alltuplesfromextens}, we can find an extension $\mc E$ which gives $(\{\alpha_i\},\{\beta_{ij}\})$ by choosing an appropriate set of lifts. By \autoref{prop:extenmakesaclass}, we see that $(\{\alpha_i'\},\{\beta_{ij}'\})$ must also come from choosing an appropriate set of lifts in $\mc E$. However, the cocycles in $Z^2( G,A)$ generated by \autoref{thm:getcocycle} from our two tuples now both represent the isomorphism class of $\mc E$ by \autoref{prop:writedowncocycle}, so these cocycles belong to the same cohomology class.

	In the other direction, name the cocycles corresponding to $(\{\alpha_i\},\{\beta_{ij}\})$ and $(\{\alpha_i'\},\{\beta_{ij}'\})$ by $c$ and $c'$ respectively, and suppose $[c]=[c']$. Then $\mc E_c\simeq\mc E_{c'}$ as extensions, but we know by the proof of \autoref{cor:alltuplesfromextens} that $(\{\alpha_i\},\{\beta_{ij}\})$ comes from choosing lifts of $\mc E_c$ and similar for $(\{\alpha_i'\},\{\beta_{ij}'\})$. In particular, because $\mc E_c\simeq\mc E_{c'}$, we know that $(\{\alpha_i'\},\{\beta_{ij}'\})$ will also come from choosing some lifts in $\mc E_c$ (recall the proof of \autoref{prop:extenmakesaclass}), so $(\{\alpha_i\},\{\beta_{ij}\})\sim(\{\alpha_i'\},\{\beta_{ij}'\})$ follows.
\end{proof}
\begin{theorem} \label{thm:classisomorphism}
	The maps described in \autoref{cor:cocycletuplesection} descend to an isomorphism of abelian groups between the equivalence classes in $\overline{\mathcal T}(G,A)$ and cohomology classes in $H^2( G,A)$.
\end{theorem}
\begin{proof}
	The fact that the maps are well-defined (in both directions) and hence injective is \autoref{cor:cohomologymakesaclass}. The fact that we had a section from tuples to cocycles implies that the map from cocycles to tuples was also surjective. Thus, we have a bona fide isomorphism.
\end{proof}

\subsection{Classification of Extensions}
We remark that we are now able to classify all extensions up to isomorphism, in some sense. At a high level, an isomorphism class of extensions corresponds to a particular cohomology class in $H^2( G,A)$, so choosing a $\{\sigma_i\}_{i=1}^m$-tuple $(\{\alpha_i\},\{\beta_{ij}\})$ corresponding to this class, we can write out a representative of this cocycle by \autoref{thm:getcocycle}, properly corresponding to the original extension by \autoref{prop:writedowncocycle}.

In fact, the cocycle in \autoref{prop:writedowncocycle} is generated by the description of the group law in \autoref{prop:multiplytwoelements}, and the entire computation only needed to use the following relations, for the appropriate choice of lifts $F_i$.
\begin{listalph}
	\item $F_ix=\sigma_i(x)F_i$ for each $i$ and $x\in A$.
	\item $F_i^{n_i}=\alpha_i$ for each $i$.
	\item $F_iF_jF_i^{-1}F_j^{-1}=\beta_{ij}$ for each $i>j$; i.e., $F_iF_j=\beta_{ij}F_jF_i$.
\end{listalph}
As such, the above relations fully describe the extension because they also specify the cocycle, and we know that this cocycle is well-defined. We summarize this discussion into the following theorem.
\begin{theorem}
	Fix everything as in the modified set-up, forgetting about the extension $\mc E$. Given a $\{\sigma_i\}_{i=1}^m$-tuple $(\{\alpha_i\},\{\beta_{ij}\})$, define the group $\mc E(\{\alpha_i\},\{\beta_{ij}\})$ as being generated by $A$ and elements $\{F_i\}_{i=1}^n$ having the following relations.
	\begin{listalph}
		\item $F_ix=\sigma_i(x)F_i$ for each $i$ and $x\in A$.
		\item $F_i^{n_i}=\alpha_i$ for each $i$.
		\item $F_iF_j=\beta_{ij}F_jF_i$ for each $i>j$.
	\end{listalph}
	Then the natural embedding $A\into\mc E(\{\alpha_i\},\{\beta_{ij}\})$ and projection $\pi\colon\mc E(\{\alpha_i\},\{\beta_{ij}\})\onto G$ by $F_i\mapsto\sigma_i$ makes $\mc E(\{\alpha_i\},\{\beta_{ij}\})$ into an extension. In fact, all extensions are isomorphic to some $\mc E(\{\alpha_i\},\{\beta_{ij}\})$.
\end{theorem}
\begin{proof}
	This follows from the preceding discussion, though we will provide a few more words in this proof. The exactness of
	\[1\to A\to\mc E(\{\alpha_i\},\{\beta_{ij}\})\stackrel\pi\to G\to1\]
	follows quickly. Further, the action of conjugation of $\mc E$ on $A$ corresponds correctly to the $ G$-action by (a). So we do indeed have an extension.

	It remains to show that all extensions are isomorphic to one of this type. Well, note that \autoref{prop:multiplytwoelements} and \autoref{prop:writedowncocycle} use only the above relations to write down a cocycle representing the isomorphism class of $\mc E(\{\alpha_i\},\{\beta_{ij}\})$, and it is the cocycle corresponding to the $\{\sigma_i\}_{i=1}^m$-tuple $(\{\alpha_i\},\{\beta_{ij}\})$ itself as described in \autoref{thm:getcocycle}.

	However, we know that as the equivalence class of $(\{\alpha_i\},\{\beta_{ij}\})$ changes, we will hit all cohomology classes in $H^2( G,A)$ by \autoref{thm:classisomorphism}. Thus, because every extension is represented by some cohomology class, every extension will be isomorphic to some $\mc E(\{\alpha_i\},\{\beta_{ij}\})$. This completes the proof.
\end{proof}

\subsection{Change of Group}
We continue in the modified set-up, but we will no longer need access to an extension $\mc E$. In this subsection, we are interested in what happens to tuples when the cocycle operations of $\op{Inf}\colon H^2\left(G/H,A^H\right)\to H^2(G,A)$ and $\op{Res}\colon H^2(G,A)\to H^2(H,A)$ are applied, where $H\subseteq G$ is some subgroup.

In general, this is difficult because the structure of a subgroup $H\subseteq G$ might not be particularly amenable to forming a tuple from a tuple in $G$. More concretely, $H$ might have generators which look very different from those of $G$. However, it will be enough for our purposes to restrict our attention to the subgroups of the form
\[H=\langle\sigma_1^{t_1},\ldots,\sigma_m^{t_m}\rangle,\]
where the $\{t_i\}_{i=1}^m$ are some positive integers. With that said, here are our computations. We begin with inflation.
\begin{lemma} \label{lem:tupleinflation}
	Fix everything as in the modified set-up, forgetting about the extension $\mc E$. Further, let $H\coloneqq\langle\sigma_1^{t_1},\ldots,\sigma_m^{t_m}\rangle$ be a subgroup with $t_\bullet\mid n_\bullet$, and let $\overline\sigma_i$ be the image of $\sigma_i$ in $G/H$. Consider the inflation map $\op{Inf}\colon H^2\left(G/H,A^H\right)\to H^2(G,A)$.
	
	If the cocycle $\overline c\in Z^2\left(G/H,A^H\right)$ gives the $\left\{\overline\sigma_i\right\}_{i=1}^m$-tuple $(\{\overline\alpha_i\},\{\overline\beta_{ij}\})$ (by \autoref{cor:cocycletuplesection}), then the cocycle $\op{Inf}\overline c\in Z^2(G,A)$ gives the $\{\sigma_i\}_{i=1}^m$-tuple
	\[\op{Inf}(\{\overline\alpha_i\},\{\overline\beta_{ij}\})\coloneqq(\{\alpha_i\},\{\beta_{ij}\})=\left(\left\{\overline\alpha_i^{n_i/\gcd(t_i,n_i)}\right\},\{\overline\beta_{ij}\}\right).\]
	Notably, $\gcd(t_i,n_i)$ is the order of $\overline\sigma_i\in G/H$.
\end{lemma}
\begin{proof}
	The point is to use the explicit formulae for the $\alpha_i$ and $\beta_{ij}$ of \autoref{lem:explicitalphabeta}.
	
	More explicitly, the map of \autoref{cor:cocycletuplesection} tells us that we can compute the tuple for $\op{Inf}\overline c$ by using our explicit formulae for $\alpha_i$ and $\beta_{ij}$ on the $2$-cocycle $\op{Inf}\overline c\in Z^2(G,A)$. For some $\alpha_i$, the computation is
	\begin{align*}
		\alpha_i &= \prod_{k=0}^{n_i-1}(\op{Inf}c)\left(\sigma_i^k,\sigma_i\right) \\
		&= \prod_{k=0}^{n_i-1}\overline c\left(\overline\sigma_i^k,\overline\sigma_i\right) \\
		&= \Bigg(\prod_{k=0}^{\gcd(n_i,t_i)-1}\overline c\left(\overline\sigma_i^k,\overline\sigma_i\right)\Bigg)^{n_i/\gcd(n_i,t_i)}
	\end{align*}
	where the last equality is because $\overline\sigma_i^{\gcd(n_i,t_i)}=1$ in $G/H$. In fact, $\gcd(n_i,t_i)$ is the order of $\overline\sigma_i$, so the product is just $\overline\alpha_i$ by \autoref{lem:explicitalphabeta} and how we defined $\overline\alpha_i$. It follows
	\[\alpha_i=\overline\alpha_i^{n_i/\gcd(n_i,t_i)}.\]
	Continuing, for some $\beta_{ij}$, we have
	\begin{align*}
		\beta_{ij} &= \frac{(\op{Inf}\overline c)(\sigma_i,\sigma_j)}{(\op{Inf}\overline c)(\sigma_j,\sigma_i)} \\
		&= \frac{\overline c(\overline\sigma_i,\overline\sigma_j)}{\overline c(\overline\sigma_j,\overline\sigma_i)} \\
		&= \overline\beta_{ij},
	\end{align*}
	where the last equality is by how we defined $\overline\beta_{ij}$. These computations complete the proof.
\end{proof}
\begin{remark} \label{rem:tupleinflationcommutativediagram}
	We can also the statement of \autoref{lem:tupleinflation} as asserting that the diagram
	% https://q.uiver.app/?q=WzAsNCxbMCwwLCJcXG1hdGhjYWwgVFxcbGVmdChHL0gsQV5IXFxyaWdodCkiXSxbMSwwLCJcXG1hdGhjYWwgVChHLEEpIl0sWzAsMSwiWl4yXFxsZWZ0KEcvSCxBXkhcXHJpZ2h0KSJdLFsxLDEsIlpeMihHLEEpIl0sWzAsMSwiXFxvcHtJbmZ9Il0sWzAsMiwiIiwyLHsic3R5bGUiOnsiaGVhZCI6eyJuYW1lIjoiZXBpIn19fV0sWzEsMywiIiwwLHsic3R5bGUiOnsiaGVhZCI6eyJuYW1lIjoiZXBpIn19fV0sWzIsMywiXFxvcHtJbmZ9IiwyXV0=&macro_url=https%3A%2F%2Fraw.githubusercontent.com%2FdFoiler%2Fnotes%2Fmaster%2Fnir.tex
	\[\begin{tikzcd}
		{Z^2\left(G/H,A^H\right)} & {Z^2(G,A)} \\
		{\mathcal T\left(G/H,A^H\right)} & {\mathcal T(G,A)}
		\arrow["{\op{Inf}}", from=1-1, to=1-2]
		\arrow[two heads, from=1-1, to=2-1]
		\arrow[two heads, from=1-2, to=2-2]
		\arrow["{\op{Inf}}", from=2-1, to=2-2]
	\end{tikzcd}\]
	commutes, where the vertical morphisms are from \autoref{cor:cocycletuplesection}.
\end{remark}
\begin{remark} \label{rem:inflationclasses}
	In light of the fact that the cohomology class of some $\op{Inf}\overline c\in Z^2(G,A)$ is only defined up to the cohomology class of $\overline c\in Z^2\left(G/H,A^H\right)$, changing an input tuple $(\{\overline\alpha_i\},\{\overline\beta_{ij}\})\in\mathcal T\left(G/H,A^H\right)$ up to equivalence will not change the cohomology class of the associated cocycle in $\overline c\in Z^2\left(G/H,A^H\right)$ and hence will not change the cohomology class of $\op{Inf}\overline c$ nor the equivalence class of $\op{Inf}(\{\overline\alpha_i\},\{\overline\beta_{ij}\})\in\mathcal T\left(G,A\right)$. All this is to say that we have a well-defined map
	\[\op{Inf}\colon\overline{\mathcal T}\left(G/H,A^H\right)\to\overline{\mathcal T}(G,A)\]
	and commutative diagram
	% https://q.uiver.app/?q=WzAsNCxbMCwwLCJcXG92ZXJsaW5le1xcbWF0aGNhbCBUfVxcbGVmdChHL0gsQV5IXFxyaWdodCkiXSxbMSwwLCJcXG92ZXJsaW5le1xcbWF0aGNhbCBUfShHLEEpIl0sWzAsMSwiSF4yXFxsZWZ0KEcvSCxBXkhcXHJpZ2h0KSJdLFsxLDEsIkheMihHLEEpIl0sWzAsMSwiXFxvcHtJbmZ9Il0sWzAsMiwiIiwyLHsic3R5bGUiOnsidGFpbCI6eyJuYW1lIjoiaG9vayIsInNpZGUiOiJ0b3AifSwiaGVhZCI6eyJuYW1lIjoiZXBpIn19fV0sWzEsMywiIiwwLHsic3R5bGUiOnsidGFpbCI6eyJuYW1lIjoiaG9vayIsInNpZGUiOiJ0b3AifSwiaGVhZCI6eyJuYW1lIjoiZXBpIn19fV0sWzIsMywiXFxvcHtJbmZ9IiwyXV0=&macro_url=https%3A%2F%2Fraw.githubusercontent.com%2FdFoiler%2Fnotes%2Fmaster%2Fnir.tex
	\[\begin{tikzcd}
		{\overline{\mathcal T}\left(G/H,A^H\right)} & {\overline{\mathcal T}(G,A)} \\
		{H^2\left(G/H,A^H\right)} & {H^2(G,A)}
		\arrow["{\op{Inf}}", from=1-1, to=1-2]
		\arrow[hook, two heads, from=1-1, to=2-1]
		\arrow[hook, two heads, from=1-2, to=2-2]
		\arrow["{\op{Inf}}", from=2-1, to=2-2]
	\end{tikzcd}\]
	induced by modding out from \autoref{rem:tupleinflationcommutativediagram}.
\end{remark}
Restriction is similar.
\begin{lemma} \label{lem:restricttuple}
	Fix everything as in the modified set-up, forgetting about the extension $\mc E$. Further, let $H\coloneqq\langle\sigma_1^{t_1},\ldots,\sigma_m^{t_m}\rangle$ be a subgroup with $t_\bullet\mid n_\bullet$. Consider the inflation map $\op{Res}\colon H^2\left(G,A\right)\to H^2(H,A)$.
	
	If the cohomology class $[c]\in H^2\left(G,A\right)$ is represented by the $\left\{\sigma_i\right\}_{i=1}^m$-tuple $(\{\alpha_i\},\{\beta_{ij}\})$, then the cohomology class $[\op{Res}\overline c]$ is represented by the $\{\sigma_i\}_{i=1}^m$-tuple
	\[(\{\overline\alpha_i\},\{\overline\beta_{ij}\})=\left(\left\{\alpha_i^{1_{n_i\mid t_i}}\right\},\left\{\beta_{ij}^{(\gcd(t_i,n_i)1_{n_i\mid t_i},\,\gcd(t_j,n_j)1_{n_i\mid t_i})}\right\}\right).\]
\end{lemma}
\begin{proof}
	By replacing $t_i$ with $\gcd(t_i,n_i)$ (which does not affect $\langle\sigma_i^{t_i}\rangle$ and hence does not affect $H$), we may assume that $t_i=\gcd(t_i,n_i)$. As in the previous proof, we will simply define $c$ by \autoref{thm:getcocycle}, and we will use the formulae of \autoref{lem:explicitalphabeta} to retrieve the $\left\{\sigma_i^{t_i}\right\}$-tuple for $\op{Res}c$. Indeed, we compute
	\begin{align*}
		\overline\alpha_i &= \prod_{k=0}^{n_i/t_i-1}(\op{Res}c)\left(\sigma_i^{t_ik},\sigma_i^{t_i}\right) \\
		&= \prod_{k=0}^{n_i/t_i-1}c\left(\sigma_i^{t_ik},\sigma_i^{t_i}\right) \\
		&= \prod_{k=0}^{n_i/t_i-1}\alpha_i^{\floor{t_i(k+1)/n_i}},
	\end{align*}
	where in the last equality we have used the construction of $c$. Now, if $n_i\mid t_i$, then $n_i=t_i$, and the product is empty, and we get $1$; otherwise, the last term of the product $k=n_i/t_i-1$ is the only term which does not return $1$, and it returns $\alpha_i$. So this matches the claimed $\alpha_i^{1_{n_i\mid t_i}}$.

	Continuing, we compute
	\begin{align*}
		\overline\beta_{ij} &= \frac{(\op{Res}c)\left(\sigma_i^{t_i},\sigma_j^{t_j}\right)}{(\op{Res}c)\left(\sigma_j^{t_j},\sigma_i^{t_i}\right)} \\
		&= \frac{c\left(\sigma_i^{t_i},\sigma_j^{t_j}\right)}{c\left(\sigma_j^{t_j},\sigma_i^{t_i}\right)} \\
		&= c\left(\sigma_i^{t_i},\sigma_j^{t_j}\right),
	\end{align*}
	where in the last step we have used the construction of $c$. Now, if $n_i\mid t_i$ or $n_i\mid t_j$, then we are computing $c\left(1,\sigma_j^{t_j}\right)$ or $c\left(\sigma_i^{t_i},1\right)$, which are both $1$, as needed. Otherwise, $t_i<n_i$ and $t_j<n_j$, so
	\[\overline\beta_{ij}=\beta_{ij}^{(t_it_j)},\]
	which again is as claimed.
\end{proof}
Thankfully, we will really only care about inflation in the following discussion, but we will say that there are analogues of \autoref{rem:tupleinflationcommutativediagram} and \autoref{rem:inflationclasses}.

\subsection{Profinite Groups}
In this subsection, we will use our results on change of group to extend our results a little to allow profinite groups. As such, we will want to slightly modify our set-up; we will call the following set-up the ``profinite set-up.''

Let $\mathcal I$ be a poset category such that any pair of elements has an upper bound (i.e., a directed set), and let the functor $G_\bullet\colon\mathcal I\opp\to\op{FinAbGrp}$ be an inverse system of finite abelian groups. These will create a profinite group
\[G\coloneqq\limit_{i\in\mathcal I}G_i.\]
In order to be able to apply our theory, we will assume that $G$ is a finite direct sum of procyclic groups as
\[G\simeq\bigoplus_{k=1}^m\overline{\langle\sigma_k\rangle}\]
for some elements $\{\sigma_k\}_{k=1}^m\subseteq G$. Further, we will require that the kernel $N_i$ of the map $G\onto G_i$ to take the form
\[N_i\coloneqq\overline{\left\langle\sigma_1^{t_{i,1}},\ldots,\sigma_m^{t_{i,m}}\right\rangle}.\]
In short, our restriction on the $N_i$ will allow our inflation maps to be computable in the sense of \autoref{lem:tupleinflation}. We quickly remark that, because the topology on $G$ is the coarsest one making the projections $G\onto G_i$ continuous, the subsets $\{N_i\}_{i\in\mathcal I}$ give a fundamental system of open neighborhoods around the identity.
\begin{remark}
	Of course, one could also start with $G$ being a finite direct sum of procyclic groups and then define the $N_i$ and $G_i$ accordingly. We have chosen the above approach because in application one might only have access to select $G_i$s, and it is not obvious how to choose these from such a ``top-down'' approach.
\end{remark}
\begin{example}
	To show that we are still allowing interesting groups, we can set 
	\[G_{m,\nu}\coloneqq\op{Gal}\left(\QQ_p(\zeta_{p^m-1})\QQ_p(\zeta_{p^\nu})/\QQ_p\right)\simeq\op{Gal}\left(\QQ_p(\zeta_{p^m-1})/\QQ_p\right)\oplus\op{Gal}\left(\QQ_p(\zeta_{p^\nu})/\QQ_p\right),\]
	which becomes $G=\op{Gal}\left(\QQ_p^{\op{ab}}/\QQ_p\right)\simeq\widehat\ZZ\oplus\ZZ_p^\times$ upon taking the inverse limit. It is not very hard to check that the kernels are generated correctly; for example, when $p$ is odd, we have $\ZZ_p^\times\cong\ZZ/(p-1)\ZZ\oplus\ZZ_p$, and under our isomorphisms, we will have
	\[\op{Gal}\left(\QQ(\zeta_{p^\nu})/\QQ_p\right)\simeq\ZZ/(p-1)\ZZ\oplus\ZZ_p/p^{\nu-1}\ZZ_p,\]
	so the kernel of $G\onto G_{m,\nu}$ is $m\widehat\ZZ\oplus(\ZZ/(p-1)\ZZ)^{1_{\nu=0}}\oplus p^{\nu-1}\ZZ_p$.
	% let $K$ be a local field with $\op{char}K=0$, and set $G_{\pi,n,m}\coloneqq\op{Gal}(K_{\pi,n}K_m/K)$, where $\{K_{\pi,n}\}$ is an ascending chain of Lubin--Tate extension and $K_m$ is the unramified extension of degree $m$. Then
	% \begin{align*}
	% 	\colimit_{n,m}\op{Gal}(K_{\pi,n}K_m/K) &\cong \op{Gal}(K^{\mathrm{ab}}/K) \\
	% 	&\cong \op{Gal}(K^{\op{unr}}/K)\oplus\op{Gal}(K_\pi) \\
	% 	&\cong \overline{\langle\op{Frob}_K\rangle}\oplus\mathcal O_K^\times \\
	% 	&\cong \widehat\ZZ\oplus\FF_\mf p^\times\oplus(1+\mf p) \\
	% 	&\cong \widehat\ZZ\oplus\FF_\mf p^\times\oplus\ZZ/p^a\ZZ\oplus\ZZ_p^{[K:\QQ_p]}
	% \end{align*}
	% for some sufficiently large $a\in\NN$; here the last isomorphism is by the logarithm map, which exists because $\op{char}K=0$. (For details, see \cite{neukirch-alg-nt}, Proposition II.5.7.)
\end{example}
\begin{remark}
	I'm not sure if such an explicit construction can be extended to other local fields $K$ (say, via Lubin--Tate theory). Because $K^\times$ is not topologically finitely generated when $K$ is in positive characteristic (see for example \cite{neukirch-alg-nt}, Proposition~II.5.7) such a construction must do something subtle.
\end{remark}
Let $A$ be a discrete $G$-module. The main goal of this subsection is to be able to provide a notion of a ``compatible system'' of tuples from each individual $H^2(G_i,A)$ to be able to exactly describe an element of $H^2(G,A)$. To effect this, we have the following somewhat annoying checks.
% \begin{lemma}
% 	Fix everything as in the profinite set-up, and let $N\subseteq G$ be an open normal subgroup. If $\sigma\in G$ is such that $[\sigma]_N\in G/N$ has finite order $n_\sigma$, then there exists some $i\in\mathcal I$ such that the order of $[\sigma]_{N_i}\in G/N_i=G_i$ has order divisible by $n_\sigma$.
% \end{lemma}
% \begin{proof}
% 	We proceed in steps. Let $1$ denote the identity of $G$.
% 	\begin{enumerate}
% 		\item Suppose that $p\coloneqq n_\sigma$ is prime. We proceed by contraposition. Namely, suppose there is no $N_i$ such that $\sigma^p\in N_i$, and we will show that $[\sigma]_N\in G/N$ cannot have order $p$. We may assume that $\sigma^p\in N$, which means that we are actually interested in showing $\sigma\in N$.

% 		Well, we claim that $\sigma$ is a limit point of $\langle\sigma^p\rangle$. To see this, we need to show that any open neighborhood $U$ around $\sigma$ has nontrivial intersection with $\langle\sigma^p\rangle$. Indeed, $\sigma^{-1}U$ is an open set containing the identity, but because the $\{N_i\}_{i\in\mathcal I}$ form a fundamental system of open neighborhoods around the identity, we have $N_i\subseteq\sigma^{-1}U$ for some $i\in\mathcal I$. Thus, it suffices to show that 
% 		\[\sigma N_i\cap\langle\sigma^p\rangle\ne\emp.\]
% 		Now, the order of $\sigma N_i$ is not divisible by $p$, so $\langle\sigma^pN_i\rangle=\langle\sigma N_i\rangle$ and in particular $\langle\sigma^pN_i\rangle$ contains $\sigma N_i$. Concretely, let's say $\sigma^{pk}N_i=\sigma N_i$; then $\sigma^{pk}\in\sigma N_i\cap\langle\sigma^p\rangle$, finishing.

% 		In total, because $N$ is an open subgroup and hence closed, we see that $\sigma^p\in N$ must also contain $\langle\sigma^p\rangle$ and hence must contain the limit point $\sigma$. This finishes.

% 		\item Next suppose that $n_\sigma=p^\nu$ is a power of a prime. We proceed by induction on $\nu$. When $\nu=0$, there is nothing to say because the order of a group element is always divisible by $1$; when $\nu=1$, this is the previous step. Otherwise, when $\nu>1$, we note that $\left[\sigma^p\right]_N=[\sigma]_N^p$ has order $p^{\nu-1}$: certainly $[\sigma]_N^{p\cdot p^{\nu-1}}$ vanishes, so the order divides $p^{\nu-1}$, but no smaller of $p$ will do because this would make the order of $[\sigma]_N$ too small.

% 		Thus, by the inductive hypothesis, there exists some $i\in\mathcal I$ such that the order of $[\sigma^p]_{N_i}$ has order divisible by $p^{\nu-1}$. We claim $[\sigma]_{N_i}$ has order divisible by $p^\nu$. Indeed, if not, then there exists some $k$ with $p\nmid k$ such that
% 		\[[\sigma]_{N_i}^{p^{\nu-1}k}=[1]_{N_i},\]
% 		from which we conclude that the order of $[\sigma^p]$ divides $p^{\nu-2}k$, which is not divisible by $p^{\nu-1}$.

% 		\item To finish, we show a version of multiplicativity: if the order of $[\sigma]_{N_p}$ is divisible by $n_p$, and the order of $[\sigma]_{N_q}$ is divisible by $n_q$ with $\gcd(n_p,n_q)=1$, then there exists $r\in\mathcal I$ such that the order of $[\sigma]_{N_r}$ is divisible by $n_pn_q$.

% 		Indeed, because $\mathcal I$ is a directed set, there exists some $r\in\mathcal I$ with morphisms $i\to r$ and $j\to r$. These correspond to having morphisms $G_r\to G_i$ and $G_r\to G_j$, and the fact that these morphisms are well-defined requires $N_r\subseteq N_i,N_j$.

% 		Now, let's say that the order of $[\sigma]_{N_r}$ is $n_r$; we want to show $n_pn_q\mid n_r$. Because $\gcd(n_p,n_q)=1$, it suffices (by symmetry) to show $n_p\mid n_r$. Well, $\sigma^{n_r}\in N_r\subseteq N_p$, so
% 		\[[\sigma]_{N_p}^{n_r}=[1]_{N_p},\]
% 		so the order of $[\sigma]_{N_p}$ divides $n_r$. In particular, $n_p\mid n_r$.
% 	\end{enumerate}
% 	We now note that, for the general case of $n\in\NN$, we can prime-factor $n$ into coprime factors, use step 2 to create a list of $N_i$, one for each prime factor, and then use step 3 to glue them all together. This completes the proof.
% \end{proof}
% \begin{lemma}
% 	Fix everything as in the profinite set-up. Then, for any open normal subgroup $N\subseteq G$, there exists $i\in\mathcal I$ so that $N$ contains $N_i$.
% \end{lemma}
% \begin{proof}
% 	This follows directly from the fact that the collection $\{N_i\}_{i\in\mathcal I}$ is a fundamental system of open neighborhoods around the identity of $G$. In particular, $N$ contains the identity and is open.
% 	% Because $G$ is compact (it's profinite), we have $[G:N]<\infty$. In particular, for any $\sigma\in G$, we must have $\sigma^{[G:N]}\in N$.
% 	% Thus, it will roughly speaking be enough to show that, for any $\sigma_k$ and $t\in\NN$, we have $\sigma_k^t\in N_i$ for some $i\in\mathcal I$. We have two cases.
% 	% \begin{itemize}
% 	% 	\item Suppose that $\sigma_k\in G$ has infinite order. 
% 	% \end{itemize}
% \end{proof}
\begin{lemma} \label{lem:colimitfiltered}
	Suppose that $\mathcal P$ is a directed set, and let $\mathcal P'\subseteq\mathcal P$ be a subcategory such that any $x\in\mathcal P$ has some $x'\in\mathcal P'$ such that $x\le x'$. Then, given a functor $F\colon\mathcal P\to\mathcal C$, we have
	\[\colimit_\mathcal PF\simeq\colimit_{\mathcal P'}F,\]
	provided that both colimits exist.
\end{lemma}
\begin{proof}
	For concreteness, if $x\le y$ in $\mathcal P$, we will let $f_{yx}\colon x\to y$ be the corresponding morphism; in particular, $x\le y\le z$ has $f_{zx}=f_{zy}f_{yx}$. Now, for brevity, set
	\[X\coloneqq\colimit_\mathcal PF\qquad\text{and}\qquad X'\coloneqq\colimit_{\mathcal P'}F.\]
	By the Yoneda lemma, it suffices to fix some object $Y\in\mathcal C$ and show that $\op{Mor}_\mathcal C(X,Y)\simeq\op{Mor}_\mathcal C(X',Y)$. Well, morphsims $X\to Y$ are in (natural) bijection with cones under $F$ with nadir $Y$, and morphisms $X'\to Y$ are in (natural) bijection with cones under $F'\coloneqq F|_{\mathcal P'}$ with nadir $Y$.

	Thus, it suffices to give a natural bijection between cones under $F$ with nadir $Y$ and cones under $F'$ with nadir $Y$. Well, given a cone under $F$ with nadir $Y$, we can simply restrict it to $\mathcal P'$ to get a cone under $F'$. In the other direction, given a cone under $F'$ with nadir $Y$, we can build a cone under $F$ with nadir $Y$ as follows; let $\varphi_{x'}\colon F(x')\to Y$ for $x'\in\mathcal P'$ be the corresponding morphisms in our cone.
	
	For any $x\in\mathcal P$, find $x'\in\mathcal P'$ such that $x\le x'$. Then set
	\[\varphi_x\coloneqq\varphi_{x'}\circ f_{x'x}\]
	Note that $\varphi_x$ is in fact independent of our choice of $x'$: if $x\le x_1'$ and $x\le x_2'$, then because $\mathcal P$ is a directed set, we can find $y\in\mathcal P$ such that $x_1',x_2'\le y$ and then $y'\in\mathcal P'$ with $y\le y'$. Then
	\begin{align*}
		\varphi_{x_\bullet'}\circ f_{x_\bullet'x} &= \varphi_{y'}\circ f_{y'x_\bullet'}\circ f_{x_\bullet'x} \\
		&= \varphi_{y'}\circ f_{y'x}
	\end{align*}
	for $x_\bullet'\in\{x_1',x_2'\}$. Anyway, we can check that the morphisms $\varphi$ do assemble to a cone under $F'$: if $x\le y$ in $\mathcal P$, then find $y'\in\mathcal P$ with $x\le y\le y'$, and we compute
	\begin{align*}
		\varphi_y\circ f_{yx} &= \varphi_{y'}\circ f_{y'y}\circ f_{yx} \\
		&= \varphi_{y'}\circ f_{y'x} \\
		&= \varphi_x.
	\end{align*}
	Thus, we do have a natural, well-defined map sending cones under $F'$ with nadir $Y$ to cones under $F$ with nadir $Y$. It is not too hard to see that these maps are inverse to each other (for example, the cone under $F'$, extended to $F$, does indeed restrict back to $F'$ properly), which completes the proof.
\end{proof}
\begin{remark}
	One can remove the hypothesis that the colimits exist and use essentially the same proof.
\end{remark}
\begin{proposition} \label{prop:bettercohomlimit}
	Fix everything as in the profinite set-up. Then, given a discrete $G$-module $A$,
	\[H^2(G,A)\simeq\colimit_{i\in\mathcal I}H^2\left(G_i,A^{N_i}\right).\]
	Here, the morphisms between the collection of $H^2\left(G_i,A^{N_i}\right)$ are induced by inflation: if $i\to j$ in $\mathcal I$, then $G_j\to G_i$ in $\mathrm{FinAbGrp}$, giving an inflation map $\op{Inf}\colon H^2\left(G_i,A^{N_i}\right)\to H^2\left(G_j,A^{N_j}\right)$.
\end{proposition}
\begin{proof}
	Let $\mathcal N$ be the poset category of open normal subgroups of $G$, reverse ordered under inclusion; i.e., $N_1\subseteq N_2$ in $G$ induces a map $N_2\to N_1$. Then it is already known that
	\[H^2(G,A)\simeq\colimit_{N\in\mathcal N}H^2\left(G/N,A^N\right).\]
	On the other hand, observe that $i\le j$ in $\mathcal I$ induces $G_j\to G_i$, so $N_j\subseteq N_i$. In other words, $i\mapsto N_i$ will define a functor $\mathcal I\to\mathcal N$; functoriality follows because $\mathcal I$ and $\mathcal N$ are poset categories. Letting $\mathcal N'$ denote the image of $\mathcal I$ in $\mathcal N$, we see
	\[\colimit_{i\in\mathcal I}H^2\left(G_i,A^{N_i}\right)\simeq\colimit_{N\in\mathcal N'}H^2\left(G/N,A^N\right).\]
	Notably, the inflation maps $\op{Inf}\colon H^2\left(G_i,A^{N_i}\right)\to H^2\left(G_j,A^{N_j}\right)$ when $i\le j$ become the inflation maps $\op{Inf}\colon H^2\left(G/N,A^N\right)\to H^2\left(G/N',A^{N'}\right)$ when $N'\subseteq N$. So if we let $F\colon\mathcal N\to\op{AbGrp}$ be the functor taking $N$ to $H^2\left(G/N,A^N\right)$ (and $N\subseteq N'$ to the inflation map), we are trying to show
	\[\colimit_{\mathcal N}F=\colimit_{\mathcal N'}F.\]
	For this, we use \autoref{lem:colimitfiltered}. Indeed, for a given open normal subgroup $N\in\mathcal N$, we need to find some $N'\in\mathcal N'$ such that $N\le N'$, which means $N'\subseteq N$.
	
	However, the elements of $\mathcal N'$ are the collection $\{N_i\}_{i\in\mathcal I}$, which form a fundamental system of open neighborhoods around the identity. Thus, the fact that $N$ is an open set containing the identity implies there is some $N_i\in\mathcal N'$ such that $N_i\subseteq N$. This finishes the proof.
\end{proof}
Observe that the above proofs did not use the extra hypotheses on $G$ nor $N_i$ to be products of procyclic groups. We use these hypotheses now.
% \begin{definition}
% 	Fix everything as in the profinite set-up, and let $A$ be a discrete $G$-module. Then a \textit{compatible system of $\{\sigma_p\}_{p=1}^m$-tuples} is an indexed set
% 	\[\left(\{\alpha_{i,p}\},\{\beta_{i,pq}\}\right)_{i\in\mathcal I}\]
% 	such that $\left(\{\alpha_{i,p}\},\{\beta_{i,pq}\}\right)$ is a $\{\sigma_pN_i\}_{p=1}^m$-tuple (corresponding to a class in $H^2\left(G_i,A^{N_i}\right)$) and
% 	\[\op{Inf}\left(\{\alpha_{i,p}\},\{\beta_{i,pq}\}\right)\sim\left(\{\alpha_{j,p}\},\{\beta_{j,pq}\}\right)\]
% 	as tuples corresponding to $H^2\left(G_j,A^{N_j}\right)$, whenever $i\le j$ in $\mathcal I$. We also define the relation $\sim$ of equivalence between compatible systems if and only if they are pointwise equivalent.
% \end{definition}
% The precise definition above is one of technical convenience, as we will shortly see.
To work more concretely, we note that any $i\in\mathcal I$ has
\[G_i\simeq\frac G{N_i}\simeq\bigoplus_{p=1}^m\overline{\langle\sigma_p\rangle}/\overline{\langle\sigma_p^{t_{i,p}}\rangle}\simeq\bigoplus_{p=1}^m\langle\sigma_p\rangle/\langle\sigma_p^{t_{i,p}}\rangle\subseteq\bigoplus_{p=1}^m\ZZ/t_{i,p}\ZZ\]
is a finite abelian group generated by the elements $\sigma_pN_i$. As a warning, the order of $\sigma_pN_i$ might not be $t_{i,p}$, for example if $\sigma_p$ itself has some small finite order which $t_{i,p}$ is not properly capitalizing on. More concretely, $\ZZ_5/3\ZZ_5=0$.

Regardless, the main point is that, given a discrete $G$-module $A$, we can consider the $\{\sigma_pN_i\}_{p=1}^m$-tuples $\mathcal T\left(G_i,A^{N_i}\right)$. Now, as discussed above, $i\le j$ in $\mathcal I$ induces a quotient map $G_j\simeq G/N_j\onto G_i/N_i$. From this, we have the following coherence check.
\begin{lemma} \label{lem:tupleinflationcommutes}
	Fix everything as in the profinite set-up, and let $A$ be a discrete $G$-module. Then, given $i\le j\le k$ in $\mathcal I$, the diagram
	% https://q.uiver.app/?q=WzAsMyxbMCwwLCJcXG1hdGhjYWwgVFxcbGVmdChHX2ksQV9pXntOX2l9XFxyaWdodCkiXSxbMSwwLCJcXG1hdGhjYWwgVFxcbGVmdChHX2osQV9qXntOX2p9XFxyaWdodCkiXSxbMSwxLCJcXG1hdGhjYWwgVFxcbGVmdChHX2ssQV9rXntOX2t9XFxyaWdodCkiXSxbMCwxLCJcXG9we0luZn0iXSxbMSwyLCJcXG9we0luZn0iXSxbMCwyLCJcXG9we0luZn0iLDJdXQ==&macro_url=https%3A%2F%2Fraw.githubusercontent.com%2FdFoiler%2Fnotes%2Fmaster%2Fnir.tex
	\[\begin{tikzcd}
		{\mathcal T\left(G_i,A^{N_i}\right)} & {\mathcal T\left(G_j,A^{N_j}\right)} \\
		& {\mathcal T\left(G_k,A^{N_k}\right)}
		\arrow["{\op{Inf}}", from=1-1, to=1-2]
		\arrow["{\op{Inf}}", from=1-2, to=2-2]
		\arrow["{\op{Inf}}"', from=1-1, to=2-2]
	\end{tikzcd}\]
	commutes. Here, the $\op{Inf}$ maps are defined as in \autoref{lem:tupleinflation}.
\end{lemma}
\begin{proof}
	For each $i\in\mathcal I$, we let $n_{i,p}$ denote the order of $\sigma_pN_i\in G_i$. Using the definition of $\op{Inf}$ from \autoref{lem:tupleinflation}, we just pick up some $\{\sigma_pN_p\}_{p=1}^m$-tuple $(\{\alpha_p\},\{\beta_{pq}\})$-tuple in $\mathcal T\left(G_i,A^{N_i}\right)$ and track through the diagram as follows.
	% https://q.uiver.app/?q=WzAsNCxbMCwwLCIoXFx7XFxhbHBoYV9wXFx9LFxce1xcYmV0YV97cHF9XFx9KSJdLFsxLDAsIlxcbGVmdChcXGJpZ1xce1xcYWxwaGFfcF57bl97aixwfS9uX3tpLHB9fVxcYmlnXFx9LFxce1xcYmV0YV97cHF9XFx9XFxyaWdodCkiXSxbMSwxLCJcXGxlZnQoXFxiaWdcXHtcXGFscGhhX3BeeyhuX3tqLHB9L25fe2kscH0pKG5fe2sscH0vbl97aixwfSl9XFxiaWdcXH0sXFx7XFxiZXRhX3twcX1cXH1cXHJpZ2h0KSJdLFswLDEsIlxcbGVmdChcXGJpZ1xce1xcYWxwaGFfcF57bl97ayxwfS9uX3tpLHB9fVxcYmlnXFx9LFxce1xcYmV0YV97cHF9XFx9XFxyaWdodCkiXSxbMCwxLCJcXG9we0luZn0iXSxbMSwyLCJcXG9we0luZn0iXSxbMCwzLCJcXG9we0luZn0iLDJdLFszLDIsIiIsMix7ImxldmVsIjoyLCJzdHlsZSI6eyJoZWFkIjp7Im5hbWUiOiJub25lIn19fV1d&macro_url=https%3A%2F%2Fraw.githubusercontent.com%2FdFoiler%2Fnotes%2Fmaster%2Fnir.tex
	\[\begin{tikzcd}
		{(\{\alpha_p\},\{\beta_{pq}\})} & {\left(\big\{\alpha_p^{n_{j,p}/n_{i,p}}\big\},\{\beta_{pq}\}\right)} \\
		{\left(\big\{\alpha_p^{n_{k,p}/n_{i,p}}\big\},\{\beta_{pq}\}\right)} & {\left(\big\{\alpha_p^{(n_{j,p}/n_{i,p})(n_{k,p}/n_{j,p})}\big\},\{\beta_{pq}\}\right)}
		\arrow["{\op{Inf}}", from=1-1, to=1-2]
		\arrow["{\op{Inf}}", from=1-2, to=2-2]
		\arrow["{\op{Inf}}"', from=1-1, to=2-1]
		\arrow[Rightarrow, no head, from=2-1, to=2-2]
	\end{tikzcd}\]
	This completes the proof.
\end{proof}
% Now, if we let $n_{i,p}$ denote the actual order of $\sigma_{i,p}N_i\in G_i$, then we may compute the inflation map $\op{Inf}\colon H^2\left(G_i,A^{N_i}\right)\to H^2\left(G_j,A^{N_j}\right)$ by
% \[\op{Inf}\left(\{\alpha_{i,p}\},\{\beta_{i,pq}\}\right)=\left(\{\alpha_{i,p}^{n_{j,p}}\},\{\beta_{i,pq}\}\right),\]
% so we are asking for
% \[\left(\{\alpha_{i,p}^{n_{j,p}}\},\{\beta_{i,pq}\}\right)\sim\left(\{\alpha_{j,p}\},\{\beta_{j,pq}\}\right)\]
% in the coherence condition for a compatible tuple.
% \begin{remark}
% 	From the above description, we can see why we ``have'' to allow the equivalence relation into our notion of compatibility. For example, if one of the $G_i$ is the trivial group, and $A^G$ is trivial, then we would be requiring all the $\beta_{i,pq}$ elements to be trivial for all $i\in\mathcal I$. This is not good.
% \end{remark}
And here is the result.
\begin{theorem}
	Fix everything as in the profinite set-up, and let $A$ be a discrete $G$-module. Then the isomorphisms of \autoref{thm:classisomorphism} upgrade into an isomorphism
	\[H^2(G,A)\simeq\colimit_{i\in\mathcal I}\overline{\mathcal T}\left(G_i,A^{N_i}\right).\]
	Here the morphisms between the $\overline{\mathcal T}\left(G_i,A^{N_i}\right)$ are inflation maps of \autoref{lem:tupleinflation}.
\end{theorem}
\begin{proof}
	Note that the objects $\overline{\mathcal T}\left(G_i,A^{N_i}\right)$ do make a directed system over $\mathcal I$ because of the commutativity of \autoref{lem:tupleinflationcommutes}. Namely, the lemma checks that $\mathcal I\to\op{AbGrp}$ by $i\mapsto\overline{\mathcal T}\left(G_i,A^{N_i}\right)$ is actually functorial; technically we must also check that the maps $\overline{\mathcal T}\left(G_i,A^{N_i}\right)\to\overline{\mathcal T}\left(G_i,A^{N_i}\right)$ are the identity, but this follows from the definition.

	Now, by \autoref{prop:bettercohomlimit}, we have
	\[H^2(G,A)\simeq\colimit_{i\in\mathcal I}H^2\left(G_i,A^{N_i}\right),\]
	but now the natural isomorphism induced by \autoref{rem:inflationclasses} induces an isomorphism of direct limits
	\[\colimit_{i\in\mathcal I}H^2\left(G_i,A^{N_i}\right)\simeq\colimit_{i\in\mathcal I}\overline{\mathcal T}\left(G_i,A^{N_i}\right)\]
	given by the isomorphism of \autoref{thm:classisomorphism} acting pointwise. This completes the proof.
\end{proof}
Because there are reasonably explicit descriptions of direct limits of abelian groups, and we already have an explicit description of each $\overline{\mathcal T}\left(G_i,A^{N_i}\right)$ term in addition to a description of the inflation maps between them, we will be content with our sufficiently explicit description of $H^2(G,A)$. So we call it done here.

\section{Tuples as Encoding Modules} \label{sec:tuplestudy}
% !TEX root = ../abeliangerbs.tex

The story so far has been able to generalize the one-variable results from \autoref{sec:general} to results using all generators of an abelian group in \autoref{sec:abelian}. It remains to prove \autoref{thm:getcocycle}, which is the main goal of this section.

\subsection{Set-Up and Overview} \label{sec:overview}
The approach here will be to attempt to abstract our data away from the $ G$-module $A$ as much as possible. To set up our discussion, we continue with
\[G\simeq\bigoplus_{i=1}^mG_i,\]
where $G_i=\langle\sigma_i\rangle\subseteq G$ and $\sigma_i$ has order $n_k$. These variables allow us to define
\[T_i\coloneqq(\sigma_i-1)\qquad\text{and}\qquad N_i\coloneqq\sum_{p=0}^{n_i-1}\sigma_i^p\]
for each index $i$. In fact, it will be helpful to also have notation
\[\sigma^{(a)}\coloneqq\sum_{p=0}^{a-1}\sigma^p\]
for any $\sigma\in G$ and nonnegative integer $a\ge0$; in particular, $\sigma^{(0)}=0$ and $\sigma_i^{(n_i)}=N_i$. The main benefits to this notation will be the facts that
\[\sigma^{(a+b)}=\sigma^{(a)}+\sigma^a\sigma^{(b)}\qquad\text{and}\qquad\sigma_i^a=T_i\sigma_i^{(a)}+1,\]
which can be seen by direct expansion. Given $g\in\prod_{p=1}^n\sigma_p^{a_p}$, we will also define the notation
\[g_i\coloneqq\prod_{p=1}^{i-1}\sigma_p^{a_p}\]
for $i\ge0$. In particular $g_0=g_1=1$ and $g_{n+1}=g$.

Now, our tool in the proof of \autoref{thm:getcocycle} will be the magical map $\mathcal F\colon\ZZ[G]^m\times\ZZ[G]^{\binom m2}\to\ZZ[G]^m$ defined by
\[\mathcal F\colon\big((x_i)_{i=1}^m,(y_{ij})_{i>j}\big)\mapsto\Bigg(x_iN_i-\sum_{j=1}^{i-1}y_{ij}T_j+\sum_{j=i+1}^my_{ji}T_j\Bigg)_{i=1}^m.\]
This is of course a $G$-module homomorphism. We will go ahead and state the main results we will prove. Roughly speaking, $\mathcal F$ is manufactured to make the following result true.
\begin{prop} \label{prop:manufacturedcocycle}
	Fix everything as in the set-up. Then the function
	\[\overline c(g)\coloneqq\left(g_i\sigma_i^{(a_i)}\right)_{i=1}^m,\]
	where $g\coloneqq\prod_{i=1}^m\sigma_i^{a_i}$, is a $1$-cocycle in $Z^1(G,\coker\mathcal F)$.
\end{prop}
The reason we care about this cocycle is that we can pass it through a boundary morphism induced by the short exact sequence
\[0\to\underbrace{\frac{\ZZ[G]^m\times\ZZ[G]^{\binom m2}}{\ker\mathcal F}}_{X\coloneqq}\stackrel{\mathcal F}\to\ZZ[G]^m\to\coker\mathcal F\to0,\]
so we have a $2$-cocycle $\delta(\overline c)\in Z^2(G,X)$; in fact, we will be able to explicitly compute $\delta(\overline c)$ as a result of the proof of \autoref{prop:manufacturedcocycle}.

Only now will we bring in tuples. The first result provides an alternate description of tuples.
\begin{restatable}{prop}{propalternativetuple} \label{prop:alternativetuple}
	Fix everything as in the set-up, and now let $A$ be a $G$-module. Then $\{\sigma_i\}_{i=1}^m$-tuples are canonically isomorphic to $\op{Hom}_{\ZZ[G]}(X,A)=H^0(G,\op{Hom}_\ZZ(X,A))$.
\end{restatable}
\noindent The second result brings in the last ingredient, the cup product.
\begin{restatable}{theorem}{thmyesitisacocycle} \label{thm:yesitisacocycle}
	Fix everything as in the set-up. Further, fix a $G$-module $A$ and a $\{\sigma_i\}_{i=1}^m$-tuple $\left(\{\alpha_i\},\{\beta_{ij}\}\right)$. Then observe there is a natural cup product map
	\[\cup\colon H^2(G,X)\times H^0(G,\op{Hom}_\ZZ(X,A))\to H^2(G,A)\]
	by using the evaluation map $X\otimes_\ZZ\op{Hom}_\ZZ(X,A)\to A$. Then, using the isomorphism of \autoref{prop:alternativetuple}, the cocycle defined in \autoref{thm:getcocycle} is simply the output of $\delta(\overline c)\cup\left(\{\alpha_i\},\{\beta_{ij}\}\right)$ on cocycles.
\end{restatable}
\noindent Because we know that the cup product sends cocycles to cocycles, this will show that the cocycle of \autoref{thm:getcocycle} is in fact well-defined.

% it might be worth stating the main results we are going to prove here, but they are somewhat notation-heavy

\subsection{Preliminary Work}
We continue in the set-up of the previous subsection.
% The goal of this subsection is to prove \autoref{prop:manufacturedcocycle}. In fact, we will show the following stronger result.
% \begin{proposition} \label{prop:allmanufacturedcocycles}
% 	Fix everything as in the set-up. Then $H^1(G,\coker\mathcal F)$ is cyclic generated by the class $[\overline c]$ represented by $\overline c$, where
% 	\[\overline c(g)\coloneqq\left(g_i\sigma_i^{(a_i)}\right)_{i=1}^m,\]
% 	with $g\coloneqq\prod_{i=1}^m\sigma_i^{a_i}$
% \end{proposition}
Before jumping into any hard logic, we define some (more) notation which will be useful later on as well. First, in $\ZZ[G]^m\times\ZZ[G]^{\binom m2}$, we define
\[\kappa_p\coloneqq\big((1_{i=p})_i,(0)_{i>j}\big)\in X\qquad\text{and}\qquad\lambda_{pq}\coloneqq\big((0)_i,(1_{(i,j)=(p,q)})_{i>j}\big)\]
for all relevant indices $p$ and $q$ so that the $\kappa_p$ and $\lambda_{pq}$ are a basis for $\ZZ[G]^m\times\ZZ[G]^{\binom m2}$ as a $\ZZ[G]$-module. Secondly, we define
\[\varepsilon_p\coloneqq(1_{i=p})_{i=1}^m\]
for all indices $p$, again giving a basis for $\ZZ[G]^m$ as a $\ZZ[G]$-module. For example, this notation lets us write
\begin{equation}
	\mathcal F\left(\sum_{i=1}^mx_i\kappa_i+\sum_{i>j}y_{ij}\lambda_{ij}\right)=\sum_{i=1}^mx_iN_i\varepsilon_i+\sum_{i>j}y_{ij}(T_i\varepsilon_j-T_j\varepsilon_i), \label{eq:betterf}
\end{equation}
and
\[\overline c(g)=\sum_{i=1}^mg_i\sigma_i^{(a_i)}\varepsilon_i\]
where $g\coloneqq\prod_{i=1}^m\sigma_i^{a_i}$.

Additionally, so that we do not need to interrupt our discussion later, we establish a few lemmas which will aide our proof of \autoref{prop:manufacturedcocycle}.
\begin{lemma} \label{lem:separatenijs}
	Fix everything as in the set-up. Then, for any set of distinct indices $(i_1,\ldots,i_k)$, we have
	\[\bigcap_{p=1}^k\im N_{i_p}=\im\prod_{p=1}^kN_{i_p},\]
	where we are identifying $x\in\ZZ[G]$ with its associated multiplication map $x\colon\ZZ[G]\to\ZZ[G]$.
\end{lemma}
\begin{proof}
	The point is that the elements of $\bigcap_{p=1}^k\im N_{i_p}$ and $\im\prod_{p=1}^kN_{i_p}$ are both simply the elements whose expansion in the form $\sum_gc_gg\in\ZZ[G]$ have $c_j$ ``constant in $\sigma_p$ and $\sigma_q$.'' More explicitly, of course, $\prod_{p=1}^kN_{i_p}\in\bigcap_{p=1}^k\im N_{i_p}$, so
	\[\im\prod_{p=1}^kN_{i_p}\subseteq\bigcap_{p=1}^k\im N_{i_p}.\]
	In the other direction, suppose that we have some element
	\[z\coloneqq\sum_{(a_i)_i}c_{(a_i)_i}\sigma_1^{a_1}\cdots\sigma_m^{a_m}\in\bigcap_{p=1}^k\im N_{i_p},\]
	the sum is over sequences $(a_i)_{i=1}^m$ such that $0\le a_i<n_i$ for each index $i$. We will show $z\in\im\prod_{p=1}^kN_{i_p}$.
	
	Now, $z\in\im N_r$ for $r\in\{p,q\}$ is equivalent to $z\in\ker T_r$, but upon multiplying by $(\sigma_r-1)$ we see that we are asking for
	\[\sum_{(a_i)_i}c_{(a_i)_i}\sigma_1^{a_1}\cdots\sigma_{r-1}^{a_{r-1}}\sigma_r^{a_r}\sigma_{r+1}^{a_{r+1}}\cdots\sigma_n^{a_n}=\sum_{(a_i)_i}c_{(a_i)_i}\sigma_1^{a_1}\cdots\sigma_{r-1}^{a_{r-1}}\sigma_r^{a_r+1}\sigma_{r+1}^{a_{r+1}}\cdots\sigma_n^{a_n}.\]
	In other words, this is asking for $c_{(a_i)_i}=c_{(a_i)_i+(1_{i=r})_i}$, or more succinctly just that $c$ is constant in the $i=r$ coordinate.

	Thus, $c$ is constant in all the $i=i_p$ coordinates for each index $i_p$. Thus, we let $d_{(a_i)_{i\notin\{i_p\}}}$ be the restricted function equal to $c_{(a_i)_i}$ but forgetting the information input from any of the $a_{i_p}$. This allows us to write
	\begin{align*}
		z &= \sum_{(a_i)_i}c_{(a_i)_i}\sigma_1^{a_1}\cdots\sigma_m^{a_m} \\
		&= \sum_{(a_i)_{i\notin\{i_p\}}}\sum_{a_{i_1}=0}^{n_{i_1}-1}\cdots\sum_{a_{i_k}=0}^{n_{i_k}-1}d_{(a_i)_{i\notin\{i_p\}}}\sigma_1^{a_1}\cdots\sigma_m^{a_m} \\
		&= \Bigg(\sum_{(a_i)_{i\notin\{i_p\}}}d_{(a_i)_{i\notin\{i_p\}}}\prod_{\substack{i=0\\i\notin\{i_p\}}}^m\sigma_i^{a_i}\Bigg)\Bigg(\sum_{a_{i_1}=0}^{n_{i_1}-1}\sigma_{i_1}^{a_{i_1}}\Bigg)\cdots\Bigg(\sum_{a_{i_k}=0}^{n_{i_k}-1}\sigma_{i_k}^{a_{i_k}}\Bigg),
	\end{align*}
	which is now manifestly in $\im\prod_{p=1}^kN_{i_p}$.
\end{proof}
\begin{lemma} \label{lem:expandgi}
	Fix everything as in the set-up. Then, given $g\coloneqq\prod_{i=1}^m\sigma_i^{a_i}$, we have
	\[g_i=1+\sum_{p=1}^{i-1}g_p\sigma_p^{(a_p)}T_p\]
	for $i\ge1$.
\end{lemma}
\begin{proof}
	This is by induction. For $i=1$, there is nothing to say. For the inductive step, we take $i>1$ where we may assume the statement for $i-1$. Via some relabeling, we may make our inductive hypothesis assert
	\[\prod_{p=2}^{i-1}\sigma_p^{a_p}=1+\sum_{p=2}^{i-1}\Bigg(\prod_{q=2}^{p-1}\sigma_q^{a_q}\Bigg)\sigma_p^{(a_p)}T_p.\]
	In particular, multiplying through by $\sigma_1^{a_1}$ yields
	\begin{align*}
		g_i &= \sigma_1^{a_1}\cdot\prod_{p=2}^{i-1}\sigma_p^{a_p} \\
		&= \sigma_1^{a_1}+\sigma_1^{a_1}\sum_{p=2}^{i-1}\Bigg(\prod_{q=2}^{p-1}\sigma_q^{a_q}\Bigg)\sigma_p^{(a_p)}T_p \\
		&= \sigma_1^{a_1}+\sum_{p=2}^{i-1}g_p\sigma_p^{(a_p)}T_p \\
		&= 1+\sigma_1^{(a_1)}T_1+\sum_{p=2}^{i-1}g_p\sigma_p^{(a_p)}T_p,
	\end{align*}
	which is exactly what we wanted, after a little more rearrangement.
\end{proof}
And mostly because we can, we show that our main short exact sequence splits.
\begin{lemma} \label{lem:sessplits}
	Fix everything as in the set-up. Then consider $\ZZ$-module map $\rho\colon\ZZ[G]^m\to\ZZ[G]^m$ defined by
	\[\rho(g\varepsilon_i)\coloneqq g_i\big(\sigma_i^{a_i}-N_i1_{a_i=n_i-1}\big)\varepsilon_i+\sum_{j=i+1}^mg_j\sigma_j^{(a_j)}T_i\varepsilon_j,\]
	where $g\coloneqq\prod_{i=1}^m\sigma_i^{a_i}$ with $0\le a_i<n_i$. Then $\rho$ descends to a map $\overline\rho\colon\coker\mathcal F\to\ZZ[G]^m$ witnessing the splitting of the short exact sequence
	\[0\to X\to\ZZ[G]^m\to\coker\mathcal F\to0\]
	over $\ZZ$.
\end{lemma}
\begin{proof}
	Observe that we have a well-defined map $\rho\colon\ZZ[G]^m\to\ZZ[G]^m$ because $\ZZ[G]^m$ is a free abelian group generated by $g\varepsilon_i$ for $g\in G$ and indices $i$. It remains to show that $\im\mathcal F\subseteq\ker\rho$ to get a map $\overline\rho\colon\coker\mathcal F\to\ZZ[G]^m$ and then to show that $\rho(z)\equiv z\pmod{\im\mathcal F}$ to get the splitting. We show these individually.

	To show that $\im\mathcal F\subseteq\ker\rho$, we note from \autoref{eq:betterf} that $\im\mathcal F$ is generated over $\ZZ[G]$ by the elements $N_i\varepsilon_i$ and $T_i\varepsilon_j-T_j\varepsilon_i$ for relevant indices $i$ and $j$. Thus, $\im\mathcal F$ is generated over $\ZZ$ by the elements $gN_i\varepsilon_i$ and $gT_i\varepsilon_j-gT_j\varepsilon_i$ for relevant indices $i$ and $j$. Thus, we fix any $g\coloneqq\prod_{i=1}^n\sigma_i^{a_i}$ and show that $gN_i\varepsilon_i\in\ker\rho$ and $gT_i\varepsilon_j-gT_j\varepsilon_i\in\ker\rho$ for relevant indices $i$ and $j$.
	\begin{itemize}
		\item We show $gN_i\varepsilon_i\in\ker\rho$ for any $i$. Because $gN_i=g\sigma_iN_i$, we may as well as assume that $a_i=0$. Then
		\[\rho\left(g\sigma_i^a\varepsilon_i\right)=g_i\big(\sigma_i^{a}-N_i1_{a=n_i-1}\big)\varepsilon_i+\sum_{j=i+1}^mg_j\sigma_i^a\sigma_j^{(a_j)}T_i\varepsilon_j.\]
		As $a$ varies from $0$ to $n_i-1$, we note that the term $g_i\big(\sigma_i^{a}-N_i1_{a=n_i-1}\big)\varepsilon_i$ will only get the $-N_i$ contribution exactly once at $a=n_i-1$. Summing, we thus see that
		\[\rho(gN_i\varepsilon_i)=g_i\Bigg(-N_i+\sum_{a=0}^{n_i-1}\sigma_i^{a}\Bigg)\varepsilon_i+\sum_{a=0}^{n_i-1}\sum_{j=i+1}^mg_j\sigma_i^a\sigma_j^{(a_j)}T_i\varepsilon_j.\]
		The left term vanishes because $N_i=\sum_{a=0}^{n_i-1}\sigma_i^a$. Additionally, the right term vanishes because we can factor $T_i\sum_{a=0}^{n_i-1}\sigma_i^a=T_iN_i=0$. So $gN_i\varepsilon_i\in\ker\rho$.
		\item We show $gT_p\varepsilon_q-gT_q\varepsilon_p\in\ker\rho$ for any $p>q$. Equivalently, we will show that $\rho(g\sigma_p\varepsilon_q)-\rho(g\varepsilon_q)=\rho(g\sigma_q\varepsilon_p)-\rho(g\varepsilon_p)$. On one hand, note
		\begin{align*}
			\rho(g\sigma_p\varepsilon_q) &= g_q\big(\sigma_q^{a_q}-N_i1_{a_q=n_q-1}\big)\varepsilon_q \\
			&\qquad\qquad+\sum_{j=q+1}^{p-1}g_j\sigma_j^{(a_j)}T_q\varepsilon_j \\
			&\qquad\qquad+g_p\left(\sigma_p^{(a_p+1)}-N_p1_{a_p=n_p-1}\right)T_q\varepsilon_p \\
			&\qquad\qquad+\sum_{j=p+1}^m\sigma_pg_j\sigma_j^{(a_j)}T_q\varepsilon_j
		\end{align*}
		because $g_j$ doesn't ``see'' the extra $\sigma_p$ term until $j>p$. (For the $j=p$ term, we would like to write $\sigma_p^{(a_p+1)}$ above, but when $a_p=n_p-1$, we actually end up with $\sigma_p^{(0)}=0$ and hence have to subtract out $\sigma_p^{(n_p)}=N_p$.) Thus,
		\[\rho(g\sigma_p\varepsilon_q)-\rho(g\varepsilon_q) = g_p\left(\sigma_p^{a_p}-N_p1_{a_p=n_p-1}\right)T_q\varepsilon_p+\sum_{j=p+1}^mg_j\sigma_j^{(a_j)}T_pT_q\varepsilon_j.\]
		On the other hand, we have
		\[\rho(g\sigma_q\varepsilon_p) = \sigma_qg_p\big(\sigma_p^{a_p}-N_p1_{a_p=n_p-1}\big)\varepsilon_p+\sum_{j=p+1}^m\sigma_qg_j\sigma_j^{(a_j)}T_p\varepsilon_j\]
		where this time all $j>p$ also have $j>q$ and so $(\sigma_qg)_j=\sigma_qg_j$. Thus,
		\[\rho(g\sigma_q\varepsilon_p)-\rho(g\varepsilon_p) = g_p\left(\sigma_p^{a_p}-N_p1_{a_p=n_p-1}\right)T_q\varepsilon_p+\sum_{j=p+1}^mg_j\sigma_j^{(a_j)}T_pT_q\varepsilon_j,\]
		as desired.
	\end{itemize}
	We now check the splitting. For this, we simply need to check that $\rho(g\varepsilon_i)\equiv g\varepsilon_i\pmod{\im\mathcal F}$, and we will get the result for all elements of $\ZZ[G]^m$ by additivity of $\rho$. Well, using \autoref{lem:expandgi}, we write
	\begin{align*}
		g\varepsilon_i &= g_i\sigma_i^{a_i}\Bigg(\prod_{j=i+1}^m\sigma_j^{a_j}\Bigg)\varepsilon_i \\
		&= g_i\sigma_i^{a_i}\Bigg(1+\sum_{j=i+1}^m\Bigg(\prod_{q=i+1}^{j-1}\sigma_q^{a_q}\Bigg)\sigma_j^{(a_j)}T_j\Bigg)\varepsilon_i \\
		&= g_i\sigma_i^{a_i}\varepsilon_i+\sum_{j=i+1}^mg_i\sigma_i^{a_i}\Bigg(\prod_{q=i+1}^{j-1}\sigma_q^{a_q}\Bigg)\sigma_j^{(a_j)}T_j\varepsilon_i \\
		&\equiv g_i\sigma_i^{a_i}\varepsilon_i+\sum_{j=i+1}^mg_j\sigma_j^{(a_j)}T_i\varepsilon_j,
	\end{align*}
	where in the last step we have used the fact that $T_j\varepsilon_i\equiv T_j\varepsilon_i\pmod{\im\mathcal F}$. Lastly, we note that $hN_i\varepsilon_i\equiv h\varepsilon_i\pmod{\im\mathcal F}$ for any $h\in G$, so in fact
	\[g\varepsilon_i\equiv g_i\left(\sigma_i^{a_i}-N_i1_{a_i=n_i-1}\right)\varepsilon_i+\sum_{j=i+1}^mg_j\sigma_j^{(a_j)}T_i\varepsilon_j,\]
	and now the right-hand side is $\rho(g\varepsilon_i)$.
\end{proof}
% \begin{remark}
% 	The purpose of \autoref{lem:sessplits} is to give an injective map from $\coker\mathcal F$ to a more controlled setting. In particular, it is somewhat annoying to check if an element $z\in\ZZ[G]^m$ lives in $\im\mathcal F$, but it is easier to check the equivalent condition $\overline\rho(z)=0$.
% \end{remark}
% We are now ready to more directly attack the proof of \autoref{prop:allmanufacturedcocycles}. We begin by reducing the amount of data we have to carry around in a cocycle.
% \begin{lemma} \label{lem:compresscocycle}
% 	Fix everything as in the set-up, and let $A$ be a $G$-module. Then, if $f\in Z^1(G,A)$ is a cocycle, then
% 	\[f(g)=\sum_{i=1}^mg_i\sigma_i^{(a_i)}f(\sigma_i),\]
% 	where $g\coloneqq\prod_{i=1}^m\sigma_i^{a_i}$ with $a_i\ge0$.
% \end{lemma}
% \begin{proof}
% 	Unsurprisingly, this is by induction. To begin, we claim that
% 	\[f\left(\sigma^a\right)=\sigma^{(a)}f(\sigma)\]
% 	by induction on $a$. When $a=0$, we are showing that $f(1)=0$, for which we note that the $1$-cocycle condition implies $f(1)=f(1)+f(1)$ and so $f(1)=0$. Then for the inductive step, we assume $f(\sigma^a)=\sigma^{(a)}f(\sigma)$ and note
% 	\[f\left(\sigma^{a+1}\right)=\sigma f\left(\sigma^a\right)+f(\sigma)=\left(1+\sigma\sigma^{(a)}\right)f(\sigma)=\sigma^{(a+1)}f(\sigma),\]
% 	finishing.

% 	We now show the original statement by an induction on $m$. For $m=0$, this is asserting $f(1)=0$, which is true. Then for the inductive step, we assume for $m-1$ and note that $m>1$ has
% 	\[f\left(g_m\sigma_m^{a_m}\right)=f(g_m)+g_mf\left(\sigma_m^{a_m}\right)=\sum_{i=1}^{m-1}g_i\sigma_i^{(a_i)}f(\sigma_i)+g_m\sigma_m^{(a_m)}f(\sigma_m),\]
% 	which is what we wanted.
% \end{proof}
% Thus, to build a $1$-cocycle, we only have to specify $f(\sigma_i)$ for indices $i$ and then check the $1$-cocycle condition to make sure we are okay.

% As such, we now run through what the $1$-cocycle check requires.
% \begin{lemma} \label{lem:cocycleforcecoord}
% 	Fix everything as in the set-up. Further, fix some $z\in\ZZ[G]^m$. Then $N_iz\in\im\mathcal F$ if and only if $[z]\in\coker\mathcal F$ has a representative of the form $a_i\varepsilon_i\in\ZZ[G]^m$ where $a_i\in\ZZ[G]$.
% \end{lemma}
% \begin{proof}
% 	In one direction, if $z\equiv a_i\varepsilon_i\pmod{\im\mathcal F}$, then
% 	\[N_iz\equiv a_i\cdot N_i\varepsilon_i\equiv a_i\cdot0\equiv0\pmod{\im\mathcal F}\]
% 	because $N_i\varepsilon_i\in\im\mathcal F$.

% 	In the other direction, we pass through $\overline\rho$ of \autoref{lem:sessplits}. By possibly rearranging our $\sigma$s, we may set $i=1$. As such, suppose $N_1z\in\im\mathcal F$, and write
% 	\[z\coloneqq\sum_{i=1}^mz_i\varepsilon_i\]
% 	where $z_i\in\ZZ[G]$. By using the fact that $T_i\varepsilon_1\equiv T_1\varepsilon_i\pmod{\im\mathcal F}$ for any index $i$, we can find a representative for $z$ in $\ZZ[G]^m$ such that $z_i$ has no $\sigma_1$ powers for each $i>1$; without loss of generality, replace $z$ with this representative.
	
% 	We thus claim that $w\coloneqq z-z_1\varepsilon_1\in\im\mathcal F$, which means that $z$ is represented by $z_1\varepsilon_1$; to show this, we already know that $N_1w=N_1(z-z_1\varepsilon_1)\in\im\mathcal F$, so we pass through $\overline\rho$. In other words, it suffices to show that $\rho(w)=0$ from $\rho(N_1w)=0$ and the fact that $w$ features no $\sigma_1$ nor $\varepsilon_1$ terms.
	
% 	Well, because $w$ features no $\sigma_1$ nor $\varepsilon_1$ terms, the only terms we care about have $g\varepsilon_i$ where $g$ has no $\sigma_1$ and $i>1$; in this case,
% 	\[\rho\left(g\sigma_1^a\varepsilon_i\right)\coloneqq\sigma_1^ag_i\big(\sigma_i^{a_i}-N_i1_{a_i=n_i-1}\big)\varepsilon_i+\sum_{j=i+1}^m\sigma_1^ag_j\sigma_j^{(a_j)}T_i\varepsilon_j=\sigma_1^a\rho(g\varepsilon_i),\]
% 	where $g\coloneqq\prod_{i=2}^m\sigma_i^{a_i}$ with $0\le a_i<n_i$. Looping over all possible $g$ and $\varepsilon_i$, we see $\rho(\sigma_1^aw)=\sigma_1^a\rho(w)$, so
% 	\[N_1\rho(w)=\rho(N_1w)=0.\]
% 	Thus, $\rho(w)\in\im T_1$, so say $\rho(w)=(\sigma_1-1)w'$. However, because $w$ has no $\varepsilon_1$ terms nor any term with a $\sigma_1$, we can see from the expansion of $\rho(w)$ that $\rho(w)$ will have no $\sigma_1$ terms. It follows that $\rho(w)\in\ZZ[G]^m$ is preserved upon applying $\sigma_1\mapsto1$, but then $(\sigma_1-1)w'$ gets sent to $0$, so it follows $\rho(w)=0$. This finishes.
% \end{proof}
% \begin{lemma} \label{lem:cocycleforcecohere}
% 	Fix everything as in the set-up. Suppose we have $\{z_i\}_{i=1}^m\subseteq\ZZ[G]$ such that
% 	\[T_iz_j\varepsilon_j=T_jz_i\varepsilon_i\]
% 	in $\coker\mathcal F$, for any pair of indices $(i,j)$. Then there exists $z\in\ZZ[G]$ such that $z\varepsilon_i=z\varepsilon_i$ (in $\coker\mathcal F$) for each index $i$.
% \end{lemma}
% \begin{proof}
% 	We proceed by induction on $m$. For $m=1$, we simply set $z\coloneqq z_1$. For the inductive step, take $m>1$, and we are given elements $\{z_i\}_{i=1}^m\subseteq\ZZ[G]$ such that
% 	\[T_iz_j\varepsilon_j=T_jz_i\varepsilon_i\]
% 	for any pair of indices $(i,j)$. By the inductive hypothesis, we may use the equations with indices less than $m$ to conjure some $z\in\ZZ[G]$ such that
% 	\[z\varepsilon_i\equiv z_i\varepsilon_i\pmod{\im\mathcal F}\]
% 	for each $i<m$. It remains to deal with the equations which have $m$ as an index; namely, for each $i<m$, we have an equation
% 	\[T_iz_m\varepsilon_m\equiv T_mz_i\varepsilon_i\equiv T_mz\varepsilon_i\pmod{\im\mathcal F}.\]
% 	Now, $T_m\varepsilon_i\equiv T_i\varepsilon_m\pmod{\im\mathcal F}$, so this is equivalent to asserting
% 	\[T_i(z_m-z)\varepsilon_m\equiv0\pmod{\im\mathcal F}\]
% 	for each index $i<m$. Thus, $T_i(z_m-z)\varepsilon_m\in\im\mathcal F$ for each $i$, which we will use by passing through the $\rho$ of \autoref{lem:sessplits}: this is equivalent to $\rho(T_i(z_m-z)\varepsilon_m)=0$ for each $i<m$. Now, we note that any $g=\prod_{j=1}^m\sigma_j^{a_j}\sigma\in G$ and $i<m$ will have
% 	\[\rho(\sigma_ig\varepsilon_m)=\sigma_ig_m\big(\sigma_m^{a_m}-N_i1_{a_m=n_m-1}\big)\varepsilon_m=\sigma_i\rho(g\varepsilon_m),\]
% 	where in particular the sum in $\rho$ vanished because $m$ is the largest index. (Also, we note $(\sigma_ig)_m=\sigma_ig_m$ because $i<m$.) Extending this linearly over all $g\in G$, we see that
% 	\[0=\rho(T_i(z_m-z)\varepsilon_m)=T_i\rho((z_m-z)\varepsilon_m)\]
% 	for each $i<m$. In particular, letting $\rho((z_m-z)\varepsilon_m)=r\varepsilon_m$, we see$r\in\im N_i$ for each $i<m$, so it follows from \autoref{lem:separatenijs} that $r\in\im N_1\cdots N_{m-1}$, so we can find $w\in\ZZ[G]$ such that
% 	\[\rho((z_m-z)\varepsilon_m)=N_1\cdots N_{m-1}w\varepsilon_m.\]
% 	Now, for technical reasons we note that any $g=\prod_{j=1}^m\sigma_j^{a_j}$ gives
% 	\[\rho(g\varepsilon_m)=g_m\big(\sigma_m^{a_m}-N_i1_{a_m=n_m-1}\big)\varepsilon_m,\]
% 	which can have no $\sigma_m^{n_m-1}$ term in it because this would have to come from $\big(\sigma_m^{a_m}-N_i1_{a_m=n_m-1}\big)$, which manually kills all such terms. As such, $N_1\cdots N_{m-1}w$ should have no $\sigma_m^{n_m-1}$ terms, which means $w$ itself should have no such terms.

% 	With this in mind, we set $z'\coloneqq z+N_1\cdots N_{m-1}w$. To check that we haven't broken anything, we note that any $i<m$ has
% 	\[z'\varepsilon_i=z\varepsilon_i+N_1\cdots N_{m-1}w\varepsilon_i\equiv z\varepsilon_i\equiv z_i\varepsilon_i\pmod{\im\mathcal F}\]
% 	where we note that $N_i\varepsilon_i\equiv0\pmod{\im\mathcal F}$. It remains to deal with $i=m$. Because $w$ features no $\sigma_m^{a_m-1}$ terms, we can check that any $g=\prod_{j=1}^m\sigma_j^{a_j}$ with $a_m<n_m-1$ has
% 	\[\rho(g\varepsilon_m)=g_m\big(\sigma_m^{a_m}-N_i1_{a_m=n_m-1}\big)\varepsilon_m=g_m\sigma_m^{a_m}\varepsilon_m=g\varepsilon_m,\]
% 	so $\rho$ will just act as the identity on $w$! Extending this linearly, we see that
% 	\begin{align*}
% 		\rho((z_m-z')\varepsilon_m) &= \rho((z_m-z)\varepsilon_m)-\rho(N_1\cdots N_{m-1}w\varepsilon_m) \\
% 		&= N_1\cdots N_{m-1}w\varepsilon_m-N_1\cdots N_{m-1}w\varepsilon_m \\
% 		&= 0.
% 	\end{align*}
% 	Thus, $(z_m-z')\varepsilon_m\in\im\mathcal F$, so $z_m\varepsilon_m\equiv z\varepsilon_m\pmod{\im\mathcal F}$ as well.
% \end{proof}
% We are now ready to classify our $1$-cocycles.
% \begin{proposition} \label{prop:cocycleclassify}
% 	Fix everything as in the set-up. If $f\in Z^1(G,\coker\mathcal F)$ is a $1$-cocycle, then there exists $z\in\ZZ[G]$ such that $f(\sigma_i)=z\varepsilon_i$ for each index $i$. Combined with the formula in \autoref{lem:compresscocycle}, this fully determines $f$.
% \end{proposition}
% \begin{proof}
% 	We start by noting that each index $i$ has
% 	\[0=f(1)=f\left(\sigma_i^{n_i}\right)=\sigma_i^{(n_i)}f(\sigma_i)=N_i(f(\sigma_i))\]
% 	by plugging in $\sigma_i^{n_i}$ into \autoref{lem:compresscocycle}. Thus, \autoref{lem:cocycleforcecoord} grants us some $z_i\in\ZZ[G]$ such that $f(\sigma_i)=z_i\varepsilon_i$ for each index $i$.

% 	Continuing, we note that each pair of indices $(i,j)$ has
% 	\[\sigma_if(\sigma_j)+f(\sigma_i)=f(\sigma_i\sigma_j)=f(\sigma_j\sigma_i)=\sigma_jf(\sigma_i)+f(\sigma_j),\]
% 	so
% 	\[T_iz_j\varepsilon_j=T_if(\sigma_j)=T_jf(\sigma_i)=T_jz_i\varepsilon_i.\]
% 	Thus, we know from \autoref{lem:cocycleforcecohere} that there exists $z\in\ZZ[G]$ such that $f(\sigma_i)=z_i\varepsilon_i=z\varepsilon_i$ for each index $i$. This completes the proof.
% \end{proof}
% Note that \autoref{prop:cocycleclassify} does not say that all the conjured $1$-cocycles are actually $1$-cocycles. It will be beneficial for us to show this by hand, so we postpone it to the next subsection.

\subsection{Verification of 1-Cocycles}
Here we prove \autoref{prop:manufacturedcocycle}.
% verify that all the $1$-cocycles of \autoref{prop:cocycleclassify} are indeed $1$-cocycles.
Namely, we show that the $1$-cochain $\overline c\in C^1(G,\coker\mathcal F)$ defined by
\[\overline c(g)=\sum_{i=1}^mg_i\sigma_i^{(a_i)}\varepsilon_i\]
where $g\coloneqq\prod_{i=1}^m\sigma_i^{a_i}$ is actually a $1$-cocycle. It will be beneficial for us to do this by hand, which is a matter of brute force. Set $c\in C^1\left(G,\ZZ[G]^m\right)$ defined by
\[c(g)\coloneqq\left(g_i\sigma_i^{(a_i)}\right)^m_{i=1},\]
where $g\coloneqq\prod_{i=1}^m\sigma_i^{a_i}$. We will show that $\im dc\subseteq\im\mathcal F$, which we will mean that $\im\overline{dc}=\im d\overline c=0$, where $f\mapsto\overline f$ is the map $C^\bullet\left(G,\ZZ[G]^m\right)\onto C^\bullet\left(G,\coker\mathcal F\right)$ induced by modding out.

As such, we set $g\coloneqq\prod_{i=1}^m\sigma_i^{a_i}$ and $h\coloneqq\prod_{i=1}^m\sigma_i^{b_i}$ with $0\le a_i,b_i<n_i$ for each $i$. Then, using the division algorithm, write
\[a_i+b_i=n_iq_i+r_i\]
where $q_i\in\{0,1\}$ and $0\le r_i<n_i$ for each $i$. Now, we want to show $dc(g,h)\in\im\mathcal F$, so we begin by writing
\begin{align}
	dc(g,h) &= gc(h)-c(gh)+c(g) \notag \\
	&= g\left(h_i\sigma_i^{(b_i)}\right)_{i=1}^m-\Bigg(\prod_{p=0}^{i-1}\sigma_p^{r_p}\cdot\sigma_i^{(r_i)}\Bigg)_{i=1}^m+\left(g_i\sigma_i^{(a_i)}\right)_{i=1}^m \notag \\
	&= \left(gh_i\sigma_i^{(b_i)}\right)_{i=1}^m-\left(g_ih_i\sigma_i^{(r_i)}\right)_{i=1}^m+\left(g_i\sigma_i^{(a_i)}\right)_{i=1}^m. \label{eq:expandedcocycle}
\end{align}
We now go term-by-term in \autoref{eq:expandedcocycle}. The easiest is the middle term of \autoref{eq:expandedcocycle}, for which we write
\begin{align*}
	g_ih_i\sigma_i^{(r_i)} &= g_ih_i\sigma_i^{(a_i+b_i)}-g_ih_i\sigma_i^{r_i}\sigma_i^{(n_iq_i)} \\
	&= g_ih_i\sigma_i^{(a_i+b_i)}-g_ih_i\sigma_i^{a_i+b_i}\cdot q_iN_i \\
	&= g_ih_i\sigma_i^{(a_i+b_i)}-g_ih_i\cdot q_iN_i,
\end{align*}
where the last equality is because $\sigma_iN_i=N_i$. Thus,
\begin{align*}
	-\left(g_ih_i\sigma_i^{(r_i)}\right)_{i=1}^m &= -\left(g_ih_i\sigma_i^{(a_i+b_i)}\right)_{i=1}^m+\left(g_ih_i\cdot q_iN_i\right)_{i=1}^m \\
	&= -\left(g_ih_i\sigma_i^{(a_i+b_i)}\right)_{i=1}^m+\mathcal F\big((g_ih_iq_i)_i,(0)_{i>j}\big).
\end{align*}
Now, using \autoref{lem:expandgi}, the $i$th coordinate of the left term of \autoref{eq:expandedcocycle} is
\begin{align*}
	gh_i\sigma_i^{(b_i)} &= g_i\sigma_i^{a_i}\Bigg(\prod_{ j=i+1}^{m}\sigma_j^{a_j}\Bigg)h_i\sigma_i^{(b_i)} \\
	&= g_i\Bigg(1+\sum_{j=i+1}^{m}\Bigg(\prod_{q=i+1}^{j-1}\sigma_q^{a_q}\Bigg)\sigma_j^{(a_j)}T_j\Bigg)h_i\sigma_i^{a_i}\sigma_i^{(b_i)} \\
	&= g_ih_i\sigma_i^{a_i}\sigma_i^{(b_i)}+\sum_{j=i+1}^{m}\Bigg(g_i\sigma_i^{a_i}\prod_{q=i+1}^{j-1}\sigma_q^{a_q}\Bigg)h_i\sigma_j^{(a_j)}\sigma_i^{(b_i)}T_j \\
	&= g_ih_i\sigma_i^{a_i}\sigma_i^{(b_i)}+\sum_{j=i+1}^{m}g_jh_i\sigma_j^{(a_j)}\sigma_i^{(b_i)}T_j.
\end{align*}
And lastly, for the right term of \autoref{eq:expandedcocycle}, the $i$th coordinate is
\begin{align*}
	g_i\sigma_i^{(a_i)} &= g_i\Bigg(h_i-\sum_{j=1}^{i-1}h_j\sigma_j^{(b_j)}T_j\Bigg)\sigma_i^{(a_i)} \\
	&= g_ih_i\sigma_i^{(a_i)}-\sum_{j=1}^{i-1}g_ih_j\sigma_i^{(a_i)}\sigma_j^{(b_j)}T_j.
\end{align*}
So to finish, we continue from \autoref{eq:expandedcocycle}, which gives
\begin{align*}
	dc(g,h)-\mathcal F\big((g_ih_iq_i)_i,(0)_{i>j}\big) &= \left(g_ih_i\sigma_i^{a_i}\sigma_i^{(b_i)}\right)_{i=1}^m-\left(g_ih_i\sigma_i^{(a_i+b_i)}\right)_{i=1}^m+\left(g_ih_i\sigma_i^{(a_i)}\right)_{i=1}^m \\
	&\qquad\qquad+\Bigg(\sum_{j=i+1}^{m}g_jh_i\sigma_j^{(a_j)}\sigma_i^{(b_i)}T_j-\sum_{j=1}^{i-1}g_ih_j\sigma_i^{(a_i)}\sigma_j^{(b_j)}T_j\Bigg)_{i=1}^m \\
	&= \Bigg(-\sum_{j=1}^{i-1}g_ih_j\sigma_i^{(a_i)}\sigma_j^{(b_j)}T_j+\sum_{j=i+1}^{m}g_jh_i\sigma_j^{(a_j)}\sigma_i^{(b_i)}T_j\Bigg)_{i=1}^m \\
	&= \mathcal F\left((0)_i,\big(g_ih_j\sigma_i^{(a_i)}\sigma_j^{(b_j)}\big)_{i>j}\right).
\end{align*}
Thus,
\begin{equation}
	dc(g,h) = \mathcal F\left((g_ih_iq_i)_i,\big(g_ih_j\sigma_i^{(a_i)}\sigma_j^{(b_j)}\big)_{i>j}\right)\in\im\mathcal F. \label{eq:computedelta}
\end{equation}
This completes the proof of \autoref{prop:manufacturedcocycle}.

In fact, the above proof has found an explicit element $z$ so that $\mathcal F(z)=dc(g,h)$ for each $g,h\in G$. As such, we recall that we set
\[X\coloneqq\frac{\ZZ[G]^m\times\ZZ[G]^{\binom m2}}{\ker\mathcal F}\]
to give the short exact sequence
\[0\to X\stackrel{\mathcal F}\to\ZZ[G]^m\to\coker\mathcal F\to0.\]
In particular, we can track $\overline c\in Z^1(G,\coker\mathcal F)$ through a boundary morphism: we already have a chosen lift $c\in Z^1(G,\ZZ[G]^m)$ for $\overline c$, and we have also computed $\mathcal F^{-1}\circ dc$ from the above work. This gives the following result.
\begin{cor} \label{cor:deltaccomputation}
	Fix everything as in the set-up. Then the $\overline c$ of \autoref{prop:manufacturedcocycle} has
	\[\delta(c)(g,h)\coloneqq\left((g_ih_iq_i)_i,\big(g_ih_j\sigma_i^{(a_i)}\sigma_j^{(b_j)}\big)_{i>j}\right)\in Z^2(G,X)\]
	where $\delta$ is induced by
	\[0\to X\stackrel{\mathcal F}\to\ZZ[G]^m\to\coker\mathcal F\to0.\]
\end{cor}
\begin{proof}
	This follows from tracking how $\delta$ behaves, using \autoref{eq:computedelta}.
\end{proof}
\begin{remark}
	In some sense, this $\delta(c)$ is exactly the cocycle of \autoref{thm:getcocycle}, where we have abstracted away everything about $A$. We will rigorize this notion in our proof of \autoref{thm:yesitisacocycle}.
\end{remark}
% We are now ready to complete the proof of \autoref{prop:allmanufacturedcocycles}. In fact, we show the following stronger result.
% \begin{proposition} \label{prop:computeh1cokerF}
% 	Fix everything as in the set-up. Further, let $\varepsilon\colon\ZZ[G]\to\ZZ$ be the augmentation map sending $\sigma_i\mapsto1$ for each $i$. Then the following are true.
% 	\begin{listalph}
% 		\item Given any $z\in\ZZ[G]$, the formula
% 		\[f(g)=\sum_{i=1}^mg_i\sigma_i^{(a_i)}\cdot z\varepsilon_i=(z\cdot\overline c)(g)\]
% 		for $g\coloneqq\prod_{i=1}^m\sigma_i^{a_i}$ defines a $1$-cocycle in $Z^1(G,\coker\mathcal F)$. These are all the $1$-cocycles.
% 		\item If $f\in Z^1(G,\coker\mathcal F)$ is a $1$-cocycle, then $[f]=[\varepsilon(z)\cdot\overline c]$ in $H^1(G,\coker\mathcal F)$, for the $z\in\ZZ[G]$ of \autoref{prop:cocycleclassify}. In particular, $H^1(G,\coker\mathcal F)$ is a cyclic abelian group generated by $[\overline c]$.
% 	\end{listalph}
% \end{proposition}
% \begin{proof}
% 	We proceed one at a time.
% 	\begin{listalph}
% 		\item Given $z\in\ZZ[G]$, to see that $f$ is a $1$-cocycle, note that $f=z\cdot\overline c$. Thus, for the $1$-cocycle check, we just note that any $g,h\in G$ have
% 		\begin{align*}
% 			f(gh) &= z\cdot\overline c(gh) \\
% 			&= z\cdot(g\overline c(h)+\overline c(g)) \\
% 			&= gf(h)+f(g)
% 		\end{align*}
% 		because we already know that $\overline c\in Z^1(G,\coker\mathcal F)$.

% 		To see that these are all the $1$-cocycles, let $f\in Z^1(G,\coker\mathcal F)$ be any $1$-cocycle. Then \autoref{prop:cocycleclassify} promises $z\in\ZZ[G]$ such that $f(\sigma_i)=z\varepsilon_i$ for each index $i$, for which \autoref{lem:compresscocycle} tells us that
% 		\[f(g)=\sum_{i=1}^mg_i\sigma_i^{(a_i)}f(\sigma_i)=\sum_{i=1}^mg_i\sigma_i^{(a_i)}\cdot z\varepsilon_i\]
% 		for $g\coloneqq\prod_{i=1}^m\sigma_i^{a_i}$. So $f$ does have the desired form.

% 		\item Fix $f\in Z^1(G,\coker\mathcal F)$, and conjure the corresponding $z\in\ZZ[G]$ of \autoref{prop:cocycleclassify}. We note in part (a) that $f=z\cdot\overline c$, so it remains to show that $[z\cdot\overline c]=[\varepsilon(z)\cdot\overline c]$ in $H^1(G,\coker\mathcal F)$.
		
% 		By linearity of $\ZZ[G]$, it suffices to show that $[g\cdot\overline c]=[\overline c]$ for each $g\in G$. By induction on the number of generators $\sigma_i$ appearing in $g\in G$, it suffices to show that $[\sigma_i\cdot\overline c]=[\overline c]$ for each index $i$. Lastly, by rearranging the $\sigma_i$, it suffices to show that $[\sigma_1\cdot\overline c]=[\overline c]$.

% 		Well, for any $\sigma_i$, we note that
% 		\[(\sigma_1\overline c-\overline c)(\sigma_i)=\sigma_1\varepsilon_i-\varepsilon_i=T_1\varepsilon_i=T_i\varepsilon_1,\]
% 		where in the last equality we have used that we're living in $\coker\mathcal F$. Letting $d\colon C^0(G,\coker\mathcal F)\to B^1(G,\coker\mathcal F)$ denote the corresponding differential, we see
% 		\[(\sigma_1\overline c-\overline c-d\varepsilon_1)(\sigma_i)=T_i\varepsilon_1-(\sigma_i-1)\varepsilon_1=0\]
% 		for each index $i$. Thus, $\sigma_1\overline c-\overline c-d\varepsilon_1)\in Z^1(G,\coker\mathcal F)$ vanishes on all $\sigma_i$, so \autoref{lem:compresscocycle} tells us that it vanishes on all $g\in G$. It follows $[\sigma_1\overline c-\overline c]=[0]$, which finishes.
% 	\end{listalph}
% 	The above parts complete the proof.
% \end{proof}
% \begin{cor} \label{cor:computeh2x}
% 	Fix everything as in the set-up. Then $H^2(G,X)$ is a cyclic abelian group generated by $[\delta(\overline c)]$, where $\delta$ is induced by
% 	\[0\to X\stackrel{\mathcal F}\to\ZZ[G]^m\to\coker\mathcal F\to0.\]
% \end{cor}
% \begin{proof}
% 	From the long exact sequence of cohomology, we see that
% 	\[\delta\colon H^1(G,\coker\mathcal F)\to H^2(G,X)\]
% 	is an isomorphism because $\ZZ[G]^m$ is projective and hence acyclic. Thus, this follows from (b) of \autoref{prop:computeh1cokerF}.
% \end{proof}

\subsection{Tuples via Cohomology}
We continue in the set-up of the previous subsection. The goal of this subsection is to prove \autoref{prop:alternativetuple}. The main idea is that we will be able to finitely generate $\ker\mathcal F$ essentially using the relations of a $\{\sigma_i\}_{i=1}^m$-tuple.

We start with the following basic result.
\begin{lemma} \label{lem:getgens}
	Fix everything as in the set-up. Then $\ker\mathcal F$ contains the following elements.
	\begin{listalph}
		\item $T_p\kappa_p$ for any index $p$.
		\item $N_pN_q\lambda_{pq}$ for any pair of indices $(p,q)$ with $p>q$.
		\item $T_q\kappa_p+N_p\lambda_{pq}$ for any pair of indices $(p,q)$ with $p>q$.
		\item $T_p\kappa_q-N_q\lambda_{pq}$ for any pair of indices $(p,q)$ with $p>q$.
		\item $T_q\lambda_{pr}-T_r\lambda_{pq}-T_p\lambda_{qr}$ for any triplet of indices $(p,q,r)$ with $p>q>r$.
	\end{listalph}
\end{lemma}
\begin{proof}
	We start by showing that all the listed elements are in fact in $\ker\mathcal F$.
	\begin{listalph}
		\item Note that $\mathcal F$ only ever takes the $x_i$ term to $x_iN_i$, so if $x_i=T_i$, then the effect of $x_i$ vanishes.
		\item Similarly, note that $\mathcal F$ only ever takes the $y_{ij}$ term to $y_{ij}T_i$ or $y_{ij}T_j$. As such, if $y_{ij}=N_iN_j$, then the effect of $y_{ij}$ vanishes again.
		\item The only relevant terms are at indices $p$ and $q$. Here, $i=p$ has $\mathcal F$ output
		\[T_qN_p-N_pT_q+0=0.\]
		For $i=q$, we have no $x_q$ term, so we are left with $N_pT_p=0$.
		\item Again, the only relevant terms are at indices $p$ and $q$. This time the interesting term is at $i=q$, where we have
		\[T_pN_q-0+(-N_q)T_p=0.\]
		Then at $i=p$, we simply have $0N_p-(-N_q)T_q+0=0$.
		\item The relevant terms, as usual, are for $i\in\{p,q,r\}$.
		\begin{itemize}
			\item At $i=p$, we have $0-(T_qT_r+(-T_r)T_q)+0=0.$
			\item At $i=q$, we have $0-(-(T_p)T_r)+((-T_r)T_p)=0$.
			\item At $i=r$, we have $0-0+(T_qT_p+(-T_p)T_q)=0$.
		\end{itemize}
	\end{listalph}
	The above checks complete this part of the proof.
\end{proof}
\begin{remark}
	The above elements are intended to encode the relations to be a $\{\sigma_i\}_{i=1}^n$-tuple. We will see this made rigorous in the proof of \autoref{prop:alternativetuple}.
\end{remark}
In fact, the following is true.
\begin{lemma} \label{lem:havegens}
	Fix everything as in the set-up. Then the elements (a)--(e) of \autoref{lem:getgens}, with (b) removed, generate $\ker\mathcal F$.
\end{lemma}
\begin{proof}
	We remark that we callously removed (b) because it is implied (c): $T_q\kappa_p+N_p\lambda_{pq}\in\ker\mathcal F$ implies that
	\[N_q\cdot(T_q\kappa_p+N_p\lambda_{pq})=N_pN_q\lambda_{pq}\]
	is also in $\ker\mathcal F$. Anyway, this proof is long and annoying and hence relegated to \autoref{sec:havegensproof}.
\end{proof}
Here is the payoff for the hard work in \autoref{lem:havegens}.
\propalternativetuple*
\begin{proof}
	Let $\mathcal T$ denote the set of $\{\sigma_i\}_{i=1}^m$-tuples. We now define the map $\varphi\colon\op{Hom}_{\ZZ[G]}(X,A)\to\mathcal T$ by
	\[\varphi\colon f\mapsto\Big(\big(f(\kappa_i)\big)_i,\big(f(\lambda_{ij})\big)_{i>j}\Big).\]
	In other words, we simply read off the values of $f$ from indicators on the coordinates of $X$. It's not hard to see that $\varphi$ is in fact a $G$-module homomorphism, but we will have to check that $\varphi$ is well-defined, for which we have to check the conditions on being a $\{\sigma_i\}_{i=1}^m$-tuple.
	\begin{lemma} \label{lem:kernelisrelations}
		Fix everything as in the set-up, and let $A$ be a $G$-module. Then, given $f\colon\ZZ[G]^m\times\ZZ[G]^{\binom m2}$, we have $\ker\mathcal F\subseteq\ker f$ if and only if
		\[\Big(\big(f(\kappa_i)\big)_i,\big(f(\lambda_{ij})\big)_{i>j}\Big)\]
		is a $\{\sigma_i\}_{i=1}^m$-tuple.
	\end{lemma}
	\begin{proof}
		By \autoref{lem:havegens}, we see $\ker\mathcal F\subseteq\ker f$ if and only if $f$ vanishes on the elements given in \autoref{lem:getgens}. As such, we now run the following checks.
		\begin{enumerate}
			\item We discuss \autoref{eq:tuplefields}. For one, note that $f(\lambda_{ij})\in A$ essentially for free. Now, we note
			\begin{align*}
				f(\kappa_i)\in A^{\langle\sigma_i\rangle} &\iff T_if(\kappa_i)=0 \\
				&\iff f(T_i\kappa_i)=0 \\
				&\iff T_i\kappa_i\in\ker f.
			\end{align*}
			\item We discuss \autoref{eq:tuplerelations}. On one hand, note that $i>j$ has
			\begin{align*}
				N_if(\lambda_{ij})=-T_jf(\lambda_i) &\iff f(N_i\lambda_{ij}+T_j\lambda_i) \\
				&\iff N_i\lambda_{ij}+T_j\lambda_i\in\ker f.
			\end{align*}
			On the other hand,
			\begin{align*}
				-N_jf(\lambda_{ij})=-T_if(\lambda_j) &\iff f(N_j\lambda_{ij}+T_i\lambda_j)=0 \\
				&\iff N_j\lambda_{ij}+T_i\lambda_j\in\ker f.
			\end{align*}
			\item We discuss \autoref{eq:betarelations}. Simply note indices $i>j>k$ have
			\begin{align*}
				T_jf(\lambda_{ik})=T_kf(\lambda_{ij})+T_if(\lambda_{jk}) &\iff f(T_j\lambda_{ik}-T_k\lambda_{ij}-T_i\lambda_{jk})=0 \\
				&\iff T_j\lambda_{ik}-T_k\lambda_{ij}-T_i\lambda_{jk}\in\ker f.
			\end{align*}
		\end{enumerate}
		In total, we see that satisfying the relations to be a $\{\sigma_i\}_{i=1}^m$-tuple exactly encodes the data of having the generators of $\ker\mathcal F$ live in $\ker f$.
	\end{proof}
	So indeed, given $f\colon X\to A$, the above lemma applied to the composite
	\[\ZZ[G]^m\times\ZZ[G]^{\binom m2}\onto X\stackrel{f}\to A\]
	shows that $\varphi(f)\in\mathcal T$.

	To show that $\varphi$ is an isomorphism, we exhibit its inverse; fix some $(\{\alpha_i\},\{\beta_{ij}\}_{i>j})\in\mathcal T$. Well, $\ZZ[G]\times\ZZ[G]^{\binom m2}$ has as a basis the $\kappa_i$ and $\lambda_{ij}$, so we can uniquely define a $G$-module homomorphism $f\colon X\to A$ by
	\[f(\kappa_i)\coloneqq\alpha_i\qquad\text{and}\qquad f(\lambda_{ij})\coloneqq\beta_{ij}\]
	for all relevant indices $i,j$, and in fact the map $\mathcal T\to\op{Hom}_\ZZ\left(\ZZ[G]^m\times\ZZ[G]^{\binom m2},A\right)$ we can see to be a $G$-module homomorphism. However, because these outputs are a $\{\sigma_i\}_{i=1}^m$-tuple, we can read \autoref{lem:kernelisrelations} backward to say that $f$ has kernel containing $\ker\mathcal F$, so in fact we induce a map $\overline f\colon X\to A$.
	
	So in total, we get a $G$-module homomorphism $\psi\colon\mathcal T\to\op{Hom}_{\ZZ[G]}(X,A)$ by
	\[\psi\colon(\{\alpha_i\},\{\beta_{ij}\}_{i>j})\mapsto\overline f,\]
	where $\overline f$ is defined on the basis elements above. Further, $\psi$ is the inverse of $\varphi$ essentially because the $\{\kappa_i\}_i\cup\{\lambda_{ij}\}_{i>j}$ form a basis of $\ZZ[G]^m\times\ZZ[G]^{\binom m2}$. This completes the proof.
\end{proof}
And now because it is so easy, we might as well prove \autoref{thm:yesitisacocycle}.
\thmyesitisacocycle*
\begin{proof}
	The main point is that we have a computation of $\delta(\overline c)$ from \autoref{cor:deltaccomputation}, which we merely need to track through. In particular, fix a $\{\sigma_i\}_{i=1}^m$-tuple $(\{\alpha_i\}_i,\{\beta_{ij}\}_{i>j})$, and let $f\in H^0(G,\op{Hom}_\ZZ(X,A))$ be the corresponding morphism. As such, we may compute
	\[\delta(\overline c)\cup f\colon(g,h)\mapsto\delta(\overline c)(g,h)\otimes_\ZZ gh\cdot f=\delta(\overline c)(g,h)\otimes_\ZZ f.\]
	To pass through evaluation, we set $g\coloneqq\prod_i\sigma_i^{a_i}$ and $h\coloneqq\prod_i\sigma_i^{b_i}$, from which we get
	\begin{align*}
		f(\delta(\overline c)(g,h)) &= f\left((g_ih_iq_i)_i,\big(g_ih_j\sigma_i^{(a_i)}\sigma_j^{(b_j)}\big)_{i>j}\right) \\
		&= \sum_{i=1}^mg_ih_i\floor{\frac{a_i+b_i}{n_i}}\cdot\alpha_i+\sum_{\substack{i,j=1\\i>j}}^mg_ih_j\sigma_i^{(a_i)}\sigma_j^{(b_j)}\cdot\beta_{ij} \\
		&= \sum_{\substack{i,j=1\\i>j}}^m\Bigg(\prod_{p<i}\sigma_p^{a_p}\Bigg)\Bigg(\prod_{q<j}\sigma_q^{b_q}\Bigg)\sigma_i^{(a_i)}\sigma_j^{(b_j)}\beta_{ij}+\sum_{i=1}^mg_ih_i\alpha_i^{\floor{\frac{a_i+b_i}{n_i}}}.
	\end{align*}
	Doing a little more rearrangement and writing this multiplicatively exactly recovers the cocycle of \autoref{thm:getcocycle}. This completes the proof.
\end{proof}

% \subsection{Some Loose Ends}
% We continue in the set-up and notation of the previous subsection.
Though we have proven everything we set out to do in \autoref{sec:overview}, there is more to discuss with our alternate description of tuples. As a taste, we prove the following extension of \autoref{prop:alternativetuple}.
\begin{proposition} \label{prop:alternativetupleclass}
	Fix everything as in the set-up, and let $A$ be a $G$-module. Then the isomorphism of \autoref{prop:alternativetuple} descends to an isomorphism between equivalence classes of $\{\sigma_i\}_{i=1}^m$-tuples are canonically isomorphic to $\widehat H^0(G,\op{Hom}_\ZZ(X,A))$.
\end{proposition}
\begin{proof}
	Recall that the short exact sequence
	\[0\to X\stackrel{\mathcal F}\to\ZZ[G]^m\to\coker\mathcal F\to 0\]
	of $G$-modules splits as $\ZZ$-modules by \autoref{lem:sessplits}, so we have a short exact sequence
	\[0\to\op{Hom}_\ZZ(\coker\mathcal F,A)\to\op{Hom}_\ZZ(\ZZ[G]^m,A)\stackrel{-\circ\mathcal F}\to\op{Hom}_\ZZ(X,A)\to 0.\]
	Now, the key trick will be to compare regular group cohomology with Tate cohomology. To begin, we note that our cohomology theories give the following commutative diagram with exact rows.
	% https://q.uiver.app/?q=WzAsNixbMCwwLCJIXjAoRyxcXG9we0hvbX1fXFxaWihcXFpaW0ddXm0sQSkpIl0sWzEsMCwiSF4wKEcsXFxvcHtIb219X1xcWlooWCxBKSkiXSxbMSwxLCJcXHdpZGVoYXQgSF4wKEcsXFxvcHtIb219X1xcWlooWCxBKSkiXSxbMiwwLCJIXjEoRyxcXG9we0hvbX1fXFxaWihcXGNva2VyXFxtYXRoY2FsIEYsQSkpIl0sWzIsMSwiXFx3aWRlaGF0IEheMShHLFxcb3B7SG9tfV9cXFpaKFxcY29rZXJcXG1hdGhjYWwgRixBKSkiXSxbMCwxLCIwIl0sWzEsMiwiIiwwLHsic3R5bGUiOnsiaGVhZCI6eyJuYW1lIjoiZXBpIn19fV0sWzMsNCwiIiwwLHsibGV2ZWwiOjIsInN0eWxlIjp7ImhlYWQiOnsibmFtZSI6Im5vbmUifX19XSxbMSwzXSxbMiw0XSxbNSwyXSxbMCwxLCItXFxjaXJjXFxtYXRoY2FsIEYiXV0=&macro_url=https%3A%2F%2Fraw.githubusercontent.com%2FdFoiler%2Fnotes%2Fmaster%2Fnir.tex
	\begin{equation}
		\begin{tikzcd}
			{H^0(G,\op{Hom}_\ZZ(\ZZ[G]^m,A))} & {H^0(G,\op{Hom}_\ZZ(X,A))} & {H^1(G,\op{Hom}_\ZZ(\coker\mathcal F,A))} \\
			0 & {\widehat H^0(G,\op{Hom}_\ZZ(X,A))} & {\widehat H^1(G,\op{Hom}_\ZZ(\coker\mathcal F,A))}
			\arrow[two heads, from=1-2, to=2-2]
			\arrow[Rightarrow, no head, from=1-3, to=2-3]
			\arrow[from=1-2, to=1-3]
			\arrow[from=2-2, to=2-3]
			\arrow[from=2-1, to=2-2]
			\arrow["{-\circ\mathcal F}", from=1-1, to=1-2]
		\end{tikzcd} \label{eq:crazycohomology}
	\end{equation}
	Here, the middle vertical map is reduction modulo $\im N_G$. The rows are exact from the long exact sequences, and the square commutes by construction of Tate cohomology. Now, the point is that the diagram induces the isomorphism
	\begin{equation}
		\frac{H^0(G,\op{Hom}_\ZZ(X,A))}{\im(-\circ\mathcal F)}\simeq\widehat H^0(G,\op{Hom}_\ZZ(X,A)), \label{eq:crazyinducediso}
	\end{equation}
	which simply sends $[f]\mapsto[f]$.

	Thus, the main content here will be to track through the image of $-\circ\mathcal F$ in \autoref{eq:crazycohomology}. Let $\mathcal T$ denote the set of $\{\sigma_i\}_{i=1}^m$-triples of $A$, and let $\mathcal T_0$ denote the set (in fact, equivalence class) of triples corresponding to $[0]\in H^2(G,A)$. Letting $\varphi\colon H^0(G,\op{Hom}_\ZZ(X,A))\to\mathcal T$ be defined by
	\[\varphi\colon f\mapsto\Big(\big(f(\kappa_i)\big)_i,\big(f(\lambda_{ij})\big)_{i>j}\Big)\]
	be the isomorphism of \autoref{prop:alternativetuple}, we claim that the image of $-\circ\mathcal F$ in $H^0(G,\op{Hom}_\ZZ(X,A))$ corresponds under $\varphi$ to exactly $\mathcal T_0$.

	Indeed, we take a $G$-module homomorphism $f\colon\ZZ[G]^m\to A$ to the $G$-module homomorphism $(f\circ\mathcal F)\colon X\to A$. Then we compute
	\begin{align*}
		(f\circ\mathcal F)(\kappa_i) &= f(N_i\varepsilon_i) \\
		&= N_if(\varepsilon_i) \\
		(f\circ\mathcal F)(\lambda_{ij}) &= f(T_i\varepsilon_j-T_j\varepsilon_i) \\
		&= T_if(\varepsilon_j)-T_jf(\varepsilon_i)
	\end{align*}
	for all relevant indices $i$ and $j$. Thus,
	\[\varphi(f\circ\mathcal F)=\left(\big(N_if(\varepsilon_i)\big)_{i},\big(T_if(\varepsilon_j)-T_jf(\varepsilon_i)\big)_{i>j}\right),\]
	which we can see lives in $\mathcal T_0$ by definition of our equivalence relation (upon using multiplicative notation). In fact, as $f$ varies, we see that the values of $f(\varepsilon_i)$ may vary over all $A$, so the image of $f\mapsto\varphi(f\circ\mathcal F)$ is exactly all of $\mathcal T_0$. Thus, $\varphi$ induces an isomorphism
	\[\overline\varphi\colon\frac{H^0(G,\op{Hom}_\ZZ(X,A))}{\im(-\circ\mathcal F)}\simeq\frac{\mathcal T}{\mathcal T_0}.\]
	Composing this with the ``identity'' map \autoref{eq:crazyinducediso} finishes the proof.
	% The main point is that the cup product with $\delta(\overline c)$ will induce an isomorphism
	% \[\widehat H^0(G,\op{Hom}_\ZZ(X,L^\times))\to H^2(G,L^\times).\]
	% Indeed, note that $\mathcal F\colon X\to\ZZ[G]^m$ is an embedding of $\ZZ$-modules, so $X$ is a free abelian group because $\ZZ[G]^m$ is. It follows that $X$ is the character group of an algebraic torus $\mathcal T=\op{Hom}_\ZZ(X,\mathbb G_m)$, so we write $X=X^*(\mathcal T)$. Now, the main point is that we can realize the cup-product map of \autoref{thm:yesitisacocycle} in Tate cohomology as
	% \[\cup\colon\widehat H^0(G,\mathcal T(L))\times\widehat H^2(G,X^*(\mathcal T))\to H^2(G,L^\times).\]
	% However, by Tate--Nakayama duality, we know that this pairing is non-degenerate. In particular, because $\delta(\overline c)$ generates $H^2(G,X^*(\mathcal T))=H^2(G,X)$ by \autoref{cor:computeh2x}, we know that the map
	% \[\delta(\overline c)\cup-\colon\widehat H^0(G,\op{Hom}_\ZZ(X,L^\times))\to H^2(G,L^\times)\]
	% must be injective. On the other hand, by taking a cohomology class $[c]\in H^2(G,L^\times)$, lifting to a representative $\{\sigma_i\}_{i=1}^m$-tuple (as in \autoref{thm:classisomorphism}) gives an input to the above cup product map hitting $[c]$. Thus, the above cup product map we already know to be surjective, so it is an isomorphism.
	% We now attack the statement directly. Let $\mathcal T$ denote the set of $\{\sigma_i\}_{i=1}^m$-tuples and $\mathcal T_0$ denote the set (in fact, equivalence class) of tuples corresponding to the trivial cohomology class in $H^2(G,L^\times)$. Then we draw the following diagram, which we claim commutes and is made of isomorphisms.
	% % https://q.uiver.app/?q=WzAsMyxbMCwwLCJcXG1hdGhjYWwgVC9cXG1hdGhjYWwgVF8wIl0sWzAsMSwiXFx3aWRlaGF0IEheMChHLFxcb3B7SG9tfV9cXFpaKFgsTF5cXHRpbWVzKSkiXSxbMSwxLCJcXHdpZGVoYXQgSF4yKEcsTF5cXHRpbWVzKSJdLFswLDJdLFswLDFdLFsxLDJdXQ==&macro_url=https%3A%2F%2Fraw.githubusercontent.com%2FdFoiler%2Fnotes%2Fmaster%2Fnir.tex
	% \[\begin{tikzcd}
	% 	{\mathcal T/\mathcal T_0} \\
	% 	{\widehat H^0(G,\op{Hom}_\ZZ(X,L^\times))} & {\widehat H^2(G,L^\times)}
	% 	\arrow[from=1-1, to=2-2]
	% 	\arrow[from=1-1, to=2-1, dashed]
	% 	\arrow[from=2-1, to=2-2]
	% \end{tikzcd}\]
	% Namely, the map $\mathcal T/\mathcal T_0\to\widehat H^2(G,L^\times)$ sends an equivalence class of tuples to its cocycle, and it is an isomorphism by \autoref{thm:classisomorphism}. Further, the map $\widehat H^0(G,\op{Hom}_\ZZ(X,L^\times))\to\widehat H^2(G,L^\times)$ is the cup product with $\delta(\overline c)$, and it is an isomorphism as described above.
	% Lastly, $\mathcal T/\mathcal T_0\to\widehat H^0(G,\op{Hom}_\ZZ(X,L^\times))$ is descended from the morphism of \autoref{prop:alternativetuple}, so the diagram does indeed commute by \autoref{thm:yesitisacocycle}. In particular, this vertical map is well-defined and in fact an isomorphism by the commutativity of the diagram. This completes the proof.
\end{proof}
% Another loose end we have to tie up is that we showed $H^2(G,X)$ is cyclic generated by $[\delta(\overline c)]$, but we do not actually know the order. Tracking through Tate--Nakayama duality in the proof will tell us that the order is $\#G$, but this requires $G$ to be a Galois group. Thankfully, we are able to work this out for general $G$ using the rest of the theory that we have built.
% \begin{lemma} \label{lem:zivanish}
% 	Fix everything as in the set-up. If $z\in\ZZ[G]$ has $z\varepsilon_i=0$ in $\coker\mathcal F$, then $z\in\im N_i$.
% \end{lemma}
% \begin{proof}
% 	The point is to pass through $\rho$ of \autoref{lem:sessplits}. By possibly rearranging the $\sigma_i$, we may assume that $i=m$. Then, for any $g\coloneqq\prod_{i=1}^m\sigma_i^{a_i}$, we see
% 	\[\rho(g\varepsilon_m)=g_m\big(\sigma_m^{a_m}-N_m1_{a_m=n_m-1}\big)\varepsilon_m=g\varepsilon_m-g_m1_{a_m=n_m-1}\cdot N_m\varepsilon_m.\]
% 	Namely, $\rho(g\varepsilon_m)-g\varepsilon_m=N_mz_g\varepsilon_m$ for some $z_g\in\ZZ[G]$.
	
% 	Extending this linearly, we see that
% 	\[\rho(z\varepsilon_m)-z\varepsilon_m=w\cdot N_m\varepsilon_m\]
% 	for some $w\in\ZZ[G]$, but $z\varepsilon_m=0$ in $\coker\mathcal F$ makes this say $z\varepsilon_m=-w\cdot N_m\varepsilon_m$. Because this is now an equality in $\ZZ[G]^m$, we conclude $z=-w\cdot N_m\in N_m$.
% \end{proof}
% \begin{lemma} \label{lem:computeordc}
% 	Fix everything as in the set-up. Then $z\cdot\overline c=0$ in $Z^1(G,\coker\mathcal F)$ if and only if $z\in\im N_G$, where $N_G=\sum_{g\in G}g$.
% \end{lemma}
% \begin{proof}
% 	In one direction, if $z=N_Gw$, then
% 	\[z\varepsilon_i=N_Gw\varepsilon_i\equiv0\pmod{\im\mathcal F}\]
% 	for each index $i$, so it follows that $(z\cdot\overline c)(\sigma_i)=z\varepsilon_i=0$ for each $\sigma_i$. Thus, using \autoref{lem:compresscocycle}, we conclude that $z\cdot\overline c=0$.

% 	The other direction is more difficult. Suppose that $z\cdot\overline c=0$. In particular, it follows that $(z\cdot\overline c)(\sigma_i)=z\varepsilon_i$ must equal $0$ for each index $i$. In particular, by \autoref{lem:zivanish}, we conclude that $z\in\im N_i$ for each index $i$, which by \autoref{lem:separatenijs} tells us that
% 	\[z\in\im N_1\cdots N_m=\im N_G.\]
% 	This completes the proof.
% \end{proof}
% \begin{prop} \label{prop:finishh1cokerFcomputation}
% 	Fix everything as in the set-up. Then $H^1(G,\coker\mathcal F)$ is cyclic of order $\#G$, generated by $[\overline c]$.
% \end{prop}
% \begin{proof}
% 	To help us use \autoref{prop:computeh1cokerF}, let $\varepsilon\colon\ZZ[G]\to\ZZ$ denote the augmentation map.
	
% 	Note that we already know $H^1(G,\coker\mathcal F)$ is cyclic generated by $[\overline c]$ by \autoref{prop:computeh1cokerF}, so it only remains to compute the order of $[\overline c]$. On one hand, we have an upper bound on the order of $[\overline c]$ because $H^1(G,\coker\mathcal F)$ is $\#G$-torsion, but we can also see this directly: note that \autoref{lem:computeordc} tells us that
% 	\[[0]=[N_G\cdot\overline c].\]
% 	However, $[N_G\cdot\overline c]=[\varepsilon(N_G)\cdot\overline c]=[\#G\cdot\overline c]$ by \autoref{prop:computeh1cokerF}, so we do see that $\#G\cdot\overline c=0$.
	
% 	It remains to show that $[\overline c]$ has order at least $\#G$. As such, it suffices to show that if $n$ has $[n\cdot\overline c]=[0]$, then $\#G\mid n$. In particular, $n\cdot\overline c$ is a coboundary, so letting $d\colon C^0(G,\coker\mathcal F)\to B^1(G,\coker\mathcal F)$ denote the corresponding differential, we have
% 	\[n\cdot\overline c=d\left(\sum_{i=1}^mb_i\varepsilon_i\right)=\sum_{i=1}^mb_i(d\varepsilon_i)\]
% 	for some $\{b_i\}_{i=1}^m\subseteq\ZZ[G]$. Now, $(d\varepsilon_i)(\sigma_j)=T_j\varepsilon_i=T_i\varepsilon_j$ for any pair of indices $(i,j)$, so by the uniqueness of the extension in \autoref{lem:compresscocycle}, we conclude $d\varepsilon_i=T_i\overline c$. Thus, we set
% 	\[z\coloneqq n-\sum_{i=1}^mb_iT_i\]
% 	so that $\varepsilon(z)=n$ and $z\cdot\overline c=0$.

% 	To finish, we note \autoref{lem:computeordc} now tells us that $z\in\im N_G$, so letting $z=N_Gw$, we see that
% 	\[n=\varepsilon(z)=\varepsilon(N_G)\varepsilon(w)=\#G\cdot\varepsilon(w),\]
% 	so $\#G\mid n$. This completes the proof.
% \end{proof}
% \begin{cor}
% 	Fix everything as in the set-up. Then $H^1(G,\coker\mathcal F)$ is cyclic of order $\#G$, generated by $[\delta(\overline c)]$, where $\delta$ is induced by
% 	\[0\to X\stackrel{\mathcal F}\to\ZZ[G]^m\to\coker\mathcal F\to0.\]
% \end{cor}
% \begin{proof}
% 	As in the proof of \autoref{cor:computeh2x}, we note $\delta\colon H^1(G,\coker\mathcal F)\to H^2(G,X)$ is an isomorphism, so this follows from \autoref{prop:finishh1cokerFcomputation}.
% \end{proof}

\subsection{Some Cup Product Computations}
We take a brief intermission to establish a little theory on cup products. In this section, we let $G$ denote a generic finite group (not necessarily assumed to be abelian) and $A$ a $G$-module.
\begin{lemma}[\cite{bonn-lectures}, Proposition~I.5.3] \label{lem:cupproductmorphism}
	Let $G$ be a finite group. Given any $G$-modules $A,B,C$ with a $G$-module homomorphism $\varphi\colon B\to C$, the following diagram commutes for any $p,q\in\ZZ$ and $[a]\in\widehat H^p(G,A)$.
	% https://q.uiver.app/?q=WzAsNCxbMCwwLCJcXHdpZGVoYXQgSF5wKEcsQikiXSxbMSwwLCJcXHdpZGVoYXQgSF5wKEcsQykiXSxbMCwxLCJcXHdpZGVoYXQgSF57cCtxfShHLEFcXG90aW1lc19cXFpaIEIpIl0sWzEsMSwiXFx3aWRlaGF0IEhee3ArcX0oRyxBXFxvdGltZXNfXFxaWiBDKSJdLFswLDEsIlxcdmFycGhpIl0sWzIsMywiXFx2YXJwaGkiXSxbMCwyLCJhXFxjdXAgLSIsMl0sWzEsMywiYVxcY3VwIC0iXV0=&macro_url=https%3A%2F%2Fraw.githubusercontent.com%2FdFoiler%2Fnotes%2Fmaster%2Fnir.tex
	\[\begin{tikzcd}
		{\widehat H^q(G,B)} & {\widehat H^q(G,C)} \\
		{\widehat H^{p+q}(G,A\otimes_\ZZ B)} & {\widehat H^{p+q}(G,A\otimes_\ZZ C)}
		\arrow["\varphi", from=1-1, to=1-2]
		\arrow["\id_A\otimes\varphi", from=2-1, to=2-2]
		\arrow["{[a]\cup -}"', from=1-1, to=2-1]
		\arrow["{[a]\cup -}", from=1-2, to=2-2]
	\end{tikzcd}\]
\end{lemma}
\begin{proof}
	When $p,q\ge0$, we can argue directly. Indeed, we claim that the diagram commutes on the level of homogeneous cochains: let $[a]\in\widehat H^p(G,A)$ and $[b]\in\widehat H^q(G,B)$ be cohomology classes represented by the homogeneous cochains $a\in[a]$ and $b\in[b]$. Tracking along the top of the diagram, we see
	\begin{align*}
		(a\cup\varphi(b))(g_0,\ldots,g_{p+q}) &= a(g_0,\ldots,g_p)\otimes\varphi(b)(g_p,\ldots,g_{p+1}) \\
		&= a(g_0,\ldots,g_p)\otimes\varphi(b(g_p,\ldots,g_{p+1})).
	\end{align*}
	Tracking along the bottom of the diagram, we see
	\begin{align*}
		(\id_A\otimes\varphi)(a\cup b)(g_0,\ldots,g_{p+q}) &= (\id_A\otimes\varphi)(a(g_0,\ldots,g_p)\otimes b(g_p,\ldots,g_{p+q})) \\
		&= a(g_0,\ldots,g_p)\otimes\varphi(b(g_p,\ldots,g_{p+q})),
	\end{align*}
	which is equal. This completes the proof in the case of $p,q\ge0$.

	We will only need the case of $p,q\ge0$ in the application, but we will go ahead and do the general case now; we dimension-shift $p$ and $q$ downwards. For example, to shift $p$ downwards, we note that the (split) short exact sequence
	\begin{equation}
		0\to A\otimes_\ZZ I_G\to A\otimes_\ZZ\ZZ[G]\to A\to0 \label{eq:standardashift}
	\end{equation}
	induces the isomorphism $\delta\colon\widehat H^{p-1}(G,A)\to\widehat H^p(G,I_G\otimes_\ZZ A)$. As such, given $a\in\widehat H^{p-1}(G,A)$, the inductive hypothesis reassures that the following diagram commutes.
	% https://q.uiver.app/?q=WzAsNCxbMCwwLCJcXHdpZGVoYXQgSF5wKEcsQikiXSxbMSwwLCJcXHdpZGVoYXQgSF5wKEcsQykiXSxbMCwxLCJcXHdpZGVoYXQgSF57cCtxfShHLElfR1xcb3RpbWVzX1xcWlogQVxcb3RpbWVzX1xcWlogQikiXSxbMSwxLCJcXHdpZGVoYXQgSF57cCtxfShHLElfR1xcb3RpbWVzX1xcWlogQVxcb3RpbWVzX1xcWlogQykiXSxbMCwxLCJcXHZhcnBoaSJdLFsyLDMsIlxcdmFycGhpIl0sWzAsMiwiXFxkZWx0YShhKVxcY3VwIC0iLDJdLFsxLDMsIlxcZGVsdGEoYSlcXGN1cCAtIl1d&macro_url=https%3A%2F%2Fraw.githubusercontent.com%2FdFoiler%2Fnotes%2Fmaster%2Fnir.tex
	\[\begin{tikzcd}
		{\widehat H^q(G,B)} & {\widehat H^q(G,C)} \\
		{\widehat H^{p+q}(G,I_G\otimes_\ZZ A\otimes_\ZZ B)} & {\widehat H^{p+q}(G,I_G\otimes_\ZZ A\otimes_\ZZ C)}
		\arrow["\varphi", from=1-1, to=1-2]
		\arrow["\varphi", from=2-1, to=2-2]
		\arrow["{\delta(a)\cup -}"', from=1-1, to=2-1]
		\arrow["{\delta(a)\cup -}", from=1-2, to=2-2]
	\end{tikzcd}\]
	In other words, all $b\in\widehat H^q(G,B)$ have $\varphi(\delta(a)\cup b)=\delta(a)\cup\varphi(b)$.
	
	Now, because \autoref{eq:standardashift} is split, we can hit it with $-\otimes_\ZZ B$ and $-\otimes_\ZZ C$ to induce the following commutative diagram with exact rows.
	% https://q.uiver.app/?q=WzAsMTAsWzAsMCwiMCJdLFsxLDAsIkFcXG90aW1lc19cXFpaIElfR1xcb3RpbWVzX1xcWlogQiJdLFsyLDAsIkFcXG90aW1lc19cXFpaXFxaWltHXVxcb3RpbWVzX1xcWlogQiJdLFszLDAsIkFcXG90aW1lc19cXFpaIEIiXSxbNCwwLCIwIl0sWzEsMSwiQVxcb3RpbWVzX1xcWlogSV9HXFxvdGltZXNfXFxaWiBDIl0sWzIsMSwiQVxcb3RpbWVzX1xcWlpcXFpaW0ddXFxvdGltZXNfXFxaWiBDIl0sWzMsMSwiQVxcb3RpbWVzX1xcWlogQyJdLFswLDEsIjAiXSxbNCwxLCIwIl0sWzAsMV0sWzEsMl0sWzIsM10sWzMsNF0sWzgsNV0sWzUsNl0sWzYsN10sWzcsOV0sWzEsNSwiXFx2YXJwaGkiLDJdLFsyLDYsIlxcdmFycGhpIiwyXSxbMyw3LCJcXHZhcnBoaSIsMl1d&macro_url=https%3A%2F%2Fraw.githubusercontent.com%2FdFoiler%2Fnotes%2Fmaster%2Fnir.tex
	\[\begin{tikzcd}
		0 & {A\otimes_\ZZ I_G\otimes_\ZZ B} & {A\otimes_\ZZ\ZZ[G]\otimes_\ZZ B} & {A\otimes_\ZZ B} & 0 \\
		0 & {A\otimes_\ZZ I_G\otimes_\ZZ C} & {A\otimes_\ZZ\ZZ[G]\otimes_\ZZ C} & {A\otimes_\ZZ C} & 0
		\arrow[from=1-1, to=1-2]
		\arrow[from=1-2, to=1-3]
		\arrow[from=1-3, to=1-4]
		\arrow[from=1-4, to=1-5]
		\arrow[from=2-1, to=2-2]
		\arrow[from=2-2, to=2-3]
		\arrow[from=2-3, to=2-4]
		\arrow[from=2-4, to=2-5]
		\arrow["\varphi"', from=1-2, to=2-2]
		\arrow["\varphi"', from=1-3, to=2-3]
		\arrow["\varphi"', from=1-4, to=2-4]
	\end{tikzcd}\]
	Letting $\delta_B\colon\widehat H^{p-1}(A\otimes_\ZZ B)\to\widehat H^p(I_G\otimes_\ZZ A\otimes_\ZZ B)$ and $\delta_C\colon\widehat H^{p-1}(A\otimes_\ZZ B)\to\widehat H^p(I_G\otimes_\ZZ A\otimes_\ZZ B)$ denote the corresponding isomorphisms (note that the middle terms are induced and hence acyclic), we note that the functoriality of boundary morphisms tells us that $\varphi\delta_B=\delta_C\varphi$. In total, it follows that $b\in\widehat H^q(G,B)$ will have
	\[\delta_C(\varphi(a\cup b))=\varphi(\delta_B(a\cup b))=\varphi(\delta(a)\cup b)\stackrel*=\delta(a)\cup\varphi(b)=\delta_C(a\cup\varphi(b)),\]
	where we have used the inductive hypothesis at $\stackrel*=$. Because $\delta_C$ is an isomorphism, this completes the step to shift $p$ downwards to $p-1$.

	Shifting $q$ downwards is similar. This time we start with the following commutative diagram whose rows are (split) short exact sequences.
	% https://q.uiver.app/?q=WzAsMTAsWzAsMCwiMCJdLFsxLDAsIklfR1xcb3RpbWVzX1xcWlogQiJdLFsyLDAsIlxcWlpbR11cXG90aW1lc19cXFpaIEIiXSxbMywwLCJCIl0sWzQsMCwiMCJdLFsxLDEsIklfR1xcb3RpbWVzX1xcWlogQyJdLFsyLDEsIlxcWlpbR11cXG90aW1lc19cXFpaIEMiXSxbMywxLCJDIl0sWzAsMSwiMCJdLFs0LDEsIjAiXSxbMCwxXSxbMSwyXSxbMiwzXSxbMyw0XSxbOCw1XSxbNSw2XSxbNiw3XSxbNyw5XSxbMSw1LCJcXHZhcnBoaSIsMl0sWzIsNiwiXFx2YXJwaGkiLDJdLFszLDcsIlxcdmFycGhpIiwyXV0=&macro_url=https%3A%2F%2Fraw.githubusercontent.com%2FdFoiler%2Fnotes%2Fmaster%2Fnir.tex
	\[\begin{tikzcd}
		0 & {I_G\otimes_\ZZ B} & {\ZZ[G]\otimes_\ZZ B} & B & 0 \\
		0 & {I_G\otimes_\ZZ C} & {\ZZ[G]\otimes_\ZZ C} & C & 0
		\arrow[from=1-1, to=1-2]
		\arrow[from=1-2, to=1-3]
		\arrow[from=1-3, to=1-4]
		\arrow[from=1-4, to=1-5]
		\arrow[from=2-1, to=2-2]
		\arrow[from=2-2, to=2-3]
		\arrow[from=2-3, to=2-4]
		\arrow[from=2-4, to=2-5]
		\arrow["\varphi"', from=1-2, to=2-2]
		\arrow["\varphi"', from=1-3, to=2-3]
		\arrow["\varphi"', from=1-4, to=2-4]
	\end{tikzcd}\]
	In particular, we let $\delta_B'\colon\widehat H^{q-1}(G,B)\to\widehat H^q(G,I_G\otimes_\ZZ B)$ and $\delta_C'\colon\widehat H^{q-1}(G,B)\to\widehat H^q(G,I_G\otimes_\ZZ C)$ denote the induced isomorphisms, and again functoriality of the boundary morphisms tells us that $\varphi\delta_B=\delta_C\varphi$. Now, the inductive hypothesis tells us that the following diagram commutes for any $a\in\widehat H^p(G,A)$.
	% https://q.uiver.app/?q=WzAsNCxbMCwwLCJcXHdpZGVoYXQgSF5wKEcsSV9HXFxvdGltZXNfXFxaWiBCKSJdLFsxLDAsIlxcd2lkZWhhdCBIXnAoRyxJX0dcXG90aW1lc19cXFpaIEMpIl0sWzAsMSwiXFx3aWRlaGF0IEhee3ArcX0oRyxBXFxvdGltZXNfXFxaWiBJX0dcXG90aW1lc19cXFpaIEIpIl0sWzEsMSwiXFx3aWRlaGF0IEhee3ArcX0oRyxBXFxvdGltZXNfXFxaWiBJX0dcXG90aW1lc19cXFpaIEMpIl0sWzAsMSwiXFx2YXJwaGkiXSxbMiwzLCJcXHZhcnBoaSJdLFswLDIsImFcXGN1cCAtIiwyXSxbMSwzLCJhXFxjdXAgLSJdXQ==&macro_url=https%3A%2F%2Fraw.githubusercontent.com%2FdFoiler%2Fnotes%2Fmaster%2Fnir.tex
	\[\begin{tikzcd}
		{\widehat H^q(G,I_G\otimes_\ZZ B)} & {\widehat H^q(G,I_G\otimes_\ZZ C)} \\
		{\widehat H^{p+q}(G,A\otimes_\ZZ I_G\otimes_\ZZ B)} & {\widehat H^{p+q}(G,A\otimes_\ZZ I_G\otimes_\ZZ C)}
		\arrow["\varphi", from=1-1, to=1-2]
		\arrow["\varphi", from=2-1, to=2-2]
		\arrow["{a\cup -}"', from=1-1, to=2-1]
		\arrow["{a\cup -}", from=1-2, to=2-2]
	\end{tikzcd}\]
	Namely, any $b\in\widehat H^{p-1}(G,B)$ has
	\begin{align*}
		\delta_C'(a\cup\varphi(b)) &= (-1)^p\big(a\cup\delta_C'(\varphi(b))\big) \\
		&= (-1)^p\big(a\cup\varphi(\delta_B'(b))\big) \\
		&\stackrel*= (-1)^p\varphi(a\cup\delta_B'(b)) \\
		&= (-1)^p\cdot(-1)^p\varphi(\delta_B'(a\cup b)) \\
		&= \delta_C'(\varphi(a\cup b)),
	\end{align*}
	where we've applied the inductive hypothesis at $\stackrel*=$. Because $\delta_C'$ is an isomorphism, this completes shifting $q$ downwards to $q-1$.
\end{proof}
\begin{remark}
	An analogous argument shows that a $G$-module homomorphism $\psi\colon A\to B$ induces the following commutative diagram, for any $p,q\in\ZZ$ and $c\in\widehat H^q(G,C)$.
	% https://q.uiver.app/?q=WzAsNCxbMCwwLCJcXHdpZGVoYXQgSF5wKEcsQSkiXSxbMSwwLCJcXHdpZGVoYXQgSF5wKEcsQikiXSxbMCwxLCJcXHdpZGVoYXQgSF57cCtxfShHLEFcXG90aW1lc19cXFpaIEMpIl0sWzEsMSwiXFx3aWRlaGF0IEhee3ArcX0oRyxCXFxvdGltZXNfXFxaWiBDKSJdLFswLDEsIlxccHNpIl0sWzIsMywiXFxwc2kiXSxbMCwyLCItXFxjdXAgYyIsMl0sWzEsMywiLVxcY3VwIGMiLDJdXQ==&macro_url=https%3A%2F%2Fraw.githubusercontent.com%2FdFoiler%2Fnotes%2Fmaster%2Fnir.tex
	\[\begin{tikzcd}
		{\widehat H^p(G,A)} & {\widehat H^p(G,B)} \\
		{\widehat H^{p+q}(G,A\otimes_\ZZ C)} & {\widehat H^{p+q}(G,B\otimes_\ZZ C)}
		\arrow["\psi", from=1-1, to=1-2]
		\arrow["\psi", from=2-1, to=2-2]
		\arrow["{-\cup c}"', from=1-1, to=2-1]
		\arrow["{-\cup c}"', from=1-2, to=2-2]
	\end{tikzcd}\]
\end{remark}
In a different direction, we will want a duality result. To begin, we recall the following.
\begin{prop}[\cite{cartan-eilenberg}, Corollary~XII.6.5] \label{prop:ceduality}
	Let $G$ be a finite group and $A$ be any $G$-module. Then the cup-product pairing induces an isomorphism
	\[\widehat H^{i-1}(G,\op{Hom}_\ZZ(A,\QQ/\ZZ))\to\op{Hom}_\ZZ\left(\widehat H^{-i}(G,A),\widehat H^{-1}(G,\QQ/\ZZ)\right)\]
	for all $i\in\ZZ$. Indeed, this is a duality upon identifying $\widehat H^{-1}(G,\QQ/\ZZ)$ with $\QQ/\ZZ$.
\end{prop}
We will use this to prove the following.
\begin{proposition} \label{prop:abstractintegralduality}
	Let $G$ be a finite group, and let $X$ be a finitely generated $\ZZ$-free $G$-module. Then the cup-product pairing induces an isomorphism
	\[\widehat H^i(G,\op{Hom}_\ZZ(X,\ZZ))\to\op{Hom}_\ZZ\left(\widehat H^{-i}(G,X),\widehat H^0(G,\ZZ)\right)\]
	for all $i\in\ZZ$. Indeed, this is a duality upon identifying $\widehat H^0(G,\ZZ)$ with $\frac1{\#G}\ZZ/\ZZ\subseteq\QQ/\ZZ$.
\end{proposition}
\begin{proof}
	This proof is analogous to \cite{cartan-eilenberg}, Theorem XII.6.6. The key to the proof is the short exact sequence
	\begin{equation}
		0\to\ZZ\to\QQ\to\QQ/\ZZ\to0. \label{eq:divisibleses}
	\end{equation}
	The main point is that $X$ being finitely generated and $\ZZ$-free implies that $X$ is projective (as an abelian group), so we can apply $\op{Hom}_\ZZ(X,-)$ to get out the short exact sequence
	\begin{equation}
		0\to\op{Hom}_\ZZ(X,\ZZ)\to\op{Hom}_\ZZ(X,\QQ)\to\op{Hom}_\ZZ(X,\QQ/\ZZ)\to0. \label{eq:homdivisibleses}
	\end{equation}
	Now, note that the multiplication-by-$n$ endomorphism on $\op{Hom}_\ZZ(X,\QQ)$ is an isomorphism (namely, $\QQ$ is a divisible abelian group), so the same will be true of $\widehat H^i(G,\op{Hom}_\ZZ(X,\QQ))$ for any $i\in\ZZ$. However, these cohomology groups must be $\#G$-torsion, so in fact $\widehat H^i(G,\op{Hom}_\ZZ(X,\QQ))=0$ for all $i\in\ZZ$.

	Similarly, we note that we can hit \autoref{eq:homdivisibleses} with the functor $-\otimes_\ZZ X$ to get another short exact sequence
	\begin{equation}
		0\to\op{Hom}_\ZZ(X,\ZZ)\otimes_\ZZ X\to\op{Hom}_\ZZ(X,\QQ)\otimes_\ZZ X\to\op{Hom}_\ZZ(X,\QQ/\ZZ)\otimes_\ZZ X\to0. \label{eq:tensorhomdivisibleses}
	\end{equation}
	Notably, this is exact because $X$ is a finitely generated, torsion-free $\ZZ$-module and hence flat as a $\ZZ$-module. Now, $\op{Hom}_\ZZ(X,\QQ)\otimes_\ZZ X$ is still a divisible abelian group, so again $\widehat H^i(G,\op{Hom}_\ZZ(X,\QQ))=0$ for all $i\in\ZZ$.

	The rest of the proof is tracking boundary morphisms around. Fix some $i\in\ZZ$. Note \autoref{eq:divisibleses} and \autoref{eq:homdivisibleses} and \autoref{eq:tensorhomdivisibleses} induce boundary isomorphisms
	\[\arraycolsep=1.4pt\begin{array}{rlcl}
		\delta \colon& \widehat H^{-1}(G,\QQ/\ZZ) &\to& \widehat H^0(G,\ZZ) \\
		\delta_h \colon& \widehat H^{i-1}(G,\op{Hom}_\ZZ(X,\QQ/\ZZ))&\to&\widehat H^i(G,\op{Hom}_\ZZ(X,\ZZ)) \\
		\delta_t \colon& \widehat H^{-1}(G,\op{Hom}_\ZZ(\QQ/\ZZ)\otimes_\ZZ X)&\to&\widehat H^0(G,\op{Hom}_\ZZ(X,\ZZ)\otimes_\ZZ X).
	\end{array}\]
	We also note that we have a morphism of short exact sequences
	% https://q.uiver.app/?q=WzAsMTAsWzAsMCwiMCJdLFsxLDAsIlxcb3B7SG9tfV9cXFpaKFgsXFxaWilcXG90aW1lc19cXFpaIFgiXSxbMiwwLCJcXG9we0hvbX1fXFxaWihYLFxcUVEpXFxvdGltZXNfXFxaWiBYIl0sWzMsMCwiXFxvcHtIb219X1xcWlooWCxcXFFRL1xcWlopXFxvdGltZXNfXFxaWiBYIl0sWzAsMSwiMCJdLFs0LDAsIjAiXSxbNCwxLCIwIl0sWzEsMSwiXFxaWiJdLFsyLDEsIlxcUVEiXSxbMywxLCJcXFFRL1xcWloiXSxbMCwxXSxbMSwyXSxbMiwzXSxbMyw1XSxbNCw3XSxbNyw4XSxbOCw5XSxbOSw2XSxbMSw3LCJcXGV0YV9cXFpaIiwyXSxbMiw4LCJcXGV0YV9cXFFRIiwyXSxbMyw5LCJcXGV0YV97XFxRUS9cXFpafSIsMl1d&macro_url=https%3A%2F%2Fraw.githubusercontent.com%2FdFoiler%2Fnotes%2Fmaster%2Fnir.tex
	\[\begin{tikzcd}
		0 & {\op{Hom}_\ZZ(X,\ZZ)\otimes_\ZZ X} & {\op{Hom}_\ZZ(X,\QQ)\otimes_\ZZ X} & {\op{Hom}_\ZZ(X,\QQ/\ZZ)\otimes_\ZZ X} & 0 \\
		0 & \ZZ & \QQ & {\QQ/\ZZ} & 0
		\arrow[from=1-1, to=1-2]
		\arrow[from=1-2, to=1-3]
		\arrow[from=1-3, to=1-4]
		\arrow[from=1-4, to=1-5]
		\arrow[from=2-1, to=2-2]
		\arrow[from=2-2, to=2-3]
		\arrow[from=2-3, to=2-4]
		\arrow[from=2-4, to=2-5]
		\arrow["{\eta_\ZZ}"', from=1-2, to=2-2]
		\arrow["{\eta_\QQ}"', from=1-3, to=2-3]
		\arrow["{\eta_{\QQ/\ZZ}}"', from=1-4, to=2-4]
	\end{tikzcd}\]
	where the $\eta_\bullet$ are evaluation maps. For peace of mind, we can check that the squares commute by the following lemma.
	\begin{lemma} \label{lem:evcommutes}
		Let $G$ be a group and $A,B,C$ be $G$-modules with a $G$-module homomorphism $\varphi\colon B\to C$. Then the diagram
		% https://q.uiver.app/?q=WzAsNCxbMCwwLCJBXFxvdGltZXNfXFxaWlxcb3B7SG9tfShBLEIpIl0sWzEsMCwiQVxcb3RpbWVzX1xcWlooQSxDKSJdLFswLDEsIkIiXSxbMSwxLCJDIl0sWzAsMSwiXFx2YXJwaGkiXSxbMiwzLCJcXHZhcnBoaSJdLFswLDJdLFsxLDNdXQ==&macro_url=https%3A%2F%2Fraw.githubusercontent.com%2FdFoiler%2Fnotes%2Fmaster%2Fnir.tex
		\[\begin{tikzcd}
			{A\otimes_\ZZ\op{Hom}_\ZZ(A,B)} & {A\otimes_\ZZ\op{Hom}_\ZZ(A,C)} \\
			B & C
			\arrow["\varphi", from=1-1, to=1-2]
			\arrow["\varphi", from=2-1, to=2-2]
			\arrow[from=1-1, to=2-1]
			\arrow[from=1-2, to=2-2]
		\end{tikzcd}\]
		commutes, where the vertical homomorphisms are evaluation.
	\end{lemma}
	\begin{proof}
		We simply pick up some $a\otimes f\in A\otimes_\ZZ\op{Hom}_\ZZ(A,B)$ and track through
		% https://q.uiver.app/?q=WzAsNCxbMCwwLCJhXFxvdGltZXMgZiJdLFsxLDAsImFcXG90aW1lc1xcdmFycGhpXFxjaXJjIGYiXSxbMCwxLCJmKGEpIl0sWzEsMSwiXFx2YXJwaGkoZihhKSkiXSxbMCwxLCJcXHZhcnBoaSIsMCx7InN0eWxlIjp7InRhaWwiOnsibmFtZSI6Im1hcHMgdG8ifX19XSxbMiwzLCJcXHZhcnBoaSIsMCx7InN0eWxlIjp7InRhaWwiOnsibmFtZSI6Im1hcHMgdG8ifX19XSxbMCwyLCIiLDEseyJzdHlsZSI6eyJ0YWlsIjp7Im5hbWUiOiJtYXBzIHRvIn19fV0sWzEsMywiIiwxLHsic3R5bGUiOnsidGFpbCI6eyJuYW1lIjoibWFwcyB0byJ9fX1dXQ==&macro_url=https%3A%2F%2Fraw.githubusercontent.com%2FdFoiler%2Fnotes%2Fmaster%2Fnir.tex
		\[\begin{tikzcd}
			{a\otimes f} & {a\otimes\varphi\circ f} \\
			{f(a)} & {\varphi(f(a))}
			\arrow["\varphi", maps to, from=1-1, to=1-2]
			\arrow["\varphi", maps to, from=2-1, to=2-2]
			\arrow[maps to, from=1-1, to=2-1]
			\arrow[maps to, from=1-2, to=2-2]
		\end{tikzcd}\]
		which finishes the proof.
	\end{proof}
	Now, \autoref{prop:ceduality} tells us that
	\[\arraycolsep=1.4pt\begin{array}{ccc}
		\widehat H^{i-1}(G,\op{Hom}_\ZZ(X,\QQ/\ZZ)) &\to& \op{Hom}_\ZZ\left(\widehat H^{-i}(G,X),\widehat H^{-1}(G,\QQ/\ZZ)\right) \\
		a &\mapsto& (b\mapsto\eta_{\QQ/\ZZ}(a\cup b))
	\end{array}\]
	is an isomorphism. Composing this with various other isomorphisms, we can build the isomorphism
	\[\arraycolsep=1.4pt\begin{array}{ccccccc}
		\widehat H^i(G,X_*) &\to& \widehat H^{i-1}(G,X^*) &\to& \op{Hom}\left(\widehat H^{-i}(G,X),\widehat H^{-1}(G,\QQ/\ZZ)\right) &\to& \op{Hom}\left(\widehat H^{-i}(G,X),\widehat H^0(G,\QQ/\ZZ)\right)  \\
		a &\mapsto& \delta_h^{-1}a &\mapsto& \left(b\mapsto\eta_{\QQ/\ZZ}(\delta_h^{-1}a\cup b)\right) &\mapsto& \left(b\mapsto\delta\eta_{\QQ/\ZZ}(\delta_h^{-1}a\cup b)\right)
	\end{array}\]
	where $X_*\coloneqq\op{Hom}_\ZZ(X,\ZZ)$ and $X^*\coloneqq\op{Hom}_\ZZ(X,\QQ/\ZZ)$, for brevity. This gives an isomorphism between the desired objects, but to prove the result we need to show that the above map is $a\mapsto(b\mapsto\eta_\ZZ(a\cup b))$. Well, given $a\in\widehat H^i(G,\op{Hom}_\ZZ(X,\ZZ))$ and $b\in\widehat H^{-i}(G,X)$, properties of the boundary morphisms tells us
	\begin{align*}
		\delta\eta_{\QQ/\ZZ}\left(\delta_h^{-1}a\cup b\right) &= \eta_\ZZ\delta_t\left(\delta_h^{-1}a\cup b\right) \\
		&= \eta_\ZZ\left(\delta_h\delta_h^{-1}a\cup b\right) \\
		&= \eta_\ZZ(a\cup b),
	\end{align*}
	which is what we wanted.
\end{proof}
\begin{remark}
	The hypothesis that $X$ be $\ZZ$-free is necessary: the statement is false for $X=\ZZ/\#G\ZZ$ and $i=0$, for example.
\end{remark}
We close this subsection with a dimension-shifting result.
\begin{lemma} \label{lem:hompreservesinduced}
	Let $G$ be a finite group and $X$ a $G$-module. If $M$ is an induced $G$-module, then $\op{Hom}_\ZZ(X,M)$ is also an induced $G$-module.
\end{lemma}
\begin{proof}
	By definition, we can write $M\coloneqq\op{Hom}_\ZZ(\ZZ[G],A)$ for some $G$-module $A$, where $A$ has perhaps trivial $G$-action. Now, we claim that
	\[\arraycolsep=1.4pt\begin{array}{cccc}
		\varphi\colon& \op{Hom}_\ZZ(X,\op{Hom}_\ZZ(\ZZ[G],A)) &\simeq& \op{Hom}_\ZZ(\ZZ[G],\op{Hom}_\ZZ(X,A)) \\
		\varphi\colon& f &\mapsto& \big(z\mapsto(x\mapsto f(x)(z))\big)
	\end{array}\]
	is an isomorphism of $G$-modules. This will finish because the right-hand $G$-module is induced.
	
	Now, $\varphi$ s a homomorphism of abelian groups because
	\[\varphi(f+f')(z)(x)=(f+g)(x)(z)=\varphi(f)(z)(x)+\varphi(f')(z)(x)\]
	for any $x$ and $z$ and $f,f'\in\op{Hom}_\ZZ(X,\op{Hom}_\ZZ(\ZZ[G],A))$. This is a $G$-module homomorphism because any $g\in G$ and $f\in\op{Hom}_\ZZ(X,\op{Hom}_\ZZ(\ZZ[G],A))$ has
	\begin{align*}
		\varphi(gf)(z)(x) &= \big(g\cdot\varphi(f)(g^{-1}z)\big)(x) \\
		&= g\cdot\varphi(f)(g^{-1}z)(g^{-1}x) \\
		&= g\cdot f(g^{-1}x)(g^{-1}z) \\
		&= \big(g\cdot f(g^{-1}x)\big)(z) \\
		&= (gf)(x)(z) \\
		&= \varphi(gf)(x)(z)
	\end{align*}
	for each $x$ and $z$.

	Now, we define
	\[\arraycolsep=1.4pt\begin{array}{cccc}
		\psi\colon& \op{Hom}_\ZZ(\ZZ[G],\op{Hom}_\ZZ(X,A)) &\simeq& \op{Hom}_\ZZ(X,\op{Hom}_\ZZ(\ZZ[G],A)) \\
		\psi\colon& f &\mapsto& \big(x\mapsto(z\mapsto f(z)(x))\big)
	\end{array}\]
	to be the inverse morphism. The exact same checks show that this is a $G$-module homomorphism, and it is not hard to see that
	\[\varphi\psi(f)(z)(x)=\psi(f)(z)(x)=f(x)(z),\]
	so $\varphi\circ\psi$ is the identity; similarly, $\psi\circ\varphi$ is the identity.
\end{proof}
\begin{proposition} \label{prop:dimshiftcupisos}
	Let $G$ be a finite group and $X$ a $G$-module. Further, suppose that we have indices $p,q\in\ZZ$ and $c\in H^p(G,X)$ such that the cup-product map
	\[c\cup-\colon\widehat H^q(G,\op{Hom}_\ZZ(X,A))\to\widehat H^{p+q}(G,A)\]
	is an isomorphism for all $G$-modules $A$. Then the cup-product map
	\[c\cup-\colon\widehat H^j(G,\op{Hom}_\ZZ(X,A))\to\widehat H^{p+j}(G,A)\]
	is an isomorphism for all $G$-modules $A$ and indices $j\in\ZZ$.
\end{proposition}
\begin{proof}
	We merely have to shift $q$ up and down. To shift downwards, we suppose that the cup-product map is always an isomorphism for $j$, and we show that it is always an isomorphism $j-1$. Namely, fix a $G$-module $A$, and we are interested in showing that the cup-product map
	\[c\cup-\colon\widehat H^{j-1}(G,\op{Hom}_\ZZ(X,Z))\to\widehat H^{p+j-1}(G,A)\]
	is an isomorphism. To do so, we note the short exact sequence
	\begin{equation}
		0\to I_G\to\ZZ[G]\to\ZZ\to0 \label{eq:shiftingses}
	\end{equation}
	which splits over $\ZZ$ and thus gives us the short exact sequences
	% https://q.uiver.app/?q=WzAsMTUsWzAsMCwiMCJdLFsxLDAsIlxcb3B7SG9tfV9cXFpaKFgsSV9HXFxvdGltZXNfXFxaWiBBKSJdLFsyLDAsIlxcb3B7SG9tfV9cXFpaKFgsXFxaWltHXVxcb3RpbWVzX1xcWlogQSkiXSxbMywwLCJcXG9we0hvbX1fXFxaWihYLEEpIl0sWzQsMCwiMCJdLFswLDEsIjAiXSxbNCwxLCIwIl0sWzEsMSwiWFxcb3RpbWVzX1xcWlpcXG9we0hvbX1fXFxaWihYLElfR1xcb3RpbWVzX1xcWlogQSkiXSxbMiwxLCJYXFxvdGltZXNfXFxaWlxcb3B7SG9tfV9cXFpaKFgsXFxaWltHXVxcb3RpbWVzX1xcWlogQSkiXSxbMywxLCJYXFxvdGltZXNfXFxaWlxcb3B7SG9tfV9cXFpaKFgsQSkiXSxbMCwyLCIwIl0sWzEsMiwiSV9HXFxvdGltZXNfXFxaWiBBIl0sWzIsMiwiXFxaWltHXVxcb3RpbWVzX1xcWlogQSJdLFszLDIsIkEiXSxbNCwyLCIwIl0sWzAsMV0sWzEsMl0sWzIsM10sWzMsNF0sWzUsN10sWzcsOF0sWzgsOV0sWzksNl0sWzEwLDExXSxbMTEsMTJdLFsxMiwxM10sWzEzLDE0XSxbNywxMSwiXFxldGFfe0lfR30iLDJdLFs4LDEyLCJcXGV0YV97XFxaWltHXX0iLDJdLFs5LDEzLCJcXGV0YV9BIiwyXV0=&macro_url=https%3A%2F%2Fraw.githubusercontent.com%2FdFoiler%2Fnotes%2Fmaster%2Fnir.tex
	\[\begin{tikzcd}
		0 & {\op{Hom}_\ZZ(X,I_G\otimes_\ZZ A)} & {\op{Hom}_\ZZ(X,\ZZ[G]\otimes_\ZZ A)} & {\op{Hom}_\ZZ(X,A)} & 0 \\
		0 & {X\otimes_\ZZ\op{Hom}_\ZZ(X,I_G\otimes_\ZZ A)} & {X\otimes_\ZZ\op{Hom}_\ZZ(X,\ZZ[G]\otimes_\ZZ A)} & {X\otimes_\ZZ\op{Hom}_\ZZ(X,A)} & 0 \\
		0 & {I_G\otimes_\ZZ A} & {\ZZ[G]\otimes_\ZZ A} & A & 0
		\arrow[from=1-1, to=1-2]
		\arrow[from=1-2, to=1-3]
		\arrow[from=1-3, to=1-4]
		\arrow[from=1-4, to=1-5]
		\arrow[from=2-1, to=2-2]
		\arrow[from=2-2, to=2-3]
		\arrow[from=2-3, to=2-4]
		\arrow[from=2-4, to=2-5]
		\arrow[from=3-1, to=3-2]
		\arrow[from=3-2, to=3-3]
		\arrow[from=3-3, to=3-4]
		\arrow[from=3-4, to=3-5]
		\arrow["{\eta_{I_G}}"', from=2-2, to=3-2]
		\arrow["{\eta_{\ZZ[G]}}"', from=2-3, to=3-3]
		\arrow["{\eta_A}"', from=2-4, to=3-4]
	\end{tikzcd}\]
	where the $\eta_\bullet$s are evaluation maps; in particular, the bottom two rows commute by \autoref{lem:evcommutes} and thus give a morphism of short exact sequences. These short exact sequences give us boundary morphisms
	\[\arraycolsep=1.4pt\begin{array}{rlcl}
		\delta\colon& \widehat H^{p+j-1}(G,A) &\to& \widehat H^{p+j}(G,I_G\otimes_\ZZ A) \\
		\delta_h\colon& \widehat H^{j-1}(G,\op{Hom}_\ZZ(X,A)) &\to& \widehat H^j(G,\op{Hom}_\ZZ(X,I_G\otimes_\ZZ A)) \\
		\delta_t\colon& \widehat H^{p+j-1}(G,X\otimes_\ZZ\op{Hom}_\ZZ(X,A)) &\to& \widehat H^{p+j}(G,X\otimes_\ZZ\op{Hom}_\ZZ(X,I_G\otimes_\ZZ A)).
	\end{array}\]
	Notably, $\delta$ is an isomorphism because $\ZZ[G]\otimes_\ZZ A$ is induced; from this it follows that $\op{Hom}_\ZZ(X,\ZZ[G]\otimes_\ZZ A)$ is also induced by \autoref{lem:hompreservesinduced}, implying that $\delta_h$ is also an isomorphism.
	
	Now, the key to this dimension-shifting is claiming that the diagram
	% https://q.uiver.app/?q=WzAsNCxbMCwwLCJcXHdpZGVoYXQgSF57ai0xfShHLFxcb3B7SG9tfV9cXFpaKFgsQSkpIl0sWzEsMCwiXFx3aWRlaGF0IEhee3Arai0xfShHLEEpIl0sWzAsMSwiXFx3aWRlaGF0IEhee2p9KEcsXFxvcHtIb219X1xcWlooWCxJX0dcXG90aW1lc19cXFpaIEEpKSJdLFsxLDEsIlxcd2lkZWhhdCBIXntwK2p9KEcsSV9HXFxvdGltZXNfXFxaWiBBKSJdLFswLDEsImNcXGN1cC0iXSxbMiwzLCJjXFxjdXAtIl0sWzAsMiwiXFxkZWx0YV9oIiwyXSxbMSwzLCJcXGRlbHRhIiwyXV0=&macro_url=https%3A%2F%2Fraw.githubusercontent.com%2FdFoiler%2Fnotes%2Fmaster%2Fnir.tex
	\[\begin{tikzcd}
		{\widehat H^{j-1}(G,\op{Hom}_\ZZ(X,A))} & {\widehat H^{p+j-1}(G,A)} \\
		{\widehat H^{j}(G,\op{Hom}_\ZZ(X,I_G\otimes_\ZZ A))} & {\widehat H^{p+j}(G,I_G\otimes_\ZZ A)}
		\arrow["{c\cup-}", from=1-1, to=1-2]
		\arrow["{c\cup-}", from=2-1, to=2-2]
		\arrow["{\delta_h}"', from=1-1, to=2-1]
		\arrow["(-1)^p\delta"', from=1-2, to=2-2]
	\end{tikzcd}\]
	commutes. Indeed, this will be enough because the bottom row is an isomorphism by the inductive hypothesis, and the left and morphisms are isomorphisms as discussed above, which makes the top row into an isomorphism. Well, to see that the diagram commutes, we expand the diagram as follows.
	% https://q.uiver.app/?q=WzAsNixbMCwwLCJcXHdpZGVoYXQgSF57ai0xfShHLFxcb3B7SG9tfV9cXFpaKFgsQSkpIl0sWzEsMCwiXFx3aWRlaGF0IEhee3Aran0oRyxYXFxvdGltZXNfXFxaWlxcb3B7SG9tfV9cXFpaKEEpKSJdLFswLDEsIlxcd2lkZWhhdCBIXntqfShHLFxcb3B7SG9tfV9cXFpaKFgsSV9HXFxvdGltZXNfXFxaWiBBKSkiXSxbMSwxLCJcXHdpZGVoYXQgSF57cCtqfShHLFhcXG90aW1lc19cXFpaXFxvcHtIb219X1xcWlooSV9HXFxvdGltZXNfXFxaWiBBKSkiXSxbMiwwLCJcXHdpZGVoYXQgSF57cCtqLTF9KEcsQSkiXSxbMiwxLCJcXHdpZGVoYXQgSF57cCtqfShHLElfR1xcb3RpbWVzX1xcWlogQSkiXSxbMCwxLCJjXFxjdXAtIl0sWzIsMywiY1xcY3VwLSJdLFswLDIsIlxcZGVsdGFfaCIsMl0sWzEsMywiXFxkZWx0YV90IiwyXSxbMSw0LCJcXGV0YV9BIl0sWzMsNSwiXFxldGFfe0lfR30iXSxbNCw1LCJcXGRlbHRhIiwyXV0=&macro_url=https%3A%2F%2Fraw.githubusercontent.com%2FdFoiler%2Fnotes%2Fmaster%2Fnir.tex
	\[\begin{tikzcd}
		{\widehat H^{j-1}(G,\op{Hom}_\ZZ(X,A))} & {\widehat H^{p+j-1}(G,X\otimes_\ZZ\op{Hom}_\ZZ(X,A))} & {\widehat H^{p+j-1}(G,A)} \\
		{\widehat H^{j}(G,\op{Hom}_\ZZ(X,I_G\otimes_\ZZ A))} & {\widehat H^{p+j}(G,X\otimes_\ZZ\op{Hom}_\ZZ(X.I_G\otimes_\ZZ A))} & {\widehat H^{p+j}(G,I_G\otimes_\ZZ A)}
		\arrow["{c\cup-}", from=1-1, to=1-2]
		\arrow["{c\cup-}", from=2-1, to=2-2]
		\arrow["{\delta_h}"', from=1-1, to=2-1]
		\arrow["{(-1)^p\delta_t}"', from=1-2, to=2-2]
		\arrow["{\eta_A}", from=1-2, to=1-3]
		\arrow["{\eta_{I_G}}", from=2-2, to=2-3]
		\arrow["(-1)^p\delta"', from=1-3, to=2-3]
	\end{tikzcd}\]
	The left square commutes because cup products commute with boundary morphisms; the right square commutes by functoriality of boundary morphisms.

	Shifting upwards is similar. Suppose that the cup-product in question is always an isomorphism for $j$, and we show that it is always an isomorphism for $j+1$. Namely, fix a $G$-module $A$, and we are interested in showing that the cup-product map
	\[c\cup-\colon\widehat H^{j+1}(G,\op{Hom}_\ZZ(X,A))\to\widehat H^{p+j+1}(G,A)\]
	is an isomorphism. As before, we use \autoref{eq:shiftingses} to induce the short exact sequences
	% https://q.uiver.app/?q=WzAsMTUsWzAsMCwiMCJdLFsxLDAsIlxcb3B7SG9tfV9cXFpaKFgsQSkiXSxbMiwwLCJcXG9we0hvbX1fXFxaWihYLFxcb3B7SG9tfV9cXFpaKFxcWlpbR10sQSkpIl0sWzMsMCwiXFxvcHtIb219X1xcWlooWCxcXG9we0hvbX1fXFxaWihJX0csQSkpIl0sWzQsMCwiMCJdLFswLDEsIjAiXSxbNCwxLCIwIl0sWzEsMSwiWFxcb3RpbWVzX1xcWlpcXG9we0hvbX1fXFxaWihYLEEpIl0sWzIsMSwiWFxcb3RpbWVzX1xcWlpcXG9we0hvbX1fXFxaWihYLFxcb3B7SG9tfV9cXFpaKFxcWlpbR10sQSkpIl0sWzMsMSwiWFxcb3RpbWVzX1xcWlpcXG9we0hvbX1fXFxaWihYLFxcb3B7SG9tfV9cXFpaKElfRyxBKSkiXSxbMCwyLCIwIl0sWzEsMiwiQSJdLFsyLDIsIlxcb3B7SG9tfV9cXFpaKFxcWlpbR10sQSkiXSxbMywyLCJcXG9we0hvbX1fXFxaWihJX0csQSkiXSxbNCwyLCIwIl0sWzAsMV0sWzEsMl0sWzIsM10sWzMsNF0sWzUsN10sWzcsOF0sWzgsOV0sWzksNl0sWzEwLDExXSxbMTEsMTJdLFsxMiwxM10sWzEzLDE0XSxbNywxMSwiXFxldGFfQSIsMl0sWzgsMTIsIlxcZXRhX3tcXFpaW0ddfSIsMl0sWzksMTMsIlxcZXRhX3tJX0d9IiwyXV0=&macro_url=https%3A%2F%2Fraw.githubusercontent.com%2FdFoiler%2Fnotes%2Fmaster%2Fnir.tex
	\[\begin{tikzcd}[column sep=12pt]
		0 & {\op{Hom}_\ZZ(X,A)} & {\op{Hom}_\ZZ(X,\op{Hom}_\ZZ(\ZZ[G],A))} & {\op{Hom}_\ZZ(X,\op{Hom}_\ZZ(I_G,A))} & 0 \\
		0 & {X\otimes_\ZZ\op{Hom}_\ZZ(X,A)} & {X\otimes_\ZZ\op{Hom}_\ZZ(X,\op{Hom}_\ZZ(\ZZ[G],A))} & {X\otimes_\ZZ\op{Hom}_\ZZ(X,\op{Hom}_\ZZ(I_G,A))} & 0 \\
		0 & A & {\op{Hom}_\ZZ(\ZZ[G],A)} & {\op{Hom}_\ZZ(I_G,A)} & 0
		\arrow[from=1-1, to=1-2]
		\arrow[from=1-2, to=1-3]
		\arrow[from=1-3, to=1-4]
		\arrow[from=1-4, to=1-5]
		\arrow[from=2-1, to=2-2]
		\arrow[from=2-2, to=2-3]
		\arrow[from=2-3, to=2-4]
		\arrow[from=2-4, to=2-5]
		\arrow[from=3-1, to=3-2]
		\arrow[from=3-2, to=3-3]
		\arrow[from=3-3, to=3-4]
		\arrow[from=3-4, to=3-5]
		\arrow["{\eta_A}"', from=2-2, to=3-2]
		\arrow["{\eta_{\ZZ[G]}}"', from=2-3, to=3-3]
		\arrow["{\eta_{I_G}}"', from=2-4, to=3-4]
	\end{tikzcd}\]
	where the $\eta_\bullet$s are (renamed) evaluation maps. Again, the bottom rows commute by \autoref{lem:evcommutes} and hence given a morphism of short exact sequences. As before, we have the boundary morphisms
	\[\arraycolsep=1.4pt\begin{array}{rlcl}
		\delta\colon& \widehat H^{p+j}(G,\op{Hom}_\ZZ(I_G,A)) &\to& \widehat H^{p+j+1}(G,A) \\
		\delta_h\colon& \widehat H^{j}(G,\op{Hom}_\ZZ(X,\op{Hom}_\ZZ(I_G,A))) &\to& \widehat H^{j+1}(G,\op{Hom}_\ZZ(X,A)) \\
		\delta_t\colon& \widehat H^{p+j}(G,X\otimes_\ZZ\op{Hom}_\ZZ(X,\op{Hom}_\ZZ(I_G,A))) &\to& \widehat H^{p+j+1}(G,X\otimes_\ZZ\op{Hom}_\ZZ(X,A)).
	\end{array}\]
	We again note that $\delta$ is an isomorphism because $\op{Hom}_\ZZ(\ZZ[G],A)$ is an induced module; thus, \autoref{lem:hompreservesinduced} tells us that $\op{Hom}_\ZZ(X,\op{Hom}_\ZZ(\ZZ[G],A))$ is also induced, making $\delta_h$ an isomorphism as well.

	Once more, the key to the dimension-shifting will be the claim that the diagram
	% https://q.uiver.app/?q=WzAsNCxbMCwwLCJcXHdpZGVoYXQgSF57an0oRyxcXG9we0hvbX1fXFxaWihYLFxcb3B7SG9tfV9cXFpaKElfRyxBKSkpIl0sWzAsMSwiXFx3aWRlaGF0IEhee2orMX0oRyxcXG9we0hvbX1fXFxaWihYLEEpKSJdLFsxLDAsIlxcd2lkZWhhdCBIXntwK2p9KEcsXFxvcHtIb219X1xcWlooSV9HLEEpKSJdLFsxLDEsIlxcd2lkZWhhdCBIXntwK2orMX0oRyxBKSJdLFswLDEsIlxcZGVsdGFfaCIsMl0sWzIsMywiXFxkZWx0YSIsMl0sWzAsMiwiY1xcY3VwLSJdLFsxLDMsImNcXGN1cC0iXV0=&macro_url=https%3A%2F%2Fraw.githubusercontent.com%2FdFoiler%2Fnotes%2Fmaster%2Fnir.tex
	\[\begin{tikzcd}
		{\widehat H^{j}(G,\op{Hom}_\ZZ(X,\op{Hom}_\ZZ(I_G,A)))} & {\widehat H^{p+j}(G,\op{Hom}_\ZZ(I_G,A))} \\
		{\widehat H^{j+1}(G,\op{Hom}_\ZZ(X,A))} & {\widehat H^{p+j+1}(G,A)}
		\arrow["{\delta_h}"', from=1-1, to=2-1]
		\arrow["(-1)^p\delta"', from=1-2, to=2-2]
		\arrow["{c\cup-}", from=1-1, to=1-2]
		\arrow["{c\cup-}", from=2-1, to=2-2]
	\end{tikzcd}\]
	commutes. This will be enough because the top arrow is an isomorphism by the inductive hypothesis, and the left and right arrows are isomorphisms as discussed above, thus making the bottom arrow also an isomorphism. Now, to see that the diagram commutes, we expand out our cup products as follows.
	% https://q.uiver.app/?q=WzAsNixbMCwwLCJcXHdpZGVoYXQgSF57an0oRyxcXG9we0hvbX1fXFxaWihYLFxcb3B7SG9tfV9cXFpaKElfRyxBKSkpIl0sWzAsMSwiXFx3aWRlaGF0IEhee2orMX0oRyxcXG9we0hvbX1fXFxaWihYLEEpKSJdLFsxLDAsIlxcd2lkZWhhdCBIXntwK2p9KEcsWFxcb3RpbWVzX1xcWlpcXG9we0hvbX1fXFxaWihJX0csQSkpIl0sWzEsMSwiXFx3aWRlaGF0IEhee3AraisxfShHLFhcXG90aW1lc19cXFpaXFxvcHtIb219X1xcWlooWCxBKSkiXSxbMiwxLCJcXHdpZGVoYXQgSF57cCtqKzF9KEcsQSkiXSxbMiwwLCJcXHdpZGVoYXQgSF57cCtqfShHLFxcb3B7SG9tfV9cXFpaKElfRyxBKSkiXSxbMCwxLCJcXGRlbHRhX2giLDJdLFsyLDMsIigtMSlecFxcZGVsdGFfdCIsMl0sWzAsMiwiY1xcY3VwLSJdLFsxLDMsImNcXGN1cC0iXSxbMiw1LCJcXGV0YV97SV9HfSJdLFszLDQsIlxcZXRhX0EiXSxbNSw0LCIoLTEpXnBcXGRlbHRhIiwyXV0=&macro_url=https%3A%2F%2Fraw.githubusercontent.com%2FdFoiler%2Fnotes%2Fmaster%2Fnir.tex
	\[\begin{tikzcd}
		{\widehat H^{j}(G,\op{Hom}_\ZZ(X,\op{Hom}_\ZZ(I_G,A)))} & {\widehat H^{p+j}(G,X\otimes_\ZZ\op{Hom}_\ZZ(I_G,A))} & {\widehat H^{p+j}(G,\op{Hom}_\ZZ(I_G,A))} \\
		{\widehat H^{j+1}(G,\op{Hom}_\ZZ(X,A))} & {\widehat H^{p+j+1}(G,X\otimes_\ZZ\op{Hom}_\ZZ(X,A))} & {\widehat H^{p+j+1}(G,A)}
		\arrow["{\delta_h}"', from=1-1, to=2-1]
		\arrow["{(-1)^p\delta_t}"', from=1-2, to=2-2]
		\arrow["{c\cup-}", from=1-1, to=1-2]
		\arrow["{c\cup-}", from=2-1, to=2-2]
		\arrow["{\eta_{I_G}}", from=1-2, to=1-3]
		\arrow["{\eta_A}", from=2-2, to=2-3]
		\arrow["{(-1)^p\delta}"', from=1-3, to=2-3]
	\end{tikzcd}\]
	The left square commutes because cup products commute with boundary morphisms, and the right square commutes by functoriality of boundary morphisms. This finishes.
\end{proof}
And here are some nice corollaries, tying back into our theory.
\begin{cor} \label{cor:xhasallcupisos}
	Fix notation as in \autoref{sec:overview}. Then, for any $G$-module $A$ and index $i\in\ZZ$, the cup-product map
	\[[\delta(\overline c)]\cup-\colon\widehat H^i(G,\op{Hom}_\ZZ(X,A))\to\widehat H^{i+2}(G,A)\]
	is an isomorphism.
\end{cor}
\begin{proof}
	Set $p=2$ and $q=0$ and $c$ to $[\delta(\overline c)]$ in \autoref{prop:dimshiftcupisos}; the hypothesis is satisfied by combining the cup-product map of \autoref{thm:yesitisacocycle} with \autoref{prop:alternativetupleclass}. (Namely, the cup-product map is sending an equivalence class of tuples to the corresponding cohomology class, which is an isomorphism by \autoref{thm:classisomorphism}.) Anyway, \autoref{prop:dimshiftcupisos} does indeed give the result.
\end{proof}
\begin{cor}
	Fix notation as in \autoref{sec:overview}. Then $\widehat H^2(G,X)\simeq\ZZ/\#G\ZZ$, generated by $[\delta(\overline c)]$.
\end{cor}
\begin{proof}
	For brevity, set $n\coloneqq\#G$. By \autoref{cor:xhasallcupisos}, we have the isomorphism
	\[[\delta(\overline c)]\cup-\colon\widehat H^{-2}(G,\op{Hom}_\ZZ(X,\ZZ))\to\widehat H^0(G,\ZZ)=\ZZ/n\ZZ.\]
	In particular, $\widehat H^{-2}(G,\op{Hom}_\ZZ(X,\ZZ))\simeq\ZZ/n\ZZ$, generated by some element $[\delta(\overline c)]^\lor$ such that $[\delta(\overline c)]\cup[\delta(\overline c)]^\lor=[1]$.

	Now, note that the embedding $\mathcal F\colon X\into\ZZ[G]^m$ implies that $X$ is $\ZZ$-free, so we may apply \autoref{prop:abstractintegralduality} to say that the cup-product pairing induces an isomorphism
	\[\ZZ/n\ZZ\simeq\widehat H^{-2}(G,\op{Hom}_\ZZ(X,\ZZ))\to\op{Hom}_\ZZ\left(\widehat H^2(G,X),\widehat H^0(G,\ZZ)\right)\simeq\op{Hom}_\ZZ\left(\widehat H^2(G,X),\textstyle\frac1n\ZZ/\ZZ\right).\]
	Because $\widehat H^2(G,X)$ is $n$-torsion, homomorphisms $\widehat H^2(G,X)\to\QQ/\ZZ$ must have image in $\frac1n\ZZ/\ZZ$, so in fact the rightmost group is the dual of $\widehat H^2(G,X)$. Because an abelian group is isomorphic to its dual, we see that $\widehat H^2(G,X)$ is in fact cyclic of order $n$.

	It remains to show that $[\delta(\overline c)]$ is a generator; for this, we show that $[\delta(\overline c)]$ has order at least $n$, which will be enough because $H^2(G,X)$ is cyclic of order $n$. Well, if $k[\delta(\overline c)]=0$, then
	\[[k]=k\big([\delta(\overline c)]\cup[\delta(\overline c)]^\lor\big)=k[\delta(\overline c)]\cup[\delta(\overline c)]^\lor=[0]\cup[\delta(\overline c)]^\lor=[0]\]
	in $\widehat H^0(G,\ZZ)$, so $n\mid k$. This finishes.
\end{proof}

\subsection{A Perfect Pairing}
The main goal of this subsection is to prove the following result.
\begin{theorem} \label{thm:abstractperfectpairing}
	Let $G$ be a finite group, and let $X$ and $A$ be $G$-modules. Then, if there exists an element $c\in H^2(G,X)$ such that the cup-product maps
	\begin{align*}
		c\cup-&\colon\widehat H^{-2}(G,\op{Hom}_\ZZ(X,\ZZ))\to\widehat H^0(G,\ZZ) \\
		c\cup-&\colon\widehat H^0(G,\op{Hom}_\ZZ(X,A))\to\widehat H^{2}(G,A)
	\end{align*}
	are isomorphisms, then the cup-product pairing induces an isomorphism
	\[\widehat H^2(G,A)\to\op{Hom}_\ZZ\left(\widehat H^{-2}(G,\op{Hom}_\ZZ(X,\ZZ)),\widehat H^0(G,\op{Hom}_\ZZ(X,A))\right).\]
\end{theorem}
The main step in the proof is the following lemma.
\begin{lemma}
	Let $G$ be a finite group, and let $X$ and $A$ be $G$-modules. Pick up another $G$-module $A$. Then, given any $i\in\ZZ$ and $c\in\widehat H^2(G,X)$ and $u\in\widehat H^2(G,A)$, the following diagram commutes, where all arrows are cup-product maps.
	% https://q.uiver.app/?q=WzAsNCxbMCwwLCJcXHdpZGVoYXQgSF57aS0yfShHLFxcb3B7SG9tfV9cXFpaKFgsXFxaWikpIl0sWzEsMCwiXFx3aWRlaGF0IEheaShHLFxcb3B7SG9tfV9cXFpaKFgsQSkpIl0sWzAsMSwiXFx3aWRlaGF0IEheaShHLFxcWlopIl0sWzEsMSwiXFx3aWRlaGF0IEhee2krMn0oRyxBKSJdLFswLDEsIi1cXGN1cCB1Il0sWzAsMiwiY1xcY3VwLSIsMl0sWzIsMywiLVxcY3VwIHUiLDJdLFsxLDMsImNcXGN1cC0iXV0=&macro_url=https%3A%2F%2Fraw.githubusercontent.com%2FdFoiler%2Fnotes%2Fmaster%2Fnir.tex
	\[\begin{tikzcd}
		{\widehat H^{i-2}(G,\op{Hom}_\ZZ(X,\ZZ))} & {\widehat H^i(G,\op{Hom}_\ZZ(X,A))} \\
		{\widehat H^i(G,\ZZ)} & {\widehat H^{i+2}(G,A)}
		\arrow["{-\cup u}", from=1-1, to=1-2]
		\arrow["{c\cup-}"', from=1-1, to=2-1]
		\arrow["{-\cup u}"', from=2-1, to=2-2]
		\arrow["{c\cup-}", from=1-2, to=2-2]
	\end{tikzcd}\]
\end{lemma}
\begin{proof}
	Formally, our cup-product maps are induced by the following ``evaluation morphisms.''
	\begin{itemize}
		\item For the left arrow, we have $\eta_L\colon X\otimes_\ZZ\op{Hom}_\ZZ(X,\ZZ)\to\ZZ$ by evaluation.
		\item For the top arrow, we have $\eta_T\colon\op{Hom}_\ZZ(X,\ZZ)\otimes_\ZZ A\to\op{Hom}_\ZZ(X,A)$ by $f\otimes a\mapsto(x\mapsto f(x)a)$.
		\item For the bottom arrow, we have $\eta_B\colon\ZZ\otimes_\ZZ A\to A$ by $k\otimes a\mapsto ka$.
		\item For the right arrow, we have $\eta_R\colon X\otimes_\ZZ\op{Hom}_\ZZ(X,A)\to A$ by evaluation.
	\end{itemize}
	In particular, these maps are defined so that the following diagram commutes.
	% https://q.uiver.app/?q=WzAsNCxbMCwwLCJYXFxvdGltZXNfXFxaWlxcb3B7SG9tfV9cXFpaKFgsXFxaWilcXG90aW1lc19cXFpaIEEiXSxbMSwwLCJYXFxvdGltZXNfXFxaWlxcb3B7SG9tfV9cXFpaKFgsQSkiXSxbMCwxLCJcXFpaXFxvdGltZXNfXFxaWiBBIl0sWzEsMSwiQSJdLFswLDEsIlxcZXRhX1QiXSxbMCwyLCJcXGV0YV9MIiwyXSxbMSwzLCJcXGV0YV9SIl0sWzIsMywiXFxldGFfQiIsMl1d&macro_url=https%3A%2F%2Fraw.githubusercontent.com%2FdFoiler%2Fnotes%2Fmaster%2Fnir.tex
	\begin{equation}
		\begin{tikzcd}
			{X\otimes_\ZZ\op{Hom}_\ZZ(X,\ZZ)\otimes_\ZZ A} & {X\otimes_\ZZ\op{Hom}_\ZZ(X,A)} \\
			{\ZZ\otimes_\ZZ A} & A
			\arrow["{\eta_T}", from=1-1, to=1-2]
			\arrow["{\eta_L}"', from=1-1, to=2-1]
			\arrow["{\eta_R}", from=1-2, to=2-2]
			\arrow["{\eta_B}"', from=2-1, to=2-2]
		\end{tikzcd} \label{eq:innermorphismcoherence}
	\end{equation}
	Indeed, we can just compute along the following diagram.
	% https://q.uiver.app/?q=WzAsNCxbMCwwLCJ4XFxvdGltZXMgZlxcb3RpbWVzIGEiXSxbMSwwLCJ4XFxvdGltZXMoeCdcXG1hcHN0byBmKHgnKWEpIl0sWzAsMSwiZih4KVxcb3RpbWVzIGEiXSxbMSwxLCJmKHgpYSJdLFswLDEsIlxcZXRhX1QiLDAseyJzdHlsZSI6eyJ0YWlsIjp7Im5hbWUiOiJtYXBzIHRvIn19fV0sWzAsMiwiXFxldGFfTCIsMix7InN0eWxlIjp7InRhaWwiOnsibmFtZSI6Im1hcHMgdG8ifX19XSxbMSwzLCJcXGV0YV9SIiwwLHsic3R5bGUiOnsidGFpbCI6eyJuYW1lIjoibWFwcyB0byJ9fX1dLFsyLDMsIlxcZXRhX0IiLDIseyJzdHlsZSI6eyJ0YWlsIjp7Im5hbWUiOiJtYXBzIHRvIn19fV1d&macro_url=https%3A%2F%2Fraw.githubusercontent.com%2FdFoiler%2Fnotes%2Fmaster%2Fnir.tex
	\[\begin{tikzcd}
		{x\otimes f\otimes a} & {x\otimes(x'\mapsto f(x')a)} \\
		{f(x)\otimes a} & {f(x)a}
		\arrow["{\eta_T}", maps to, from=1-1, to=1-2]
		\arrow["{\eta_L}"', maps to, from=1-1, to=2-1]
		\arrow["{\eta_R}", maps to, from=1-2, to=2-2]
		\arrow["{\eta_B}"', maps to, from=2-1, to=2-2]
	\end{tikzcd}\]
	Now, the core of the proof is in drawing the following very large diagram.
	% https://q.uiver.app/?q=WzAsOSxbMCwwLCJcXHdpZGVoYXQgSF57aS0yfShHLFxcb3B7SG9tfV9cXFpaKFgsXFxaWikpIl0sWzEsMCwiXFx3aWRlaGF0IEheaShHLFxcb3B7SG9tfV9cXFpaKFgsXFxaWilcXG90aW1lc19cXFpaIEEpIl0sWzIsMCwiXFx3aWRlaGF0IEheaShHLFxcb3B7SG9tfV9cXFpaKFgsQSkpIl0sWzAsMSwiXFx3aWRlaGF0IEheaShHLFhcXG90aW1lc19cXFpaXFxvcHtIb219X1xcWlooWCxcXFpaKSkiXSxbMSwxLCJcXHdpZGVoYXQgSF57aSsyfShHLFhcXG90aW1lc19cXFpaXFxvcHtIb219X1xcWlooWCxcXFpaKVxcb3RpbWVzX1xcWlogQSkiXSxbMiwxLCJcXHdpZGVoYXQgSF57aSsyfShHLFhcXG90aW1lc19cXFpaXFxvcHtIb219X1xcWlooWCxBKSkiXSxbMiwyLCJcXHdpZGVoYXQgSF57aSsyfShHLEEpIl0sWzAsMiwiXFx3aWRlaGF0IEheaShHLFxcWlopIl0sWzEsMiwiXFx3aWRlaGF0IEheMihHLFhcXG90aW1lc19cXFpaIEEpIl0sWzAsMSwiLVxcY3VwIHUiXSxbMyw0LCItXFxjdXAgdSJdLFs3LDgsIi1cXGN1cCB1Il0sWzAsMywiY1xcY3VwIC0iLDJdLFsxLDQsImNcXGN1cCAtIiwyXSxbMiw1LCJjXFxjdXAgLSIsMl0sWzEsMiwiXFxldGFfVCJdLFs0LDUsIlxcZXRhX1QiXSxbOCw2LCJcXGV0YV9CIl0sWzMsNywiXFxldGFfTCIsMl0sWzQsOCwiXFxldGFfTCIsMl0sWzUsNiwiXFxldGFfUiIsMl0sWzEyLDEzLCIoMSkiLDMseyJzaG9ydGVuIjp7InNvdXJjZSI6MjAsInRhcmdldCI6MjB9LCJzdHlsZSI6eyJib2R5Ijp7Im5hbWUiOiJub25lIn0sImhlYWQiOnsibmFtZSI6Im5vbmUifX19XSxbMTMsMTQsIigyKSIsMyx7InNob3J0ZW4iOnsic291cmNlIjoyMCwidGFyZ2V0IjoyMH0sInN0eWxlIjp7ImJvZHkiOnsibmFtZSI6Im5vbmUifSwiaGVhZCI6eyJuYW1lIjoibm9uZSJ9fX1dLFsxOCwxOSwiKDMpIiwzLHsic2hvcnRlbiI6eyJzb3VyY2UiOjIwLCJ0YXJnZXQiOjIwfSwic3R5bGUiOnsiYm9keSI6eyJuYW1lIjoibm9uZSJ9LCJoZWFkIjp7Im5hbWUiOiJub25lIn19fV0sWzE5LDIwLCIoNCkiLDMseyJzaG9ydGVuIjp7InNvdXJjZSI6MjAsInRhcmdldCI6MjB9LCJzdHlsZSI6eyJib2R5Ijp7Im5hbWUiOiJub25lIn0sImhlYWQiOnsibmFtZSI6Im5vbmUifX19XV0=&macro_url=https%3A%2F%2Fraw.githubusercontent.com%2FdFoiler%2Fnotes%2Fmaster%2Fnir.tex
	\[\begin{tikzcd}
		{\widehat H^{i-2}(G,\op{Hom}_\ZZ(X,\ZZ))} & {\widehat H^i(G,\op{Hom}_\ZZ(X,\ZZ)\otimes_\ZZ A)} & {\widehat H^i(G,\op{Hom}_\ZZ(X,A))} \\
		{\widehat H^i(G,X\otimes_\ZZ\op{Hom}_\ZZ(X,\ZZ))} & {\widehat H^{i+2}(G,X\otimes_\ZZ\op{Hom}_\ZZ(X,\ZZ)\otimes_\ZZ A)} & {\widehat H^{i+2}(G,X\otimes_\ZZ\op{Hom}_\ZZ(X,A))} \\
		{\widehat H^i(G,\ZZ)} & {\widehat H^{i+2}(G,X\otimes_\ZZ A)} & {\widehat H^{i+2}(G,A)}
		\arrow["{-\cup u}", from=1-1, to=1-2]
		\arrow["{-\cup u}", from=2-1, to=2-2]
		\arrow["{-\cup u}", from=3-1, to=3-2]
		\arrow[""{name=0, anchor=center, inner sep=0}, "{c\cup -}"', from=1-1, to=2-1]
		\arrow[""{name=1, anchor=center, inner sep=0}, "{c\cup -}"', from=1-2, to=2-2]
		\arrow[""{name=2, anchor=center, inner sep=0}, "{c\cup -}"', from=1-3, to=2-3]
		\arrow["{\eta_T}", from=1-2, to=1-3]
		\arrow["{\eta_T}", from=2-2, to=2-3]
		\arrow["{\eta_B}", from=3-2, to=3-3]
		\arrow[""{name=3, anchor=center, inner sep=0}, "{\eta_L}"', from=2-1, to=3-1]
		\arrow[""{name=4, anchor=center, inner sep=0}, "{\eta_L}"', from=2-2, to=3-2]
		\arrow[""{name=5, anchor=center, inner sep=0}, "{\eta_R}"', from=2-3, to=3-3]
		\arrow["{(1)}"{marking}, Rightarrow, draw=none, from=0, to=1]
		\arrow["{(2)}"{marking}, Rightarrow, draw=none, from=1, to=2]
		\arrow["{(3)}"{marking}, Rightarrow, draw=none, from=3, to=4]
		\arrow["{(4)}"{marking}, Rightarrow, draw=none, from=4, to=5]
	\end{tikzcd}\]
	We are being asked to show that the outer square commutes; we will show that each inner square commutes, which will be enough.
	\begin{enumerate}[label=(\arabic*)]
		\item This square commutes by the associativity of the cup product.
		\item This square commutes by \autoref{lem:cupproductmorphism}.
		\item This square commutes by \autoref{lem:cupproductmorphism}.
		\item This square commutes by functoriality of $\widehat H^{i+2}(G,-)$ applied to \autoref{eq:innermorphismcoherence}.
	\end{enumerate}
	The above checks complete the proof.
\end{proof}
We may now proceed directly with \autoref{thm:abstractperfectpairing}.
\begin{proof}[Proof of \autoref{thm:abstractperfectpairing}]
	We use the lemma to assert that, for any $u\in H^2(G,A)$, the diagram
	\[\begin{tikzcd}
		{\widehat H^{-2}(G,\op{Hom}_\ZZ(X,\ZZ))} & {\widehat H^0(G,\op{Hom}_\ZZ(X,A))} \\
		{\widehat H^0(G,\ZZ)} & {\widehat H^{2}(G,A)}
		\arrow["{-\cup u}", from=1-1, to=1-2]
		\arrow["{c\cup-}"', from=1-1, to=2-1]
		\arrow["{-\cup u}"', from=2-1, to=2-2]
		\arrow["{c\cup-}", from=1-2, to=2-2]
	\end{tikzcd}\]
	commutes. By hypothesis, the left and right arrows are isomorphisms, so the commutativity means that showing
	\[\arraycolsep=1.4pt\begin{array}{ccc}
		\widehat H^2(G,A) &\to& \op{Hom}_\ZZ\left(\widehat H^{-2}(G,\op{Hom}_\ZZ(X,\ZZ)),\widehat H^0(G,\op{Hom}_\ZZ(X,A))\right) \\
		u &\mapsto& (a\mapsto (a\cup u))
	\end{array}\]
	is an isomorphism is the same as showing that
	\[\arraycolsep=1.4pt\begin{array}{ccc}
		\widehat H^2(G,A) &\to& \op{Hom}_\ZZ\left(\widehat H^0(G,\ZZ),\widehat H^2(G,A)\right) \\
		u &\mapsto& (k\mapsto (k\cup u))
	\end{array}\]
	is an isomorphism. Setting $n\coloneqq\#G$, we see $\widehat H^0(G,\ZZ)=\ZZ/n\ZZ$, and the cup product we are looking at sends $k\in\ZZ/n\ZZ$ and $u\in\widehat H^2(G,A)$ to $k\cup u=ku$ by how the ``evaluation'' map $\ZZ\otimes_\ZZ A\simeq A$ behaves. Thus, we are showing that
	\[\arraycolsep=1.4pt\begin{array}{ccc}
		\widehat H^2(G,A) &\to& \op{Hom}_\ZZ\left(\ZZ/n\ZZ,\widehat H^2(G,A)\right) \\
		u &\mapsto& (k\mapsto ku)
	\end{array}\]
	is an isomorphism.
	
	However, $\widehat H^2(G,A)$ is $n$-torsion, so in fact maps $\ZZ\to\widehat H^2(G,A)$ automatically have $n\ZZ$ in their kernel and hence reduce to maps $\ZZ/n\ZZ\to\widehat H^2(G,A)$. Conversely, any map $\ZZ/n\ZZ\to\widehat H^2(G,A)$ can be extended by $\ZZ\onto\ZZ/n\ZZ$ to a map $\ZZ\to\widehat H^2(G,A)$, so we have a natural isomorphism
	\[\arraycolsep=1.4pt\begin{array}{ccc}
		\op{Hom}_\ZZ\left(\ZZ/n\ZZ,\widehat H^2(G,A)\right) &\simeq& \op{Hom}_\ZZ\left(\ZZ,\widehat H^2(G,A)\right) \\
		f &\mapsto& (k\mapsto f([k])) \\
		([k]\mapsto f(k)) &\mapsfrom& f.
	\end{array}\]
	In particular, it suffices to show that
	\[\arraycolsep=1.4pt\begin{array}{ccc}
		\widehat H^2(G,A) &\to& \op{Hom}_\ZZ\left(\ZZ,\widehat H^2(G,A)\right) \\
		u &\mapsto& (k\mapsto ku)
	\end{array}\]
	is an isomorphism. But this is a standard fact about the functor $\op{Hom}_\ZZ\colon\mathrm{AbGrp}\to\mathrm{AbGrp}$, so we are done.
\end{proof}
We now synthesize the theory we have been building.
\begin{cor}
	Fix notation as in \autoref{sec:overview}. Then, given a $G$-module $A$, the cup-product pairing induces an isomorphism
	\[\widehat H^2(G,A)\to\op{Hom}_\ZZ\left(\widehat H^{-2}(G,\op{Hom}_\ZZ(X,\ZZ)),\widehat H^0(G,\op{Hom}_\ZZ(X,A))\right).\]
\end{cor}
\begin{proof}
	We apply \autoref{thm:abstractperfectpairing} to our case; here $X$ is defined as in \autoref{sec:overview}, and we take $c$ to be $[\delta(\overline c)]$. Note $X$ is $\ZZ$-free because of the embedding $\mathcal F\colon X\into\ZZ[G]^m$. The cup-product maps in question are isomorphisms by \autoref{cor:xhasallcupisos}. Thus, \autoref{thm:abstractperfectpairing} kicks in, completing the proof.
\end{proof}

% \section{Torus Reciprocity}
% In this section we will apply the theory we have built to the specific case where $G$ is a Galois group of an extension of local fields $L/K$.

% In particular, we keep the notation as in \autoref{sec:tuplestudy} while asserting that $G=\op{Gal}(L/K)$ for some finite abelian extension of local fields $L/K$. Now, we note that the embedding
% \[X\stackrel{\mathcal F}\into\ZZ[G]^m\]
% tells us that $X$ embeds into a free abelian group and hence must be a free abelian group. In particular, because $X$ has a $G$-action---which extends to a $\op{Gal}(K^{\op{sep}}/K)$-action by taking quotients---we are promised a $K$-torus $T$ such that
% \[X^*(T)=X.\]
% By dualizing again, we see that $T\simeq\op{Hom}_\ZZ(X,\mathbb G_m-)$. As such, we just set $T\coloneqq\op{Hom}_\ZZ(X,\mathbb G_m-)$, which gives the following result.
% \begin{cor} \label{cor:torustupledescription}
% 	Fix notation as above. Then $H^0(L/K,T(L))$ is in natural bijection with $\{\sigma_i\}_{i=1}^m$-tuples, and $\widehat H^0(L/K,T(L))$ is in natural bijection with equivalence classes of $\{\sigma_i\}_{i=1}^m$-tuples.
% \end{cor}
% \begin{proof}
% 	Because $T(L)=\op{Hom}_\ZZ(X^*(T),L^\times)$, we may plug into \autoref{prop:alternativetuple} and \autoref{prop:alternativetupleclass}.
% \end{proof}
% To continue our discussion, we recall the following generalization of Artin reciprocity.
% \begin{theorem} \label{thm:torusreciprocity}
% 	Let $K$ be a local field and $T$ a $K$-torus. Suppose that $L/K$ is an extension of fields such that the base change $T_L$ is a split torus. Then cup product with the fundamental class $u_{L/K}\in H^2(L/K,L^\times)$ induces an isomorphism
% 	\[-\cup u_{L/K}\colon\widehat H^n(L/K,X_*(T))\to\widehat H^{n+2}(L/K,T(L))\]
% 	for all integers $n$.
% \end{theorem}
% \begin{proof}
% 	Omitted.
% \end{proof}
% Importantly, our torus $T$ is split over $L$ because (say) all characters in $X^*(T)$ are defined over $L$.\todo{I'm not sure if this makes sense.}

% In light of \autoref{cor:torustupledescription}, we are particularly interested in the case of $n=-2$ in \autoref{thm:torusreciprocity}, giving an isomorphism
% \[-\cup u_{L/K}\colon\widehat H^{-2}(L/K,X_*(T))\to\widehat H^0(L/K,T(L)).\]
% For example, we may use \autoref{thm:yesitisacocycle} to create the following diagram.
% % https://q.uiver.app/?q=WzAsMyxbMCwwLCJcXHdpZGVoYXQgSF57LTJ9KEwvSyxYXyooVCkpIl0sWzEsMCwiXFx3aWRlaGF0IEheMChML0ssVChMKSkiXSxbMSwxLCJcXHdpZGVoYXQgSF4yKEwvSyxMXlxcdGltZXMpIl0sWzAsMSwiLVxcY3VwIHVfe0wvS30iXSxbMSwyLCJcXGRlbHRhKFxcb3ZlcmxpbmUgYylcXGN1cC0iXV0=&macro_url=https%3A%2F%2Fraw.githubusercontent.com%2FdFoiler%2Fnotes%2Fmaster%2Fnir.tex
% \[\begin{tikzcd}
% 	{\widehat H^{-2}(L/K,X_*(T))} & {\widehat H^0(L/K,T(L))} \\
% 	& {\widehat H^2(L/K,L^\times)}
% 	\arrow["{-\cup u_{L/K}}", from=1-1, to=1-2]
% 	\arrow["{[\delta(\overline c)]\cup-}", from=1-2, to=2-2]
% \end{tikzcd}\]
% In particular, the vertical arrow is now an isomorphism because we know that equivalence classes of tuples uniquely correspond to cocycles from \autoref{thm:classisomorphism}.

% To complete the above suggestive diagram, we have the following lemma.
% \begin{lemma} \label{lem:torusdiagram}
% 	Let $L/K$ be an extension of local fields, and let $T$ be a $K$-torus which splits over $L$. Then, given arbitrary classes $u\in H^2(L/K,L^\times)$ and $c\in H^2(L/K,X^*(T))$, the following diagram commutes.
% 	% https://q.uiver.app/?q=WzAsNCxbMCwwLCJcXHdpZGVoYXQgSF57LTJ9KEwvSyxYXyooVCkpIl0sWzEsMCwiXFx3aWRlaGF0IEheMChML0ssVChMKSkiXSxbMSwxLCJcXHdpZGVoYXQgSF4yKEwvSyxMXlxcdGltZXMpIl0sWzAsMSwiXFx3aWRlaGF0IEheMChML0ssXFxaWikiXSxbMCwxLCItXFxjdXAgdSJdLFsxLDIsIlxcZGVsdGEoXFxvdmVybGluZSBjKVxcY3VwLSJdLFszLDIsIi1cXGN1cCB1IiwyXSxbMCwzLCJcXGRlbHRhKFxcb3ZlcmxpbmUgYylcXGN1cC0iLDJdXQ==&macro_url=https%3A%2F%2Fraw.githubusercontent.com%2FdFoiler%2Fnotes%2Fmaster%2Fnir.tex
% 	\[\begin{tikzcd}
% 		{\widehat H^{-2}(L/K,X_*(T))} & {\widehat H^0(L/K,T(L))} \\
% 		{\widehat H^0(L/K,\ZZ)} & {\widehat H^2(L/K,L^\times)}
% 		\arrow["{-\cup u}", from=1-1, to=1-2]
% 		\arrow["{c\cup-}", from=1-2, to=2-2]
% 		\arrow["{-\cup u}"', from=2-1, to=2-2]
% 		\arrow["{c\cup-}"', from=1-1, to=2-1]
% 	\end{tikzcd}\]
% \end{lemma}
% \begin{proof}
% 	At a high level, this should follow from the associativity of the cup product, but some care is required because the cup product maps are augmented in various ways throughout the diagram. For peace of mind, we will actually track through these maps.
	
% 	We use the standard resolution by homogeneous cochains; let $G\coloneqq\op{Gal}(L/K)$. We can represent a class $[x]\in\widehat H^{-2}(L/K,X_*(T))$ by an element $x\in\op{Hom}_G(P_{-2},X_*(T))$, where $P_{-2}\coloneqq\op{Hom}_\ZZ(P_1,\ZZ)=\op{Hom}_\ZZ(\ZZ[G]^2,\ZZ)$. To compute our cup products, we also need to define the notation
% 	\[(g_1^*,\ldots,g_q^*)\in P_{-q}\coloneqq\op{Hom}_\ZZ(P_{q-1},\ZZ)=\op{Hom}_\ZZ(\ZZ[G]^q,\ZZ)\]
% 	which behaves as the indicator function for $(g_1,\ldots,g_q)$ on $G^q$. We are now ready to track through our diagram
% 	\begin{itemize}
% 		\item Along the bottom, we start with $c\cup x\in\widehat H^0(L/K,X^*(T)\otimes_\ZZ X_*(T))$, which is
% 		\[(c\cup x)(g)=\sum_{s_1,s_2\in G}c(g,s_1,s_2)\otimes x(s_2^*,s_1^*).\]
% 		Now, passing through $X^*(T)\otimes_\ZZ X_*(T)\to\op{Hom}(\mathbb G_m,\mathbb G_m)$ by $(f\otimes g)\mapsto(f\circ g)$, we get the map
% 		\[(c\cup x)(g)=\prod_{s_1,s_2\in G}c(g,s_1,s_2)\circ x(s_2^*,s_1^*).\]
% 		In particular, the right-hand side is an algebraic map $L^\times\to L^\times$, which we know must take the form $z\mapsto z^{r_g}$ for some $r_g\in\ZZ$. The mapping $g\mapsto r_g$ is the $0$-cocycle in $\widehat H^0(L/K,\ZZ)$.

% 		Next we must compute $(c\cup x)\cup u\in\widehat H^2(L/K,\ZZ\otimes_\ZZ L^\times)$, which is
% 		\[((c\cup x)\cup u)(g_0,g_1,g_2)=r_{g_0}\otimes u(g_0,g_1,g_2).\]
% 		Passing through $\ZZ\otimes_\ZZ L^\times\to L^\times$ by $r\otimes z\mapsto z^r$, we see that we get
% 		\begin{align}
% 			((c\cup x)\cup u)(g_0,g_1,g_2) &= u(g_0,g_1,g_2)^{r_{g_0}} \notag \\
% 			&= \prod_{s_1,s_2\in G}\big(c(g_0,s_1,s_2)\circ x(s_2^*,s_1^*)\big)\big(u(g_0,g_1,g_2)\big), \label{eq:cupsleft}
% 		\end{align}
% 		where in the last equality we applied the definition of $r_{g_0}$.

% 		\item We now track along the top. Starting with $x\cup u\in\widehat H^0(L/K,X_*(T)\otimes_\ZZ L^\times)$, we have
% 		\[(x\cup u)(g)=\sum_{s_1,s_2\in G}x(s_1^*,s_2^*)\otimes u(s_2,s_1,g).\]
% 		Passing through $X_*(T)\otimes_\ZZ L^\times\to T(L)$ by $(f\otimes z)\mapsto f(z)$, we get
% 		\[(x\cup u)(g)=\prod_{s_1,s_2\in G}x(s_1^*,s_2^*)\big(u(s_2,s_1,g)\big).\]
% 		Continuing, we compute $(c\cup(x\cup u))\in\widehat H^2(L/K,X*(T)\otimes_\ZZ T(L))$ as
% 		\[(c\cup(x\cup u))(g_0,g_1,g_2)=c(g_0,g_1,g_2)\otimes\prod_{s_1,s_2\in G}x(s_1^*,s_2^*)\big(u(s_2,s_1,g_2)\big).\]
% 		And to finish, we pass through $X^*(T)\otimes_\ZZ T(L)\to L^\times$ by $(f\otimes z)\mapsto f(z)$, which gives
% 		\begin{equation}
% 			(c\cup(x\cup u))(g_0,g_1,g_2)=\prod_{s_1,s_2\in G}\big(c(g_0,g_1,g_2)\circ x(s_1^*,s_2^*)\big)\big(u(s_2,s_1,g_2)\big). \label{eq:cupsright}
% 		\end{equation}
% 	\end{itemize}
% 	At this point, it might look like we're in trouble because \autoref{eq:cupsleft} and \autoref{eq:cupsright} look different from each other. However, this is just an outcome of how the cup product is defined. Indeed, we know abstractly that we must have $(c\cup x)\cup u=c\cup(x\cup u)$ in $\widehat H^2(L/K,X^*(T)\otimes_\ZZ X_*(T)\otimes_\ZZ L^\times)$, but these are
% 	\begin{align*}
% 		((c\cup x)\cup u)(g_0,g_1,g_2) &= (c\cup x)(g_0)\otimes u(g_0,g_1,g_2) \\
% 		&= \sum_{s_1,s_2\in G}c(g_0,s_1,s_2)\otimes x(s_2^*,s_1^*)\otimes u(g_0,g_1,g_2),
% 	\end{align*}
% 	and
% 	\begin{align*}
% 		(c\cup(x\cup u))(g_0,g_1,g_2) &= c(g_0,g_1,g_2)\otimes(x\cup u)(g_0) \\
% 		&= \sum_{s_1,s_2\in G}c(g_0,g_1,g_2)\otimes x(s_1^*,s_2^*)\otimes u(s_2,s_1,g_0).
% 	\end{align*}
% 	The above two cocycles need to be equal up to a coboundary in $B^2(G,X^*(T)\otimes_\ZZ X_*(T)\otimes_\ZZ L^\times)$, so we have that the map sending $(g_0,g_1,g_2)$ to
% 	\[\sum_{s_1,s_2\in G}c(g_0,s_1,s_2)\otimes x(s_2^*,s_1^*)\otimes u(g_0,g_1,g_2)-\sum_{s_1,s_2\in G}c(g_0,g_1,g_2)\otimes x(s_1^*,s_2^*)\otimes u(s_2,s_1,g_0)\]
% 	lives in $B^2(G,X^*(T)\otimes_\ZZ X_*(T)\otimes_\ZZ L^\times)$. Passing this all through the evaluation map $X^*(T)\otimes_\ZZ X_*(T)\otimes_\ZZ L^\times\to L^\times$ by $(f\otimes g\otimes z)\mapsto(f\circ g)(z)$, we see that the map sending $(g_0,g_1,g_2)$ to
% 	\[\prod_{s_1,s_2\in G}\big(c(g_0,s_1,s_2)\circ x(s_2^*,s_1^*)\big)\big(u(g_0,g_1,g_2)\big)\bigg/\prod_{s_1,s_2\in G}\big(c(g_0,g_1,g_2)\circ x(s_1^*,s_2^*)\big)\big(u(s_2,s_1,g_2)\big)\]
% 	is a coboundary in $B^2(G,L^\times)$. (Namely, homomorphisms $M\to M'$ induce homomorphisms $B^\bullet(G,M)\to B^\bullet(G,M)$.) Thus, \autoref{eq:cupsleft} and \autoref{eq:cupsright} are indeed in the same cocycle class.\todo{Run this check without gimmicks. Maybe general indices?}
% \end{proof}
% The point of \autoref{lem:torusdiagram} is the following result.
% \begin{theorem} \label{thm:goodcuppairing}
% 	Let $L/K$ be an extension of local fields, and let $T$ be a $K$-torus which splits over $L$. Further, suppose there is some $c\in H^2(L/K,X^*(T))$ such that
% 	\[c\cup-\colon\widehat H^0(L/K,T(L))\to\widehat H^2(L/K,L^\times)\]
% 	is an isomorphism. Now, if $u,u'\in H^2(L/K,L^\times)$ has
% 	\[x\cup u=x\cup u'\in\widehat H^0(L/K,T(L))\]
% 	for all $x\in\widehat H^{-2}(L/K,X_*(T))$, then $u=u'$.
% \end{theorem}
% \begin{proof}
% 	In the language of \autoref{lem:torusdiagram}, we are being told that $v=u$ and $v=u'$ are inducing the same top row of the following commutative diagram.
% 	\begin{equation}
% 		\begin{tikzcd}
% 			{\widehat H^{-2}(L/K,X_*(T))} & {\widehat H^0(L/K,T(L))} \\
% 			{\widehat H^0(L/K,\ZZ)} & {\widehat H^2(L/K,L^\times)}
% 			\arrow["{-\cup v}", from=1-1, to=1-2]
% 			\arrow["{c\cup-}", from=1-2, to=2-2]
% 			\arrow["{-\cup v}"', from=2-1, to=2-2]
% 			\arrow["{c\cup-}"', from=1-1, to=2-1]
% 		\end{tikzcd} \label{eq:torusdiagram}
% 	\end{equation}
% 	Quickly, note that $v=u_{L/K}$ makes the top row an isomorphism by \autoref{thm:torusreciprocity} as well as the bottom row an isomorphism (e.g., the generator $[1]\in\widehat H^0(L/K,\ZZ)$ maps to the generator $[1]\cup u_{L/K}=u^1_{L/K}=u_{L/K}$). Additionally, the right arrow is an isomorphism by hypothesis on $c$. Thus, the commutativity of the diagram tells us that the left arrow is also an isomorphism.

% 	Now, because $-\cup u,-\cup u'\colon H^{-2}\colon\widehat H^{-2}(L/K,X_*(T))\to\widehat H^0(L/K,T(L))$ induce the same map along the top row of \autoref{eq:torusdiagram}, and the vertical arrows of \autoref{eq:torusdiagram} are isomorphisms, we see that
% 	\[-\cup u,-\cup u'\colon\widehat H^0(L/K,\ZZ)\to\widehat H^2(L/K,L^\times)\]
% 	must also be the same map (along the bottom row). However, for any $v\in H^2(L/K,L^\times)$, we see that
% 	\[[1]\cup v=v^1=v,\]
% 	so it follows $u=[1]\cup u=[1]\cup u'=u'$. This finishes.
% \end{proof}
% \begin{cor}
% 	Fix notation and in particular the torus $T$ as above. If $u\in H^2(L/K,L^\times)$ induces the same isomorphism of \autoref{thm:torusreciprocity} via the cup-product map $-\cup u$, then $u=u_{L/K}$.
% \end{cor}
% \begin{proof}
% 	We apply \autoref{thm:goodcuppairing} with $c$ set to be $[\delta(\overline c)]$; notably,
% 	\[[\delta(\overline c)]\cup-\colon\widehat H^0(L/K,T(L))\to\widehat H^2(L/K,L^\times)\]
% 	is an isomorphism by combining \autoref{prop:alternativetupleclass} with \autoref{thm:yesitisacocycle}. Thus, \autoref{thm:goodcuppairing} applies and achieves the result upon setting $u'\coloneqq u_{L/K}$.
% \end{proof}
% \begin{remark}
% 	The above corollary is false when $T=\mathbb G_m$ and $G$ is non-cyclic. What makes our $T$ special is that we have some $[\delta(\overline c)]\in H^2(L/K,X^*(T))$ such that
% 	\[[\delta(\overline c)]\cup-\colon\widehat H^0(L/K,T(L))\to\widehat H^2(L/K,L^\times)\]
% 	is an isomorphism. No such element exists when $T=\mathbb G_m$ and $G$ is non-cyclic because the above groups need not even be isomorphic!
% \end{remark}

\section{Local Gerbs} \label{sec:local}
In the following two sections, we will use the results of (largely) \autoref{sec:general} in order to provide explicit group laws for some of the Kottwitz gerbs \cite{kottwitz}. In this section, we will focus on abelian extensions of local fields. The approch here is similar to the approach for global fundamental classes in \cite{explicit-fund-classes}, though we work in more generality than multiquadratic extensions.

\subsection{Set-Up} \label{sec:setup}
Fix an extension local fields $L/K$. Then let $K_m$ be the largest unramified subextension, which we will give degree $m$; let $\overline\sigma_K\in\op{Gal}(L/K)$ denote the Frobenius automorphism, which lets us set
\[K_{\pi,\nu}\coloneqq L^{\langle\overline\sigma_K\rangle}.\]
In particular, $K_{\pi,\nu}/K$ is totally ramified because, for example, the residue fields of $K_{\pi,\nu}$ and $K$ have the same order.
\begin{example}
	For $K=\QQ_p$, we can take $K_m=\QQ_p\left(\zeta_{p^m-1}\right)$ and $K_{\pi,\nu}=\QQ_p\left(\zeta_{p^\nu}\right)$.
\end{example}
For some fixed $\nu$ and $m$, we let $L\coloneqq K_{\pi,\nu}K_m$. This gives us the following tower of fields.
% https://q.uiver.app/?q=WzAsNCxbMSwyLCJLIl0sWzAsMSwiS197XFxwaSxcXG51fSJdLFsyLDEsIktfbSJdLFsxLDAsIkwiXSxbMCwxLCIiLDAseyJzdHlsZSI6eyJoZWFkIjp7Im5hbWUiOiJub25lIn19fV0sWzEsMywiIiwwLHsic3R5bGUiOnsiaGVhZCI6eyJuYW1lIjoibm9uZSJ9fX1dLFswLDIsIiIsMix7InN0eWxlIjp7ImhlYWQiOnsibmFtZSI6Im5vbmUifX19XSxbMiwzLCIiLDIseyJzdHlsZSI6eyJoZWFkIjp7Im5hbWUiOiJub25lIn19fV1d&macro_url=https%3A%2F%2Fraw.githubusercontent.com%2FdFoiler%2Fnotes%2Fmaster%2Fnir.tex
\[\begin{tikzcd}
	& L \\
	{K_{\pi,\nu}} && {K_m} \\
	& K
	\arrow[no head, from=3-2, to=2-1]
	\arrow[no head, from=2-1, to=1-2]
	\arrow[no head, from=3-2, to=2-3]
	\arrow[no head, from=2-3, to=1-2]
\end{tikzcd}\]
To help us a little later, we will assume that the extension $L/K$ is neither totally ramified nor unramified.
\begin{remark}
	Assuming that $L/K$ is neither totally ramified nor unramified is not actually very big of a problem because we can apply inflation to $u_{L/K}$ to read off the fundamental class for the totally ramified and unramified parts.
\end{remark}
We provide some quick commentary on these extensions.
\begin{itemize}
	\item The extension $K_m/K$ is unramified of degree $f\coloneqq m$; note we are assuming $1<f<n$. Its Galois group is thus generated by the Frobenius element defined by $\overline\sigma_K$.
	\item The extension $K_{\pi,\nu}/K$ is totally ramified of degree $[K_{\pi,\nu}:K]$. Because we are assuming this Galois group is abelian, we may write
	\[\op{Gal}(K_{\pi,\nu}/K)\simeq\Gamma_1\times\cdots\times\Gamma_t\]
	where $\Gamma_i=\langle\tau_i\rangle\subseteq\op{Gal}(K_{\pi,\nu}/K)$ is a cyclic group of order $n_i$.
	% Its Galois group is thus isomorphic to $\left(\ZZ/p^\nu\ZZ\right)^\times$, where the isomorphism takes $x\in\left(\ZZ/p^\nu\ZZ\right)^\times$ to
	% \[\sigma_x\colon\zeta_{p^\nu}\mapsto\zeta_{p^\nu}^{x^{-1}}.\]
	% The group $\left(\ZZ/p^\nu\ZZ\right)^\times$ is cyclic, so we will fix a generator $x$, which gives us a distinguished generator $\sigma_x\in\op{Gal}\left(\QQ(\zeta_{p^\nu})/\QQ_p\right)$.
	\item Because $K_{\pi,\nu}/K$ is totally ramified and $K_m/K$ is unramified, we have that the fields $K_{\pi,\nu}$ and $K_m$ are linearly disjoint over $K$. As such, $L=K_{\pi,\nu}K_m$ has
	\begin{align*}
		\op{Gal}(L/K_{\pi,\nu}) &\simeq \op{Gal}(K_m/K)=\langle\overline\sigma_K\rangle \\
		\op{Gal}(L/K_m) &\simeq \op{Gal}(K_{\pi,\nu}/K)=\Gamma_1\times\cdots\times\Gamma_t \\
		\op{Gal}(L/K) &\simeq \op{Gal}(K_m/K)\times\op{Gal}(K_{\pi,\nu}/K)=\langle\overline\sigma_K\rangle\times\Gamma_1\times\cdots\times\Gamma_t.
	\end{align*}
	In light of these isomorphisms, we will upgrade $\overline\sigma_K$ to the automorphism of $L/K$ which restricts properly on $K_m/K$ and fixing $K_{\pi,\nu}$; we do analogously for the $\tau_i$. We also acknowledge that our degree is
	\[n\coloneqq[L:K]=[K_m:K]\cdot[K_{\pi,\nu}:K]=f\cdot[K_{\pi,\nu}:K].\]
\end{itemize}
For brevity, we will also set $L_i\coloneqq L^{\langle\tau_i\rangle}$ for each $i$, which makes the fields under $L$ look like the following.
% https://q.uiver.app/?q=WzAsNyxbMSwzLCJLIl0sWzIsMiwiS19tIl0sWzAsMSwiTF8wPUtfe1xccGksXFxudX0iXSxbMSwwLCJMIl0sWzIsMSwiTF8xIl0sWzMsMSwiXFxjZG90cyJdLFs0LDEsIkxfdCJdLFswLDEsIiIsMix7InN0eWxlIjp7ImhlYWQiOnsibmFtZSI6Im5vbmUifX19XSxbMiwzLCIiLDAseyJzdHlsZSI6eyJoZWFkIjp7Im5hbWUiOiJub25lIn19fV0sWzAsMiwiIiwwLHsic3R5bGUiOnsiaGVhZCI6eyJuYW1lIjoibm9uZSJ9fX1dLFs0LDMsIiIsMix7InN0eWxlIjp7ImhlYWQiOnsibmFtZSI6Im5vbmUifX19XSxbMSw0LCIiLDIseyJzdHlsZSI6eyJoZWFkIjp7Im5hbWUiOiJub25lIn19fV0sWzEsNiwiIiwyLHsic3R5bGUiOnsiaGVhZCI6eyJuYW1lIjoibm9uZSJ9fX1dLFs2LDMsIiIsMix7InN0eWxlIjp7ImhlYWQiOnsibmFtZSI6Im5vbmUifX19XV0=&macro_url=https%3A%2F%2Fraw.githubusercontent.com%2FdFoiler%2Fnotes%2Fmaster%2Fnir.tex
\[\begin{tikzcd}
	& L \\
	{K_{\pi,\nu}} && {L_1} & \cdots & {L_t} \\
	&& {K_m} \\
	& K
	\arrow[no head, from=4-2, to=3-3]
	\arrow[no head, from=2-1, to=1-2]
	\arrow[no head, from=4-2, to=2-1]
	\arrow[no head, from=2-3, to=1-2]
	\arrow[no head, from=3-3, to=2-3]
	\arrow[no head, from=3-3, to=2-5]
	\arrow[no head, from=2-5, to=1-2]
\end{tikzcd}\]
In particular, $\op{Gal}(L/L_i)$ is cyclic for each $i$.

Now, the main idea in the computation is to use an unramified extension $M\coloneqq K_n$ of the same degree as $L/K$. This modifies our diagram of fields as follows.
% https://q.uiver.app/?q=WzAsNixbMSwzLCJLIl0sWzAsMiwiS197XFxwaSxcXG51fSJdLFsxLDAsIk1MIl0sWzAsMSwiTCJdLFsxLDIsIktfbSJdLFsyLDEsIk0iXSxbMCwxLCJcXHRleHR7cmFtfSIsMCx7InN0eWxlIjp7ImhlYWQiOnsibmFtZSI6Im5vbmUifX19XSxbMSwzLCJcXHRleHR7dW5yfSIsMCx7InN0eWxlIjp7ImhlYWQiOnsibmFtZSI6Im5vbmUifX19XSxbMCw0LCJcXHRleHR7dW5yfSIsMix7InN0eWxlIjp7ImhlYWQiOnsibmFtZSI6Im5vbmUifX19XSxbNCw1LCJcXHRleHR7dW5yfSIsMix7InN0eWxlIjp7ImhlYWQiOnsibmFtZSI6Im5vbmUifX19XSxbMywyLCJcXHRleHR7dW5yfSIsMCx7InN0eWxlIjp7ImhlYWQiOnsibmFtZSI6Im5vbmUifX19XSxbNSwyLCJcXHRleHR7cmFtfSIsMix7InN0eWxlIjp7ImhlYWQiOnsibmFtZSI6Im5vbmUifX19XSxbNCwzLCJcXHRleHR7cmFtfSIsMix7InN0eWxlIjp7ImhlYWQiOnsibmFtZSI6Im5vbmUifX19XV0=&macro_url=https%3A%2F%2Fraw.githubusercontent.com%2FdFoiler%2Fnotes%2Fmaster%2Fnir.tex
\[\begin{tikzcd}
	& ML \\
	L && M \\
	{K_{\pi,\nu}} & {K_m} \\
	& K
	\arrow["{\text{ram}}", no head, from=4-2, to=3-1]
	\arrow["{\text{unr}}", no head, from=3-1, to=2-1]
	\arrow["{\text{unr}}"', no head, from=4-2, to=3-2]
	\arrow["{\text{unr}}"', no head, from=3-2, to=2-3]
	\arrow["{\text{unr}}", no head, from=2-1, to=1-2]
	\arrow["{\text{ram}}"', no head, from=2-3, to=1-2]
	\arrow["{\text{ram}}"', no head, from=3-2, to=2-1]
\end{tikzcd}\]
We have labeled the unramified extensions by ``$\textrm{unr}$'' and the totally ramified extensions by ``$\textrm{ram}$.''

% For brevity, we set $K\coloneqq\QQ_p$ and $L\coloneqq\QQ_p(\zeta_N)$ and $M\coloneqq\QQ_p(\zeta_{N'})$ so that $ML=\QQ_p(\zeta_N,\zeta_{N'})$. This abbreviates our diagram into the following.
% % https://q.uiver.app/?q=WzAsNCxbMSwzLCJLIl0sWzAsMSwiTCJdLFsyLDEsIk0iXSxbMSwwLCJNTCJdLFsxLDMsIiIsMCx7InN0eWxlIjp7ImhlYWQiOnsibmFtZSI6Im5vbmUifX19XSxbMCwxLCIiLDAseyJzdHlsZSI6eyJoZWFkIjp7Im5hbWUiOiJub25lIn19fV0sWzAsMiwiIiwyLHsic3R5bGUiOnsiaGVhZCI6eyJuYW1lIjoibm9uZSJ9fX1dLFsyLDMsIiIsMix7InN0eWxlIjp7ImhlYWQiOnsibmFtZSI6Im5vbmUifX19XV0=&macro_url=https%3A%2F%2Fraw.githubusercontent.com%2FdFoiler%2Fnotes%2Fmaster%2Fnir.tex
% \[\begin{tikzcd}
% 	& ML \\
% 	L && M \\
% 	\\
% 	& K
% 	\arrow[no head, from=2-1, to=1-2]
% 	\arrow[no head, from=4-2, to=2-1]
% 	\arrow[no head, from=4-2, to=2-3]
% 	\arrow[no head, from=2-3, to=1-2]
% \end{tikzcd}\]
As before, we provide some comments on the field extensions.
\begin{itemize}
	\item The extension $M/K$ is unramified of degree $n$. As before, its Galois group is cyclic, generated by the Frobenius element $\sigma_K$. Observe that $\sigma_K$ restricted to $K_m$ is $\overline\sigma_K$, explaining our notation. In particular, $\sigma_K$ has order $n$, but $\overline\sigma_K$ has order $f<n$.
	\item As before, note that $K_{\pi,\nu}$ and $M$ are linearly disjoint over $K$ because $K_{\pi,\nu}/K$ is totally ramified while $M/K$ is unramified. As such, we may say that
	\begin{align*}
		\op{Gal}(ML/M) &\simeq \op{Gal}(K_{\pi,\nu}/K) = \Gamma_1\times\cdots\times\Gamma_t \\
		\op{Gal}(ML/K_{\pi,\nu}) &\simeq \op{Gal}(M/K) = \langle\sigma_K\rangle \\
		\op{Gal}(ML/K) &\simeq \op{Gal}(M/K)\times\op{Gal}(K_{\pi,\nu}/K) = \langle\sigma_K\rangle\times\Gamma_1\times\cdots\times\Gamma_t.
	\end{align*}
	Again, we will upgrade $\sigma_K$ and the $\tau_i$ to their corresponding automorphisms on any subfield of $ML$.
	\item We take a moment to compute
	\begin{align*}
		\op{Gal}(ML/L) &\simeq \left\{\sigma_K^{a}\tau\in\op{Gal}(ML/K):\sigma_K^{a}\tau|_L=\id_L\right\}.
	\end{align*}
	Because $L$ is $K_{\pi,\nu}K_m$, it suffices to fix each of these fields individually. Well, to fix $K_{\pi,\nu}$, we need $\tau$ to vanish, so we might as well force $\tau=1$. But to fix $K_m$, we need $\sigma_K^{a}|_{K_m}=\overline\sigma_K^{a}$ to be the identity, so we are actually requiring that $f\mid a$ here. As such,
	\[\op{Gal}(ML/L)=\langle\sigma_K^f\rangle.\]
\end{itemize}
These comments complete the Galois-theoretic portion of the analysis.

\section{Idea}
We will begin by briefly describe the outline for the computation. For a finite extension of local fields $L/K$, let $u_{L/K}\in H^2(L/K)$ denote the fundamental class.

Now, take variables as in our set-up in \autoref{sec:setup}. The main idea is to translate what we know about the unramified extension $M/K$ over to the general extension $L/K$. In particular, we are able to compute the fundamental class $u_{M/K}\in H^2(M/K)$, so we observe that
\[\op{Inf}_{M/K}^{ML/K}u_{M/K}=[ML:M]u_{M/K}=n\cdot u_{ML/K}=[ML:L]u_{ML/L}=\op{Inf}_{L/K}^{ML/K}u_{L/K}.\]
As such, we will be able to compute $u_{L/K}$ as long as we are able to invert the inflation map $\op{Inf}\colon H^2(L/K)\to H^2(ML/K)$. This is not actually very easy to do in general,\footnote{The difficulty comes from the fact that a generic cocycle might be off from an inflated cocycle by some truly hideous coboundary.} but we are in luck because this inflation map here comes from the Inflation--Restriction exact sequence
\[0\to H^2(L/K)\stackrel{\op{Inf}}\to H^2(ML/K)\stackrel{\op{Res}}\to H^2(ML/L).\]
The argument for the Inflation--Restriction exact sequence is an explicit computation on cocycles (involving some dimension shifting), but it can be tracked backwards to give the desired cocycle.

\subsection{Computation}
In this section we record the details of the computation.

\subsubsection{Explicit Inflation--Restriction}
The results and commentary here mirror \cite[Section~2]{explicit-fund-classes}. Throughout this section, $G$ will be a group (usually finite) and $H\subseteq G$ will be a subgroup (usually normal).

We begin by recalling the statement of the Inflation--Restriction exact sequence; we will provide the proof for completeness because we will use the proof for our computation.
\begin{theorem}[{\cite[Proposition~5]{atiyah-wall}}] \label{thm:infres}
	Let $G$ be a finite group with normal subgroup $H\subseteq G$. Given a $G$-module $A$, suppose that the $H^i(H,A)=0$ for $1\le i<q$ for some index $q\ge1$. Then the sequence
	\[0\to H^q\left(G/H,A^H\right)\stackrel{\op{Inf}}\to H^q(G,A)\stackrel{\op{Res}}\to H^q(H,A)\]
	is exact.
\end{theorem}
\begin{proof}
	The proof is by induction on $q$, via dimension shifting. For $q=1$, we can just directly check this on $1$-cocycles. The main point is the exactness at $H^q(G,A)$: if $c\in Z^1(G,A)$ has $\op{Res}(c)\in B^1(H,A)$, then find $a\in A$ with
	\[\op{Res}(c)(a)\coloneqq h\cdot a-a.\]
	As such, we define $f_a\in B^1(G,A)$ by $f_a(g)\coloneqq g\cdot a-a$, which implies that $c-f_a$ vanishes on $H$. It is then possible to stare at the $1$-cocycle condition
	\[(c-f_a)(gg')=(c-f_a)(g)+g\cdot(c-f_a)(g')\]
	to check that $c-f_a$ only depends on the cosets of $H$ (e.g., by taking $g'\in H$) and that $\im(c-f_a)\subseteq A^H$ (e.g., by taking $g\in H$).

	For $q>1$, we use dimension shifting via the following lemma. Indeed, suppose the statement is true for $q$. Then the short exact sequence
	\[0\to A\to\op{Hom}_\ZZ(\ZZ[G],A)\to\op{Hom}_\ZZ(I_G,A)\to0\]
	induces vertical isomorphisms in the following commutative diagram.
	% \begin{lemma}[Dimension shifting] \label{lem:dimshift}
	% 	Let $G$ be a group with subgroup $H\subseteq G$. Given a $G$-module $A$, all indices $q\ge1$ have
	% 	\[\delta\colon H^q(H,\op{Hom}_\ZZ(I_G,A))\simeq H^{q+1}(H,A).\]
	% \end{lemma}
	% \begin{proof}
	% 	Recall that we have the short exact sequence of $\ZZ[H]$-modules
	% 	\[0\to I_G\to\ZZ[G]\to\ZZ\to0.\]
	% 	In fact, this short exact sequence splits over $\ZZ$, so it will still be short exact after applying $\op{Hom}_\ZZ(-,A)$, which gives the short exact sequence
	% 	\[0\to A\to\op{Hom}_\ZZ(\ZZ[G],A)\to\op{Hom}_\ZZ(I_G,A)\to0\]
	% 	of $\ZZ[H]$-modules. The result now follows from the long exact sequence of cohomology upon noting that $\op{Hom}_\ZZ(\ZZ[G],A)$ is coinduced and hence acyclic for cohomology.
	% \end{proof}
	% Using the above lemma, we have the following the commutative diagram with vertical arrows which are isomorphisms.
	% https://q.uiver.app/?q=WzAsOCxbMCwwLCIwIl0sWzAsMSwiMCJdLFsxLDAsIkhecVxcbGVmdChHL0gsXFxvcHtIb219X1xcWlooSV9HLEEpXkhcXHJpZ2h0KSJdLFsxLDEsIkhee3ErMX1cXGxlZnQoRy9ILEFeSFxccmlnaHQpIl0sWzIsMSwiSF57cSsxfShHLEEpIl0sWzIsMCwiSF5xKEcsXFxvcHtIb219X1xcWlooSV9HLEEpKSJdLFszLDEsIkhee3ErMX0oSCxBKSJdLFszLDAsIkhecShILFxcb3B7SG9tfV9cXFpaKElfRyxBKSkiXSxbMSwzXSxbMyw0XSxbNCw2XSxbMiwzLCJcXGRlbHRhIl0sWzUsNCwiXFxkZWx0YSJdLFs3LDYsIlxcZGVsdGEiXSxbNSw3XSxbMiw1XSxbMCwyXV0=&macro_url=https%3A%2F%2Fraw.githubusercontent.com%2FdFoiler%2Fnotes%2Fmaster%2Fnir.tex
	\[\begin{tikzcd}
		0 & {H^q\left(G/H,\op{Hom}_\ZZ(I_G,A)^H\right)} & {H^q(G,\op{Hom}_\ZZ(I_G,A))} & {H^q(H,\op{Hom}_\ZZ(I_G,A))} \\
		0 & {H^{q+1}\left(G/H,A^H\right)} & {H^{q+1}(G,A)} & {H^{q+1}(H,A)}
		\arrow[from=2-1, to=2-2]
		\arrow[from=2-2, to=2-3]
		\arrow[from=2-3, to=2-4]
		\arrow["\delta", from=1-2, to=2-2]
		\arrow["\delta", from=1-3, to=2-3]
		\arrow["\delta", from=1-4, to=2-4]
		\arrow[from=1-3, to=1-4]
		\arrow[from=1-2, to=1-3]
		\arrow[from=1-1, to=1-2]
	\end{tikzcd}\]
	The top row is exact by the inductive hypothesis, so the bottom row is therefore also exact.
\end{proof}
Our goal is to make the above proof explicit in the case of $q=2$, which is the only reason we sketched the above proofs at all. We begin by making the dimension shifting explicit.
\begin{lemma}[{\cite[Lemma~2.1]{explicit-fund-classes}}] \label{lem:explicitdimshift}
	Let $G$ be a group with subgroup $H\subseteq G$, and let $\{g_\alpha\}_{\alpha\in\lambda}$ be coset representatives for $H\backslash G$. Now, given a $G$-module $A$, the maps
	\begin{align*}
		\delta_H\colon Z^1(H,\op{Hom}_\ZZ(I_G,A))&\to Z^2(H,A) \\
		c&\mapsto\left[(h,h')\mapsto h\cdot c(h')(h^{-1}-1)\right] \\
		\left[h\mapsto\big((h'g_\bullet-1)\mapsto h'\cdot u((h')^{-1},h)\big)\right]&\mapsfrom u
	\end{align*}
	are group homomorphisms descending to the isomorphism $\overline\delta\colon H^1(H,\op{Hom}_\ZZ(I_G,A))\simeq H^2(H,A)$. The map $\delta_H$ above is surjective, and the reverse map is a section; when $H=G$, these are isomorphisms.
\end{lemma}
\begin{proof}
	To show that $\delta_H$ descends to an isomorphism properly, we could track through dimension-shifting by hand, or we can use the machinery we've built. Namely, setting $X=\ZZ$ in \autoref{cor:cupdown} told us that the $1$-cocycle $\chi\in Z^1(G,I_G)$ defined by
	\[\chi(\sigma)\coloneqq(1-\sigma)\]
	provides an isomorphism
	\[(\chi\cup-)\colon\widehat H^i(G,\op{Hom}_\ZZ(I_G,A))\to\widehat H^{i+1}(G,A).\]
	Computing our cup product, we have
	\begin{align*}
		(\chi\cup c)(h,h') &= \big(hc(h')\big)\big(\chi(h)\big) \\
		&= h\cdot c(h')\big(h^{-1}(1-h)\big) \\
		&= h\cdot c(h')\big(h^{-1}-1\big).
	\end{align*}
	So we see that $\delta_H=(\chi\cup-)$ on cocycles and therefore descends to the needed isomorphism. Additionally, it is a homomorphism by properties of the cup product.
	% We begin by noting that our short exact sequence can be written more explicitly as follows.
	% % https://q.uiver.app/?q=WzAsOSxbMCwwLCIwIl0sWzEsMCwiQSJdLFsyLDAsIlxcb3B7SG9tfV9cXFpaKFxcWlpbR10sQSkiXSxbMywwLCJcXG9we0hvbX1fXFxaWihJX0csQSkiXSxbNCwwLCIwIl0sWzEsMSwiYSJdLFsyLDEsImFcXG1hcHN0byh6XFxtYXBzdG9cXHZhcmVwc2lsb24oeilhKSJdLFsyLDIsImYiXSxbMywyLCJmfF97SV9HfSJdLFswLDFdLFsxLDJdLFsyLDNdLFszLDRdLFs1LDYsIiIsMCx7InN0eWxlIjp7InRhaWwiOnsibmFtZSI6Im1hcHMgdG8ifX19XSxbNyw4LCIiLDAseyJzdHlsZSI6eyJ0YWlsIjp7Im5hbWUiOiJtYXBzIHRvIn19fV1d&macro_url=https%3A%2F%2Fraw.githubusercontent.com%2FdFoiler%2Fnotes%2Fmaster%2Fnir.tex
	% \[\begin{tikzcd}[row sep=0.0em]
	% 	0 & A & {\op{Hom}_\ZZ(\ZZ[G],A)} & {\op{Hom}_\ZZ(I_G,A)} & 0 \\
	% 	& a & {(z\mapsto\varepsilon(z)a)} \\
	% 	&& f & {f|_{I_G}}
	% 	\arrow[from=1-1, to=1-2]
	% 	\arrow[from=1-2, to=1-3]
	% 	\arrow[from=1-3, to=1-4]
	% 	\arrow[from=1-4, to=1-5]
	% 	\arrow[maps to, from=2-2, to=2-3]
	% 	\arrow[maps to, from=3-3, to=3-4]
	% \end{tikzcd}\]
	% We now track through the induced boundary morphism $\delta\colon H^1(H,\op{Hom}_\ZZ(I_G,A))\to H^2(H,Q)$.
	% \begin{itemize}
	% 	\item We begin with $c\in Z^1(H,\op{Hom}_\ZZ(I_G,A))$, which means that we have $c(h)\colon I_G\to A$ for each $h,h'\in H$, and we satisfy
	% 	\[c(hh')=c(h)+h\cdot c(h').\]
	% 	Tracking through the action of $H$ on $\op{Hom}_\ZZ(I_G,A)$, this means that
	% 	\[c(hh')(g-1)=c(h)(g-1)+h\cdot c(h')(h^{-1}g-h^{-1})\]
	% 	for any $g\in G$.
	% 	\item To pull $c$ back to $C^1(H,\op{Hom}_\ZZ(\ZZ[G],A))$, we need to lift $c(h)\colon I_G\to A$ to a $\widetilde c(h)\colon\ZZ[G]\to A$. Recalling that we only need to preserve group structure, we simply precompose $c(h)$ with the map $\ZZ[G]\to I_G$ given by $z\mapsto z-\varepsilon(z)$. That is, we define
	% 	\[\widetilde c(h)(z)\coloneqq c(h)(z-\varepsilon(z)).\]
	% 	\item We now push $\widetilde c$ through $d\colon C^1(H,\op{Hom}_\ZZ(\ZZ[G],A))\to Z^2(H,\op{Hom}_\ZZ(\ZZ[G],A))$. This gives
	% 	\[(d\widetilde c)(h,h')=g\widetilde c(h')-\widetilde c(hh')+\widetilde c(h)\]
	% 	for any $h,h'\in H$. Concretely, plugging in some $z\in\ZZ[G]$ makes this look like
	% 	\begin{align*}
	% 		(d\widetilde{c})(h,h')(z) &= (h\widetilde c(h'))(z)-\widetilde c(hh')(z)+\widetilde c(h)(z) \\
	% 		&= h\cdot c(h')\left(h^{-1}z-\varepsilon(h^{-1}z)\right)-c(hh')(z-\varepsilon(z))+c(h)(z-\varepsilon(z)) \\
	% 		&= h\cdot c(h')\left(h^{-1}z-\varepsilon(z)\right)-c(hh')(z-\varepsilon(z))+c(h)(z-\varepsilon(z)).
	% 	\end{align*}
	% 	Now, from the $1$-cocycle condition on $c$, we recall
	% 	\[-c(hh')(z-\varepsilon(z))+c(h)(z-\varepsilon(z))=-h\cdot(c(h')(h^{-1}z-\varepsilon(z)h^{-1})),\]
	% 	so
	% 	\begin{align*}
	% 		(d\widetilde{c})(h,h')(z) &= h\cdot c(h')\left(\varepsilon(z)h^{-1}-\varepsilon(z)\right) \\
	% 		&= \varepsilon(z)\cdot\left(h\cdot c(h')\left(h^{-1}-1\right)\right).
	% 	\end{align*}
	% 	In particular, we see that $d\widetilde c\in Z^2(H,\op{Hom}_\ZZ(\ZZ[G],A))$ pulls back to $(h,h')\mapsto h\cdot c(h')\left(h^{-1}-1\right)$ in $Z^2(H,A)$. It is not too difficult to check that we have in fact defined a $2$-cocycle, but we will not do so because it is not necessary for the proof.
	% \end{itemize}
	% Now, we do know that $\delta_H$ is a homomorphism abstractly on elements of our cohomology classes by the Snake lemma, but it is also not too hard to see that
	% \[\delta_H\colon Z^1(H,\op{Hom}_\ZZ(I_G,A))\to Z^2(H,A)\]
	% is in fact a homomorphism of groups directly from the construction. In short,
	% \[\delta_H(c+c')(h,h')=h'\cdot c(h)\left(h^{-1}-1\right)+h'\cdot c'(h)\left(h^{-1}-1\right)=(\delta_H(c)+\delta_H(c'))(h,h')\]
	% for any $h,h'\in H$.

	It remains to prove the last sentence. We run the following checks; given $u\in Z^2(H,A)$, define $c_u\in C^1(H,\op{Hom}_\ZZ(I_G,A))$ by
	\[c_u(h)(h'g_\bullet-1)=h'\cdot u\left((h')^{-1},h\right).\]
	Note that this is enough data to define $c_u(h)\colon I_G\to A$ because $I_G$ is a free $\ZZ$-module generated by $\{g-1:g\in G\}$.
	\begin{itemize}
		\item We verify that $c_u$ is a $1$-cocycle. This is a matter of force. Pick up $h,h'\in H$ and $g_\bullet h''\in G$ and write
		\begin{align*}
			&\phantom{{}={}}(hc_u(h'))(h''g_\bullet-1)+c_u(hh')(h''g_\bullet-1)+c_u(h)(h''g_\bullet-1) \\
			&= h\cdot c_u(h')\left(h^{-1}h''g_\bullet -h^{-1}\right)+c_u(hh')(h''g_\bullet-1)+c_u(h)(h''g_\bullet-1) \\
			&= h\cdot\left(h^{-1}h''u\left((h'')^{-1}h,h'\right)-h^{-1}u(h,h')\right)+h''u\left((h'')^{-1},hh'\right)+h''u\left((h'')^{-1},h\right) \\
			&= h''u\left((h'')^{-1}h,h'\right)-u(h,h')+h''u\left((h'')^{-1},hh'\right)+h''u\left((h'')^{-1},h\right).
		\end{align*}
		This is just the $2$-cocycle condition for $u$ upon dividing out by $h''$, so we are done.
		\item For $u\in Z^2(H,A)$, we verify that $\delta_H(c_u)=u$. Indeed, given $h,h'\in H$, we check
		\begin{align*}
			\delta_H(c_u)(h,h') &= h\cdot c_u(h')\left(h^{-1}-1\right) \\
			&= h\cdot h^{-1}\cdot u(h,h') \\
			&= u(h,h').
		\end{align*}
	\end{itemize}
	So far we have verified that $\delta$ has section $u\mapsto c_u$ and hence must be surjective. Lastly, we take $H=G$ and show that $c_{\delta c}=c$ to finish. Indeed, for $g,g'\in G=H$, we write
	\begin{align*}
		c_{\delta_H c}(g)(g'-1) &= g'\cdot(\delta_H c)\left((g')^{-1},g\right) \\
		&= g'(g')^{-1}\cdot c(g)(g'-1) \\
		&= c(g)(g'-1),
	\end{align*}
	which is what we wanted.
\end{proof}
We also have used dimension shifting to show that $H^1\left(G/H,\op{Hom}_\ZZ(I_G,A)^H\right)\to H^2\left(G/H,A^H\right)$ is an isomorphism, but this requires a little more trickery. To begin, we discuss how to lift from $\op{Hom}_\ZZ(I_G,A)^H$ to $\op{Hom}_\ZZ(\ZZ[G],A)^H$.
\begin{lemma} \label{lem:howtolift}
	Let $G$ be a group with subgroup $H\subseteq G$. Fix a $G$-module $A$ with $H^1(H,A)=0$. Then, for any $\psi\in\op{Hom}_\ZZ(I_G,A)^H$, the function $h\mapsto h\psi\left(h^{-1}-1\right)$ is a cocycle in $Z^1(H,A)=B^1(H,A)$, so we can define a function $\eta_\bullet\colon\op{Hom}_\ZZ(I_G,A)^H\to A$ such that
	\[\psi(h-1)=h\cdot \eta_\varphi-\eta_\varphi\]
	for all $h\in H$. In fact, given $\varphi\in\op{Hom}_\ZZ(I_G,A)^H$, we can construct $\widetilde\varphi\in\op{Hom}_\ZZ(\ZZ[G],A)^H$ by
	\[\widetilde\varphi(z)\coloneqq\varphi(z-\varepsilon(z))+\varepsilon(z)\eta_\varphi\]
	so that $\widetilde\varphi|_{I_G}=\varphi$.
\end{lemma}
\begin{proof}
	We will just run the checks directly.
	\begin{itemize}
		\item We start by checking $\psi\in\op{Hom}_\ZZ(I_G,A)^H$ give $1$-cocycles $c(h)\coloneqq \varphi\left(h-1\right)$ in $Z^1(A,H)$. To begin, we note that $\psi\in\op{Hom}_\ZZ(I_G,A)^H$ simply means that any $z-\varepsilon(z)\in I_G$ has
		\[\psi(z-\varepsilon(z))=(h\psi)(z-\varepsilon(z))=h\psi\left(h^{-1}z-h^{-1}\varepsilon(z)\right)\]
		for all $h\in H$. In particular, replacing $h$ with $h^{-1}$ tells us that
		\[h\psi(z-\varepsilon(z))=\psi(hz-h\varepsilon(z)).\]
		Now, we can just compute
		\begin{align*}
			(dc)(h,h') &= hc(h')-c(hh')+c(h) \\
			&= hc\left(h'-1\right)-c\left(hh'-1\right)+c\left(h-1\right) \\
			&= c\left(hh'-h\right)-c\left(hh'-1\right)+c\left(h-1\right),
		\end{align*}
		where in the last equality we used the fact that $\psi\in\op{Hom}_\ZZ(I_G,A)^H$. Now, $(dc)(h,h')$ manifestly vanishes, so we are done.
		\item Note that $\widetilde\varphi\in\op{Hom}_\ZZ(\ZZ[G],A)$ because it is a linear combination of (compositions of) homomorphisms.
		\item Note that any $z\in I_G$ has $\varepsilon(z)=0$, so
		\[\widetilde\varphi(z)=\varphi(z-0)+0\cdot \eta_\varphi=\varphi(z),\]
		so $\widetilde\varphi|_{I_G}=\varphi$.
		\item It remains to check that $\widetilde\varphi$ is fixed by $H$. This requires a little more effort. Recall that $\varphi\in\op{Hom}_\ZZ(I_G,A)^H$ means that any $z-\varepsilon(z)\in I_G$ has
		\[h\varphi(z-\varepsilon(z))=\varphi\left(hz-h\varepsilon(z)\right)\]
		for any $h\in H$. Now, we just compute
		\begin{align*}
			(h\widetilde\varphi)(z) &= h\widetilde\varphi\left(h^{-1}z\right) \\
			&= h\left(\varphi\left(h^{-1}z-\varepsilon(h^{-1}z)\right)+\varepsilon(h^{-1}z)\eta_\varphi\right) \\
			&= \varphi\left(z-h\varepsilon(z)\right)+\varepsilon(z)\cdot h\eta_\varphi \\
			&= \varphi\left(z-h\varepsilon(z)\right)+\varepsilon(z)\varphi(h-1)+\varepsilon(z)\eta_\varphi \\
			&= \varphi(z-\varepsilon(z))+\varepsilon(z)\eta_\varphi \\
			&= \widetilde\varphi(z).
		\end{align*}
	\end{itemize}
	The above checks complete the proof.
\end{proof}
% \begin{remark}
% 	For motivation, the $\widetilde\varphi$ was constructed by tracking through the following diagram.
% 	% https://q.uiver.app/?q=WzAsOCxbMSwwLCJcXGRpc3BsYXlzdHlsZVxcZnJhY3tDXjAoSCxBKX17Ql4wKEgsQSl9Il0sWzEsMSwiWl4xKEgsQSk9Ql4xKEgsQSkiXSxbMiwxLCJaXjEoSCxcXG9we0hvbX1fXFxaWihcXFpaW0ddLEEpKSJdLFszLDEsIlpeMShILFxcb3B7SG9tfV9cXFpaKElfRyxBKSkiXSxbMywwLCJcXGRpc3BsYXlzdHlsZVxcZnJhY3tDXjAoSCxcXG9we0hvbX1fXFxaWihJX0csQSkpfXtCXjAoSCxcXG9we0hvbX1fXFxaWihJX0csQSkpfSJdLFsyLDAsIlxcZGlzcGxheXN0eWxlXFxmcmFje0NeMChILFxcb3B7SG9tfV9cXFpaKFxcWlpbR10sQSkpfXtCXjAoSCxcXG9we0hvbX1fXFxaWihcXFpaW0ddLEEpKX0iXSxbNCwwLCIwIl0sWzAsMSwiMCJdLFs3LDFdLFsxLDJdLFsyLDNdLFswLDVdLFs1LDRdLFs0LDZdLFswLDFdLFs1LDJdLFs0LDNdXQ==&macro_url=https%3A%2F%2Fraw.githubusercontent.com%2FdFoiler%2Fnotes%2Fmaster%2Fnir.tex
% 	\[\begin{tikzcd}
% 		& {\displaystyle\frac{C^0(H,A)}{B^0(H,A)}} & {\displaystyle\frac{C^0(H,\op{Hom}_\ZZ(\ZZ[G],A))}{B^0(H,\op{Hom}_\ZZ(\ZZ[G],A))}} & {\displaystyle\frac{C^0(H,\op{Hom}_\ZZ(I_G,A))}{B^0(H,\op{Hom}_\ZZ(I_G,A))}} & 0 \\
% 		0 & {Z^1(H,A)=B^1(H,A)} & {Z^1(H,\op{Hom}_\ZZ(\ZZ[G],A))} & {Z^1(H,\op{Hom}_\ZZ(I_G,A))}
% 		\arrow[from=2-1, to=2-2]
% 		\arrow[from=2-2, to=2-3]
% 		\arrow[from=2-3, to=2-4]
% 		\arrow[from=1-2, to=1-3]
% 		\arrow[from=1-3, to=1-4]
% 		\arrow[from=1-4, to=1-5]
% 		\arrow[from=1-2, to=2-2]
% 		\arrow[from=1-3, to=2-3]
% 		\arrow[from=1-4, to=2-4]
% 	\end{tikzcd}\]
% 	In short, take $\varphi\in Z^0(H,\op{Hom}_\ZZ(I_G,A))=\op{Hom}_\ZZ(I_G,A)^H$, pull it back to $z\mapsto\varphi(z-\varepsilon(z))$. Pushing this down to $Z^1(H,\op{Hom}_\ZZ(\ZZ[G],A))$ and pulling back to $Z^1(H,A)$ takes us to the $1$-cocycle $h\mapsto h\varphi\left(h^{-1}-1\right)$. Here we use the $H^1(H,A)=0$ condition above and adjust our lift $z\mapsto\varphi(z-\varepsilon(z))$ accordingly.
% \end{remark}
And now we can now make our dimension shifting explicit.
\begin{lemma} \label{lem:dimshift2}
	Work in the context of \autoref{lem:howtolift} and assume that $H\subseteq G$ is normal. We track through the isomorphism
	\[\delta\colon H^1\left(G/H,\op{Hom}_\ZZ(I_G,A)^H\right)\simeq H^2\left(G/H,A^H\right)\]
	given by the exact sequence
	\[0\to A^H\to\op{Hom}_\ZZ(\ZZ[G],A)^H\to\op{Hom}_\ZZ(I_G,A)^H\to0.\]
\end{lemma}
\begin{proof}
	We begin with some $c\in H^1\left(G/H,\op{Hom}_\ZZ(I_G,A)^H\right)$. To track through the $\delta$, we define
	\[\widetilde c(gH)\coloneqq c(gH)(z-\varepsilon(z))+\eta_{c(gH)}\varepsilon(z)\]
	to be the lift given in \autoref{lem:howtolift}. Now, we are given that $dc=0$, which here means that any $z\in\ZZ[G]$ and $gH,g'H\in G/H$ will have
	\begin{align*}
		0 &= (dc)(gH,g'H)(z-\varepsilon(z)) \\
		0 &= (gH\cdot c(g'H)-c(gg'H)+c(gH))(z-\varepsilon(z)) \\
		0 &= g\cdot c(g'H)\left(g^{-1}z-g^{-1}\varepsilon(z)\right)-c(gg'H)(z-\varepsilon(z))+c(gH)(z-\varepsilon(z)) \\
		g\cdot c(g'H)\left(g^{-1}-1\right)\varepsilon(z) &= g\cdot c(g'H)\left(g^{-1}z-\varepsilon(z)\right)-c(gg'H)(z-\varepsilon(z))+c(gH)(z-\varepsilon(z)) \\
		g\cdot c(g'H)\left(g^{-1}-1\right)\varepsilon(z) &= g\cdot c(g'H)\left(g^{-1}z-\varepsilon(g^{-1}z)\right)-c(gg'H)(z-\varepsilon(z))+c(gH)(z-\varepsilon(z)).
	\end{align*}
	We now directly compute that
	\begin{align*}
		(d\widetilde c)(gH,g'H)(z) &= (gH\cdot c(g'H)-c(gg'H)+c(gH))(z) \\
		&= g\cdot c(g'H)\left(g^{-1}z-\varepsilon(g^{-1}z)\right)+g\eta_{c(g'H)}\varepsilon(z) \\
		&\phantom{{}={}}-c(gg'H)(z-\varepsilon(z))-\eta_{c(gg'H)}\varepsilon(z) \\
		&\phantom{{}={}}+c(gH)(z-\varepsilon(z))+\eta_{c(gH)}\varepsilon(z) \\
		&= \left(g\cdot c(g'H)\left(g^{-1}-1\right)+g\cdot \eta_{c(g'H)}-\eta_{c(gg'H)}+\eta_{c(gH)}\right)\varepsilon(z)
	\end{align*}
	% We quickly recall that $c(g'H)\in\op{Hom}_\ZZ(I_G,A)^H$ implies that $h\cdot c(g'H)(z)=(h\cdot c(g'H))\left(h^{-1}z\right)=c(g'H)\left(h^{-1}-1\right)$, so in fact we can write
	% \[(d\widetilde c)(gH,g'H)(z) = \left(c(g'H)\left(1-g\right)-g\cdot \eta_{c(g'H)}+\eta_{c(gg'H)}-\eta_{c(gH)}\right)\varepsilon(z).\]
	As such, we have pulled ourselves back to the $2$-cocycle given by
	\[\boxed{u(gH,g'H)\coloneqq g\cdot c(g'H)\left(g^{-1}-1\right)+g\cdot \eta_{c(g'H)}-\eta_{c(gg'H)}+\eta_{c(gH)}}.\]
	We quickly note that this is in fact independent of our choice of representative $g\in gH$: changing representative of $g$ to $gh$ for $h\in H$ will only affect the terms
	\[h\cdot c(g'H)\left(h^{-1}g^{-1}-1\right)+h\eta_{c(g'H)}=c(g'H)\left(g^{-1}-h\right)+c(g'H)\left(h-1\right)+\eta_{c(g'H)}=c(g'H)\left(g^{-1}-1\right)+\eta_{c(g'H)},\]
	so we are indeed safe. This completes the proof.
	% Even though it is not necessary, we will run the following checks on $u$.
	% \begin{itemize}
	% 	\item We verify that $\im u\subseteq A^H$. The main point is that any $h\in H$ will have $h\cdot \eta_\varphi=h\varphi\left(h^{-1}-1\right)+\eta_\varphi=\varphi(1-h)+\eta_\varphi$ for any $\varphi\in\op{Hom}_\ZZ(I_G,A)^H$. Thus,
	% 	\begin{align*}
	% 		h\cdot u(gH,g'H) &= hg\cdot c(g'H)\left(g^{-1}-1\right)-hg\eta_{c(g'H)}+h\eta_{c(gg'H)}-h\eta_{c(gH)} \\
	% 		&= gg^{-1}hg\cdot c(g'H)\left(g^{-1}-1\right)-gg^{-1}hg\eta_{c(g'H)}+h\eta_{c(gg'H)}-h\eta_{c(gH)} \\
	% 		&= g\cdot c(g'H)\left(g^{-1}h-g^{-1}hg\right) \\
	% 		&\phantom{{}={}}-g\left(c(g'H)(1-g^{-1}hg)+\eta_{c(g'H)}\right) \\
	% 		&\phantom{{}={}}+c(gg'H)(1-h)+\eta_{c(gg'H)} \\
	% 		&\phantom{{}={}}-c(gH)(1-h)-\eta_{c(gH)} \\
	% 		&= g\cdot c(g'H)(g^{-1}h-1)+c(gg'H)(1-h)-c(gH)(1-h)-g\eta_{c(g'H)}+\eta_{c(gg'H)}-\eta_{c(gH)} \\
	% 		&= g\cdot c(g'H)(g^{-1}h-1)+g\cdot c(g'H)(1-h)-g\eta_{c(g'H)}+\eta_{c(gg'H)}-\eta_{c(gH)} \\
	% 		&= g\cdot c(g'H)(g^{-1}h-h)-g\eta_{c(g'H)}+\eta_{c(gg'H)}-\eta_{c(gH)} \\
	% 	\end{align*}
	% 	Because 
	% \end{itemize}
\end{proof}
We now make \autoref{thm:infres} explicit in the case of $q=2$.
\begin{lemma}[{\cite[Lemma~2.3]{explicit-fund-classes}}] \label{lem:explicitresinf}
	Let $G$ be a group with normal subgroup $H\subseteq G$. Fix a $G$-module $A$ with $H^1(H,A)=0$, and define the function $\eta_\bullet\colon\op{Hom}_\ZZ(I_G,A)^H\to A$ of \autoref{lem:howtolift}. Given $c\in Z^2(G,A)$ such that $\op{Res}^G_Hc\in B^2(H,A)$; in particular, suppose we have $b\in\op{Hom}_\ZZ(I_G,A)$ such that all $h\in H$ have
	\[\op{Res}^G_H(\delta^{-1}c)(h)=(db)(h)=h\cdot b-h,\]
	where $\delta^{-1}$ is the inverse isomorphism of \autoref{lem:explicitdimshift}. Then we find $u\in Z^2\left(G/H,A^H\right)$ such that
	\[[\op{Inf}u]=[c]\]
	in $H^2(G,A)$.
\end{lemma}
\begin{proof}
	The main point is that boundary morphisms $\delta$ commute with $\op{Res}$ and $\op{Inf}$. By construction, we have that $\left(\op{Res}^G_H\delta^{-1}c\right)-db=0$ in $Z^1(H,\op{Hom}_\ZZ(I_G,A))$. Pulling back to $Z^1(G,\op{Hom}_\ZZ(I_G,A))$, we note that
	\[c'\coloneqq\left(\delta^{-1}c-db\right)\in Z^1(G,\op{Hom}_\ZZ(I_G,A))\]
	vanishes on $H$ by hypothesis. Because $\delta^{-1}c-db$ is a $1$-cocycle, we are able to write
	\[c'(gg')=c'(g)+gc'(g').\]
	Letting $g'$ vary over $H$, we see that $\delta^{-1}c-db$ is well-defined on $G/H$. On the other hand, for any $h\in H$ and $g\in G$, we note that $g^{-1}hg\in H$, so
	\[c'(g)=c'\left(g\cdot g^{-1}hg\right)=c'\left(hg\right)=c'\left(h\right)+hc(g),\]
	implying that $c'(g)\in\op{Hom}_\ZZ(I_G,A)^H$.

	We are now ready to apply \autoref{lem:dimshift2}, which we use on $c'$, thus defining $u\coloneqq\delta(c')$. Explicitly, we have
	\[
		\boxed{u(gH,g'H) = g\cdot c'(g'H)\left(g^{-1}-1\right)+g\cdot \eta_{c'(g'H)}-\eta_{c'(gg'H)}+\eta_{c'(gH)}}.
	\]
	This is explicit enough for our purposes. Observe that $[\op{Inf}u]=[c]$ because $[\op{Inf}c']=[\delta^{-1}c]$, and $\delta$ commutes with $\op{Inf}$.
	% Thus, we define $\overline c'\in C^1\left(G/H,\op{Hom}_\ZZ(I_G,A)^H\right)$ by $\overline c'(gH)\coloneqq c'(g)$. Note $\overline c'\in Z^1\left(G/H,\op{Hom}_\ZZ(I_G,A)^H\right)$ because each $g,g'\in G$ give
	% \[\overline c'(gH\cdot g'H)=c'(gg')=c'(g)+gc'(g')=\overline c'(gH)+gH\cdot \overline c'(gH).\]
	% We take a moment to understand $\op{Hom}_\ZZ(I_G,A)^H$. Given $f\in\op{Hom}_\ZZ(I_G,A)$, the condition that $f$ is fixed by $H$ is saying that all $h\in H$ will have $hf=f$. Concretely, we require each $g\in G$ to have
	% \[f(g-1)=(hf)(g-1)=h\cdot f\left(h^{-1}g-h^{-1}\right).\]
\end{proof}

\subsubsection{Computing the Cocycle}
Given a finite Galois extension $E/F$ of local fields, we will let $c_{E/F}$ denote a representative of the fundamental class in $H^2(\op{Gal}(E/F),E^\times)$.

We now return to the set-up in \autoref{sec:setup} and track through \autoref{lem:explicitresinf} in our case. For reference, the following is the diagram that we will be chasing around; here $G\coloneqq\op{Gal}(ML/K)$ and $H\coloneqq\op{Gal}(ML/L)$.
% https://q.uiver.app/?q=WzAsOSxbMiwwLCJIXjIoXFxvcHtHYWx9KE0vSyksTV5cXHRpbWVzKSJdLFsyLDEsIkheMihcXG9we0dhbH0oTUwvSyksTUxeXFx0aW1lcykiXSxbMywxLCJIXjIoXFxvcHtHYWx9KE1ML0wpLE1MXlxcdGltZXMpIl0sWzMsMiwiSF4xKFxcb3B7R2FsfShNTC9MKSxcXG9we0hvbX1fXFxaWihJX3tcXG9we0dhbH0oTUwvSyl9LE1MXlxcdGltZXMpKSJdLFsyLDIsIkheMShcXG9we0dhbH0oTUwvSyksXFxvcHtIb219X1xcWlooSV97XFxvcHtHYWx9KE1ML0spfSxNTF5cXHRpbWVzKSkiXSxbMSwyLCJIXjEoXFxvcHtHYWx9KEwvSyksXFxvcHtIb219X1xcWlooSV97XFxvcHtHYWx9KE1ML0spfSxNTF5cXHRpbWVzKV57XFxvcHtHYWx9KEwvSyl9KSJdLFsxLDEsIkheMihcXG9we0dhbH0oTC9LKSxMXlxcdGltZXMpIl0sWzAsMSwiMCJdLFswLDIsIjAiXSxbNSw2LCJcXGRlbHRhIiwyXSxbNCwxLCJcXGRlbHRhIiwyXSxbMywyLCJcXGRlbHRhIiwyXSxbMSwyLCJcXG9we1Jlc30iXSxbNiwxLCJcXG9we0luZn0iXSxbMCwxLCJcXG9we0luZn0iXSxbNSw0XSxbNCwzXSxbNyw2XSxbOCw1XV0=&macro_url=https%3A%2F%2Fraw.githubusercontent.com%2FdFoiler%2Fnotes%2Fmaster%2Fnir.tex
\[\begin{tikzcd}
	&& {H^2(\op{Gal}(M/K),M^\times)} \\
	0 & {H^2(\op{Gal}(L/K),L^\times)} & {H^2(G,ML^\times)} & {H^2(\op{Gal}(ML/L),ML^\times)} \\
	0 & {H^1(G/H,\op{Hom}_\ZZ(I_{G},ML^\times)^{H})} & {H^1(G,\op{Hom}_\ZZ(I_{G},ML^\times))} & {H^1(H,\op{Hom}_\ZZ(I_{G},ML^\times))}
	\arrow["\delta"', from=3-2, to=2-2]
	\arrow["\delta"', from=3-3, to=2-3]
	\arrow["\delta"', from=3-4, to=2-4]
	\arrow["{\op{Res}}", from=2-3, to=2-4]
	\arrow["{\op{Inf}}", from=2-2, to=2-3]
	\arrow["{\op{Inf}}", from=1-3, to=2-3]
	\arrow["\op{Inf}", from=3-2, to=3-3]
	\arrow["\op{Res}", from=3-3, to=3-4]
	\arrow[from=2-1, to=2-2]
	\arrow[from=3-1, to=3-2]
\end{tikzcd}\]
To begin, we know that we can write
\[c_{M/K}\left(\sigma_K^i,\sigma_K^j\right)\coloneqq\pi^{\floor{\frac{i+j}n}}=\begin{cases}
	1 & i+j<n, \\
	\pi & i+j\ge n,
\end{cases}\]
where $\pi$ is a uniformizer of $K$. Inflating this down to $H^2(G,ML^\times)$ gives
\[(\op{Inf}c_{M/K})\left(\sigma_K^{a_1}\tau,\sigma_K^{b_1}\tau'\right)=\pi^{\floor{\frac{a_1+b_1}n}}.\]
Now, we use \autoref{lem:explicitdimshift} to move down to $H^1(G,\op{Hom}_\ZZ(I_G,ML^\times))$ as
\[\delta^{-1}(\op{Inf}c_{M/K})\left(\sigma_K^{a_1}\tau\right)\left(\sigma_K^{b_1}\tau'-1\right)=\sigma_K^{b_1}\tau'\cdot (\op{Inf}u_{M/K})\left(\sigma_K^{[-b_1]}(\tau')^{-1},\sigma_K^{a_1}\tau\right)=\pi^{\floor{\frac{a_1+[-b_1]}n}},\]
where $[k]$ denote the integer $0\le[k]<n$ such that $k\equiv[k]\pmod n$.

It will be helpful to see explicitly that the restriction to $H=\langle\sigma_k^f\rangle$ is a coboundary. That is, we need to find $b\in\op{Hom}_\ZZ(I_G,ML^\times)$ such that
\[\delta^{-1}(\op{Inf}u_{M/K})\left(\sigma_K^{fa_1}\right)=\frac{\sigma_K^{fa_1}\cdot b}b.\]
Because $I_G$ is freely generated by elements of the form $g-1$ for $g\in G$, it suffices to plug in some arbitrary $\sigma_K^{b_1}\tau'-1$, which we see requires
\begin{align*}
	\pi^{\floor{\frac{fa_1+[-b_1]}n}} &= \frac{\big(\sigma_K^{fa_1}\cdot b\big)\left(\sigma_K^{b_1}\tau'-1\right)}{b\left(\sigma_K^{b_1}\tau'-1\right)} \\
	&= \frac{\sigma_K^{fa_1} b\left(\sigma_K^{b_1-fa_1}\tau'-1\right)}{\sigma_K^{fa_1} b\left(\sigma_K^{-fa_1}-1\right)b\left(\sigma_K^{b_1}\tau'-1\right)}.
\end{align*}
We can see that $b$ should not depend on $\tau'$, so we define $\hat b\left(\sigma_K^a\right)=b\left(\sigma_K^a\tau'-1\right)$; the above is then equivalent to
\begin{align*}
	\pi^{\floor{\frac{fa_1+[-b_1]}n}} &= \frac{\sigma_K^{fa_1}\hat b\left(\sigma_K^{b_1-fa_1}\right)}{\sigma_K^{fa_1}\hat b\left(\sigma_K^{-fa_1}\right)\hat b\left(\sigma_K^{b_1}\right)} \\
	\pi^{\floor{\frac{fa_1+b_1}n}} &= \frac{\hat b\left(\sigma_K^{-b_1-fa_1}\right)}{\hat b\left(\sigma_K^{-fa_1}\right)\sigma_K^{-fa_1}\hat b\left(\sigma_K^{-b_1}\right)},
\end{align*}
where we have negated $b_1$ in the last step. At this point, the right-hand side will look a lot more natural if we set $\tau_K\coloneqq\sigma_K^{-1}$, which turns this into
\[\frac{\hat b\left(\tau_K^{fa_1}\right)\tau_K^{fa_1}\hat b\left(\tau_K^{b_1}\right)}{\hat b\left(\tau_K^{b_1fa_1}\right)} = (1/\pi)^{\floor{\frac{fa_1+b_1}n}}\]
after taking reciprocals. Thus, we see that $\hat b$ should be counting carries of $\tau_K$s. With this in mind, we let $\varpi$ be a uniformizer of $K_{\pi,\nu}$ and note that $\varpi$ is also a uniformizer of $L$ because $L/K_{\pi,\nu}$ is an unramified extension. It follows that
\[\varpi^{[ML:L]}\in\op N_{ML/L}\left(ML^\times\right).\]
Further, $\varpi^{[ML:L]}$ has the same absolute value as $\pi$ because $K_{\pi,\nu}/K$ is a totally ramified extension of degree $[K_{\pi,\nu}:K]=[ML:M]=[ML:L]$. Thus, $\pi/\varpi^{[ML:L]}$ and therefore $\pi$ is a norm in $\op N_{ML/L}(ML^\times)$ because $ML/L$ is unramified and so $\mathcal O_L^\times\subseteq\op N_{ML/L}(ML^\times)$. Thus, we find $\gamma\in ML^\times$ such that
\[\op N_{ML/L}(\gamma)=\pi.\]
The point of doing all of this is so that we can codify our carrying by writing
\[\hat b\left(\tau_K^a\right)\coloneqq\prod_{i=0}^{\floor{a/f}-1}\tau_K^{if}(\gamma)^{-1}.\]
Tracking out $\hat b$ backwards to $b$, our desired $b\in\op{Hom}_\ZZ(I_G,ML^\times)$ is given by
\[\boxed{b\left(\sigma_K^{a}\tau-1\right)=\prod_{i=0}^{\floor{[-a]/f}-1}\sigma_K^{-if}(\gamma)^{-1}}.\]
We take a moment to write out $c\coloneqq\delta^{-1}(\op{Inf}c_{M/K})/db$, which looks like
\begin{align*}
	c\left(\sigma_K^{a_1}\tau\right)\left(\sigma_K^{b_1}\tau'-1\right) &= \frac{\delta^{-1}(\op{Inf}c_{M/K})}{db}\left(\sigma_K^{a_1}\tau\right)\left(\sigma_K^{b_1}\tau'-1\right) \\
	&= \frac{\delta^{-1}(\op{Inf}c_{M/K})\left(\sigma_K^{a_1}\tau\right)\left(\sigma_K^{b_1}\tau'-1\right)}{\left(\sigma_K^{a_1}\tau\cdot b\right)\left(\sigma_K^{b_1}\tau'-1\right)/b\left(\sigma_K^{b_1}\tau'-1\right)} \\
	&= \frac{\pi^{\floor{(a_1+[-b_1])/n}}}{\sigma_K^{a_1}\tau\cdot b\left(\sigma_K^{b_1-a_1}\tau'\tau^{-1}-\sigma_K^{-a_1}\tau^{-1}\right)/b\left(\sigma_K^{b_1}\tau'-1\right)} \\
	&= \pi^{\floor{(a_1+[-b_1])/n}}\cdot\hat b\left(\sigma_K^{b_1}\right)\cdot\sigma_K^{a_1}\tau\left(\frac{\hat b\left(\sigma_K^{-a_1}\right)}{\hat b\left(\sigma_K^{b_1-a_1}\right)}\right).
\end{align*}
Before proceeding, we discuss a few special cases.
\begin{itemize}
	\item Taking $\sigma_K^{a_1}\tau=\tau_i$ for some $\tau_i$, we get
	\begin{align*}
		c\left(\tau_i\right)\left(\sigma_K^{b_1}\tau'-1\right) &= \pi^{\floor{(0+[-b_1])/n}}\cdot\hat b\left(\sigma_K^{b_1}\right)\cdot\tau_i\left(\frac{1}{\hat b\left(\sigma_K^{b_1}\right)}\right) \\
		&= \hat b\left(\sigma_K^{b_1}\right)/\tau_i\hat b\left(\sigma_K^{b_1}\right).
	\end{align*}
	In particular, $c\left(\tau_i\right)\left(\sigma_K^{-1}-1\right)=1$, provided that $f>1$. Additionally, $c(\tau_i)\left(\tau'-1\right)=1$.
	
	Our general theory says that $h\mapsto c(\tau_i)(h-1)$ is a $1$-cocycle in $Z^1(H,ML^\times)$ (though we could also check this directly), so Hilbert's Theorem 90 promises us a magical element $\eta_i\in ML^\times$ such that
	\[\frac{\sigma_K^{fb_1}\eta_i}{\eta_i}=\frac{\hat b\left(\sigma_K^{fb_1}\right)}{\tau_i\hat b\left(\sigma_K^{fb_1}\right)}\]
	for all $\sigma_K^{fb_1}\in H$. This condition will be a little clearer if we write everything in terms of $\tau_K\coloneqq\sigma_K^{-1}$, which transforms this into
	\[\frac{\tau_K^{fb_1}\eta_i}{\eta_i}=\frac{\hat b\left(\tau_K^{-fb_1}\right)}{\tau_i\hat b\left(\tau_K^{-fb_1}\right)}=\prod_{i=0}^{b_1-1}\frac{\tau_K^{if}(\gamma^{-1})}{\tau_i\tau_K^{if}(\gamma^{-1})}=\prod_{i=0}^{b_1-1}\frac{\tau_i\tau_K^{if}(\gamma)}{\tau_K^{if}(\gamma)}.\]
	Because we are dealing with a cyclic group $H$, it is not too hard to see that it suffices merely for $b_1=1$ to hold, so our magical element $\eta_i$ merely requires
	\[\boxed{\frac{\sigma_K^{-f}\left(\eta_i\right)}{\eta_i}=\frac{\tau_i(\gamma)}{\gamma}}\]
	after inverting $\tau_K$ back to $\sigma_K$.
	\item Taking $\sigma_K^{a_1}\tau=\sigma_K$, we get
	\begin{align*}
		c\left(\sigma_K\right)\left(\sigma_K^{b_1}\tau'-1\right) &= \pi^{\floor{(1+[-b_1])/n}}\cdot\hat b\left(\sigma_K^{b_1}\right)\cdot\sigma_K\left(\frac{\hat b\left(\sigma_K^{-1}\right)}{\hat b\left(\sigma_K^{b_1-1}\right)}\right).
	\end{align*}
	In particular, $\sigma_K^{b_1}\tau'=\tau_i^{-1}$ will give $c(\sigma_K)\left(\tau_i^{-1}-1\right)=1$. We will also want $c(\sigma_K)\left(\sigma_K^{-b_1}-1\right)$ for $0\le b_1<f$. We now have two cases.
	\begin{itemize}
		\item Suppose that $L/K$ is not totally ramified so that $f>1$. Using the fact that $f<n$ and $f>1$, it is not too hard to see that everything will cancel down to $1$ except in the case where $b_1=f-1$, where we get
		\[c(\sigma_K)\left(\sigma_K^{-(f-1)}-1\right)=\sigma_K\left(\frac1{\hat b\left(\sigma_K^{-f}\right)}\right)=\sigma_K(\gamma).\]
		\item Otherwise, our extension is totally ramified so that $f=1$. Here, $b_1=0$ is forced, so we are computing $c(\sigma_K)(\tau'-1)=1$. (Our extension being unramified promises $n>1$.)
	\end{itemize}
	Continuing as before, our general theory says that $h\mapsto c(\sigma_K)(h-1)$ is a $1$-cocycle in $Z^1(H,ML^\times)$, though again we could just check this directly. It follows that Hilbert's Theorem 90 promises us a magical element $\eta_K\in ML^\times$ such that
	\[\frac{\sigma_K^{fb_1}\eta_K}{\eta_K}=\pi^{\floor{(1+[-fb_1])/n}}\cdot\hat b\left(\sigma_K^{fb_1}\right)\cdot\sigma_K\left(\frac{\hat b\left(\sigma_K^{-1}\right)}{\hat b\left(\sigma_K^{fb_1-1}\right)}\right)\]
	for all $\sigma_K^{fb_1}\in H$. To simplify this condition, we once again split into two cases.
	\begin{itemize}
		\item Suppose that $L/K$ is not totally ramified so that $f>1$. Using $f>1$, this collapses down to
		\[\frac{\sigma_K^{fb_1}\eta_K}{\eta_K}=\frac{\hat b\left(\sigma_K^{fb_1}\right)}{\sigma_K\hat b\left(\sigma_K^{fb_1-1}\right)}.\]
		As before, this condition will be a little clearer if we set $\tau_K\coloneqq\sigma_K^{-1}$, which turns the condition into
		\[\frac{\tau_K^{fb_1}\eta_K}{\eta_K}=\frac{\hat b\left(\tau_K^{fb_1}\right)}{\sigma_K\hat b\left(\tau_K^{fb_1+1}\right)}=\prod_{i=0}^{b_1-1}\frac{\tau_K^{if}(\gamma^{-1})}{\sigma_K\tau_K^{if}(\gamma^{-1})}=\prod_{i=0}^{b_1-1}\frac{\sigma_K\tau_K^{if}(\gamma)}{\tau_K^{if}(\gamma)}.\]
		(Notably, $\hat b\left(\tau_K^{fb_1}\right)=\hat b\left(\tau_K^{fb_1+1}\right)$ because $f>1$.) Again, because $H$ is cyclic generated by $\tau_K^f$, an induction shows that it suffices to check this condition for $b_1=1$, which means that our magical element $\eta_K\in ML^\times$ is constructed so that
		\[\boxed{\frac{\sigma_K^{-f}\left(\eta_K\right)}{\eta_K}=\frac{\sigma_K(\gamma)}{\gamma}}\]
		where we have again inverted back from $\tau_K$ to $\sigma_K$.
		\item Suppose that $L/K$ is totally ramified so that $f=1$. Switching over to $\tau_K$ as usual, we can evaluate
		\[\frac{\tau_K^{b_1}\eta_K}{\eta_K} = \pi^{\floor{(1+b_1)/n}}\cdot\hat b\left(\tau_K^{b_1}\right)\cdot\tau_K^{-1}\left(\frac{\hat b\left(\tau_K\right)}{\hat b\left(\tau_K^{b_1+1}\right)}\right)=\pi^{\floor{(1+b_1)/n}}\cdot\tau_K^{-1}\left(\frac{\tau_K\hat b\big(\tau_K^{b_1}\big)\cdot\hat b(\tau_K)}{\hat b\big(\tau_K^{b_1+1}\big)}\right).\]
		When $b_1<n-1$, everything will cancel out. When $b_1=n-1$, then $\hat b\big(\tau_K^{b_1+1}\big)=1$ while $\tau_K\hat b\big(\tau_K^{b_1}\big)\cdot\hat b(\tau_K)$ collects to $\prod_{i=0}^{n-1}\tau_K(\gamma)^{-1}=\pi^{-1}$, so everything will still cancel out. So we want
		\[\frac{\tau_K^{b_1}\eta_K}{\eta_K}=1,\]
		for which $\eta_K=1$ suffices. As usual, it suffices to just look at $b_1=1$ by an induction, so we are asking for $\tau_K\eta_K/\eta_K=1$.
	\end{itemize}
	\item We will not actually need a more concrete description of this, but we remark that we can run the same story for any $g\in G$ through to get an element $\eta_g\in ML^\times$ such that
	\[\frac{\sigma_K^{fb_1}\eta_g}{\eta_g}=\frac1{c(g)(\sigma_K^{fb_1}-1)}\]
	for any $\sigma_K^{fb_1}\in H$. As usual, this follows from our general theory.
\end{itemize}
We are now ready to describe the local fundamental class. Piecing what we have so far, we know from \autoref{lem:explicitresinf} that we can write
\[c_{L/K}(g,g')\coloneqq g\cdot c(g')\left(g^{-1}-1\right)\cdot\frac{g\eta_{g'}\cdot \eta_{g}}{\eta_{gg'}}.\]
This will be explicit enough for us.

\subsubsection{Computing the Tuple}
We now use our computation of $c_{L/K}$ representing $u_{L/K}$ from the previous subsection to compute the tuple corresponding to $c_{L/K}$. Here are the values that we care about for our specific computation; for consistency, we set $\tau_0\coloneqq\sigma_K$ and $n_0\coloneqq f$ to be the order of $\tau_0$.
\begin{itemize}
	\item We write
	\begin{align*}
		c_{L/K}(\sigma_K,\tau_i) &= \sigma_Kc(\tau_i)\left(\sigma_K^{-1}-1\right)\cdot\frac{\sigma_K \eta_i\cdot \eta_K}{\eta_{\sigma_K\tau_i}} \\
		&= \frac{\sigma_K \eta_i\cdot \eta_K}{\eta_{\sigma_K\sigma_x}}.
	\end{align*}
	\item We write
	\begin{align*}
		c_{L/K}(\tau_i,\sigma_K) &= \tau_ic(\sigma_K)\left(\tau_i^{-1}-1\right)\cdot\frac{\tau_i\eta_K\cdot \eta_i}{\eta_{\sigma_x\sigma_K}} \\
		&= \frac{\tau_i\eta_K\cdot \eta_i}{\eta_{\sigma_x\sigma_K}}.
	\end{align*}
	\item In particular, we know that we can set $\beta_{i0}$ to
	\begin{align*}
		\beta_{i0} \coloneqq{}& \frac{c_{L/K}(\tau_i,\sigma_K)}{c_{L/K}(\sigma_K,\tau_i)} \\
		={}& \frac{\tau_i\eta_K\cdot \eta_i/\eta_{\sigma_x\sigma_K}}{\sigma_K \eta_i\cdot \eta_K/\eta_{\sigma_K\sigma_x}} \\
		\Aboxed{\beta_{i0}={}& \frac{\eta_i}{\sigma_K\left(\eta_i\right)}\cdot\frac{\tau_i\left(\eta_K\right)}{\eta_K}}.
	\end{align*}
	% As a sanity check, we can hit this $\beta$ with $\sigma_K^{-f}$ to show that $\beta\in(ML)^H=L$; namely, $\sigma_K^{-f}\eta_{c(\sigma_K)}=\frac{\sigma_K\alpha}\alpha\cdot \eta_{c(\sigma_K)}$ and $\sigma_K^{-f}\eta_{c\sigma(x)}=\frac{\sigma_x\alpha}\alpha\cdot \eta_{c(\sigma_x)}$ by construction, so we can see that everything will appropriately cancel out.
	\item We write
	\begin{align*}
		c_{L/K}(\tau_i,\tau_j) &= \tau_ic(\tau_j)\left(\tau_j^{-1}-1\right)\cdot\frac{\tau_i\eta_j\cdot \eta_{i}}{\eta_{\tau_i\tau_j}} \\
		&= \frac{\tau_i\eta_j\cdot \eta_{i}}{\eta_{\tau_i\tau_j}}.
	\end{align*}
	\item Thus, for $i>j>0$, we can set $\beta_{ij}$ to
	\begin{align*}
		\beta_{ij} \coloneqq{}& \frac{c_{L/K}(\tau_i,\tau_j)}{c_{L/K}(\tau_j,\tau_i)} \\
		={}& \frac{\tau_i\eta_j\cdot \eta_{i}/\eta_{\tau_i\tau_j}}{\tau_j\eta_i\cdot \eta_{j}/\eta_{\tau_i\tau_j}} \\
		\Aboxed{\beta_{ij}={}& \frac{\eta_{i}}{\tau_j\eta_i}\cdot\frac{\tau_i\eta_j}{\eta_{j}}}.
	\end{align*}
	\item For $\alpha_0$, our element is given by
	\begin{align*}
		\alpha_0 \coloneqq{}& \prod_{i=0}^{f-1}c_{L/K}\left(\sigma_K^i,\sigma_K\right) \\
		={}& \prod_{i=0}^{f-1}\left(\sigma_K^ic\left(\sigma_K\right)\left(\sigma_K^{-i}-1\right)\cdot\frac{\sigma_K^i\eta_K\cdot \eta_{\sigma_K^i}}{\eta_{\sigma_K^{i+1}}}\right).
	\end{align*}
	Recall from our general theory that $\eta_g$ only depends on the coset of $g$ in $G/H$, so we see that the product of the quotients $\eta_{\sigma_K^i}/\eta_{\sigma_K^{i+1}}$ will cancel out.
	
	To finish the computation, we have two cases.
	\begin{itemize}
		\item If $L/K$ is not totally ramified, we know from our computation that this is $1$ until $i=f-1$, which gives $\sigma_K(\gamma)$. As such, we collapse down to
		\[\alpha_0=\sigma_K^f(\gamma)\cdot\prod_{i=0}^{f-1}\sigma_K^i\left(\eta_K\right)=\boxed{\sigma_K^f(\gamma)\cdot\sigma_K^{(f)}(\eta_K)}.\]
		\item If $L/K$ is totally ramified so that $f=1$, then we computed $c(\sigma_K)(\sigma_K^{-i}-1)=0$ always---note we only have an $i=0$ term---so we are left with just $\boxed{\eta_K=\sigma_K^{(f)}\eta_K}$.
	\end{itemize}
	\item For $\alpha_i$ with $i>0$, our element is given by
	\begin{align*}
		\alpha_i \coloneqq{}& \prod_{p=0}^{n_i-1}c_{L/K}\left(\tau_i^p,\tau_i\right) \\
		={}& \prod_{p=0}^{n_i-1}\tau_i^pc(\tau_i)\left(\tau_i^{-p}-1\right)\cdot\frac{\tau_i^p\eta_i\cdot \eta_{\tau_i^p}}{\eta_{\tau_i^{p+1}}}.
	\end{align*}
	Recalling that $\tau_i$ has order $n_i$, our quotient term $\eta_{\tau_i^i}/\eta_{\tau_i^{i+1}}$ will again cancel out. Additionally, the cocycle $c$ always spits out $1$ on these inputs, so we are left with
	\[\alpha_i=\prod_{p=0}^{n_i-1}\tau_i^p\left(\eta_i\right)=\boxed{\tau_i^{(n_i)}(\eta_i)}.\]
\end{itemize}
We summarize the results above in the following theorem.
\begin{theorem} \label{thm:fundtriple}
	Fix everything as in the set-up. Then there exists some $\gamma\in ML^\times$ such that $\op N_{ML/L}(\gamma)=\pi$ and elements in $\eta_K,\eta_i\in ML^\times$ for $1\le i\le t$ such that
	\[\frac{\sigma_K^{-f}\left(\eta_K\right)}{\eta_K}=\begin{cases}
		\sigma_K(\gamma)/\gamma & L/K\text{ not totally ramified}, \\
		1 & L/K\text{ totally ramified},
	\end{cases}\qquad\text{and}\qquad\frac{\sigma_K^{-f}\left(\eta_i\right)}{\eta_i}=\frac{\tau_i(\gamma)}{\gamma}.\]
	Then the tuple given by
	\[\alpha_i\coloneqq\begin{cases}
		\sigma_K^f(\gamma)\cdot\sigma_K^{(f)}(\eta_K) & i=0,\,L/K\text{ not totally ramified}, \\
		\tau_i^{(n_i)}(\eta_i) & \text{else},
	\end{cases}\qquad\text{and}\qquad\beta_{ij}\coloneqq\frac{\eta_{i}}{\tau_j\eta_i}\cdot\frac{\tau_i\eta_j}{\eta_{j}},\]
	where $n_0=f$ and $\tau_0=\sigma_K$, corresponds to the fundamental class $u_{L/K}\in H^2(\op{Gal}(L/K),L^\times)$.
\end{theorem}
We remark that we can replace $\sigma_K^f(\gamma)$ with merely $\gamma$ (which still has norm $p$) while keeping all other variables the same; this gives us the following slightly prettier presentation. Note that we have multiplied the equations for $\eta_\bullet$ by $\sigma_K^f$ on both sides.
\begin{corollary} \label{cor:fundtriple}
	Fix everything as in the set-up. Then there exists some $\gamma\in ML^\times$ such that $\op N_{ML/L}(\gamma)=\pi$ and elements in $\eta_K,\eta_i\in ML^\times$ (for $1\le i\le t$) such that
	\[\frac{\eta_K}{\sigma_K^{f}\left(\eta_K\right)}=\begin{cases}
		\sigma_K(\gamma)/\gamma & L/K\text{ not totally ramified}, \\
		1 & L/K\text{ totally ramified},
	\end{cases}\qquad\text{and}\qquad\frac{\eta_i}{\sigma_K^{f}\left(\eta_i\right)}=\frac{\tau_i(\gamma)}{\gamma}.\]
	Then the tuple given by
	\[\alpha_i\coloneqq\begin{cases}
		\gamma\cdot\sigma_K^{(f)}(\eta_K) & i=0,\,L/K\text{ not totally ramified}, \\
		\tau_i^{(n_i)}(\eta_i) & \text{else},
	\end{cases}\qquad\text{and}\qquad\beta_{ij}\coloneqq\frac{\eta_{i}}{\tau_j\eta_i}\cdot\frac{\tau_i\eta_j}{\eta_{j}},\]
	where $n_0=f$ and $\tau_0=\sigma_K$, corresponds to the fundamental class $u_{L/K}\in H^2(\op{Gal}(L/K),L^\times)$.
\end{corollary}
For brevity later on, we will give a name to these conditions.
\begin{definition}
	Fix an extension $L/K$. The $\{\sigma_i\}_{i=1}^m$-tuples constructed in \autoref{cor:fundtriple} will be called \textit{fundamental tuples}.
\end{definition}
We will show shortly that fundamental tuples actually give the entire equivalence class of $\{\sigma_i\}_{i=1}$-tuples associated to the fundamental class.
\begin{remark}
	This result is essentially a stronger version of Dwork's theorem \cite[Theorem~XIII.2]{local-fields}. Namely, Dwork and Serre are interested in computing the reciprocity map, which roughly means we only want access to the $\alpha$s, but above we are interested in computing the full fundamental class.
\end{remark}
\begin{remark}
	The $\eta_\bullet$s have a degree of freedom in that these elements are unique only up to multiplication by a nonzero element of $ML^{\langle\sigma_K^f\rangle}=L$. As the $\eta_\bullet$s vary (with $\gamma$ fixed), it is not too hard to see directly from the formulae that we will encounter a full equivalence class of tuples. We will not write this out.
\end{remark}
\begin{remark}
	Even when $L/K$ is unramified, which we technically disallowed for our computation, we can see that $M=L$ and then $\gamma=\pi$ so that $\eta_K=1$ are all forced, giving $\alpha_0=\pi$. So indeed, the above formulation does work for all abelian extensions $L/K$.
\end{remark}

% \subsubsection{Checks}
% In this section we run some checks and discuss some consequences of \autoref{thm:fundtriple}, in the form of \autoref{cor:fundtriple}. For these results, we recall that we set $L\coloneqq\QQ_p(\zeta_N)$ and $L_1\coloneqq\QQ_p(\zeta_{p^\nu})$ and $L_2\coloneqq\QQ_p(\zeta_m)$ so that $\overline\sigma_K=\sigma_K|_{L_1}$ generates $\op{Gal}(L/L_1)$ and $\sigma_x$ generates $\op{Gal}(L/L_2)$.

In the discussion which follows, we will make repeated use of the fact that (using notation of \autoref{cor:fundtriple})
\[\sigma_K^f\left(\eta_K\right)=\frac{\gamma}{\sigma_K(\gamma)}\cdot \eta_K\qquad\text{and}\qquad\sigma_K^f\left(\eta_i\right)=\frac\gamma{\tau_i(\gamma)}\cdot \eta_i.\]
And here are our checks; we start by showing that our elements are in the right field.
\begin{lemma}
	Fix a tuple $(\alpha_0,\alpha_i),(\beta_{ij})$ as in \autoref{cor:fundtriple}. Then the following are true.
	\begin{listalph}
		\item $\alpha_0\in K_{\pi,\nu}^\times$.
		\item $\alpha_i\in L_i^\times$ for each $i\ge1$.
		\item $\beta_{ij}\in L^\times$ for each $i>j$.
	\end{listalph}
\end{lemma}
\begin{proof}
	We run the checks one at a time.
	\begin{listalph}
		\item It suffices to show that $\alpha_0$ is fixed by $\op{Gal}(M/K_{\pi,\nu})=\langle\sigma_K\rangle$. As such, we simply compute
		\begin{align*}
			\sigma_K(\alpha_0) &= \sigma_K\left(\gamma\cdot\prod_{i=0}^{f-1}\sigma_K^i\left(\eta_K\right)\right) \\
			&= \sigma_K(\gamma)\cdot\prod_{i=0}^{f-1}\sigma_K^{i+1}\left(\eta_K\right) \\
			&= \sigma_K(\gamma)\cdot\sigma_K^f\left(\eta_K\right)\prod_{i=1}^{f-1}\sigma_K^{i+1}\left(\eta_K\right) \\
			&= \gamma\cdot \eta_K\prod_{i=1}^{f-1}\sigma_K^{i+1}\left(\eta_K\right) \\
			&= \gamma\cdot\prod_{i=0}^{f-1}\sigma_K^{i+1}\left(\eta_K\right) \\
			&= \alpha_0.
		\end{align*}
		\item It suffices to show that $\alpha_i$ is fixed by $\op{Gal}(M/L_i)=\langle\sigma_K^f,\tau_i\rangle$. On one hand,
		\begin{align*}
			\sigma_K^f(\alpha_i) &= \sigma_K^f\left(\prod_{p=0}^{n_i-1}\tau_i^p\left(\eta_i\right)\right) \\
			&= \prod_{p=0}^{n_i-1}\tau_i^p\left(\sigma_K^f\eta_i\right) \\
			&= \left(\prod_{p=0}^{n_i-1}\tau_i^p\left(\frac{\gamma}{\tau_i(\gamma)}\right)\right)\cdot\left(\prod_{p=0}^{n_i-1}\tau_i^p\left(\eta_i\right)\right) \\
			&= \left(\prod_{p=0}^{n_i-1}\frac{\tau_i^p(\gamma)}{\tau_i^{p+1}(\gamma)}\right)\cdot\alpha_i \\
			&= \alpha_i,
		\end{align*}
		where the product telescopes because $\tau_i$ has order $n_i$.

		On the other hand,
		\begin{align*}
			\tau_i(\alpha_i) &= \tau_i\left(\prod_{p=0}^{n_i-1}\tau_i^p\left(\eta_i\right)\right) \\
			&= \prod_{p=0}^{n_i-1}\tau_i^{p+1}\left(\eta_i\right) \\
			&= \prod_{p=0}^{n_i-1}\tau_i^p\left(\eta_i\right),
		\end{align*}
		where we have again used the fact that $\tau_i$ has order $n_i$. This last product is $\alpha_i$, so we are done.
		\item It suffices to show that $\beta_{ij}$ is fixed by $\op{Gal}(M/L)=\langle\sigma_K^f\rangle$. Applying force, we see
		\begin{align*}
			\sigma_K^f(\beta_{ij}) &= \sigma_K^f\left(\frac{\eta_{i}}{\tau_j\eta_i}\cdot\frac{\tau_i\eta_j}{\eta_{j}}\right) \\
			&= \frac{\sigma_K^f\eta_{i}}{\tau_j\sigma_K^f\eta_i}\cdot\frac{\tau_i\sigma_K^f\eta_j}{\sigma_K^f\eta_{j}} \\
			&= \frac{\eta_i\cdot\gamma/\tau_i\gamma}{\tau_j\eta_i\cdot\tau_j\gamma/\tau_i\tau_j\gamma}\cdot
			\frac{\eta_j\cdot\tau_i\gamma/\tau_i\tau_j\gamma}{\eta_j\cdot\gamma/\tau_j\gamma} \\
			&= \frac{\eta_i}{\tau_j\eta_i}\cdot
			\frac{\eta_j}{\eta_j} \\
			&= \beta_{ij}.
		\end{align*}
	\end{listalph}
	The above checks complete the proof.
\end{proof}
Next we show the relations between the $\alpha$s and $\beta$s.
\begin{lemma}
	Fix a tuple $(\alpha_0,\alpha_i),(\beta_{ij})$ as in \autoref{cor:fundtriple}. Then the following are true.
	\begin{listalph}
		\item $\op N_{L/L_i}(\beta_{ij})=\alpha_i/\tau_j\alpha_i$ for $i>j\ge0$.
		\item $\op N_{L/L_0}(\beta_{i0}^{-1})=\alpha_0/\tau_i\alpha_0$.
		\item $\op N_{L/L_j}(\beta_{ij}^{-1})=\alpha_j/\tau_i\alpha_j$ for $i>j>0$.
	\end{listalph}
\end{lemma}
\begin{proof}
	We go one at a time.
	\begin{listalph}
		\item Note $\op{Gal}(L/L_i)=\langle\tau_i\rangle$, so we compute
		\begin{align*}
			\op N_{L/L_i}(\beta_{ij}) &= \prod_{p=0}^{n_i-1}
			\tau_i^p(\beta_{ij}) \\
			&= \prod_{p=0}^{n_i-1}
			\tau_i^p\left(\frac{\eta_{i}}{\tau_j\eta_i}\cdot\frac{\tau_i\eta_j}{\eta_{j}}\right) \\
			&= \prod_{p=0}^{n_i-1}
			\frac{\tau_i^p\eta_{i}}{\tau_j\tau_i^p\eta_i}\cdot\prod_{p=0}^{n_i-1}\frac{\tau_i^{p+1}\eta_j}{\tau_i^p\eta_{j}} \\
			&= \left(\prod_{p=0}^{n_i-1}\tau_i^p\eta_{i}\bigg/\tau_j\prod_{p=0}^{n_i-1}\tau_i^p\eta_i\right)\cdot\frac{\tau_i^{n_i}\eta_j}{\eta_{j}},
		\end{align*}
		which collapses into $\alpha_i/\tau_j\alpha_i$, as needed.
		\item Note $\op{Gal}(L/L_0)=\langle\overline\sigma_K\rangle$. In particular, $\overline\sigma_K$ has order $f$, so we can just compute out
		\begin{align*}
			\op N_{L/L_0}(\beta_{i0}) &= \prod_{p=0}^{f-1}\sigma_K^p(\beta_{i0}) \\
			&= \prod_{p=0}^{f-1}\sigma_K^p\left(\frac{\eta_i}{\sigma_K\eta_i}\cdot\frac{\tau_i\eta_K}{\eta_K}\right) \\
			&= \prod_{p=0}^{f-1}\frac{\sigma_K^p\eta_i}{\sigma_K^{p+1}\eta_i}\cdot\prod_{p=0}^{f-1}\frac{\tau_i\sigma_K^p\eta_K}{\sigma_K^p\eta_K} \\
			&= \frac{\eta_i}{\sigma_K^f\eta_i}\cdot\prod_{p=0}^{f-1}\tau_i\sigma_K^p\eta_K\bigg/\prod_{p=0}^{f-1}\sigma_K^p\eta_K \\
			&= \tau_i\Bigg(\gamma\prod_{p=0}^{f-1}\sigma_K^p\eta_K\Bigg)\bigg/\Bigg(\gamma\prod_{p=0}^{f-1}\sigma_K^p\eta_K\Bigg),
		\end{align*}
		which is what we wanted after taking reciprocals.
		\item This time around, we have $\op{Gal}(L/L_j)=\langle\tau_j\rangle$. As such, we proceed similarly to (a), writing
		\begin{align*}
			\op N_{L/L_j}(\beta_{ij}) &= \prod_{p=0}^{n_j-1}
			\tau_j^p(\beta_{ij}) \\
			&= \prod_{p=0}^{n_j-1}
			\tau_j^p\left(\frac{\eta_{i}}{\tau_j\eta_i}\cdot\frac{\tau_i\eta_j}{\eta_{j}}\right) \\
			&= \prod_{p=0}^{n_j-1}
			\frac{\tau_j^p\eta_{i}}{\tau_j^{p+1}\eta_i}\cdot\prod_{p=0}^{n_j-1}\frac{\tau_i\tau_j^p\eta_j}{\tau_j^p\eta_{j}} \\
			&= \frac{\eta_i}{\tau_j^{n_j}\eta_i}\cdot\left(\tau_i\prod_{p=0}^{n_j-1}\tau_j^p\eta_j\bigg/\prod_{p=0}^{n_j-1}\tau_j^p\eta_j\right),
		\end{align*}
		which again collapses into $\tau_i\alpha_j/\alpha_j$. Taking reciprocals finishes.
	\end{listalph}
	The above checks complete the proof.
\end{proof}
Lastly, here are the relations between the $\beta$s.
\begin{lemma}
	Fix a tuple $(\alpha_0,\alpha_i),(\beta_{ij})$ as in \autoref{cor:fundtriple}. Then, for $i>j>k$, we have
	\[\frac{\tau_j\beta_{ik}}{\beta_{ik}}=\frac{\tau_k\beta_{ij}}{\beta_{ij}}\cdot\frac{\tau_i\beta_{jk}}{\beta_{jk}}.\]
\end{lemma}
\begin{proof}
	As usual, we apply force. Note
	\begin{align*}
		\frac{\tau_k\beta_{ij}}{\beta_{ij}}\cdot\frac{\tau_i\beta_{jk}}{\beta_{jk}} &= \frac{\displaystyle\frac{\tau_k\eta_i}{\tau_k\tau_j\eta_i}\cdot\frac{\tau_k\tau_i\eta_j}{\tau_k\eta_j}}
		{\displaystyle\frac{\eta_i}{\tau_j\eta_i}\cdot\frac{\tau_i\eta_j}{\eta_j}}\cdot
		\frac{\displaystyle\frac{\tau_i\eta_j}{\tau_i\tau_k\eta_j}\cdot\frac{\tau_i\tau_j\eta_k}{\tau_i\eta_k}}
		{\displaystyle\frac{\eta_j}{\tau_k\eta_j}\cdot\frac{\tau_j\eta_k}{\eta_k}} \\
		&= \frac{\tau_k\eta_i}{\tau_k\tau_j\eta_i}\cdot\frac{\tau_k\tau_i\eta_j}{\tau_k\eta_j}\cdot
		\frac{\tau_j\eta_i}{\eta_i}\cdot\frac{\eta_j}{\tau_i\eta_j}\cdot
		\frac{\tau_i\eta_j}{\tau_i\tau_k\eta_j}\cdot\frac{\tau_i\tau_j\eta_k}{\tau_i\eta_k}\cdot
		\frac{\tau_k\eta_j}{\eta_j}\cdot\frac{\eta_k}{\tau_j\eta_k} \\
		&= \frac{\tau_k\eta_i}{\tau_k\tau_j\eta_i}\cdot
		\frac11\cdot
		\frac{\tau_j\eta_i}{\eta_i}\cdot
		\frac11\cdot
		\frac11\cdot
		\frac{\tau_i\tau_j\eta_k}{\tau_i\eta_k}\cdot
		\frac11\cdot\frac{\eta_k}{\tau_j\eta_k} \\
		&= \frac{\tau_j\eta_i}{\tau_k\tau_j\eta_i}\cdot
		\frac{\tau_i\tau_j\eta_k}{\tau_j\eta_k}\cdot
		\frac{\tau_k\eta_i}{\eta_i}\cdot
		\frac{\eta_k}{\tau_i\eta_k},
	\end{align*}
	which is what we wanted.
\end{proof}

% \subsubsection{Consequences}
% \begin{warn}
	The following section does not use the notation of the rest of the article.
\end{warn}
With some checks out of the way, here are some actual consequences. To begin, we state Hilbert's Theorem 90.
\begin{lemma} \label{lem:hilbert90}
	Suppose that $L/K$ is a (finite) cyclic extension of fields such that $\Gamma\coloneqq\op{Gal}(L/K)$ is generated by $\sigma\in\Gamma$. Given some $\alpha\in L^\times$ such that $\op N(\alpha)=1$, there exists $\beta_0\in L^\times$ such that $\alpha=\beta_0/\sigma\beta_0$. In fact, this $\beta_0$ is unique ``up to a multiple in $K^\times$'' in the sense that
	\[\left\{\beta\in L^\times:\alpha=\beta/\sigma\beta\right\}=\left\{x\beta_0:x\in K^\times\right\}.\]
\end{lemma}
\begin{proof}
	That such a $\beta_0$ exists follows directly from Hilbert's Theorem 90. For the last sentence, of course any $\beta\coloneqq x\beta_0\in L^\times$ with $x\in K^\times$ will have
	\[\frac\beta{\sigma\beta}=\frac{\beta_0}{\sigma\beta_0}=\alpha.\]
	In the other direction, if $\beta\in L^\times$ has $\beta/\sigma\beta=\alpha$, then
	\[\sigma(\beta/\beta_0)=(\sigma\beta)/(\sigma\beta_0)=\beta/\beta_0,\]
	so $\beta/\beta_0\in K^\times$ and $\beta=(\beta/\beta_0)\cdot\beta_0$.
\end{proof}
And here are some quick consequences of this.
\begin{cor}
	Fix everything as in the set-up, and fix $\alpha\in ML^\times$ such that $\op N_{ML/L}(\alpha)=p$. Choosing some $\sigma\in\{\sigma_K,\sigma_x\}$, the elements $\eta_{\sigma}$ satisfying
	\[\frac{\eta_{\sigma}}{\sigma_K^f\left(\eta_{\sigma}\right)}=\frac{\sigma(\alpha)}{\alpha}\]
	are unique up to a multiple in $L^\times$, in the sense of \autoref{lem:hilbert90}.
\end{cor}
\begin{proof}
	Note that $\op{Gal}(ML/L)=\langle\sigma_K^f\rangle$ is cyclic generated by $\sigma_K^f$ and $\op N_{ML/L}(\sigma\alpha/\alpha)=p/p=1$, so we may simply apply \autoref{lem:hilbert90} directly to get the result.
\end{proof}
We might be worried that our choice $\alpha$ is affecting the set of $\eta_{c(\sigma_K)}$ or $\eta_{c(\sigma_x)}$, but in fact they are not, more or less.
\begin{cor} \label{cor:updatealpha}
	Fix everything as in the set-up, and choose $\sigma\in\{\sigma_K,\sigma_x\}$. Given $\alpha\in ML^\times$ such that $\op N_{ML/L}(\alpha)=p$, define
	\[S_\alpha\coloneqq\left\{\eta_\sigma\in ML^\times:\frac{\eta_{\sigma}}{\sigma_K^f\left(\eta_{\sigma}\right)}=\frac{\sigma(\alpha)}{\alpha}\right\}.\]
	Then the set $S_\alpha$ is ``unique up to a multiple in $ML^\times$'' in the sense that two $\alpha,\alpha'\in ML^\times$ with $\op N_{ML/L}(\alpha)=\op N_{ML/L}(\alpha')=p$ have some $\chi\in ML^\times$ such that
	\[S_{\alpha}=\chi\cdot S_{\alpha'}\coloneqq\{\chi\cdot \eta_\sigma:\eta_\sigma\in S_{\alpha'}\}.\]
\end{cor}
\begin{proof}
	Suppose $\alpha,\alpha'\in ML^\times$ satisfy $\op N_{ML/L}(\alpha)=\op N_{ML/L}(\alpha')=p$. The key point is that
	\[\op N_{ML/L}(\alpha/\alpha')=p/p=1,\]
	so \autoref{lem:hilbert90} promises us some $\gamma\in ML^\times$ such that $\alpha/\alpha'=\gamma/\sigma_K^f(\gamma)$. As such, we see that
	\[\frac{\sigma(\alpha)}{\alpha}=\frac{\sigma(\alpha/\alpha')}{\alpha/\alpha'}\cdot\frac{\sigma(\alpha')}{\alpha'}=\frac{(\sigma\gamma/\gamma)}{\sigma_K^f(\sigma\gamma/\gamma)}\cdot\frac{\sigma(\alpha')}{\alpha'}.\]
	As such, we set $\chi\coloneqq(\sigma\gamma/\gamma)$.
	
	To finish, we check that $S_\alpha\subseteq \chi\cdot S_{\alpha'}$, and the other inclusion is similar. Well, if $\eta_\sigma\in S_{\alpha'}$, then
	\[\frac{\chi \eta_\sigma}{\sigma_K^f(\chi \eta_\sigma)}=\frac \chi {\sigma_K^f(\chi )}\cdot\frac{\eta_\sigma}{\sigma_K^f(\eta_\sigma)}=\frac{(\sigma\gamma/\gamma)}{\sigma_K^f(\sigma\gamma/\gamma)}\cdot\frac{\sigma(\alpha')}{\alpha'}=\frac{\sigma(\alpha)}\alpha,\]
	so $\chi \eta_\sigma\in S_\alpha$. This finishes.
	% \begin{itemize}
	% 	\item We check $x\cdot S_{\alpha'}\subseteq S_\alpha$. Well, if $\eta_\sigma\in S_{\alpha'}$, then
	% 	\[\frac{x\eta_\sigma}{\sigma_K^f(x\eta_\sigma)}=\frac x{\sigma_K^f(x)}\cdot\frac{\eta_\sigma}{\sigma_K^f(\eta_\sigma)}=\frac{(\sigma\gamma/\gamma)}{\sigma_K^f(\sigma\gamma/\gamma)}\cdot\frac{\sigma(\alpha')}{\alpha'}=\frac{\sigma(\alpha)}\alpha,\]
	% 	so $x\eta_\sigma\in S_\alpha$.
	% 	\item We check $S_\alpha\subseteq x\cdot S_{\alpha'}$. Again, if $\eta_\sigma\in S_\alpha$, then
	% 	\[\frac{x^{-1}\eta_\sigma}{\sigma_K^f(x^{-1}\eta_\sigma)}=\frac {x^{-1}}{\sigma_K^f(x^{-1})}\cdot\frac{\eta_\sigma}{\sigma_K^f(\eta_\sigma)}=\left(\frac{(\sigma\gamma/\gamma)}{\sigma_K^f(\sigma\gamma/\gamma)}\right)^{-1}\cdot\frac{\sigma(\alpha)}{\alpha}=\frac{\sigma(\alpha')}{\alpha'},\]
	% 	so $\eta_\sigma=x\cdot x^{-1}\eta_\sigma\in x\cdot S_{\alpha'}$.
	% \end{itemize}
	% The above two inclusions complete the proof.
\end{proof}
We now return to describing triples.
\begin{cor} \label{cor:fullclass}
	Fix everything as in the set-up, and fix $\alpha\in ML^\times$ such that $\op N_{ML/L}(\alpha)=p$. Then, for any triple $(\alpha_1',\alpha_2',\beta')$ corresponding to the fundamental class, there exist elements $\eta_{c(\sigma_K)}',\eta_{c(\sigma_x)}'\in ML^\times$ with
	\[\frac{\eta_{c(\sigma_K)}'}{\sigma_K^f\left(\eta_{c(\sigma_K)}'\right)}=\frac{\sigma_K(\alpha)}{\alpha}\qquad\text{and}\qquad\frac{\eta_{c(\sigma_x)}'}{\sigma_K^f\left(\eta_{c(\sigma_x)}'\right)}=\frac{\sigma_x(\alpha)}{\alpha}\]
	such that
	\[(\alpha_1',\alpha_2',\beta')=\left(\alpha\cdot\prod_{i=0}^{f-1}\sigma_K^i\left(\eta_{c(\sigma_K)}'\right),\quad\prod_{i=0}^{\varphi\left(p^\nu\right)-1}\sigma_x^i\left(\eta_{c(\sigma_x)}'\right),\quad\frac{\sigma_K\left(\eta_{c(\sigma_x)}'\right)}{\eta_{c(\sigma_x)}'}\cdot\frac{\eta_{c(\sigma_K)}'}{\sigma_x\left(\eta_{c(\sigma_K)}'\right)}\right).\]
	In other words, all triples corresponding to the fundamental class come from the recipe described in \autoref{cor:fundtriple}.
\end{cor}
\begin{proof}
	By \autoref{cor:fundtriple}, we can certainly find some elements $\eta_{c(\sigma_K)},\eta_{c(\sigma_x)}\in ML^\times$ such that
	\[\frac{\eta_{c(\sigma_K)}}{\sigma_K^f\left(\eta_{c(\sigma_K)}\right)}=\frac{\sigma_K(\alpha)}{\alpha}\qquad\text{and}\qquad\frac{\eta_{c(\sigma_x)}}{\sigma_K^f\left(\eta_{c(\sigma_x)}\right)}=\frac{\sigma_x(\alpha)}{\alpha},\]
	for which
	\[(\alpha_1,\alpha_2,\beta)\coloneqq\left(\alpha\cdot\prod_{i=0}^{f-1}\sigma_K^i\left(\eta_{c(\sigma_K)}\right),\quad\prod_{i=0}^{\varphi\left(p^\nu\right)-1}\sigma_x^i\left(\eta_{c(\sigma_x)}\right),\quad\frac{\sigma_K\left(\eta_{c(\sigma_x)}\right)}{\eta_{c(\sigma_x)}}\cdot\frac{\eta_{c(\sigma_K)}}{\sigma_x\left(\eta_{c(\sigma_K)}\right)}\right)\]
	corresponds to the fundamental class $u_{L/K}\in H^2(\op{Gal}(L/K),L^\times)$. In particular, $(\alpha_1,\alpha_2,\beta)$ and $(\alpha_1',\alpha_2',\beta')$ both correspond to the same cohomology class and hence in the same equivalence class of triples, so we know that there exist $m_1,m_2\in L^\times$ such that
	\[\alpha_1'=\alpha_1\cdot\op N_{L/L_1}(m_1),\quad\alpha_2'=\alpha_2\cdot\op N_{L/L_2}(m_2),\quad\beta'=\beta\cdot\frac{\sigma_K(m_2)}{m_2}\cdot\frac{m_1}{\sigma_x(m_1)}.\]
	As such, we set $\eta_{c(\sigma_K)}'\coloneqq \eta_{c(\sigma_K)}\cdot m_1$ and $\eta_{c(\sigma_x)}'\coloneqq \eta_{c(\sigma_x)}\cdot m_2$, and these can be checked to work. For example, $\eta_{c(\sigma_K)}'$ satisfies
	\[\frac{\eta_{c(\sigma_K)}'}{\sigma_K^f\left(\eta_{c(\sigma_K)}'\right)}=\frac{\sigma_K(\alpha)}{\alpha}\qquad\text{and}\qquad\frac{\eta_{c(\sigma_x)}'}{\sigma_K^f\left(\eta_{c(\sigma_x)}'\right)}=\frac{\sigma_x(\alpha)}{\alpha}\]
	by \autoref{lem:hilbert90}. The rest of the checks are similar.
\end{proof}
\begin{corollary} \label{cor:classofa1}
	Fix everything as in the set-up, and let $\pi_1\in L_1^\times$ be a uniformizer. If the triple $(\alpha_1,\alpha_2,\beta)$ is a triple corresponding to the fundamental class, then
	\[\alpha_1\equiv\pi_1\pmod{\op N_{L/L_1}(L^\times)}.\]
\end{corollary}
\begin{proof}[Proof by triples]
	Note that $L/L_1$ is an unramified extension, so all elements of absolute value $1$ are norms, so there is in fact a class of elements containing all uniformizers in $L_1^\times/\op N_{L/L_1}(L^\times)$. Further, because $\alpha_1$ is also only defined up to a multiple in $\op N_{L/L_1}(L^\times)$, to show that the classes in $L^\times/\op N_{L/L_1}(L^\times)$ coincide, it thus suffices to exhibit a single triple $(\alpha_1,\alpha_2,\beta)$ such that $\alpha_1\in L_1^\times$ is a uniformizer.

	This is a matter of force. To begin, we can use \autoref{cor:fundtriple} to find some $\alpha$ with $\op N_{ML/L}(\alpha)=p$ and $\eta_{c(\sigma_K)},\eta_{c(\sigma_x)}\in ML^\times$ giving the triple $(\alpha_1,\alpha_2,\beta)$ as described. The idea is to force $\eta_{c(\sigma_K)}$ to have valuation zero.
	
	Let $v_{ML}$ be the fixed valuation of $ML$ extending the standard valuation $v_{\QQ_p}$ on $\QQ_p$, and let $v_{L}$ be its restriction to $L$. Because $ML/L$ is an unramified, the image of $v_{ML}$ and $v_L$ in $\QQ$ is the same. In particular, we can find some $m_1\in L_1^\times$ such that
	\[v_{ML}\left(\eta_{c(\sigma_K)}\right)=v_L(m_1).\]
	Thus, we replace $\eta_{c(\sigma_K)}$ with $\eta_{c(\sigma_K)}/m_1$, and we still satisfy the conditions of \autoref{cor:fundtriple} by \autoref{lem:hilbert90} while getting $v_{ML}\left(\eta_{c(\sigma_K)}\right)=0$. Now, the corresponding $\alpha_1$ looks like
	\[\alpha_1=\alpha\cdot\prod_{i=0}^{f-1}\sigma_K^i\left(\eta_{c(\sigma_K)}\right).\]
	In particular, defining $v_{L_1}\coloneqq v_L|_{L_1}$, it follows
	\[v_{L_1}(\alpha_1)=v_{ML}(\alpha_1)=v_{ML}(\alpha),\]
	However, $\op N_{ML/L}(\alpha)=p$ by construction, so we see that
	\[[ML:L]v_{ML}(\alpha)=v_{ML}(p)=v_{\QQ_p}(p)=1.\]
	Explicitly, we see that
	\[[ML:L]=[\QQ(\zeta_{N'}):\QQ(\zeta_m)]=\frac{[\QQ(\zeta_{N'}):\QQ_p]}{[\QQ_p(\zeta_m):\QQ_p]}=\frac nf=\varphi\left(p^\nu\right).\]
	However, $L_1/K$ has ramification degree $\varphi\left(p^\nu\right)$ (from the maximal totally ramified subextension $\QQ_p(\zeta_{p^\nu})$), so its uniformizers are the elements of valuation $1/\varphi\left(p^\nu\right)$. Thus, we have computed that $\alpha_1$ has the correct valuation and hence is a uniformizer.
\end{proof}
\begin{proof}[Proof by the Artin map]
	We take a moment to say that there is an alternate derivation of \autoref{cor:classofa1} using the Artin map: one can show that, if $u\in Z^2(L/K)$ is a representative of the fundamental class of an abelian extension $L/K$, then
	\begin{align*}
		\op{Gal}(L/K) &\to K^\times/\op N(L^\times) \\
		\sigma &\mapsto \prod_{g\in\op{Gal}(L/K)}u(g,\sigma)
	\end{align*}
	is the inverse Artin map. In particular, from our explicit formula for $\alpha_1$, we see
	\[\alpha_1=\prod_{g\in\op{Gal}(L/L_1)}u(g,\overline\sigma_K)=\theta_{L/L_1}^{-1}(\overline\sigma_K).\]
	However, $\overline\sigma_K$ is the Frobenius automorphism of $L/L_1$ because the extension $L_1/K$ is totally ramified, implying that the residue field of $L_1$ is the same as $K=\QQ_p$. Thus, $\theta_{L/L_1}^{-1}(\overline\sigma_K)$ is the class containing the uniformizers of $L_1^\times$.
\end{proof}
We close with a sanity check.
\begin{cor}
	Fix everything as in the set-up, and let $T_\alpha$ denote the set of triples $(\alpha_1,\alpha_2,\beta)$ generated by some element $\alpha\in ML^\times$ with $\op N_{ML/L}(\alpha)=p$ via \autoref{cor:fundtriple}. Then $T_\alpha$ is independent of $\alpha$.
\end{cor}
\begin{proof}
	The main idea is to use (the proof of) \autoref{cor:updatealpha}. Fix $\alpha,\alpha'\in ML^\times$ with $\op N_{ML/L}(\alpha)=\op N_{ML/L}(\alpha')=p$, and we need to show that $T_\alpha=T_{\alpha'}$. By symmetry, it will be enough for $T_\alpha\subseteq T_{\alpha'}$.

	Following the proof of \autoref{cor:updatealpha}, note that $\op N_{ML/L}(\alpha/\alpha')=1$, so we are promised $\gamma\in ML^\times$ such that $\alpha/\alpha'=\gamma/\sigma_K^f(\gamma)$. Then we showed that any $\sigma\in\{\sigma_K,\sigma_x\}$ can set
	\[\chi_\sigma\coloneqq\frac{\sigma(\gamma)}\gamma\]
	to give $S_{\alpha,\sigma} x=\cdot S_{\alpha',\sigma}$, where $S_{\alpha,\sigma}$ is the set of possible $\eta_\sigma$ defined in \autoref{cor:updatealpha}.

	We now proceed directly with the proof. Suppose that we have some triple $(\alpha_1,\alpha_2,\beta)\in T_\alpha$, which we know that we can write down as
	\[(\alpha_1,\alpha_2,\beta)=\left(\alpha\cdot\prod_{i=0}^{f-1}\sigma_K^i\left(\eta_{\sigma_K}\right),\quad\prod_{i=0}^{\varphi\left(p^\nu\right)-1}\sigma_x^i\left(\eta_{\sigma_x}\right),\quad\frac{\sigma_K\left(\eta_{\sigma_x}\right)}{\eta_{\sigma_x}}\cdot\frac{\eta_{\sigma_K}}{\sigma_x\left(\eta_{\sigma_K}\right)}\right)\]
	for some $\eta_{\sigma_K}\in S_{\alpha,\sigma_K}$ and $\eta_{\sigma_x}\in S_{\alpha,\sigma_x}$. We need to show that $(\alpha_1,\alpha_2,\beta)\in T_{\alpha'}$. Well, by \autoref{cor:updatealpha}, we can set
	\[I'_{\sigma}\coloneqq \eta_{\sigma}/\chi_\sigma\in S_{\alpha',\sigma}\]
	for $\sigma\in\{\sigma_K,\sigma_x\}$. We now compute
	\begin{align*}
		\alpha_1 ={}& \alpha\cdot\prod_{i=0}^{f-1}\sigma_K^i(\eta_{\sigma_K}) \\
		={}& \alpha\cdot\prod_{i=0}^{f-1}\sigma_K^i(\chi_\sigma)\cdot\prod_{i=0}^{f-1}\sigma_K^i(\eta_{\sigma_K'}) \\
		={}& \alpha\cdot\prod_{i=0}^{f-1}\sigma_K^i\left(\frac{\sigma_K\gamma}{\gamma}\right)\cdot\prod_{i=0}^{f-1}\sigma_K^i(\eta_{\sigma_K'}) \\
		={}& \alpha\cdot\frac{\sigma_K^f(\gamma)}{\gamma}\cdot\prod_{i=0}^{f-1}\sigma_K^i(\eta_{\sigma_K'}) \\
		={}& \alpha'\cdot\prod_{i=0}^{f-1}\sigma_K^i(\eta_{\sigma_K'}),
	\end{align*}
	where the last equality holds by definition of $\gamma$. Similarly, we see
	\begin{align*}
		\alpha_2 ={}& \prod_{i=0}^{\varphi\left(p^\nu\right)-1}\sigma_x^i(\eta_{\sigma_x}) \\
		={}& \prod_{i=0}^{\varphi\left(p^\nu\right)-1}\sigma_x^i(\chi_{\sigma_x})\cdot\prod_{i=0}^{\varphi\left(p^\nu\right)-1}\sigma_x^i(\eta_{\sigma_x}') \\
		={}& \prod_{i=0}^{\varphi\left(p^\nu\right)-1}\sigma_x^i\left(\frac{\sigma_x(\gamma)}{\gamma}\right)\cdot\prod_{i=0}^{\varphi\left(p^\nu\right)-1}\sigma_x^i(\eta_{\sigma_x}') \\
		={}& \prod_{i=0}^{\varphi\left(p^\nu\right)-1}\sigma_x^i(\eta_{\sigma_x}'),
	\end{align*}
	where the product telescopes in the last equality because $\sigma_x$ has order $\varphi\left(p^\nu\right)$. Lastly, we set
	\begin{align*}
		\beta &= \frac{\sigma_K\left(\eta_{\sigma_x}\right)}{\eta_{\sigma_x}}\cdot\frac{\eta_{\sigma_K}}{\sigma_x\left(\eta_{\sigma_K}\right)} \\
		&= \frac{\sigma_K\left(\chi_{\sigma_x}\right)}{\chi_{\sigma_x}}\cdot\frac{\chi_{\sigma_K}}{\sigma_x\left(\chi_{\sigma_K}\right)}\cdot\frac{\sigma_K\left(\eta_{\sigma_x}'\right)}{\eta_{\sigma_x}'}\cdot\frac{\eta_{\sigma_K}'}{\sigma_x\left(\eta_{\sigma_K}'\right)} \\
		&= \frac{\sigma_K\sigma_x\gamma/\sigma_K\gamma}{\sigma_x\gamma/\gamma}\cdot\frac{\sigma_K\gamma/\gamma}{\sigma_x\sigma_K\gamma/\sigma_x\gamma}\cdot\frac{\sigma_K\left(\eta_{\sigma_x}'\right)}{\eta_{\sigma_x}'}\cdot\frac{\eta_{\sigma_K}'}{\sigma_x\left(\eta_{\sigma_K}'\right)} \\
		&= \frac{\sigma_K\left(\eta_{\sigma_x}'\right)}{\eta_{\sigma_x}'}\cdot\frac{\eta_{\sigma_K}'}{\sigma_x\left(\eta_{\sigma_K}'\right)}.
	\end{align*}
	Thus,
	\[(\alpha_1,\alpha_2,\beta)=\left(\alpha'\cdot\prod_{i=0}^{f-1}\sigma_K^i\left(\eta_{\sigma_K}'\right),\quad\prod_{i=0}^{\varphi\left(p^\nu\right)-1}\sigma_x^i\left(\eta_{\sigma_x}'\right),\quad\frac{\sigma_K\left(\eta_{\sigma_x}'\right)}{\eta_{\sigma_x}'}\cdot\frac{\eta_{\sigma_K}'}{\sigma_x\left(\eta_{\sigma_K}'\right)}\right)\in T_{\alpha'},\]
	which finishes.
\end{proof}

\subsection{Tame Ramification}
In this section, we work through \autoref{cor:fundtriple} very explicitly in a basic case. Let $p$ be an odd prime because the following discussion has no content in the case of $p=2$. Set $K\coloneqq\QQ_p$ and $K_m\coloneqq\QQ_p(\zeta_m)$ with $f\coloneqq[\QQ_p(\zeta_m):\QQ_p]$.

The main simplification we will make which allows explicit computation is that we will set $K_{\pi,\nu}\coloneqq\QQ_p(\zeta_p)$. Continuing with the set-up, we see $L=\QQ_p(\zeta_p,\zeta_m)$ with $n\coloneqq (p-1)\cdot f$; as such, set $N'\coloneqq p^n-1$ so that $M=\QQ_p(\zeta_{N'})$. Here is the diagram of our fields.
% https://q.uiver.app/?q=WzAsNixbMSwzLCJcXFFRX3AiXSxbMCwyLCJcXFFRX3AoXFx6ZXRhX3ApIl0sWzIsMiwiXFxRUV9wKFxcemV0YV9tKSJdLFsxLDEsIlxcUVFfcChcXHpldGFfcCxcXHpldGFfbSkiXSxbMywxLCJcXFFRX3AoXFx6ZXRhX3tOJ30pIl0sWzIsMCwiXFxRUV9wKFxcemV0YV9wLFxcemV0YV97Tid9KSJdLFswLDEsIiIsMCx7InN0eWxlIjp7ImhlYWQiOnsibmFtZSI6Im5vbmUifX19XSxbMCwyLCIiLDIseyJzdHlsZSI6eyJoZWFkIjp7Im5hbWUiOiJub25lIn19fV0sWzEsMywiIiwwLHsic3R5bGUiOnsiaGVhZCI6eyJuYW1lIjoibm9uZSJ9fX1dLFsyLDMsIiIsMix7InN0eWxlIjp7ImhlYWQiOnsibmFtZSI6Im5vbmUifX19XSxbMiw0LCIiLDIseyJzdHlsZSI6eyJoZWFkIjp7Im5hbWUiOiJub25lIn19fV0sWzQsNSwiIiwyLHsic3R5bGUiOnsiaGVhZCI6eyJuYW1lIjoibm9uZSJ9fX1dLFszLDUsIiIsMix7InN0eWxlIjp7ImhlYWQiOnsibmFtZSI6Im5vbmUifX19XV0=&macro_url=https%3A%2F%2Fraw.githubusercontent.com%2FdFoiler%2Fnotes%2Fmaster%2Fnir.tex
\[\begin{tikzcd}
	&& {\QQ_p(\zeta_p,\zeta_{N'})} \\
	& {\QQ_p(\zeta_p,\zeta_m)} && {\QQ_p(\zeta_{N'})} \\
	{\QQ_p(\zeta_p)} && {\QQ_p(\zeta_m)} \\
	& {\QQ_p}
	\arrow[no head, from=4-2, to=3-1]
	\arrow[no head, from=4-2, to=3-3]
	\arrow[no head, from=3-1, to=2-2]
	\arrow[no head, from=3-3, to=2-2]
	\arrow[no head, from=3-3, to=2-4]
	\arrow[no head, from=2-4, to=1-3]
	\arrow[no head, from=2-2, to=1-3]
\end{tikzcd}\]
So that we are able to isolate our set-up, we note that
\[\op{Gal}(\QQ(\zeta_p)/\QQ)\simeq(\ZZ/p\ZZ)^\times\]
is cyclic, so we choose some $x\in(\ZZ/p\ZZ)^\times$ to generate, which corresponds to the automorphism $\sigma_x\colon\zeta_p\mapsto\zeta_p^x$. Namely, we may set $\tau_1\coloneqq\sigma_x$.

Now, the reason we set $K_{\pi,\nu}=\QQ_p(\zeta_p)$ is that we will be able to set
\[\gamma\coloneqq(-p)^{1/(p-1)}\in\QQ_p(\zeta_p).\]
Indeed, we sneakily set $\pi=-p$ to be our uniformizer of $\QQ_p$ so that $\op N_{ML/L}(\gamma)=\gamma^{p-1}=-p$. Because it will be helpful for us shortly, we will actually give a construction of $(-p)^{1/(p-1)}$, for completeness.
\begin{lemma} \label{lem:findrootofp}
	Let $p$ be a prime. Then we can find some $\gamma\coloneqq(-p)^{1/(p-1)}$ in $\QQ_p(\zeta_p)$. In fact, we can take $\gamma\equiv c\varpi\pmod{\varpi^2}$ for any $c\in\FF_p^\times$, where $\varpi\coloneqq\zeta_p-1$ is a uniformizer.
\end{lemma}
\begin{proof}
	That a root $\gamma$ exists is well-known. The factorization
	\[x^{p-1}-1\equiv\prod_{c\in\FF_p^\times}(x-c)\pmod p\]
	lifts to a factorization in $\ZZ_p$ by \cite[Lemma~II.4.6]{neukirch-alg-nt}. As such, as soon as we have one root $\gamma$ of $x^{p-1}+p$, observe that $|\gamma|=p^{1/(p-1)}=|\varpi|$, so $\gamma$ is a uniformizer as well, meaning that the $c$ in
	\[\zeta_{p-1}\gamma\equiv c\varpi\pmod{\varpi^2}\]
	is nonzero and will vary across all representatives in $\FF_p^\times$ as we exchange the root $\gamma$ with $\zeta_{p-1}\gamma$ for various $\zeta_{p-1}$.
	% We follow Professor Andrew Sutherland's \href{https://math.mit.edu/classes/18.785/2021fa/LectureNotes20.pdf\#theorem.2.5}{Lemma~20.5}. Set $\pi\coloneqq\zeta_p-1$ to be a uniformizer of $\QQ_p(\zeta_p)$. Now, the minimal polynomial of $\zeta_p$ is
	% \[f(T)\coloneqq\frac{(T+1)^p-1}T,\]
	% which is $p$-Eisenstein. To properly apply Hensel's lemma to solve $T^{p-1}+p$, we see that any solution should be divisible by $\pi$, so we divide out by this first. Note $v(\pi)=1/(p-1)$, so $u\coloneqq-\pi^{p-1}/p\in\mathcal O_{\QQ_p(\zeta_p)}^\times$. In fact, we can see from the polynomial $f$ that
	% \[\pi^{p-1}+p\equiv0\pmod{p\pi},\]
	% so $u\equiv-1\pmod\pi$. As such, we now note that $g(T)\coloneqq T^{p-1}-u$ has
	% \[g(c)\equiv0\pmod\pi\qquad\text{and}\qquad g'(c)=(p-1)c\not\equiv0\pmod\pi,\]
	% for any $c\in\FF_p^\times$, so we can lift $c$ to a root $\beta_c\in\mathcal O_{\QQ_p(\zeta_p)}$. From here, we see $(\pi/\beta_c)^{p-1}=\pi^{p-1}/u=-p$, so $\pi/\beta_c$ is our desired root. For the last statement, we see
	% \[\pi/\beta_c\equiv c^{-1}\pi\pmod{\pi^2},\]
	% so as $c\in\FF_p^\times$ varies, we do indeed get all equivalence classes.
\end{proof}
In light of \autoref{lem:findrootofp}, we will just take $\gamma$ to have $\gamma^{p-1}=-p$ with $\gamma\equiv c\pi\pmod{\pi^2}$ for any particular $c\in\FF_p^\times$. This satisfies $\op N_{ML/L}(\gamma)=-p$ as discussed above.

We will now compute the tuple. We start with the unramified side because it is easier. Namely, $\gamma\in\QQ_p(\zeta_p)$ is fixed by the Frobenius automorphism $\sigma_K$, so we may set $\eta_K\coloneqq1$ to have
\[\frac{\eta_K}{\sigma_K^f(\eta_K)}=1=\frac{\sigma_K(\gamma)}{\gamma}.\]
The corresponding $\alpha_0$ is thus
\[\boxed{\alpha_0=\gamma}.\]
We now deal with ramification.
% Observe $\op{Gal}(\QQ_p(\zeta_p)/\QQ_p)\simeq(\ZZ/p\ZZ)^\times$ is cyclic, but we must choose a generator nonetheless. Let $x\in(\ZZ/p\ZZ)^\times$ be a generator, and let $\sigma_x\colon\zeta_p\mapsto\zeta_p^x$ be the corresponding automorphism; namely, $\tau_1\coloneqq\sigma_x$. (Notably, this is not the automorphism generated by the Artin map; we will return to this point later.)
We begin with a computational lemma, tying in what we have with Teichm\"uller lifts.
\begin{lemma}
	Fix everything as above. Then $\zeta_{p-1}\coloneqq\sigma_x(\gamma)/\gamma$ is a primitive $(p-1)$st root of unity and in particular lies in $\QQ_p$. In fact, $\zeta_{p-1}\equiv x\pmod p$.
\end{lemma}
Note that we are defining $\zeta_{p-1}$ above, which is okay: in the worst case, we might have to adjust the definitions of $\zeta_{N'}$ and $\zeta_m$ to correspond with this particular $\zeta_{p-1}$, but otherwise $\zeta_{p-1}$ may be any fixed primitive $(p-1)$st root of unity.
\begin{proof}
	To see that $\zeta_{p-1}$ is a $(p-1)$st root of unity, we note that $\sigma_x(\gamma)=\zeta_{p-1}\cdot\gamma$, so an induction shows that
	\[\sigma_x^k(\gamma)=\zeta_{p-1}^k\cdot\gamma.\]
	Setting $k=p-1$ shows that $\zeta_{p-1}^{p-1}=1$, so $\zeta_{p-1}$ is a $(p-1)$st root of unity.
	% To show that $\zeta_{p-1}$ is primitive, we know that $\zeta_{p-1}^k=1$ above would imply that $\sigma_x^k(\gamma)=\gamma$, but $\QQ_p(\gamma)=\QQ_p(\zeta_p)$ (we already know $\QQ_p(\gamma)\subseteq\QQ_p(\zeta_p)$, but both of these extensions have degree $p-1$), so in fact $\sigma_x^k=\id$. So $x\in(\ZZ/p\ZZ)^\times$ being a generator requires $p-1\mid k$. So indeed, the least positive integer $k$ with $\zeta_p^k=1$ is $k=p-1$.

	% We now quickly note that $\QQ_p$ contains all $(p-1)$st roots of unity by Hensel's lemma because the polynomial $T^{p-1}-1\in\FF_p[T]$ fully splits into $p-1$ distinct factors; in particular, $\zeta_{p-1}\in\QQ_p$. In fact, Hensel's lemma tells us that the $p-1$st roots of unity of $\QQ_p$ fully represent $(\ZZ/p\ZZ)^\times$, so there is a chance for $\zeta_{p-1}\equiv x\pmod p$.

	We next show $\zeta_{p-1}\equiv x\pmod p$; this will automatically imply that $\zeta_{p-1}$ is primitive because it will force $\zeta_{p-1}$ to have at least the order of $x\pmod p$, which is $p-1$. Let $\varpi\coloneqq\zeta_p-1$ be a uniformizer of $\QQ_p(\zeta_p)$. Because $\zeta_{p-1},x\in\QQ_p$, it is enough for $v_{\QQ_p}(\zeta_{p-1}-x)>0$; as such, we will show that
	\[\zeta_{p-1}\stackrel?\equiv x\pmod\varpi.\]
	To see this, recall $\gamma\equiv c\varpi\pmod{\varpi^2}$, so
	\[\zeta_{p-1}=\frac{\sigma_x(\gamma)}{\gamma}\equiv\frac{c\cdot\sigma_x(\varpi)}{c\cdot\varpi}\equiv\frac{\sigma_x(\varpi)}{\varpi}\pmod\varpi.\]
	However, $\sigma_x(\varpi)=\zeta_p^x-1$, so
	\[\frac{\sigma_x(\varpi)}\varpi=\frac{\zeta_p^x-1}{\zeta_p-1}\equiv1+\zeta_p+\cdots+\zeta_p^{x-1}\equiv\underbrace{1+\cdots+1}_x\equiv x\pmod\varpi,\]
	finishing.
\end{proof}
We are almost able to compute $\eta_x\coloneqq\eta_1$. To do this, we pick up a quick lemma.
\begin{lemma}
	Let $p$ and $f$ be integers. Then
	\[\frac{p^{f(p-1)}-1}{(p-1)\left(p^f-1\right)}\in\ZZ.\]
\end{lemma}
\begin{proof}
	Observe
	\[\frac{p^{f(p-1)}-1}{p^f-1} = \sum_{k=0}^{p-1}p^{fk} \equiv \sum_{k=0}^{p-1}1=p-1\equiv0\pmod{p-1}.\]
	This finishes.
\end{proof}
In light of the above lemma, we define
\[z\coloneqq-\frac{p^{f(p-1)}-1}{(p-1)\left(p^f-1\right)}.\]
Note the sign here: it is very important! It follows that $\eta_x\coloneqq\zeta_{N'}^z$ will have
\begin{align*}
	\frac{\eta_x}{\sigma_K^f(\eta_x)} &= \frac{\zeta_{N'}^z}{\zeta_{N'}^{zp^f}} = \zeta_{N'}^{-z\left(p^f-1\right)} = \zeta_{N'}^{N'/(p-1)} = \zeta_{p-1},
\end{align*}
which is indeed $\sigma_x(\gamma)/\gamma$. Thus, the corresponding $\alpha_1$ is
\begin{align*}
	\alpha_1 &= \prod_{i=0}^{p-1}\sigma_x^i(\eta_i) \\
	&= \eta_i^{p-1} \\
	&= \zeta^{z(p-1)}_{N'} \\
	&= \zeta_{N'}^{-N'/\left(p^f-1\right)} \\
	\Aboxed{\alpha_1 &= \zeta_{p^f-1}^{-1}}.
\end{align*}
Lastly, we compute our $\beta_{10}$ as
\begin{align*}
	\beta_{10} &= \frac{\eta_K}{\sigma_x\eta_K}\cdot\frac{\sigma_K\eta_x}{\eta_x} \\
	&= \zeta_{N'}^{z(p-1)} \\
	\Aboxed{\beta_{10} &= \zeta_{p^f-1}^{-1}}.
\end{align*}
In total, we get the following nice result.
\begin{theorem}
	Let $p$ be an odd prime, and fix $K\coloneqq\QQ_p$ and $L\coloneqq\QQ_p(\zeta_p,\zeta_m)$, where $p\nmid m$. Further, set $L_0\coloneqq\QQ_p(\zeta_p)$ and $L_1\coloneqq\QQ_p(\zeta_m)$ so that $L=L_0L_1$ and $L_0\cap L_1=K$. Now, pick up the following data.
	\begin{itemize}
		\item Suppose the order of $p$ modulo $m$ is $f$.
		\item Let $\sigma_x\colon\zeta_p\mapsto\zeta_p^x$ be a generator of $\op{Gal}(\QQ_p(\zeta_p)/\QQ_p)$.
		\item Find $\gamma\in\QQ_p(\zeta_p)$ such that $\gamma^{p-1}+p=0$ and $\sigma_x(\gamma)/\gamma=\zeta_{p-1}$. (Equivalently, set $\zeta_{p-1}\coloneqq\sigma_x(\gamma)/\gamma$.)
	\end{itemize}
	Then the fundamental class $u_{L/K}\in H^2(\op{Gal}(L/K),L^\times)$ is represented by the triple
	\[(\alpha_0,\alpha_1,\beta_{10})=\left(\gamma,\zeta_{p^f-1}^{-1},\zeta_{p^f-1}^{-1}\right).\]
\end{theorem}
\begin{remark}
	We verify Artin reciprocity for $\QQ_p(\zeta_p)/\QQ_p$. Let $c\in Z^2(\op{Gal}(L/K),L^\times)$ represent the fundamental class. The explicit formula for $\alpha_1$ tells us that
	\[\alpha_1=\prod_{i=0}^{p-1}c\left(\sigma_x^i,\sigma_x\right)=[\sigma_x]\cup\op{Res}u_{L/\QQ_p}=[\sigma_x]\cup u_{L/\QQ_p(\zeta_m)}=\theta^{-1}_{L/\QQ_p(\zeta_m)}(\sigma_x).\]
	Taking norms down to $K^\times$, we see on one hand that
	\[\op N_{\QQ_p(\zeta_m)/\QQ_p}(\alpha_1)=\prod_{i=0}^{f-1}\zeta_{p^f-1}^{-p^i}=\zeta_{p^f-1}^{-\left(1+p+\cdots+p^{f-1}\right)}=\zeta_{p^f-1}^{-\left(p^f-1\right)/(p-1)}=\zeta_{p-1}^{-1}\equiv x^{-1}\pmod p.\]
	On the other hand,
	\[\op N_{\QQ_p(\zeta_m)/\QQ_p}\theta^{-1}_{L/\QQ_p(\zeta_m)}(\sigma_x)=\theta^{-1}_{L/\QQ_p}(\sigma_x)=\theta^{-1}_{\QQ_p(\zeta_p)/\QQ_p}(\sigma_x).\]
	So $\theta^{-1}_{\QQ_p(\zeta_p)/\QQ_p}$ sends $\sigma_x\colon\zeta_p\mapsto\zeta_p^x$ to $x^{-1}\pmod p$, as predicted by Lubin--Tate theory.
\end{remark}

\subsection{Towers}
In this section, we will use the notions but not the exact notation as in the set-up. Instead, we will build a ``tower set-up'' below. Our goal is to be able to force some compatibility among the data in the tuples of \autoref{cor:fundtriple} in towers. This is particularly simple in the case where we fix some unramified extension and allow our ramification to ascend in a tower.

As such, fix a base field $K$ and unramified extension $K_m$, and we also fix a tower of totally ramified extensions
\[K\coloneqq K_{\pi,0}\subseteq K_{\pi,1}\subseteq K_{\pi,2}\subseteq\cdots.\]
For example, we might choose Lubin--Tate extensions for this purpose. For brevity, we set
\[K_\pi\coloneqq\bigcup_{i\ge0}K_{\pi,i}\]
to be the (very large) composite totally ramified extension. Now, for each $i\ge0$, we define $L_i\coloneqq K_mK_{\pi,i}$ for each and $M_i$ to be the unramified extension of degree $[L_i:K]$ over $K$; notably, $[K_m:K]\mid[L_i:K]$, so $K_m\subseteq M_i$ for each $i\ge0$. Here is our diagram.
% https://q.uiver.app/?q=WzAsMTUsWzMsNSwiSyJdLFs0LDQsIktfbSJdLFs1LDMsIk1fMSJdLFsyLDQsIktfe1xccGksMX0iXSxbMSwzLCJLX3tcXHBpLDJ9Il0sWzAsMiwiXFxkZG90cyJdLFszLDMsIkxfMSJdLFsyLDIsIkxfMiJdLFs2LDIsIlxcaWRkb3RzIl0sWzQsMiwiTV8xTF8xIl0sWzMsMSwiTV8xTF8yIl0sWzUsMSwiXFxpZGRvdHMiXSxbMSwxLCJcXGRkb3RzIl0sWzIsMCwiXFxkZG90cyJdLFs0LDAsIlxcaWRkb3RzIl0sWzAsMywiIiwwLHsic3R5bGUiOnsiaGVhZCI6eyJuYW1lIjoibm9uZSJ9fX1dLFszLDQsIiIsMCx7InN0eWxlIjp7ImhlYWQiOnsibmFtZSI6Im5vbmUifX19XSxbNCw1LCIiLDAseyJzdHlsZSI6eyJoZWFkIjp7Im5hbWUiOiJub25lIn19fV0sWzAsMSwiIiwyLHsic3R5bGUiOnsiaGVhZCI6eyJuYW1lIjoibm9uZSJ9fX1dLFsxLDIsIiIsMix7InN0eWxlIjp7ImhlYWQiOnsibmFtZSI6Im5vbmUifX19XSxbMiw4LCIiLDIseyJzdHlsZSI6eyJoZWFkIjp7Im5hbWUiOiJub25lIn19fV0sWzEsNiwiIiwyLHsic3R5bGUiOnsiaGVhZCI6eyJuYW1lIjoibm9uZSJ9fX1dLFsyLDksIiIsMix7InN0eWxlIjp7ImhlYWQiOnsibmFtZSI6Im5vbmUifX19XSxbNiw5LCIiLDEseyJzdHlsZSI6eyJoZWFkIjp7Im5hbWUiOiJub25lIn19fV0sWzksMTEsIiIsMSx7InN0eWxlIjp7ImhlYWQiOnsibmFtZSI6Im5vbmUifX19XSxbNiw3LCIiLDEseyJzdHlsZSI6eyJoZWFkIjp7Im5hbWUiOiJub25lIn19fV0sWzksMTAsIiIsMSx7InN0eWxlIjp7ImhlYWQiOnsibmFtZSI6Im5vbmUifX19XSxbMTAsMTQsIiIsMSx7InN0eWxlIjp7ImhlYWQiOnsibmFtZSI6Im5vbmUifX19XSxbMTAsMTMsIiIsMSx7InN0eWxlIjp7ImhlYWQiOnsibmFtZSI6Im5vbmUifX19XSxbNywxMiwiIiwxLHsic3R5bGUiOnsiaGVhZCI6eyJuYW1lIjoibm9uZSJ9fX1dLFs0LDcsIiIsMSx7InN0eWxlIjp7ImhlYWQiOnsibmFtZSI6Im5vbmUifX19XSxbNywxMCwiIiwxLHsic3R5bGUiOnsiaGVhZCI6eyJuYW1lIjoibm9uZSJ9fX1dLFszLDYsIiIsMCx7InN0eWxlIjp7ImhlYWQiOnsibmFtZSI6Im5vbmUifX19XV0=&macro_url=https%3A%2F%2Fraw.githubusercontent.com%2FdFoiler%2Fnotes%2Fmaster%2Fnir.tex
\[\begin{tikzcd}[column sep={1cm,between origins}, row sep={1cm,between origins}]
	&& \ddots && \iddots \\
	& \ddots && {M_1L_2} && \iddots \\
	\ddots && {L_2} && {M_1L_1} && \iddots \\
	& {K_{\pi,2}} && {L_1} && {M_1} \\
	&& {K_{\pi,1}} && {K_m} \\
	&&& K
	\arrow[no head, from=6-4, to=5-3]
	\arrow[no head, from=5-3, to=4-2]
	\arrow[no head, from=4-2, to=3-1]
	\arrow[no head, from=6-4, to=5-5]
	\arrow[no head, from=5-5, to=4-6]
	\arrow[no head, from=4-6, to=3-7]
	\arrow[no head, from=5-5, to=4-4]
	\arrow[no head, from=4-6, to=3-5]
	\arrow[no head, from=4-4, to=3-5]
	\arrow[no head, from=3-5, to=2-6]
	\arrow[no head, from=4-4, to=3-3]
	\arrow[no head, from=3-5, to=2-4]
	\arrow[no head, from=2-4, to=1-5]
	\arrow[no head, from=2-4, to=1-3]
	\arrow[no head, from=3-3, to=2-2]
	\arrow[no head, from=4-2, to=3-3]
	\arrow[no head, from=3-3, to=2-4]
	\arrow[no head, from=5-3, to=4-4]
\end{tikzcd}\]
Arrows going up and to the left are unramified; arrows going up and to the right are (totally) ramified. Now, we are interested in constructing a ``compatible'' system of tuples representing fundamental classes for the ascending chain of extensions $L_1/K$, $L_2/K$, $L_3/K$, etc.

For coherence reasons, we will also place a few assumptions on our Galois groups. Namely, we will assume that
\[\op{Gal}(K_\pi/K)=\bigoplus_{i=1}^m\overline{\langle\tau_i\rangle}\]
is a direct sum of finitely many procyclic groups. For example, if we are using Lubin--Tate extensions, and we are in characteristic $0$, then this is automatic. Additionally, we will assume that our quotients are
\[\op{Gal}(K_{\pi,i}/K)=\bigoplus_{i=1}^m\langle\tau_i|_{K_i}\rangle\]
for each $i\ge0$. This requirement, though strong, is essentially the only way we could hope for compatibility among our tuples---namely, it tells us that each $L_i/K$ has Galois group generated by the same elements (up to restriction) and hence have more or less the same requirements to yield a fundamental tuple. As an example, this requirement is satisfied when $K=\QQ_p$ and $K_i=\QQ_p(\zeta_{p^i})$; in fact, $m\in\{1,2\}$ in this case.

The main focus of the construction is to construct compatible $\gamma$ elements, but the notion of compatibility will in fact extend. As such, we will codify this into the following definition.
\begin{definition}
	Fix everything as above. Then a sequence $\{x_i\}_{i=0}^\infty$ of elements $x_i\in M_iL_i$ is \textit{compatible in towers} if and only if
	\[\op N_{M_{i+1}L_{i+1}/M_iL_{i+1}}(x_{i+1})=x_i.\]
\end{definition}
This definition is written down sequentially so that verifying its existence is easy.
\begin{lemma}
	Fix a uniformizer $\pi_K\in K$. There is a sequence $\{\gamma_i\}_{i=0}^\infty$ of elements compatible in towers such that $\gamma_0\in M_0L_0=K_m$ is $\gamma_0=\pi_K$.
\end{lemma}
\begin{proof}
	This comes down to a norm argument and an induction. Extend a valuation $v_K\colon K\to\ZZ$ to all fields above. Suppose we have constructed $\gamma_i$ such that $\gamma_i$ is a uniformizer of $M_iL_i$. We claim that we can construct $\gamma_{i+1}$ to also be a uniformizer of $M_{i+1}L_{i+1}$ and with
	\[\op N_{M_{i+1}L_{i+1}/M_iL_{i+1}}(\gamma_{i+1})=\gamma_i.\]
	This claim will finish the proof inductively.
	
	Now, observe that the extension $M_iL_{i+1}/M_iL_i$ is a totally ramified extension, so if we let $\varpi$ denote a uniformizer of $M_iL_{i+1}$, we have
	\begin{equation}
		v\left(\varpi^{[M_iL_{i+1}:M_iL_i]}\right)=v(\gamma_i). \label{eq:totramtowervaluation}
	\end{equation}
	Continuing, $M_{i+1}L_{i+1}/M_iL_{i+1}$ is an unramified extension, so in fact $\varpi$ continues to be a uniformizer up in $M_{i+1}L_{i+1}$. As such, we see that it suffices to construct $u\in M_{i+1}L_{i+1}$ such that
	\[\op N_{M_{i+1}L_{i+1}/M_iL_{i+1}}(u)=\frac{\gamma_i}{\op N_{M_{i+1}L_{i+1}/M_iL_{i+1}}(\varpi)}.\]
	But the right-hand side is a unit because it has valuation $0$ from \autoref{eq:totramtowervaluation}, so we can construct a unit $u$ for the left-hand side as well because the norm map surjects from units to units in unramified extensions. In total, $\gamma_{i+1}\coloneqq u\varpi$ is the element we are looking for.
\end{proof}
However, the definition of compatibility does not actually tell us that each of these $\gamma_i$ will behave the way that we need them to as required by \autoref{cor:fundtriple}. The compatibility is also a little unnatural because it only moves one step at a time. To fix both of these issues, we have the following.
\begin{lemma}
	Suppose that the sequence $\{x_i\}_{i=0}^\infty$ is compatible in towers. Then for any nonnegative integers $p\ge q$, we have
	\[\op N_{M_pL_p/M_qL_p}(x_p)=x_q.\]
\end{lemma}
\begin{proof}
	This will require us to actually describe the Galois groups involved. Set $\sigma_K\in\op{Gal}(K^{\op{unr}}/K)$ to the Frobenius automorphism on $K$, but extend $\sigma_K$ to all $K^{\op{ab}}$ by acting trivially on totally ramified extensions. Additionally, for brevity we set
	\[f\coloneqq[K_m:K]\qquad\text{and}\qquad e_i\coloneqq[K_{\pi,i}:K]\]
	for each $i\ge0$. Now, the extension $M_pL_p/M_qL_p$ is unramified and hence has Galois group generated by its Frobenius element. The Frobenius element of $M_qL_p$ is equal to the Frobenius element of $M_q$ because the extension $M_jL_i/M_j$ is totally ramified, and because $M_q/K$ is unramified, we may compute the Frobenius element of $M_q$ as
	\[\sigma_K^{[M_q:K]},\]
	where $[M_q:K]=[L_q:K]=[L_q:K_{\pi,q}]\cdot[K_{\pi,q}:K]=[K_m:K]\cdot[K_{\pi,q}:K]=fe_q$. As for the order of $\op{Gal}(M_pL_p/M_qL_p)$, we first compute, for any $i\ge0$,
	\[[M_iL_i:L_i]=\frac{[M_iL_i:K]}{[L_i:K]}=\frac{[M_iL_i:M_i]\cdot[M_i:K]}{[L_i:K]}=[M_iL_i:M_i]=[K_{\pi,i}:K]=e_i,\]
	so the degree we want is $e_p/e_q$. Thus,
	\[\op N_{M_pL_p/M_qL_p}(x_p)=\prod_{i=0}^{e_p/e_q-1}\sigma_K^{fe_qi}(x_p).\]
	Now, we show that this equals $x_q$ by induction on $p$. When $p=q$, there is nothing to say. Then, supposing we have the equality at $p$, we write
	\begin{align*}
		\op N_{M_{p+1}L_{p+1}/M_qL_{p+1}}(x_{p+1}) &= \prod_{i=0}^{e_{p+1}/e_q-1}\sigma_K^{fe_qi}(x_{p+1}) \\
		&= \prod_{b=0}^{e_p/e_q-1}\prod_{a=0}^{e_{p+1}/e_p-1}\sigma_K^{fe_q(a(e_p/e_q)+b)}(x_{p+1}) \\
		&= \prod_{b=0}^{e_p/e_q-1}\sigma_K^{fe_qb}\Bigg(\prod_{a=0}^{e_{p+1}/e_p-1}\sigma_K^{fe_pa}(x_{p+1})\Bigg) \\
		&= \prod_{b=0}^{e_p/e_q-1}\sigma_K^{fe_qb}\Bigg(\prod_{a=0}^{e_{p+1}/e_p-1}\sigma_K^{fe_pa}(x_{p+1})\Bigg).
	\end{align*}
	Doing the same Galois theory, we see $\op{Gal}(M_{p+1}L_{p+1}/M_pL_{p+1})$ is cyclic generated by $\sigma_K^{fe_p}$ of order $e_{p+1}/e_p$, so the inner term is $\op N_{M_{p+1}L_{p+1}/M_pL_{p+1}}(x_{p+1})$, which we know to be $x_p$. Now, $x_p\in M_pL_p$, so in fact the entire product collapses to
	\[\op N_{M_{p+1}L_{p+1}/M_qL_{p+1}}(x_{p+1})=\op N_{M_pL_p/M_qL_p}(x_p)=x_q,\]
	which is what we wanted. This completes the proof.
\end{proof}
In particular, our sequence $\{\gamma_i\}_{i=0}^\infty$ compatible in towers with $\gamma_0=0$ will have
\[\op N_{M_iL_i/L_i}(\gamma_i)=\op N_{M_iL_i/M_0L_i}(\gamma_i)=\gamma_0=\pi_K\]
for each $i\ge0$, so these $\gamma_i\in M_iL_i$ do in fact satisfy the needed requirement of \autoref{cor:fundtriple}.

Thus, we have described how to construct our $\gamma$ terms in the tower, from which the rest of the fundamental tuple follows. However, we do remark that it is possible to choose the $\eta$ terms to be compatible in towers as well.
\begin{lemma}
	Fix everything as above. Further, fix some $\sigma\in\op{Gal}\big(\bigcup_{i\ge0}L_i/K\big)$. Then there exists a sequence $\{\eta_i\}_{i=0}^\infty$ compatible in towers such that
	\begin{equation}
		\frac{\eta_i}{\sigma_K^f(\eta_i)}=\frac{\sigma(\gamma_i)}{\gamma_i} \label{eq:etaequation}
	\end{equation}
	for each $i\ge0$.
\end{lemma}
\begin{proof}
	Well, to begin we have $\gamma_0=\pi_K$, which is fixed by $\sigma$, so the right-hand side is $1$, meaning that we might as well take $\eta_0=1$. We now claim that, given $\eta_i$ satisfying \autoref{eq:etaequation} which is a unit, we can construct $\eta_{i+1}$ with
	\[\op N_{M_{i+1}L_{i+1}/M_iL_{i+1}}(\eta_{i+1})=\eta_i\]
	also satisfying \autoref{eq:etaequation} (for $i+1$) which is a unit. For brevity, set ${\op N}\coloneqq\op N_{M_{i+1}L_{i+1}/M_iL_{i+1}}$. To begin, we note that $\eta_i$ is a unit in $M_iL_{i+1}$ as well, so because $M_{i+1}L_{i+1}/M_iL_{i+1}$ is unramified, we may simply guess any $\eta\in M_{i+1}L_{i+1}$ such that
	\[\op N(\eta)=\eta_i.\]
	We now need to correct for \autoref{eq:etaequation}. Well, we start by noting we're pretty close because
	\begin{align*}
		\op N\Bigg(\frac{\eta}{\sigma_K^f(\eta)}\bigg/\frac{\sigma(\gamma_{i+1})}{\gamma_{i+1}}\Bigg) &= \frac{\op N\eta}{\sigma_K^f(\op N\eta)}\bigg/\frac{\sigma(\op N\gamma_{i+1})}{\op N\gamma_{i+1}} \\
		&= \frac{\eta_i}{\sigma_K^f(\eta_i)}\bigg/\frac{\sigma(\gamma_{i})}{\gamma_{i}} \\
		&= 1.
	\end{align*}
	Now, $M_{i+1}L_{i+1}/M_iL_{i+1}$ is unramified and hence cyclic, and we know that its Galois group is generated by $\sigma_K^{fe_i}$ as computed earlier, so Hilbert's theorem 90 allows us to find some $u\in M_{i+1}L_{i+1}$ such that
	\[\frac{\eta}{\sigma_K^f(\eta)}\bigg/\frac{\sigma(\gamma_{i+1})}{\gamma_{i+1}}=\frac u{\sigma_K^{fe_i}u}.\]
	Quickly, note that we may multiply $u$ by any element in $M_iL_{i+1}$ without adjusting the equality. Thus, taking $\varpi$ to be a uniformizer of $M_iL_{i+1}$, we note that we can divide out $u$ by some number of $\varpi$s to force $u$ to be a unit because the extension $M_{i+1}L_{i+1}/M_iL_{i+1}$ is unramified, making $\varpi$ also a uniformizer of $M_{i+1}L_{i+1}$. This is all to say that we may assume that $u$ is a unit.

	Now, we note that
	\[\frac u{\sigma_K^{fe_i}u}=\prod_{k=0}^{e_i-1}\frac{\sigma_K^{fk}u}{\sigma_K^{f(k+1)}u}=\underbrace{\Bigg(\prod_{k=0}^{e_i-1}\sigma_K^{fk}u\Bigg)}_{v\coloneqq}/\sigma_K^f\Bigg(\prod_{k=0}^{e_i-1}\sigma_K^{fk}u\Bigg)=\frac{v}{\sigma_K^fv}.\]
	Because $u$ is a unit, $v$ is as well. In total, we see that
	\[\frac{\eta}{\sigma_K^f(\eta)}\bigg/\frac{\sigma(\gamma_{i+1})}{\gamma_{i+1}}=\frac u{\sigma_K^{fe_i}u}=\frac{v}{\sigma_K^fv}\]
	now implies that
	\[\frac{\eta/v}{\sigma_K^f(\eta/v)}=\frac{\sigma(\gamma_{i+1})}{\gamma_{i+1}}.\]
	Thus, we set $\eta_{i+1}\coloneqq\eta/v$, which we know to be a unit because both $\eta$ and $v$ are. This completes the inductive step and hence the proof.
\end{proof}
As such, we define $\{\eta_{\sigma,i}\}_{i=0}^\infty$ for each $\sigma\in\op{Gal}\big(\bigcup_{i\ge0}L_i/K\big)$ as constructed above, and we know these to be compatible in towers.

To finish our discussion, we note that because the expressions for the $\alpha_i$ and $\beta_{ij}$ are multiplicative and because norms commute with automorphisms in abelian extensions, choosing the $\gamma$s and $\eta$s to be compatible in towers will imply that the entire fundamental tuples will be (pointwise) compatible in towers.

As an example, we write this compatibility out for $\alpha_0$; the rest of the terms are similar. We define
\[\alpha_{0,i}\coloneqq\gamma_i\cdot\prod_{k=0}^{f-1}\sigma_K^k(\eta_{\sigma_K,i})\]
in accordance with \autoref{cor:fundtriple}. To check that this is compatible in towers, we set ${\op N}\coloneqq\op N_{M_{i+1}L_{i+1}/M_iL_{i+1}}$ for some index $i$ and compute
\begin{align*}
	\op N(\alpha_{0,i+1}) &= \op N\Bigg(\gamma_{i+1}\cdot\prod_{k=0}^{f-1}\sigma_K^k(\eta_{\sigma_K,i+1})\Bigg) \\
	&= \op N\gamma_{i+1}\cdot\prod_{k=0}^{f-1}\sigma_K^k(\op N\eta_{\sigma_K,i+1}) \\
	&= \gamma_i\cdot\prod_{k=0}^{f-1}\sigma_K^k(\eta_{\sigma_K,i}) \\
	&= \alpha_{0,i},
\end{align*}
which is what we wanted.

\section{Global Fundamental Classes}
% !TEX root = ../abeliangerbs.tex

\subsection{Set-Up and Idea}
Given an extension of global fields $L/K$, let $u_{L/K}\in H^2(\op{Gal}(L/K),\AA_L^\times/L^\times)$ denote the global fundamental class. Let $L/\QQ$ be a finite abelian extension of fields with Galois group $G\coloneqq\op{Gal}(L/K)$, where $G$ is a product of cyclic groups generated by $\tau_1,\ldots,\tau_m\in G$.

Now, let $n\coloneqq[L:\QQ]$. The idea for this computation is to choose a very large auxiliary prime $p$ such that $p\equiv1\pmod n$. In particular, because there are only finitely many primes $p$ such that $L\cap\QQ(\zeta_p)$ is strictly larger than $\QQ$, so we may find some $p\equiv1\pmod n$ such that $L\cap\QQ(\zeta_p)=\QQ$. For brevity, set $M\coloneqq\QQ(\zeta_p)$ and $K\coloneqq\QQ$.

As such, here is the set-up so far.
% https://q.uiver.app/?q=WzAsNCxbMSwzLCJcXFFRIl0sWzAsMiwiTCJdLFszLDEsIlxcUVEoXFx6ZXRhX3ApIl0sWzIsMCwiTChcXHpldGFfcCkiXSxbMCwxLCJuIiwwLHsic3R5bGUiOnsiaGVhZCI6eyJuYW1lIjoibm9uZSJ9fX1dLFswLDIsInAtMSIsMix7InN0eWxlIjp7ImhlYWQiOnsibmFtZSI6Im5vbmUifX19XSxbMiwzLCJuIiwyLHsic3R5bGUiOnsiaGVhZCI6eyJuYW1lIjoibm9uZSJ9fX1dLFsxLDMsInAtMSIsMCx7InN0eWxlIjp7ImhlYWQiOnsibmFtZSI6Im5vbmUifX19XV0=&macro_url=https%3A%2F%2Fraw.githubusercontent.com%2FdFoiler%2Fnotes%2Fmaster%2Fnir.tex
\[\begin{tikzcd}[column sep={1.5cm,between origins}]
	&& {ML} \\
	&&& {M} \\
	L \\
	& K
	\arrow["n", no head, from=4-2, to=3-1]
	\arrow["{p-1}"', no head, from=4-2, to=2-4]
	\arrow["n"', no head, from=2-4, to=1-3]
	\arrow["{p-1}", no head, from=3-1, to=1-3]
\end{tikzcd}\]
Here are some notable facts about this set-up.
\begin{itemize}
	\item The extension $\QQ(\zeta_p)/\QQ$ is totally ramified at $(p)$, so we can explicitly compute the global fundamental class $u_{M/K}$.\footnote{In fact, this extension has cyclic Galois group and therefore has inert primes, so this provides another means to compute the global fundamental class.}
	\item We have
	\begin{align*}
		\op{Inf}^{ML}_Lu_{L/K} &= [ML:L]u_{ML/K} \\
		&= (p-1)u_{ML/K} \\
		&= \frac{p-1}n\cdot nu_{ML/K} \\
		&= \frac{p-1}n\cdot[ML:M]u_{ML/K} \\
		&= \frac{p-1}n\cdot\op{Inf}^{ML}_Mu_{M/K}.
	\end{align*}
	\item Lastly, there is the inflation--restriction exact sequence
	\[0\to H^2\left(\op{Gal}(L/K),\AA_L^\times/L^\times\right)\to H^2\left(\op{Gal}(ML/K),\AA_{ML}^\times/ML^\times\right)\to H^2\left(\op{Gal}(ML/L),\AA_{ML}^\times/ML^\times\right),\]
	which tells us that $\op{Inf}^{ML}_L$ is injective and even provides an explicit way to invert it when possible.
\end{itemize}
Combining the above three observations will let us compute $\op{Inf}^{ML}_Lu_{L/K}$ and then invert inflation to recover $u_{L/K}$.

\subsection{Computation}

\section{Global Gerbs} \label{sec:global}
In this section we provide a concrete description of the Kottwitz gerbs $\mathcal E_2$ and $\mathcal E_3$ from \cite{kottwitz} associated to the global extension $\QQ(\zeta_{p^m})/\QQ$ when $p$ is a prime.

\subsection{Set-Up} \label{sec:globalsetup}
We quickly recall the construction of $\mathcal E_2$. Given a global field $K$, let $V_K$ denote the set of places of $K$. We follow \cite{kottwitz} and \cite{tate-torus}.

Fix an extension of global fields $L/K$ with Galois group $G\coloneqq\op{Gal}(L/K)$. For later use, we will also let $G_v\subseteq G$ denote the decomposition group of a place $v\in V_L$. Now, we have the two short exact sequences. To begin, we note that the augmentation map $\ZZ[V_K]\onto\ZZ$ induces the short exact sequence
\[0\to\ZZ[V_L]_0\to\ZZ[V_L]\to\ZZ\to0\label{eq:sesx}\tag{$X$}\]
where $\ZZ[V_L]$ is the kernel of $\ZZ[V_L]\onto\ZZ$. We also have the short exact sequence
\[1\to L^\times\to\AA_L^\times\to\AA_L^\times/L^\times\to1\tag{$A$}\]
where the inclusion $L^\times\into\AA_L^\times$ is diagonal.

Let $\mathbb D_2\coloneqq\op{Hom}_\ZZ(\ZZ[V_L],-)$ denote the protorus with character group $\ZZ[V_L]$. Then $\mathcal E_2(L/K)$ is the Galois gerb associated to a particular class $\alpha_2\in H^2\left(G,\mathbb D(\mathbb A_L)\right)$. To construct this class, we need the following lemma.
\begin{lemma}[{\cite[][p.~714]{tate-torus}}] \label{lem:magicaltate}
	Let $L/K$ be an extension of global fields with Galois group $G$, and let $V_L$ and $V_K$ denote the set of places of $L$ and $K$ respectively. Given a place $v\in V_L$, let $G_v\subseteq G$ denote its decomposition group. Then, for any $i\in\ZZ$,
	\[\widehat H^i(G,\op{Hom}_\ZZ(\ZZ[V_L],M))\simeq\prod_{u\in V_K}\widehat H^i(G_{v(u)},M),\]
	where the product is over places $u\in V_K$ taking a fixed place $v(u)\in V_L$ above $u$.
\end{lemma}
\begin{proof}
	We give the proof for later use. This is essentially a matter of separating our places and then applying Shapiro's lemma. For each $u\in V_K$, let $V_{u}\subseteq V_L$ denote the set of places in $L$ above $u$. Then we see
	\[\ZZ[V_L]\simeq\bigoplus_{u\in V_K}\ZZ[V_u]\]
	as $G$-modules because the $G$-orbit of a place $v\in V_L$ lying over a place $u\in V_K$ is exactly $V_u$. Thus, we have the isomorphisms
	\begin{align*}
		\widehat H^i(G,\op{Hom}_\ZZ(\ZZ[V_L],M)) &\simeq \widehat H^i\left(G,\op{Hom}_\ZZ\Bigg(\bigoplus_{u\in V_L}\ZZ[V_u],M\Bigg)\right) \\
		&\simeq \widehat H^i\left(G,\prod_{u\in V_K}\op{Hom}_\ZZ(\ZZ[V_u],M)\right) \\
		&\simeq \prod_{u\in V_K}\widehat H^i\left(G,\op{Hom}_\ZZ(\ZZ[V_u],M)\right).
	\end{align*}
	It remains to show that
	\[\widehat H^i\left(G,\op{Hom}_\ZZ(\ZZ[V_u],M)\right)\stackrel?\simeq\widehat H^i(G_{v(u)},M).\]
	Well, for each place $u\in V_K$, find a place $v(u)\in V_L$ above it. As discussed above, $V_u$ is a transitive $G$-set, and the stabilizer of $v(u)$ is $G_{v(u)}$. Thus, $V_u\simeq G_{v(u)}\backslash G$ as $G$-sets (note the distinction between left and right $G$-sets is somewhat irrelevant because $gG_v=G_vg$ for each $g\in G_v$), so $\ZZ[V_u]\simeq\ZZ[G_{v(u)}\backslash G]$ as $G$-modules. Thus, we may write
	\begin{align*}
		\widehat H^i\left(G,\op{Hom}_\ZZ(\ZZ[V_u],M)\right) &\simeq \widehat H^i\left(G,\op{Hom}_\ZZ(\ZZ[G_{v(u)}\backslash G],M)\right) \\
		&\simeq \widehat H^i\left(G,\op{Mor}_{\mathrm{Set}}(G_{v(u)}\backslash G,M)\right) \\
		&\simeq \widehat H^i\big(G,\op{CoInd}_{G_{v(u)}}^G(M)\big),
	\end{align*}
	where the last isomorphism is because $\op{Mor}_{\mathrm{Set}}(G_{v(u)}\backslash G,M)\simeq\op{CoInd}_H^G(M)$ by taking $f\colon G_{v(u)}\backslash G\to M$ to the function $g\mapsto gf\left(G_vg^{-1}\right)$. Now, this last cohomology group is isomorphic to $\widehat H^i(G_{v(u)},M)$ by Shapiro's lemma, thus finishing.
\end{proof}
\begin{remark} \label{rem:forwardshapiro}
	Tracking through the application of Shapiro's lemma above, we can see that the isomorphism behaves as
	\[\widehat H^i(G,\op{Hom}_\ZZ(\ZZ[V_L],M))\stackrel{\op{Res}}\to\widehat H^i(G_{v},\op{Hom}_\ZZ(\ZZ[V_L],M))\stackrel{\op{eval}_v}\to\widehat H^i(G_{v},M)\]
	on components; here $\op{eval}_v$ is induced by the evaluation-at-$v$ map $\op{Hom}_\ZZ(\ZZ[V_L],M)\to M$.
\end{remark}
Thus, to specify $\alpha_2\in\widehat H^2(G,\mathbb D_2(\AA_L))$, it is enough to specify a set of classes
\[\alpha_2(u)\in\widehat H^2\left(G_{v(u)},\AA_L^\times\right)\]
for each $u\in V_K$. To do so, we note that $G_{v(u)}=\op{Gal}(L_{v(u)}/K_u)$, so we use the natural embedding $i_v\colon L_v\into\AA_L^\times$ (for $u\in V_L$) to set
\[\alpha_2(u)\coloneqq i_{v(u)}\big(\alpha(L_{v(u)}/K_u)\big),\]
where $\alpha(L_{v(u)}/K_u)\in\widehat H^2\left(G_{v(u)},L_{v(u)}^\times\right)$ is the local fundamental class.

\subsection{An Explicit Cocycle} \label{subsec:computee2}
We continue in the context of \autoref{sec:globalsetup}, in the case of $K\coloneqq\QQ$ and $L\coloneqq\QQ(\zeta_{p^\nu})$; for brevity, set $\zeta\coloneqq\zeta_{p^\nu}$. The goal of the computation is to fully reverse \autoref{lem:magicaltate} to be able to write down a $2$-cocycle in $Z^2(G,\mathbb D_2(\AA_L))$ representing $\alpha_2$, which will then specify a gerb in the correct equivalence class of $\mathcal E_2$. As such, for each $u\in V_K$, we choose some $v(u)\in V_L$ above $u$.

\subsubsection{Extracting Elements}
We are going to choose our local fundamental class representatives to be compatible with a choice of global fundamental class for $L/K$. However, this will require extracting certain magical elements of $L^\times$, so we will go ahead and extract these before getting into the computation.

To begin, we need to write down $G\coloneqq\op{Gal}(\QQ(\zeta)/\QQ)$ in some concrete way, so we pick a generator $x\in\left(\ZZ/p^\nu\ZZ\right)^\times$ (recall that $p$ is odd) so that $\sigma\colon\zeta\mapsto\zeta^x$ is a generator of $G$ of order $n\coloneqq\varphi\left(p^\nu\right)=(p-1)p^{\nu-1}$. To be able to properly localize, for each prime $q\ne p$, we define $k_q\ge0$ to have
\[x^{k_q}\equiv q\pmod p\]
so that $\sigma^{k_q}\colon\zeta\mapsto\zeta^q$. We also set $d_q\coloneqq\gcd(k_q,n)$ so that $\langle\sigma^{k_q}\rangle=\langle\sigma^{d_q}\rangle$ with order $n_q\coloneqq n/d_q$.

Additionally, we let $\mf P$ denote the prime of $L$ above $(p)$ of $K$; notably, $L/K$ is totally ramified at $(p)$, so there is in fact one prime $\mf P$ here. In particular, we can check that
\[c_p(\sigma^i,\sigma^j)=x^{-\floor{\frac{i+j}n}}\]
is a $2$-cocycle in $Z^2(G,L_\mf P/K_{(p)})$ representing the local fundamental class in $\widehat H^2(G,L_\mf P^\times)$. Passing $c_\mf P$ through $L_\mf P^\times\into\AA_L^\times\onto\AA_L^\times/L^\times$, we see that
\[i_\mf Pc_p(\sigma^i,\sigma^j)=i_\mf Px^{-\floor{\frac{i+j}n}}\]
has cohomology class of global invariant $1/n$ and therefore represents the global fundamental class $u_{L/K}\in\widehat H^2(G,\AA_L^\times/L^\times)$.

We now start choosing elements of $L^\times$. The following conjures the element that we need for infinite places. Set $\tau\coloneqq\sigma^{n/2}$ to be the ``conjugation'' action on $L$.
\begin{lemma}
	Let $v\coloneqq v(\infty)$ be our chosen infinite place, and set $G_v=\{1,\tau\}$. Then there exists $\xi_\infty\in L^{\langle\tau\rangle}$ such that
	\[\xi_\infty\equiv i_v(-1)\cdot i_\mf Px\pmod{N_{\langle\tau\rangle}\AA_L^\times}.\]
\end{lemma}
\begin{proof}
	It is a fact that we can represent the local fundamental class of $L_v/K_\infty$ by
	\[c_v(\tau^i,\tau^j)=(-1)^{\floor{\frac{i+j}2}}.\]
	Again, embedding this into $\AA_L^\times/L^\times$, we see that
	\[i_v c_v(\tau^i,\tau^j)=i_v(-1)^{\floor{\frac{i+j}2}}\]
	has global invariant $1/2$ and therefore should live in the same cohomology class as $\op{Res}_{G_v}i_\mf Pc_\mf P$. In particular, we place $[\tau]\in\widehat H^{-2}(G_v,\ZZ)$ and note that
	\[[i_v c_v]\cup[\tau]=[n/2\cdot i_\mf Pc_p]\cup[\tau]\]
	as elements in $\widehat H^0(G_v,\AA_L^\times/L^\times)$. Rearranging, this implies that
	\[[1]=[i_v(-1)\cdot i_\mf Px]\]
	as elements in $\widehat H^0(G_v,\AA_L^\times/L^\times)$. Now, this group is $\AA_L^\times/L^\times$ modded out by $N_{G_v}\AA_L^\times$, so we can unwind this as promising some $\xi_\infty\in L^\times$ such that
	\[\xi_\infty\equiv i_v(-1)\cdot i_\mf Px\pmod{N_{G_v}\AA_L^\times}.\]
	It remains to show that $\xi_\infty\in L^{\langle\tau\rangle}$. Well, the above turns into
	\[\xi_\infty=i_v(-1)\cdot i_\mf Px\cdot a\cdot\tau a\]
	for some $a\in\AA_L^\times$, and this equality has each factor on the right-hand side fixed by $\tau$.
\end{proof}
\begin{remark}
	For certain primes, one can choose $\xi_\infty$ from the circulant units of $\QQ(\zeta_p)$, making $\xi_\infty$ effectively computable. However, in general this does not work; this fails first for $\QQ(\zeta_{29})$.
\end{remark}
Continuing, we note that, because $G_v$ is preserved by conjugation, we have
\[g\xi_\infty\equiv i_{gv}(-1)\cdot i_\mf Px\pmod{N_{G_{gv}}\AA_L^\times}\]
as well, so we set $\xi_{gv}\coloneqq g\xi_\infty$. Because $\xi_\infty$ is preserved by $\tau$, the choice of $g\in G$ yielding $gv$ is irrelevant.

We are going to want to ``inflate'' $\xi_v$ to be helpful with larger subgroups, for which we establish the following lemma.
\begin{lemma}
	Fix everything as above. Picking any infinite place $v\mid\infty$ and subgroup $H\subseteq G$ containing $\tau$, the element
	\[\xi_{v,H}\coloneqq\prod_{g\langle\tau\rangle\in H/\langle\tau\rangle}g\xi_v\]
	has
	\[\xi_{v,H}\in L^H\qquad\text{and}\qquad\xi_{v,H}\equiv i_\mf Px^{\#H/2}\cdot\prod_{w\in Hv}i_w(-1)\pmod{N_H\AA_L^\times}.\]
\end{lemma}
Technically, we must choose some coset representatives for $H/\langle\tau\rangle$ to define $\xi_{v,H}$, but because $\xi_v$ is fixed by $\tau$, they all yield the same element of $L^H$.
\begin{proof}
	By construction,
	\[\xi_v=i_\mf Px\cdot i_v(-1)\cdot N_{\langle\tau\rangle}a\]
	for some $a\in\AA_L^\times$. Now, we choose coset representatives $\{g_1,\ldots,g_m\}$ for $H/\langle\tau\rangle$ so that
	\begin{align*}
		\xi_{v,H} &= \prod_{k=1}^mg_k\xi_v \\
		&= \Bigg(\prod_{k=1}^mg_ki_\mf Px\Bigg)\Bigg(\prod_{k=1}^mg_ki_v(-1)\Bigg)\Bigg(\prod_{k=1}^mg_k(a\cdot\tau a)\Bigg) \\
		&= \Bigg(\prod_{k=1}^mi_{g_k\mf P}(g_kx)\Bigg)\Bigg(\prod_{k=1}^mi_{g_kv}g_k(-1)\Bigg)\Bigg(\prod_{k=1}^mg_ka\cdot g_k\tau a\Bigg) \\
		&= i_\mf Px^{\#H/2}\Bigg(\prod_{k=1}^mi_{g_kv}(-1)\Bigg)N_Ha.
	\end{align*}
	Quickly, we show that the (multi)set of $g_kv$ is the same as $Hv$. Well, $gv=v$ if and only if $g\in\langle\tau\rangle$, so the stabilizer of $v$ in the $H$-set in $Hv$ is $\langle\tau\rangle$. It follows that there is an isomorphism $H/\langle\tau\rangle\cong Hv$ of $H$-sets, which is what we wanted.

	Thus,
	\[\xi_{v,H}=i_\mf Px^{\#H/2}\Bigg(\prod_{w\in Hv}i_w(-1)\Bigg)N_Ha.\]
	To show $\xi_{v,H}\in L^H$, we observe that the above factors are each fixed by $H$, finishing.
\end{proof}
Next we turn to our finite unramified places. The following is the key idea.
\begin{lemma} \label{lem:magicalprimes}
	Fix everything as above. For each subgroup $H\subseteq G$ and ideal class $c\in\op{Cl}L^H$, there exists a prime $L^H$-ideal $\mf r_{H,c}$ satisfying the following constraints.
	\begin{itemize}
		\item $\mf r_{H,c}$ has ideal class $c$.
		\item $\mf r_{H,c}$ splits completely in $L$.
	\end{itemize}
\end{lemma}
\begin{proof}
	This is an application of the Chebotarev density theorem. Let $M$ be the Hilbert class field of $L^H$, yielding the following tower of fields.
	% https://q.uiver.app/?q=WzAsNSxbMSwwLCJNTCJdLFswLDEsIk0iXSxbMSwyLCJMXkgiXSxbMiwxLCJMIl0sWzEsMywiSyJdLFs0LDIsIiIsMCx7InN0eWxlIjp7ImhlYWQiOnsibmFtZSI6Im5vbmUifX19XSxbMiwxLCIiLDAseyJzdHlsZSI6eyJoZWFkIjp7Im5hbWUiOiJub25lIn19fV0sWzEsMCwiIiwwLHsic3R5bGUiOnsiaGVhZCI6eyJuYW1lIjoibm9uZSJ9fX1dLFsyLDMsIiIsMCx7InN0eWxlIjp7ImhlYWQiOnsibmFtZSI6Im5vbmUifX19XSxbMywwLCIiLDEseyJzdHlsZSI6eyJoZWFkIjp7Im5hbWUiOiJub25lIn19fV1d&macro_url=https%3A%2F%2Fraw.githubusercontent.com%2FdFoiler%2Fnotes%2Fmaster%2Fnir.tex
	\[\begin{tikzcd}
		& ML \\
		M && L \\
		& {L^H} \\
		& K
		\arrow[no head, from=4-2, to=3-2]
		\arrow[no head, from=3-2, to=2-1]
		\arrow[no head, from=2-1, to=1-2]
		\arrow[no head, from=3-2, to=2-3]
		\arrow[no head, from=2-3, to=1-2]
	\end{tikzcd}\]
	The main claim is that $M\cap L=L^H$. Certainly $M\cap L$ contains $L^H$, so we make the following two observations.
	\begin{itemize}
		\item Because $M\cap L$ is a subextension of the unramified extension $L^H\subseteq M$, the extension $L^H\subseteq M\cap L$ is also unramified.
		\item Because the extension $L^H\subseteq L$ is totally ramified, the only way for a sub-extension to be unramified is for the subextension to be $L^H$.
	\end{itemize}
	Combining the above two observations forces $M\cap L=L^H$.

	It follows that $M$ and $L$ are linearly disjoint over $L^H$, so
	\[\op{Gal}(ML/L^H)\simeq\op{Gal}\big(M/L^H\big)\times\op{Gal}\big(L/L^H\big)\simeq\op{Cl}L^H\times H.\]
	Thus, choose $g\in\op{Gal}(M/L^H)$ corresponding to $c\in\op{Cl}L^H$ and then use the Chebotarev density theorem to find a prime $L^H$-ideal $\mf r$ such that $\op{Frob}_\mf r=(g,1)$. We claim that $\mf r_{H,c}\coloneqq\mf r$ will do the trick.
	
	For concreteness, let $\mf R$ be a prime of $ML$ above $\mf r$, and set $\mf R_M\coloneqq\mf R\cap M$ and $\mf R_L\coloneqq\mf R\cap L$. Then
	\[\op{Frob}_{\mf R_M/\mf r}=\op{Res}_M\op{Frob}_{\mf R/\mf r}=g,\]
	so $\mf r$ has the correct ideal class. Similarly,
	\[\op{Frob}_{\mf R_L/\mf r}=\op{Res}_L\op{Frob}_{\mf R/\mf r}=1,\]
	so $\mf r$ splits completely up in $L$.
\end{proof}
Now, let $(q)\ne(p)$ be a finite prime of $K$, and choose some place $v\coloneqq v(u)\in V_L$ above $(q)$ corresponding to the prime $\mf Q$. Intersecting down, set $\mf q\coloneqq\mf Q\cap L^{G_v}$.

We will want to choose a well-behaved uniformizer of $\mf q$ to represent our local fundamental class. Choosing $q\in\mf q$ turns out to cause difficulties when $\mf q$ is not inert in $L$. Instead, we use \autoref{lem:magicalprimes} to find the constructed $L^{G_v}$-prime $\mf r_u$ such that $\mf r_u$ splits completely in $L$ and $\mf q\mf r_u$ is principal. As such, we find $\varpi_u\in L^{G_v}$ such that
\[\mf q\mf r_u=(\varpi_u).\]
Observe that if we work with $gv(u)$ instead of $v(u)$ for some $g\in G$, we can analogously write
\[(g\mf q)(g\mf r_u)=(g\varpi_u),\]
so we set $\varpi_{gv(u)}\coloneqq g\varpi_u$ for $g\in G$. Observe that this is well-defined: $gv(u)=g'v(u)$ implies that $g^{-1}g'\in G_v$, so $g^{-1}g\varpi_u=\varpi_u$, so $g\varpi_u=g'\varpi_u$.

% We have one more technical point to cover. Given a subgroup $H\subseteq G$ and $e\in\ZZ/\#H\ZZ$, it is technically possible for
% \[\left[i_\mf Px^e\right]=[1]\]
% as elements of $\widehat H^0(H,\AA_L^\times/L^\times)$. As such, there exists some $\delta_{H,e}\in L^\times$ such that
% \[\delta_{H,e}\equiv i_\mf Px^e\pmod{N_H\AA_L^\times}.\]
% As usual, writing out $\delta_{H,e}=i_\mf Px^e\cdot N_Ha$ for some $a\in\AA_L^\times$ reveals that $\delta_{H,e}$ is preserved by $H$, meaning that $\delta_{H,e}\in L^H$.
% \begin{remark}
% 	If $H\subseteq L^{\langle\tau\rangle}$, it is possible to have a totally positive unit in $\mathcal O_{L^H}$; these form examples of $\delta$ elements. These sorts of elements are annoying to pin down exactly, so the above elements appear difficult to write down explicitly.
% \end{remark}

\subsubsection{Choosing Local Fundamental Cocycles}
To work up \autoref{lem:magicaltate}, we must find explicit $2$-cocycles to represent the various $i_{v(u)}\alpha(L_{v(u)}/K_u)$s. Some of these will be easy. For example, for $v=v((p))=\mf P$, we can set
\[c_p(\sigma^i,\sigma^j)=x^{-\floor{\frac{i+j}n}}\]
to represent $u_{L_\mf P/K_{(p)}}\in\widehat H^2(G,L_\mf P^\times)$, so we set $\widetilde c_p\coloneqq i_\mf Pc_p$.

Additionally, for $v=v(\infty)$, we set
\[c_\infty(\tau^i,\tau^j)=(-1)^{\floor{\frac{i+j}2}}\]
to represent $u_{L_v/K_\infty}\in\widehat H^2(G,L_v^\times)$. However, we won't want to use $i_vc_\infty$ for our $2$-cocycle. Instead, we recall that
\[[i_vc_\infty]\cup[\tau]=[i_v(-1)]=[\xi_\infty/i_\mf Px]\]
as elements of $\widehat H^0(G_v,\AA_L^\times)$. Thus, $[i_vc_\infty]$ is also represented by
\[\widetilde c_\infty(\tau^i,\tau^j)\coloneqq(\xi_\infty/i_\mf Px)^{\floor{\frac{i+j}2}}\]
by cupping with $(\tau^i,\tau^j)\mapsto\floor{\frac{i+j}2}$, which represents the generator of $\widehat H^2(G_v,\ZZ)$.

Lastly, we let $u=(q)\ne(p)$ denote a finite (unramified) place of $V_K$, and we set $v\coloneqq v(u)$ associated to the finite prime $\mf Q$. For brevity, set $H\coloneqq G_v$, and note $H=\langle\sigma^{k_q}\rangle$ because $\sigma^{k_q}\colon\zeta\mapsto\zeta^{k_q}$. Now, because our chosen $\varpi_u$ is a uniformizer of $\mf Q\cap L^{G_v}$, we can set
\[\left(\sigma^{k_qi},\sigma^{k_qj}\right)\mapsto\varpi_u^{\floor{\frac{i+j}{n_q}}}\]
to represent $u_{L_v/K_u}\in\widehat H^2(H,L_v^\times)$. It will be helpful to be able to change between generators, so we pick up the following lemma.
\begin{lemma} \label{lem:cocyclegenchange}
	Let $G=\langle\sigma\rangle$ be a finite cyclic group of order $n$. Further, suppose $k\in\ZZ$ has $\gcd(k,n)=1$. Then define $\chi,\chi_d\in Z^2(G,\ZZ)$ by
	\[\chi\left(\sigma^i,\sigma^j\right)\coloneqq\floor{\frac{i+j}n}\qquad\text{and}\qquad\chi_k\left(\sigma^{ki},\sigma^{kj}\right)\coloneqq\floor{\frac{i+j}n},\]
	where $0\le i,j<n$. Then $[\chi]=k[\chi_k]$ in $H^2(G,\ZZ)$.
\end{lemma}
\begin{proof}
	It is well-known that
	\begin{equation}
		(-\cup[\chi_k])\colon\widehat H^0(G,\ZZ)\to\widehat H^2(G,\ZZ) \label{eq:cycliccupiso}
	\end{equation}
	is an isomorphism. Now, for $m\in\ZZ$, we see that $[m]\cup[\chi_k]=[m\chi_k]$, so we see that we can actually invert the above isomorphism explicitly because
	\[\sum_{g\in G}(m\chi_k)\left(g,\sigma^k\right)=m\sum_{\ell k=0}^{n-1}\chi_k\left(\sigma^{\ell k},\sigma^k\right)=m,\]
	so $[c]\mapsto[c]\cup\left[\sigma^k\right]=\Big[\sum_{g\in G}c(g,\sigma^k)\Big]$ describes the inverse of \autoref{eq:cycliccupiso}. As such, we pick up $\chi$ and compute
	\[\sum_{g\in G}\chi\left(g,\sigma^k\right)=\sum_{\ell=0}^{n-1}\chi\left(\sigma^\ell,\sigma^k\right)=k.\]
	Thus, $[k]\cup[\chi_k]=[\chi]$, which is what we wanted.
\end{proof}
As such, we set $\chi_{d_q}\in Z^2(G,\ZZ)$ by $\chi_{d_q}\colon\left(\sigma^{d_qi},\sigma^{d_qj}\right)\mapsto\floor{\frac{i+j}n}$. Then \autoref{lem:cocyclegenchange} tells us that
\[[\chi_{d_q}]=(k_q/d_q)[\chi_{k_q}].\]
Thus, we find $y_q\in\ZZ$ with $y_q\cdot k_q/d_q\equiv1\pmod{n_q}$ so that we can represent $\alpha(L_{v}/K_u)$ by
\[([\varpi_u]\cup y_q\chi_{d_q})\colon\left(\sigma^{d_qi},\sigma^{d_qj}\right)\mapsto\varpi_u^{y_q\floor{\frac{i+j}n}}.\]
For brevity, let this $2$-cocycle be $c_q\in Z^2\big(H,L_{v}^\times\big)$.

Again, we won't want to represent $i_vu_{L_v/K_u}\in\widehat H^2(H,\AA_L^\times)$ by $i_vc_q$. To find the desired representative, we begin by embedding $\varpi_u\in L^\times$ to $\AA_L^\times$, yielding
\[\varpi_u=\prod_{w\in V_L}i_w\varpi_u.\]
We claim that if $v'\in V_L$ is a finite place not lying over $(p)$, $\mf q$, nor $\mf r$, then
\begin{equation}
	\prod_{w\in Hv'}i_w\varpi_u \label{eq:singleplacenorm}
\end{equation}
is a norm in $N_{H}\AA_L^\times$. Indeed, all places in $Hv'$ are unramified (they don't lie over $(p)$), and the fact that $v'$ avoids both $\mf q$ and $\mf r$ implies that $\varpi_u\in\mathcal O_w^\times$ for each $w\in Hv'$. In particular, there is some $a_{v'}\in L_{v'}$ such that $\varpi_u=N_{H_{v'}}a$, so
\[N_H(i_{v'}a_{v'})=\prod_{h\in H}i_{hv'}ha_{v'}=\prod_{[h_0]\in H/H_{v'}}i_{h_0v'}\Bigg(h_0\prod_{h\in H_{v'}}ha_{v'}\Bigg)=\prod_{w\in Hv'}i_w\varpi_u,\]
where the last equality used the fact that $\varpi_u$ is fixed by $h_0\in H$. Now, multiplying elements of the form \autoref{eq:singleplacenorm} together, we conclude that
\begin{equation}
	\varpi_u\equiv i_v\varpi_u\cdot i_\mf P\varpi_u\cdot\prod_{w\mid\mf r}i_w\varpi_u\cdot\prod_{w\mid\infty}i_w\varpi_u\pmod{N_H\AA_L^\times}. \label{eq:varpiinitial}
\end{equation}
We deal with the remaining terms one at a time, in sequence.
\begin{lemma} \label{lem:constructgeneralxis}
	Fix everything as above, with finite place $u$ not above $(p)$ chosen. Then there exists $\xi_u\in L^\times$ and $e_u\in\ZZ$ such that
	\[\xi_u\varpi_u\equiv i_v\varpi_u\cdot i_\mf Px^{e_u}\pmod{N_H\AA_L^\times}.\]
\end{lemma}
\begin{proof}
	Looking at \autoref{eq:varpiinitial}, we have to deal with places about $\mf r$ and places above $\infty$. We deal with these separately.

	Let's begin with the places above $\mf r$. Fix some $v'$ above $\mf r$. Because $\mf r$ is totally split in $L$, we have $H_{v'}=\{1\}$, so
	\[N_H(i_{v'}\varpi_u)=\prod_{h\in H}i_{hv'}\varpi_u=\prod_{w\mid\infty}i_w\varpi_u.\]
	So the places over $\mf r$ actually dissolve into a norm, implying
	\[\varpi_u\equiv i_v\varpi_u\cdot i_\mf P\varpi_u\cdot\prod_{w\mid\infty}i_w\varpi_u\pmod{N_H\AA_L^\times}.\]
	Next we turn to the infinite places. We begin by fixing some infinite place $v'\mid\infty$. We have two cases.
	\begin{itemize}
		\item If $\tau\notin H$, then we see that
		\[N_Hi_{v'}\varpi_u=\prod_{h\in H}i_{hv'}h\varpi_u=\prod_{w\in Hv'}i_w\varpi_u,\]
		where the last step is because $hv'=h'v'$ for $h,h'\in H$ implies $h=h'$. Thus, these are all norms.
		\item Otherwise, $\tau\in H$. For concreteness, associate $v'$ to the embedding $\sigma\colon L\to\CC$; note $hv$ is associated to the embedding $L\stackrel h\to L\to\CC$. In fact, $\sigma(L^H)\subseteq\RR$ because $L^H$ is fixed by $\tau\in H$, so we'll consider
		\[i_{v'}\sqrt{\sigma(\varepsilon_{u,v'}\varpi_u)}\in\AA_L^\times,\]
		where the sign $\varepsilon_{u,v'}\in\{\pm1\}$ is chosen to ensure $\sigma(\varepsilon_{u,v'}\varpi_u)>0$. Thinking concretely, $\sqrt{\varepsilon_{u,v'}\sigma\varpi_u}$ is a Cauchy sequence of elements of $L^H$ under the metric induced by $\sigma\colon L^H\to\RR$, whose square approaches $\varepsilon_{u,v'}\sigma\varpi_u>0$. Notably, we may choose a Cauchy sequence for our square root from $L^H$ because $\sigma(\varepsilon_{u,v'}\varpi_u)>0$.
		
		Applying $h\colon L_{v'}\to L_{hv'}$ to this Cauchy sequence, we get another Cauchy sequence, but this time the Cauchy sequence is under the metric induced by $\sigma h^{-1}\colon L^H\to\RR$ and approaches $\varepsilon_{u,v'}\sigma h\varpi_u$. However, these metric are the same, and $h\varpi_u=\varpi_u$, meaning that applying $h$ here merely produced another $\sqrt{\varepsilon_{u,v'}\sigma\varpi_u}\in L_{hv'}$. The whole point of this is to be able to write
		\begin{align*}
			N_Hi_{v'}\sqrt{\sigma(\varepsilon_{u,v'}\varpi_u)} &= \prod_{h\in H}hi_{v'}\sqrt{\sigma(\varepsilon_{u,v'}\varpi_u)} \\
			&= \prod_{h\langle\tau\rangle\in H/\langle\tau\rangle}i_{hv'}\left(\sqrt{\sigma(\varepsilon_{u,v'}\varpi_u)}\cdot\tau\sqrt{\sigma(\varepsilon_{u,v'}\varpi_u)}\right) \\
			&= \prod_{w\in Hv'}i_w(\varepsilon_{u,v'}\varpi_u).
		\end{align*}
		In total, we see that
		\[\prod_{w\in Hv'}i_w\varpi_u\equiv\prod_{w\in Hv'}i_w(\varepsilon_{u,v'})\equiv\left(\xi_{v',H}\cdot i_\mf Px^{-\#H/2}\right)^{(1-\varepsilon_{u,v'})/2}\]
		by \autoref{lem:constructgeneralxis}.
	\end{itemize}
	We now synthesize. If $\tau\in H$, then we take $\xi_u=1$ and $e_u=0$ so that \autoref{eq:varpiinitial} gives
	\[\varpi_u\equiv i_v\varpi_u\cdot i_\mf P\varpi_u\pmod{N_H\AA_L^\times}.\]
	When $\tau\in H$, this is a little more complicated. For notational reasons, we will let $V_\infty$ denote the set of infinite places in $V_L$, letting us write
	\begin{align*}
		\prod_{w\in V_\infty}i_w\varpi_u &= \prod_{[v']\in V_\infty/H}\prod_{w\in Hv'}i_{hw}\varpi_u \\
		&\equiv \prod_{[v']\in V_\infty/H}\left(\xi_{v',H}\cdot i_\mf Px^{-\#H/2}\right)^{(1-\varepsilon_{u,v'})/2} \\
		&\equiv \prod_{[v']\in V_\infty/H}\xi_{v',H}^{^{(1-\varepsilon_{u,v'})/2}}\cdot\prod_{[v']\in V_\infty/H} i_\mf Px^{-\#H/2\cdot(1-\varepsilon_{u,v'})/2} \pmod{N_H\AA_L^\times}.
	\end{align*}
	So we can collapse this product down to $\xi_u^{-1}\cdot i_\mf Px^{e_u}$ as above. Plugging into \autoref{eq:varpiinitial} gets the result.
\end{proof}
Lastly, we fix the $i_\mf P$ term. For this, we use the following lemma.
\begin{lemma} \label{lem:fixpplace}
	Fix everything as above. Suppose that we have a subgroup $H\subseteq G$ and power $e\in\ZZ$ such that
	\[[i_\mf Px^e]=[1]\]
	as elements of $\widehat H^0(H,\AA_L^\times/L^\times)$. Then
	\[i_\mf Px^e\equiv1\pmod{N_H\AA_L^\times}.\]
\end{lemma}
\begin{proof}
	The point is to show that $\#H\mid e$. Let $H=\langle\sigma^d\rangle$ for a fixed $d\mid n$. We have already established that
	\[(\sigma^i,\sigma^j)\mapsto i_\mf Px^{-\floor{\frac{i+j}n}}\]
	represents the fundamental class of $\widehat H^2(G,\AA_L^\times/L^\times)$, so restricting implies that
	\[(\sigma^{di},\sigma^{dj})\mapsto i_\mf Px^{-\floor{\frac{i+j}{n/d}}}\]
	represents the fundamental class of $\widehat H^2(H,\AA_L^\times/L^\times)\simeq\ZZ/\#H\ZZ$. Cupping with $[\sigma^d]\in\widehat H^{-2}(H,\ZZ)$ reveals that $i_\mf Px^{-1}$ is a generator of $\widehat H^0(H,\AA_L^\times/L^\times)$ of order $\#H$.

	Thus,
	\[[i_\mf Px]^e=[1]\]
	as elements of $\widehat H^0(H,\AA_L^\times/L^\times)$ implies that $\#H\mid e$. In particular, we conclude that $\#H\mid e$. To finish, we see that
	\[N_Hi_\mf Px^{e/\#H}=i_\mf Px^e,\]
	finishing.
\end{proof}
\begin{remark}
	The above lemma has the amusing corollary that all totally positive units of $\QQ(\zeta_{p^m})$ must be equivalent to $1\pmod{\mf P}$, where $\mf P=(1-\zeta_{p^m})$ is the (unique) prime lying above $(p)$.
\end{remark}
Currently, we have some $\xi_u$ and $e_u$ such that
\[\xi_u\varpi_u\equiv i_v\varpi_u\cdot i_\mf Px^{e_u}\pmod{N_H\AA_L^\times}.\]
However, we know abstractly that the $2$-cocycles $i_vc_q$ and $\op{Res}\widetilde c_p$ both represent the fundamental class of $\widehat H^2(H,\AA_L^\times/L^\times)$, which means that they need to have the same cup product with $\left[\sigma^{d_q}\right]$, giving the equality
\[\left[i_v\varpi_u^{y_q}\right]=\left[i_\mf Px^{-1}\right]\]
as elements of $\widehat H^0(H,\AA_L^\times/L^\times)$. Combining,
\[[1]=\left[i_v\varpi_u^{y_q}\cdot i_\mf Px^{y_qe_u}\right]=[i_\mf Px^{y_qe_u-1}]=[i_\mf Px]^{y_qe_u-1}\]
as elements of $\widehat H^0(H,\AA_L^\times/L^\times)$. Thus, \autoref{lem:fixpplace} lets us conclude that
\[i_\mf Px^{y_qe_u}\equiv i_\mf Px\pmod{N_H\AA_L^\times}.\]
Thus,
\[(\xi_u\varpi_u)^{y_q}\equiv i_v\varpi_u^{y_q}\cdot i_\mf Px\pmod{N_H\AA_L^\times}.\]
In total, we can choose
\[\widetilde c_q\left(\sigma^i,\sigma^j\right)\coloneqq\left(\xi_u^{y_q}\varpi_u^{y_q}/i_\mf Px\right)^{\floor{\frac{i+j}{n_q}}}\]
to represent $i_vu_{L_v/K_u}\in\widehat H^2(H,\AA_L^\times)$.

To synthesize all places, we set
\begin{equation}
	\omega_u\coloneqq\begin{cases}
		1 & u=(p), \\
		\xi_\infty & u=\infty, \\
		\xi_u^{y_q}\varpi_u^{y_q} & u\notin\{(p),\infty\},
	\end{cases}\qquad\text{and}\qquad d_u\coloneqq\begin{cases}
		d_q & u=q\ne p\text{ is finite}, \\
		1 & u=p, \\
		n/2 & u=\infty,
	\end{cases} \label{eq:omegadef}
\end{equation}
so that
\[\widetilde c_u\left(\sigma^{d_ui},\sigma^{d_uj}\right)=(\omega_u/i_\mf Px)^{\floor{\frac{i+j}{n/d_u}}}\]
in all cases.

% With that out of the way, here are our local fundamental classes.
% \begin{itemize}
% 	\item Now, for a finite place $u\coloneqq q\ne p$, we note that $q$ is unramified, so $v(q)\in V_L$ has decomposition group $G_{v(q)}$ cyclic generated by the Frobenius automorphism $\sigma^{k_q}\colon\zeta\mapsto\zeta^{q}$. As such, the local fundamental class here is represented by
% 	\[\left(\sigma^{k_qi},\sigma^{k_qj}\right)\mapsto q^{\floor{(i+j)/n}}.\]
% 	In particular, if we set $\chi_{k_q}\in Z^2(G,\ZZ)$ by $\chi_{k_q}\colon\left(\sigma^{k_qi},\sigma^{k_qj}\right)\mapsto\floor{\frac{i+j}n}$, we see that $\alpha(L_{v(u)}/K_u)=[q]\cup[\chi_{k_q}]$, where $[q]\in\widehat H^0(G,L_{v(u)}^\times)$.

% 	It will be beneficial, psychologically speaking, to change generators from $\sigma^{k_q}$ to $\sigma^{d_q}$. As such, we set $\chi_{d_q}\in Z^2(G,\ZZ)$ by $\chi_{d_q}\colon\left(\sigma^{d_qi},\sigma^{d_qj}\right)\mapsto\floor{\frac{i+j}n}$. Then \autoref{lem:cocyclegenchange} tells us that
% 	\[[\chi_{d_q}]=(k_q/d_q)[\chi_{k_q}].\]
% 	Thus, we find $y_q\in\ZZ$ with $y_q\cdot k_q/d_q\equiv1\pmod{n_q}$ so that we can represent $\alpha(L_{v(u)}/K_u)$ by
% 	\[(q\cup y_q\chi_{d_q})\colon\left(\sigma^{d_qi},\sigma^{d_qj}\right)\mapsto q^{y_q\floor{(i+j)/n}}.\]
% 	For brevity, let this $2$-cocycle be $c_q\in Z^2\big(G_{v(u)},L_{v(u)}^\times\big)$.

% 	\item For the finite place $u\coloneqq q=p$, we note that $L_{v(p)}/K_p$ is totally ramified. Using Lubin--Tate theory and the fact that the local fundamental class is uniquely determined by the local Artin reciprocity map for cyclic extensions, we can just directly compute that
% 	\[\left(\sigma^i,\sigma^j\right)\mapsto x^{-\floor{(i+j)/n}}\]
% 	represents $\alpha\left(L_{v(p)}/K_p\right)$. Let this $2$-cocycle be $c_p\in Z^2\big(G,L_{v(p)}^\times\big)$.

% 	\item Lastly, for the infinite place $u\coloneqq\infty$, set $v(\infty)$ to be a complex place $L$. Then $G_v\coloneqq\op{Gal}(L_{v(u)}/K_u)=\op{Gal}(\CC/\RR)$ is cyclic generated by $\sigma^{n/2}$ of order $2$. As such, the $2$-cocycle
% 	\[\left(\sigma^{in/2},\sigma^{jn/2}\right)\mapsto(-1)^{\floor{(i+j)/2}}\]
% 	represents $\alpha(L_{v(u)}/K_u)$. Let this $2$-cocycle be $c_\infty\in Z^2\big(G_{v(u)},L_{v(\infty)}^\times\big)$.
% \end{itemize}
% In order to talk about our $2$-cocycles in a unified way, we define
% \[\omega_u\coloneqq\begin{cases}
% 	q^{y_q} & u=q\ne p\text{ is finite}, \\
% 	x^{-1} & u=p, \\
% 	-1 & u=\infty,
% \end{cases}\qquad\text{and}\qquad d_u\coloneqq\begin{cases}
% 	d_q & u=q\ne p\text{ is finite}, \\
% 	1 & u=p, \\
% 	n/2 & u=\infty,
% \end{cases}\]
% and $n_u\coloneqq n/d_u$ for each $u\in V_K$. Thus, we see that $c_u\in Z^2\big(G_{v(u)},L_{v(u)}^\times\big)$ is defined by
% \[c_u\left(\sigma^{d_ui},\sigma^{d_uj}\right)=\omega_u^{\floor{(i+j)/n_u}}\]
% for any place $u\in V_K$. Thus, we see that $\alpha_2(u)$ is now represented by
% \[(i_{v(u)}c_u)\left(\sigma^{d_ui},\sigma^{d_uj}\right)=(i_{v(u)}\omega_u)^{\floor{(i+j)/n_u}},\]
% where $i_{v(u)}\colon L_{v(u)}\into\AA_L$ is the canonical embedding. We quickly observe that our construction of $c_u$ has the remarkable properties that $d_u\mid n$ and $\omega_u\in K$ for each place $u\in V_K$.

\subsubsection{Inverting Shapiro's Lemma}
The next step in reversing \autoref{lem:magicaltate} is to invert the Shapiro's lemma isomorphism
\[\widehat H^2\left(G_{v(u)},\AA_L^\times\right)\simeq\widehat H^2\big(G,\op{CoInd}_{G_{v(u)}}^G(\AA_L^\times)\big)\]
for each place $u\in V_K$. Until the end of this section, we will fix the place $u\in V_K$ and set $v\coloneqq v(u)\in V_L$ and $H\coloneqq G_v=G_{v(u)}$ for brevity. It is known that (e.g., see \cite{kaletha-invert-shapiro}) this inverse morphism can be constructed as the composite
\[\widehat H^2\left(H,\AA_L^\times\right)\stackrel{\iota}\to\widehat H^2\big(H,\op{CoInd}^G_H\AA_L^\times\big)\stackrel{\op{cor}}\to\widehat H^2\big(G,\op{CoInd}^G_H\AA_L^\times\big),\]
where $\iota\colon\AA_L^\times\to\op{CoInd}^G_H\AA_L^\times$ takes $a$ to $\iota(a)\colon g\mapsto\big(g1_{g\in H}\big)a$.

Thus, we have two maps to track on the level of our $2$-cocycles. For the time being, we will ignore that we have chosen a specific $2$-cocycle $c_u\in Z^2(H,\AA_L^\times)$ and track everything through abstractly. To track $\iota$, we start by computing
\[(\iota c_u)\left(h,h'\right)\colon g\mapsto\left(gc_u(h,h')\right)^{1_{g\in H}}.\]
% \[(\iota c_u)\left(\sigma^{d_ui},\sigma^{d_uj}\right)\colon\sigma^c\mapsto\left(\sigma^c i_v\omega_u\right)^{1_{d_u\mid c}\floor{(i+j)/n_u}}.\]
% Because we only care about the case where $g\in H=G_v$, we see $g\circ i_v=i_{gv}\circ g=i_v\circ g$, so we get
% \[(\iota c_u)\left(\sigma^{d_ui},\sigma^{d_uj}\right)\colon\sigma^c\mapsto i_{v}\omega_u^{1_{d_u\mid c}\floor{(i+j)/n_u}}.\]
% \[(\iota c_u)\left(h,h'\right)\colon g\mapsto\left(i_vgc(h,h')\right)^{1_{g\in H}}.\]
Next we must track through $\op{cor}$. This is more difficult; we follow \cite{neukirch-cohom}.
% We start by noting that we can write the inhomogeneous $2$-cocycle as the homogeneous $2$-cocycle
% % \[\iota i_v\widetilde c_u\left(1,\sigma^{d_ui},\sigma^{d_u(i+j)}\right)\colon\sigma^c\mapsto i_{v}\omega_u^{1_{d_u\mid c}\floor{(i+j)/n_u}}.\]
% \[\iota i_v\widetilde c_u(1,h,hh')\colon g\mapsto\left(i_v\cdot gc(h,h')\right)^{1_{g\in H}}.\]
%To set up our evaluation of $\op{cor}$
To begin, we choose representatives for cosets in $H\backslash G$, letting $\overline{Hg}$ denote the representative of $H\backslash G$; for coherence reasons, we require $\overline{He}=e$, where $e\in G$ is the identity. With this notation, we may compute
\begin{align*}
	(\op{cor}\iota c_u)\left(g_1,g_2\right) &= \sum_{Hg\in H\backslash G}(\overline{Hg})^{-1}\cdot(\iota c_u)\left(\overline{Hg}g_1\overline{Hgg_1}^{-1},\overline{Hgg_1}g_2\overline{Hgg_1g_2}^{-1}\right).
\end{align*}
% \begin{align*}
% 	(\op{cor}\iota c_u)\left(\sigma^i,\sigma^j\right) &= \sum_{Hg\in H\backslash G}(\overline{Hg})^{-1}\cdot(\iota i_{v}\widetilde c_u)\left(1,\overline{Hg}\sigma^i\overline{Hg\sigma^i}^{-1},\overline{Hg}\sigma^{i+j}\overline{Hg\sigma^{i+j}}^{-1}\right) \\
% 	&= \sum_{\ell=0}^{d_u-1}\sigma^{-\ell}\cdot(\iota i_{v}\widetilde c_u)\left(1,\sigma^{i+\ell-[i+\ell]_{d_u}},\sigma^{i+j+\ell-[i+j+\ell]_{d_u}}\right).
% \end{align*}
Now, the $G$-action on $\op{CoInd}^G_H\AA_L^\times$ takes $f\colon G\to\AA_L^\times$ to $(gf)\colon x\mapsto f(xg)$. So when we plug in $g_0\in G$, we get
\begin{align*}
	(\op{cor}\iota c_u)\left(g_1,g_2\right)(g_0) &= \prod_{Hg\in H\backslash G}(\iota c_u)\left(\overline{Hg}g_1\overline{Hgg_1}^{-1},\overline{Hgg_1}g_2\overline{Hgg_1g_2}^{-1}\right)\left(g_0\overline{Hg}^{-1}\right) \\
	&= \prod_{Hg\in H\backslash G}\left(g_0\overline{Hg}^{-1}c_u\left(\overline{Hg}g_1\overline{Hgg_1}^{-1},\overline{Hgg_1}g_2\overline{Hgg_1g_2}^{-1}\right)\right)^{1_{g_0\overline{Hg}^{-1}\in H}}.
\end{align*}
% \[(\op{cor}\iota c_u)\left(\sigma^i,\sigma^j\right)\left(\sigma^c\right)=\prod_{\ell=0}^{d_u-1}(\iota i_{v}\widetilde c_u)\left(1,\sigma^{i+\ell-[i+\ell]_{d_u}},\sigma^{i+j+\ell-[i+j+\ell]_{d_u}}\right)\left(\sigma^{c-\ell}\right).\]
The only opportunity for a factor in the product to not output $1$ is when $g_0\overline{Hg}^{-1}\in H$, which is equivalent to $Hg_0=Hg$, yielding
\[(\op{cor}\iota c_u)\left(g_1,g_2\right)(g_0)=g_0\overline{Hg_0}^{-1}c_u\left(\overline{Hg_0}g_1\overline{Hg_0g_1}^{-1},\overline{Hg_0g_1}g_2\overline{Hg_0g_1g_2}^{-1}\right).\]
This will be explicit enough for our purposes.

Continuing, we go from $Z^2\big(G,\op{CoInd}_{G_v}^G\AA_L^\times\big)$ up to $Z^2\big(G,\op{Mor}_{\mathrm{Set}}(H\backslash G,\AA_L^\times)\big)$, for which we note that $f\in\op{CoInd}_{G_v}^G\AA_L^\times$ should be sent to $Hg\mapsto gf\left(g^{-1}\right)$. (This is well-defined because $f(hg)=hf(g)$ for $h\in H$ here.) This gives the $2$-cocycle
\[(g_1,g_2)\mapsto Hg_0\mapsto \overline{Hg_0^{-1}}^{-1}c_u\left(\overline{Hg_0^{-1}}g_1\overline{Hg_0^{-1}g_1}^{-1},\overline{Hg_0^{-1}g_1}g_2\overline{Hg_0^{-1}g_1g_2}^{-1}\right).\]
The above immediately extends to a $2$-cocycle in $Z^2\big(G,\op{Hom}_\ZZ(\ZZ[G_{v}\backslash G],\AA_L^\times)\big)$, which then turns into the $2$-cocycle
% \[c_2\left(\sigma^i,\sigma^j\right)\colon\sigma^cv(u)\mapsto i_{\sigma^cv(u)}\omega_u^{\bigg\lfloor{\frac{\big[\big\lfloor\frac{i+[-c]_{d_u}}{d_u}\big\rfloor\big]_{n_u}+\big[\big\lfloor\frac{i+j+[-c]_{d_u}}{d_u}\big\rfloor-\big\lfloor\frac{i+[-c]_{d_u}}{d_u}\big\rfloor\big]_{n_u}}{n_u}}\bigg\rfloor}\]
\[(g_1,g_2)\mapsto g_0v\mapsto\overline{Hg_0^{-1}}^{-1}c_u\left(\overline{Hg_0^{-1}}g_1\overline{Hg_0^{-1}g_1}^{-1},\overline{Hg_0^{-1}g_1}g_2\overline{Hg_0^{-1}g_1g_2}^{-1}\right)\]
in $c_2\in Z^2\big(G,\op{Hom}_\ZZ(\ZZ[V_u],\AA_L^\times)\big)$.\todo{Should this by g0 inverse v?}

Only now do we let the place $u\in V_K$ vary, extending $c_2$ accordingly to
\begin{equation}
	c_2(g_1,g_2)\colon g_0v(u)\mapsto\overline{G_{v(u)}g_0^{-1}}^{-1}c_u\left(\overline{G_{v(u)}g_0^{-1}}g_1\overline{G_{v(u)}g_0^{-1}g_1}^{-1},\overline{G_{v(u)}g_0^{-1}g_1}g_2\overline{G_{v(u)}g_0^{-1}g_1g_2}^{-1}\right) \label{eq:shapiroinverted}
\end{equation}
in $c_2\in Z^2(G,\op{Hom}_\ZZ(\ZZ[V_L],\AA_L^\times))$; this is the representative of $\alpha_2$ we are looking for.
\begin{example}
	If $g_1,g_2\in H$ and $g_0=e$, then
	\[c_2(g_1,g_2)\colon v(u)\mapsto c_u\left(g_1,g_2\right),\]
	as needed; notably, we used the requirement that $\overline{He}=e$.
\end{example}

% We continue not using the specific choices of $c_u$. We are ready to finish tracking upwards through \autoref{lem:magicaltate}. Our next step is to go from $Z^2\big(G,\op{CoInd}_{G_v}^G\AA_L^\times\big)$ up to $Z^2\big(G,\op{Mor}_{\mathrm{Set}}(G_{v(u)}\backslash G,\AA_L^\times)\big)$, for which we note that $f\in\op{CoInd}_{G_v}^G\AA_L^\times$ should be sent to $G_{v(u)}g\mapsto gf\left(g^{-1}\right)$. (This is well-defined because $f(hg)=hf(g)$ for $h\in G_{v(u)}$ here.) This gives the $2$-cocycle
% \[(g_1,g_2)\mapsto G_{v(u)}g_0\mapsto \]
% \[\left(\sigma^i,\sigma^j\right)\mapsto G_{v(u)}\sigma^c\mapsto i_{\sigma^cv(u)}\omega_u^{\bigg\lfloor{\frac{\big[\big\lfloor\frac{i+[-c]_{d_u}}{d_u}\big\rfloor\big]_{n_u}+\big[\big\lfloor\frac{i+j+[-c]_{d_u}}{d_u}\big\rfloor-\big\lfloor\frac{i+[-c]_{d_u}}{d_u}\big\rfloor\big]_{n_u}}{n_u}}\bigg\rfloor},\]
% where we are now assuming $0\le c<d_u$ without loss of generality; note that we have used the fact $\sigma^c\circ i_v=i_{\sigma^cv}\circ\sigma^c$. The above immediately extends to a $2$-cocycle in $Z^2\big(G,\op{Hom}_\ZZ(\ZZ[G_{v(u)}\backslash G],\AA_L^\times)\big)$, which then turns into the $2$-cocycle
% \[c_2\left(\sigma^i,\sigma^j\right)\colon\sigma^cv(u)\mapsto i_{\sigma^cv(u)}\omega_u^{\bigg\lfloor{\frac{\big[\big\lfloor\frac{i+[-c]_{d_u}}{d_u}\big\rfloor\big]_{n_u}+\big[\big\lfloor\frac{i+j+[-c]_{d_u}}{d_u}\big\rfloor-\big\lfloor\frac{i+[-c]_{d_u}}{d_u}\big\rfloor\big]_{n_u}}{n_u}}\bigg\rfloor}\]
% in $c_2\in Z^2\big(G,\op{Hom}_\ZZ(\ZZ[V_u],\AA_L^\times)\big)$. To finish with this step, we note that letting $u$ vary in the above expression immediately pushes the $2$-cocycle to $c_2\in Z^2\big(G,\op{Hom}_\ZZ(\ZZ[V_L],\AA_L^\times)\big)$, which is exactly the representative of $\alpha_2$ we have been looking for.
% \[(\op{cor}\iota c_u)\left(\sigma^i,\sigma^j\right)\left(\sigma^c\right)=(\iota i_{v}\widetilde c_u)\left(1,\sigma^{i+[c]_{d_u}-[i+c]_{d_u}},\sigma^{i+j+[c]_{d_u}-[i+j+c]_{d_u}}\right)(\sigma^{c-[c]_{d_u}}).\]
% Now, transitioning back to an inhomogeneous $2$-cocycle, we have
% \[(\op{cor}\iota c_u)\left(\sigma^i,\sigma^j\right)\left(\sigma^c\right)=(\iota c_u)\left(\sigma^{i+[c]_{d_u}-[i+c]_{d_u}},\sigma^{j-[i+j+c]_{d_u}+[i+c]_{d_u}}\right)(\sigma^{c-[c]_{d_u}}).\]
% We can simplify this some, but not much. Observe $i+[c]_{d_u}-[i+c]_{d_u}=d_u\floor{\frac{i+[c]_{d_u}}{d_u}}$ and $i+j+[c]_{d_u}-[i+j+c]_{d_u}=d_u\floor{\frac{i+j+[c]_{d_u}}{d_u}}$, so we have
% \[(\op{cor}\iota c_u)\left(\sigma^i,\sigma^j\right)\left(\sigma^c\right)= i_{v}\omega_u^{\bigg\lfloor{\frac{\big[\big\lfloor\frac{i+[c]_{d_u}}{d_u}\big\rfloor\big]_{n_u}+\big[\big\lfloor\frac{i+j+[c]_{d_u}}{d_u}\big\rfloor-\big\lfloor\frac{i+[c]_{d_u}}{d_u}\big\rfloor\big]_{n_u}}{n_u}}\bigg\rfloor}\]
% as our $2$-cocycle in $Z^2\big(G,\op{CoInd}_H^G\AA_L^\times\big)$.

\subsubsection{Finishing Up}
We will now be more concrete to our example. Because $G$ is cyclic, and $G_{v(u)}$ is cyclic generated by $\sigma^{d_u}$, we can set
\[\overline{G_{v(u)}\sigma^i}=\sigma^i\]
for each $0\le i<d_u$.
% We are ready to finish tracking upwards through \autoref{lem:magicaltate}. Our next step is to go from $Z^2\big(G,\op{CoInd}_{G_v}^G\AA_L^\times\big)$ up to $Z^2\big(G,\op{Mor}_{\mathrm{Set}}(G_{v(u)}\backslash G,\AA_L^\times)\big)$, for which we note that $f\in\op{CoInd}_{G_v}^G\AA_L^\times$ should be sent to $G_{v(u)}g\mapsto gf\left(g^{-1}\right)$. (This is well-defined because $f(hg)=hf(g)$ for $h\in G_{v(u)}$ here.)
This gives the $2$-cocycle
% \[\left(\sigma^i,\sigma^j\right)\mapsto G_{v(u)}\sigma^c\mapsto i_{\sigma^cv(u)}\omega_u^{\bigg\lfloor{\frac{\big[\big\lfloor\frac{i+[-c]_{d_u}}{d_u}\big\rfloor\big]_{n_u}+\big[\big\lfloor\frac{i+j+[-c]_{d_u}}{d_u}\big\rfloor-\big\lfloor\frac{i+[-c]_{d_u}}{d_u}\big\rfloor\big]_{n_u}}{n_u}}\bigg\rfloor},\]
% where we are now assuming $0\le c<d_u$ without loss of generality; note that we have used the fact $\sigma^c\circ i_v=i_{\sigma^cv}\circ\sigma^c$. The above immediately extends to a $2$-cocycle in $Z^2\big(G,\op{Hom}_\ZZ(\ZZ[G_{v(u)}\backslash G],\AA_L^\times)\big)$, which then turns into the $2$-cocycle
\[c_2\left(\sigma^i,\sigma^j\right)\colon\sigma^cv(u)\mapsto \sigma^c(\omega_u/i_\mf Px)^{\bigg\lfloor{\frac{\big[\big\lfloor\frac{i+[-c]_{d_u}}{d_u}\big\rfloor\big]_{n_u}+\big[\big\lfloor\frac{i+j+[-c]_{d_u}}{d_u}\big\rfloor-\big\lfloor\frac{i+[-c]_{d_u}}{d_u}\big\rfloor\big]_{n_u}}{n_u}}\bigg\rfloor}\]
in $c_2\in Z^2\big(G,\op{Hom}_\ZZ(\ZZ[V_L],\AA_L^\times)\big)$ after tracking through \autoref{eq:shapiroinverted}.
% To finish with this step, we note that letting $u$ vary in the above expression immediately pushes the $2$-cocycle to $c_2\in Z^2\big(G,\op{Hom}_\ZZ(\ZZ[V_L],\AA_L^\times)\big)$, which is exactly the representative of $\alpha_2$ we have been looking for.

As a last addendum, we go ahead and compute the $\alpha$ associated to $c_2$. Namely, we want to compute
\begin{align*}
	\alpha\left(\sigma^cv(u)\right) &= \prod_{i=0}^{n-1}c\left(\sigma^i,\sigma\right)\left(\sigma^cv(u)\right) \\
	&= \sigma^c(\omega_u/i_\mf Px)^{\displaystyle\sum_{i=0}^{n-1}\bigg\lfloor{\frac{\left[\big\lfloor\frac{i+[-c]_{d_u}}{d_u}\big\rfloor\right]_{n_u}+\left[\big\lfloor\frac{i+1+[-c]_{d_u}}{d_u}\big\rfloor-\big\lfloor\frac{i+[-c]_{d_u}}{d_u}\big\rfloor\right]_{n_u}}{n_u}}\bigg\rfloor}.
\end{align*}
It turns out that the giant sum is just $1$, which we outsource to the following lemma.
\begin{lemma}
	Let $n,d>0$ be positive integers. Then, for any $c\in[0,d)$, we have
	\[\sum_{i=0}^{nd-1}\bigg\lfloor{\frac{\left[\big\lfloor\frac{i+c}{d}\big\rfloor\right]_{n}+\left[\big\lfloor\frac{i+1+c}{d}\big\rfloor-\big\lfloor\frac{i+c}{d}\big\rfloor\right]_{n}}{n}}\bigg\rfloor=1.\]
\end{lemma}
\begin{proof}
	Note that each term in the sum is either $0$ or $1$ because the terms take the form $\floor{\frac{a+b}n}$ where $0\le a,b<n$. As such, we are counting the number of nonzero terms in the sum.

	Well, we claim that the term is nonzero if and only if $i=nd-c-1$. Note that $n,d>0$ and $c<d$ implies that $nd-c-1$ is a valid input in $[0,nd-1)$. Anyway, we start by showing that, if the term
	\[\bigg\lfloor{\frac{\left[\big\lfloor\frac{i+c}{d}\big\rfloor\right]_{n}+\left[\big\lfloor\frac{i+1+c}{d}\big\rfloor-\big\lfloor\frac{i+c}{d}\big\rfloor\right]_{n}}{n}}\bigg\rfloor\]
	is nonzero, then $i=nd-c-1$. Note that $\big\lfloor\frac{i+1+c}{d}\big\rfloor-\big\lfloor\frac{i+c}{d}\big\rfloor$ must be positive for this to be possible, or else the entire numerator is less than $n$. However, for this to be positive, we need $i+1+c$ to be a multiple of $d$, which means
	\[i\equiv-c-1\pmod d.\]
	Even still, we don't get much from this, only that $\big\lfloor\frac{i+1+c}{d}\big\rfloor-\big\lfloor\frac{i+c}{d}\big\rfloor=1$. As such, we're going to need
	\[\left[\left\lfloor\frac{i+c}{d}\right\rfloor\right]_n=n-1\]
	for our term to be nonzero. Of course, $i<nd$ and $c<d$, so $\frac{i+c}d<n$, so we don't even have to worry about modding out by $n$ here. As such, we really just need $\frac{i+c}d\ge n-1$, which translates into
	\[i\ge nd-c-d.\]
	Combining this with the fact that $i<nd$ and $i\equiv-c-1\pmod d$, we see that we are forced to have $i=nd-c-1$.

	We finish by remarking that $i=nd-c-1$ will give
	\[\bigg\lfloor{\frac{\left[\big\lfloor\frac{i+c}{d}\big\rfloor\right]_{n}+\left[\big\lfloor\frac{i+1+c}{d}\big\rfloor-\big\lfloor\frac{i+c}{d}\big\rfloor\right]_{n}}{n}}\bigg\rfloor=\floor{\frac{n-1+1}n}=1\]
	as discussed above. This completes the proof.
\end{proof}
In total, our value of $\alpha$ comes out to be
\[\alpha^{(2)}\colon\sigma^cv(u) \mapsto\sigma^c\omega_u/i_\mf Px.\]
For brevity, we set $\omega_{\omega^cv(u)}\coloneqq\sigma^c\omega_u$. By construction, $\omega_u\in L^{G_v}$, so $\omega_v$ does not depend on the exact choice of $\sigma^c$ among coset representatives in $G/G_v$. So we can write more succinctly that
\[\boxed{\alpha^{(2)}\colon v\mapsto\omega_v/i_\mf Px}.\]
This completes the computation.

\subsection{Localizing}
Note that there is a (unique) map $\lambda_v\colon\ZZ\to\ZZ[V_L]$ by $1\mapsto v$, which induces a map of protori $\lambda_v\colon\mathbb D\to\mathbb G_m$. With respect to $\alpha_2$, we are interested in this map as moving
\[(-\circ\lambda_v)\colon\op{Hom}_\ZZ(\ZZ[V_L],\AA_L^\times)\to\AA_L^\times,\]
which we can track as the evaluation-at-$v$ map $\op{eval}_v$. In particular, we defined $\alpha_2$ by \autoref{lem:magicaltate} to be the unique cohomology class in $\widehat H^2(G,\mathbb D(\AA_L))$ such that
\[\op{eval}_{v(u)}\op{Res}_{G_{v(u)}}\alpha_2=\alpha(L_v/K_u)\]
for each place $u\in V_K$ (see \autoref{rem:forwardshapiro}), which we now see is equivalent to
\[\lambda_{v(u)}\op{Res}_{G_{v(u)}}\alpha_2=\alpha(L_v/K_u).\]
On the level of gerbs, we are asking for $\alpha_2$ to be the unique cohomology class making the following diagram commute for all $u\in V_K$; here $v\coloneqq v(u)$.
% https://q.uiver.app/?q=WzAsMjAsWzAsMCwiMSJdLFsxLDAsIlxcbWF0aGJiIEQoXFxtYXRoYmIgQV9MKSJdLFsyLDAsIlxcbWMgRV8yKEwvSykiXSxbMywwLCJcXG9we0dhbH0oTC9LKSJdLFs0LDAsIjEiXSxbMCwxLCIxIl0sWzEsMSwiXFxtYXRoYmIgRChcXG1hdGhiYiBBX0wpIl0sWzIsMSwiXFxtYyBFXzInJyhML0spIl0sWzMsMSwiXFxvcHtHYWx9KExfdi9LX3UpIl0sWzQsMSwiMSJdLFswLDIsIjEiXSxbMSwyLCJcXG1hdGhiYiBHX20oXFxtYXRoYmIgQV9MKSJdLFsyLDIsIlxcbWMgRSdfMihML0spIl0sWzMsMiwiXFxvcHtHYWx9KExfdi9LX3UpIl0sWzQsMiwiMSJdLFswLDMsIjEiXSxbMSwzLCJcXG1hdGhiYiBHX20oTF92KSJdLFsyLDMsIlxcbWMgRShMX3YvS191KSJdLFszLDMsIlxcb3B7R2FsfShMX3YvS191KSJdLFs0LDMsIjEiXSxbMCwxXSxbMSwyXSxbMiwzXSxbMyw0XSxbNSw2XSxbNiw3XSxbNyw4XSxbOCw5XSxbMTAsMTFdLFsxMSwxMl0sWzEyLDEzXSxbMTMsMTRdLFsxNSwxNl0sWzE2LDE3XSxbMTcsMThdLFsxOCwxOV0sWzEsNiwiIiwxLHsibGV2ZWwiOjIsInN0eWxlIjp7ImhlYWQiOnsibmFtZSI6Im5vbmUifX19XSxbOCwxMywiIiwxLHsibGV2ZWwiOjIsInN0eWxlIjp7ImhlYWQiOnsibmFtZSI6Im5vbmUifX19XSxbMTMsMTgsIiIsMSx7ImxldmVsIjoyLCJzdHlsZSI6eyJoZWFkIjp7Im5hbWUiOiJub25lIn19fV0sWzgsMywiIiwxLHsic3R5bGUiOnsidGFpbCI6eyJuYW1lIjoiaG9vayIsInNpZGUiOiJ0b3AifX19XSxbNywyLCIiLDEseyJzdHlsZSI6eyJ0YWlsIjp7Im5hbWUiOiJob29rIiwic2lkZSI6InRvcCJ9fX1dLFs3LDEyLCJcXHdpZGV0aWxkZSBcXGxhbWJkYV92IiwyXSxbMTcsMTIsIlxcd2lkZXRpbGRlIGlfdiJdLFsxNiwxMSwiaV92Il0sWzYsMTEsIlxcbGFtYmRhX3YiLDJdXQ==&macro_url=https%3A%2F%2Fraw.githubusercontent.com%2FdFoiler%2Fnotes%2Fmaster%2Fnir.tex
\[\begin{tikzcd}
	1 & {\mathbb D(\mathbb A_L)} & {\mc E_2(L/K)} & {\op{Gal}(L/K)} & 1 \\
	1 & {\mathbb D(\mathbb A_L)} & {\mc E_2''(L/K)} & {\op{Gal}(L_v/K_u)} & 1 \\
	1 & {\mathbb G_m(\mathbb A_L)} & {\mc E'_2(L/K)} & {\op{Gal}(L_v/K_u)} & 1 \\
	1 & {\mathbb G_m(L_v)} & {\mc E(L_v/K_u)} & {\op{Gal}(L_v/K_u)} & 1
	\arrow[from=1-1, to=1-2]
	\arrow[from=1-2, to=1-3]
	\arrow[from=1-3, to=1-4]
	\arrow[from=1-4, to=1-5]
	\arrow[from=2-1, to=2-2]
	\arrow[from=2-2, to=2-3]
	\arrow[from=2-3, to=2-4]
	\arrow[from=2-4, to=2-5]
	\arrow[from=3-1, to=3-2]
	\arrow[from=3-2, to=3-3]
	\arrow[from=3-3, to=3-4]
	\arrow[from=3-4, to=3-5]
	\arrow[from=4-1, to=4-2]
	\arrow[from=4-2, to=4-3]
	\arrow[from=4-3, to=4-4]
	\arrow[from=4-4, to=4-5]
	\arrow[Rightarrow, no head, from=1-2, to=2-2]
	\arrow[Rightarrow, no head, from=2-4, to=3-4]
	\arrow[Rightarrow, no head, from=3-4, to=4-4]
	\arrow[hook, from=2-4, to=1-4]
	\arrow[hook, from=2-3, to=1-3]
	\arrow["{\widetilde \lambda_v}"', from=2-3, to=3-3]
	\arrow["{\widetilde i_v}", from=4-3, to=3-3]
	\arrow["{i_v}", from=4-2, to=3-2]
	\arrow["{\lambda_v}"', from=2-2, to=3-2]
\end{tikzcd}\]
Here, the morphisms $\widetilde\lambda_v$ and $\widetilde i_v$ are induced by the rest of the diagram.

\subsubsection{Choosing Lifts}
We now work in a little more generality, taking $L/K$ to be the extension $\QQ(\zeta_N)/\QQ$, where $N$ is odd.
\begin{remark}
	We will take $N$ to be odd entirely for psychological reasons. The arguments below in fact extend to allow $N$ to satisfy any of the following conditions:
	\begin{itemize}
		\item $N$ is not divisible by $8$,
		\item $N$ is not divisible by $3$, or
		\item $N$ divisible by $9$.
	\end{itemize}
\end{remark}
Taking a prime factorization of $N$, we write
\[N=p_1^{a_1}\cdot\ldots\cdot p_m^{a_m}\]
and so choose generators $x_i\in\left(\ZZ/p_i^{a_i}\ZZ\right)^\times$ so that
\[\sigma_i\colon\zeta_{p_i^{a_i}}\mapsto\zeta_{p_i^{a_i}}^{x_i}\]
extends to an automorphism $\sigma_i\in\op{Gal}(L/K)$ (namely, acting as the identity on the other $\zeta_{p^a}$s) so that
\[\op{Gal}(L/K)\simeq\bigoplus_{i=1}^m\langle\sigma_i\rangle.\]
Now, when we localize to some place $v\in V_L$ lying over a finite place $q=u\in V_K$, the unramified part of the decomposition group $G_v$ will be generated by the Frobenius automorphism
\[\sigma_q\colon\zeta\mapsto\zeta^q,\]
where $\zeta=\zeta_{N/q^a}$ with $\gcd(N/q^a,q)=1$.\todo{}

Our goal for this subsection is to choose lifts $f_i\in\mc E_2(L/K)$ so that the $\widetilde\lambda_vf_i$ commute as much as possible in $\mc E_2'(L/K)$. In particular, when $v\coloneqq v(u)\in V_L$ lies over $u\in V_K$, we claim that we can arrange things so that
\[(\widetilde\lambda_vf_i)(\widetilde\lambda_vf_j)=(\widetilde\lambda_vf_j)(\widetilde\lambda_vf_i)\]
as long as neither $p_i$ nor $p_j$ are primes corresponding to the place $u$. To begin, we note that
\[\widehat H^2\left(G,\op{Hom}_\ZZ(\ZZ[V_L],\AA_L^\times)\right)\simeq\prod_{u\in V_K}\widehat H^2\left(G,\op{Hom}_\ZZ(\ZZ[V_u],\AA_L^\times)\right)\]
is an isomorphism at the level of $2$-cocycles simply by gluing all the local $\alpha_2$s together. Namely, we may choose whatever $2$-cocycles we want from $Z^2\left(G,\op{Hom}_\ZZ(\ZZ[V_u],\AA_L^\times)\right)$ (as long as they cohere correctly via Shapiro's lemma according to \autoref{rem:forwardshapiro}), and we know that they will combine into a coherent $2$-cocycle for $\alpha_2$.

This is all to say that we may set all the $\widetilde\lambda_vf_i\in\AA_L^\times$ independently and not worry about coherence issues. As such, we now fix $u\in V_K$ and $v\coloneqq v(u)\in V_L$. So, for the time being, we set $c_u$ to represent $\alpha(L_v/K_u)$ by some triple and extend $c_u$ up to
\[c_{2u}\coloneqq\op{cor}\iota i_vc_u\in Z^2\left(G,\op{Hom}_\ZZ(\ZZ[V_u],\AA_L^\times)\right)\]
as in \autoref{eq:shapiroinverted}. We will simply set
\[\widetilde\lambda_vf_i\coloneqq(1,\sigma_i)\]
and see how far it gets us. In particular, we can compute
\[(\widetilde\lambda_vf_i)(\widetilde\lambda_vf_j)(\widetilde\lambda_vf_i)^{-1}(\widetilde\lambda_vf_j)^{-1}=\frac{c_{2u}(\sigma_i,\sigma_j)(v)}{c_{2u}(\sigma_j,\sigma_i)(v)},\]
so we want to force $c_{2u}(\sigma_i,\sigma_j)=c_{2u}(\sigma_j,\sigma_i)$ as much as possible. Thus, we expand
\[c_{2u}(\sigma_i,\sigma_j)\colon v\mapsto i_vc_u\left(g_1\overline{G_vg_1}^{-1},\overline{G_vg_1}g_2\overline{G_vg_1g_2}^{-1}\right).\]
Now, by definition of $c_u$, we note that
\[c_u(1,g)=c_u(g,1)=1\]
for each $g\in G_v$, so we have at least have a chance of forcing things to work out.

Let $S$ be the image of $G_v\backslash G\to G$ given by $G_vg\mapsto\overline{G_vg}$, which essentially makes our degrees of freedom in defining $c_{2u}$. It will not matter very much if $v$ is ramified or unramified, so we will just assume (roughly without loss of generality) that $u=p_m$ so that we are interested in showing the $\widetilde\lambda_vf_i$ for $i<m$; in the unramified cases, we should just skip this step of the construction and replace $m$ with $m-1$ going forward.

Now, to begin, we claim that we can pack $S$ to contain all but at most one of the $\sigma_i$. \todo{Finish this.}

\subsection{Computing \texorpdfstring{$\mathcal E_3$}{E3}}
In this section we continue the computation with $L\coloneqq\QQ(\zeta_{p^m})$ and $K\coloneqq\QQ$ from \autoref{subsec:computee2}. Namely, at the end we computed that
\[\widetilde c_2\left(\sigma^i,\sigma^j\right)\colon v\mapsto\left(\omega_v/i_\mf Px\right)^{\floor{\frac{i+j}n}}\]
represents $\alpha_2\in\widehat H^2(G,\AA_L^\times)$. We now recall that
\[c_1(\sigma^i,\sigma^j)\coloneqq i_\mf Px^{-\floor{\frac{i+j}n}}\]
represents the global fundamental class $\alpha_1\in\widehat H^2(G,\AA_L^\times/L^\times)$. However, our careful choice of $c_2$ and $c_1$ implies that the following diagram commutes for all $g,g'\in G$.
% https://q.uiver.app/?q=WzAsNCxbMCwwLCJcXFpaW1ZfTF0iXSxbMSwwLCJcXFpaIl0sWzEsMSwiXFxtYXRoYmIgQV9MXlxcdGltZXMvTF5cXHRpbWVzIl0sWzAsMSwiXFxtYXRoYmIgQV9MXlxcdGltZXMiXSxbMCwzLCJjXzIoZyxnJykiLDJdLFswLDFdLFsxLDIsImNfMShnLGcnKSJdLFszLDJdXQ==&macro_url=https%3A%2F%2Fraw.githubusercontent.com%2FdFoiler%2Fnotes%2Fmaster%2Fnir.tex
\[\begin{tikzcd}
	{\ZZ[V_L]} & \ZZ \\
	{\mathbb A_L^\times} & {\mathbb A_L^\times/L^\times}
	\arrow["{c_2(g,g')}"', from=1-1, to=2-1]
	\arrow[from=1-1, to=1-2]
	\arrow["{c_1(g,g')}", from=1-2, to=2-2]
	\arrow[from=2-1, to=2-2]
\end{tikzcd}\]
These two morphisms induce a unique morphism $c_1(g,g')\colon\ZZ[V_L]_0\to L^\times$ as follows.
% https://q.uiver.app/?q=WzAsMTAsWzIsMCwiXFxaWltWX0xdIl0sWzMsMCwiXFxaWiJdLFszLDEsIlxcbWF0aGJiIEFfTF5cXHRpbWVzL0xeXFx0aW1lcyJdLFsyLDEsIlxcbWF0aGJiIEFfTF5cXHRpbWVzIl0sWzEsMCwiXFxaWltWX0xdXzAiXSxbMSwxLCJMXlxcdGltZXMiXSxbNCwwLCIwIl0sWzQsMSwiMCJdLFswLDAsIjAiXSxbMCwxLCIwIl0sWzAsMywiY18yKGcsZycpIl0sWzAsMV0sWzEsMiwiY18xKGcsZycpIl0sWzMsMl0sWzQsMF0sWzQsNSwiY18zKGcsZycpIiwwLHsic3R5bGUiOnsiYm9keSI6eyJuYW1lIjoiZGFzaGVkIn19fV0sWzUsM10sWzgsNF0sWzksNV0sWzEsNl0sWzIsN11d&macro_url=https%3A%2F%2Fraw.githubusercontent.com%2FdFoiler%2Fnotes%2Fmaster%2Fnir.tex
\[\begin{tikzcd}
	0 & {\ZZ[V_L]_0} & {\ZZ[V_L]} & \ZZ & 0 \\
	0 & {L^\times} & {\mathbb A_L^\times} & {\mathbb A_L^\times/L^\times} & 0
	\arrow["{c_2(g,g')}", from=1-3, to=2-3]
	\arrow[from=1-3, to=1-4]
	\arrow["{c_1(g,g')}", from=1-4, to=2-4]
	\arrow[from=2-3, to=2-4]
	\arrow[from=1-2, to=1-3]
	\arrow["{c_3(g,g')}", dashed, from=1-2, to=2-2]
	\arrow[from=2-2, to=2-3]
	\arrow[from=1-1, to=1-2]
	\arrow[from=2-1, to=2-2]
	\arrow[from=1-4, to=1-5]
	\arrow[from=2-4, to=2-5]
\end{tikzcd}\]
In fact, because we have
\[\frac{gc_i(g',g'')\cdot c_i(g,g'g'')}{c_i(g,g')\cdot c_i(gg',g'')}=1\]
for all $g,g',g''\in G$ and $i\in\{1,2\}$, the uniqueness of the induced arrow $c_3$ implies that the same relation must hold for $i=3$ above. In particular, $c_3$ is a $2$-cocycle, and by construction $c_3$ represents $\alpha_3$.

We can even write down $c_3$ explicitly. Indeed, given $v-v'\in\ZZ[V_L]_0$, we have
\[c_2(\sigma^i,\sigma^j)(v-v')=(\omega_v/\omega_{v'})^{\floor{\frac{i+j}n}}\in L^\times,\]
so we have
\[c_3(\sigma^i,\sigma^j)(v-v')=(\omega_v/\omega_{v'})^{\floor{\frac{i+j}n}}.\]
In particular, our value of $\alpha$ comes out to be
\[\boxed{\alpha^{(3)}\colon(v-v')\mapsto\omega_v/\omega_{v'}}.\]
We quickly recall that $\omega_{\sigma^cv(u)}\coloneqq\sigma^c\omega_u$, where $\omega_u$ was defined in \autoref{eq:omegadef}.

% \subsection{Finding the Correct Local Cocycles}
% In this subsection we will explain the mysterious choice of representatives for the local fundamental classes above. As before, set $L\coloneqq\QQ(\zeta_p)$ and $K\coloneqq\QQ$ so that $L/K$ is a cyclic extension with Galois group $G\coloneqq\op{Gal}(L/K)$. Let $\mf P$ denote the prime of $L$ above $(p)$ in $K$. Then $L_\mf P/K_{(p)}$ is cyclic and totally ramified, so we can solve for this local fundamental class in $H^2\left(G,L_\mf p^\times\right)$ as having $\alpha$ value $x^{-1}.$ In particular, we chose
% \[\alpha_\mf P\coloneqq\left[i_\mf Px^{-1}\right]\in\widehat H^0\left(G,\AA_L^\times\right)\]
% to be the corresponding representative for our $\alpha_2$ at this place $\mf P$.

% We next turn to the finite unramified primes. Let $q\ne p$ be some finite unramified prime of $K$. The key to this is the following lemma.
% \begin{lemma}
% 	Let $G'\subseteq G$ be a subgroup. Then for any ideal class $c\in\op{Cl}(L^{G'})$, there exists a prime $\mf r\subseteq\mathcal O_{L^{G'}}$ satisfying the following two conditions.
% 	\begin{itemize}
% 		\item $\mf r$ represents $c$.
% 		\item The prime $(r)\coloneqq\mf r\cap K$ is inert in $L$.
% 	\end{itemize}
% \end{lemma}
% \begin{proof}
% 	The key to the proof is to encode the conditions into conditions on the Frobenius automorphism and then apply the Chebotarev density theorem. Let $H$ denote the Hilbert class field of $L^{G'}$, giving the following diagram of fields.
% 	% https://q.uiver.app/?q=WzAsNSxbMSwzLCJLIl0sWzEsMiwiTF57Ryd9Il0sWzIsMSwiTCJdLFswLDEsIkgiXSxbMSwwLCJITCJdLFs0LDMsIiIsMCx7InN0eWxlIjp7ImhlYWQiOnsibmFtZSI6Im5vbmUifX19XSxbMywxLCIiLDAseyJzdHlsZSI6eyJoZWFkIjp7Im5hbWUiOiJub25lIn19fV0sWzEsMCwiIiwwLHsic3R5bGUiOnsiaGVhZCI6eyJuYW1lIjoibm9uZSJ9fX1dLFs0LDIsIiIsMix7InN0eWxlIjp7ImhlYWQiOnsibmFtZSI6Im5vbmUifX19XSxbMiwxLCIiLDIseyJzdHlsZSI6eyJoZWFkIjp7Im5hbWUiOiJub25lIn19fV1d&macro_url=https%3A%2F%2Fraw.githubusercontent.com%2FdFoiler%2Fnotes%2Fmaster%2Fnir.tex
% 	\[\begin{tikzcd}
% 		& HL \\
% 		H && L \\
% 		& {L^{G'}} \\
% 		& K
% 		\arrow[no head, from=1-2, to=2-1]
% 		\arrow[no head, from=2-1, to=3-2]
% 		\arrow[no head, from=3-2, to=4-2]
% 		\arrow[no head, from=1-2, to=2-3]
% 		\arrow[no head, from=2-3, to=3-2]
% 	\end{tikzcd}\]
% 	It is a fact that $H$ is Galois over $K$, so we may consider $\op{Gal}(HL/K)$. Very quickly, we claim that $H\cap L=L^{G'}$. Indeed, $H\cap L$ is a sub-extension of $L^{G'}\subseteq H$ and hence unramified, but all sub-extensions of $L^{G'}\subseteq L$ are either trivial or ramified at $\mf P\cap L^{G'}$. So the claim follows.

% 	On one hand, we let $g_c\in\op{Gal}(H/L^{G'})$ denote the element corresponding to the class $c\in\op{Cl}(L^{G'})\simeq\op{Gal}(H/L^{G'})$. Now, the short exact sequence
% 	\[1\to\op{Gal}(HL/L)\to\op{Gal}(HL/K)\to\op{Gal}(L/K)\to1\]
% 	implies that we may find $g\in\op{Gal}(HL/K)$ such that $g|_L=\sigma$ and 

% 	Now, we may extend $\sigma|_{L^{G'}}\in\op{Gal}(L^{G'}/K)$ to an automorphism of $\op{Gal}(H/K)$ because $K\subseteq H$ is a Galois extension. By abusing our notation, we let this automorphism be $\sigma|_H$; now,
% 	\[\sigma|_H|_{L_{G'}}=\sigma|_{L^G'}\]
% 	by construction.
% \end{proof}

\printbibliography[title={References}]

\newpage
\appendix
\section{Verification of the Cocycle} \label{sec:verifycocycle}
% !TEX root = ../abeliangerbs.tex

In this section, we verify \autoref{thm:getcocycle}. As such, in this section, we will work under the modified set-up, forgetting about the extension $\mc E$ but letting $(\{\alpha_i\},\{\beta_{ij}\})$ be some $\{\sigma_i\}_{i=1}^m$-tuple.

Here the formula looks like
\[c(g,g')\coloneqq\left[\prod_{1\le j<i\le m}\Bigg(\prod_{1\le k<j}\sigma_k^{a_k+b_k}\Bigg)\Bigg(\prod_{j\le k<i}\sigma_k^{a_k}\Bigg)\beta_{ij}^{(a_ib_j)}\right]\left[\prod_{i=1}^m\Bigg(\prod_{1\le k<i}\sigma_k^{a_k+b_k}\Bigg)\alpha_i^{\floor{\frac{a_i+b_i}{n_i}}}\right],\]
where $g=\prod_i\sigma_i^{a_i}$ and $g'=\prod_i\sigma_i^{b_i}$ with $0\le a_i,b_i<n_i$ and $q_i\coloneqq\floor{(a_i+b_i)/n_i}$. To make this more digestible, we define
\[g_i\coloneqq\prod_{1\le k<i}\sigma_k^{a_k}\]
for any $g=\prod_i\sigma_i^{a_i}\in G$, so we can write down our formula as
\[c(g,g')\coloneqq\left[\prod_{1\le j<i\le m}g_ig'_j\beta_{ij}^{(a_ib_j)}\right]\left[\prod_{i=1}^mg_ig'_i\alpha_i^{\floor{\frac{a_i+b_i}{n_i}}}\right].\]
Now, given $g,g',g''\in G$, we would like to check
\[gc(g',g'')\cdot c(g,g'g'')\stackrel?=c(gg',g'')\cdot c(g,g'),\]
where $g=\prod_i\sigma_i^{a_i}$ and $g'=\prod_i\sigma_i^{b_i}$ and $g''=\prod_i\sigma_i^{c_i}$ with $0\le a_i,b_i,c_i<n_i$.

\subsection{Carries}
We will begin our verification by dealing with carries; we start with the following lemma, intended to beef up our relation \autoref{eq:tuplerelations}.
\begin{lemma}
	Given indices $i>j$ with $a_i,a_j,q_i,q_j\ge0$, we have
	\[\beta_{ij}^{(a_ia_j)}=\beta_{ij}^{(a_i+q_in_i,a_j)}\left(\frac{\sigma_j^{a_j}(\alpha_i)}{\alpha_i}\right)^{q_i}\qquad\text{and}\qquad\beta_{ij}^{(a_ia_j)}=\beta_{ij}^{(a_i,a_j+q_jn_j)}\left(\frac{\alpha_j}{\sigma_i^{a_i}(\alpha_j)}\right)^{q_j}.\]
\end{lemma}
\begin{proof}
	This is a matter of force. For one, we compute
	\begin{align*}
		\beta_{ij}^{(a_i+n_iq_i,a_j)} &= \prod_{p=0}^{a_i+n_iq_i-1}\prod_{q=0}^{a_j-1}\sigma_i^p\sigma_j^q\beta_{ij} \\
		&= \left(\prod_{p=0}^{a_i-1}\prod_{q=0}^{a_j-1}\sigma_i^p\sigma_j^q\beta_{ij}\right)\left(\prod_{q=0}^{a_j-1}\prod_{p=a_i}^{a_i+n_iq_i-1}\sigma_i^p\sigma_j^q\beta_{ij}\right) \\
		&= \beta_{ij}^{(a_ia_j)}\left(\prod_{q=0}^{a_j-1}\sigma_j^q\op N_{L/L_i}(\beta_{ij})\right)^{q_i}.
	\end{align*}
	Now, using the relation $\op N_{L/L_i}(\beta_{ij})=\alpha_i/\sigma_j(\alpha_i)$ from \autoref{eq:tuplerelations}, this becomes
	\begin{align*}
		\beta_{ij}^{(a_i+n_iq_i,a_j)} &= \beta_{ij}^{(a_ia_j)}\left(\prod_{q=0}^{a_j-1}\frac{\sigma_j^q\alpha_i}{\sigma^{j+1}\alpha_i}\right)^{q_i} \\
		&= \beta_{ij}^{(a_ia_j)}\left(\frac{\alpha_i}{\sigma^{a_j}\alpha_i}\right)^{q_i},
	\end{align*}
	which rearranges into what we wanted.

	For the other, we again just compute
	\begin{align*}
		\beta_{ij}^{(a_i,a_j+n_jq_j)} &= \prod_{p=0}^{a_i-1}\prod_{q=0}^{a_j+n_jq_j-1}\sigma_i^p\sigma_j^q\beta_{ij} \\
		&= \left(\prod_{p=0}^{a_i-1}\prod_{q=0}^{a_j-1}\sigma_i^p\sigma_j^q\beta_{ij}\right)\left(\prod_{p=0}^{a_i-1}\prod_{q=q_j}^{a_j+n_jq_j-1}\sigma_i^p\sigma_j^q\beta_{ij}\right) \\
		&= \beta_{ij}^{(a_ia_j)}\left(\prod_{p=0}^{a_i-1}\sigma_i^p\op N_{L/L_q}(\beta_{ij})\right)^{q_i}.
	\end{align*}
	This time, we use the relation $\op N_{L/L_j}(\beta_{ij})=\sigma_i(\alpha_j)/\alpha_j$, which gives
	\begin{align*}
		\beta_{ij}^{(a_i,a_j+n_jq_j)} &= \beta_{ij}^{(a_ia_j)}\left(\prod_{p=0}^{a_i-1}\frac{\sigma_i^{p+1}(\alpha_j)}{\sigma_i^p(\alpha_j)}\right)^{q_i} \\
		&= \beta_{ij}^{(a_ia_j)}\left(\frac{\sigma_i^{a_j}(\alpha_j)}{\alpha_j}\right)^{q_i},
	\end{align*}
	which again rearranges into the desired.
\end{proof}
We are now ready to begin the computation, dealing with carries to start. Use the division algorithm to write
\[a_i+b_i=n_iu_i+x_i\qquad\text{and}\qquad b_i+c_i=n_iv_i+y_i,\]
where $u_i,v_i\in\{0,1\}$ and $0\le x_i,y_i<n_i$ for each $i$. We start by collecting remainder terms on the side of $gc(g',g'')\cdot c(g,g'g'')$.
\begin{enumerate}
	\item Note
	\begin{align*}
		gc(g',g'') &= g\left[\prod_{1\le j<i\le m}g_i'g''_j\beta_{ij}^{(b_ic_j)}\right]\cdot g\left[\prod_{i=1}^mg'_ig''_i\alpha_i^{v_i}\right],
	\end{align*}
	so we set
	\[R_1\coloneqq\prod_{i=1}^mgg'_ig''_i\alpha_i^{v_i}\]
	to be our remainder term.
	\item Note
	\begin{align*}
		c(g,g'g'') &= \left[\prod_{1\le j<i\le m}g_ig'_jg''_j\beta_{ij}^{(a_iy_j)}\right]\left[\prod_{i=1}^mg_ig'_ig''_i\alpha_i^{\floor{\frac{a_i+y_i}{n_i}}}\right] \\
		&= \left[\prod_{1\le j<i\le m}g_ig'_jg''_j\beta_{ij}^{(a_i,b_j+c_j)}\cdot g_ig'_jg''_j\left(\frac{\alpha_j}{\sigma_i^{a_i}\alpha_j}\right)^{v_i}\right]\left[\prod_{i=1}^mg_ig'_ig''_i\alpha_i^{\floor{\frac{a_i+y_i}{n_i}}}\right] \\
		&= \left[\prod_{1\le j<i\le m}g_ig'_jg''_j\beta_{ij}^{(a_i,b_j+c_j)}\right]\left[\prod_{1\le j<i\le m}g_ig'_jg''_j\left(\frac{\alpha_j}{\sigma_i^{a_i}\alpha_j}\right)^{v_i}\right]\left[\prod_{i=1}^mg_ig'_ig''_i\alpha_i^{\floor{\frac{a_i+y_i}{n_i}}}\right],
	\end{align*}
	so we set
	\[R_2\coloneqq\left[\prod_{1\le j<i\le m}g_ig'_jg''_j\left(\frac{\alpha_j}{\sigma_i^{a_i}\alpha_j}\right)^{v_i}\right]\left[\prod_{i=1}^mg_ig'_ig''_i\alpha_i^{\floor{\frac{a_i+y_i}{n_i}}}\right]\]
	to be our remainder term.
	\item Lastly, we collect our remainders. Observe
	\begin{align*}
		R_2 &= \left[\prod_{j=1}^mg'_jg''_j\Bigg(\prod_{i=j+1}^mg_i\cdot\frac{\alpha_j}{\sigma_i^{a_i}\alpha_j}\Bigg)^{v_i}\right]\left[\prod_{i=1}^mg_ig'_ig''_i\alpha_i^{\floor{\frac{a_i+y_i}{n_i}}}\right] \\
		&= \left[\prod_{j=1}^mg'_jg''_j\Bigg(\prod_{i=j+1}^m\frac{(\sigma_1^{a_1}\cdots\sigma_{i-1}^{a_{i-1}})\alpha_j}{(\sigma_1^{a_1}\cdots\sigma_{i-1}^{a_{i-1}})\sigma_i^{a_i}\alpha_j}\Bigg)^{v_i}\right]\left[\prod_{i=1}^mg_ig'_ig''_i\alpha_i^{\floor{\frac{a_i+y_i}{n_i}}}\right] \\
		&= \left[\prod_{j=1}^mg'_jg''_j\Bigg(\prod_{i=j+1}^m\frac{g_i\alpha_j}{g_{i+1}\alpha_j}\Bigg)^{v_i}\right]\left[\prod_{i=1}^mg_ig'_ig''_i\alpha_i^{\floor{\frac{a_i+y_i}{n_i}}}\right] \\
		&= \left[\prod_{j=1}^mg'_jg''_j\cdot\frac{g_{j+1}\alpha_j^{v_j}}{g\alpha_j^{v_j}}\right]\left[\prod_{i=1}^mg_ig'_ig''_i\alpha_i^{\floor{\frac{a_i+y_i}{n_i}}}\right].
	\end{align*}
	We now note that $g_{j+1}\alpha_j=g_j\alpha_j$ because $\alpha_j$ is fixed by $\sigma_j$. As such,
	\begin{align*}
		R_1R_2 &= \left[\prod_{i=1}^mgg'_ig''_i\alpha_i^{v_i}\right]\left[\prod_{i=1}^mg'_ig''_i\cdot\frac{g_i\alpha_i^{v_i}}{g\alpha_i^{v_i}}\right]\left[\prod_{i=1}^mg_ig'_ig''_i\alpha_i^{\floor{\frac{a_i+y_i}{n_i}}}\right] \\
		&= \prod_{i=1}^mg_ig'_ig''_i\alpha_i^{v_i+\floor{\frac{a_i+y_i}{n_i}}},
	\end{align*}
	which is nice enough for us now.
\end{enumerate}
Now, we collect remainder terms from $c(gg',g'')\cdot c(g,g')$.
\begin{enumerate}
	\item Note
	\begin{align*}
		c(gg',g'') &= \left[\prod_{1\le j<i\le m}g_ig'_ig''_j\beta_{ij}^{(x_ic_j)}\right]\left[\prod_{i=1}^mg_ig'_ig''_i\alpha_i^{\floor{\frac{x_i+c_i}{n_i}}}\right] \\
		&= \left[\prod_{1\le j<i\le m}g_ig'_ig''_j\beta_{ij}^{(a_i+b_i,c_j)}\cdot g_ig'_ig''_j\left(\frac{\sigma_j^{c_j}\alpha_i}{\alpha_i}\right)^{u_i}\right]\left[\prod_{i=1}^mg_ig'_ig''_i\alpha_i^{\floor{\frac{x_i+c_i}{n_i}}}\right] \\
		&= \left[\prod_{1\le j<i\le m}g_ig'_ig''_j\beta_{ij}^{(a_i+b_i,c_j)}\right]\left[\prod_{1\le j<i\le m}g_ig'_ig''_j\left(\frac{\sigma_j^{c_j}\alpha_i}{\alpha_i}\right)^{u_i}\right]\left[\prod_{i=1}^mg_ig'_ig''_i\alpha_i^{\floor{\frac{x_i+c_i}{n_i}}}\right],
	\end{align*}
	so we set
	\[R_3\coloneqq\left[\prod_{1\le j<i\le m}g_ig'_ig''_j\left(\frac{\sigma_j^{c_j}\alpha_i}{\alpha_i}\right)^{u_i}\right]\left[\prod_{i=1}^mg_ig'_ig''_i\alpha_i^{\floor{\frac{x_i+c_i}{n_i}}}\right].\]
	\item Note
	\[c(g,g') = \left[\prod_{1\le j<i\le m}g_ig'_j\beta_{ij}^{(a_ib_j)}\right]\left[\prod_{i=1}^mg_ig'_i\alpha_i^{u_i}\right],\]
	so we set
	\[R_4\coloneqq\left[\prod_{i=1}^mg_ig'_i\alpha_i^{u_i}\right].\]
	\item Lastly, we collect our remainder terms. Observe
	\begin{align*}
		R_3 &= \left[\prod_{i=1}^mg_ig'_i\Bigg(\prod_{j=1}^{i-1}g''_j\cdot\frac{\sigma_j^{c_j}\alpha_i}{\alpha_i}\Bigg)^{u_i}\right]\left[\prod_{i=1}^mg_ig'_ig''_i\alpha_i^{\floor{\frac{x_i+c_i}{n_i}}}\right] \\
		&= \left[\prod_{i=1}^mg_ig'_i\Bigg(\prod_{j=1}^{i-1}\frac{(\sigma_1^{c_1}\cdots\sigma_{j-1}^{c_{j-1}})\sigma_j^{c_j}\alpha_i}{(\sigma_1^{c_1}\cdots\sigma_{j-1}^{c_{j-1}})\alpha_i}\Bigg)^{u_i}\right]\left[\prod_{i=1}^mg_ig'_ig''_i\alpha_i^{\floor{\frac{x_i+c_i}{n_i}}}\right] \\
		&= \left[\prod_{i=1}^mg_ig'_i\Bigg(\prod_{j=1}^{i-1}\frac{g''_{j+1}\alpha_i}{g''_j\alpha_i}\Bigg)^{u_i}\right]\left[\prod_{i=1}^mg_ig'_ig''_i\alpha_i^{\floor{\frac{x_i+c_i}{n_i}}}\right] \\
		&= \left[\prod_{i=1}^mg_ig'_i\cdot\frac{g''_i\alpha_i^{u_i}}{\alpha_i^{u_i}}\right]\left[\prod_{i=1}^mg_ig'_ig''_i\alpha_i^{\floor{\frac{x_i+c_i}{n_i}}}\right].
	\end{align*}
	Thus,
	\begin{align*}
		R_3R_4 &= \left[\prod_{i=1}^mg_ig'_i\cdot\frac{g''_i\alpha_i^{u_i}}{\alpha_i^{u_i}}\right]\left[\prod_{i=1}^mg_ig'_ig''_i\alpha_i^{\floor{\frac{x_i+c_i}{n_i}}}\right]\left[\prod_{i=1}^mg_ig'_i\alpha_i^{u_i}\right] \\
		&= \prod_{i=1}^mg_ig'_ig''_i\alpha_i^{u_i+\floor{\frac{x_i+c_i}{n_i}}},
	\end{align*}
	which is again simple enough for our purposes.
\end{enumerate}
We now note that, for each $i$,
\[u_i+\floor{\frac{x_i+c_i}{n_i}}=\floor{\frac{a_i+b_i+c_i}{n_i}}=v_i+\floor{\frac{a_i+y_i}{n_i}}\]
by how carried addition behaves. It follows that
\[R_1R_2=\prod_{i=1}^mg_ig'_ig''_i\alpha_i^{v_i+\floor{\frac{a_i+y_i}{n_i}}}=\prod_{i=1}^mg_ig'_ig''_i\alpha_i^{u_i+\floor{\frac{x_i+c_i}{n_i}}}=R_3R_4.\]
Thus, it suffices to show that
\[\frac{gc(g',g'')}{R_1}\cdot\frac{c(g,g'')}{R_2}\stackrel?=\frac{c(gg',g'')}{R_3}\cdot\frac{c(g,g')}{R_4},\]
which is equivalent to
\[g\left[\prod_{1\le j<i\le m}g_i'g''_j\beta_{ij}^{(b_ic_j)}\right]\cdot\left[\prod_{1\le j<i\le m}g_ig'_jg''_j\beta_{ij}^{(a_i,b_j+c_j)}\right]\stackrel?=\left[\prod_{1\le j<i\le m}g_ig'_ig''_j\beta_{ij}^{(a_i+b_i,c_j)}\right]\cdot\left[\prod_{1\le j<i\le m}g_ig'_j\beta_{ij}^{(a_ib_j)}\right]\]
by the work above.

\subsection{Finishing}
We need to verify that
\[g\left[\prod_{1\le j<i\le m}g_i'g''_j\beta_{ij}^{(b_ic_j)}\right]\cdot\left[\prod_{1\le j<i\le m}g_ig'_jg''_j\beta_{ij}^{(a_i,b_j+c_j)}\right]\stackrel?=\left[\prod_{1\le j<i\le m}g_ig'_ig''_j\beta_{ij}^{(a_i+b_i,c_j)}\right]\cdot\left[\prod_{1\le j<i\le m}g_ig'_j\beta_{ij}^{(a_ib_j)}\right]\]
as discussed in the previous subsection.

Before beginning the check, we recall the relations on the $\beta$s from \autoref{eq:betarelations} can be written as
\[\frac{\sigma_2(\beta_{31})}{\beta_{31}}=\frac{\sigma_1(\beta_{32})}{\beta_{32}}\cdot\frac{\sigma_3(\beta_{21})}{\beta_{21}},\]
because we only have one triple $(i,j,k)$ of indices with $i>j>k$. This is somewhat difficult to deal with directly, so we quickly show a more general version.
\begin{lemma} \label{lem:betterbetarelation}
	Fix indices with $i>j>k$, and let $a_i,a_j,a_k\ge0$. Then
	\[\frac{\sigma_j^{a_j}\beta_{ik}^{(a_ia_k)}}{\beta_{ik}^{(a_ia_k)}}=\frac{\sigma_k^{a_k}\beta_{ij}^{(a_ia_j)}}{\beta_{ij}^{(a_ia_j)}}\cdot\frac{\sigma_i^{a_i}\beta_{jk}^{(a_ja_k)}}{\beta_{jk}^{(a_ja_k)}}.\]
\end{lemma}
\begin{proof}
	We simply compute
	\begin{align*}
		\frac{\sigma_i^{a_i}\beta_{jk}^{(a_ja_k)}}{\beta_{jk}^{(a_ja_k)}}\cdot\frac{\sigma_k^{a_k}\beta_{ij}^{(a_ia_j)}}{\beta_{ij}^{(a_ia_j)}} &= \prod_{r=0}^{a_i-1}\frac{\sigma_i^{r+1}\beta_{jk}^{(a_ja_k)}}{\sigma_i^r\beta_{jk}^{(a_ja_k)}}\cdot\prod_{p=0}^{a_k-1}\frac{\sigma_k^{p+1}\beta_{ij}^{(a_ia_j)}}{\sigma_k^p\beta_{ij}^{(a_ia_j)}} \\
		&= \prod_{p=0}^{a_k-1}\prod_{q=0}^{a_j-1}\prod_{r=0}^{a_i-1}\left(\frac{\sigma_k^p\sigma_j^q\sigma_i^{r+1}\beta_{jk}}{\sigma_k^p\sigma_j^q\sigma_i^r\beta_{jk}}\cdot\frac{\sigma_k^{p+1}\sigma_j^q\sigma_i^r\beta_{ij}}{\sigma_k^p\sigma_j^q\sigma_i^r\beta_{ij}}\right) \\
		&= \prod_{p=0}^{a_k-1}\prod_{q=0}^{a_j-1}\prod_{r=0}^{a_i-1}\sigma_k^p\sigma_j^q\sigma_i^r\left(\frac{\sigma_i\beta_{jk}}{\beta_{jk}}\cdot\frac{\sigma_k\beta_{ij}}{\beta_{ij}}\right) \\
		&= \prod_{p=0}^{a_k-1}\prod_{q=0}^{a_j-1}\prod_{r=0}^{a_i-1}\sigma_k^p\sigma_j^q\sigma_i^r\left(\frac{\sigma_j\beta_{ik}}{\beta_{ik}}\right),
	\end{align*}
	where in the last equality we have use the relation on the $\beta$s. Continuing,
	\begin{align*}
		\frac{\sigma_i^{a_i}\beta_{jk}^{(a_ja_k)}}{\beta_{jk}^{(a_ja_k)}}\cdot\frac{\sigma_k^{a_k}\beta_{ij}^{(a_ia_j)}}{\beta_{ij}^{(a_ia_j)}} &= \prod_{q=0}^{a_j-1}\left(\prod_{p=0}^{a_k-1}\prod_{r=0}^{a_i-1}\frac{\sigma_j^{q+1}\sigma_k^p\sigma_i^r\beta_{ik}}{\sigma_j^q\sigma_k^p\sigma_i^r\beta_{ik}}\right) \\
		&= \prod_{q=0}^{a_j-1}\frac{\sigma_j^{q+1}\beta_{ik}^{(a_ia_k)}}{\sigma_j^q\beta_{ik}^{(a_ia_k)}} \\
		&= \frac{\sigma_j^{a_j}\beta_{ik}^{(a_ia_k)}}{\beta_{ik}^{(a_ia_k)}},
	\end{align*}
	which is what we wanted.
\end{proof}
We now proceed with the check, by induction. More precisely, we claim that any $m'\le m$ gives
\[g_{m'+1}\left[\prod_{j<i\le m'}g_i'g''_j\beta_{ij}^{(b_ic_j)}\right]\left[\prod_{j<i\le m'}g_ig'_jg''_j\beta_{ij}^{(a_i,b_j+c_j)}\right]\stackrel?=\left[\prod_{j<i\le m'}g_ig'_ig''_j\beta_{ij}^{(a_i+b_i,c_j)}\right]\left[\prod_{j<i\le m'}g_ig'_j\beta_{ij}^{(a_ib_j)}\right]\]
which we will show by induction on $m'$. For $m'=1$, there is nothing to say because there are no indices $i>j$.

So now suppose we have equality for $m'<m$, and we give equality for $m''\coloneqq m'+1$. That is, we want to show that
\[g_{m'+2}\prod_{j<i\le m'+1}g_i'g''_j\beta_{ij}^{(b_ic_j)}\cdot\prod_{j<i\le m'+1}g_ig'_jg''_j\beta_{ij}^{(a_i,b_j+c_j)}\stackrel?=\prod_{j<i\le m'+1}g_ig'_ig''_j\beta_{ij}^{(a_i+b_i,c_j)}\cdot\prod_{j<i\le m'+1}g_ig'_j\beta_{ij}^{(a_ib_j)}\]
but by the inductive hypothesis it suffices for
\[\frac{\displaystyle g_{m''+1}\prod_{j<i\le m'+1}g_i'g''_j\beta_{ij}^{(b_ic_j)}}{\displaystyle g_{m'+1}\prod_{j<i\le m'}g_i'g''_j\beta_{ij}^{(b_ic_j)}}\cdot
\frac{\displaystyle\prod_{j<i\le m'+1}g_ig'_jg''_j\beta_{ij}^{(a_i,b_j+c_j)}}{\displaystyle\prod_{j<i\le m'}g_ig'_jg''_j\beta_{ij}^{(a_i,b_j+c_j)}}
\stackrel?=
\frac{\displaystyle\prod_{j<i\le m'+1}g_ig'_ig''_j\beta_{ij}^{(a_i+b_i,c_j)}}{\displaystyle\prod_{j<i\le m'}g_ig'_ig''_j\beta_{ij}^{(a_i+b_i,c_j)}}\cdot
\frac{\displaystyle\prod_{j<i\le m'+1}g_ig'_j\beta_{ij}^{(a_ib_j)}}{\displaystyle\prod_{j<i\le m'}g_ig'_j\beta_{ij}^{(a_ib_j)}}\]
which is collapses to
\[\frac{\displaystyle g_{m''+1}\prod_{j<i\le m'+1}g_i'g''_j\beta_{ij}^{(b_ic_j)}}{\displaystyle g_{m'+1}\prod_{j<i\le m'}g_i'g''_j\beta_{ij}^{(b_ic_j)}}\cdot
\prod_{j\le m'}g_{m''}g'_jg''_j\beta_{m''j}^{(a_{m''},b_j+c_j)}
\stackrel?=
\prod_{j\le m'}g_{m''}g'_{m''}g''_j\beta_{m''j}^{(a_{m''}+b_{m''},c_j)}\cdot
\displaystyle\prod_{j\le m'}g_{m''}g'_j\beta_{ij}^{(a_{m''}b_j)}\]
because the terms with $i<m''=m'+1$ got cancelled in the rightmost three products. Rearranging, this is the same as
\[\frac{\displaystyle g_{m''+1}\prod_{j<i\le m'+1}g_i'g''_j\beta_{ij}^{(b_ic_j)}}{\displaystyle g_{m'+1}\prod_{j<i\le m'}g_i'g''_j\beta_{ij}^{(b_ic_j)}}
\stackrel?=
\frac{\displaystyle\prod_{j<m''}g_{m''}g'_{m''}g''_j\beta_{m''j}^{(a_{m''}+b_{m''},c_j)}\cdot
\displaystyle\prod_{j<m''}g_{m''}g'_j\beta_{m''j}^{(a_{m''}b_j)}}
{\displaystyle\prod_{j<m''}g_{m''}g'_jg''_j\beta_{m''j}^{(a_{m''},b_j+c_j)}}.\]
Peeling off the $i=m''=m'+1$ terms from the left-hand side numerator, we're showing
\[\frac{\displaystyle g_{m''+1}\prod_{j<i\le m'}g_i'g''_j\beta_{ij}^{(b_ic_j)}}{\displaystyle g_{m'+1}\prod_{j<i\le m'}g_i'g''_j\beta_{ij}^{(b_ic_j)}}
\stackrel?=
\frac{\displaystyle\prod_{j<m''}g_{m''}g'_{m''}g''_j\beta_{m''j}^{(a_{m''}+b_{m''},c_j)}\cdot
\displaystyle\prod_{j<m''}g_{m''}g'_j\beta_{m''j}^{(a_{m''}b_j)}}
{\displaystyle\prod_{j<m''}g_{m''+1}g_{m''}'g''_j\beta_{m''j}^{(b_{m''}c_j)}\cdot
\prod_{j<m''}g_{m''}g'_jg''_j\beta_{m''j}^{(a_{m''},b_j+c_j)}}.\]
We take a moment to simplify the left-hand side with \autoref{lem:betterbetarelation} by writing
\begin{align*}
	g_{m'+1}\prod_{j<i\le m'}g_i'g''_j\left(\frac{\sigma_{m''}^{a_{m''}}\beta_{ij}^{(b_ic_j)}}{\beta_{ij}^{(b_ic_j)}}\right) &= g_{m''}\prod_{j<i\le m'}g_i'g''_j\left(\frac{\sigma_i^{b_i}\beta_{m''j}^{(a_{m''}c_j)}}{\beta_{m''j}^{(a_{m''}c_j)}}\cdot\frac{\beta_{m''i}^{(a_{m''}b_i)}}{\sigma_j^{c_j}\beta_{m''i}^{(a_{m''}b_i)}}\right) \\
	&= g_{m''}\left[\prod_{j=1}^{m'}g''_j\prod_{i=j+1}^{m'}g_i'\left(\frac{\sigma_i^{b_i}\beta_{m''j}^{(a_{m''}c_j)}}{\beta_{m''j}^{(a_{m''}c_j)}}\right)\cdot
	\prod_{i=1}^{m'}g_i'\prod_{j=1}^{i-1}g_j''\left(\frac{\beta_{m''i}^{(a_{m''}b_i)}}{\sigma_j^{c_j}\beta_{m''i}^{(a_{m''}b_i)}}\right)\right] \\
	&= g_{m''}\left[\prod_{j=1}^{m'}\frac{g'_{m'+1}g''_j\beta_{m''j}^{(a_{m''}c_j)}}{g'_{j+1}g''_j\beta_{m''j}^{(a_{m''}c_j)}}\cdot
	\prod_{i=1}^{m'}\frac{g_i'\beta_{m''i}^{(a_{m''}b_i)}}{g_i'g_i''\beta_{m''i}^{(a_{m''}b_i)}}\right] \\
	&= g_{m''}\left[\prod_{j<m''}\frac{g'_{m''}g''_j\beta_{m''j}^{(a_{m''}c_j)}}{g'_{j+1}g''_j\beta_{m''j}^{(a_{m''}c_j)}}\cdot
	\prod_{j<m''}\frac{g_j'\beta_{m''j}^{(a_{m''}b_j)}}{g_j'g_j''\beta_{m''j}^{(a_{m''}b_j)}}\right]
\end{align*}
after doing a lot of telescoping. Now, we can remove $g_{m''}$ everywhere to give
\[\prod_{j<m''}\frac{g'_{m''}g''_j\beta_{m''j}^{(a_{m''}c_j)}}{g'_{j+1}g''_j\beta_{m''j}^{(a_{m''}c_j)}}\cdot
\prod_{j<m''}\frac{g_j'\beta_{m''j}^{(a_{m''}b_j)}}{g_j'g_j''\beta_{m''j}^{(a_{m''}b_j)}}
\stackrel?=
\frac{\displaystyle\prod_{j<m''}g'_{m''}g''_j\beta_{m''j}^{(a_{m''}+b_{m''},c_j)}\cdot
\displaystyle\prod_{j<m''}g'_j\beta_{m''j}^{(a_{m''}b_j)}}
{\displaystyle\prod_{j<m''}g'_{m''+1}g''_j\beta_{m''j}^{(b_{m''}c_j)}\cdot
\prod_{j<m''}g'_jg''_j\beta_{m''j}^{(a_{m''},b_j+c_j)}},\]
or
\[\prod_{j<m''}\frac{g'_{m''}g''_j\beta_{m''j}^{(a_{m''}c_j)}}{g'_{j+1}g''_j\beta_{m''j}^{(a_{m''}c_j)}}
\stackrel?=
\frac{\displaystyle\prod_{j<m''}g'_{m''}g''_j\beta_{m''j}^{(a_{m''}+b_{m''},c_j)}\cdot
\displaystyle\prod_{j<m''}g'_jg''_j\beta_{m''j}^{(a_{m''}b_j)}}
{\displaystyle\prod_{j<m''}g'_{m''+1}g''_j\beta_{m''j}^{(b_{m''}c_j)}\cdot
\prod_{j<m''}g'_jg''_j\beta_{m''j}^{(a_{m''},b_j+c_j)}}.\]
Rearranging, we want
\[\prod_{j<m''}
\frac{g'_jg''_j\beta_{m''j}^{(a_{m''},b_j+c_j)}}
{g'_jg''_j\beta_{m''j}^{(a_{m''}b_j)}\cdot
g'_{j+1}g''_j\beta_{m''j}^{(a_{m''}c_j)}}
\stackrel?=\prod_{j<m''}
\frac{g'_{m''}g''_j\beta_{m''j}^{(a_{m''}+b_{m''},c_j)}}
{g'_{m''}g''_j\beta_{m''j}^{(a_{m''}c_j)}\cdot
g'_{m''+1}g''_j\beta_{m''j}^{(b_{m''}c_j)}},\]
which is
\[\prod_{j<m''}
g'_jg''_j\left(\frac{\beta_{m''j}^{(a_{m''},b_j+c_j)}}
{\beta_{m''j}^{(a_{m''}b_j)}\cdot
\sigma_j^{b_j}\beta_{m''j}^{(a_{m''}c_j)}}\right)
\stackrel?=\prod_{j<m''}
g'_{m''}g''_j\left(\frac{\beta_{m''j}^{(a_{m''}+b_{m''},c_j)}}
{\beta_{m''j}^{(a_{m''}c_j)}\cdot
\sigma_{m''}^{a_{m''}}\beta_{m''j}^{(b_{m''}c_j)}}\right).\]
However, by definition of the $\beta_{ij}^{(xy)}$, we see that
\[\frac{\beta_{m''j}^{(a_{m''},b_j+c_j)}}
{\beta_{m''j}^{(a_{m''}b_j)}\cdot
\sigma_j^{b_j}\beta_{m''j}^{(a_{m''}c_j)}}=\frac{\beta_{m''j}^{(a_{m''}+b_{m''},c_j)}}
{\beta_{m''j}^{(a_{m''}c_j)}\cdot
\sigma_{m''}^{a_{m''}}\beta_{m''j}^{(b_{m''}c_j)}}=1,\]
so everything does indeed cancel out properly. This completes the check.

\section{Computation of \texorpdfstring{$\ker\mathcal F$}{ker F}} \label{sec:havegensproof}
% !TEX root = ../abeliangerbs.tex

In this section we give a proof of \autoref{lem:havegens}. As such, we will use all the context from the statement and proceed directly with the proof; as mentioned earlier, we may add (b) back to our list of generators because it is induced by (c). Pick up some $z\coloneqq((x_i)_i,(y_{ij})_{i>j})\in\ker\mathcal F$, which is equivalent to saying
\[x_iN_i-\sum_{j=1}^{i-1}y_{ij}T_j+\sum_{j=i+1}^my_{ji}T_j=0\]
for each index $i$. We want to write $z$ as a $\ZZ[G]$-linear combination of the elements from (a)--(e). The main idea will be to slowly subtract out $\ZZ[G]$-linear combinations of the above elements (which does not affect $z\in\ker\mathcal F$) until we can prove that we have $0$ left over. We start with the $x_i$ terms, which we do in two steps.
\begin{enumerate}
	\item We begin by dealing with the $x_i$ terms. Fix some index $p$, and we will subtract out a suitable $\ZZ[G]$-linear combination of the above generators to set $x_p=0$ while not changing the other $x_i$ terms. Well, using the element
	\[\kappa_pT_p,\tag{a}\]
	we may assume that $x_p$ has no $\sigma_p$ terms because $\sigma_p\equiv1\pmod{T_p}$. Then for each $q<p$, we can subtract out a suitable multiple of
	\[T_q\kappa_p+N_p\lambda_{pq}\tag{c}\]
	to make it so that we may assume $x_p$ has no $\sigma_q$ terms because $\sigma_q\equiv1\pmod{T_q}$. Similarly, for each $q>p$, we can subtract out a suitable multiple of
	\[T_q\kappa_p-N_p\lambda_{pq}\tag{d}\]
	to make it so that we may assume $x_p$ has no $\sigma_q$ terms because $\sigma_q\equiv1\pmod{T_q}$.

	\item Thus, the above process allows us to assume that $x_p\in\ZZ$, and the above linear combinations have not affected any $x_i$ for $i\ne p$. We now use the fact that $z\in\ker\mathcal F$. Indeed, we know that
	\[x_pN_p-\sum_{j=1}^{p-1}y_{pj}T_j+\sum_{j=p+1}^my_{jp}T_j=0.\]
	Applying the augmentation map $\varepsilon\colon\ZZ[G]\to\ZZ$, sending $\varepsilon\colon\sigma_i\mapsto1$ for each index $i$, we see that $x_p\in\ZZ$ implying that $x_p$ remains fixed. On the other hand $\varepsilon\colon T_j\mapsto0$ for each index $j$ and $\varepsilon\colon N_p\mapsto n_p$, so we are left with
	\[n_px_p=0.\]
	Because $n_p\ne0$ (it's the order of $\sigma_p$), we conclude that $x_p=0$. Applying this argument to the other $x_i$ terms, we conclude that we may assume $x_i=0$ for each $i$.
\end{enumerate}
It remains to deal with the $y_{ij}$ terms, which is a little more involved. For reference, we are showing that
\[-\sum_{j=1}^{i-1}y_{ij}T_j+\sum_{j=i+1}^my_{ji}T_j=0\]
for each index $i$ implies that $z=((0)_i,(y_{ij})_{i>j})$ is a $\ZZ[G]$-linear combination of the terms from (b) and (e).

We will now more or less proceed with the $y_{ij}$ by induction on $m$, allowing the group $G$ (in its number of generators $m$) to be changed in the process. For $m=1$, there is nothing to say because there is no $y_{ij}$ term at all. For a taste of how we will use \autoref{lem:separatenijs}, we also work out $m=2$: our equations read
\[\underbrace{-y_{21}T_1=0}_{i=1}\qquad\text{and}\qquad\underbrace{y_{21}T_2=0}_{i=2}.\]
Thus, $y_{21}\in(\ker T_1)\cap(\ker T_2)=(\im N_1)\cap(\im N_2)$, which is $\im N_1N_2$ by \autoref{lem:separatenijs}.

We now proceed with the general case; take $m>2$. Let $G'\coloneqq\langle\sigma_2,\ldots,\sigma_m\rangle$, which has $m-1$ generators. By the inductive hypothesis, we may assume the statement for $G'$. Explicitly, we will assume that, if $(y_{ij}')_{i>j\ge2}\in\ZZ[G']^{\binom{m-1}2}$ are variables satisfying
\[-\sum_{j=2}^{i-1}y_{ij}'T_j+\sum_{j=i+1}^my_{ji}'T_j=0\]
for each index $i\ge2$, then $y_{ij}'$ are a linear combination of terms from the elements from (b) and (e) above, only using indices at least $2$.

We will again proceed in steps, for clarity.
\begin{enumerate}
	\item To apply the inductive hypothesis, we need to force $y_{pq}\in\ZZ[G']$ for each pair of indices $(p,q)$ with $p>q\ge2$. Well, we use the relation (e) so that we can subtract multiples of
	\[T_q\lambda_{p1}-T_1\lambda_{pq}-T_p\lambda_{q1}.\]
	In particular, this element will subtract out $T_1$ from $y_{pq}$ while only introducing chaos to the elements $y_{p1}$ and $y_{q1}$ in the process. Thus, subtracting a suitable multiple allows us to assume that $y_{pq}$ has no $\sigma_1$ terms while not affecting any other $y_{ij}$ with $i>j\ge2$.

	Applying this process to all $y_{ij}$ with $i>j\ge2$, we do indeed get $y_{ij}\in\ZZ[G']$ for each $i>j\ge2$.

	\item We are now ready to apply the inductive hypothesis. For each index $i\ge2$, we have the equation
	\[-y_{i1}T_1-\sum_{j=2}^{i-1}y_{ij}T_j+\sum_{j=i+1}^my_{ji}T_j=0.\]
	Because each $y_{pq}$ term with $p>q\ge2$ features no $\sigma_1$, applying the transformation $\sigma_1\mapsto1$ will affect no term in the sums while causing $y_{i1}T_1$ to vanish. Thus, we have the equations
	\[-\sum_{j=2}^{i-1}y_{ij}T_j+\sum_{j=i+1}^my_{ji}T_j=0\]
	for each index $i\ge2$. Because $y_{ij}\in\ZZ[G']$ for $i>j\ge2$ already, we see that we may apply the inductive hypothesis to assert that the $y_{ij}$ are $\ZZ[G']$-linear combinations of terms from (b) and (e) (only using indices at least $2$).
	
	Subtracting these linear combinations out, we may assume $y_{ij}=0$ for each $i>j\ge2$.

	\item To take stock, our equations for $i\ge2$ now read
	\[-y_{i1}T_1=0,\]
	which simply tells us that $y_{i1}\in\im N_1$ for each $i\ge2$. As such, we pick up $w_i\in\ZZ[G]$ so that $y_{i1}=w_iN_1$ for each $i\ge2$; because $\sigma_1N_1=N_1$, we may assume that $w_i\in\ZZ[G']$ for each $i\ge2$.

	Now the equation for $i=1$ reads
	\[\sum_{j=2}^my_{j1}T_j=0,\]
	or
	\[\sum_{i=2}^mw_iN_1T_i=0.\]
	Sending $\sigma_1\mapsto1$, we see that $w_i$ and $T_i$ are both fixed because they feature no $\sigma_1$s, so we merely have
	\[n_1\sum_{i=2}^mw_iT_i=0.\]
	Dividing out by $n_1$, we are left with
	\[\sum_{i=2}^mw_iT_i=0.\]

	\item At this point, we may appear stuck, but we have one final trick: taking indices $p>q\ge2$, subtracting out multiples of
	\[\big(T_q\lambda_{p1}-T_1\lambda_{pq}-T_p\lambda_{q1}\big)\cdot N_1\]
	will not affect the $y_{pq}$ term because $T_1N_1$. Indeed, subtracting this term out looks like
	\[T_qN_1\lambda_{p1}-T_pN_1\lambda_{q1},\]
	which after factoring out $N_1$ takes $w_p\mapsto w_p-T_q$ and $w_q\mapsto w_q+T_p$.

	In particular, fixing any $q\ge2$ and then applying this trick for all $p>q$, we may assume that $w_q$ does not feature any $\sigma_p$ terms for $p>q$. Thus, looking at our equation
	\[\sum_{i=2}^mw_iT_i=0,\]
	we are now able to show that $w_i\in\ker T_i=\im N_i$ for each $i\ge2$, which will finish because it shows $y_{i1}\in N_iN_1$. Indeed, starting with $i=2$, we see that $w_2$ features no $\sigma_p$ for $p>2$, so we may take $\sigma_p\mapsto1$ for each $p>2$ safely, giving the equation
	\[w_2T_2=0,\]
	finishing for $w_2$. Thus, we are left with the equation
	\[\sum_{i=3}^mw_iT_i=0,\]
	from which we see we can induct downwards (this has fewer variables) to finish.
\end{enumerate}
The above steps complete the proof, as advertised.

\end{document}