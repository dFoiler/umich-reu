Throughout, we will let $u_{L/K}$ denote a representative of the fundamental class in $H^2(L/K)$ rather than the actual cohomology class, mostly out of laziness.

We now return to the set-up in \autoref{sec:setup} and track through \autoref{lem:explicitresinf} in our case. For reference, the following is the diagram that we will be chasing around; here $G\coloneqq\op{Gal}(ML/K)$ and $H\coloneqq\op{Gal}(ML/L)$.
% https://q.uiver.app/?q=WzAsOSxbMiwwLCJIXjIoXFxvcHtHYWx9KE0vSyksTV5cXHRpbWVzKSJdLFsyLDEsIkheMihcXG9we0dhbH0oTUwvSyksTUxeXFx0aW1lcykiXSxbMywxLCJIXjIoXFxvcHtHYWx9KE1ML0wpLE1MXlxcdGltZXMpIl0sWzMsMiwiSF4xKFxcb3B7R2FsfShNTC9MKSxcXG9we0hvbX1fXFxaWihJX3tcXG9we0dhbH0oTUwvSyl9LE1MXlxcdGltZXMpKSJdLFsyLDIsIkheMShcXG9we0dhbH0oTUwvSyksXFxvcHtIb219X1xcWlooSV97XFxvcHtHYWx9KE1ML0spfSxNTF5cXHRpbWVzKSkiXSxbMSwyLCJIXjEoXFxvcHtHYWx9KEwvSyksXFxvcHtIb219X1xcWlooSV97XFxvcHtHYWx9KE1ML0spfSxNTF5cXHRpbWVzKV57XFxvcHtHYWx9KEwvSyl9KSJdLFsxLDEsIkheMihcXG9we0dhbH0oTC9LKSxMXlxcdGltZXMpIl0sWzAsMSwiMCJdLFswLDIsIjAiXSxbNSw2LCJcXGRlbHRhIiwyXSxbNCwxLCJcXGRlbHRhIiwyXSxbMywyLCJcXGRlbHRhIiwyXSxbMSwyLCJcXG9we1Jlc30iXSxbNiwxLCJcXG9we0luZn0iXSxbMCwxLCJcXG9we0luZn0iXSxbNSw0XSxbNCwzXSxbNyw2XSxbOCw1XV0=&macro_url=https%3A%2F%2Fraw.githubusercontent.com%2FdFoiler%2Fnotes%2Fmaster%2Fnir.tex
\[\begin{tikzcd}
	&& {H^2(\op{Gal}(M/K),M^\times)} \\
	0 & {H^2(\op{Gal}(L/K),L^\times)} & {H^2(G,ML^\times)} & {H^2(\op{Gal}(ML/L),ML^\times)} \\
	0 & {H^1(G/H,\op{Hom}_\ZZ(I_{G},ML^\times)^{H})} & {H^1(G,\op{Hom}_\ZZ(I_{G},ML^\times))} & {H^1(H,\op{Hom}_\ZZ(I_{G},ML^\times))}
	\arrow["\delta"', from=3-2, to=2-2]
	\arrow["\delta"', from=3-3, to=2-3]
	\arrow["\delta"', from=3-4, to=2-4]
	\arrow["{\op{Res}}", from=2-3, to=2-4]
	\arrow["{\op{Inf}}", from=2-2, to=2-3]
	\arrow["{\op{Inf}}", from=1-3, to=2-3]
	\arrow["\op{Inf}", from=3-2, to=3-3]
	\arrow["\op{Res}", from=3-3, to=3-4]
	\arrow[from=2-1, to=2-2]
	\arrow[from=3-1, to=3-2]
\end{tikzcd}\]
To begin, we know that we can write
\[u_{M/K}\left(\sigma_K^i,\sigma_K^j\right)=\pi^{\floor{\frac{i+j}n}}=\begin{cases}
	1 & i+j<n, \\
	\pi & i+j\ge n,
\end{cases}\]
where $\pi$ is a uniformizer of $K$. Inflating this down to $H^2(G,ML^\times)$ gives
\[(\op{Inf}u_{M/K})\left(\sigma_K^{a_1}\tau,\sigma_K^{b_1}\tau'\right)=\pi^{\floor{\frac{a_1+b_1}n}}.\]
Now, we use \autoref{lem:dimshift} to move down to $H^1(G,\op{Hom}_\ZZ(I_G,ML^\times))$ as
\[\delta^{-1}(\op{Inf}u_{M/K})\left(\sigma_K^{a_1}\tau\right)\left(\sigma_K^{b_1}\tau'-1\right)=\sigma_K^{b_1}\tau'\cdot (\op{Inf}u_{M/K})\left(\sigma_K^{[-b_1]}(\tau')^{-1},\sigma_K^{a_1}\tau\right)=p^{\floor{\frac{a_1+[-b_1]}n}},\]
where $[k]$ denote the integer $0\le[k]<n$ such that $k\equiv[k]\pmod n$.

Now, we need to show that the restriction to $H=\langle\sigma_k^f\rangle$ is a coboundary. That is, we need to find $b\in\op{Hom}_\ZZ(I_G,ML^\times)$ such that
\[\delta^{-1}(\op{Inf}u_{M/K})\left(\sigma_K^{fa_1}\right)=\frac{\sigma_K^{fa_1}\cdot b}b.\]
Because $I_G$ is freely generated by elements of the form $g-1$ for $g\in G$, it suffices to plug in some arbitrary $\sigma_K^{b_1}\tau'-1$, which we see requires
\begin{align*}
	\pi^{\floor{\frac{fa_1+[-b_1]}n}} &= \frac{\big(\sigma_K^{fa_1}\cdot b\big)\left(\sigma_K^{b_1}\tau'-1\right)}{b\left(\sigma_K^{b_1}\tau'-1\right)} \\
	&= \frac{\sigma_K^{fa_1} b\left(\sigma_K^{b_1-fa_1}\tau'-1\right)}{\sigma_K^{fa_1} b\left(\sigma_K^{-fa_1}-1\right)b\left(\sigma_K^{b_1}\tau'-1\right)}.
\end{align*}
We can see that $b$ should not depend on $\tau'$, so we define $\hat b\left(\sigma_K^a\right)=b\left(\sigma_K^a\tau'-1\right)$; the above is then equivalent to
\begin{align*}
	\pi^{\floor{\frac{fa_1+[-b_1]}n}} &= \frac{\sigma_K^{fa_1}\hat b\left(\sigma_K^{b_1-fa_1}\right)}{\sigma_K^{fa_1}\hat b\left(\sigma_K^{-fa_1}\right)\hat b\left(\sigma_K^{b_1}\right)} \\
	\pi^{\floor{\frac{fa_1+b_1}n}} &= \frac{\hat b\left(\sigma_K^{-b_1-fa_1}\right)}{\hat b\left(\sigma_K^{-fa_1}\right)\sigma_K^{-fa_1}\hat b\left(\sigma_K^{-b_1}\right)},
\end{align*}
where we have negated $b_1$ in the last step. At this point, the right-hand side will look a lot more natural if we set $\tau\coloneqq\sigma_K^{-1}$, which turns this into
\[\frac{\hat b\left(\tau_K^{fa_1}\right)\tau_K^{fa_1}\hat b\left(\tau_K^{b_1}\right)}{\hat b\left(\tau_K^{b_1fa_1}\right)} = (1/\pi)^{\floor{\frac{fa_1+b_1}n}}\]
after taking reciprocals. Thus, we see that $\hat b$ should be counting carries of $\tau$s. With this in mind, we let $\varpi$ be a uniformizer of $K_{\pi,\nu}$ and note that $\varpi\in L$ be a uniformizer because $L/K_{\pi,\nu}$ is an unramified extension. It follows that
\[\varpi^{[ML:L]}\in\op N_{ML/L}\left(ML^\times\right).\]
Further, $\varpi^{[ML:L]}$ has the same absolute value as $\pi$ because $K_{\pi,\nu}/K$ is a totally ramified extension of degree $[K_{\pi,\nu}:K]=[ML:M]=[ML:L]$. Thus, $\pi$ is a norm in $\op N_{ML/L}(ML^\times)$ because $ML/L$ is unramified and so $\mathcal O_L^\times\subseteq\op N_{ML/L}(ML^\times)$. Thus, we find $\gamma\in ML^\times$ such that
\[\op N_{ML/L}(\gamma)=\pi.\]
The point of doing all of this is so that we can codify our carrying by writing
\[\hat b\left(\tau_K^a\right)\coloneqq\prod_{i=0}^{\floor{a/f}-1}\tau^{if}(\gamma)^{-1}.\]
Tracking out $\hat b$ backwards to $b$, our desired $b\in\op{Hom}_\ZZ(I_G,ML^\times)$ is given by
\[\boxed{b\left(\sigma_K^{a_1}\tau-1\right)=\prod_{i=0}^{\floor{[-a_1]/f}-1}\sigma_K^{-if}(\gamma)^{-1}}.\]
We take a moment to write out $c\coloneqq\delta^{-1}(\op{Inf}u_{M/K})/db$, which looks like
\begin{align*}
	c\left(\sigma_K^{a_1}\tau\right)\left(\sigma_K^{b_1}\tau'-1\right) &= \frac{\delta^{-1}(\op{Inf}u_{M/K})}{db}\left(\sigma_K^{a_1}\tau\right)\left(\sigma_K^{b_1}\tau'-1\right) \\
	&= \frac{\delta^{-1}(\op{Inf}u_{M/K})\left(\sigma_K^{a_1}\tau\right)\left(\sigma_K^{b_1}\tau'-1\right)}{\left(\sigma_K^{a_1}\tau b\right)\left(\sigma_K^{b_1}\tau'-1\right)/b\left(\sigma_K^{b_1}\tau'-1\right)} \\
	&= \frac{\pi^{\floor{(a_1+[-b_1])/n}}}{\sigma_K^{a_1}\tau b\left(\sigma_K^{b_1-a_1}\tau'\tau^{-1}-\sigma_K^{-a_1}\tau^{-1}\right)/b\left(\sigma_K^{b_1}\tau'-1\right)} \\
	&= \pi^{\floor{(a_1+[-b_1])/n}}\cdot\hat b\left(\sigma_K^{b_1}\right)\cdot\sigma_K^{a_1}\tau\left(\frac{\hat b\left(\sigma_K^{-a_1}\right)}{\hat b\left(\sigma_K^{b_1-a_1}\right)}\right).
\end{align*}
Before proceeding, we discuss a few special cases.
\begin{itemize}
	\item Taking $\sigma_K^{a_1}\tau=\tau_i$ for some $\tau_i$, we get
	\begin{align*}
		c\left(\tau_i\right)\left(\sigma_K^{b_1}\tau'-1\right) &= \pi^{\floor{(0+[-b_1])/n}}\cdot\hat b\left(\sigma_K^{b_1}\right)\cdot\tau_i\left(\frac{1}{\hat b\left(\sigma_K^{b_1}\right)}\right) \\
		&= \hat b\left(\sigma_K^{b_1}\right)/\tau_i\hat b\left(\sigma_K^{b_1}\right).
	\end{align*}
	In particular, $c\left(\sigma_x\right)\left(\sigma_K^{-1}-1\right)=1$, provided that $f>1$. Additionally, $c(\tau_i)\left(\tau'-1\right)=1$.
	
	Our general theory says that $h\mapsto c(\sigma_x)(h-1)$ is a $1$-cocycle in $Z^1(H,ML^\times)$ (though we could also check this directly), so Hilbert's Theorem 90 promises us a magical element $\eta_i\in ML^\times$ such that
	\[\frac{\sigma_K^{fb_1}\eta_i}{\eta_i}=\frac{\hat b\left(\sigma_K^{fb_1}\right)}{\tau_i\hat b\left(\sigma_K^{fb_1}\right)}\]
	for all $\sigma_K^{fb_1}\in H$. This condition will be a little clearer if we write everything in terms of $\tau_K\coloneqq\sigma_K^{-1}$, which transforms this into
	\[\frac{\tau_K^{fb_1}\eta_i}{\eta_i}=\frac{\hat b\left(\tau_K^{-fb_1}\right)}{\tau_i\hat b\left(\tau_K^{-fb_1}\right)}=\prod_{i=0}^{b_1-1}\frac{\tau_K^{if}(\gamma^{-1})}{\tau_i\tau_K^{if}(\gamma^{-1})}=\prod_{i=0}^{b_1-1}\frac{\tau_i\tau_K^{if}(\gamma)}{\tau_K^{if}(\gamma)}.\]
	Because we are dealing with a cyclic group $H$, it is not too hard to see that it suffices merely for $b_1=1$ to hold, so our magical element $\eta_{c(\sigma_x)}$ merely requires
	\[\boxed{\frac{\sigma_K^{-f}\left(\eta_i\right)}{\eta_i}=\frac{\tau_i(\gamma)}{\gamma}}\]
	after inverting $\tau_K$ back to $\sigma_K$.
	\item Taking $\sigma_K^{a_1}\tau=\sigma_K$, we get
	\begin{align*}
		c\left(\sigma_K\right)\left(\sigma_K^{b_1}\tau'-1\right) &= \pi^{\floor{(1+[-b_1])/n}}\cdot\hat b\left(\sigma_K^{b_1}\right)\cdot\sigma_K\left(\frac{\hat b\left(\sigma_K^{-1}\right)}{\hat b\left(\sigma_K^{b_1-1}\right)}\right).
	\end{align*}
	In particular, $\sigma_K^{b_1}\tau'=\tau_i^{-1}$ will give $c(\sigma_K)\left(\tau_i^{-1}-1\right)=1$. We will also want $c(\sigma_K)\left(\sigma_K^{-b_1}-1\right)$ for $0\le b_1<f$. Using the fact that $f<n$ and $f>1$, it is not too hard to see that everything will cancel down to $1$ except in the case where $b_1=f-1$, where we get
	\[c(\sigma_K)\left(\sigma_K^{-(f-1)}-1\right)=\sigma_K\left(\frac1{\hat b\left(\sigma_K^{-f}\right)}\right)=\sigma_K(\gamma).\]
	Continuing as before, our general theory says that $h\mapsto c(\sigma_K)(h-1)$ is a $1$-cocycle in $Z^1(H,ML^\times)$, though again we could just check this directly. It follows that Hilbert's Theorem 90 promises us a magical element $\eta_K\in ML^\times$ such that
	\[\frac{\sigma_K^{fb_1}\eta_K}{\eta_K}=p^{\floor{(1+[-fb_1])/n}}\cdot\hat b\left(\sigma_K^{fb_1}\right)\cdot\sigma_K\left(\frac{\hat b\left(\sigma_K^{-1}\right)}{\hat b\left(\sigma_K^{fb_1-1}\right)}\right)\]
	for all $\sigma_K^{fb_1}\in H$. Using $f>1$, this collapses down to
	\[\frac{\sigma_K^{fb_1}\eta_K}{\eta_K}=\frac{\hat b\left(\sigma_K^{fb_1}\right)}{\sigma_K\hat b\left(\sigma_K^{fb_1-1}\right)}.\]
	As before, this condition will be a little clearer if we set $\tau_K\coloneqq\sigma_K^{-1}$, which turns the condition into
	\[\frac{\tau_K^{fb_1}\eta_K}{\eta_K}=\frac{\hat b\left(\tau_K^{fb_1}\right)}{\sigma_K\hat b\left(\tau_K^{fb_1+1}\right)}=\prod_{i=0}^{b_1-1}\frac{\tau_K^{if}(\gamma^{-1})}{\sigma_K\tau_K^{if}(\gamma^{-1})}=\prod_{i=0}^{b_1-1}\frac{\sigma_K\tau_K^{if}(\gamma)}{\tau^{if}(\gamma)}.\]
	(Notably, $\hat b\left(\tau^{fb_1}\right)=\hat b\left(\tau^{fb_1+1}\right)$ because $f>1$.) Again, because $H$ is cyclic generated by $\tau^f$, an induction shows that it suffices to check this condition for $b_1=1$, which means that our magical element $\eta_K\in ML^\times$ is constructed so that
	\[\boxed{\frac{\sigma_K^{-f}\left(\eta_K\right)}{\eta_K}=\frac{\sigma_K(\gamma)}{\gamma}}\]
	where we have again inverted back from $\tau_K$ to $\sigma_K$.
	\item We will not actually need a more concrete description of this, but we remark that we can run the same story for any $g\in G$ through to get an element $\eta_g\in ML^\times$ such that
	\[\frac{\sigma_K^{fb_1}\eta_g}{\eta_g}=\frac1{c(g)(\sigma_K^{fb_1}-1)}\]
	for any $\sigma_K^{fb_1}\in H$. As usual, this follows from our general theory.
\end{itemize}
We are now ready to describe the local fundamental class. Piecing what we have so far, we know from \autoref{lem:explicitresinf} that we can write
\[u_{L/K}(g,g')\coloneqq gc(g')\left(g^{-1}-1\right)\cdot\frac{g\eta_{g'}\cdot \eta_{g}}{\eta_{gg'}}.\]
Here are the values that we care about for our specific computation; for consistency, we set $\tau_0\coloneqq\sigma_K$ and $n_0\coloneqq f$ to be the order of $\tau_0$.
\begin{itemize}
	\item We write
	\begin{align*}
		u_{L/K}(\sigma_K,\tau_i) &= \sigma_Kc(\tau_i)\left(\sigma_K^{-1}-1\right)\cdot\frac{\sigma_K \eta_i\cdot \eta_K}{\eta_{\sigma_K\tau_i}} \\
		&= \frac{\sigma_K \eta_i\cdot \eta_K}{\eta_{\sigma_K\sigma_x}}.
	\end{align*}
	\item We write
	\begin{align*}
		u_{L/K}(\tau_i,\sigma_K) &= \tau_ic(\sigma_K)\left(\tau_i^{-1}-1\right)\cdot\frac{\tau_i\eta_K\cdot \eta_i}{\eta_{\sigma_x\sigma_K}} \\
		&= \frac{\tau_i\eta_K\cdot \eta_i}{\eta_{\sigma_x\sigma_K}}.
	\end{align*}
	\item In particular, we know that we can set $\beta_{i0}$ to
	\begin{align*}
		\beta_{i0} \coloneqq{}& \frac{u_{L/K}(\tau_i,\sigma_K)}{u_{L/K}(\sigma_K,\tau_i)} \\
		={}& \frac{\tau_i\eta_K\cdot \eta_i/\eta_{\sigma_x\sigma_K}}{\sigma_K \eta_i\cdot \eta_K/\eta_{\sigma_K\sigma_x}} \\
		\Aboxed{\beta_{i0}={}& \frac{\eta_i}{\sigma_K\left(\eta_i\right)}\cdot\frac{\tau_i\left(\eta_K\right)}{\eta_K}}.
	\end{align*}
	% As a sanity check, we can hit this $\beta$ with $\sigma_K^{-f}$ to show that $\beta\in(ML)^H=L$; namely, $\sigma_K^{-f}\eta_{c(\sigma_K)}=\frac{\sigma_K\alpha}\alpha\cdot \eta_{c(\sigma_K)}$ and $\sigma_K^{-f}\eta_{c\sigma(x)}=\frac{\sigma_x\alpha}\alpha\cdot \eta_{c(\sigma_x)}$ by construction, so we can see that everything will appropriately cancel out.
	\item We write
	\begin{align*}
		u_{L/K}(\tau_i,\tau_j) &= \tau_ic(\tau_j)\left(\tau_j^{-1}-1\right)\cdot\frac{\tau_i\eta_j\cdot \eta_{i}}{\eta_{\tau_i\tau_j}} \\
		&= \frac{\tau_i\eta_j\cdot \eta_{i}}{\eta_{\tau_i\tau_j}}.
	\end{align*}
	\item Thus, for $i>j>0$, we can set $\beta_{ij}$ to
	\begin{align*}
		\beta_{ij} \coloneqq{}& \frac{u_{L/K}(\tau_i,\tau_j)}{u_{L/K}(\tau_j,\tau_i)} \\
		={}& \frac{\tau_i\eta_j\cdot \eta_{i}/\eta_{\tau_i\tau_j}}{\tau_j\eta_i\cdot \eta_{j}/\eta_{\tau_i\tau_j}} \\
		\Aboxed{\beta_{ij}={}& \frac{\eta_{i}}{\tau_j\eta_i}\cdot\frac{\tau_i\eta_j}{\eta_{j}}}.
	\end{align*}
	\item We will go ahead and compute $\alpha_0$ and the $\alpha_i$, for completeness. For $\alpha_0$, our element is given by
	\begin{align*}
		\alpha_0 \coloneqq{}& \prod_{i=0}^{f-1}u_{L/K}\left(\sigma_K^i,\sigma_K\right) \\
		={}& \prod_{i=0}^{f-1}\left(\sigma_K^ic\left(\sigma_K,\sigma_K^{-i}-1\right)\cdot\frac{\sigma_K^i\eta_K\cdot \eta_{\sigma_K^i}}{\eta_{\sigma_K^{i+1}}}\right).
	\end{align*}
	Recall from our general theory that $\eta_g$ only depends on the coset of $g$ in $G/H$, so we see that the product of the quotients $\eta_{\sigma_K^i}/\eta_{\sigma_K^{i+1}}$ will cancel out. As for the $c$ term, we know from our computation that this is $1$ until $i=f-1$, which gives $\sigma_K(\gamma)$. As such, we collapse down to
	\[\boxed{\alpha_0=\sigma_K^f(\gamma)\cdot\prod_{i=0}^{f-1}\sigma_K^i\left(\eta_K\right)}.\]
	\item For $\alpha_i$ with $i>0$, our element is given by
	\begin{align*}
		\alpha_i \coloneqq{}& \prod_{p=0}^{n_i-1}u_{L/K}\left(\tau_i^p,\tau_i\right) \\
		={}& \prod_{p=0}^{n_i-1}\tau_i^pc(\tau_i)\left(\tau_i^{-p}-1\right)\cdot\frac{\tau_i^p\eta_i\cdot \eta_{\tau_i^p}}{\eta_{\tau_i^{p+1}}}.
	\end{align*}
	Recalling that $\tau_i$ has order $n_i$, our quotient term $\eta_{\tau_i^i}/\eta_{\tau_i^{i+1}}$ will again cancel out. Additionally, the cocycle $c$ always spits out $1$ on these inputs, so we are left with
	\[\boxed{\alpha_i=\prod_{p=0}^{n_i-1}\tau_i^p\left(\eta_i\right)}.\]
\end{itemize}
We summarize the results above in the following theorem.
\begin{theorem} \label{thm:fundtriple}
	Fix everything as in the set-up. Then there exists some $\gamma\in ML^\times$ such that $\op N_{ML/L}(\gamma)=\pi$ and elements in $\eta_K,\eta_i\in ML^\times$ (for $1\le i\le t$) such that
	\[\frac{\sigma_K^{-f}\left(\eta_K\right)}{\eta_K}=\frac{\sigma_K(\gamma)}{\gamma}\qquad\text{and}\qquad\frac{\sigma_K^{-f}\left(\eta_i\right)}{\eta_i}=\frac{\tau_i(\gamma)}{\gamma}.\]
	Then the tuple
	\[\big((\alpha_0,\alpha_i),(\beta_{ij})\big)\coloneqq
	\left(\sigma_K^f(\gamma)\cdot\prod_{i=0}^{f-1}\sigma_K^i\left(\eta_K\right),\quad
	\prod_{p=0}^{n_i-1}\tau_i^p\left(\eta_i\right),\quad
	\frac{\eta_{i}}{\tau_j\eta_i}\cdot\frac{\tau_i\eta_j}{\eta_{j}}\right)\]
	corresponds to the fundamental class $u_{L/K}\in H^2(\op{Gal}(L/K),L^\times)$.
\end{theorem}
We remark that we can replace $\gamma$ with $\sigma_K^f(\gamma)$ (which still has norm $p$) while keeping all other variables the same; this gives us the following slightly prettier presentation. Note that we have multiplied the equations for $\eta_\bullet$ by $\sigma_K^f$ on both sides.
\begin{corollary} \label{cor:fundtriple}
	Fix everything as in the set-up. Then there exists some $\gamma\in ML^\times$ such that $\op N_{ML/L}(\gamma)=\pi$ and elements in $\eta_K,\eta_i\in ML^\times$ (for $1\le i\le t$) such that
	\[\frac{\eta_K}{\sigma_K^{f}\left(\eta_K\right)}=\frac{\sigma_K(\gamma)}{\gamma}\qquad\text{and}\qquad\frac{\eta_i}{\sigma_K^f\left(\eta_i\right)}=\frac{\tau_i(\gamma)}{\gamma}.\]
	Then the tuple
	\[\big((\alpha_0,\alpha_i),(\beta_{ij})\big)\coloneqq
	\left(\gamma\cdot\prod_{i=0}^{f-1}\sigma_K^i\left(\eta_K\right),\quad
	\prod_{p=0}^{n_i-1}\tau_i^p\left(\eta_i\right),\quad
	\frac{\eta_{i}}{\tau_j\eta_i}\cdot\frac{\tau_i\eta_j}{\eta_{j}}\right)\]
	corresponds to the fundamental class $u_{L/K}\in H^2(\op{Gal}(L/K),L^\times)$.
\end{corollary}
For brevity later on, we will give a name to these conditions.
\begin{definition}
	Fix an extension $L/K$. The $\{\sigma_i\}_{i=1}^m$-tuples constructed in \autoref{cor:fundtriple} will be called \textit{fundamental tuples}.
\end{definition}
We will show shortly that fundamental tuples actually give the entire equivalence class of $\{\sigma_i\}_{i=1}$-tuples associated to the fundamental class.
\begin{remark}
	This result is essentially a stronger version of Dwork's theorem (1958), recorded in Serre's \textit{Local Fields}, chapter~XIII, Theorem~2. Namely, Dwork and Serre are interested in computing the reciprocity map, which roughly means we only want access to the $\alpha$s, but above we are interested in computing the full fundamental class.
\end{remark}