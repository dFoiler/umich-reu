Fix a finite abelian extension of local fields $L/K$ which is not unramified.
\begin{remark}
	Assuming that $L/K$ is not unramified is a purely technical requirement; indeed, most of the arguments go through in this case. Regardless, when unramified, there already exist descriptions of the local fundamental class.
\end{remark}
Then let $K_m$ be the largest unramified subextension, which we will give degree $m$; let $\overline\sigma_K\in\op{Gal}(L/K)$ denote the Frobenius automorphism, which lets us set
\[K_{\pi,\nu}\coloneqq L^{\langle\overline\sigma_K\rangle}.\]
In particular, $K_{\pi,\nu}/K$ is totally ramified because, for example, the residue fields of $K_{\pi,\nu}$ and $K$ have the same order.
\begin{example}
	For $K=\QQ_p$, we can take $K_m=\QQ_p\left(\zeta_{p^m-1}\right)$ and $K_{\pi,\nu}=\QQ_p\left(\zeta_{p^\nu}\right)$.
\end{example}
This gives us the following tower of fields.
% https://q.uiver.app/?q=WzAsNCxbMSwyLCJLIl0sWzAsMSwiS197XFxwaSxcXG51fSJdLFsyLDEsIktfbSJdLFsxLDAsIkwiXSxbMCwxLCIiLDAseyJzdHlsZSI6eyJoZWFkIjp7Im5hbWUiOiJub25lIn19fV0sWzEsMywiIiwwLHsic3R5bGUiOnsiaGVhZCI6eyJuYW1lIjoibm9uZSJ9fX1dLFswLDIsIiIsMix7InN0eWxlIjp7ImhlYWQiOnsibmFtZSI6Im5vbmUifX19XSxbMiwzLCIiLDIseyJzdHlsZSI6eyJoZWFkIjp7Im5hbWUiOiJub25lIn19fV1d&macro_url=https%3A%2F%2Fraw.githubusercontent.com%2FdFoiler%2Fnotes%2Fmaster%2Fnir.tex
\[\begin{tikzcd}
	& L \\
	{K_{\pi,\nu}} && {K_m} \\
	& K
	\arrow[no head, from=3-2, to=2-1]
	\arrow[no head, from=2-1, to=1-2]
	\arrow[no head, from=3-2, to=2-3]
	\arrow[no head, from=2-3, to=1-2]
\end{tikzcd}\]
Quickly, we note that $L/K_{\pi,\nu}$ has Galois group generated by the Frobenius $\overline\sigma_K$ and therefore has degree $m$, so we have that $K_{\pi,\nu}$ and $K_m$ are linearly disjoint over $K$ and
\[[L:K]=[L:K_{\pi,\nu}]\cdot[K_{\pi,\nu}:K]=[K_m:K]\cdot[K_{\pi,\nu}:K],\]
which implies that $L=K_{\pi,\nu}K_m$ as well.

We provide some quick commentary on these extensions.
\begin{itemize}
	\item The extension $K_m/K$ is unramified of degree $f\coloneqq m$; note we are assuming $L\ne K_m$ and hence $f<n$. Its Galois group is thus generated by the Frobenius element defined by $\overline\sigma_K$.
	\item The extension $K_{\pi,\nu}/K$ is totally ramified of degree $[K_{\pi,\nu}:K]$. Because we are assuming this Galois group is abelian, we may write
	\[\op{Gal}(K_{\pi,\nu}/K)\simeq\Gamma_1\times\cdots\times\Gamma_t\]
	where $\Gamma_i=\langle\tau_i\rangle\subseteq\op{Gal}(K_{\pi,\nu}/K)$ is a cyclic group of order $n_i$.
	% Its Galois group is thus isomorphic to $\left(\ZZ/p^\nu\ZZ\right)^\times$, where the isomorphism takes $x\in\left(\ZZ/p^\nu\ZZ\right)^\times$ to
	% \[\sigma_x\colon\zeta_{p^\nu}\mapsto\zeta_{p^\nu}^{x^{-1}}.\]
	% The group $\left(\ZZ/p^\nu\ZZ\right)^\times$ is cyclic, so we will fix a generator $x$, which gives us a distinguished generator $\sigma_x\in\op{Gal}\left(\QQ(\zeta_{p^\nu})/\QQ_p\right)$.
	\item Because $K_{\pi,\nu}/K$ is totally ramified and $K_m/K$ is unramified, we have that the fields $K_{\pi,\nu}$ and $K_m$ are linearly disjoint over $K$. As such, $L=K_{\pi,\nu}K_m$ has
	\begin{align*}
		\op{Gal}(L/K_{\pi,\nu}) &\simeq \op{Gal}(K_m/K)=\langle\overline\sigma_K\rangle \\
		\op{Gal}(L/K_m) &\simeq \op{Gal}(K_{\pi,\nu}/K)=\Gamma_1\times\cdots\times\Gamma_t \\
		\op{Gal}(L/K) &\simeq \op{Gal}(K_m/K)\times\op{Gal}(K_{\pi,\nu}/K)=\langle\overline\sigma_K\rangle\times\Gamma_1\times\cdots\times\Gamma_t.
	\end{align*}
	In light of these isomorphisms, we will upgrade $\overline\sigma_K$ to the automorphism of $L/K$ which restricts properly on $K_m/K$ and fixing $K_{\pi,\nu}$; we do analogously for the $\tau_i$. We also acknowledge that our degree is
	\[n\coloneqq[L:K]=[K_m:K]\cdot[K_{\pi,\nu}:K]=f\cdot[K_{\pi,\nu}:K].\]
\end{itemize}
For brevity, we will also set $L_i\coloneqq L^{\langle\tau_i\rangle}$ for each $i$, which makes the fields under $L$ look like the following.
% https://q.uiver.app/?q=WzAsNyxbMSwzLCJLIl0sWzIsMiwiS19tIl0sWzAsMSwiTF8wPUtfe1xccGksXFxudX0iXSxbMSwwLCJMIl0sWzIsMSwiTF8xIl0sWzMsMSwiXFxjZG90cyJdLFs0LDEsIkxfdCJdLFswLDEsIiIsMix7InN0eWxlIjp7ImhlYWQiOnsibmFtZSI6Im5vbmUifX19XSxbMiwzLCIiLDAseyJzdHlsZSI6eyJoZWFkIjp7Im5hbWUiOiJub25lIn19fV0sWzAsMiwiIiwwLHsic3R5bGUiOnsiaGVhZCI6eyJuYW1lIjoibm9uZSJ9fX1dLFs0LDMsIiIsMix7InN0eWxlIjp7ImhlYWQiOnsibmFtZSI6Im5vbmUifX19XSxbMSw0LCIiLDIseyJzdHlsZSI6eyJoZWFkIjp7Im5hbWUiOiJub25lIn19fV0sWzEsNiwiIiwyLHsic3R5bGUiOnsiaGVhZCI6eyJuYW1lIjoibm9uZSJ9fX1dLFs2LDMsIiIsMix7InN0eWxlIjp7ImhlYWQiOnsibmFtZSI6Im5vbmUifX19XV0=&macro_url=https%3A%2F%2Fraw.githubusercontent.com%2FdFoiler%2Fnotes%2Fmaster%2Fnir.tex
\[\begin{tikzcd}
	& L \\
	{K_{\pi,\nu}} && {L_1} & \cdots & {L_t} \\
	&& {K_m} \\
	& K
	\arrow[no head, from=4-2, to=3-3]
	\arrow[no head, from=2-1, to=1-2]
	\arrow[no head, from=4-2, to=2-1]
	\arrow[no head, from=2-3, to=1-2]
	\arrow[no head, from=3-3, to=2-3]
	\arrow[no head, from=3-3, to=2-5]
	\arrow[no head, from=2-5, to=1-2]
\end{tikzcd}\]
In particular, $\op{Gal}(L/L_i)=\langle\tau_i\rangle$ is cyclic for each $i$.

Now, the main idea in the computation is to use an unramified extension $M\coloneqq K_n$ of the same degree $n$ as $L/K$. This modifies our diagram of fields as follows.
% https://q.uiver.app/?q=WzAsNixbMSwzLCJLIl0sWzAsMiwiS197XFxwaSxcXG51fSJdLFsxLDAsIk1MIl0sWzAsMSwiTCJdLFsxLDIsIktfbSJdLFsyLDEsIk0iXSxbMCwxLCJcXHRleHR7cmFtfSIsMCx7InN0eWxlIjp7ImhlYWQiOnsibmFtZSI6Im5vbmUifX19XSxbMSwzLCJcXHRleHR7dW5yfSIsMCx7InN0eWxlIjp7ImhlYWQiOnsibmFtZSI6Im5vbmUifX19XSxbMCw0LCJcXHRleHR7dW5yfSIsMix7InN0eWxlIjp7ImhlYWQiOnsibmFtZSI6Im5vbmUifX19XSxbNCw1LCJcXHRleHR7dW5yfSIsMix7InN0eWxlIjp7ImhlYWQiOnsibmFtZSI6Im5vbmUifX19XSxbMywyLCJcXHRleHR7dW5yfSIsMCx7InN0eWxlIjp7ImhlYWQiOnsibmFtZSI6Im5vbmUifX19XSxbNSwyLCJcXHRleHR7cmFtfSIsMix7InN0eWxlIjp7ImhlYWQiOnsibmFtZSI6Im5vbmUifX19XSxbNCwzLCJcXHRleHR7cmFtfSIsMix7InN0eWxlIjp7ImhlYWQiOnsibmFtZSI6Im5vbmUifX19XV0=&macro_url=https%3A%2F%2Fraw.githubusercontent.com%2FdFoiler%2Fnotes%2Fmaster%2Fnir.tex
\[\begin{tikzcd}
	& ML \\
	L && M \\
	{K_{\pi,\nu}} & {K_m} \\
	& K
	\arrow["{\text{ram}}", no head, from=4-2, to=3-1]
	\arrow["{\text{unr}}", no head, from=3-1, to=2-1]
	\arrow["{\text{unr}}"', no head, from=4-2, to=3-2]
	\arrow["{\text{unr}}"', no head, from=3-2, to=2-3]
	\arrow["{\text{unr}}", no head, from=2-1, to=1-2]
	\arrow["{\text{ram}}"', no head, from=2-3, to=1-2]
	\arrow["{\text{ram}}"', no head, from=3-2, to=2-1]
\end{tikzcd}\]
We have labeled the unramified extensions by ``$\textrm{unr}$'' and the totally ramified extensions by ``$\textrm{ram}$.''

% For brevity, we set $K\coloneqq\QQ_p$ and $L\coloneqq\QQ_p(\zeta_N)$ and $M\coloneqq\QQ_p(\zeta_{N'})$ so that $ML=\QQ_p(\zeta_N,\zeta_{N'})$. This abbreviates our diagram into the following.
% % https://q.uiver.app/?q=WzAsNCxbMSwzLCJLIl0sWzAsMSwiTCJdLFsyLDEsIk0iXSxbMSwwLCJNTCJdLFsxLDMsIiIsMCx7InN0eWxlIjp7ImhlYWQiOnsibmFtZSI6Im5vbmUifX19XSxbMCwxLCIiLDAseyJzdHlsZSI6eyJoZWFkIjp7Im5hbWUiOiJub25lIn19fV0sWzAsMiwiIiwyLHsic3R5bGUiOnsiaGVhZCI6eyJuYW1lIjoibm9uZSJ9fX1dLFsyLDMsIiIsMix7InN0eWxlIjp7ImhlYWQiOnsibmFtZSI6Im5vbmUifX19XV0=&macro_url=https%3A%2F%2Fraw.githubusercontent.com%2FdFoiler%2Fnotes%2Fmaster%2Fnir.tex
% \[\begin{tikzcd}
% 	& ML \\
% 	L && M \\
% 	\\
% 	& K
% 	\arrow[no head, from=2-1, to=1-2]
% 	\arrow[no head, from=4-2, to=2-1]
% 	\arrow[no head, from=4-2, to=2-3]
% 	\arrow[no head, from=2-3, to=1-2]
% \end{tikzcd}\]
As before, we provide some comments on the field extensions.
\begin{itemize}
	\item The extension $M/K$ is unramified of degree $n$. As before, its Galois group is cyclic, generated by the Frobenius element $\sigma_K\in\op{Gal}(M/K)$. Observe that $\sigma_K$ restricted to $K_m$ is $\overline\sigma_K$, explaining our notation. In particular, $\sigma_K$ has order $n$, but $\overline\sigma_K$ has order $f<n$.
	\item As before, note that $K_{\pi,\nu}$ and $M$ are linearly disjoint over $K$ because $K_{\pi,\nu}/K$ is totally ramified while $M/K$ is unramified. As such, we may say that
	\begin{align*}
		\op{Gal}(ML/M) &\simeq \op{Gal}(K_{\pi,\nu}/K) = \Gamma_1\times\cdots\times\Gamma_t \\
		\op{Gal}(ML/K_{\pi,\nu}) &\simeq \op{Gal}(M/K) = \langle\sigma_K\rangle \\
		\op{Gal}(ML/K) &\simeq \op{Gal}(M/K)\times\op{Gal}(K_{\pi,\nu}/K) = \langle\sigma_K\rangle\times\Gamma_1\times\cdots\times\Gamma_t.
	\end{align*}
	Again, we will upgrade $\sigma_K$ and the $\tau_i$ to their corresponding automorphisms on any subfield of $ML$.
	\item We take a moment to compute
	\begin{align*}
		\op{Gal}(ML/L) &\simeq \left\{\sigma_K^{a}\tau\in\op{Gal}(ML/K):\sigma_K^{a}\tau|_L=\id_L\right\}.
	\end{align*}
	Because $L$ is $K_{\pi,\nu}K_m$, it suffices to fix each of these fields individually. Well, to fix $K_{\pi,\nu}$, we need $\tau$ to vanish, so we might as well force $\tau=1$. But to fix $K_m$, we need $\sigma_K^{a}|_{K_m}=\overline\sigma_K^{a}$ to be the identity, so we are actually requiring that $f\mid a$ here. As such,
	\[\op{Gal}(ML/L)=\langle\sigma_K^f\rangle.\]
\end{itemize}
These comments complete the Galois-theoretic portion of the analysis.

\subsection{Idea}
We will begin by briefly describe the outline for the computation. For a finite extension of local fields $L/K$, let $u_{L/K}\in H^2(L/K)$ denote the fundamental class.

Now, take variables as in our set-up in \autoref{sec:setup}. The main idea is to translate what we know about the unramified extension $M/K$ over to the general extension $L/K$. In particular, we are able to compute the fundamental class $u_{M/K}\in H^2(M/K)$, so we observe that, by \autoref{prop:functorialfundclass},
\[\op{Inf}_{M/K}^{ML/K}u_{M/K}=[ML:M]u_{M/K}=n\cdot u_{ML/K}=[ML:L]u_{ML/L}=\op{Inf}_{L/K}^{ML/K}u_{L/K}.\]
As such, we will be able to compute $u_{L/K}$ as long as we are able to invert the inflation map $\op{Inf}\colon H^2(L/K)\to H^2(ML/K)$. This is not actually very easy to do in general, but we are in luck because this inflation map here comes from the Inflation--Restriction exact sequence
\[0\to H^2(L/K)\stackrel{\op{Inf}}\to H^2(ML/K)\stackrel{\op{Res}}\to H^2(ML/L).\]
The argument for the Inflation--Restriction exact sequence is an explicit computation on cocycles (involving some dimension shifting), but it can be tracked backwards to give the desired cocycle.