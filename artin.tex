\documentclass{article}
\usepackage[utf8]{inputenc}

\newcommand{\nirpdftitle}{An Explicit Artin Map}
\usepackage{import}
\inputfrom{../notes}{nir}

\pagestyle{contentpage}

\title{An Explicit Artin Map}
\author{Nir Elber}
\date{\today}
\usepackage{graphicx}

\begin{document}

\maketitle

\begin{abstract}
	We track through the Artin map, given the local fundamental class $u\in H^2\left(\op{Gal}(L/K),L^\times\right)$, via the cup product.
\end{abstract}

\section{Via Tate's Theorem}
It is possible to simply track through the proof of Tate's theorem by hand in order to recover the result. The point is that we end up needing to compute
\[\op N_\Gamma(T_\sigma)=\sum_{g\in\Gamma}gT_\sigma.\]
However, the action is given by
\[gT_\sigma=T_{g\sigma}-T_g+u(g,\sigma).\]
Thus, we get
\[\prod_{g\in G}u(g,\sigma)\]
at the end of the computation.

\section{Via the Cup Product}
Fix a Galois extension of local fields $L/K$, with Galois group $\Gamma\coloneqq\op{Gal}(L/K)$. To begin, we will simply write down the Artin map as the long string of isomorphisms
\[\Gamma^{\op{ab}}\simeq I_\Gamma/I_\Gamma^2=H_0(\Gamma,I_\Gamma)\stackrel\delta\leftarrow H_1\left(\Gamma,\ZZ\right)=\widehat H^{-2}\left(\Gamma,\ZZ\right)\stackrel{u_{L/K}\cup-}\to\widehat H^0\left(\Gamma,L^\times\right)=K^\times/\op N^L_K\left(L^\times\right).\]
We will explain each of these objects as they come up. For now, let's move from the left to right, describing the morphisms one at a time; please note that we will not show that all the morphisms are in fact isomorphisms. Pick up some $\sigma\in\Gamma$.
\begin{enumerate}
	\item In $\Gamma^{\op{ab}}$, we have the coset $[\sigma]\in\Gamma^{\op{ab}}$.

	\item The isomorphism $\Gamma^{\op{ab}}\simeq I_\Gamma/I_\Gamma^2$ takes $[\sigma]$ to $[\sigma-1]_{I_\Gamma^2}\in I_\Gamma/I_\Gamma^2$.

	\item It is not completely obvious how to make group homology into something that we can write down; we will use homology by free resolutions. In particular, for future reference, we will set $P_n\coloneqq\ZZ\left[\Gamma^{n+1}\right]$ and define $d_n\colon P_n\to P_{n+1}$ by
	\[d_n(g_0,\ldots,g_n)=\sum_{i=0}^n(-1)^i(g_0,\ldots,g_{i-1},g_{i+1},\ldots,g_n).\]
	Further, we define $\varepsilon\colon P_0\to\ZZ$ by sending $\varepsilon\colon g\mapsto1$ for all $g\in\Gamma$. Then we have a free $\ZZ[\Gamma]$-module resolution of $\ZZ$ by
	\[\cdots\to P_2\stackrel{d_2}\to P_1\stackrel{d_1}\to P_0\stackrel\varepsilon\to\ZZ\to0.\]
	Then, given a $\Gamma$-module $A$, it happens that we can compute group homology as the homology groups of the resolution $P_\bullet\otimes_{\ZZ[\Gamma]} A$; more precisely,
	\[H_q(\Gamma,A)=H_q(P_\bullet\otimes_{\ZZ[\Gamma]} A)=\frac{\ker(d_q\otimes A)}{\im(d_{q+1}\otimes A)},\]
	where $d_0\colon P_0\to0$ is the zero map; for later use, we set $C_q(\Gamma,A)\coloneqq P_q\otimes_{\ZZ[\Gamma]}A$ to be $q$-chains, and $Z_q(\Gamma,A)\coloneqq\ker(d_q\otimes A)$ to be the $q$-cycles, and $B_q(\Gamma,A)\coloneqq\im(d_{q+1}\otimes A)$ to be the $q$-boundaries so that $H_q(\Gamma,A)=Z_q(\Gamma,A)/B_q(\Gamma,A)$. In particular, the natural association from $I_\Gamma/I_\Gamma^2$ to $H_0(\Gamma,I_\Gamma)$ sends $[\sigma-1]_{I_\Gamma^2}$ to the $0$-cycle $[1\otimes(\sigma-1)]\in H_0(\Gamma,I_\Gamma)$.
	\item Our next step is to track through connecting morphism $\delta$, which is induced by the short exact sequence
	\[0\to I_\Gamma\to\ZZ[\Gamma]\to\ZZ\to0.\]
	In particular, $\delta$ is induced by the Snake lemma from the following snake diagram.
	% https://q.uiver.app/?q=WzAsOCxbMSwwLCJcXGRpc3BsYXlzdHlsZVxcZnJhY3tDXzAoXFxHYW1tYSxJX1xcR2FtbWEpfXtCXzAoXFxHYW1tYSxJX1xcR2FtbWEpfSJdLFsxLDEsIlpfMShcXEdhbW1hLElfXFxHYW1tYSkiXSxbMiwwLCJcXGRpc3BsYXlzdHlsZVxcZnJhY3tDXzAoXFxHYW1tYSxcXFpaW1xcR2FtbWFdKX17Ql8wKFxcR2FtbWEsXFxaWltcXEdhbW1hXSl9Il0sWzMsMCwiXFxkaXNwbGF5c3R5bGVcXGZyYWN7Q18wKFxcR2FtbWEsXFxaWil9e0JfMChcXEdhbW1hLFxcWlopfSJdLFsyLDEsIlpfMShcXEdhbW1hLFxcWlpbXFxHYW1tYV0pIl0sWzMsMSwiWl8xKFxcWlopIl0sWzQsMCwiMCJdLFswLDEsIjAiXSxbMCwyXSxbMiwzXSxbMyw2XSxbNywxXSxbMSw0XSxbNCw1XSxbMCwxLCJkXzAiXSxbMiw0LCJkXzAiXSxbMyw1LCJkXzAiXV0=&macro_url=https%3A%2F%2Fraw.githubusercontent.com%2FdFoiler%2Fnotes%2Fmaster%2Fnir.tex
	\[\begin{tikzcd}
		& {\displaystyle\frac{C_1(\Gamma,I_\Gamma)}{B_1(\Gamma,I_\Gamma)}} & {\displaystyle\frac{C_1(\Gamma,\ZZ[\Gamma])}{B_1(\Gamma,\ZZ[\Gamma])}} & {\displaystyle\frac{C_1(\Gamma,\ZZ)}{B_1(\Gamma,\ZZ)}} & 0 \\
		0 & {Z_0(\Gamma,I_\Gamma)} & {Z_0(\Gamma,\ZZ[\Gamma])} & {Z_0(\ZZ)}
		\arrow[from=1-2, to=1-3]
		\arrow[from=1-3, to=1-4]
		\arrow[from=1-4, to=1-5]
		\arrow[from=2-1, to=2-2]
		\arrow[from=2-2, to=2-3]
		\arrow[from=2-3, to=2-4]
		\arrow["{d_1}", from=1-2, to=2-2]
		\arrow["{d_1}", from=1-3, to=2-3]
		\arrow["{d_1}", from=1-4, to=2-4]
	\end{tikzcd}\]
	We begin by tracking through a generic $1$-cycle
	\[\sum_{x\in\ZZ[\Gamma]}x\otimes a_x\in Z_1(\Gamma,\ZZ),\]
	where the $a_x$ are all integers. Note that we can always expand out $x\in P_1=\ZZ\left[\Gamma^2\right]$ into a $\ZZ$-linear sum over terms of the form $(g,h)\in\Gamma^2$, so we might as well write our generic $1$-cycle as
	\[\sum_{g,h\in\Gamma}(g,h)\otimes a_{g,h}\in Z_1(\Gamma,\ZZ),\]
	where the $a_{g,h}$ are all integers. However, observe that
	\[(g,h)\otimes a_{g,h}=\left(1,hg^{-1}\right)g\otimes a_{g,h}=\left(1,hg^{-1}\right)\otimes ga_{g,h}=\left(1,hg^{-1}\right)\otimes a_{g,h}.\]
	Here, the last equality is because $a_{g,h}\in\ZZ$ has trivial $\Gamma$-action. Anyway, we might as well assume that our generic $1$-cycle looks like
	\[\sum_{g\in\Gamma}(1,g)\otimes a_g\in Z^1(\Gamma,\ZZ).\]
	% Only now do we say that $1$-cycle condition is requiring that
	% \[(d_1\otimes\id_\ZZ)\left(\sum_{g\in\gamma}(1,g)\otimes a_g\right)=\sum_{g\in\Gamma}d_1(1,g)\otimes a_g=\sum_{g\in\Gamma}(g-1)\otimes a_g=\sum_{g\in\Gamma}1\otimes(g-1)a_g=1\otimes\sum_{g\in\Gamma}a_g\]
	% to vanish; i.e., we need $\sum_{g\in\Gamma}a_g=0$.

	Anyway, we now track through our $1$-cycle through $\delta$. We have the following steps.
	\begin{itemize}
		\item Observe that we can lift $\sum_{g\in\Gamma}(1,g)\otimes a_g$ from $Z_1(\Gamma,\ZZ)$ to $C_1(\Gamma,\ZZ[\Gamma])$ just by moving the integers $a_g\in\ZZ$ to $a_g\in\ZZ[\Gamma]$. Observe that we no longer necessarily have a $1$-cycle because the $\Gamma$-action on the $a_g$ is no longer trivial.
		\item Next, we push our $1$-chain through $d_1$, which now looks like
		\[(d_1\otimes\id_{\ZZ[\Gamma]})\left(\sum_{g\in\gamma}(1,g)\otimes a_g\right)=\sum_{g\in\Gamma}d_1(1,g)\otimes a_g=\sum_{g\in\Gamma}(g-1)\otimes a_g=\sum_{g\in\Gamma}1\otimes(g-1)a_g.\]
		Observe that $(g-1)a_g\in I_\Gamma$, so this is a valid $0$-chain in $C_0(\Gamma,I_\Gamma)$. In particular, we have found what $\delta$ does to our $1$-cycles.
	\end{itemize}
	Only now do we observe that the $1$-cycle $[(1,\sigma)\otimes1]\in H_1(\Gamma,\ZZ)$ goes to the needed element $(\sigma-1)\otimes1\in H_0(\Gamma,I_\Gamma)$, so it is our element.

	\item To be able to apply the cup product to our Tate cohomology, we will need to put our $1$-cycle $(1,\sigma)\otimes1$ into a complete free resolution of $\ZZ$ to unify homology and cohomology into Tate cohomology. For this, we note that there is a (contravariant) functor $(\cdot)^*\colon\mathrm{Mod}_G\opp\to\mathrm{Mod}_G$ taking some $G$-module $A$ to the $G$-module
	\[A^*\coloneqq\op{Hom}_\ZZ(A,\ZZ).\]
	The functor also takes morphisms $f\colon A\to B$ to the morphism $f^*\colon B^*\to A^*$ defined by $f^*(g)\colon x\mapsto g(f(x))$. In particular, our free resolution
	\[\cdots\to P_2\stackrel{d_2}\to P_1\stackrel{d_1}\to P_0\stackrel\varepsilon\to\ZZ\to0\]
	becomes
	\[0\to\ZZ\stackrel{\varepsilon^*}\to P_0^*\stackrel{d_1^*}\to P_1\stackrel{d_2^*}\to P_2\to\cdots.\]
	Note that this second sequence is still long exact because the $\op{Hom}$ functor is exact on free objects. We now set $P_{-n}\coloneqq P_{n-1}^*$ and $d_0\coloneqq\varepsilon^*\circ\varepsilon$ and $d_{-n}\coloneqq d_n^*$ for $n<0$ to attach our two long exact sequences to get a very long exact sequence
	\[\cdots\to P_2\stackrel{d_2}\to P_1\stackrel{d_1}\to P_0\stackrel{d_0}\to P_{-1}\stackrel{d_{-1}}\to P_{-2}\stackrel{d_{-2}}\to P_{-3}\to\cdots.\]
	Then one can check that the Tate cohomology groups $\widehat H^q(\Gamma,A)$ arise as
	\[\widehat H^q(\Gamma,A)=H^q(\op{Hom}_\Gamma(P_\bullet,A))=\frac{\ker(\op{Hom}_\Gamma(d_q,A))}{\im(\op{Hom}_\Gamma(d_{q-1},A))}.\]
	For example, if $q\ge1$, then this is simply group cohomology via homogeneous cochains.

	It requires some work to make sense of this construction to match the usual notions of Tate cohomology. We will only be interested in the case of $q=-2$ and $q=0$. In the case of $q=0$, it happens that the element of $\ker(\op{Hom}_G(d_0,A))$ we care about should just be a constant map.

	Discussing $q=-2$ needs a little more work. We remark that the below approach will in fact work for all $q\le-2$, though we will not track all of this through. We will describe how $P_1\otimes_{\ZZ[\Gamma]}A\simeq\op{Hom}_\Gamma(P_{-2},A)$, and this isomorphism will be natural enough for our purposes (namely, correctly tracking through the differential maps $d$ as well). To begin, we note that we have a $\Gamma$-module isomorphism $\sigma\colon P_1\otimes_{\ZZ[\Gamma]}A\to\op{Hom}_\ZZ(P_1^*,A)$ by
	\[\sigma(p\otimes a)\colon(f\mapsto f(p)a),\]
	where $f\in P_1^*$. Then our isomorphism is
	\[P_1\otimes_{\ZZ[\Gamma]}A\simeq(P_1\otimes_\ZZ A)_\Gamma\stackrel{\op N_\Gamma}\to(P_1\otimes_\ZZ A)^\Gamma\stackrel\sigma\to\op{Hom}_\ZZ(P_1^*,A)^\Gamma=\op{Hom}_\Gamma(P_1^*,A).\]
	Observe $\op N_\Gamma$ is an isomorphism because $P_1\otimes_\ZZ A$ is co-induced and hence has trivial $\widehat H^0$ and $\widehat H^1$ terms. Thus, we track through our $1$-cycle $[(1,\sigma)\otimes1]\in H_1(\Gamma,\ZZ)$ as follows.
	\begin{itemize}
		\item The isomorphism $P_1\otimes_{\ZZ[\Gamma]}A\simeq(P_1\otimes_\ZZ A)_\Gamma$ preserves $(1,\sigma)\otimes1$.
		\item The norm map sends $(1,\sigma)\otimes1$ to
		\[\op N_\Gamma((1,\sigma)\otimes1)=\sum_{g\in\Gamma}g(1,\sigma)\otimes g1=\sum_{g\in\Gamma}(g,g\sigma)\otimes1.\]
		\item Next, $\sigma$ sends us to a map
		\[f\mapsto\sum_{g\in\Gamma}f(g,g\sigma).\]
		Let this map be $c_\sigma\in\op{Hom}_\Gamma(P_1^*,A)$.
	\end{itemize}
	This last point provides us with our representative in $\widehat H^{-2}(\Gamma,\ZZ)$.

	\item We are now ready to compute the cup product. We will let $u$ represent the local fundamental class $u_{L/K}\in H^2\left(\Gamma,L^\times\right)$ as an inhomogeneous $2$-cocycle, and we will let $\overline u$ be the corresponding homogeneous $2$-cocycle.

	To finish the computation, we need to compute $([\overline u]\cup[c_\sigma])\in\widehat H^0\left(\Gamma,L^\times\right)$. In this case, $c_\sigma\in\widehat H^{-2}\left(\Gamma,\ZZ\right)$ and $[\overline u]=u_{L/K}\in\widehat H^2\left(\Gamma,L^\times\right)$, so we can look up the formula for the cup product.
	
	Given $s_1,s_2\in\Gamma$, let $(s_1^*,s_2^*)\colon P_1\to\ZZ$ be defined by $(s_1,s_2)(g,h)=1_{g=s_1,h=s_2}$; these elements form a basis of $P_1^*$. As such, our cup product is computed as
	\[([\overline u]\cup[c_\sigma])(g_0)=\sum_{s_1,s_2\in\Gamma}u(g_0,s_1,s_2)\otimes c_\sigma(s_2^*,s_1^*).\]
	Now, we compute that
	\[c_\sigma(s_2^*,s_1^*)=\sum_{g\in\Gamma}(s_2^*,s_1^*)(g,g\sigma)=\sum_{g\in\Gamma}1_{s_2=g,s_1=g\sigma}.\]
	The only possible way for a term in the sum to be nonzero is for $g=s_2$, in which case we are asking for $s_1=s_2\sigma$. So in fact, we have $c_\sigma(s_2^*,s_1^*)=1_{s_1=s_2\sigma}$. Removing the nonzero terms from $([\overline u]\cup[c_\sigma])(g_0)$, we are left with
	\[([\overline u]\cup[c_\sigma])(g_0)=\sum_{g\in\Gamma}u(g_0,g\sigma,g)\otimes1.\]
	This actually comes out to the inverse map.\todo{Fix this}
\end{enumerate}

\end{document}