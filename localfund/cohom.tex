The results and commentary here mirror \cite[Section~2]{explicit-fund-classes}. Throughout this section, $G$ will be a group (usually finite) and $H\subseteq G$ will be a subgroup (usually normal).

We begin by recalling the statement of the Inflation--Restriction exact sequence; we will provide the proof for completeness because we will use the proof for our computation.
\begin{theorem}[{\cite[Proposition~5]{atiyah-wall}}] \label{thm:infres}
	Let $G$ be a finite group with normal subgroup $H\subseteq G$. Given a $G$-module $A$, suppose that the $H^i(H,A)=0$ for $1\le i<q$ for some index $q\ge1$. Then the sequence
	\[0\to H^q\left(G/H,A^H\right)\stackrel{\op{Inf}}\to H^q(G,A)\stackrel{\op{Res}}\to H^q(H,A)\]
	is exact.
\end{theorem}
\begin{proof}
	The proof is by induction on $q$, via dimension shifting. For $q=1$, we can just directly check this on $1$-cocycles. The main point is the exactness at $H^q(G,A)$: if $c\in Z^1(G,A)$ has $\op{Res}(c)\in B^1(H,A)$, then find $a\in A$ with
	\[\op{Res}(c)(a)\coloneqq h\cdot a-a.\]
	As such, we define $f_a\in B^1(G,A)$ by $f_a(g)\coloneqq g\cdot a-a$, which implies that $c-f_a$ vanishes on $H$. It is then possible to stare at the $1$-cocycle condition
	\[(c-f_a)(gg')=(c-f_a)(g)+g\cdot(c-f_a)(g')\]
	to check that $c-f_a$ only depends on the cosets of $H$ (e.g., by taking $g'\in H$) and that $\im(c-f_a)\subseteq A^H$ (e.g., by taking $g\in H$).

	For $q>1$, we use dimension shifting via the following lemma. Indeed, suppose the statement is true for $q$. Then the short exact sequence
	\[0\to A\to\op{Hom}_\ZZ(\ZZ[G],A)\to\op{Hom}_\ZZ(I_G,A)\to0\]
	induces vertical isomorphisms in the following commutative diagram.
	% \begin{lemma}[Dimension shifting] \label{lem:dimshift}
	% 	Let $G$ be a group with subgroup $H\subseteq G$. Given a $G$-module $A$, all indices $q\ge1$ have
	% 	\[\delta\colon H^q(H,\op{Hom}_\ZZ(I_G,A))\simeq H^{q+1}(H,A).\]
	% \end{lemma}
	% \begin{proof}
	% 	Recall that we have the short exact sequence of $\ZZ[H]$-modules
	% 	\[0\to I_G\to\ZZ[G]\to\ZZ\to0.\]
	% 	In fact, this short exact sequence splits over $\ZZ$, so it will still be short exact after applying $\op{Hom}_\ZZ(-,A)$, which gives the short exact sequence
	% 	\[0\to A\to\op{Hom}_\ZZ(\ZZ[G],A)\to\op{Hom}_\ZZ(I_G,A)\to0\]
	% 	of $\ZZ[H]$-modules. The result now follows from the long exact sequence of cohomology upon noting that $\op{Hom}_\ZZ(\ZZ[G],A)$ is coinduced and hence acyclic for cohomology.
	% \end{proof}
	% Using the above lemma, we have the following the commutative diagram with vertical arrows which are isomorphisms.
	% https://q.uiver.app/?q=WzAsOCxbMCwwLCIwIl0sWzAsMSwiMCJdLFsxLDAsIkhecVxcbGVmdChHL0gsXFxvcHtIb219X1xcWlooSV9HLEEpXkhcXHJpZ2h0KSJdLFsxLDEsIkhee3ErMX1cXGxlZnQoRy9ILEFeSFxccmlnaHQpIl0sWzIsMSwiSF57cSsxfShHLEEpIl0sWzIsMCwiSF5xKEcsXFxvcHtIb219X1xcWlooSV9HLEEpKSJdLFszLDEsIkhee3ErMX0oSCxBKSJdLFszLDAsIkhecShILFxcb3B7SG9tfV9cXFpaKElfRyxBKSkiXSxbMSwzXSxbMyw0XSxbNCw2XSxbMiwzLCJcXGRlbHRhIl0sWzUsNCwiXFxkZWx0YSJdLFs3LDYsIlxcZGVsdGEiXSxbNSw3XSxbMiw1XSxbMCwyXV0=&macro_url=https%3A%2F%2Fraw.githubusercontent.com%2FdFoiler%2Fnotes%2Fmaster%2Fnir.tex
	\[\begin{tikzcd}
		0 & {H^q\left(G/H,\op{Hom}_\ZZ(I_G,A)^H\right)} & {H^q(G,\op{Hom}_\ZZ(I_G,A))} & {H^q(H,\op{Hom}_\ZZ(I_G,A))} \\
		0 & {H^{q+1}\left(G/H,A^H\right)} & {H^{q+1}(G,A)} & {H^{q+1}(H,A)}
		\arrow[from=2-1, to=2-2]
		\arrow[from=2-2, to=2-3]
		\arrow[from=2-3, to=2-4]
		\arrow["\delta", from=1-2, to=2-2]
		\arrow["\delta", from=1-3, to=2-3]
		\arrow["\delta", from=1-4, to=2-4]
		\arrow[from=1-3, to=1-4]
		\arrow[from=1-2, to=1-3]
		\arrow[from=1-1, to=1-2]
	\end{tikzcd}\]
	The top row is exact by the inductive hypothesis, so the bottom row is therefore also exact.
\end{proof}
Our goal is to make the above proof explicit in the case of $q=2$, which is the only reason we sketched the above proofs at all. We begin by making the dimension shifting explicit.
\begin{lemma}[{\cite[Lemma~2.1]{explicit-fund-classes}}] \label{lem:explicitdimshift}
	Let $G$ be a group with subgroup $H\subseteq G$, and let $\{g_\alpha\}_{\alpha\in\lambda}$ be coset representatives for $H\backslash G$. Now, given a $G$-module $A$, the maps
	\begin{align*}
		\delta_H\colon Z^1(H,\op{Hom}_\ZZ(I_G,A))&\to Z^2(H,A) \\
		c&\mapsto\left[(h,h')\mapsto h\cdot c(h')(h^{-1}-1)\right] \\
		\left[h\mapsto\big((h'g_\bullet-1)\mapsto h'\cdot u((h')^{-1},h)\big)\right]&\mapsfrom u
	\end{align*}
	are group homomorphisms descending to the isomorphism $\overline\delta\colon H^1(H,\op{Hom}_\ZZ(I_G,A))\simeq H^2(H,A)$. The map $\delta_H$ above is surjective, and the reverse map is a section; when $H=G$, these are isomorphisms.
\end{lemma}
\begin{proof}
	To show that $\delta_H$ descends to an isomorphism properly, we could track through dimension-shifting by hand, or we can use the machinery we've built. Namely, setting $X=\ZZ$ in \autoref{cor:cupdown} told us that the $1$-cocycle $\chi\in Z^1(G,I_G)$ defined by
	\[\chi(\sigma)\coloneqq(1-\sigma)\]
	provides an isomorphism
	\[(\chi\cup-)\colon\widehat H^i(G,\op{Hom}_\ZZ(I_G,A))\to\widehat H^{i+1}(G,A).\]
	Computing our cup product, we have
	\begin{align*}
		(\chi\cup c)(h,h') &= \big(hc(h')\big)\big(\chi(h)\big) \\
		&= h\cdot c(h')\big(h^{-1}(1-h)\big) \\
		&= h\cdot c(h')\big(h^{-1}-1\big).
	\end{align*}
	So we see that $\delta_H=(\chi\cup-)$ on cocycles and therefore descends to the needed isomorphism. Additionally, it is a homomorphism by properties of the cup product.
	% We begin by noting that our short exact sequence can be written more explicitly as follows.
	% % https://q.uiver.app/?q=WzAsOSxbMCwwLCIwIl0sWzEsMCwiQSJdLFsyLDAsIlxcb3B7SG9tfV9cXFpaKFxcWlpbR10sQSkiXSxbMywwLCJcXG9we0hvbX1fXFxaWihJX0csQSkiXSxbNCwwLCIwIl0sWzEsMSwiYSJdLFsyLDEsImFcXG1hcHN0byh6XFxtYXBzdG9cXHZhcmVwc2lsb24oeilhKSJdLFsyLDIsImYiXSxbMywyLCJmfF97SV9HfSJdLFswLDFdLFsxLDJdLFsyLDNdLFszLDRdLFs1LDYsIiIsMCx7InN0eWxlIjp7InRhaWwiOnsibmFtZSI6Im1hcHMgdG8ifX19XSxbNyw4LCIiLDAseyJzdHlsZSI6eyJ0YWlsIjp7Im5hbWUiOiJtYXBzIHRvIn19fV1d&macro_url=https%3A%2F%2Fraw.githubusercontent.com%2FdFoiler%2Fnotes%2Fmaster%2Fnir.tex
	% \[\begin{tikzcd}[row sep=0.0em]
	% 	0 & A & {\op{Hom}_\ZZ(\ZZ[G],A)} & {\op{Hom}_\ZZ(I_G,A)} & 0 \\
	% 	& a & {(z\mapsto\varepsilon(z)a)} \\
	% 	&& f & {f|_{I_G}}
	% 	\arrow[from=1-1, to=1-2]
	% 	\arrow[from=1-2, to=1-3]
	% 	\arrow[from=1-3, to=1-4]
	% 	\arrow[from=1-4, to=1-5]
	% 	\arrow[maps to, from=2-2, to=2-3]
	% 	\arrow[maps to, from=3-3, to=3-4]
	% \end{tikzcd}\]
	% We now track through the induced boundary morphism $\delta\colon H^1(H,\op{Hom}_\ZZ(I_G,A))\to H^2(H,Q)$.
	% \begin{itemize}
	% 	\item We begin with $c\in Z^1(H,\op{Hom}_\ZZ(I_G,A))$, which means that we have $c(h)\colon I_G\to A$ for each $h,h'\in H$, and we satisfy
	% 	\[c(hh')=c(h)+h\cdot c(h').\]
	% 	Tracking through the action of $H$ on $\op{Hom}_\ZZ(I_G,A)$, this means that
	% 	\[c(hh')(g-1)=c(h)(g-1)+h\cdot c(h')(h^{-1}g-h^{-1})\]
	% 	for any $g\in G$.
	% 	\item To pull $c$ back to $C^1(H,\op{Hom}_\ZZ(\ZZ[G],A))$, we need to lift $c(h)\colon I_G\to A$ to a $\widetilde c(h)\colon\ZZ[G]\to A$. Recalling that we only need to preserve group structure, we simply precompose $c(h)$ with the map $\ZZ[G]\to I_G$ given by $z\mapsto z-\varepsilon(z)$. That is, we define
	% 	\[\widetilde c(h)(z)\coloneqq c(h)(z-\varepsilon(z)).\]
	% 	\item We now push $\widetilde c$ through $d\colon C^1(H,\op{Hom}_\ZZ(\ZZ[G],A))\to Z^2(H,\op{Hom}_\ZZ(\ZZ[G],A))$. This gives
	% 	\[(d\widetilde c)(h,h')=g\widetilde c(h')-\widetilde c(hh')+\widetilde c(h)\]
	% 	for any $h,h'\in H$. Concretely, plugging in some $z\in\ZZ[G]$ makes this look like
	% 	\begin{align*}
	% 		(d\widetilde{c})(h,h')(z) &= (h\widetilde c(h'))(z)-\widetilde c(hh')(z)+\widetilde c(h)(z) \\
	% 		&= h\cdot c(h')\left(h^{-1}z-\varepsilon(h^{-1}z)\right)-c(hh')(z-\varepsilon(z))+c(h)(z-\varepsilon(z)) \\
	% 		&= h\cdot c(h')\left(h^{-1}z-\varepsilon(z)\right)-c(hh')(z-\varepsilon(z))+c(h)(z-\varepsilon(z)).
	% 	\end{align*}
	% 	Now, from the $1$-cocycle condition on $c$, we recall
	% 	\[-c(hh')(z-\varepsilon(z))+c(h)(z-\varepsilon(z))=-h\cdot(c(h')(h^{-1}z-\varepsilon(z)h^{-1})),\]
	% 	so
	% 	\begin{align*}
	% 		(d\widetilde{c})(h,h')(z) &= h\cdot c(h')\left(\varepsilon(z)h^{-1}-\varepsilon(z)\right) \\
	% 		&= \varepsilon(z)\cdot\left(h\cdot c(h')\left(h^{-1}-1\right)\right).
	% 	\end{align*}
	% 	In particular, we see that $d\widetilde c\in Z^2(H,\op{Hom}_\ZZ(\ZZ[G],A))$ pulls back to $(h,h')\mapsto h\cdot c(h')\left(h^{-1}-1\right)$ in $Z^2(H,A)$. It is not too difficult to check that we have in fact defined a $2$-cocycle, but we will not do so because it is not necessary for the proof.
	% \end{itemize}
	% Now, we do know that $\delta_H$ is a homomorphism abstractly on elements of our cohomology classes by the Snake lemma, but it is also not too hard to see that
	% \[\delta_H\colon Z^1(H,\op{Hom}_\ZZ(I_G,A))\to Z^2(H,A)\]
	% is in fact a homomorphism of groups directly from the construction. In short,
	% \[\delta_H(c+c')(h,h')=h'\cdot c(h)\left(h^{-1}-1\right)+h'\cdot c'(h)\left(h^{-1}-1\right)=(\delta_H(c)+\delta_H(c'))(h,h')\]
	% for any $h,h'\in H$.

	It remains to prove the last sentence. We run the following checks; given $u\in Z^2(H,A)$, define $c_u\in C^1(H,\op{Hom}_\ZZ(I_G,A))$ by
	\[c_u(h)(h'g_\bullet-1)=h'\cdot u\left((h')^{-1},h\right).\]
	Note that this is enough data to define $c_u(h)\colon I_G\to A$ because $I_G$ is a free $\ZZ$-module generated by $\{g-1:g\in G\}$.
	\begin{itemize}
		\item We verify that $c_u$ is a $1$-cocycle. This is a matter of force. Pick up $h,h'\in H$ and $g_\bullet h''\in G$ and write
		\begin{align*}
			&\phantom{{}={}}(hc_u(h'))(h''g_\bullet-1)+c_u(hh')(h''g_\bullet-1)+c_u(h)(h''g_\bullet-1) \\
			&= h\cdot c_u(h')\left(h^{-1}h''g_\bullet -h^{-1}\right)+c_u(hh')(h''g_\bullet-1)+c_u(h)(h''g_\bullet-1) \\
			&= h\cdot\left(h^{-1}h''u\left((h'')^{-1}h,h'\right)-h^{-1}u(h,h')\right)+h''u\left((h'')^{-1},hh'\right)+h''u\left((h'')^{-1},h\right) \\
			&= h''u\left((h'')^{-1}h,h'\right)-u(h,h')+h''u\left((h'')^{-1},hh'\right)+h''u\left((h'')^{-1},h\right).
		\end{align*}
		This is just the $2$-cocycle condition for $u$ upon dividing out by $h''$, so we are done.
		\item For $u\in Z^2(H,A)$, we verify that $\delta_H(c_u)=u$. Indeed, given $h,h'\in H$, we check
		\begin{align*}
			\delta_H(c_u)(h,h') &= h\cdot c_u(h')\left(h^{-1}-1\right) \\
			&= h\cdot h^{-1}\cdot u(h,h') \\
			&= u(h,h').
		\end{align*}
	\end{itemize}
	So far we have verified that $\delta$ has section $u\mapsto c_u$ and hence must be surjective. Lastly, we take $H=G$ and show that $c_{\delta c}=c$ to finish. Indeed, for $g,g'\in G=H$, we write
	\begin{align*}
		c_{\delta_H c}(g)(g'-1) &= g'\cdot(\delta_H c)\left((g')^{-1},g\right) \\
		&= g'(g')^{-1}\cdot c(g)(g'-1) \\
		&= c(g)(g'-1),
	\end{align*}
	which is what we wanted.
\end{proof}
We also have used dimension shifting to show that $H^1\left(G/H,\op{Hom}_\ZZ(I_G,A)^H\right)\to H^2\left(G/H,A^H\right)$ is an isomorphism, but this requires a little more trickery. To begin, we discuss how to lift from $\op{Hom}_\ZZ(I_G,A)^H$ to $\op{Hom}_\ZZ(\ZZ[G],A)^H$.
\begin{lemma} \label{lem:howtolift}
	Let $G$ be a group with subgroup $H\subseteq G$. Fix a $G$-module $A$ with $H^1(H,A)=0$. Then, for any $\psi\in\op{Hom}_\ZZ(I_G,A)^H$, the function $h\mapsto h\psi\left(h^{-1}-1\right)$ is a cocycle in $Z^1(H,A)=B^1(H,A)$, so we can define a function $\eta_\bullet\colon\op{Hom}_\ZZ(I_G,A)^H\to A$ such that
	\[\psi(h-1)=h\cdot \eta_\varphi-\eta_\varphi\]
	for all $h\in H$. In fact, given $\varphi\in\op{Hom}_\ZZ(I_G,A)^H$, we can construct $\widetilde\varphi\in\op{Hom}_\ZZ(\ZZ[G],A)^H$ by
	\[\widetilde\varphi(z)\coloneqq\varphi(z-\varepsilon(z))+\varepsilon(z)\eta_\varphi\]
	so that $\widetilde\varphi|_{I_G}=\varphi$.
\end{lemma}
\begin{proof}
	We will just run the checks directly.
	\begin{itemize}
		\item We start by checking $\psi\in\op{Hom}_\ZZ(I_G,A)^H$ give $1$-cocycles $c(h)\coloneqq \varphi\left(h-1\right)$ in $Z^1(A,H)$. To begin, we note that $\psi\in\op{Hom}_\ZZ(I_G,A)^H$ simply means that any $z-\varepsilon(z)\in I_G$ has
		\[\psi(z-\varepsilon(z))=(h\psi)(z-\varepsilon(z))=h\psi\left(h^{-1}z-h^{-1}\varepsilon(z)\right)\]
		for all $h\in H$. In particular, replacing $h$ with $h^{-1}$ tells us that
		\[h\psi(z-\varepsilon(z))=\psi(hz-h\varepsilon(z)).\]
		Now, we can just compute
		\begin{align*}
			(dc)(h,h') &= hc(h')-c(hh')+c(h) \\
			&= hc\left(h'-1\right)-c\left(hh'-1\right)+c\left(h-1\right) \\
			&= c\left(hh'-h\right)-c\left(hh'-1\right)+c\left(h-1\right),
		\end{align*}
		where in the last equality we used the fact that $\psi\in\op{Hom}_\ZZ(I_G,A)^H$. Now, $(dc)(h,h')$ manifestly vanishes, so we are done.
		\item Note that $\widetilde\varphi\in\op{Hom}_\ZZ(\ZZ[G],A)$ because it is a linear combination of (compositions of) homomorphisms.
		\item Note that any $z\in I_G$ has $\varepsilon(z)=0$, so
		\[\widetilde\varphi(z)=\varphi(z-0)+0\cdot \eta_\varphi=\varphi(z),\]
		so $\widetilde\varphi|_{I_G}=\varphi$.
		\item It remains to check that $\widetilde\varphi$ is fixed by $H$. This requires a little more effort. Recall that $\varphi\in\op{Hom}_\ZZ(I_G,A)^H$ means that any $z-\varepsilon(z)\in I_G$ has
		\[h\varphi(z-\varepsilon(z))=\varphi\left(hz-h\varepsilon(z)\right)\]
		for any $h\in H$. Now, we just compute
		\begin{align*}
			(h\widetilde\varphi)(z) &= h\widetilde\varphi\left(h^{-1}z\right) \\
			&= h\left(\varphi\left(h^{-1}z-\varepsilon(h^{-1}z)\right)+\varepsilon(h^{-1}z)\eta_\varphi\right) \\
			&= \varphi\left(z-h\varepsilon(z)\right)+\varepsilon(z)\cdot h\eta_\varphi \\
			&= \varphi\left(z-h\varepsilon(z)\right)+\varepsilon(z)\varphi(h-1)+\varepsilon(z)\eta_\varphi \\
			&= \varphi(z-\varepsilon(z))+\varepsilon(z)\eta_\varphi \\
			&= \widetilde\varphi(z).
		\end{align*}
	\end{itemize}
	The above checks complete the proof.
\end{proof}
% \begin{remark}
% 	For motivation, the $\widetilde\varphi$ was constructed by tracking through the following diagram.
% 	% https://q.uiver.app/?q=WzAsOCxbMSwwLCJcXGRpc3BsYXlzdHlsZVxcZnJhY3tDXjAoSCxBKX17Ql4wKEgsQSl9Il0sWzEsMSwiWl4xKEgsQSk9Ql4xKEgsQSkiXSxbMiwxLCJaXjEoSCxcXG9we0hvbX1fXFxaWihcXFpaW0ddLEEpKSJdLFszLDEsIlpeMShILFxcb3B7SG9tfV9cXFpaKElfRyxBKSkiXSxbMywwLCJcXGRpc3BsYXlzdHlsZVxcZnJhY3tDXjAoSCxcXG9we0hvbX1fXFxaWihJX0csQSkpfXtCXjAoSCxcXG9we0hvbX1fXFxaWihJX0csQSkpfSJdLFsyLDAsIlxcZGlzcGxheXN0eWxlXFxmcmFje0NeMChILFxcb3B7SG9tfV9cXFpaKFxcWlpbR10sQSkpfXtCXjAoSCxcXG9we0hvbX1fXFxaWihcXFpaW0ddLEEpKX0iXSxbNCwwLCIwIl0sWzAsMSwiMCJdLFs3LDFdLFsxLDJdLFsyLDNdLFswLDVdLFs1LDRdLFs0LDZdLFswLDFdLFs1LDJdLFs0LDNdXQ==&macro_url=https%3A%2F%2Fraw.githubusercontent.com%2FdFoiler%2Fnotes%2Fmaster%2Fnir.tex
% 	\[\begin{tikzcd}
% 		& {\displaystyle\frac{C^0(H,A)}{B^0(H,A)}} & {\displaystyle\frac{C^0(H,\op{Hom}_\ZZ(\ZZ[G],A))}{B^0(H,\op{Hom}_\ZZ(\ZZ[G],A))}} & {\displaystyle\frac{C^0(H,\op{Hom}_\ZZ(I_G,A))}{B^0(H,\op{Hom}_\ZZ(I_G,A))}} & 0 \\
% 		0 & {Z^1(H,A)=B^1(H,A)} & {Z^1(H,\op{Hom}_\ZZ(\ZZ[G],A))} & {Z^1(H,\op{Hom}_\ZZ(I_G,A))}
% 		\arrow[from=2-1, to=2-2]
% 		\arrow[from=2-2, to=2-3]
% 		\arrow[from=2-3, to=2-4]
% 		\arrow[from=1-2, to=1-3]
% 		\arrow[from=1-3, to=1-4]
% 		\arrow[from=1-4, to=1-5]
% 		\arrow[from=1-2, to=2-2]
% 		\arrow[from=1-3, to=2-3]
% 		\arrow[from=1-4, to=2-4]
% 	\end{tikzcd}\]
% 	In short, take $\varphi\in Z^0(H,\op{Hom}_\ZZ(I_G,A))=\op{Hom}_\ZZ(I_G,A)^H$, pull it back to $z\mapsto\varphi(z-\varepsilon(z))$. Pushing this down to $Z^1(H,\op{Hom}_\ZZ(\ZZ[G],A))$ and pulling back to $Z^1(H,A)$ takes us to the $1$-cocycle $h\mapsto h\varphi\left(h^{-1}-1\right)$. Here we use the $H^1(H,A)=0$ condition above and adjust our lift $z\mapsto\varphi(z-\varepsilon(z))$ accordingly.
% \end{remark}
And now we can now make our dimension shifting explicit.
\begin{lemma} \label{lem:dimshift2}
	Work in the context of \autoref{lem:howtolift} and assume that $H\subseteq G$ is normal. We track through the isomorphism
	\[\delta\colon H^1\left(G/H,\op{Hom}_\ZZ(I_G,A)^H\right)\simeq H^2\left(G/H,A^H\right)\]
	given by the exact sequence
	\[0\to A^H\to\op{Hom}_\ZZ(\ZZ[G],A)^H\to\op{Hom}_\ZZ(I_G,A)^H\to0.\]
\end{lemma}
\begin{proof}
	We begin with some $c\in H^1\left(G/H,\op{Hom}_\ZZ(I_G,A)^H\right)$. To track through the $\delta$, we define
	\[\widetilde c(gH)\coloneqq c(gH)(z-\varepsilon(z))+\eta_{c(gH)}\varepsilon(z)\]
	to be the lift given in \autoref{lem:howtolift}. Now, we are given that $dc=0$, which here means that any $z\in\ZZ[G]$ and $gH,g'H\in G/H$ will have
	\begin{align*}
		0 &= (dc)(gH,g'H)(z-\varepsilon(z)) \\
		0 &= (gH\cdot c(g'H)-c(gg'H)+c(gH))(z-\varepsilon(z)) \\
		0 &= g\cdot c(g'H)\left(g^{-1}z-g^{-1}\varepsilon(z)\right)-c(gg'H)(z-\varepsilon(z))+c(gH)(z-\varepsilon(z)) \\
		g\cdot c(g'H)\left(g^{-1}-1\right)\varepsilon(z) &= g\cdot c(g'H)\left(g^{-1}z-\varepsilon(z)\right)-c(gg'H)(z-\varepsilon(z))+c(gH)(z-\varepsilon(z)) \\
		g\cdot c(g'H)\left(g^{-1}-1\right)\varepsilon(z) &= g\cdot c(g'H)\left(g^{-1}z-\varepsilon(g^{-1}z)\right)-c(gg'H)(z-\varepsilon(z))+c(gH)(z-\varepsilon(z)).
	\end{align*}
	We now directly compute that
	\begin{align*}
		(d\widetilde c)(gH,g'H)(z) &= (gH\cdot c(g'H)-c(gg'H)+c(gH))(z) \\
		&= g\cdot c(g'H)\left(g^{-1}z-\varepsilon(g^{-1}z)\right)+g\eta_{c(g'H)}\varepsilon(z) \\
		&\phantom{{}={}}-c(gg'H)(z-\varepsilon(z))-\eta_{c(gg'H)}\varepsilon(z) \\
		&\phantom{{}={}}+c(gH)(z-\varepsilon(z))+\eta_{c(gH)}\varepsilon(z) \\
		&= \left(g\cdot c(g'H)\left(g^{-1}-1\right)+g\cdot \eta_{c(g'H)}-\eta_{c(gg'H)}+\eta_{c(gH)}\right)\varepsilon(z)
	\end{align*}
	% We quickly recall that $c(g'H)\in\op{Hom}_\ZZ(I_G,A)^H$ implies that $h\cdot c(g'H)(z)=(h\cdot c(g'H))\left(h^{-1}z\right)=c(g'H)\left(h^{-1}-1\right)$, so in fact we can write
	% \[(d\widetilde c)(gH,g'H)(z) = \left(c(g'H)\left(1-g\right)-g\cdot \eta_{c(g'H)}+\eta_{c(gg'H)}-\eta_{c(gH)}\right)\varepsilon(z).\]
	As such, we have pulled ourselves back to the $2$-cocycle given by
	\[\boxed{u(gH,g'H)\coloneqq g\cdot c(g'H)\left(g^{-1}-1\right)+g\cdot \eta_{c(g'H)}-\eta_{c(gg'H)}+\eta_{c(gH)}}.\]
	We quickly note that this is in fact independent of our choice of representative $g\in gH$: changing representative of $g$ to $gh$ for $h\in H$ will only affect the terms
	\[h\cdot c(g'H)\left(h^{-1}g^{-1}-1\right)+h\eta_{c(g'H)}=c(g'H)\left(g^{-1}-h\right)+c(g'H)\left(h-1\right)+\eta_{c(g'H)}=c(g'H)\left(g^{-1}-1\right)+\eta_{c(g'H)},\]
	so we are indeed safe. This completes the proof.
	% Even though it is not necessary, we will run the following checks on $u$.
	% \begin{itemize}
	% 	\item We verify that $\im u\subseteq A^H$. The main point is that any $h\in H$ will have $h\cdot \eta_\varphi=h\varphi\left(h^{-1}-1\right)+\eta_\varphi=\varphi(1-h)+\eta_\varphi$ for any $\varphi\in\op{Hom}_\ZZ(I_G,A)^H$. Thus,
	% 	\begin{align*}
	% 		h\cdot u(gH,g'H) &= hg\cdot c(g'H)\left(g^{-1}-1\right)-hg\eta_{c(g'H)}+h\eta_{c(gg'H)}-h\eta_{c(gH)} \\
	% 		&= gg^{-1}hg\cdot c(g'H)\left(g^{-1}-1\right)-gg^{-1}hg\eta_{c(g'H)}+h\eta_{c(gg'H)}-h\eta_{c(gH)} \\
	% 		&= g\cdot c(g'H)\left(g^{-1}h-g^{-1}hg\right) \\
	% 		&\phantom{{}={}}-g\left(c(g'H)(1-g^{-1}hg)+\eta_{c(g'H)}\right) \\
	% 		&\phantom{{}={}}+c(gg'H)(1-h)+\eta_{c(gg'H)} \\
	% 		&\phantom{{}={}}-c(gH)(1-h)-\eta_{c(gH)} \\
	% 		&= g\cdot c(g'H)(g^{-1}h-1)+c(gg'H)(1-h)-c(gH)(1-h)-g\eta_{c(g'H)}+\eta_{c(gg'H)}-\eta_{c(gH)} \\
	% 		&= g\cdot c(g'H)(g^{-1}h-1)+g\cdot c(g'H)(1-h)-g\eta_{c(g'H)}+\eta_{c(gg'H)}-\eta_{c(gH)} \\
	% 		&= g\cdot c(g'H)(g^{-1}h-h)-g\eta_{c(g'H)}+\eta_{c(gg'H)}-\eta_{c(gH)} \\
	% 	\end{align*}
	% 	Because 
	% \end{itemize}
\end{proof}
We now make \autoref{thm:infres} explicit in the case of $q=2$.
\begin{lemma}[{\cite[Lemma~2.3]{explicit-fund-classes}}] \label{lem:explicitresinf}
	Let $G$ be a group with normal subgroup $H\subseteq G$. Fix a $G$-module $A$ with $H^1(H,A)=0$, and define the function $\eta_\bullet\colon\op{Hom}_\ZZ(I_G,A)^H\to A$ of \autoref{lem:howtolift}. Given $c\in Z^2(G,A)$ such that $\op{Res}^G_Hc\in B^2(H,A)$; in particular, suppose we have $b\in\op{Hom}_\ZZ(I_G,A)$ such that all $h\in H$ have
	\[\op{Res}^G_H(\delta^{-1}c)(h)=(db)(h)=h\cdot b-h,\]
	where $\delta^{-1}$ is the inverse isomorphism of \autoref{lem:explicitdimshift}. Then we find $u\in Z^2\left(G/H,A^H\right)$ such that
	\[[\op{Inf}u]=[c]\]
	in $H^2(G,A)$.
\end{lemma}
\begin{proof}
	The main point is that boundary morphisms $\delta$ commute with $\op{Res}$ and $\op{Inf}$. By construction, we have that $\left(\op{Res}^G_H\delta^{-1}c\right)-db=0$ in $Z^1(H,\op{Hom}_\ZZ(I_G,A))$. Pulling back to $Z^1(G,\op{Hom}_\ZZ(I_G,A))$, we note that
	\[c'\coloneqq\left(\delta^{-1}c-db\right)\in Z^1(G,\op{Hom}_\ZZ(I_G,A))\]
	vanishes on $H$ by hypothesis. Because $\delta^{-1}c-db$ is a $1$-cocycle, we are able to write
	\[c'(gg')=c'(g)+gc'(g').\]
	Letting $g'$ vary over $H$, we see that $\delta^{-1}c-db$ is well-defined on $G/H$. On the other hand, for any $h\in H$ and $g\in G$, we note that $g^{-1}hg\in H$, so
	\[c'(g)=c'\left(g\cdot g^{-1}hg\right)=c'\left(hg\right)=c'\left(h\right)+hc(g),\]
	implying that $c'(g)\in\op{Hom}_\ZZ(I_G,A)^H$.

	We are now ready to apply \autoref{lem:dimshift2}, which we use on $c'$, thus defining $u\coloneqq\delta(c')$. Explicitly, we have
	\[
		\boxed{u(gH,g'H) = g\cdot c'(g'H)\left(g^{-1}-1\right)+g\cdot \eta_{c'(g'H)}-\eta_{c'(gg'H)}+\eta_{c'(gH)}}.
	\]
	This is explicit enough for our purposes. Observe that $[\op{Inf}u]=[c]$ because $[\op{Inf}c']=[\delta^{-1}c]$, and $\delta$ commutes with $\op{Inf}$.
	% Thus, we define $\overline c'\in C^1\left(G/H,\op{Hom}_\ZZ(I_G,A)^H\right)$ by $\overline c'(gH)\coloneqq c'(g)$. Note $\overline c'\in Z^1\left(G/H,\op{Hom}_\ZZ(I_G,A)^H\right)$ because each $g,g'\in G$ give
	% \[\overline c'(gH\cdot g'H)=c'(gg')=c'(g)+gc'(g')=\overline c'(gH)+gH\cdot \overline c'(gH).\]
	% We take a moment to understand $\op{Hom}_\ZZ(I_G,A)^H$. Given $f\in\op{Hom}_\ZZ(I_G,A)$, the condition that $f$ is fixed by $H$ is saying that all $h\in H$ will have $hf=f$. Concretely, we require each $g\in G$ to have
	% \[f(g-1)=(hf)(g-1)=h\cdot f\left(h^{-1}g-h^{-1}\right).\]
\end{proof}