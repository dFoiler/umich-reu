% !TEX root = ../intertwining.tex

\section{\texorpdfstring{$q$}{q}-Combinatorial Inputs} \label{sec:qcombo}
In this section, we discuss the eigenvalues of some antitriangular matrices. Essentially the only method in the literature to access the eigenvalues of an antitriangular matrix is to do some educated guessing in order to make the give matrix upper-triangular. See \cite{britnell-antitriangular} for a thorough discussion of a special case; the work in this subsection can be seen as a $q$-analogue for some of their results. For the first two subsections of this section, we discuss some (purely!) combinatorial inputs into our main results. Notably, these subsections will not use the notation of \Cref{sec:rep-theory}, and $q$ will be treated as a free variable.

\subsection{A Couple \texorpdfstring{$q$}{q}-Identities}
In this quick subsection, we pick up a couple $q$-identities which will be useful in the sequel. Throughout, we freely use the packages \texttt{qZeil} and \texttt{qMultiSum} developed by Axel Riese; see \cite{riese-zeil,riese-multisum} for a description of these packages.

The following identity is used for the linear groups.
\begin{proposition} \label{prop:gl-q-identity}
    For any nonnegative integers $m,n\in\ZZ$, we have
    \begin{align*}
        &q^{-m^{2}+mn}\sum_{i=0}^{m}\left(-1\right)^{i}q^{\frac{1}{2}i\left(i-1\right)-ni}\frac{\left(q;q\right)_m^{2}}{\left(q;q\right)_i\left(q;q\right)_{m-i}^{2}} \\
        ={}&\sum_{i+j+k=n}\left(-1\right)^{i}q^{\frac{1}{2}i\left(i-1\right)-mi}\frac{\left(q;q\right)_n}{\left(q;q\right)_i\left(q;q\right)_j\left(q;q\right)_k}.
    \end{align*}
\end{proposition}
\begin{proof}
    Let the left-hand side be $L_{m,n}(q)$ and the right-hand side be $R_{m,n}(q)$ so that we want to show that $L_{m,n}(q)=R_{m,n}(q)$. We will show that $L_{m,n}(q)$ and $R_{m,n}(q)$ satisfy the same recurrence in $n$ and then check that $L_{m,n}(q)=R_{m,n}(q)$ for some small $n$. With this outline in mind, we have the following steps.
    \begin{enumerate}
        \item After some rearranging, \texttt{qMultiSum} shows that $n\ge0$ has
        \[R_{m,n+2}\left(q\right)+\left(q^{1+n-m}-2\right)R_{m,n+1}\left(q\right)-\left(q^{1+n}-1\right)R_{m,n}\left(q\right)=0.\]
        We would like to show that $L_{m,n}(q)$ satisfies the same recurrence in $n$. Well, define $\widetilde L_{m,n}(q)$ to be
        \[L_{m,n+2}\left(q\right)+\left(q^{1+n-m}-2\right)L_{m,n+1}\left(q\right)-\left(q^{1+n}-1\right)L_{m,n}\left(q\right)=0,\]
        Some elementary rearranging shows that $\widetilde L_{m,n}(q)$ is
        \[\sum_{i=0}^{m}\left(-1\right)^{i}q^{\frac{1}{2}i\left(i-1\right)-ni}\left(q^{2m-2i}-2q^{m-i}+q^{1+n-i}+1-q^{1+n}\right)\frac{\left(q;q\right)_m^{2}}{(q;q)_i(q;q)_{m-i}^{2}}\]
        up to some powers of $q$ that we have divided out. Now, \texttt{qZeil} is able to show that this sum vanishes.
        \item It remains to check that $L_{m,n}(q)=R_{m,n}(q)$ for $n\in\{0,1\}$. For $n=0$, \texttt{qZeil} shows $L_{m,0}(q)=1$, which agrees with $R_{m,0}(q)$. For $n=1$, \texttt{qZeil} shows
        \[L_{m,1}(q)=\frac{2q^m-1}{2q^m-q}L_{m-1,1}(q)\]
        and checks that $R_{m,1}(q)$ satisfies the same recurrence in $m$. So we complete the proof upon computing $L_{1,0}(q)=R_{1,0}(q)=1$.
        \qedhere
    \end{enumerate}
\end{proof}
The following identity is used for the symplectic and orthogonal groups.
\begin{prop} \label{prop:sp-q-identity}
    For any nonnegative integers $m,n\in\ZZ$, we have
    \[q^{\frac{-m^2+m}2+mn}\sum_{i=0}^{\floor{m/2}}(-1)^iq^{i(i-1)-2in}\frac{(q;q)_m}{\left(q^2;q^2\right)_i(q;q)_{i-2j}}=\sum_{j=0}^n(-1)^jq^{j(j-i)}\frac{\left(q^2;q^2\right)_{n}}{\left(q^2;q^2\right)_j(q;q)_{n-j}}.\]
\end{prop}
\begin{proof}
    Let the left-hand side be $L_{m,n}(q)$, and let the right-hand side by $R_{m,n}(q)$. We will show that $L_{m,n}(q)$ and $R_{m,n}(q)$ satisfy the same recurrence in $n$ and then check that $L_{m,n,}(q)=R_{m,n}(q)$ for some small $n$. With this outline in mind, we have the following steps.
    \begin{enumerate}
        \item The package \texttt{qZeil} shows that $n\ge2$ has
        \[R_{m,n}\left(q\right)+\frac{\left(q^{2n}-q^{m+1}-q^{m+2}\right)}{q^{m+1}}R_{m,n-1}\left(q\right)+q\left(1-q^{2n-2}\right)R_{m,n-2}\left(q\right)=0.\]
        We would like to show that $L_{m,n}(q)$ satisfies the same recurrence in $m$. Well, define $\widetilde L_{m,n}(q)$ to be
        \[L_{m,n}\left(q\right)+\frac{\left(q^{2n}-q^{m+1}-q^{m+2}\right)}{q^{m+1}}L_{m,n-1}\left(q\right)+q\left(1-q^{2n-2}\right)L_{m,n-2}\left(q\right),\]
        and we want to show that $\widetilde L_{m,n}(q)$ vanishes. Some simplification shows that $\widetilde L_{m,n}(q)$ equals
        \begin{align*}
            \sum_{i=0}^{\floor{n/2}}&(-1)^iq^{-1+i^2-\frac{m^2}2+m\left(-\frac32+n\right)-i(1+2n)} \\
            &\cdot\left(q^{2+4i}-q^{1+2i+m}-q^{2+2i+m}+q^{1+2m}+q^{2(i+n)}-q^{2(2i+n)}\right) \\
            &\cdot\frac{\left(q;q\right)_n}{\left(q;q\right)_{n-2i}\left(q^{2};q^{2}\right)_i}.
        \end{align*}
        The package \texttt{qZeil} is able to show that this sum vanishes.
        \item It remains to check that $L_{m,n}(q)=R_{m,n}(q)$ for $n\in\{0,1\}$. For $n=0$, \texttt{qZeil} shows that $L_{m,0}(q)=1$, which agrees with $R_{m,0}(q)$. For $n=1$, \texttt{qZeil} shows that
        \[L_{m,1}(q)=\frac{q-q^m-q^{m+1}}{q^2-q^m-q^{m+1}}L_{m-1,1}(q)\]
        and checks that $R_{m,1}(q)$ satisfies the same recurrence. So it is enough to check that $L_{0,1}(q)=R_{0,1}(q)=1$.
        \qedhere
    \end{enumerate}
\end{proof}

\subsection{Eigenvalues for Linear Groups}
We continue with the notation of \Cref{sec:rep-theory} with $G\in\{\GL_{2n},\SL_{2n}\}$. In this subsection, we will compute the eigenvalues of the intertwining operator when $\beta_\chi=1$.

For expositional reasons, we begin by computing the eigenvalues of a certain helper matrix.
\begin{proposition} \label{prop:gl-helper}
    Fix a positive integer $n$. For indices $i,j\in\{0,1,\ldots,n\}$ such that $i+j-n\ge0$, define the sign
    \[\varepsilon_A(i,j)\coloneqq(-1)^{i+j-n},\]
    and the power
    \[Q_A(i,j)\coloneqq q^{\binom{i+j-n+1}2-\left(i+1\right)^{2}},\]
    and then the entry
    \[A(i,j)\coloneqq\varepsilon_A(i,j)Q_A(i,j)\frac{(q;q)_i^{2}}{(q;q)_{n-j}^{2}(q;q)_{i+j-n}}.\]
    Further, define $A(i,j)=0$ for other indices $i$ and $j$. Then the antitriangular matrix $[A(i,j)]_{0\le i,j\le n}$ is diagonalizable with eigenvalues
    \[\left\{(-1)^{n-i}q^{\binom{i+1}2-\binom{n+2}2}:0\le i\le n\right\}.\]
\end{proposition}
\begin{proof}
    For indices $i,j\in\{0,1,\ldots,n\}$ such that $j\ge i$, define the sign
    \[\varepsilon_B(i,j)\coloneqq(-1)^i\]
    and the power
    \[Q_B(i,j)\coloneqq q^{-(n+1)(j+1)+\binom{i+1}2}\]
    and then the entry
    \[B(i,j)\coloneqq\varepsilon_B(i,j)Q_B(i,j)\sum_{k=0}^{j-i}\frac{(q;q)_j}{(q;q)_i(q;q)_k(q;q)_{j-i-k}}.\]
    Further, define $B(i,j)=0$ for other indices $i$ and $j$. Then we claim that $A$ is similar to $B$, which will complete the proof upon reading off the diagonal entries of $B$.

    It remains to show the claim. We will conjugate $A$ by the matrix $M$ defined by
    \[M(i,j)\coloneqq q^{-(i+1)(j+1)}\]
    for $i,j\in\{0,1,\ldots,n\}$. To show $M^{-1}AM=B$, it is enough to show that $AM=MB$. (Note that $M$ is automatically invertible because it is a Vandermonde matrix.) Thus, for indices $i$ and $k$, we want to show that
    \[(AM)_{ik}\stackrel?=(MB)_{ik},\]
    which is equivalent to
    \[\sum_{j=0}^nA_{ij}M_{jk}\stackrel?=\sum_{j=0}^nM_{ij}B_{jk}.\]
    Now, $A_{ij}$ will vanish unless $i+j\ge n$, and $B_{jk}$ will vanish unless $j\le k$, so we go ahead and re-index the sums so that we want to show
    \[\sum_{j=0}^iA_{i,n-j}M_{n-j,k}\stackrel?=\sum_{j=0}^kM_{ij}B_{jk}.\]
    Upon plugging in our definitions and simplifying (and notably cancelling many powers of $q$), this reduces to
    \begin{align*}
        &q^{-i^{2}+ik}\sum_{j=0}^{i}\left(-1\right)^{j}q^{\frac{1}{2}j\left(j-1\right)-kj}\frac{(q;q)_i^{2}}{(q;q)_j(q;q)_{i-j}^{2}} \\
        ={}&\sum_{j=0}^{k}\sum_{j'=0}^{k-j}\left(-1\right)^{j}q^{\frac{1}{2}j\left(j-1\right)-ij}\frac{(q;q)_k}{(q;q)_j(q;q)_{j')(q;q)_{k-j-j'}}},
    \end{align*}
    which is \Cref{prop:gl-q-identity}.
\end{proof}
\begin{example}
    For $n=4$, the matrix in \Cref{prop:gl-helper} is
    \[\begin{bmatrix}  &   &   &   & q^{-1}\\  &   &   & q^{-4}  & \frac{ ( q - 1 ) }{1} q^{-3}\\  &   & q^{-9}  & \frac{ ( q^{2} - 1 ) ^{2} }{ ( q - 1 ) } q^{-8}  & \frac{ ( q - 1 ) ( q^{2} - 1 ) }{1} q^{-6}\\  & q^{-16}  & \frac{ ( q^{3} - 1 ) ^{2} }{ ( q - 1 ) } q^{-15}  & \frac{ ( q^{2} - 1 ) ( q^{3} - 1 ) ^{2} }{ ( q - 1 ) } q^{-13}  & \frac{ ( q - 1 ) ( q^{2} - 1 ) ( q^{3} - 1 ) }{1} q^{-10}\\q^{-25}  & \frac{ ( q^{4} - 1 ) ^{2} }{ ( q - 1 ) } q^{-24}  & \frac{ ( q^{3} - 1 ) ^{2} ( q^{4} - 1 ) ^{2} }{ ( q - 1 ) ( q^{2} - 1 ) } q^{-22}  & \frac{ ( q^{2} - 1 ) ( q^{3} - 1 ) ( q^{4} - 1 ) ^{2} }{ ( q - 1 ) } q^{-19}  & \frac{ ( q - 1 ) ( q^{2} - 1 ) ( q^{3} - 1 ) ( q^{4} - 1 ) }{1} q^{-15}\\\end{bmatrix}.\]
    The negative powers of $q$ will go away in the sequel because we will multiply through by a large power of $q$ when working with the intertwining operator.
\end{example}
\begin{theorem} \label{thm:eigens-gl}
    Take $G\in\{\GL_{2n},\SL_{2n}\}$. Fix a character $\chi\colon P\to\CC^\times$, which we write as $\chi=(\alpha\circ m)(\beta\circ\chi_{\det})$. Assume that $\beta=1$ so that $\chi=\chi^J$. Then the operator $I$ on $\Ind_P^G\chi$ is diagonalizable and has eigenvalues given by
    \[\left\{(-1)^{n-i}q^{\binom n2+\binom{i+1}2}:0\le i\le n\right\}.\]
\end{theorem}
\begin{proof}
    Identify $I$ with its matrix representation. We apply \Cref{prop:gl-helper} after conjugating $f$ by the $(n+1)\times(n+1)$ matrix $S$ defined by
    \[S_{ij}\coloneqq\begin{cases}
        0 & \text{if }i+j<n, \\
        1 & \text{if }i+j\ge n.
    \end{cases}\]
    One can quickly check that $S$ is invertible with inverse $T$ given by
    \[T_{ij}=\begin{cases}
        -1 & \text{if }i+j=n-1, \\
        1 & \text{if }i+j=n, \\
        0 & \text{otherwise}.
    \end{cases}\]
    Now, define $A$ as in \Cref{prop:gl-helper} with $n=n-1$. Then we claim that
    \begin{equation}
        S^{-1}IS \stackrel?= q^{n^2}\begin{bmatrix}
            -\sigma A\sigma \\
            (1,\ldots,1) & 1
        \end{bmatrix}, \label{eq:conj-intertwining-gl}
    \end{equation}
    where $\sigma$ is the permutation sending $e_i\mapsto e_{n-1-i}$ for all $i\in\{0,\ldots,n-1\}$, and $(1,\ldots,1)$ is a row vector consisting of all $1$s. Before providing the claim, we explain how it implies the result. Taking the eigenvalues of \Cref{prop:gl-helper} (and some simplification) checks that our eigenvalues in the theorem are correct from the above equation. Because all the eigenvalues are distinct, we see that the right-hand matrix is diagonalizable.

    It remains to show \eqref{eq:conj-intertwining-gl}. In fact, we will show that $TI=q^{n^2}\begin{bsmallmatrix}
        -\sigma A\sigma \\ (1,\ldots,1) & 1
    \end{bsmallmatrix}T$ because $T$ is simpler than $S$. Well, choosing some indices $i$ and $k$, we would like to show that
    \[(TI)_{ik}\stackrel?=\left(q^{n^2}\begin{bmatrix}
        -\sigma A\sigma \\
        (1,\ldots,1) & 1
    \end{bmatrix}T\right)_{ik}\]
    or equivalently
    \[\sum_{j=0}^nT_{ij}T_{jk}\stackrel?=q^{n^2}\sum_{j=0}^n\begin{bmatrix}
        -\sigma A\sigma \\
        (1,\ldots,1) & 1
    \end{bmatrix}_{ij}T_{jk}.\]
    Using the definition of $T$, we see that we want to show that
    \[I_{n-i,k}-1_{i<n}I_{n-i-1,k}\stackrel?=q^{n^2}\left(\begin{bmatrix}
        -\sigma A\sigma \\
        (1,\ldots,1) & 1
    \end{bmatrix}_{i,n-k}-1_{k<n}\begin{bmatrix}
        -\sigma A\sigma \\
        (1,\ldots,1) & 1
    \end{bmatrix}_{i,n-k-1}\right).\]
    We verify this by rather tedious casework on $i$ and $k$. Denote the left-hand side by $L$ and the right-hand side by $R$.
    \begin{itemize}
        \item Suppose $i=k=n$. Then the definition of $I$ yields $L=q^{n^2}$, and glancing at the bottom row of our matrix in $R$ yields $R=q^{n^2}$ as well.
        \item Suppose $i=n$ but $k<n$. Then we see $L=0$ by construction of $I$'s matrix, and one can check that $R=1-1=0$.
        \item Suppose $i<n$ but $k=n$. Then some expansion with the definitions of $I$ and $A$ reveals that
        \[L=(-1)^{n-i}q^{n^2-\binom{n-i+1}2}(q;q)_{n-i-1}=R.\]
        \item Suppose $k<i<n$. Then $(n-i)+k-n<0$ and $(n-1-i)+(n-1-(n-k-1))<n-1$, so all coefficients vanish.
        \item Suppose $i=k<n$. Then some expansion with the definitions of $I$ and $A$ reveals that
        \[L=q^{n^2-(n-i)^2}=R.\]
        \item Suppose $i<k<n$. Then some expansion with the definitions of $I$ and $A$ reveals that
        \[L=(-1)^{k-i}q^{n^2-(n-i)^2+\binom{k-i}2}\frac{(q;q)_{n-i-1}^2}{(q;q)_{n-k}^2(q;q)_{k-i}}\left(1-2q^{n-i}+q^{2n-i-k}\right)=R.\]
    \end{itemize}
    The above casework completes the proof.
\end{proof}

\subsection{A Helper Matrix} \label{subsec:helper}
For this subsection, $q$ will return to being a free variable. Akin to \Cref{prop:gl-helper}, we describe a general helper matrix which shows up in the upper-triangularization for the groups $G\in\{\GO_{2n},\O_{2n},\Sp_{2n},\GSp_{2n}\}$, so we handle it here. For some fixed nonnegative integer $n$ and sign $\varepsilon\in\{\pm1\}$ and $a\in\CC$, we select indices $0\le i,j\le n$ such that $i+j-n$ is an even nonnegative integer and define the sign
\[\varepsilon_A(i,j)\coloneqq\varepsilon^{\frac{i-j-n}2}(-1)^{\frac{i+j-n}2}\]
and the power
\[Q_A(i,j)\coloneqq q^{-\binom{i+a}2+\frac{i+j-n}{2}\left(\frac{i+j-n}{2}+a-1\right)}\]
and then the entry
\[A(i,j)\coloneqq\varepsilon_A(i,j)Q_A(i,j)\frac{(q;q)_i}{(q^2;q^2)_{(i+j-n)/2}(q;q)_{n-j}}.\]
Further, define $A(i,j)=0$ for other indices $i$ and $j$. This $(n+1)\times(n+1)$ matrix will be of interest to us.
\begin{example}
    For $(n,\varepsilon,a)=(4,1,1)$, we can compute $A$ equals
    \[\begin{bmatrix}  &   &   &   & 1\\  &   &   & q^{-1}\\  &   & q^{-3}  &   & \frac{ ( q - 1 ) }{1} q^{-2}\\  & q^{-6}  &   & \frac{ ( q^{3} - 1 ) }{1} q^{-5}\\q^{-10}  &   & \frac{ ( q^{3} - 1 ) ( q^{4} - 1 ) }{ ( q^{2} - 1 ) } q^{-9}  &   & \frac{ ( q - 1 ) ( q^{3} - 1 ) }{1} q^{-6}\\\end{bmatrix}.\]
    The negative powers of $q$ are desirable because we would prefer to work with $q^{-\left|U\right|}I$ instead of $I$ for normalization purposes, where $I$ is the intertwining operator.
\end{example}
% We begin by computing its inverse, which is useful because the inverse of $A$ is more closely related to our intertwining operator.
% \begin{proposition} \label{prop:helper-matrix-inv}
%     Define $n$, $\varepsilon$, $a$, and $A$ as above. For indices $0\le i,j\le n$ such that $i+j-n$ is an even nonnegative integer and define the sign
%     \[\varepsilon_B(i,j)\coloneqq\varepsilon^{\frac{i-j-n}2}(-1)^{(i+j-n)/2}\]
%     and the power
%     \[Q_B(i,j)\coloneqq q^{-\binom{i+a}2+\left(a+\frac{i+j-n}2-1\right)\left(\frac{i+j-n}2\right)}\]
%     and then the entry
%     \[B(n-i,n-j)\coloneqq\varepsilon_B(i,j)Q_B(i,j)\frac{(q;q)_i}{(q^2;q^2)_{(i+j-n)/2}(q;q)_{n-j}}.\]
%     Then $A$ is invertible, and $B=A^{-1}$.
% \end{proposition}
% \begin{proof}
%     It is enough to show $AB=1_{n+1}$. Thus, we select indices $i$ and $k$, and we want to show
%     \[\sum_{j=0}^nA_{ij}B_{jk}=(AB)_{ik}\stackrel?=1_{i=k}.\]
%     Well, $A_{ij}=0$ unless $i+j\ge n$, and $B_{jk}=0$ unless $(n-j)+(n-k)\ge0$, so we may assume that $j\in[n-i,n-k]$. In particular, our sum can only be nontrivial when $i\ge k$. We now apply the substitution $j\mapsto n+j-i$ so that we want to show
%     \[\sum_{j=0}^{i-k}A_{i,n+j-i}B_{n+j-i,k}\stackrel?=1_{i=k}.\]
%     Now, the $A$ term is zero unless $i+(n+j-i)-n\equiv0\pmod2$, which amounts to requiring $j\equiv0\pmod2$; checking parities for $B$ produces the requirement $(i-j)+(n-k)-n\equiv0\pmod2$, so we see that the sum only has a chance to be nontrivial when $i\equiv k\pmod2$. Thus, we may assume that $i-k$ is even, in which case we want to show
%     \[\sum_{j=0}^{(i-k)/2}A_{i,n+2j-i}B_{n+2j-i,k}\stackrel?=1_{i=k}.\]
%     Now, plugging in the definitions of $A$ and $B$ and simplifying shows that the left-hand side is
%     \[(-\varepsilon)^{\frac{i-k}2}q^{a\left(\frac{i-k}2\right)}\frac{(q;q)_i}{(q;q)_k}\sum_{j=0}^{(i-k)/2}(-1)^jq^{2\binom{\frac{i-k}2-j}2}\frac1{(q^2;q^2)_j(q^2;q^2)_{(i-k)/2-j}}.\]
%     If $i=k$, then the sum only has one term, and we can compute that we get $1$. If $i\ne k$, then the sum vanishes by the Cauchy binomial theorem.
% \end{proof}
We now compute the eigenvalues of $A$ when $n$ is even. No great effort has been made to the simplification of these eigenvalues in order to preserve their ``order.'' (It will notably be important to keep track of the parity of $i$ in applications.)
\begin{proposition} \label{prop:helper-matrix-even}
    Define $n$, $\varepsilon$, $a$, and $A$ as above. If $n=2m$, then the antitriangular matrix $[A(i,j)]_{0\le i,j\le n}$ is diagonalizable with eigenvalues
    \[\left\{\varepsilon^{m}(-1)^{\floor{\frac i2}}q^{-\binom{a+m}2-\binom{m+1}2+\left(m-\floor{\frac i2}\right)^2}:0\le i\le n\right\}.\]
\end{proposition}
\begin{proof}
    For indices $0\le i,j\le n$ such that $j-i$ is a nonnegative even integer, define the sign
    \[\varepsilon_B(i,j)\coloneqq\varepsilon^{m}\left(-1\right)^{\floor{\frac{i}{2}}}\]
    and the power
    \[Q_B(i,j)\coloneqq q^{-\binom{a+m}2-\binom{m+1}2+\left(m-\floor{\frac{i}{2}}\right)^{2}-\frac{j-i}{2}\left(2m-i+2\floor{\frac{i}{2}}\right)}\]
    and then the entry
    \[B(i,j)\coloneqq\varepsilon_B(i,j)Q_B(i,j)\frac{(q^2;q^2)_{\floor{j/2}}}{(q^2;q^2)_{i/2}(q;q)_{(j-i)/2}}.\]
    Further, define $B(i,j)=0$ for other indices $i$ and $j$. Then we claim that $A$ is similar to $B$, which will complete the proof upon reading off the diagonal entries of $B$. Notably, even though some eigenvalues are equal, $B$ splits into a direct sum of operators on the even basis vectors and on the odd basis vectors, and the operators on these subspaces have distinct eigenvalues.

    It remains to show the claim. We will conjugate $A$ by the matrix $M$ defined by
    \[M(i,j)\coloneqq\varepsilon^{\floor{i/2}}q^{\floor{i/2}(2\floor{j/2}+a)}\]
    for indices $i$ and $j$ such that $i\equiv j\pmod2$ and zero elsewhere. To show $M^{-1}AM=B$, it is enough to show that $AM=MB$. (Note that $M$ is automatically invertible because its determinant is essentially computed by two Vandermonde determinants.) Thus, for indices $i$ and $k$, we want to show that
    \[(AM)_{ik}\stackrel?=(MB)_{ik}.\]
    If $i$ and $k$ fail to have the same parity, then we note that both sides equal $0$ because $A$, $M$, and $B$ all send even (and odd) basis vectors to linear combinations of even (and odd) basis vectors. Thus, we may assume that $i\equiv k\pmod2$. Now, we are left to verify the identity
    \[\sum_{j=0}^nA_{ij}M_{jk}\stackrel?=\sum_{j=0}^nM_{ij}B_{jk}.\]
    Now, $A_{ik}$ will vanish unless $i+j-n$ is a nonnegative even integer, and $B_{jk}$ will vanish unless $k-j$ is a nonnegative even integer, so we go ahead and re-index the sums so that we want to show
    \[\sum_{j=0}^{\floor{i/2}}A_{i,n-i+2j}M_{n-i+2j,k}\stackrel?=\sum_{j=0}^{\floor{k/2}}M_{i,k-2j}B_{k-2j,k}.\]
    Upon plugging in our definitions and simplifying (and notably cancelling many powers of $q$), this reduces to
    \begin{align*}
        &\sum_{j=0}^{\floor{i/2}}\left(-1\right)^{j}q^{j\left(j-1\right)-2j\floor{\frac k2}}\frac{\left(q;q\right)_i}{\left(q^{2};q^2\right)_j\left(q;q\right)_{i-2j}} \\
        ={}& q^{\frac{i(i-1)}2-i\floor{\frac k2}}\sum_{j=0}^{\floor{k/2}}\left(-1\right)^{j}q^{\left(j^{2}-ij\right)}\frac{\left(q^{2};q^2\right)_{\floor{k/2}}}{\left(q^{2};q^2\right)_j\left(q;q\right)_{\floor{k/2}-j}},
    \end{align*}
    which follows from \Cref{prop:sp-q-identity}.
\end{proof}
One can upgrade the eigenvalue computations in the even case to the odd case as follows.
\begin{proposition} \label{prop:helper-matrix-odd}
    Define $n$, $\varepsilon$, $a$, and $A$ as above. If $n$ is odd with $n=2m+1$, then the antitriangular matrix $[A(i,j)]_{0\le i,j\le n}$ is diagonalizable with eigenvalues
    \[\left\{\pm\sqrt\varepsilon q^{-\frac a2-\binom{a+m}2-\binom{m+1}2+(m-i)^2-i}:0\le i\le m\right\}.\]
\end{proposition}
\begin{proof}
    Denote the defined $(n+1)\times(n+1)$ matrix by $A_n$. We are interested in the eigenvalues of $A_{2m+1}$ for some $m\ge0$. For some $m\ge0$, note that $A_{2m}$ sends even (and odd) basis vectors to a linear combination of even (and odd) basis vectors, so we let $A_{2m}^+$ (and $A_{2m}^-$) denote the submatrices of $A_{2m}$ consisting of the even (and odd) and columns and rows (respectively). Thus, by rearranging the rows and columns of $A_{2m}$, we see that $A_{2m}$ is similar to 
    \[\begin{bmatrix}
        A_{2m}^+ \\ & A_{2m}^-
    \end{bmatrix}.\]
    On the other hand, we see that $A_{2m+1}$ sends even (and odd) basis vectors to a linear combination of odd (and even) basis vectors. Defining $A_{2m+1}^+$ (and $A_{2m+1}^-$) to be the submatrix consisting of the even (odd) columns and odd (even) rows, we thus see that $A_{2m+1}$ is similar to
    \[\begin{bmatrix}
        & A_{2m+1}^- \\ A_{2m+1}^+
    \end{bmatrix}.\]
    Now, we note that $A_n$ is a submatrix of $A_{n+1}$ always, but even columns and rows become odd columns and rows up to sign $\varepsilon$; thus, $A_n^+=\varepsilon A_{n+1}^-$. In particular, we see that $A_{2m+1}$ is similar to the antitriangular matrix
    \[\varepsilon\begin{bmatrix}
        & A_{2m}^+ \\ A_{2m+2}^-
    \end{bmatrix}.\]
    The proof of \Cref{prop:helper-matrix-even} explains that $A_{2m}^+$ is similar to an upper-triangular matrix with diagonal entries
    \[\left\{\varepsilon^{m+1}(-1)^{i}q^{-\binom{a+m}2-\binom{m+1}2+\left(m-i\right)^2}:0\le i\le m\right\},\]
    and $A_{2m+2}^-$ is similar to an upper-triangular matrix with diagonal entries
    \[\left\{\varepsilon^{m}(-1)^{i}q^{-\binom{a+m+1}2-\binom{m+2}2+(m+1-i)^2}:0\le i\le m\right\}\]
    where we conjugate by the same matrix! Thus, viewing $A_{2m+1}$ as an $(m+1)\times(m+1)$ matrix with entries that are $2\times2$ (block) matrices, we see that $A_{2m+1}$ is similar to a (block) upper-triangular matrix with diagonal entries
    \[\left\{\varepsilon^{m+1}(-1)^iq^{-\binom{a+m}2-\binom{m+1}2+(m-i)^2}\begin{bmatrix}
        & 1 \\
        \varepsilon q^{-a-2i}
    \end{bmatrix}:0\le i\le m\right\}.\]
    The result follows from diagonalizing these $2\times2$ matrices and noting that all the eigenvalues are distinct.
\end{proof}

\subsection{Eigenvalues for Orthogonal Groups}
We continue with the notation of \Cref{sec:rep-theory}, taking $G\in\{\GO_{2n},\O_{2n}\}$. We (essentially) begin with the case $\beta=1$.
\begin{theorem} \label{thm:o-trivial-eigens}
    Take $G\in\{\GO_{2n},\O_{2n}\}$. Fix a character $\chi\colon P\to\CC^\times$, which we write as $\chi=(\alpha\circ m)(\beta\circ\chi_{\det})$. Assume either that $\beta=1$ or $\beta^2=1$ for $G=\O_{2n}$.
    \begin{listalph}
        \item If $n=2m$ is even, then the intertwining operator $I$ on $\Ind_P^G\chi$ is diagonalizable and has eigenvalues given by
        \[\left\{(-1)^{m-\floor{\frac i2}}q^{m(m-1)+\floor{\frac i2}^2}:1\le i\le n+1\right\}.\]
        \item If $n=2m+1$ is odd, then the intertwining operator $I$ on $\Ind_P^G\chi$ is diagonalizable and has eigenvalues given by
        \[\left\{\pm q^{m^2+i(i+1)}:0\le i\le m\right\}.\]
    \end{listalph}
\end{theorem}
\begin{proof}
    The assumptions imply that the intertwining operator $I$ has a uniform matrix representation given in \Cref{prop:trivial-matrix-coeffs,prop:sp-quadratic-matrix}. Now, we define $A$ as in \Cref{subsec:helper} with $(n,\varepsilon,a)=(n,1,0)$ so that $q^{\binom n2}A$ is the matrix representation of $I$. The result now follows from combining \Cref{prop:helper-matrix-even,prop:helper-matrix-odd} and simplifying the eigenvalues.
\end{proof}
It remains to cover the case where $\beta^2=1$ but $\beta\ne1$ when $G=\GO_{2n}$. This will follow by submatrix considerations via \Cref{lem:general-from-special-matrix}.
\begin{theorem} \label{thm:go-quad-eigens}
    Take $G=\GO_{2n}$. Fix a character $\chi\colon P\to\CC^\times$, which we write as $\chi=(\alpha\circ m)(\beta\circ\chi_{\det})$. Assume that $\beta^2=1$ but $\beta\ne1$.
    \begin{listalph}
        \item If $n=2m$ is even, then the intertwining operator $I\circ I$ on $\Ind_P^G\chi$ is diagonalizable and has eigenvalues
        \[\left\{q^{2m(m-1)+2i^2}:0\le i\le m\right\}.\]
        \item If $n=2m+1$ is odd, then the intertwining operator $I\circ I$ on $\Ind_P^G\chi$ is diagonalizable and has eigenvalues
        \[\left\{q^{2m^2+2i(i+1)}:0\le i\le m\right\}.\]
    \end{listalph}
\end{theorem}
\begin{proof}
    We combine the computations of \Cref{thm:o-trivial-eigens} with \Cref{lem:general-from-special-matrix}. Let $I^0$ be the $(n+1)\times(n+1)$ matrix representation of the corresponding operator for $\O_{2n}$, and let $I^+$ and $I^-$ be the submatrices of $I$ given in \Cref{lem:general-from-special-matrix} which are the matrix representations of $I$ on $\left(\Ind_P^G\chi\right)^\chi\to\left(\Ind_P^G\chi^J\right)^\chi$ and $\left(\Ind_P^G\chi^J\right)^\chi\to\left(\Ind_P^G\chi\right)^\chi$, respectively.

    For example, if $n$ is even, then $I^+$ and $I^-$ are both the submatrices of $I$ consisting of the even rows and columns. Tracking through the proof of \Cref{thm:sp-quadratic-eigens} (and notably its input \Cref{prop:helper-matrix-even}), we will show that $I^+$ and $I^-$ are both diagonalizable with eigenvalues
    \[\left\{(-1)^{m-\floor{\frac i2}}q^{m(m-1)+i^2}:1\le i\le m\right\},\]
    which completes the proof upon squaring our eigenvalues. Indeed, defining $A$ and $B$ as in \Cref{thm:o-trivial-eigens}, we see that $I^+=I^-$ is a submatrix of $A$, and the upper-triangularization of $A$ used a matrix $M$ which preserves the span of the even (and odd) basis vectors. Thus, the submatrix of $M$ consisting of the even rows and columns will still provide an upper-triangularization of $I^+=I^-$ (namely, the corresponding submatrix of $B$), from which we can read off the diagonal as coming from the even $i$, which produces the above list of eigenvalues.

    For the remainder of the argument, we may take $n$ to be odd. Then $I^+$ is the submatrix of $I$ consisting of the even columns and odd rows, and $I^-$ is the submatrix of $I$ consisting of the odd columns and even rows. Arguing as above, we define $A$ and $B$ as in \Cref{thm:o-trivial-eigens} so that we see that $I$ is similar to $q^{\binom{n}2}A$, which in turn \Cref{prop:helper-matrix-odd} explains is similar to the block diagonal matrix with diagonal given by the $2\times2$ matrices
    \[\left\{\beta(-1)^m(-1)^iq^{\binom{2m+2}2-2\binom{m+1}2+(m-i)^2}\begin{bmatrix}
        & 1 \\
        \beta(-1)q^{-1-2i}
    \end{bmatrix}:0\le i\le m\right\}.\]
    The similarity $I\sim q^{\binom{n}2}A$ admittedly swaps even and odd rows and columns, but the last similarity preserves linear combinations of even and odd basis vectors by its construction. Thus, we see that the matrices $I^+$ and $I^-$ will be similar to the diagonal matrices achieved by reading off the diagonals in the block diagonal matrix described above (because the upper-triangularizations of $I^+$ and $I^-$ are achieved by the same matrix $M$ by construction in \Cref{prop:helper-matrix-odd}!). As such, computing the composite $I^-\circ I^+$ tells us that the eigenvalues are
    \[\left\{q^{2m^2+2i(i+1)}:0\le i\le m\right\}\]
    after a little simplification.
\end{proof}

\subsection{Eigenvalues for Symplectic Groups}
We continue with the notation of \Cref{sec:rep-theory}, taking $G\in\{\GSp_{2n},\Sp_{2n}\}$. We begin with the case $\beta=1$.
\begin{theorem}
    Take $G\in\{\GSp_{2n},\Sp_{2n}\}$. Fix a character $\chi\colon P\to\CC^\times$, which we write as $\chi=(\alpha\circ m)(\beta\circ\chi_{\det})$. Assume that $\beta=1$ so that $\chi=\chi^J$.
    \begin{listalph}
        \item If $n=2m$ is even, then the intertwining operator $I$ on $\Ind_P^G\chi$ is diagonalizable and has eigenvalues given by
        \[\left\{\pm q^{m^2+i(i+1)}:0\le i\le m-1\right\}\sqcup\left\{q^{\binom{2m+1}2}\right\}.\]
        \item If $n=2m+1$ is odd, then the intertwining operator $I$ on $\Ind_P^G\chi$ is diagonalizable and has eigenvalues given by
        \[\left\{-(-1)^{m-\floor{\frac i2}}q^{m(m+1)+\floor{\frac i2}^2}:1\le i\le n+1\right\}.\]
    \end{listalph}
\end{theorem}
\begin{proof}
    Identify $I$ with its matrix representation. We apply \Cref{prop:helper-matrix-even,prop:helper-matrix-odd} after conjugating $I$ by the $(n+1)\times(n+1)$ matrix $S$ defined by
    \[S_{ij}\coloneqq\begin{cases}
        0 & \text{if }i+j<n, \\
        1 & \text{if }i+j\ge n.
    \end{cases}\]
    One can quickly check that $S$ is invertible with inverse $T$ given by
    \[T_{ij}=\begin{cases}
        -1 & \text{if }i+j=n-1, \\
        1 & \text{if }i+j=n, \\
        0 & \text{otherwise}.
    \end{cases}\]
    Now, define $A$ as in \Cref{subsec:helper} with $n=n-1$ and $a=2$ and $\varepsilon=1$. Then we claim that
    \begin{equation}
        S^{-1}IS\stackrel?=q^{\binom{n+1}2}\begin{bmatrix}
            -\sigma A\sigma \\
            (1,\ldots,1) & 1
        \end{bmatrix}, \label{eq:conjugate-sp-trivial}
    \end{equation}
    where $\sigma$ is the permutation matrix sending $e_i\mapsto e_{n-1-i}$ for all $i\in\{0,\ldots,n-1\}$, and $(1,\ldots,1)$ is a row vector consisting of all $1$s. Before proving the claim, we explain how it implies the result. Taking the eigenvalues of \Cref{prop:helper-matrix-even,prop:helper-matrix-odd} (and some simplification) checks that our eigenvalues in the theorem are correct from the above equation. Because $A$ is diagonalizable and does not have $1$ as an eigenvalue, we are also able to show that the right-hand side is diagonalizable.

    It remains to show \eqref{eq:conjugate-sp-trivial}. In fact, we will show that $TI=q^{\binom{n+1}2}\begin{bsmallmatrix}
        -\sigma A\sigma \\ (1,\ldots,1) & 1
    \end{bsmallmatrix}T$ because $T$ is simpler than $S$. Well, choosing some indices $i$ and $k$, we would like to show that
    \[(TI)_{ik}\stackrel?=q^{\binom{n+1}2}\left(\begin{bmatrix}
        -\sigma A\sigma \\ (1,\ldots,1) & 1
    \end{bmatrix}T\right)_{ik},\]
    or equivalently
    \[\sum_{j=0}^nT_{ij}I_{jk}\stackrel?=q^{\binom{n+1}2}\sum_{j=0}^n\begin{bmatrix}
        -\sigma A\sigma \\ (1,\ldots,1) & 1
    \end{bmatrix}_{ij}T_{jk}.\]
    Using the definition of $T$, we see that we want to show that
    \[I_{n-i,k}-1_{i<n}I_{n-i-1,k}\stackrel?=q^{\binom{n+1}2}\left(\begin{bmatrix}
        -\sigma A\sigma \\ (1,\ldots,1) & 1
    \end{bmatrix}_{i,n-k}-1_{k<n}\begin{bmatrix}
        -\sigma A\sigma \\ (1,\ldots,1) & 1
    \end{bmatrix}_{i,n-k-1}\right).\]
    We verify this by rather tedious casework on $i$ and $k$. Denote the left-hand side by $L$ and the right-hand side by $R$.
    \begin{itemize}
        \item Suppose $i=n$. Then the definition of $I$ yields $L=q^{\binom{n+1}2}1_{k=n}$, and glancing at the bottom row of our matrix in $R$ yields $R=q^{\binom{n+1}2}1_{k=n}$.
        \item Suppose $k=n$ but $i<n$. If $n-i$ is even, then $(\sigma A\sigma)_{i,n-k}=0$, so $R=0$; a short computation with the coefficients of $I$ further verify that $L=0$. Now, if $n-i$ is odd, then one can compute that
        \[L=(-1)^{\frac{n-i+1}2}q^{\binom{n+1}2-\binom{n-i+1}2+2\binom{(n-i+1)/2}2}\frac{(q;q)_{n-i-1}}{(q^2;q^2)_{(n-i-1)/2}}=R.\]
        \item Suppose $k<i<n$. Then $(n-i)+k-n<0$ and $(n-1-i)+(n-1-(n-k-1))<n-1$, so all coefficients vanish.
        \item Suppose $i=k<n$. Then $L$ has only one term, which can be computed to be $q^{\binom{n+1}2-\binom{n-i+1}2}$. From here, we can compute $R=q^{\binom{n+1}2-\binom{n-i+1}2}$ as well.
        \item Suppose $i<k<n$ and $k-i$ is even. Here, computing with the coefficients of $I$ and $A$, we see that
        \[L=(-1)^{\frac{k-i}2}q^{\binom {n+1}2-\binom{n-i+1}2+2\binom{(k-i+2)/2}2}\frac{(q;q)_{n-i-1}}{(q;q)_{n-k-1}(q^2;q^2)_{(k-i)/2}}=R.\]
        \item Suppose $i<k<n$ and $k-i$ is odd. Again, computing with the coefficients of $I$ and $A$, we see that
        \[L=(-1)^{\frac{k-i+1}2}q^{\binom{n+1}2-\binom{n-i+1}2+2\binom{(k-i+1)/2}2}\frac{(q;q)_{n-i-1}}{(q;q)_{n-k}(q^2;q^2)_{(k-i-1)/2}}=R.\]
    \end{itemize}
    The above casework completes the proof.
\end{proof}
We now move on to the case where $\beta^2=1$ but $\beta\ne1$. The argument is slightly easier for $\Sp_{2n}$, from which $\GSp_{2n}$ follows by submatrix considerations.
\begin{theorem} \label{thm:sp-quadratic-eigens}
    Take $G=\Sp_{2n}$. Fix a character $\chi\colon P\to\CC^\times$, which we write as $\chi=\beta\circ\chi_{\det}$. Assume that $\beta^2=1$ but $\beta\ne1$ so that $\chi=\chi^J$.
    \begin{listalph}
        \item If $n=2m$ is even, then the intertwining operator $I$ on $\Ind_P^G\chi$ is diagonalizable and has eigenvalues
        \[\left\{\beta(-1)^m(-1)^{m-\floor{\frac i2}}q^{m^2+\floor{\frac{i}2}^2}:0\le i\le n\right\}.\]
        \item If $n=2m+1$ is odd, then the intertwining operator $I$ on $\Ind_P^G\chi$ is diagonalizable and has eigenvalues
        \[\left\{\pm\sqrt{\beta(-1)}q^{m(m+1)+i(i+1)+\frac12}:0\le i\le m\right\}.\]
    \end{listalph}
\end{theorem}
\begin{proof}
    Define $A$ as in \Cref{subsec:helper}, with $n=n$ and $\varepsilon=\beta(-1)$ and $a=1$. A little algebra shows that the matrix representation of $I$ is the matrix $q^{\binom{n+1}2}A$. The result now follows by plugging into the eigenvalue computations of \Cref{prop:helper-matrix-even,prop:helper-matrix-odd} and simplifying.
\end{proof}
We now use submatrix arguments to compute the eigenvalues for $\GSp_{2n}$.
\begin{theorem}
    Take $G=\GSp_{2n}$. Fix a character $\chi\colon P\to\CC^\times$, which we write as $\chi=(\alpha\circ m)(\beta\circ\chi_{\det})$. Assume that $\beta^2=1$ but $\beta\ne1$.
    \begin{listalph}
        \item If $n=2m$ is even, then the intertwining operator $I\circ I$ on $\Ind_P^G\chi$ is diagonalizable and has eigenvalues
        \[\left\{q^{2m^2+2i^2}:0\le i\le m\right\}.\]
        \item If $n=2m+1$ is odd, then the intertwining operator $I\circ I$ on $\Ind_P^G\chi$ is diagonalizable and has eigenvalues
        \[\left\{\beta(-1)q^{2m(m+1)+2i(i+1)+1}:0\le i\le m\right\}.\]
    \end{listalph}
\end{theorem}
\begin{proof}
    The argument is exactly analogous to \Cref{thm:go-quad-eigens}. We combine the computations of \Cref{thm:sp-quadratic-eigens} with \Cref{lem:general-from-special-matrix}. Indeed, let $I^0$ be the $(n+1)\times(n+1)$ matrix representation of the corresponding operator for $\Sp_{2n}$, and let $I^+$ and $I^-$ be the submatrices of $I$ given in \Cref{lem:general-from-special-matrix} which are the matrix representations of $I$ on $\left(\Ind_P^G\chi\right)^\chi\to\left(\Ind_P^G\chi^J\right)^\chi$ and $\left(\Ind_P^G\chi^J\right)^\chi\to\left(\Ind_P^G\chi\right)^\chi$, respectively.

    For example, if $n$ is even, then $I^+$ and $I^-$ are both the submatrices of $I$ consisting of the even rows and columns. Tracking through the proof of \Cref{thm:sp-quadratic-eigens} (and notably its input \Cref{prop:helper-matrix-even}), we will show that $I^+$ and $I^-$ are both diagonalizable with eigenvalues
    \[\left\{\beta(-1)^m(-1)^iq^{m^2+(m-i)^2}:0\le i\le m\right\},\]
    which completes the proof upon squaring our eigenvalues. Indeed, defining $A$ and $B$ as in \Cref{thm:sp-quadratic-eigens}, we see that $I^+=I^-$ is a submatrix of $A$, and the upper-triangularization of $B$ used a matrix $M$ which preserves the span of the even (and odd) basis vectors. Thus, the submatrix of $M$ consisting of the even rows and columns will still provide an upper-triangularization of $I^+=I^-$ (resulting in a submatrix of $B$!), and we can read off the diagonal as coming from the even $i$, which produces the above list of eigenvalues.

    For the remainder of the argument, we may take $n$ to be odd. Then $I^+$ is the submatrix of $I$ consisting of the even columns and odd rows, and $I^-$ is the submatrix of $I$ consisting of the odd columns and even rows. Arguing as above, we define $A$ and $B$ as in \Cref{thm:sp-quadratic-eigens} so that we see that $I$ is similar to $q^{\binom{n+1}2}A$, which in turn \Cref{prop:helper-matrix-odd} explains is similar to the block diagonal matrix with diagonal given by the $2\times2$ matrices
    \[\left\{\beta(-1)^m(-1)^iq^{\binom{2m+2}2-2\binom{m+1}2+(m-i)^2}\begin{bmatrix}
        & 1 \\
        \beta(-1)q^{-1-2i}
    \end{bmatrix}:0\le i\le m\right\}.\]
    The similarity $I\sim q^{\binom{n+1}2}B$ admittedly swaps even and odd rows and columns, but the last similarity preserves linear combinations of even and odd basis vectors by its construction. Thus, we see that the matrices $I^+$ and $I^-$ will be similar to the diagonal matrices achieved by reading off the diagonals in the block diagonal matrix described above (because the upper-triangularizations of $I^+$ and $I^-$ are achieved by the same matrix $M$ by construction in \Cref{prop:helper-matrix-odd}!). As such, computing the composite $I^-\circ I^+$ tells us that the eigenvalues are
    \[\left\{\beta(-1)q^{2m(m+1)+2(m-i)(m-i+1)+1}:0\le i\le m\right\}\]
    after a little simplification.
\end{proof}