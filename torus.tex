\documentclass{article}
\usepackage[utf8]{inputenc}

\newcommand{\nirpdftitle}{Torus Worksheet}
\usepackage{import}
\inputfrom{../notes}{nir}

\pagestyle{contentpage}

\title{Torus Worksheet}
\author{Nir Elber}
\date{\today}
\usepackage{graphicx}
\lhead{}
\rhead{\textit{TORUS WORKSHEET}}

\begin{document}

\maketitle

\setcounter{tocdepth}{4}
\tableofcontents

\section{The Bicyclic Case}
Set $G\simeq G_1\times G_2$ where $G_1=\langle\sigma_1\rangle\subseteq G$ and $G_2=\langle\sigma_2\rangle\subseteq G$ with $n_1\coloneqq\#G_1$ and $n_2\coloneqq\#G_2$. We define the elements
\[N_i\coloneqq\sum_{k=0}^{n_i-1}\sigma_i^k\qquad\text{and}\qquad T_i\coloneqq(\sigma_i-1)\]
for $i\in\{1,2\}$. Additionally, we define
\[N\coloneqq N_1N_2=\sum_{k=0}^{n_1-1}\sum_{\ell=0}^{n_2-1}\sigma_1^k\sigma_2^\ell.\]
We will also use the elements $N_i,T_i,N$ to mean the induced multiplication maps $\ZZ[G]\to\ZZ[G]$.
\begin{remark}
	We will freely use the facts that $\ker T_i=\im N_i$ and $\im T_i=\ker N_i$. These are not too hard to prove and follow similarly as in the cyclic case.
\end{remark}
We take a moment to define the maps
\begin{align*}
	\mathcal F_1\colon\frac{\ZZ[G]}{\im T_1}\times\frac{\ZZ[G]}{\im N}\times\frac{\ZZ[G]}{\im T_2} &\to \ZZ[G]\times\ZZ[G] \\
	(x,y,z) &\mapsto (N_2z+T_1y,N_1x-T_2y) \\
	\mathcal F_2\colon\ZZ[G]\times\ZZ[G] &\to \frac{\ZZ[G]}{\im T_1}\times\frac{\ZZ[G]}{\im N}\times\frac{\ZZ[G]}{\im T_2} \\
	(a,b) &\mapsto (T_2a,N_1a-N_2b,T_1b).
\end{align*}
Observe that $\mathcal F_1$ is in fact well-defined: if $x-x'=T_1a\in\im T_1$ and $y-y'=Nb\in\im N$ and $z-z'=T_2c\in\im T_2$, then
\begin{align*}
	N_2z &= N_2z' + N_2T_2c = N_2z' \\
	N_1x &= N_1x' + N_1T_1a = N_1x' \\
	T_1y &= T_1y' + T_1Nb = T_1y' \\
	T_2y &= T_2y' + T_2Nb = T_2y',
\end{align*}
so everything is independent of choice of representative.

\subsection{The Complex}
Now, we can chain $\mathcal F_1$ and $\mathcal F_2$ into an infinite sequence as follows.
\begin{equation}
	\cdots\stackrel{\mathcal F_1}\to\ZZ[G]\times\ZZ[G]\stackrel{\mathcal F_2}\to\frac{\ZZ[G]}{\im T_1}\times\frac{\ZZ[G]}{\im N}\times\frac{\ZZ[G]}{\im T_2}\stackrel{\mathcal F_1}\to\ZZ[G]\times\ZZ[G]\stackrel{\mathcal F_2}\to\cdots \label{eq:bicycliccomplex}
\end{equation}
We make a few preliminary observations about \autoref{eq:bicycliccomplex}.
\begin{lemma} \label{lem:getcomplex}
	The sequence of maps in \autoref{eq:bicycliccomplex} makes a complex.
\end{lemma}
\begin{proof}
	We need to show that $\mathcal F_1\circ\mathcal F_2=\mathcal F_2\circ\mathcal F_1=0$. This is a matter of force. On one hand,
	\begin{align*}
		(\mathcal F_1\circ\mathcal F_2)(a,b) &= \mathcal F_1(T_2a,N_1a-N_2b,T_1b) \\
		&= \big(N_2(T_1b)+T_1(N_1a-N_2b),N_1(T_2a)-T_2(N_1a-N_2b)\big) \\
		&= \big(N_2T_1b+T_1N_1a-T_1N_2b,N_1T_2a-T_2N_1a-T_2N_2b\big) \\
		&= \big(N_2T_1b-T_1N_2b,N_1T_2a-T_2N_2b\big) \\
		&= (0,0).
	\end{align*}
	On the other hand,
	\begin{align*}
		(\mathcal F_2\circ\mathcal F_1)(x,y,z) &= \mathcal F_2(N_2z+T_1y,N_1x-T_2y) \\
		&= \big(T_2(N_2z+T_1y),N_1(N_2z+T_1y)-N_2(N_1x-T_2y),T_1(N_1x-T_2y)\big) \\
		&= \big(T_2N_2z+T_2T_1y,N_1N_2z+N_1T_1y-N_2N_1x+N_2T_2y,T_1N_1x-T_1T_2y\big) \\
		&= \big(T_2T_1y,N_1N_2z-N_2N_1x,-T_1T_2y\big),
	\end{align*}
	which we can see represents $0$ in $\ZZ[G]/\im T_1\times\ZZ[G]/\im N\times\ZZ[G]/\im T_2$.
\end{proof}
\begin{lemma} \label{lem:almostexact}
	The complex \autoref{eq:bicycliccomplex} is exact at each $\ZZ[G]/\im T_1\times\ZZ[G]/\im N\times\ZZ[G]/\im T_2$.
\end{lemma}
\begin{proof}
	We need to show that $\ker\mathcal F_1=\im\mathcal F_2$. By \autoref{lem:getcomplex}, we already know that $\im\mathcal F_2\subseteq\ker\mathcal F_2$, so it remains to get the reverse inclusion. Well, suppose $(x,y,z)\in\ker\mathcal F_1$, which means that
	\[N_2z+T_1y=N_1x-T_2y=0.\]
	Now, note that because $x\in\ZZ[G]/\im T_1$, where $\sigma_1\equiv1\pmod{T_1}$, we have
	\[N_1x=\sum_{k=0}^{n_1-1}\sigma_1^kx\equiv n_1x.\]
	However,
	\[n_1\cdot N_2x=N_2(n_1x)=N_2N_1x=N_2(N_1x-T_2y)=0,\]
	so $N_2x=0$. Thus, $x\in\ker N_2=\im T_2$, so we may write $x=T_2a$ for some $a\in\ZZ[G]$. A similar argument (replacing $1$s with $2$s and adjusting some signs) shows that we may write $z=T_1b$ for some $b\in\ZZ[G]$.

	To show $(x,y,z)\in\im\mathcal F_2$, it remains to show that $y\equiv N_1a-N_2b\pmod N$. Well, notice that
	\begin{align*}
		T_1(N_1a-N_2b-y) &= -(N_2T_1b+T_1y) \\
		&= -(N_2z+T_1y) \\
		&= 0 \\
		T_2(N_1a-N_2b-y) &= N_1T_2a-T_2y \\
		&= N_1x-T_2y \\
		&= 0,
	\end{align*}
	so $N_1a-N_2b-y\in(\ker T_1)\cap(\ker T_2)$. To finish, we would like $(\ker T_1)\cap(\ker T_2)\subseteq\im N$. Well, if $\alpha\in(\ker T_1)\cap(\ker T_2)$, then we can write
	\[\alpha=\sum_{k=0}^{n_1-1}\sum_{\ell=0}^{n_2-1}a_{k,\ell}\sigma_1^k\sigma_2^\ell.\]
	Now, the condition $T_1\alpha=0$ implies that $a_{k,\ell}-a_{k-1,\ell}=0$ for each $k$, where indices are taken modulo $n_i$ as necessary; thus, $a_{k,\ell}$ is constant with respect to $k$. Similarly, $T_2\alpha=0$ implies that $a_{k,\ell}$ is constant with respect to $\ell$, so
	\[\alpha=\sum_{k=0}^{n_1-1}\sum_{\ell=0}^{n_2-1}a_{0,0}\sigma_1^k\sigma_2^\ell=Na_{0,0}\in\im N,\]
	as desired.
\end{proof}

\subsection{A Cocycle} \label{sec:cocycle}
The benefit to our analysis above is that we have a short exact sequence
\begin{equation}
	0\to\coker\mathcal F_2\stackrel{\mathcal F_1}\to\ZZ[G]\times\ZZ[G]\to\coker\mathcal F_1\to1. \label{eq:ses}
\end{equation}
In particular, the (induced) map $\mathcal F_1\colon\coker\mathcal F_2\to\ZZ[G]\times\ZZ[G]$ is injective by \autoref{lem:almostexact}.

In the future, we will want a $2$-cocycle in $Z^2(G,\coker\mathcal F_2)$, which we will create by tracking a boundary morphism through \autoref{eq:ses}. Thus, we want to start with a $1$-cocycle in $Z^2(G,\coker\mathcal F_1)$. For this, we define the notation
\[\sigma_i^{(a_i)}\coloneqq\sum_{k=0}^{a_i-1}\sigma_i^k\]
for $i\in\{1,2\}$ and any nonnegative integer $a_i\ge0$. For example, $\sigma_i^{(n_i)}=N_i$. We also note that
\[\sigma_i^{(a_i+b_i)}=\sigma_i^{(a_i)}+\sigma_i^{a_i}\sigma_i^{(b_i)},\]
which justifies our ``almost exponential'' notation.
\begin{lemma}
	Define $u\in C^1(G,\ZZ[G]\times\ZZ[G])$ by $u\left(\sigma_1^{a_1}\sigma_2^{a_2}\right)\coloneqq\left(\sigma_1^{a_1}\sigma_2^{(a_2)},\sigma_1^{(a_1)}\right)$. Then the induced $1$-cochain $\overline u\in C^1(G,\coker\mathcal F_1)$ is a $1$-cocycle.
\end{lemma}
\begin{proof}
	We need to show that $\overline{du}=d\overline u=0$; i.e., we need to show that $\im(du)\subseteq\im\mathcal F_1$. Well, pick up $a_i,b_i$ with $0\le a_i,b_i<n_i$, and set
	\[a_i+b_i=n_iq_i+r_i\]
	by the division algorithm so that $q_i\in\{0,1\}$ and $0\le r_i<n_i$. This lets us compute
	\begin{align*}
		(du)\left(\sigma_1^{a_1}\sigma_2^{a_2},\sigma_1^{b_1}\sigma_2^{b_2}\right) &= \sigma_1^{a_1}\sigma_2^{a_2}u\left(\sigma_1^{b_1}\sigma_2^{b_2}\right)-u\left(\sigma_1^{r_1}\sigma_2^{r_2}\right)+u\left(\sigma_1^{a_1}\sigma_2^{a_2}\right) \\
		&= \sigma_1^{a_1}\sigma_2^{a_2}\left(\sigma_1^{b_1}\sigma_2^{(b_2)},\sigma_1^{(b_1)}\right)-\left(\sigma_1^{r_1}\sigma_2^{(r_2)},\sigma_1^{(r_1)}\right)+\left(\sigma_1^{a_1}\sigma_2^{(a_2)},\sigma_1^{(a_1)}\right).
	\end{align*}
	Now, note
	\begin{align*}
		\left(\sigma_1^{r_1}\sigma_2^{(r_2)},\sigma_1^{(r_1)}\right) &= \left(\sigma_1^{a_1+b_1}\sigma_2^{(r_2)},\sigma_1^{(r_1)}\right) \\
		&= \left(\sigma_1^{a_1+b_1}\sigma_2^{(a_2+b_2)}-\sigma_1^{a_1+b_1}\sigma_2^{r_2}\sigma_2^{(n_2q_2)},\sigma_1^{(a_1+b_1)}-\sigma_1^{r_1}\sigma_1^{(n_1q_1)}\right) \\
		&= \left(\sigma_1^{a_1+b_1}\sigma_2^{(a_2+b_2)},\sigma_1^{(a_1+b_1)}\right)-\left(\sigma_1^{a_1+b_1}\sigma_2^{a_2+b_2}\cdot q_2N_2,\sigma_1^{a_1+b_1}\cdot q_1N_1\right).
	\end{align*}
	On the other hand,
	\begin{align*}
		\sigma_1^{a_1}\sigma_2^{a_2}\left(\sigma_1^{b_1}\sigma_2^{(b_2)},\sigma_1^{(b_1)}\right) &= \left(\sigma_1^{a_1+b_1}\sigma_2^{a_2}\sigma_2^{(b_2)},\sigma_1^{a_1}\sigma_2^{a_2}\sigma_1^{(b_1)}\right) \\
		&= \left(\sigma_1^{a_1+b_1}\sigma_2^{a_2}\sigma_2^{(b_2)},\sigma_1^{a_1}\sigma_1^{(b_1)}\right)+\left(0,\sigma_1^{a_1}\left(\sigma_2^{a_2}-1\right)\sigma_1^{(b_1)}\right) \\
		&= \left(\sigma_1^{a_1+b_1}\sigma_2^{a_2}\sigma_2^{(b_2)},\sigma_1^{a_1}\sigma_1^{(b_1)}\right)+\left(0,\sigma_1^{a_1}\sigma_2^{(a_2)}\sigma_1^{(b_1)}T_2\right),
	\end{align*}
	and
	\begin{align*}
		\left(\sigma_1^{a_1}\sigma_2^{(a_2)},\sigma_1^{(a_1)}\right) &= \left(\sigma_1^{a_1+b_1}\sigma_2^{(a_1)},\sigma_1^{(a_1)}\right)+\left(\sigma_1^{a_1}\left(1-\sigma_1^{b_1}\right)\sigma_2^{(a_2)},0\right) \\
		&= \left(\sigma_1^{a_1+b_1}\sigma_2^{(a_1)},\sigma_1^{(a_1)}\right)+\left(-\sigma_1^{a_1}\sigma_1^{(b_1)}\sigma_2^{(a_2)}T_1,0\right).
	\end{align*}
	Synthesizing, we see that $\sigma_2^{(a_2+b_2)}=\sigma_2^{(a_2)}+\sigma_1^{a_2}\sigma_2^{(b_2)}$ causes the ``leading'' terms to vanish, leaving us with
	\begin{align*}
		(du)\left(\sigma_1^{a_1}\sigma_2^{a_2},\sigma_1^{b_1}\sigma_2^{b_2}\right) &= \left(0,\sigma_1^{a_1}\sigma_2^{(a_2)}\sigma_1^{(b_1)}T_2\right)+\left(q_2\sigma_1^{a_1+b_1}\sigma_2^{a_2+b_2}N_2,q_1\sigma_1^{a_1+b_1}N_1\right)+\left(-\sigma_1^{a_1}\sigma_1^{(b_1)}\sigma_2^{(a_2)}T_1,0\right) \\
		&= \mathcal F_1\left(q_1\sigma_1^{a_1+b_1},\quad-\sigma_1^{a_1}\sigma_1^{(b_1)}\sigma_2^{(a_2)},\quad q_2\sigma_1^{a_1+b_1}\sigma_2^{a_2+b_2}\right),
	\end{align*}
	which finishes.
\end{proof}
In particular, from our above computation of $du$, we can pull back along $\mathcal F_1$ to find $\delta u\in Z^2(G,\coker\mathcal F_2)$ is given by
\[(\delta u)\left(\sigma_1^{a_1}\sigma_2^{a_2},\sigma_1^{b_1}\sigma_2^{b_2}\right)=\left(q_1\sigma_1^{a_1+b_1},\quad-\sigma_1^{a_1}\sigma_1^{(b_1)}\sigma_2^{(a_2)},\quad q_2\sigma_1^{a_1+b_1}\sigma_2^{a_2+b_2}\right).\]
Simplifying a bit, we note that $\sigma_i\equiv1\pmod{T_i}$, so we can write this as
\begin{equation}
	(\delta u)\left(\sigma_1^{a_1}\sigma_2^{a_2},\sigma_1^{b_1}\sigma_2^{b_2}\right)=\left(q_1,\quad-\sigma_1^{a_1}\sigma_1^{(b_1)}\sigma_2^{(a_2)},\quad q_2\sigma_1^{a_1+b_1}\right). \label{eq:deltau}
\end{equation}
We will use $\delta u$ again shortly.

\subsection{Triples}
We have now built enough machinery to be able to talk about triples; for the rest of the article, we will have $G=\op{Gal}(L/K)$ for some bicyclic extension $L/K$. Let $S$ be the set of triples. The following is the main idea.
\begin{lemma} \label{lem:tripleisomorphism}
	There is a natural isomorphism between morphisms $f\in\op{Hom}_{\ZZ[G]}\left(\coker\mathcal F_2,L^\times\right)$ and the set $S$ of triples $(\alpha_1,\beta,\alpha_2)$ given by
	\[\varphi\colon f\mapsto\big(f(1,0,0),f(0,1,0),f(0,0,1)\big).\]
\end{lemma}
\begin{proof}
	For brevity, set $e_1\coloneqq(1,0,0)$ and $e_2\coloneqq(0,1,0)$ and $e_3\coloneqq(0,0,1)$ to be elements of $\ZZ[G]$. Now, let $\varphi\colon\op{Hom}_{\ZZ[G]}(\coker\mathcal F_2,L^\times)\to S$ be defined by
	\[\varphi\colon f\mapsto\big(f(e_1),f(e_2),f(e_3)\big).\]
	The main check here is that $\varphi$ is well-defined. We have the following checks.
	\begin{itemize}
		\item We check that our elements live in the correct fields. Note that $f(e_1),f(e_2),f(e_3)\in L^\times$. Further,
		\[\sigma_1(f(e_1))=f(\sigma_1,0,0)=f(1,0,0)=f(e_1)\]
		where the main step is that $\sigma_1\equiv1\pmod{T_1}$ enforces $(\sigma_1,0,0)=(1,0,0)$. It follows that $f(e_1)\in L^{\langle\sigma_1\rangle}$. An analogous argument shows that $f(e_3)\in L^{\langle\sigma_2\rangle}$.
		\item We check the relations. The main point is that
		\begin{align*}
			(0,N_1,0) &\equiv (-T_2,0,0) \pmod{\im\mathcal F_2}, \\
			(0,N_2,0) &\equiv (0,0,T_1) \pmod{\im\mathcal F_2}.
		\end{align*}
		As such, observe that
		\begin{align*}
			\op N_{L/L^{\langle\sigma_1\rangle}}f(0,1,0) &= f(0,N_1,0) \\
			&= f(-T_2,0,0) \\
			&= f(1,0,0)/\sigma_2f(1,0,0).
		\end{align*}
		The other relation is similar: observe
		\begin{align*}
			\op N_{L/L^{\langle\sigma_2\rangle}}f(0,1,0) &= f(0,N_2,0) \\
			&= f(0,0,T_1) \\
			&= \sigma_1f(0,0,1)/f(0,0,1),
		\end{align*}
		which is what we wanted.
	\end{itemize}
	We also remark that $\varphi$ is homomorphic because the operations on both $\op{Hom}_{\ZZ[G]}(\coker\mathcal F_2,L^\times)$ and $S$ are both defined pointwise.

	Now, in the other direction, we define $\psi\colon S\to\op{Hom}_{\ZZ[G]}(\coker\mathcal F_2,L^\times)$ by
	\[\psi\colon(\alpha_1,\beta,\alpha_2)\mapsto\big((z_1,z_2,z_3)\mapsto z_1\alpha\cdot z_2\beta\cdot z_3\alpha_2\big).\]
	Again, the main check is that $\psi$ is well-defined. Well, given a triple $(\alpha_1,\beta,\alpha_2)\in S$, we show that $\psi(\alpha_1,\beta,\alpha_2)$ is well-defined as a $G$-module homomorphism. To begin, we note that we can start with a function $f\colon\ZZ[G]\times\ZZ[G]\times\ZZ[G]\to L^\times$ by
	\[f\colon(z_1,z_2,z_3)\mapsto z_1\alpha_1\cdot z_2\beta\cdot z_3\alpha_2.\]
	This can be quickly checked to be a $G$-module homomorphism: note that any $\alpha\in L^\times$ makes $z\mapsto z\alpha$ is a $G$-module homomorphism $L^\times\to L^\times$, and the above is more or less a linear combination of such $G$-module homomorphisms.

	We now investigate $\ker f$. Note that
	\[f(T_1,0,0)=T_1\alpha=(\sigma_1\alpha_1)/\alpha_1=1\]
	because $\alpha_1\in L^{\langle\sigma_1\rangle}$. An identical argument shows that $f(0,0,T_2)=1$. Further,
	\[f(0,N,0)=N\beta=\op N_{L/K}(\beta)=1.\]
	In total, we see that $\im T_1\times\im N\times\im T_2\subseteq\ker f$, so we get to induce a function $\overline f\colon\ZZ[G]/\im T_1\times\ZZ[G]/\im N\times\ZZ[G]\im T_2$ by
	\[\overline f\colon(z_1,z_2,z_3)\mapsto z_1\alpha_1\cdot z_2\beta\cdot z_3\alpha_2.\]
	Further, we note that $(a,b)\in\ZZ[G]\times\ZZ[G]$ will have
	\begin{align*}
		\overline f(\mathcal F_2(a,b)) &= f(T_2a,N_1a-N_2b,T_1b) \\
		&= (T_2a)\alpha_1\cdot(N_1a-N_2b)\beta\cdot(T_1b)\alpha_2 \\
		&= a\left(\frac{\sigma_2\alpha_1}{\alpha_1}\cdot\op N_{L/L^{\langle\sigma_1\rangle}}(\beta)\right)\cdot b\left(\frac{\sigma_1\alpha_2}{\alpha_2}\cdot\op N_{L/L^{\langle\sigma_2\rangle}}(\beta^{-1})\right),
	\end{align*}
	which we see goes to $1$ by the relations between $\beta$ and the $\alpha$s. Thus, we do indeed a $G$-module homomorphism $\coker\mathcal F_2\to S$ defined as $\psi(\alpha_1,\beta,\alpha_2)$ requires.

	We now finish by stating that $\varphi$ and $\psi$ are mutually inverse, essentially by construction. Checking that $(\varphi\circ\psi)(\alpha_1,\beta,\alpha_2)=(\alpha_1,\beta,\alpha_2)$ is a matter of plugging everything through; on the other hand, to show $(\psi\circ\varphi)(f)=f$, we note that
	\[f(z_1,z_2,z_3)=z_1f(1,0,0)\cdot z_2f(0,1,0)\cdot z_3f(0,0,1)=\psi(\varphi(f))\]
	because $f$ is a $G$-module homomorphism.
\end{proof}
In fact, we can even correctly account for the equivalence classes of triples; let $S_0$ denote the subgroup of triples coming from the trivial gerb.
\begin{proposition}
	The isomorphism $\varphi$ of \autoref{lem:tripleisomorphism} descends to a natural isomorphism
	\[\overline\varphi\colon\widehat H^0(G,\op{Hom}_\ZZ(\coker\mathcal F_2,L^\times))\simeq S/S_0.\]
\end{proposition}
\begin{proof}
	Observe that
	\[\widehat H^0(G,\op{Hom}_\ZZ(\coker\mathcal F_2,L^\times))=\frac{\op{Hom}_\ZZ(\coker\mathcal F_2,L^\times)^G}{N_G(\op{Hom}_\ZZ(\coker\mathcal F_2,L^\times))},\]
	so to pass through $\varphi$ appropriately, we have to show that $\varphi(\im N_G)=S_0$.\todo{Show this.}
\end{proof}
To tie up loose ends, we promised to use $\delta u$ from \autoref{sec:cocycle}, so here is how.
\begin{proposition}
	Fix a triple $(\alpha_1,\beta,\alpha_2)\in S$. Then the cocycle in $Z^2(G,L^\times)$ corresponding to $(\alpha_1,\beta,\alpha_2)$ is
	\[(\delta u)\cup\varphi^{-1}(\alpha_1,\beta,\alpha_2),\]
	where we are implicitly passing through the evaluation map $\coker\mathcal F_2\otimes_{\ZZ}\op{Hom}_\ZZ(\coker\mathcal F_2,L^\times)\to L^\times$.
\end{proposition}
\begin{proof}
	The cup product on inhomogeneous cochains is given by
	\begin{align*}
		\big((\delta u)\cup\varphi^{-1}(\alpha_1,\beta,\alpha_2)\big)\left(\sigma_1^{a_1}\sigma_2^{a_2},\sigma_1^{b_1}\sigma_2^{b_2}\right) &= (\delta u)\left(\sigma_1^{a_1}\sigma_2^{a_2},\sigma_1^{b_1}\sigma_2^{b_2}\right)\otimes_\ZZ\sigma_1^{a_1+b_1}\sigma_2^{a_2+b_2}\varphi^{-1}(\alpha_1,\beta,\alpha_2) \\
		&= (\delta u)\left(\sigma_1^{a_1}\sigma_2^{a_2},\sigma_1^{b_1}\sigma_2^{b_2}\right)\otimes_\ZZ\varphi^{-1}(\alpha_1,\beta,\alpha_2).
	\end{align*}
	Passing through evaluation, we will get an element in $L^\times$ which looks like
	\begin{align*}
		q_1\alpha_1\cdot-\sigma_1^{a_1}\sigma_1^{(b_1)}\sigma_2^{(a_2)}\beta\cdot q_2\sigma_1^{a_1+b_1}\alpha_2,
	\end{align*}
	which is exactly what we wanted (after a little rearrangement).
\end{proof}

\end{document}