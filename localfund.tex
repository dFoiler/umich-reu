\documentclass{article}
\usepackage[utf8]{inputenc}

\newcommand{\nirpdftitle}{Local Fundamental Class}
\usepackage{import}
\inputfrom{../notes}{nir}

\pagestyle{contentpage}

\title{The Local Fundamental Class}
\author{Nir Elber}
\date{\today}
\usepackage{graphicx}
\lhead{}
\rhead{\textit{FUNDAMENTAL CLASS}}

\begin{document}

\maketitle

\begin{abstract}
	\noindent We compute the local fundamental class of the extension $\QQ_p(\zeta_N)/\QQ_p$ when $p$ is an odd prime. This requires making a number of standard group cohomology constructions fully explicit in the process.
\end{abstract}

\setcounter{tocdepth}{4}
\tableofcontents

\section{Set-Up} \label{sec:setup}
Fix an extension local fields $L/K$. Then let $K_m$ be the largest unramified subextension, which we will give degree $m$; let $\overline\sigma_K\in\op{Gal}(L/K)$ denote the Frobenius automorphism, which lets us set
\[K_{\pi,\nu}\coloneqq L^{\langle\overline\sigma_K\rangle}.\]
In particular, $K_{\pi,\nu}/K$ is totally ramified because, for example, the residue fields of $K_{\pi,\nu}$ and $K$ have the same order.
\begin{example}
	For $K=\QQ_p$, we can take $K_m=\QQ_p\left(\zeta_{p^m-1}\right)$ and $K_{\pi,\nu}=\QQ_p\left(\zeta_{p^\nu}\right)$.
\end{example}
For some fixed $\nu$ and $m$, we let $L\coloneqq K_{\pi,\nu}K_m$. This gives us the following tower of fields.
% https://q.uiver.app/?q=WzAsNCxbMSwyLCJLIl0sWzAsMSwiS197XFxwaSxcXG51fSJdLFsyLDEsIktfbSJdLFsxLDAsIkwiXSxbMCwxLCIiLDAseyJzdHlsZSI6eyJoZWFkIjp7Im5hbWUiOiJub25lIn19fV0sWzEsMywiIiwwLHsic3R5bGUiOnsiaGVhZCI6eyJuYW1lIjoibm9uZSJ9fX1dLFswLDIsIiIsMix7InN0eWxlIjp7ImhlYWQiOnsibmFtZSI6Im5vbmUifX19XSxbMiwzLCIiLDIseyJzdHlsZSI6eyJoZWFkIjp7Im5hbWUiOiJub25lIn19fV1d&macro_url=https%3A%2F%2Fraw.githubusercontent.com%2FdFoiler%2Fnotes%2Fmaster%2Fnir.tex
\[\begin{tikzcd}
	& L \\
	{K_{\pi,\nu}} && {K_m} \\
	& K
	\arrow[no head, from=3-2, to=2-1]
	\arrow[no head, from=2-1, to=1-2]
	\arrow[no head, from=3-2, to=2-3]
	\arrow[no head, from=2-3, to=1-2]
\end{tikzcd}\]
To help us a little later, we will assume that the extension $L/K$ is neither totally ramified nor unramified.
\begin{remark}
	Assuming that $L/K$ is neither totally ramified nor unramified is not actually very big of a problem because we can apply inflation to $u_{L/K}$ to read off the fundamental class for the totally ramified and unramified parts.
\end{remark}
We provide some quick commentary on these extensions.
\begin{itemize}
	\item The extension $K_m/K$ is unramified of degree $f\coloneqq m$; note we are assuming $1<f<n$. Its Galois group is thus generated by the Frobenius element defined by $\overline\sigma_K$.
	\item The extension $K_{\pi,\nu}/K$ is totally ramified of degree $[K_{\pi,\nu}:K]$. Because we are assuming this Galois group is abelian, we may write
	\[\op{Gal}(K_{\pi,\nu}/K)\simeq\Gamma_1\times\cdots\times\Gamma_t\]
	where $\Gamma_i=\langle\tau_i\rangle\subseteq\op{Gal}(K_{\pi,\nu}/K)$ is a cyclic group of order $n_i$.
	% Its Galois group is thus isomorphic to $\left(\ZZ/p^\nu\ZZ\right)^\times$, where the isomorphism takes $x\in\left(\ZZ/p^\nu\ZZ\right)^\times$ to
	% \[\sigma_x\colon\zeta_{p^\nu}\mapsto\zeta_{p^\nu}^{x^{-1}}.\]
	% The group $\left(\ZZ/p^\nu\ZZ\right)^\times$ is cyclic, so we will fix a generator $x$, which gives us a distinguished generator $\sigma_x\in\op{Gal}\left(\QQ(\zeta_{p^\nu})/\QQ_p\right)$.
	\item Because $K_{\pi,\nu}/K$ is totally ramified and $K_m/K$ is unramified, we have that the fields $K_{\pi,\nu}$ and $K_m$ are linearly disjoint over $K$. As such, $L=K_{\pi,\nu}K_m$ has
	\begin{align*}
		\op{Gal}(L/K_{\pi,\nu}) &\simeq \op{Gal}(K_m/K)=\langle\overline\sigma_K\rangle \\
		\op{Gal}(L/K_m) &\simeq \op{Gal}(K_{\pi,\nu}/K)=\Gamma_1\times\cdots\times\Gamma_t \\
		\op{Gal}(L/K) &\simeq \op{Gal}(K_m/K)\times\op{Gal}(K_{\pi,\nu}/K)=\langle\overline\sigma_K\rangle\times\Gamma_1\times\cdots\times\Gamma_t.
	\end{align*}
	In light of these isomorphisms, we will upgrade $\overline\sigma_K$ to the automorphism of $L/K$ which restricts properly on $K_m/K$ and fixing $K_{\pi,\nu}$; we do analogously for the $\tau_i$. We also acknowledge that our degree is
	\[n\coloneqq[L:K]=[K_m:K]\cdot[K_{\pi,\nu}:K]=f\cdot[K_{\pi,\nu}:K].\]
\end{itemize}
For brevity, we will also set $L_i\coloneqq L^{\langle\tau_i\rangle}$ for each $i$, which makes the fields under $L$ look like the following.
% https://q.uiver.app/?q=WzAsNyxbMSwzLCJLIl0sWzIsMiwiS19tIl0sWzAsMSwiTF8wPUtfe1xccGksXFxudX0iXSxbMSwwLCJMIl0sWzIsMSwiTF8xIl0sWzMsMSwiXFxjZG90cyJdLFs0LDEsIkxfdCJdLFswLDEsIiIsMix7InN0eWxlIjp7ImhlYWQiOnsibmFtZSI6Im5vbmUifX19XSxbMiwzLCIiLDAseyJzdHlsZSI6eyJoZWFkIjp7Im5hbWUiOiJub25lIn19fV0sWzAsMiwiIiwwLHsic3R5bGUiOnsiaGVhZCI6eyJuYW1lIjoibm9uZSJ9fX1dLFs0LDMsIiIsMix7InN0eWxlIjp7ImhlYWQiOnsibmFtZSI6Im5vbmUifX19XSxbMSw0LCIiLDIseyJzdHlsZSI6eyJoZWFkIjp7Im5hbWUiOiJub25lIn19fV0sWzEsNiwiIiwyLHsic3R5bGUiOnsiaGVhZCI6eyJuYW1lIjoibm9uZSJ9fX1dLFs2LDMsIiIsMix7InN0eWxlIjp7ImhlYWQiOnsibmFtZSI6Im5vbmUifX19XV0=&macro_url=https%3A%2F%2Fraw.githubusercontent.com%2FdFoiler%2Fnotes%2Fmaster%2Fnir.tex
\[\begin{tikzcd}
	& L \\
	{K_{\pi,\nu}} && {L_1} & \cdots & {L_t} \\
	&& {K_m} \\
	& K
	\arrow[no head, from=4-2, to=3-3]
	\arrow[no head, from=2-1, to=1-2]
	\arrow[no head, from=4-2, to=2-1]
	\arrow[no head, from=2-3, to=1-2]
	\arrow[no head, from=3-3, to=2-3]
	\arrow[no head, from=3-3, to=2-5]
	\arrow[no head, from=2-5, to=1-2]
\end{tikzcd}\]
In particular, $\op{Gal}(L/L_i)$ is cyclic for each $i$.

Now, the main idea in the computation is to use an unramified extension $M\coloneqq K_n$ of the same degree as $L/K$. This modifies our diagram of fields as follows.
% https://q.uiver.app/?q=WzAsNixbMSwzLCJLIl0sWzAsMiwiS197XFxwaSxcXG51fSJdLFsxLDAsIk1MIl0sWzAsMSwiTCJdLFsxLDIsIktfbSJdLFsyLDEsIk0iXSxbMCwxLCJcXHRleHR7cmFtfSIsMCx7InN0eWxlIjp7ImhlYWQiOnsibmFtZSI6Im5vbmUifX19XSxbMSwzLCJcXHRleHR7dW5yfSIsMCx7InN0eWxlIjp7ImhlYWQiOnsibmFtZSI6Im5vbmUifX19XSxbMCw0LCJcXHRleHR7dW5yfSIsMix7InN0eWxlIjp7ImhlYWQiOnsibmFtZSI6Im5vbmUifX19XSxbNCw1LCJcXHRleHR7dW5yfSIsMix7InN0eWxlIjp7ImhlYWQiOnsibmFtZSI6Im5vbmUifX19XSxbMywyLCJcXHRleHR7dW5yfSIsMCx7InN0eWxlIjp7ImhlYWQiOnsibmFtZSI6Im5vbmUifX19XSxbNSwyLCJcXHRleHR7cmFtfSIsMix7InN0eWxlIjp7ImhlYWQiOnsibmFtZSI6Im5vbmUifX19XSxbNCwzLCJcXHRleHR7cmFtfSIsMix7InN0eWxlIjp7ImhlYWQiOnsibmFtZSI6Im5vbmUifX19XV0=&macro_url=https%3A%2F%2Fraw.githubusercontent.com%2FdFoiler%2Fnotes%2Fmaster%2Fnir.tex
\[\begin{tikzcd}
	& ML \\
	L && M \\
	{K_{\pi,\nu}} & {K_m} \\
	& K
	\arrow["{\text{ram}}", no head, from=4-2, to=3-1]
	\arrow["{\text{unr}}", no head, from=3-1, to=2-1]
	\arrow["{\text{unr}}"', no head, from=4-2, to=3-2]
	\arrow["{\text{unr}}"', no head, from=3-2, to=2-3]
	\arrow["{\text{unr}}", no head, from=2-1, to=1-2]
	\arrow["{\text{ram}}"', no head, from=2-3, to=1-2]
	\arrow["{\text{ram}}"', no head, from=3-2, to=2-1]
\end{tikzcd}\]
We have labeled the unramified extensions by ``$\textrm{unr}$'' and the totally ramified extensions by ``$\textrm{ram}$.''

% For brevity, we set $K\coloneqq\QQ_p$ and $L\coloneqq\QQ_p(\zeta_N)$ and $M\coloneqq\QQ_p(\zeta_{N'})$ so that $ML=\QQ_p(\zeta_N,\zeta_{N'})$. This abbreviates our diagram into the following.
% % https://q.uiver.app/?q=WzAsNCxbMSwzLCJLIl0sWzAsMSwiTCJdLFsyLDEsIk0iXSxbMSwwLCJNTCJdLFsxLDMsIiIsMCx7InN0eWxlIjp7ImhlYWQiOnsibmFtZSI6Im5vbmUifX19XSxbMCwxLCIiLDAseyJzdHlsZSI6eyJoZWFkIjp7Im5hbWUiOiJub25lIn19fV0sWzAsMiwiIiwyLHsic3R5bGUiOnsiaGVhZCI6eyJuYW1lIjoibm9uZSJ9fX1dLFsyLDMsIiIsMix7InN0eWxlIjp7ImhlYWQiOnsibmFtZSI6Im5vbmUifX19XV0=&macro_url=https%3A%2F%2Fraw.githubusercontent.com%2FdFoiler%2Fnotes%2Fmaster%2Fnir.tex
% \[\begin{tikzcd}
% 	& ML \\
% 	L && M \\
% 	\\
% 	& K
% 	\arrow[no head, from=2-1, to=1-2]
% 	\arrow[no head, from=4-2, to=2-1]
% 	\arrow[no head, from=4-2, to=2-3]
% 	\arrow[no head, from=2-3, to=1-2]
% \end{tikzcd}\]
As before, we provide some comments on the field extensions.
\begin{itemize}
	\item The extension $M/K$ is unramified of degree $n$. As before, its Galois group is cyclic, generated by the Frobenius element $\sigma_K$. Observe that $\sigma_K$ restricted to $K_m$ is $\overline\sigma_K$, explaining our notation. In particular, $\sigma_K$ has order $n$, but $\overline\sigma_K$ has order $f<n$.
	\item As before, note that $K_{\pi,\nu}$ and $M$ are linearly disjoint over $K$ because $K_{\pi,\nu}/K$ is totally ramified while $M/K$ is unramified. As such, we may say that
	\begin{align*}
		\op{Gal}(ML/M) &\simeq \op{Gal}(K_{\pi,\nu}/K) = \Gamma_1\times\cdots\times\Gamma_t \\
		\op{Gal}(ML/K_{\pi,\nu}) &\simeq \op{Gal}(M/K) = \langle\sigma_K\rangle \\
		\op{Gal}(ML/K) &\simeq \op{Gal}(M/K)\times\op{Gal}(K_{\pi,\nu}/K) = \langle\sigma_K\rangle\times\Gamma_1\times\cdots\times\Gamma_t.
	\end{align*}
	Again, we will upgrade $\sigma_K$ and the $\tau_i$ to their corresponding automorphisms on any subfield of $ML$.
	\item We take a moment to compute
	\begin{align*}
		\op{Gal}(ML/L) &\simeq \left\{\sigma_K^{a}\tau\in\op{Gal}(ML/K):\sigma_K^{a}\tau|_L=\id_L\right\}.
	\end{align*}
	Because $L$ is $K_{\pi,\nu}K_m$, it suffices to fix each of these fields individually. Well, to fix $K_{\pi,\nu}$, we need $\tau$ to vanish, so we might as well force $\tau=1$. But to fix $K_m$, we need $\sigma_K^{a}|_{K_m}=\overline\sigma_K^{a}$ to be the identity, so we are actually requiring that $f\mid a$ here. As such,
	\[\op{Gal}(ML/L)=\langle\sigma_K^f\rangle.\]
\end{itemize}
These comments complete the Galois-theoretic portion of the analysis.

\section{Idea}
We will begin by briefly describe the outline for the computation. For a finite extension of local fields $L/K$, let $u_{L/K}\in H^2(L/K)$ denote the fundamental class.

Now, take variables as in our set-up in \autoref{sec:setup}. The main idea is to translate what we know about the unramified extension $M/K$ over to the general extension $L/K$. In particular, we are able to compute the fundamental class $u_{M/K}\in H^2(M/K)$, so we observe that
\[\op{Inf}_{M/K}^{ML/K}u_{M/K}=[ML:M]u_{M/K}=n\cdot u_{ML/K}=[ML:L]u_{ML/L}=\op{Inf}_{L/K}^{ML/K}u_{L/K}.\]
As such, we will be able to compute $u_{L/K}$ as long as we are able to invert the inflation map $\op{Inf}\colon H^2(L/K)\to H^2(ML/K)$. This is not actually very easy to do in general,\footnote{The difficulty comes from the fact that a generic cocycle might be off from an inflated cocycle by some truly hideous coboundary.} but we are in luck because this inflation map here comes from the Inflation--Restriction exact sequence
\[0\to H^2(L/K)\stackrel{\op{Inf}}\to H^2(ML/K)\stackrel{\op{Res}}\to H^2(ML/L).\]
The argument for the Inflation--Restriction exact sequence is an explicit computation on cocycles (involving some dimension shifting), but it can be tracked backwards to give the desired cocycle.

\section{Computation}
In this section we record the details of the computation.

\subsection{Group Cohomology}
Throughout this section, $G$ will be a group (usually finite) and $H\subseteq G$ will be a subgroup (usually normal). We denote $\ZZ[G]$ by the group ring and $I_G\subseteq\ZZ[G]$ by the augmentation ideal, defined as the kernel of the map $\varepsilon\colon\ZZ[G]\to\ZZ$ which sends $g\mapsto1$ for all $g\in G$.

We begin by recalling the statement of the Inflation--Restriction exact sequence.
\begin{theorem}[Inflation--Restriction] \label{thm:infres}
	Let $G$ be a finite group with normal subgroup $H\subseteq G$. Given a $G$-module $A$, suppose that the $H^i(H,A)=0$ for $1\le i<q$ for some index $q\ge1$. Then the sequence
	\[0\to H^q\left(G/H,A^H\right)\stackrel{\op{Inf}}\to H^q(G,A)\stackrel{\op{Res}}\to H^q(H,A)\]
	is exact.
\end{theorem}
\begin{proof}[Sketch]
	The proof is by induction on $q$, via dimension shifting. For $q=1$, we can just directly check this on $1$-cocycles. The main point is the exactness at $H^q(G,A)$: if $c\in Z^1(G,A)$ has $\op{Res}(c)\in B^1(H,A)$, then find $a\in A$ with
	\[\op{Res}(c)(a)\coloneqq h\cdot a-a.\]
	As such, we define $f_a\in B^1(G,A)$ by $f_a(g)\coloneqq g\cdot a-a$, which implies that $c-f_a$ vanishes on $H$. It is then possible to stare at the $1$-cocycle condition
	\[(c-f_a)(gg')=(c-f_a)(g)+g\cdot(c-f_a)(g')\]
	to check that $c-f_a$ only depends on the cosets of $H$ (e.g., by taking $g'\in H$) and that $\im(c-f_a)\subseteq A^H$ (e.g., by taking $g\in H$).

	For $q>1$, we use dimension shifting via the following lemma.
	\begin{lemma}[Dimension shifting] \label{lem:dimshift}
		Let $G$ be a group with subgroup $H\subseteq G$. Given a $G$-module $A$, all indices $q\ge1$ have
		\[\delta\colon H^q(H,\op{Hom}_\ZZ(I_G,A))\simeq H^{q+1}(H,A).\]
	\end{lemma}
	\begin{proof}[Sketch]
		Recall that we have the short exact sequence of $\ZZ[H]$-modules
		\[0\to I_G\to\ZZ[G]\to\ZZ\to0.\]
		In fact, this short exact sequence splits over $\ZZ$, so it will still be short exact after applying $\op{Hom}_\ZZ(-,A)$, which gives the short exact sequence
		\[0\to A\to\op{Hom}_\ZZ(\ZZ[G],A)\to\op{Hom}_\ZZ(I_G,A)\to0\]
		of $\ZZ[H]$-modules. The result now follows from the long exact sequence of cohomology upon noting that $\op{Hom}_\ZZ(\ZZ[G],A)$ is coinduced and hence acyclic for cohomology.
	\end{proof}
	Using the above lemma, we have the following the commutative diagram with vertical arrows which are isomorphisms.
	% https://q.uiver.app/?q=WzAsOCxbMCwwLCIwIl0sWzAsMSwiMCJdLFsxLDAsIkhecVxcbGVmdChHL0gsXFxvcHtIb219X1xcWlooSV9HLEEpXkhcXHJpZ2h0KSJdLFsxLDEsIkhee3ErMX1cXGxlZnQoRy9ILEFeSFxccmlnaHQpIl0sWzIsMSwiSF57cSsxfShHLEEpIl0sWzIsMCwiSF5xKEcsXFxvcHtIb219X1xcWlooSV9HLEEpKSJdLFszLDEsIkhee3ErMX0oSCxBKSJdLFszLDAsIkhecShILFxcb3B7SG9tfV9cXFpaKElfRyxBKSkiXSxbMSwzXSxbMyw0XSxbNCw2XSxbMiwzLCJcXGRlbHRhIl0sWzUsNCwiXFxkZWx0YSJdLFs3LDYsIlxcZGVsdGEiXSxbNSw3XSxbMiw1XSxbMCwyXV0=&macro_url=https%3A%2F%2Fraw.githubusercontent.com%2FdFoiler%2Fnotes%2Fmaster%2Fnir.tex
	\[\begin{tikzcd}
		0 & {H^q\left(G/H,\op{Hom}_\ZZ(I_G,A)^H\right)} & {H^q(G,\op{Hom}_\ZZ(I_G,A))} & {H^q(H,\op{Hom}_\ZZ(I_G,A))} \\
		0 & {H^{q+1}\left(G/H,A^H\right)} & {H^{q+1}(G,A)} & {H^{q+1}(H,A)}
		\arrow[from=2-1, to=2-2]
		\arrow[from=2-2, to=2-3]
		\arrow[from=2-3, to=2-4]
		\arrow["\delta", from=1-2, to=2-2]
		\arrow["\delta", from=1-3, to=2-3]
		\arrow["\delta", from=1-4, to=2-4]
		\arrow[from=1-3, to=1-4]
		\arrow[from=1-2, to=1-3]
		\arrow[from=1-1, to=1-2]
	\end{tikzcd}\]
	The top row is exact by the inductive hypothesis, so the bottom row is therefore also exact.
\end{proof}
Our goal is to make the above proof explicit in the case of $q=2$, which is the only reason we sketched the above proofs at all. We begin by making the dimension shifting explicit.
\begin{lemma} \label{lem:explicitdimshift}
	Let $G$ be a group with subgroup $H\subseteq G$, and let $\{g_\alpha\}_{\alpha\in\lambda}$ be coset representatives for $H\backslash G$. Now, given a $G$-module $A$, the maps
	\begin{align*}
		\delta_H\colon Z^1(H,\op{Hom}_\ZZ(I_G,A))&\to Z^2(H,A) \\
		c&\mapsto\left[(h,h')\mapsto h\cdot c(h')(h^{-1}-1)\right] \\
		\left[h\mapsto\big((h'g_\bullet-1)\mapsto h'\cdot u((h')^{-1},h)\big)\right]&\mapsfrom u
	\end{align*}
	are group homomorphisms which descend to the isomorphism $\overline\delta\colon H^1(H,\op{Hom}_\ZZ(I_G,A))\simeq H^2(H,A)$ of \autoref{lem:dimshift}. The map $\delta$ above is surjective, and the reverse map is a section; when $H=G$, these are isomorphisms.
\end{lemma}
\begin{proof}
	We begin by noting that our short exact sequence can be written more explicitly as follows.
	% https://q.uiver.app/?q=WzAsOSxbMCwwLCIwIl0sWzEsMCwiQSJdLFsyLDAsIlxcb3B7SG9tfV9cXFpaKFxcWlpbR10sQSkiXSxbMywwLCJcXG9we0hvbX1fXFxaWihJX0csQSkiXSxbNCwwLCIwIl0sWzEsMSwiYSJdLFsyLDEsImFcXG1hcHN0byh6XFxtYXBzdG9cXHZhcmVwc2lsb24oeilhKSJdLFsyLDIsImYiXSxbMywyLCJmfF97SV9HfSJdLFswLDFdLFsxLDJdLFsyLDNdLFszLDRdLFs1LDYsIiIsMCx7InN0eWxlIjp7InRhaWwiOnsibmFtZSI6Im1hcHMgdG8ifX19XSxbNyw4LCIiLDAseyJzdHlsZSI6eyJ0YWlsIjp7Im5hbWUiOiJtYXBzIHRvIn19fV1d&macro_url=https%3A%2F%2Fraw.githubusercontent.com%2FdFoiler%2Fnotes%2Fmaster%2Fnir.tex
	\[\begin{tikzcd}[row sep=0.0em]
		0 & A & {\op{Hom}_\ZZ(\ZZ[G],A)} & {\op{Hom}_\ZZ(I_G,A)} & 0 \\
		& a & {(z\mapsto\varepsilon(z)a)} \\
		&& f & {f|_{I_G}}
		\arrow[from=1-1, to=1-2]
		\arrow[from=1-2, to=1-3]
		\arrow[from=1-3, to=1-4]
		\arrow[from=1-4, to=1-5]
		\arrow[maps to, from=2-2, to=2-3]
		\arrow[maps to, from=3-3, to=3-4]
	\end{tikzcd}\]
	We now track through the induced boundary morphism $\delta\colon H^1(H,\op{Hom}_\ZZ(I_G,A))\to H^2(H,Q)$.
	\begin{itemize}
		\item We begin with $c\in Z^1(H,\op{Hom}_\ZZ(I_G,A))$, which means that we have $c(h)\colon I_G\to A$ for each $h,h'\in H$, and we satisfy
		\[c(hh')=c(h)+h\cdot c(h').\]
		Tracking through the action of $H$ on $\op{Hom}_\ZZ(I_G,A)$, this means that
		\[c(hh')(g-1)=c(h)(g-1)+h\cdot c(h')(h^{-1}g-h^{-1})\]
		for any $g\in G$.
		\item To pull $c$ back to $C^1(H,\op{Hom}_\ZZ(\ZZ[G],A))$, we need to lift $c(h)\colon I_G\to A$ to a $\widetilde c(h)\colon\ZZ[G]\to A$. Recalling that we only need to preserve group structure, we simply precompose $c(h)$ with the map $\ZZ[G]\to I_G$ given by $z\mapsto z-\varepsilon(z)$. That is, we define
		\[\widetilde c(h)(z)\coloneqq c(h)(z-\varepsilon(z)).\]
		\item We now push $\widetilde c$ through $d\colon C^1(H,\op{Hom}_\ZZ(\ZZ[G],A))\to Z^2(H,\op{Hom}_\ZZ(\ZZ[G],A))$. This gives
		\[(d\widetilde c)(h,h')=g\widetilde c(h')-\widetilde c(hh')+\widetilde c(h)\]
		for any $h,h'\in H$. Concretely, plugging in some $z\in\ZZ[G]$ makes this look like
		\begin{align*}
			(d\widetilde{c})(h,h')(z) &= (h\widetilde c(h'))(z)-\widetilde c(hh')(z)+\widetilde c(h)(z) \\
			&= h\cdot c(h')\left(h^{-1}z-\varepsilon(h^{-1}z)\right)-c(hh')(z-\varepsilon(z))+c(h)(z-\varepsilon(z)) \\
			&= h\cdot c(h')\left(h^{-1}z-\varepsilon(z)\right)-c(hh')(z-\varepsilon(z))+c(h)(z-\varepsilon(z)).
		\end{align*}
		Now, from the $1$-cocycle condition on $c$, we recall
		\[-c(hh')(z-\varepsilon(z))+c(h)(z-\varepsilon(z))=-h\cdot(c(h')(h^{-1}z-\varepsilon(z)h^{-1})),\]
		so
		\begin{align*}
			(d\widetilde{c})(h,h')(z) &= h\cdot c(h')\left(\varepsilon(z)h^{-1}-\varepsilon(z)\right) \\
			&= \varepsilon(z)\cdot\left(h\cdot c(h')\left(h^{-1}-1\right)\right).
		\end{align*}
		In particular, we see that $d\widetilde c\in Z^2(H,\op{Hom}_\ZZ(\ZZ[G],A))$ pulls back to $(h,h')\mapsto h\cdot c(h')\left(h^{-1}-1\right)$ in $Z^2(H,A)$. It is not too difficult to check that we have in fact defined a $2$-cocycle, but we will not do so because it is not necessary for the proof.
	\end{itemize}
	Now, we do know that $\delta_H$ is a homomorphism abstractly on elements of our cohomology classes by the Snake lemma, but it is also not too hard to see that
	\[\delta_H\colon Z^1(H,\op{Hom}_\ZZ(I_G,A))\to Z^2(H,A)\]
	is in fact a homomorphism of groups directly from the construction. In short,
	\[\delta_H(c+c')(h,h')=h'\cdot c(h)\left(h^{-1}-1\right)+h'\cdot c'(h)\left(h^{-1}-1\right)=(\delta_H(c)+\delta_H(c'))(h,h')\]
	for any $h,h'\in H$.

	It remains to prove the last sentence. We run the following checks; given $u\in Z^2(H,A)$, define $c_u\in C^1(H,\op{Hom}_\ZZ(I_G,A))$ by
	\[c_u(h)(h'g_\bullet-1)=h'\cdot u\left((h')^{-1},h\right).\]
	Note that this is enough data to define $c_u(h)\colon I_G\to A$ because $I_G$ is a free $\ZZ$-module generated by $\{g-1:g\in G\}$.
	\begin{itemize}
		\item We verify that $c_u$ is a $1$-cocycle. This is a matter of force. Pick up $h,h'\in H$ and $g_\bullet h''\in G$ and write
		\begin{align*}
			&\phantom{{}={}}(hc_u(h'))(h''g_\bullet-1)+c_u(hh')(h''g_\bullet-1)+c_u(h)(h''g_\bullet-1) \\
			&= h\cdot c_u(h')\left(h^{-1}h''g_\bullet -h^{-1}\right)+c_u(hh')(h''g_\bullet-1)+c_u(h)(h''g_\bullet-1) \\
			&= h\cdot\left(h^{-1}h''u\left((h'')^{-1}h,h'\right)-h^{-1}u(h,h')\right)+h''u\left((h'')^{-1},hh'\right)+h''u\left((h'')^{-1},h\right) \\
			&= h''u\left((h'')^{-1}h,h'\right)-u(h,h')+h''u\left((h'')^{-1},hh'\right)+h''u\left((h'')^{-1},h\right).
		\end{align*}
		This is just the $2$-cocycle condition for $u$ upon dividing out by $h''$, so we are done.
		\item For $u\in Z^2(H,A)$, we verify that $\delta_H(c_u)=u$. Indeed, given $h,h'\in H$, we check
		\begin{align*}
			\delta_H(c_u)(h,h') &= h\cdot c_u(h')\left(h^{-1}-1\right) \\
			&= h\cdot h^{-1}\cdot u(h,h') \\
			&= u(h,h').
		\end{align*}
	\end{itemize}
	So far we have verified that $\delta$ has section $u\mapsto c_u$ and hence must be surjective. Lastly, we take $H=G$ and show that $c_{\delta c}=c$ to finish. Indeed, for $g,g'\in G=H$, we write
	\begin{align*}
		c_{\delta_H c}(g)(g'-1) &= g'\cdot(\delta_H c)\left((g')^{-1},g\right) \\
		&= g'(g')^{-1}\cdot c(g)(g'-1) \\
		&= c(g)(g'-1),
	\end{align*}
	which is what we wanted.
\end{proof}
We also have used dimension shifting to show that $H^1\left(G/H,\op{Hom}_\ZZ(I_G,A)^H\right)\to H^2\left(G/H,A^H\right)$ is an isomorphism, but this requires a little more trickery. To begin, we discuss how to lift from $\op{Hom}_\ZZ(I_G,A)^H$ to $\op{Hom}_\ZZ(\ZZ[G],A)^H$.
\begin{lemma} \label{lem:howtolift}
	Let $G$ be a group with subgroup $H\subseteq G$. Fix a $G$-module $A$ with $H^1(H,A)=0$. Then, for any $\psi\in\op{Hom}_\ZZ(I_G,A)^H$, the function $h\mapsto h\psi\left(h^{-1}-1\right)$ is a cocycle in $Z^1(H,A)=B^1(H,A)$, so we can define a function $\eta_\bullet\colon\op{Hom}_\ZZ(I_G,A)^H\to A$ such that
	\[\psi(h-1)=h\cdot \eta_\varphi-\eta_\varphi\]
	for all $h\in H$. In fact, given $\varphi\in\op{Hom}_\ZZ(I_G,A)^H$, we can construct $\widetilde\varphi\in\op{Hom}_\ZZ(\ZZ[G],A)^H$ by
	\[\widetilde\varphi(z)\coloneqq\varphi(z-\varepsilon(z))+\varepsilon(z)\eta_\varphi\]
	so that $\widetilde\varphi|_{I_G}=\varphi$.
\end{lemma}
\begin{proof}
	We will just run the checks directly.
	\begin{itemize}
		\item We start by checking $\psi\in\op{Hom}_\ZZ(I_G,A)^H$ give $1$-cocycles $c(h)\coloneqq \varphi\left(h-1\right)$ in $Z^1(A,H)$. To begin, we note that $\psi\in\op{Hom}_\ZZ(I_G,A)^H$ simply means that any $z-\varepsilon(z)\in I_G$ has
		\[\psi(z-\varepsilon(z))=(h\psi)(z-\varepsilon(z))=h\psi\left(h^{-1}z-h^{-1}\varepsilon(z)\right)\]
		for all $h\in H$. In particular, replacing $h$ with $h^{-1}$ tells us that
		\[h\psi(z-\varepsilon(z))=\psi(hz-h\varepsilon(z)).\]
		Now, we can just compute
		\begin{align*}
			(dc)(h,h') &= hc(h')-c(hh')+c(h) \\
			&= hc\left(h'-1\right)-c\left(hh'-1\right)+c\left(h-1\right) \\
			&= c\left(hh'-h\right)-c\left(hh'-1\right)+c\left(h-1\right),
		\end{align*}
		where in the last equality we used the fact that $\psi\in\op{Hom}_\ZZ(I_G,A)^H$. Now, $(dc)(h,h')$ manifestly vanishes, so we are done.
		\item Note that $\widetilde\varphi\in\op{Hom}_\ZZ(\ZZ[G],A)$ because it is a linear combination of (compositions of) homomorphisms.
		\item Note that any $z\in I_G$ has $\varepsilon(z)=0$, so
		\[\widetilde\varphi(z)=\varphi(z-0)+0\cdot \eta_\varphi=\varphi(z),\]
		so $\widetilde\varphi|_{I_G}=\varphi$.
		\item It remains to check that $\widetilde\varphi$ is fixed by $H$. This requires a little more effort. Recall that $\varphi\in\op{Hom}_\ZZ(I_G,A)^H$ means that any $z-\varepsilon(z)\in I_G$ has
		\[h\varphi(z-\varepsilon(z))=\varphi\left(hz-h\varepsilon(z)\right)\]
		for any $h\in H$. Now, we just compute
		\begin{align*}
			(h\widetilde\varphi)(z) &= h\widetilde\varphi\left(h^{-1}z\right) \\
			&= h\left(\varphi\left(h^{-1}z-\varepsilon(h^{-1}z)\right)+\varepsilon(h^{-1}z)\eta_\varphi\right) \\
			&= \varphi\left(z-h\varepsilon(z)\right)+\varepsilon(z)\cdot h\eta_\varphi \\
			&= \varphi\left(z-h\varepsilon(z)\right)+\varepsilon(z)\varphi(h-1)+\varepsilon(z)\eta_\varphi \\
			&= \varphi(z-\varepsilon(z))+\varepsilon(z)\eta_\varphi \\
			&= \widetilde\varphi(z).
		\end{align*}
	\end{itemize}
	The above checks complete the proof.
\end{proof}
\begin{remark}
	For motivation, the $\widetilde\varphi$ was constructed by tracking through the following diagram.
	% https://q.uiver.app/?q=WzAsOCxbMSwwLCJcXGRpc3BsYXlzdHlsZVxcZnJhY3tDXjAoSCxBKX17Ql4wKEgsQSl9Il0sWzEsMSwiWl4xKEgsQSk9Ql4xKEgsQSkiXSxbMiwxLCJaXjEoSCxcXG9we0hvbX1fXFxaWihcXFpaW0ddLEEpKSJdLFszLDEsIlpeMShILFxcb3B7SG9tfV9cXFpaKElfRyxBKSkiXSxbMywwLCJcXGRpc3BsYXlzdHlsZVxcZnJhY3tDXjAoSCxcXG9we0hvbX1fXFxaWihJX0csQSkpfXtCXjAoSCxcXG9we0hvbX1fXFxaWihJX0csQSkpfSJdLFsyLDAsIlxcZGlzcGxheXN0eWxlXFxmcmFje0NeMChILFxcb3B7SG9tfV9cXFpaKFxcWlpbR10sQSkpfXtCXjAoSCxcXG9we0hvbX1fXFxaWihcXFpaW0ddLEEpKX0iXSxbNCwwLCIwIl0sWzAsMSwiMCJdLFs3LDFdLFsxLDJdLFsyLDNdLFswLDVdLFs1LDRdLFs0LDZdLFswLDFdLFs1LDJdLFs0LDNdXQ==&macro_url=https%3A%2F%2Fraw.githubusercontent.com%2FdFoiler%2Fnotes%2Fmaster%2Fnir.tex
	\[\begin{tikzcd}
		& {\displaystyle\frac{C^0(H,A)}{B^0(H,A)}} & {\displaystyle\frac{C^0(H,\op{Hom}_\ZZ(\ZZ[G],A))}{B^0(H,\op{Hom}_\ZZ(\ZZ[G],A))}} & {\displaystyle\frac{C^0(H,\op{Hom}_\ZZ(I_G,A))}{B^0(H,\op{Hom}_\ZZ(I_G,A))}} & 0 \\
		0 & {Z^1(H,A)=B^1(H,A)} & {Z^1(H,\op{Hom}_\ZZ(\ZZ[G],A))} & {Z^1(H,\op{Hom}_\ZZ(I_G,A))}
		\arrow[from=2-1, to=2-2]
		\arrow[from=2-2, to=2-3]
		\arrow[from=2-3, to=2-4]
		\arrow[from=1-2, to=1-3]
		\arrow[from=1-3, to=1-4]
		\arrow[from=1-4, to=1-5]
		\arrow[from=1-2, to=2-2]
		\arrow[from=1-3, to=2-3]
		\arrow[from=1-4, to=2-4]
	\end{tikzcd}\]
	In short, take $\varphi\in Z^0(H,\op{Hom}_\ZZ(I_G,A))=\op{Hom}_\ZZ(I_G,A)^H$, pull it back to $z\mapsto\varphi(z-\varepsilon(z))$. Pushing this down to $Z^1(H,\op{Hom}_\ZZ(\ZZ[G],A))$ and pulling back to $Z^1(H,A)$ takes us to the $1$-cocycle $h\mapsto h\varphi\left(h^{-1}-1\right)$. Here we use the $H^1(H,A)=0$ condition above and adjust our lift $z\mapsto\varphi(z-\varepsilon(z))$ accordingly.
\end{remark}
And now we can now make our dimension shifting explicit.
\begin{lemma} \label{lem:dimshift2}
	Work in the context of \autoref{lem:howtolift} and assume that $H\subseteq G$ is normal. We track through the isomorphism
	\[\delta\colon H^1\left(G/H,\op{Hom}_\ZZ(I_G,A)^H\right)\simeq H^2\left(G/H,A^H\right)\]
	given by the exact sequence
	\[0\to A^H\to\op{Hom}_\ZZ(\ZZ[G],A)^H\to\op{Hom}_\ZZ(I_G,A)^H\to0.\]
\end{lemma}
\begin{proof}
	We begin with some $c\in H^1\left(G/H,\op{Hom}_\ZZ(I_G,A)^H\right)$. To track through the $\delta$, we define
	\[\widetilde c(gH)\coloneqq c(gH)(z-\varepsilon(z))+\eta_{c(gH)}\varepsilon(z)\]
	to be the lift given in \autoref{lem:howtolift}. Now, we are given that $dc=0$, which here means that any $z\in\ZZ[G]$ and $gH,g'H\in G/H$ will have
	\begin{align*}
		0 &= (dc)(gH,g'H)(z-\varepsilon(z)) \\
		0 &= (gH\cdot c(g'H)-c(gg'H)+c(gH))(z-\varepsilon(z)) \\
		0 &= g\cdot c(g'H)\left(g^{-1}z-g^{-1}\varepsilon(z)\right)-c(gg'H)(z-\varepsilon(z))+c(gH)(z-\varepsilon(z)) \\
		g\cdot c(g'H)\left(g^{-1}-1\right)\varepsilon(z) &= g\cdot c(g'H)\left(g^{-1}z-\varepsilon(z)\right)-c(gg'H)(z-\varepsilon(z))+c(gH)(z-\varepsilon(z)) \\
		g\cdot c(g'H)\left(g^{-1}-1\right)\varepsilon(z) &= g\cdot c(g'H)\left(g^{-1}z-\varepsilon(g^{-1}z)\right)-c(gg'H)(z-\varepsilon(z))+c(gH)(z-\varepsilon(z)).
	\end{align*}
	We now directly compute that
	\begin{align*}
		(d\widetilde c)(gH,g'H)(z) &= (gH\cdot c(g'H)-c(gg'H)+c(gH))(z) \\
		&= g\cdot c(g'H)\left(g^{-1}z-\varepsilon(g^{-1}z)\right)+g\eta_{c(g'H)}\varepsilon(z) \\
		&\phantom{{}={}}-c(gg'H)(z-\varepsilon(z))-\eta_{c(gg'H)}\varepsilon(z) \\
		&\phantom{{}={}}+c(gH)(z-\varepsilon(z))+\eta_{c(gH)}\varepsilon(z) \\
		&= \left(g\cdot c(g'H)\left(g^{-1}-1\right)+g\cdot \eta_{c(g'H)}-\eta_{c(gg'H)}+\eta_{c(gH)}\right)\varepsilon(z)
	\end{align*}
	% We quickly recall that $c(g'H)\in\op{Hom}_\ZZ(I_G,A)^H$ implies that $h\cdot c(g'H)(z)=(h\cdot c(g'H))\left(h^{-1}z\right)=c(g'H)\left(h^{-1}-1\right)$, so in fact we can write
	% \[(d\widetilde c)(gH,g'H)(z) = \left(c(g'H)\left(1-g\right)-g\cdot \eta_{c(g'H)}+\eta_{c(gg'H)}-\eta_{c(gH)}\right)\varepsilon(z).\]
	As such, we have pulled ourselves back to the $2$-cocycle given by
	\[\boxed{u(gH,g'H)\coloneqq g\cdot c(g'H)\left(g^{-1}-1\right)+g\cdot \eta_{c(g'H)}-\eta_{c(gg'H)}+\eta_{c(gH)}}.\]
	We quickly note that this is in fact independent of our choice of representative $g\in gH$: changing representative of $g$ to $gh$ for $h\in H$ will only affect the terms
	\[h\cdot c(g'H)\left(h^{-1}g^{-1}-1\right)+h\eta_{c(g'H)}=c(g'H)\left(g^{-1}-h\right)+c(g'H)\left(h-1\right)+\eta_{c(g'H)}=c(g'H)\left(g^{-1}-1\right)+\eta_{c(g'H)},\]
	so we are indeed safe. This completes the proof.
	% Even though it is not necessary, we will run the following checks on $u$.
	% \begin{itemize}
	% 	\item We verify that $\im u\subseteq A^H$. The main point is that any $h\in H$ will have $h\cdot \eta_\varphi=h\varphi\left(h^{-1}-1\right)+\eta_\varphi=\varphi(1-h)+\eta_\varphi$ for any $\varphi\in\op{Hom}_\ZZ(I_G,A)^H$. Thus,
	% 	\begin{align*}
	% 		h\cdot u(gH,g'H) &= hg\cdot c(g'H)\left(g^{-1}-1\right)-hg\eta_{c(g'H)}+h\eta_{c(gg'H)}-h\eta_{c(gH)} \\
	% 		&= gg^{-1}hg\cdot c(g'H)\left(g^{-1}-1\right)-gg^{-1}hg\eta_{c(g'H)}+h\eta_{c(gg'H)}-h\eta_{c(gH)} \\
	% 		&= g\cdot c(g'H)\left(g^{-1}h-g^{-1}hg\right) \\
	% 		&\phantom{{}={}}-g\left(c(g'H)(1-g^{-1}hg)+\eta_{c(g'H)}\right) \\
	% 		&\phantom{{}={}}+c(gg'H)(1-h)+\eta_{c(gg'H)} \\
	% 		&\phantom{{}={}}-c(gH)(1-h)-\eta_{c(gH)} \\
	% 		&= g\cdot c(g'H)(g^{-1}h-1)+c(gg'H)(1-h)-c(gH)(1-h)-g\eta_{c(g'H)}+\eta_{c(gg'H)}-\eta_{c(gH)} \\
	% 		&= g\cdot c(g'H)(g^{-1}h-1)+g\cdot c(g'H)(1-h)-g\eta_{c(g'H)}+\eta_{c(gg'H)}-\eta_{c(gH)} \\
	% 		&= g\cdot c(g'H)(g^{-1}h-h)-g\eta_{c(g'H)}+\eta_{c(gg'H)}-\eta_{c(gH)} \\
	% 	\end{align*}
	% 	Because 
	% \end{itemize}
\end{proof}
We now make \autoref{thm:infres} explicit in the case of $q=2$.
\begin{lemma} \label{lem:explicitresinf}
	Let $G$ be a group with normal subgroup $H\subseteq G$. Fix a $G$-module $A$ with $H^1(H,A)=0$, and define the function $\eta_\bullet\colon\op{Hom}_\ZZ(I_G,A)^H\to A$ of \autoref{lem:howtolift}. Given $c\in Z^2(G,A)$ such that $\op{Res}^G_Hc\in B^2(H,A)$; in particular, suppose we have $b\in\op{Hom}_\ZZ(I_G,A)$ such that all $h\in H$ have
	\[\op{Res}^G_H(\delta^{-1}c)(h)=(db)(h)=h\cdot b-h,\]
	where $\delta^{-1}$ is the inverse isomorphism of \autoref{lem:explicitdimshift}. Then we find $u\in Z^2\left(G/H,A^H\right)$ such that
	\[[\op{Inf}u]=[c]\]
	in $H^2(G,A)$.
\end{lemma}
\begin{proof}
	The main point is that boundary morphisms $\delta$ commute with $\op{Res}$ and $\op{Inf}$. By construction, we have that $\left(\op{Res}^G_H\delta^{-1}c\right)-db=0$ in $Z^1(H,\op{Hom}_\ZZ(I_G,A))$. Pulling back to $Z^1(G,\op{Hom}_\ZZ(I_G,A))$, we note that
	\[c'\coloneqq\left(\delta^{-1}c-db\right)\in Z^1(G,\op{Hom}_\ZZ(I_G,A))\]
	vanishes on $H$ by hypothesis. Because $\delta^{-1}c-db$ is a $1$-cocycle, we are able to write
	\[c'(gg')=c'(g)+gc'(g').\]
	Letting $g'$ vary over $H$, we see that $\delta^{-1}c-db$ is well-defined on $G/H$. On the other hand, for any $h\in H$ and $g\in G$, we note that $g^{-1}hg\in H$, so
	\[c'(g)=c'\left(g\cdot g^{-1}hg\right)=c'\left(hg\right)=c'\left(h\right)+hc(g),\]
	implying that $c'(g)\in\op{Hom}_\ZZ(I_G,A)^H$.

	We are now ready to apply \autoref{lem:dimshift2}, which we use on $c'$, thus defining $u\coloneqq\delta(c')$. Explicitly, we have
	\[
		\boxed{u(gH,g'H) = g\cdot c'(g'H)\left(g^{-1}-1\right)+g\cdot \eta_{c'(g'H)}-\eta_{c'(gg'H)}+\eta_{c'(gH)}}.
	\]
	This is explicit enough for our purposes. Observe that $[\op{Inf}u]=[c]$ because $[\op{Inf}c']=[\delta^{-1}c]$, and $\delta$ commutes with $\op{Inf}$.
	% Thus, we define $\overline c'\in C^1\left(G/H,\op{Hom}_\ZZ(I_G,A)^H\right)$ by $\overline c'(gH)\coloneqq c'(g)$. Note $\overline c'\in Z^1\left(G/H,\op{Hom}_\ZZ(I_G,A)^H\right)$ because each $g,g'\in G$ give
	% \[\overline c'(gH\cdot g'H)=c'(gg')=c'(g)+gc'(g')=\overline c'(gH)+gH\cdot \overline c'(gH).\]
	% We take a moment to understand $\op{Hom}_\ZZ(I_G,A)^H$. Given $f\in\op{Hom}_\ZZ(I_G,A)$, the condition that $f$ is fixed by $H$ is saying that all $h\in H$ will have $hf=f$. Concretely, we require each $g\in G$ to have
	% \[f(g-1)=(hf)(g-1)=h\cdot f\left(h^{-1}g-h^{-1}\right).\]
\end{proof}

\subsection{Number Theory}
Throughout, we will let $u_{L/K}$ denote a representative of the fundamental class in $H^2(L/K)$ rather than the actual cohomology class, mostly out of laziness.

We now return to the set-up in \autoref{sec:setup} and track through \autoref{lem:explicitresinf} in our case. For reference, the following is the diagram that we will be chasing around; here $G\coloneqq\op{Gal}(ML/K)$ and $H\coloneqq\op{Gal}(ML/L)$.
% https://q.uiver.app/?q=WzAsOSxbMiwwLCJIXjIoXFxvcHtHYWx9KE0vSyksTV5cXHRpbWVzKSJdLFsyLDEsIkheMihcXG9we0dhbH0oTUwvSyksTUxeXFx0aW1lcykiXSxbMywxLCJIXjIoXFxvcHtHYWx9KE1ML0wpLE1MXlxcdGltZXMpIl0sWzMsMiwiSF4xKFxcb3B7R2FsfShNTC9MKSxcXG9we0hvbX1fXFxaWihJX3tcXG9we0dhbH0oTUwvSyl9LE1MXlxcdGltZXMpKSJdLFsyLDIsIkheMShcXG9we0dhbH0oTUwvSyksXFxvcHtIb219X1xcWlooSV97XFxvcHtHYWx9KE1ML0spfSxNTF5cXHRpbWVzKSkiXSxbMSwyLCJIXjEoXFxvcHtHYWx9KEwvSyksXFxvcHtIb219X1xcWlooSV97XFxvcHtHYWx9KE1ML0spfSxNTF5cXHRpbWVzKV57XFxvcHtHYWx9KEwvSyl9KSJdLFsxLDEsIkheMihcXG9we0dhbH0oTC9LKSxMXlxcdGltZXMpIl0sWzAsMSwiMCJdLFswLDIsIjAiXSxbNSw2LCJcXGRlbHRhIiwyXSxbNCwxLCJcXGRlbHRhIiwyXSxbMywyLCJcXGRlbHRhIiwyXSxbMSwyLCJcXG9we1Jlc30iXSxbNiwxLCJcXG9we0luZn0iXSxbMCwxLCJcXG9we0luZn0iXSxbNSw0XSxbNCwzXSxbNyw2XSxbOCw1XV0=&macro_url=https%3A%2F%2Fraw.githubusercontent.com%2FdFoiler%2Fnotes%2Fmaster%2Fnir.tex
\[\begin{tikzcd}
	&& {H^2(\op{Gal}(M/K),M^\times)} \\
	0 & {H^2(\op{Gal}(L/K),L^\times)} & {H^2(G,ML^\times)} & {H^2(\op{Gal}(ML/L),ML^\times)} \\
	0 & {H^1(G/H,\op{Hom}_\ZZ(I_{G},ML^\times)^{H})} & {H^1(G,\op{Hom}_\ZZ(I_{G},ML^\times))} & {H^1(H,\op{Hom}_\ZZ(I_{G},ML^\times))}
	\arrow["\delta"', from=3-2, to=2-2]
	\arrow["\delta"', from=3-3, to=2-3]
	\arrow["\delta"', from=3-4, to=2-4]
	\arrow["{\op{Res}}", from=2-3, to=2-4]
	\arrow["{\op{Inf}}", from=2-2, to=2-3]
	\arrow["{\op{Inf}}", from=1-3, to=2-3]
	\arrow["\op{Inf}", from=3-2, to=3-3]
	\arrow["\op{Res}", from=3-3, to=3-4]
	\arrow[from=2-1, to=2-2]
	\arrow[from=3-1, to=3-2]
\end{tikzcd}\]
To begin, we know that we can write
\[u_{M/K}\left(\sigma_K^i,\sigma_K^j\right)=\pi^{\floor{\frac{i+j}n}}=\begin{cases}
	1 & i+j<n, \\
	\pi & i+j\ge n,
\end{cases}\]
where $\pi$ is a uniformizer of $K$. Inflating this down to $H^2(G,ML^\times)$ gives
\[(\op{Inf}u_{M/K})\left(\sigma_K^{a_1}\tau,\sigma_K^{b_1}\tau'\right)=\pi^{\floor{\frac{a_1+b_1}n}}.\]
Now, we use \autoref{lem:dimshift} to move down to $H^1(G,\op{Hom}_\ZZ(I_G,ML^\times))$ as
\[\delta^{-1}(\op{Inf}u_{M/K})\left(\sigma_K^{a_1}\tau\right)\left(\sigma_K^{b_1}\tau'-1\right)=\sigma_K^{b_1}\tau'\cdot (\op{Inf}u_{M/K})\left(\sigma_K^{[-b_1]}(\tau')^{-1},\sigma_K^{a_1}\tau\right)=p^{\floor{\frac{a_1+[-b_1]}n}},\]
where $[k]$ denote the integer $0\le[k]<n$ such that $k\equiv[k]\pmod n$.

Now, we need to show that the restriction to $H=\langle\sigma_k^f\rangle$ is a coboundary. That is, we need to find $b\in\op{Hom}_\ZZ(I_G,ML^\times)$ such that
\[\delta^{-1}(\op{Inf}u_{M/K})\left(\sigma_K^{fa_1}\right)=\frac{\sigma_K^{fa_1}\cdot b}b.\]
Because $I_G$ is freely generated by elements of the form $g-1$ for $g\in G$, it suffices to plug in some arbitrary $\sigma_K^{b_1}\tau'-1$, which we see requires
\begin{align*}
	\pi^{\floor{\frac{fa_1+[-b_1]}n}} &= \frac{\big(\sigma_K^{fa_1}\cdot b\big)\left(\sigma_K^{b_1}\tau'-1\right)}{b\left(\sigma_K^{b_1}\tau'-1\right)} \\
	&= \frac{\sigma_K^{fa_1} b\left(\sigma_K^{b_1-fa_1}\tau'-1\right)}{\sigma_K^{fa_1} b\left(\sigma_K^{-fa_1}-1\right)b\left(\sigma_K^{b_1}\tau'-1\right)}.
\end{align*}
We can see that $b$ should not depend on $\tau'$, so we define $\hat b\left(\sigma_K^a\right)=b\left(\sigma_K^a\tau'-1\right)$; the above is then equivalent to
\begin{align*}
	\pi^{\floor{\frac{fa_1+[-b_1]}n}} &= \frac{\sigma_K^{fa_1}\hat b\left(\sigma_K^{b_1-fa_1}\right)}{\sigma_K^{fa_1}\hat b\left(\sigma_K^{-fa_1}\right)\hat b\left(\sigma_K^{b_1}\right)} \\
	\pi^{\floor{\frac{fa_1+b_1}n}} &= \frac{\hat b\left(\sigma_K^{-b_1-fa_1}\right)}{\hat b\left(\sigma_K^{-fa_1}\right)\sigma_K^{-fa_1}\hat b\left(\sigma_K^{-b_1}\right)},
\end{align*}
where we have negated $b_1$ in the last step. At this point, the right-hand side will look a lot more natural if we set $\tau\coloneqq\sigma_K^{-1}$, which turns this into
\[\frac{\hat b\left(\tau_K^{fa_1}\right)\tau_K^{fa_1}\hat b\left(\tau_K^{b_1}\right)}{\hat b\left(\tau_K^{b_1fa_1}\right)} = (1/\pi)^{\floor{\frac{fa_1+b_1}n}}\]
after taking reciprocals. Thus, we see that $\hat b$ should be counting carries of $\tau$s. With this in mind, we let $\varpi$ be a uniformizer of $K_{\pi,\nu}$ and note that $\varpi\in L$ be a uniformizer because $L/K_{\pi,\nu}$ is an unramified extension. It follows that
\[\varpi^{[ML:L]}\in\op N_{ML/L}\left(ML^\times\right).\]
Further, $\varpi^{[ML:L]}$ has the same absolute value as $\pi$ because $K_{\pi,\nu}/K$ is a totally ramified extension of degree $[K_{\pi,\nu}:K]=[ML:M]=[ML:L]$. Thus, $\pi$ is a norm in $\op N_{ML/L}(ML^\times)$ because $ML/L$ is unramified and so $\mathcal O_L^\times\subseteq\op N_{ML/L}(ML^\times)$. Thus, we find $\gamma\in ML^\times$ such that
\[\op N_{ML/L}(\gamma)=\pi.\]
The point of doing all of this is so that we can codify our carrying by writing
\[\hat b\left(\tau_K^a\right)\coloneqq\prod_{i=0}^{\floor{a/f}-1}\tau^{if}(\gamma)^{-1}.\]
Tracking out $\hat b$ backwards to $b$, our desired $b\in\op{Hom}_\ZZ(I_G,ML^\times)$ is given by
\[\boxed{b\left(\sigma_K^{a_1}\tau-1\right)=\prod_{i=0}^{\floor{[-a_1]/f}-1}\sigma_K^{-if}(\gamma)^{-1}}.\]
We take a moment to write out $c\coloneqq\delta^{-1}(\op{Inf}u_{M/K})/db$, which looks like
\begin{align*}
	c\left(\sigma_K^{a_1}\tau\right)\left(\sigma_K^{b_1}\tau'-1\right) &= \frac{\delta^{-1}(\op{Inf}u_{M/K})}{db}\left(\sigma_K^{a_1}\tau\right)\left(\sigma_K^{b_1}\tau'-1\right) \\
	&= \frac{\delta^{-1}(\op{Inf}u_{M/K})\left(\sigma_K^{a_1}\tau\right)\left(\sigma_K^{b_1}\tau'-1\right)}{\left(\sigma_K^{a_1}\tau b\right)\left(\sigma_K^{b_1}\tau'-1\right)/b\left(\sigma_K^{b_1}\tau'-1\right)} \\
	&= \frac{\pi^{\floor{(a_1+[-b_1])/n}}}{\sigma_K^{a_1}\tau b\left(\sigma_K^{b_1-a_1}\tau'\tau^{-1}-\sigma_K^{-a_1}\tau^{-1}\right)/b\left(\sigma_K^{b_1}\tau'-1\right)} \\
	&= \pi^{\floor{(a_1+[-b_1])/n}}\cdot\hat b\left(\sigma_K^{b_1}\right)\cdot\sigma_K^{a_1}\tau\left(\frac{\hat b\left(\sigma_K^{-a_1}\right)}{\hat b\left(\sigma_K^{b_1-a_1}\right)}\right).
\end{align*}
Before proceeding, we discuss a few special cases.
\begin{itemize}
	\item Taking $\sigma_K^{a_1}\tau=\tau_i$ for some $\tau_i$, we get
	\begin{align*}
		c\left(\tau_i\right)\left(\sigma_K^{b_1}\tau'-1\right) &= \pi^{\floor{(0+[-b_1])/n}}\cdot\hat b\left(\sigma_K^{b_1}\right)\cdot\tau_i\left(\frac{1}{\hat b\left(\sigma_K^{b_1}\right)}\right) \\
		&= \hat b\left(\sigma_K^{b_1}\right)/\tau_i\hat b\left(\sigma_K^{b_1}\right).
	\end{align*}
	In particular, $c\left(\sigma_x\right)\left(\sigma_K^{-1}-1\right)=1$, provided that $f>1$. Additionally, $c(\tau_i)\left(\tau'-1\right)=1$.
	
	Our general theory says that $h\mapsto c(\sigma_x)(h-1)$ is a $1$-cocycle in $Z^1(H,ML^\times)$ (though we could also check this directly), so Hilbert's Theorem 90 promises us a magical element $\eta_i\in ML^\times$ such that
	\[\frac{\sigma_K^{fb_1}\eta_i}{\eta_i}=\frac{\hat b\left(\sigma_K^{fb_1}\right)}{\tau_i\hat b\left(\sigma_K^{fb_1}\right)}\]
	for all $\sigma_K^{fb_1}\in H$. This condition will be a little clearer if we write everything in terms of $\tau_K\coloneqq\sigma_K^{-1}$, which transforms this into
	\[\frac{\tau_K^{fb_1}\eta_i}{\eta_i}=\frac{\hat b\left(\tau_K^{-fb_1}\right)}{\tau_i\hat b\left(\tau_K^{-fb_1}\right)}=\prod_{i=0}^{b_1-1}\frac{\tau_K^{if}(\gamma^{-1})}{\tau_i\tau_K^{if}(\gamma^{-1})}=\prod_{i=0}^{b_1-1}\frac{\tau_i\tau_K^{if}(\gamma)}{\tau_K^{if}(\gamma)}.\]
	Because we are dealing with a cyclic group $H$, it is not too hard to see that it suffices merely for $b_1=1$ to hold, so our magical element $\eta_{c(\sigma_x)}$ merely requires
	\[\boxed{\frac{\sigma_K^{-f}\left(\eta_i\right)}{\eta_i}=\frac{\tau_i(\gamma)}{\gamma}}\]
	after inverting $\tau_K$ back to $\sigma_K$.
	\item Taking $\sigma_K^{a_1}\tau=\sigma_K$, we get
	\begin{align*}
		c\left(\sigma_K\right)\left(\sigma_K^{b_1}\tau'-1\right) &= \pi^{\floor{(1+[-b_1])/n}}\cdot\hat b\left(\sigma_K^{b_1}\right)\cdot\sigma_K\left(\frac{\hat b\left(\sigma_K^{-1}\right)}{\hat b\left(\sigma_K^{b_1-1}\right)}\right).
	\end{align*}
	In particular, $\sigma_K^{b_1}\tau'=\tau_i^{-1}$ will give $c(\sigma_K)\left(\tau_i^{-1}-1\right)=1$. We will also want $c(\sigma_K)\left(\sigma_K^{-b_1}-1\right)$ for $0\le b_1<f$. Using the fact that $f<n$ and $f>1$, it is not too hard to see that everything will cancel down to $1$ except in the case where $b_1=f-1$, where we get
	\[c(\sigma_K)\left(\sigma_K^{-(f-1)}-1\right)=\sigma_K\left(\frac1{\hat b\left(\sigma_K^{-f}\right)}\right)=\sigma_K(\gamma).\]
	Continuing as before, our general theory says that $h\mapsto c(\sigma_K)(h-1)$ is a $1$-cocycle in $Z^1(H,ML^\times)$, though again we could just check this directly. It follows that Hilbert's Theorem 90 promises us a magical element $\eta_K\in ML^\times$ such that
	\[\frac{\sigma_K^{fb_1}\eta_K}{\eta_K}=p^{\floor{(1+[-fb_1])/n}}\cdot\hat b\left(\sigma_K^{fb_1}\right)\cdot\sigma_K\left(\frac{\hat b\left(\sigma_K^{-1}\right)}{\hat b\left(\sigma_K^{fb_1-1}\right)}\right)\]
	for all $\sigma_K^{fb_1}\in H$. Using $f>1$, this collapses down to
	\[\frac{\sigma_K^{fb_1}\eta_K}{\eta_K}=\frac{\hat b\left(\sigma_K^{fb_1}\right)}{\sigma_K\hat b\left(\sigma_K^{fb_1-1}\right)}.\]
	As before, this condition will be a little clearer if we set $\tau_K\coloneqq\sigma_K^{-1}$, which turns the condition into
	\[\frac{\tau_K^{fb_1}\eta_K}{\eta_K}=\frac{\hat b\left(\tau_K^{fb_1}\right)}{\sigma_K\hat b\left(\tau_K^{fb_1+1}\right)}=\prod_{i=0}^{b_1-1}\frac{\tau_K^{if}(\gamma^{-1})}{\sigma_K\tau_K^{if}(\gamma^{-1})}=\prod_{i=0}^{b_1-1}\frac{\sigma_K\tau_K^{if}(\gamma)}{\tau^{if}(\gamma)}.\]
	(Notably, $\hat b\left(\tau^{fb_1}\right)=\hat b\left(\tau^{fb_1+1}\right)$ because $f>1$.) Again, because $H$ is cyclic generated by $\tau^f$, an induction shows that it suffices to check this condition for $b_1=1$, which means that our magical element $\eta_K\in ML^\times$ is constructed so that
	\[\boxed{\frac{\sigma_K^{-f}\left(\eta_K\right)}{\eta_K}=\frac{\sigma_K(\gamma)}{\gamma}}\]
	where we have again inverted back from $\tau_K$ to $\sigma_K$.
	\item We will not actually need a more concrete description of this, but we remark that we can run the same story for any $g\in G$ through to get an element $\eta_g\in ML^\times$ such that
	\[\frac{\sigma_K^{fb_1}\eta_g}{\eta_g}=\frac1{c(g)(\sigma_K^{fb_1}-1)}\]
	for any $\sigma_K^{fb_1}\in H$. As usual, this follows from our general theory.
\end{itemize}
We are now ready to describe the local fundamental class. Piecing what we have so far, we know from \autoref{lem:explicitresinf} that we can write
\[u_{L/K}(g,g')\coloneqq gc(g')\left(g^{-1}-1\right)\cdot\frac{g\eta_{g'}\cdot \eta_{g}}{\eta_{gg'}}.\]
Here are the values that we care about for our specific computation; for consistency, we set $\tau_0\coloneqq\sigma_K$ and $n_0\coloneqq f$ to be the order of $\tau_0$.
\begin{itemize}
	\item We write
	\begin{align*}
		u_{L/K}(\sigma_K,\tau_i) &= \sigma_Kc(\tau_i)\left(\sigma_K^{-1}-1\right)\cdot\frac{\sigma_K \eta_i\cdot \eta_K}{\eta_{\sigma_K\tau_i}} \\
		&= \frac{\sigma_K \eta_i\cdot \eta_K}{\eta_{\sigma_K\sigma_x}}.
	\end{align*}
	\item We write
	\begin{align*}
		u_{L/K}(\tau_i,\sigma_K) &= \tau_ic(\sigma_K)\left(\tau_i^{-1}-1\right)\cdot\frac{\tau_i\eta_K\cdot \eta_i}{\eta_{\sigma_x\sigma_K}} \\
		&= \frac{\tau_i\eta_K\cdot \eta_i}{\eta_{\sigma_x\sigma_K}}.
	\end{align*}
	\item In particular, we know that we can set $\beta_{i0}$ to
	\begin{align*}
		\beta_{i0} \coloneqq{}& \frac{u_{L/K}(\tau_i,\sigma_K)}{u_{L/K}(\sigma_K,\tau_i)} \\
		={}& \frac{\tau_i\eta_K\cdot \eta_i/\eta_{\sigma_x\sigma_K}}{\sigma_K \eta_i\cdot \eta_K/\eta_{\sigma_K\sigma_x}} \\
		\Aboxed{\beta_{i0}={}& \frac{\eta_i}{\sigma_K\left(\eta_i\right)}\cdot\frac{\tau_i\left(\eta_K\right)}{\eta_K}}.
	\end{align*}
	% As a sanity check, we can hit this $\beta$ with $\sigma_K^{-f}$ to show that $\beta\in(ML)^H=L$; namely, $\sigma_K^{-f}\eta_{c(\sigma_K)}=\frac{\sigma_K\alpha}\alpha\cdot \eta_{c(\sigma_K)}$ and $\sigma_K^{-f}\eta_{c\sigma(x)}=\frac{\sigma_x\alpha}\alpha\cdot \eta_{c(\sigma_x)}$ by construction, so we can see that everything will appropriately cancel out.
	\item We write
	\begin{align*}
		u_{L/K}(\tau_i,\tau_j) &= \tau_ic(\tau_j)\left(\tau_j^{-1}-1\right)\cdot\frac{\tau_i\eta_j\cdot \eta_{i}}{\eta_{\tau_i\tau_j}} \\
		&= \frac{\tau_i\eta_j\cdot \eta_{i}}{\eta_{\tau_i\tau_j}}.
	\end{align*}
	\item Thus, for $i>j>0$, we can set $\beta_{ij}$ to
	\begin{align*}
		\beta_{ij} \coloneqq{}& \frac{u_{L/K}(\tau_i,\tau_j)}{u_{L/K}(\tau_j,\tau_i)} \\
		={}& \frac{\tau_i\eta_j\cdot \eta_{i}/\eta_{\tau_i\tau_j}}{\tau_j\eta_i\cdot \eta_{j}/\eta_{\tau_i\tau_j}} \\
		\Aboxed{\beta_{ij}={}& \frac{\eta_{i}}{\tau_j\eta_i}\cdot\frac{\tau_i\eta_j}{\eta_{j}}}.
	\end{align*}
	\item We will go ahead and compute $\alpha_0$ and the $\alpha_i$, for completeness. For $\alpha_0$, our element is given by
	\begin{align*}
		\alpha_0 \coloneqq{}& \prod_{i=0}^{f-1}u_{L/K}\left(\sigma_K^i,\sigma_K\right) \\
		={}& \prod_{i=0}^{f-1}\left(\sigma_K^ic\left(\sigma_K,\sigma_K^{-i}-1\right)\cdot\frac{\sigma_K^i\eta_K\cdot \eta_{\sigma_K^i}}{\eta_{\sigma_K^{i+1}}}\right).
	\end{align*}
	Recall from our general theory that $\eta_g$ only depends on the coset of $g$ in $G/H$, so we see that the product of the quotients $\eta_{\sigma_K^i}/\eta_{\sigma_K^{i+1}}$ will cancel out. As for the $c$ term, we know from our computation that this is $1$ until $i=f-1$, which gives $\sigma_K(\gamma)$. As such, we collapse down to
	\[\boxed{\alpha_0=\sigma_K^f(\gamma)\cdot\prod_{i=0}^{f-1}\sigma_K^i\left(\eta_K\right)}.\]
	\item For $\alpha_i$ with $i>0$, our element is given by
	\begin{align*}
		\alpha_i \coloneqq{}& \prod_{p=0}^{n_i-1}u_{L/K}\left(\tau_i^p,\tau_i\right) \\
		={}& \prod_{p=0}^{n_i-1}\tau_i^pc(\tau_i)\left(\tau_i^{-p}-1\right)\cdot\frac{\tau_i^p\eta_i\cdot \eta_{\tau_i^p}}{\eta_{\tau_i^{p+1}}}.
	\end{align*}
	Recalling that $\tau_i$ has order $n_i$, our quotient term $\eta_{\tau_i^i}/\eta_{\tau_i^{i+1}}$ will again cancel out. Additionally, the cocycle $c$ always spits out $1$ on these inputs, so we are left with
	\[\boxed{\alpha_i=\prod_{p=0}^{n_i-1}\tau_i^p\left(\eta_i\right)}.\]
\end{itemize}
We summarize the results above in the following theorem.
\begin{theorem} \label{thm:fundtriple}
	Fix everything as in the set-up. Then there exists some $\gamma\in ML^\times$ such that $\op N_{ML/L}(\gamma)=\pi$ and elements in $\eta_K,\eta_i\in ML^\times$ (for $1\le i\le t$) such that
	\[\frac{\sigma_K^{-f}\left(\eta_K\right)}{\eta_K}=\frac{\sigma_K(\gamma)}{\gamma}\qquad\text{and}\qquad\frac{\sigma_K^{-f}\left(\eta_i\right)}{\eta_i}=\frac{\tau_i(\gamma)}{\gamma}.\]
	Then the tuple
	\[\big((\alpha_0,\alpha_i),(\beta_{ij})\big)\coloneqq
	\left(\sigma_K^f(\gamma)\cdot\prod_{i=0}^{f-1}\sigma_K^i\left(\eta_K\right),\quad
	\prod_{p=0}^{n_i-1}\tau_i^p\left(\eta_i\right),\quad
	\frac{\eta_{i}}{\tau_j\eta_i}\cdot\frac{\tau_i\eta_j}{\eta_{j}}\right)\]
	corresponds to the fundamental class $u_{L/K}\in H^2(\op{Gal}(L/K),L^\times)$.
\end{theorem}
We remark that we can replace $\gamma$ with $\sigma_K^f(\gamma)$ (which still has norm $p$) while keeping all other variables the same; this gives us the following slightly prettier presentation. Note that we have multiplied the equations for $\eta_\bullet$ by $\sigma_K^f$ on both sides.
\begin{corollary} \label{cor:fundtriple}
	Fix everything as in the set-up. Then there exists some $\gamma\in ML^\times$ such that $\op N_{ML/L}(\gamma)=\pi$ and elements in $\eta_K,\eta_i\in ML^\times$ (for $1\le i\le t$) such that
	\[\frac{\eta_K}{\sigma_K^{f}\left(\eta_K\right)}=\frac{\sigma_K(\gamma)}{\gamma}\qquad\text{and}\qquad\frac{\eta_i}{\sigma_K^f\left(\eta_i\right)}=\frac{\tau_i(\gamma)}{\gamma}.\]
	Then the tuple
	\[\big((\alpha_0,\alpha_i),(\beta_{ij})\big)\coloneqq
	\left(\gamma\cdot\prod_{i=0}^{f-1}\sigma_K^i\left(\eta_K\right),\quad
	\prod_{p=0}^{n_i-1}\tau_i^p\left(\eta_i\right),\quad
	\frac{\eta_{i}}{\tau_j\eta_i}\cdot\frac{\tau_i\eta_j}{\eta_{j}}\right)\]
	corresponds to the fundamental class $u_{L/K}\in H^2(\op{Gal}(L/K),L^\times)$.
\end{corollary}
\begin{remark}
	This result is essentially a stronger version of Dwork's theorem (1958), recorded in Serre's \textit{Local Fields}, chapter~XIII, Theorem~2. Namely, Dwork and Serre are interested in computing the reciprocity map, which roughly means we only want access to the $\alpha$s, but above we are interested in computing the full fundamental class.
\end{remark}

\subsection{Checks}
In this section we run some checks and discuss some consequences of \autoref{thm:fundtriple}, in the form of \autoref{cor:fundtriple}. For these results, we recall that we set $L\coloneqq\QQ_p(\zeta_N)$ and $L_1\coloneqq\QQ_p(\zeta_{p^\nu})$ and $L_2\coloneqq\QQ_p(\zeta_m)$ so that $\overline\sigma_K=\sigma_K|_{L_1}$ generates $\op{Gal}(L/L_1)$ and $\sigma_x$ generates $\op{Gal}(L/L_2)$.

In the discussion which follows, we will make repeated use of the fact that (using notation of \autoref{cor:fundtriple})
\[\sigma_K^f\left(\eta_K\right)=\frac{\gamma}{\sigma_K(\gamma)}\cdot \eta_K\qquad\text{and}\qquad\sigma_K^f\left(\eta_i\right)=\frac\gamma{\tau_i(\gamma)}\cdot \eta_i.\]
And here are our checks; we start by showing that our elements are in the right field.
\begin{lemma}
	Fix a tuple $(\alpha_0,\alpha_i),(\beta_{ij})$ as in \autoref{cor:fundtriple}. Then the following are true.
	\begin{listalph}
		\item $\alpha_0\in K_{\pi,\nu}^\times$.
		\item $\alpha_i\in L_i^\times$ for each $i\ge1$.
		\item $\beta_{ij}\in L^\times$ for each $i>j$.
	\end{listalph}
\end{lemma}
\begin{proof}
	We run the checks one at a time.
	\begin{listalph}
		\item It suffices to show that $\alpha_0$ is fixed by $\op{Gal}(M/K_{\pi,\nu})=\langle\sigma_K\rangle$. As such, we simply compute
		\begin{align*}
			\sigma_K(\alpha_0) &= \sigma_K\left(\gamma\cdot\prod_{i=0}^{f-1}\sigma_K^i\left(\eta_K\right)\right) \\
			&= \sigma_K(\gamma)\cdot\prod_{i=0}^{f-1}\sigma_K^{i+1}\left(\eta_K\right) \\
			&= \sigma_K(\gamma)\cdot\sigma_K^f\left(\eta_K\right)\prod_{i=1}^{f-1}\sigma_K^{i+1}\left(\eta_K\right) \\
			&= \gamma\cdot \eta_K\prod_{i=1}^{f-1}\sigma_K^{i+1}\left(\eta_K\right) \\
			&= \gamma\cdot\prod_{i=0}^{f-1}\sigma_K^{i+1}\left(\eta_K\right) \\
			&= \alpha_0.
		\end{align*}
		\item It suffices to show that $\alpha_i$ is fixed by $\op{Gal}(M/L_i)=\langle\sigma_K^f,\tau_i\rangle$. On one hand,
		\begin{align*}
			\sigma_K^f(\alpha_i) &= \sigma_K^f\left(\prod_{p=0}^{n_i-1}\tau_i^p\left(\eta_i\right)\right) \\
			&= \prod_{p=0}^{n_i-1}\tau_i^p\left(\sigma_K^f\eta_i\right) \\
			&= \left(\prod_{p=0}^{n_i-1}\tau_i^p\left(\frac{\gamma}{\tau_i(\gamma)}\right)\right)\cdot\left(\prod_{p=0}^{n_i-1}\tau_i^p\left(\eta_i\right)\right) \\
			&= \left(\prod_{p=0}^{n_i-1}\frac{\tau_i^p(\gamma)}{\tau_i^{p+1}(\gamma)}\right)\cdot\alpha_i \\
			&= \alpha_i,
		\end{align*}
		where the product telescopes because $\tau_i$ has order $n_i$.

		On the other hand,
		\begin{align*}
			\tau_i(\alpha_i) &= \tau_i\left(\prod_{p=0}^{n_i-1}\tau_i^p\left(\eta_i\right)\right) \\
			&= \prod_{p=0}^{n_i-1}\tau_i^{p+1}\left(\eta_i\right) \\
			&= \prod_{p=0}^{n_i-1}\tau_i^p\left(\eta_i\right),
		\end{align*}
		where we have again used the fact that $\tau_i$ has order $n_i$. This last product is $\alpha_i$, so we are done.
		\item It suffices to show that $\beta_{ij}$ is fixed by $\op{Gal}(M/L)=\langle\sigma_K^f\rangle$. Applying force, we see
		\begin{align*}
			\sigma_K^f(\beta_{ij}) &= \sigma_K^f\left(\frac{\eta_{i}}{\tau_j\eta_i}\cdot\frac{\tau_i\eta_j}{\eta_{j}}\right) \\
			&= \frac{\sigma_K^f\eta_{i}}{\tau_j\sigma_K^f\eta_i}\cdot\frac{\tau_i\sigma_K^f\eta_j}{\sigma_K^f\eta_{j}} \\
			&= \frac{\eta_i\cdot\gamma/\tau_i\gamma}{\tau_j\eta_i\cdot\tau_j\gamma/\tau_i\tau_j\gamma}\cdot
			\frac{\eta_j\cdot\tau_i\gamma/\tau_i\tau_j\gamma}{\eta_j\cdot\gamma/\tau_j\gamma} \\
			&= \frac{\eta_i}{\tau_j\eta_i}\cdot
			\frac{\eta_j}{\eta_j} \\
			&= \beta_{ij}.
		\end{align*}
	\end{listalph}
	The above checks complete the proof.
\end{proof}
Next we show the relations between the $\alpha$s and $\beta$s.
\begin{lemma}
	Fix a tuple $(\alpha_0,\alpha_i),(\beta_{ij})$ as in \autoref{cor:fundtriple}. Then the following are true.
	\begin{listalph}
		\item $\op N_{L/L_i}(\beta_{ij})=\alpha_i/\tau_j\alpha_i$ for $i>j\ge0$.
		\item $\op N_{L/L_0}(\beta_{i0}^{-1})=\alpha_0/\tau_i\alpha_0$.
		\item $\op N_{L/L_j}(\beta_{ij}^{-1})=\alpha_j/\tau_i\alpha_j$ for $i>j>0$.
	\end{listalph}
\end{lemma}
\begin{proof}
	We go one at a time.
	\begin{listalph}
		\item Note $\op{Gal}(L/L_i)=\langle\tau_i\rangle$, so we compute
		\begin{align*}
			\op N_{L/L_i}(\beta_{ij}) &= \prod_{p=0}^{n_i-1}
			\tau_i^p(\beta_{ij}) \\
			&= \prod_{p=0}^{n_i-1}
			\tau_i^p\left(\frac{\eta_{i}}{\tau_j\eta_i}\cdot\frac{\tau_i\eta_j}{\eta_{j}}\right) \\
			&= \prod_{p=0}^{n_i-1}
			\frac{\tau_i^p\eta_{i}}{\tau_j\tau_i^p\eta_i}\cdot\prod_{p=0}^{n_i-1}\frac{\tau_i^{p+1}\eta_j}{\tau_i^p\eta_{j}} \\
			&= \left(\prod_{p=0}^{n_i-1}\tau_i^p\eta_{i}\bigg/\tau_j\prod_{p=0}^{n_i-1}\tau_i^p\eta_i\right)\cdot\frac{\tau_i^{n_i}\eta_j}{\eta_{j}},
		\end{align*}
		which collapses into $\alpha_i/\tau_j\alpha_i$, as needed.
		\item Note $\op{Gal}(L/L_0)=\langle\overline\sigma_K\rangle$. In particular, $\overline\sigma_K$ has order $f$, so we can just compute out
		\begin{align*}
			\op N_{L/L_0}(\beta_{i0}) &= \prod_{p=0}^{f-1}\sigma_K^p(\beta_{i0}) \\
			&= \prod_{p=0}^{f-1}\sigma_K^p\left(\frac{\eta_i}{\sigma_K\eta_i}\cdot\frac{\tau_i\eta_K}{\eta_K}\right) \\
			&= \prod_{p=0}^{f-1}\frac{\sigma_K^p\eta_i}{\sigma_K^{p+1}\eta_i}\cdot\prod_{p=0}^{f-1}\frac{\tau_i\sigma_K^p\eta_K}{\sigma_K^p\eta_K} \\
			&= \frac{\eta_i}{\sigma_K^f\eta_i}\cdot\prod_{p=0}^{f-1}\tau_i\sigma_K^p\eta_K\bigg/\prod_{p=0}^{f-1}\sigma_K^p\eta_K \\
			&= \tau_i\Bigg(\gamma\prod_{p=0}^{f-1}\sigma_K^p\eta_K\Bigg)\bigg/\Bigg(\gamma\prod_{p=0}^{f-1}\sigma_K^p\eta_K\Bigg),
		\end{align*}
		which is what we wanted after taking reciprocals.
		\item This time around, we have $\op{Gal}(L/L_j)=\langle\tau_j\rangle$. As such, we proceed similarly to (a), writing
		\begin{align*}
			\op N_{L/L_j}(\beta_{ij}) &= \prod_{p=0}^{n_j-1}
			\tau_j^p(\beta_{ij}) \\
			&= \prod_{p=0}^{n_j-1}
			\tau_j^p\left(\frac{\eta_{i}}{\tau_j\eta_i}\cdot\frac{\tau_i\eta_j}{\eta_{j}}\right) \\
			&= \prod_{p=0}^{n_j-1}
			\frac{\tau_j^p\eta_{i}}{\tau_j^{p+1}\eta_i}\cdot\prod_{p=0}^{n_j-1}\frac{\tau_i\tau_j^p\eta_j}{\tau_j^p\eta_{j}} \\
			&= \frac{\eta_i}{\tau_j^{n_j}\eta_i}\cdot\left(\tau_i\prod_{p=0}^{n_j-1}\tau_j^p\eta_j\bigg/\prod_{p=0}^{n_j-1}\tau_j^p\eta_j\right),
		\end{align*}
		which again collapses into $\tau_i\alpha_j/\alpha_j$. Taking reciprocals finishes.
	\end{listalph}
	The above checks complete the proof.
\end{proof}
Lastly, here are the relations between the $\beta$s.
\begin{lemma}
	Fix a tuple $(\alpha_0,\alpha_i),(\beta_{ij})$ as in \autoref{cor:fundtriple}. Then, for $i>j>k$, we have
	\[\frac{\tau_j\beta_{ik}}{\beta_{ik}}=\frac{\tau_k\beta_{ij}}{\beta_{ij}}\cdot\frac{\tau_i\beta_{jk}}{\beta_{jk}}.\]
\end{lemma}
\begin{proof}
	As usual, we apply force. Note
	\begin{align*}
		\frac{\tau_k\beta_{ij}}{\beta_{ij}}\cdot\frac{\tau_i\beta_{jk}}{\beta_{jk}} &= \frac{\displaystyle\frac{\tau_k\eta_i}{\tau_k\tau_j\eta_i}\cdot\frac{\tau_k\tau_i\eta_j}{\tau_k\eta_j}}
		{\displaystyle\frac{\eta_i}{\tau_j\eta_i}\cdot\frac{\tau_i\eta_j}{\eta_j}}\cdot
		\frac{\displaystyle\frac{\tau_i\eta_j}{\tau_i\tau_k\eta_j}\cdot\frac{\tau_i\tau_j\eta_k}{\tau_i\eta_k}}
		{\displaystyle\frac{\eta_j}{\tau_k\eta_j}\cdot\frac{\tau_j\eta_k}{\eta_k}} \\
		&= \frac{\tau_k\eta_i}{\tau_k\tau_j\eta_i}\cdot\frac{\tau_k\tau_i\eta_j}{\tau_k\eta_j}\cdot
		\frac{\tau_j\eta_i}{\eta_i}\cdot\frac{\eta_j}{\tau_i\eta_j}\cdot
		\frac{\tau_i\eta_j}{\tau_i\tau_k\eta_j}\cdot\frac{\tau_i\tau_j\eta_k}{\tau_i\eta_k}\cdot
		\frac{\tau_k\eta_j}{\eta_j}\cdot\frac{\eta_k}{\tau_j\eta_k} \\
		&= \frac{\tau_k\eta_i}{\tau_k\tau_j\eta_i}\cdot
		\frac11\cdot
		\frac{\tau_j\eta_i}{\eta_i}\cdot
		\frac11\cdot
		\frac11\cdot
		\frac{\tau_i\tau_j\eta_k}{\tau_i\eta_k}\cdot
		\frac11\cdot\frac{\eta_k}{\tau_j\eta_k} \\
		&= \frac{\tau_j\eta_i}{\tau_k\tau_j\eta_i}\cdot
		\frac{\tau_i\tau_j\eta_k}{\tau_j\eta_k}\cdot
		\frac{\tau_k\eta_i}{\eta_i}\cdot
		\frac{\eta_k}{\tau_i\eta_k},
	\end{align*}
	which is what we wanted.
\end{proof}

\subsection{Consequences}
\begin{warn}
	The following section does not use the notation of the rest of the article.
\end{warn}
With some checks out of the way, here are some actual consequences. To begin, we state Hilbert's Theorem 90.
\begin{lemma} \label{lem:hilbert90}
	Suppose that $L/K$ is a (finite) cyclic extension of fields such that $\Gamma\coloneqq\op{Gal}(L/K)$ is generated by $\sigma\in\Gamma$. Given some $\alpha\in L^\times$ such that $\op N(\alpha)=1$, there exists $\beta_0\in L^\times$ such that $\alpha=\beta_0/\sigma\beta_0$. In fact, this $\beta_0$ is unique ``up to a multiple in $K^\times$'' in the sense that
	\[\left\{\beta\in L^\times:\alpha=\beta/\sigma\beta\right\}=\left\{x\beta_0:x\in K^\times\right\}.\]
\end{lemma}
\begin{proof}
	That such a $\beta_0$ exists follows directly from Hilbert's Theorem 90. For the last sentence, of course any $\beta\coloneqq x\beta_0\in L^\times$ with $x\in K^\times$ will have
	\[\frac\beta{\sigma\beta}=\frac{\beta_0}{\sigma\beta_0}=\alpha.\]
	In the other direction, if $\beta\in L^\times$ has $\beta/\sigma\beta=\alpha$, then
	\[\sigma(\beta/\beta_0)=(\sigma\beta)/(\sigma\beta_0)=\beta/\beta_0,\]
	so $\beta/\beta_0\in K^\times$ and $\beta=(\beta/\beta_0)\cdot\beta_0$.
\end{proof}
And here are some quick consequences of this.
\begin{cor}
	Fix everything as in the set-up, and fix $\alpha\in ML^\times$ such that $\op N_{ML/L}(\alpha)=p$. Choosing some $\sigma\in\{\sigma_K,\sigma_x\}$, the elements $\eta_{\sigma}$ satisfying
	\[\frac{\eta_{\sigma}}{\sigma_K^f\left(\eta_{\sigma}\right)}=\frac{\sigma(\alpha)}{\alpha}\]
	are unique up to a multiple in $L^\times$, in the sense of \autoref{lem:hilbert90}.
\end{cor}
\begin{proof}
	Note that $\op{Gal}(ML/L)=\langle\sigma_K^f\rangle$ is cyclic generated by $\sigma_K^f$ and $\op N_{ML/L}(\sigma\alpha/\alpha)=p/p=1$, so we may simply apply \autoref{lem:hilbert90} directly to get the result.
\end{proof}
We might be worried that our choice $\alpha$ is affecting the set of $\eta_{c(\sigma_K)}$ or $\eta_{c(\sigma_x)}$, but in fact they are not, more or less.
\begin{cor} \label{cor:updatealpha}
	Fix everything as in the set-up, and choose $\sigma\in\{\sigma_K,\sigma_x\}$. Given $\alpha\in ML^\times$ such that $\op N_{ML/L}(\alpha)=p$, define
	\[S_\alpha\coloneqq\left\{\eta_\sigma\in ML^\times:\frac{\eta_{\sigma}}{\sigma_K^f\left(\eta_{\sigma}\right)}=\frac{\sigma(\alpha)}{\alpha}\right\}.\]
	Then the set $S_\alpha$ is ``unique up to a multiple in $ML^\times$'' in the sense that two $\alpha,\alpha'\in ML^\times$ with $\op N_{ML/L}(\alpha)=\op N_{ML/L}(\alpha')=p$ have some $\chi\in ML^\times$ such that
	\[S_{\alpha}=\chi\cdot S_{\alpha'}\coloneqq\{\chi\cdot \eta_\sigma:\eta_\sigma\in S_{\alpha'}\}.\]
\end{cor}
\begin{proof}
	Suppose $\alpha,\alpha'\in ML^\times$ satisfy $\op N_{ML/L}(\alpha)=\op N_{ML/L}(\alpha')=p$. The key point is that
	\[\op N_{ML/L}(\alpha/\alpha')=p/p=1,\]
	so \autoref{lem:hilbert90} promises us some $\gamma\in ML^\times$ such that $\alpha/\alpha'=\gamma/\sigma_K^f(\gamma)$. As such, we see that
	\[\frac{\sigma(\alpha)}{\alpha}=\frac{\sigma(\alpha/\alpha')}{\alpha/\alpha'}\cdot\frac{\sigma(\alpha')}{\alpha'}=\frac{(\sigma\gamma/\gamma)}{\sigma_K^f(\sigma\gamma/\gamma)}\cdot\frac{\sigma(\alpha')}{\alpha'}.\]
	As such, we set $\chi\coloneqq(\sigma\gamma/\gamma)$.
	
	To finish, we check that $S_\alpha\subseteq \chi\cdot S_{\alpha'}$, and the other inclusion is similar. Well, if $\eta_\sigma\in S_{\alpha'}$, then
	\[\frac{\chi \eta_\sigma}{\sigma_K^f(\chi \eta_\sigma)}=\frac \chi {\sigma_K^f(\chi )}\cdot\frac{\eta_\sigma}{\sigma_K^f(\eta_\sigma)}=\frac{(\sigma\gamma/\gamma)}{\sigma_K^f(\sigma\gamma/\gamma)}\cdot\frac{\sigma(\alpha')}{\alpha'}=\frac{\sigma(\alpha)}\alpha,\]
	so $\chi \eta_\sigma\in S_\alpha$. This finishes.
	% \begin{itemize}
	% 	\item We check $x\cdot S_{\alpha'}\subseteq S_\alpha$. Well, if $\eta_\sigma\in S_{\alpha'}$, then
	% 	\[\frac{x\eta_\sigma}{\sigma_K^f(x\eta_\sigma)}=\frac x{\sigma_K^f(x)}\cdot\frac{\eta_\sigma}{\sigma_K^f(\eta_\sigma)}=\frac{(\sigma\gamma/\gamma)}{\sigma_K^f(\sigma\gamma/\gamma)}\cdot\frac{\sigma(\alpha')}{\alpha'}=\frac{\sigma(\alpha)}\alpha,\]
	% 	so $x\eta_\sigma\in S_\alpha$.
	% 	\item We check $S_\alpha\subseteq x\cdot S_{\alpha'}$. Again, if $\eta_\sigma\in S_\alpha$, then
	% 	\[\frac{x^{-1}\eta_\sigma}{\sigma_K^f(x^{-1}\eta_\sigma)}=\frac {x^{-1}}{\sigma_K^f(x^{-1})}\cdot\frac{\eta_\sigma}{\sigma_K^f(\eta_\sigma)}=\left(\frac{(\sigma\gamma/\gamma)}{\sigma_K^f(\sigma\gamma/\gamma)}\right)^{-1}\cdot\frac{\sigma(\alpha)}{\alpha}=\frac{\sigma(\alpha')}{\alpha'},\]
	% 	so $\eta_\sigma=x\cdot x^{-1}\eta_\sigma\in x\cdot S_{\alpha'}$.
	% \end{itemize}
	% The above two inclusions complete the proof.
\end{proof}
We now return to describing triples.
\begin{cor} \label{cor:fullclass}
	Fix everything as in the set-up, and fix $\alpha\in ML^\times$ such that $\op N_{ML/L}(\alpha)=p$. Then, for any triple $(\alpha_1',\alpha_2',\beta')$ corresponding to the fundamental class, there exist elements $\eta_{c(\sigma_K)}',\eta_{c(\sigma_x)}'\in ML^\times$ with
	\[\frac{\eta_{c(\sigma_K)}'}{\sigma_K^f\left(\eta_{c(\sigma_K)}'\right)}=\frac{\sigma_K(\alpha)}{\alpha}\qquad\text{and}\qquad\frac{\eta_{c(\sigma_x)}'}{\sigma_K^f\left(\eta_{c(\sigma_x)}'\right)}=\frac{\sigma_x(\alpha)}{\alpha}\]
	such that
	\[(\alpha_1',\alpha_2',\beta')=\left(\alpha\cdot\prod_{i=0}^{f-1}\sigma_K^i\left(\eta_{c(\sigma_K)}'\right),\quad\prod_{i=0}^{\varphi\left(p^\nu\right)-1}\sigma_x^i\left(\eta_{c(\sigma_x)}'\right),\quad\frac{\sigma_K\left(\eta_{c(\sigma_x)}'\right)}{\eta_{c(\sigma_x)}'}\cdot\frac{\eta_{c(\sigma_K)}'}{\sigma_x\left(\eta_{c(\sigma_K)}'\right)}\right).\]
	In other words, all triples corresponding to the fundamental class come from the recipe described in \autoref{cor:fundtriple}.
\end{cor}
\begin{proof}
	By \autoref{cor:fundtriple}, we can certainly find some elements $\eta_{c(\sigma_K)},\eta_{c(\sigma_x)}\in ML^\times$ such that
	\[\frac{\eta_{c(\sigma_K)}}{\sigma_K^f\left(\eta_{c(\sigma_K)}\right)}=\frac{\sigma_K(\alpha)}{\alpha}\qquad\text{and}\qquad\frac{\eta_{c(\sigma_x)}}{\sigma_K^f\left(\eta_{c(\sigma_x)}\right)}=\frac{\sigma_x(\alpha)}{\alpha},\]
	for which
	\[(\alpha_1,\alpha_2,\beta)\coloneqq\left(\alpha\cdot\prod_{i=0}^{f-1}\sigma_K^i\left(\eta_{c(\sigma_K)}\right),\quad\prod_{i=0}^{\varphi\left(p^\nu\right)-1}\sigma_x^i\left(\eta_{c(\sigma_x)}\right),\quad\frac{\sigma_K\left(\eta_{c(\sigma_x)}\right)}{\eta_{c(\sigma_x)}}\cdot\frac{\eta_{c(\sigma_K)}}{\sigma_x\left(\eta_{c(\sigma_K)}\right)}\right)\]
	corresponds to the fundamental class $u_{L/K}\in H^2(\op{Gal}(L/K),L^\times)$. In particular, $(\alpha_1,\alpha_2,\beta)$ and $(\alpha_1',\alpha_2',\beta')$ both correspond to the same cohomology class and hence in the same equivalence class of triples, so we know that there exist $m_1,m_2\in L^\times$ such that
	\[\alpha_1'=\alpha_1\cdot\op N_{L/L_1}(m_1),\quad\alpha_2'=\alpha_2\cdot\op N_{L/L_2}(m_2),\quad\beta'=\beta\cdot\frac{\sigma_K(m_2)}{m_2}\cdot\frac{m_1}{\sigma_x(m_1)}.\]
	As such, we set $\eta_{c(\sigma_K)}'\coloneqq \eta_{c(\sigma_K)}\cdot m_1$ and $\eta_{c(\sigma_x)}'\coloneqq \eta_{c(\sigma_x)}\cdot m_2$, and these can be checked to work. For example, $\eta_{c(\sigma_K)}'$ satisfies
	\[\frac{\eta_{c(\sigma_K)}'}{\sigma_K^f\left(\eta_{c(\sigma_K)}'\right)}=\frac{\sigma_K(\alpha)}{\alpha}\qquad\text{and}\qquad\frac{\eta_{c(\sigma_x)}'}{\sigma_K^f\left(\eta_{c(\sigma_x)}'\right)}=\frac{\sigma_x(\alpha)}{\alpha}\]
	by \autoref{lem:hilbert90}. The rest of the checks are similar.
\end{proof}
\begin{corollary} \label{cor:classofa1}
	Fix everything as in the set-up, and let $\pi_1\in L_1^\times$ be a uniformizer. If the triple $(\alpha_1,\alpha_2,\beta)$ is a triple corresponding to the fundamental class, then
	\[\alpha_1\equiv\pi_1\pmod{\op N_{L/L_1}(L^\times)}.\]
\end{corollary}
\begin{proof}[Proof by triples]
	Note that $L/L_1$ is an unramified extension, so all elements of absolute value $1$ are norms, so there is in fact a class of elements containing all uniformizers in $L_1^\times/\op N_{L/L_1}(L^\times)$. Further, because $\alpha_1$ is also only defined up to a multiple in $\op N_{L/L_1}(L^\times)$, to show that the classes in $L^\times/\op N_{L/L_1}(L^\times)$ coincide, it thus suffices to exhibit a single triple $(\alpha_1,\alpha_2,\beta)$ such that $\alpha_1\in L_1^\times$ is a uniformizer.

	This is a matter of force. To begin, we can use \autoref{cor:fundtriple} to find some $\alpha$ with $\op N_{ML/L}(\alpha)=p$ and $\eta_{c(\sigma_K)},\eta_{c(\sigma_x)}\in ML^\times$ giving the triple $(\alpha_1,\alpha_2,\beta)$ as described. The idea is to force $\eta_{c(\sigma_K)}$ to have valuation zero.
	
	Let $v_{ML}$ be the fixed valuation of $ML$ extending the standard valuation $v_{\QQ_p}$ on $\QQ_p$, and let $v_{L}$ be its restriction to $L$. Because $ML/L$ is an unramified, the image of $v_{ML}$ and $v_L$ in $\QQ$ is the same. In particular, we can find some $m_1\in L_1^\times$ such that
	\[v_{ML}\left(\eta_{c(\sigma_K)}\right)=v_L(m_1).\]
	Thus, we replace $\eta_{c(\sigma_K)}$ with $\eta_{c(\sigma_K)}/m_1$, and we still satisfy the conditions of \autoref{cor:fundtriple} by \autoref{lem:hilbert90} while getting $v_{ML}\left(\eta_{c(\sigma_K)}\right)=0$. Now, the corresponding $\alpha_1$ looks like
	\[\alpha_1=\alpha\cdot\prod_{i=0}^{f-1}\sigma_K^i\left(\eta_{c(\sigma_K)}\right).\]
	In particular, defining $v_{L_1}\coloneqq v_L|_{L_1}$, it follows
	\[v_{L_1}(\alpha_1)=v_{ML}(\alpha_1)=v_{ML}(\alpha),\]
	However, $\op N_{ML/L}(\alpha)=p$ by construction, so we see that
	\[[ML:L]v_{ML}(\alpha)=v_{ML}(p)=v_{\QQ_p}(p)=1.\]
	Explicitly, we see that
	\[[ML:L]=[\QQ(\zeta_{N'}):\QQ(\zeta_m)]=\frac{[\QQ(\zeta_{N'}):\QQ_p]}{[\QQ_p(\zeta_m):\QQ_p]}=\frac nf=\varphi\left(p^\nu\right).\]
	However, $L_1/K$ has ramification degree $\varphi\left(p^\nu\right)$ (from the maximal totally ramified subextension $\QQ_p(\zeta_{p^\nu})$), so its uniformizers are the elements of valuation $1/\varphi\left(p^\nu\right)$. Thus, we have computed that $\alpha_1$ has the correct valuation and hence is a uniformizer.
\end{proof}
\begin{proof}[Proof by the Artin map]
	We take a moment to say that there is an alternate derivation of \autoref{cor:classofa1} using the Artin map: one can show that, if $u\in Z^2(L/K)$ is a representative of the fundamental class of an abelian extension $L/K$, then
	\begin{align*}
		\op{Gal}(L/K) &\to K^\times/\op N(L^\times) \\
		\sigma &\mapsto \prod_{g\in\op{Gal}(L/K)}u(g,\sigma)
	\end{align*}
	is the inverse Artin map. In particular, from our explicit formula for $\alpha_1$, we see
	\[\alpha_1=\prod_{g\in\op{Gal}(L/L_1)}u(g,\overline\sigma_K)=\theta_{L/L_1}^{-1}(\overline\sigma_K).\]
	However, $\overline\sigma_K$ is the Frobenius automorphism of $L/L_1$ because the extension $L_1/K$ is totally ramified, implying that the residue field of $L_1$ is the same as $K=\QQ_p$. Thus, $\theta_{L/L_1}^{-1}(\overline\sigma_K)$ is the class containing the uniformizers of $L_1^\times$.
\end{proof}
We close with a sanity check.
\begin{cor}
	Fix everything as in the set-up, and let $T_\alpha$ denote the set of triples $(\alpha_1,\alpha_2,\beta)$ generated by some element $\alpha\in ML^\times$ with $\op N_{ML/L}(\alpha)=p$ via \autoref{cor:fundtriple}. Then $T_\alpha$ is independent of $\alpha$.
\end{cor}
\begin{proof}
	The main idea is to use (the proof of) \autoref{cor:updatealpha}. Fix $\alpha,\alpha'\in ML^\times$ with $\op N_{ML/L}(\alpha)=\op N_{ML/L}(\alpha')=p$, and we need to show that $T_\alpha=T_{\alpha'}$. By symmetry, it will be enough for $T_\alpha\subseteq T_{\alpha'}$.

	Following the proof of \autoref{cor:updatealpha}, note that $\op N_{ML/L}(\alpha/\alpha')=1$, so we are promised $\gamma\in ML^\times$ such that $\alpha/\alpha'=\gamma/\sigma_K^f(\gamma)$. Then we showed that any $\sigma\in\{\sigma_K,\sigma_x\}$ can set
	\[\chi_\sigma\coloneqq\frac{\sigma(\gamma)}\gamma\]
	to give $S_{\alpha,\sigma} x=\cdot S_{\alpha',\sigma}$, where $S_{\alpha,\sigma}$ is the set of possible $\eta_\sigma$ defined in \autoref{cor:updatealpha}.

	We now proceed directly with the proof. Suppose that we have some triple $(\alpha_1,\alpha_2,\beta)\in T_\alpha$, which we know that we can write down as
	\[(\alpha_1,\alpha_2,\beta)=\left(\alpha\cdot\prod_{i=0}^{f-1}\sigma_K^i\left(\eta_{\sigma_K}\right),\quad\prod_{i=0}^{\varphi\left(p^\nu\right)-1}\sigma_x^i\left(\eta_{\sigma_x}\right),\quad\frac{\sigma_K\left(\eta_{\sigma_x}\right)}{\eta_{\sigma_x}}\cdot\frac{\eta_{\sigma_K}}{\sigma_x\left(\eta_{\sigma_K}\right)}\right)\]
	for some $\eta_{\sigma_K}\in S_{\alpha,\sigma_K}$ and $\eta_{\sigma_x}\in S_{\alpha,\sigma_x}$. We need to show that $(\alpha_1,\alpha_2,\beta)\in T_{\alpha'}$. Well, by \autoref{cor:updatealpha}, we can set
	\[I'_{\sigma}\coloneqq \eta_{\sigma}/\chi_\sigma\in S_{\alpha',\sigma}\]
	for $\sigma\in\{\sigma_K,\sigma_x\}$. We now compute
	\begin{align*}
		\alpha_1 ={}& \alpha\cdot\prod_{i=0}^{f-1}\sigma_K^i(\eta_{\sigma_K}) \\
		={}& \alpha\cdot\prod_{i=0}^{f-1}\sigma_K^i(\chi_\sigma)\cdot\prod_{i=0}^{f-1}\sigma_K^i(\eta_{\sigma_K'}) \\
		={}& \alpha\cdot\prod_{i=0}^{f-1}\sigma_K^i\left(\frac{\sigma_K\gamma}{\gamma}\right)\cdot\prod_{i=0}^{f-1}\sigma_K^i(\eta_{\sigma_K'}) \\
		={}& \alpha\cdot\frac{\sigma_K^f(\gamma)}{\gamma}\cdot\prod_{i=0}^{f-1}\sigma_K^i(\eta_{\sigma_K'}) \\
		={}& \alpha'\cdot\prod_{i=0}^{f-1}\sigma_K^i(\eta_{\sigma_K'}),
	\end{align*}
	where the last equality holds by definition of $\gamma$. Similarly, we see
	\begin{align*}
		\alpha_2 ={}& \prod_{i=0}^{\varphi\left(p^\nu\right)-1}\sigma_x^i(\eta_{\sigma_x}) \\
		={}& \prod_{i=0}^{\varphi\left(p^\nu\right)-1}\sigma_x^i(\chi_{\sigma_x})\cdot\prod_{i=0}^{\varphi\left(p^\nu\right)-1}\sigma_x^i(\eta_{\sigma_x}') \\
		={}& \prod_{i=0}^{\varphi\left(p^\nu\right)-1}\sigma_x^i\left(\frac{\sigma_x(\gamma)}{\gamma}\right)\cdot\prod_{i=0}^{\varphi\left(p^\nu\right)-1}\sigma_x^i(\eta_{\sigma_x}') \\
		={}& \prod_{i=0}^{\varphi\left(p^\nu\right)-1}\sigma_x^i(\eta_{\sigma_x}'),
	\end{align*}
	where the product telescopes in the last equality because $\sigma_x$ has order $\varphi\left(p^\nu\right)$. Lastly, we set
	\begin{align*}
		\beta &= \frac{\sigma_K\left(\eta_{\sigma_x}\right)}{\eta_{\sigma_x}}\cdot\frac{\eta_{\sigma_K}}{\sigma_x\left(\eta_{\sigma_K}\right)} \\
		&= \frac{\sigma_K\left(\chi_{\sigma_x}\right)}{\chi_{\sigma_x}}\cdot\frac{\chi_{\sigma_K}}{\sigma_x\left(\chi_{\sigma_K}\right)}\cdot\frac{\sigma_K\left(\eta_{\sigma_x}'\right)}{\eta_{\sigma_x}'}\cdot\frac{\eta_{\sigma_K}'}{\sigma_x\left(\eta_{\sigma_K}'\right)} \\
		&= \frac{\sigma_K\sigma_x\gamma/\sigma_K\gamma}{\sigma_x\gamma/\gamma}\cdot\frac{\sigma_K\gamma/\gamma}{\sigma_x\sigma_K\gamma/\sigma_x\gamma}\cdot\frac{\sigma_K\left(\eta_{\sigma_x}'\right)}{\eta_{\sigma_x}'}\cdot\frac{\eta_{\sigma_K}'}{\sigma_x\left(\eta_{\sigma_K}'\right)} \\
		&= \frac{\sigma_K\left(\eta_{\sigma_x}'\right)}{\eta_{\sigma_x}'}\cdot\frac{\eta_{\sigma_K}'}{\sigma_x\left(\eta_{\sigma_K}'\right)}.
	\end{align*}
	Thus,
	\[(\alpha_1,\alpha_2,\beta)=\left(\alpha'\cdot\prod_{i=0}^{f-1}\sigma_K^i\left(\eta_{\sigma_K}'\right),\quad\prod_{i=0}^{\varphi\left(p^\nu\right)-1}\sigma_x^i\left(\eta_{\sigma_x}'\right),\quad\frac{\sigma_K\left(\eta_{\sigma_x}'\right)}{\eta_{\sigma_x}'}\cdot\frac{\eta_{\sigma_K}'}{\sigma_x\left(\eta_{\sigma_K}'\right)}\right)\in T_{\alpha'},\]
	which finishes.
\end{proof}

\section{An Explicit Example}
In this section, we work through \autoref{cor:fundtriple} very explicitly in a basic case. Let $p$ be an odd prime because the following discussion has no content in the case of $p=2$. Set $K\coloneqq\QQ_p$ and $K_m\coloneqq\QQ_p(\zeta_m)$ with $f\coloneqq[\QQ_p(\zeta_m):\QQ_p]$.

The main simplification we will make which allows explicit computation is that we will set $K_{\pi,\nu}\coloneqq\QQ_p(\zeta_p)$. Continuing with the set-up, we see $L=\QQ_p(\zeta_p,\zeta_m)$ with $n\coloneqq (p-1)\cdot f$; as such, set $N'\coloneqq p^n-1$ so that $M=\QQ_p(\zeta_{N'})$. Here is the diagram of our fields.
% https://q.uiver.app/?q=WzAsNixbMSwzLCJcXFFRX3AiXSxbMCwyLCJcXFFRX3AoXFx6ZXRhX3ApIl0sWzIsMiwiXFxRUV9wKFxcemV0YV9tKSJdLFsxLDEsIlxcUVFfcChcXHpldGFfcCxcXHpldGFfbSkiXSxbMywxLCJcXFFRX3AoXFx6ZXRhX3tOJ30pIl0sWzIsMCwiXFxRUV9wKFxcemV0YV9wLFxcemV0YV97Tid9KSJdLFswLDEsIiIsMCx7InN0eWxlIjp7ImhlYWQiOnsibmFtZSI6Im5vbmUifX19XSxbMCwyLCIiLDIseyJzdHlsZSI6eyJoZWFkIjp7Im5hbWUiOiJub25lIn19fV0sWzEsMywiIiwwLHsic3R5bGUiOnsiaGVhZCI6eyJuYW1lIjoibm9uZSJ9fX1dLFsyLDMsIiIsMix7InN0eWxlIjp7ImhlYWQiOnsibmFtZSI6Im5vbmUifX19XSxbMiw0LCIiLDIseyJzdHlsZSI6eyJoZWFkIjp7Im5hbWUiOiJub25lIn19fV0sWzQsNSwiIiwyLHsic3R5bGUiOnsiaGVhZCI6eyJuYW1lIjoibm9uZSJ9fX1dLFszLDUsIiIsMix7InN0eWxlIjp7ImhlYWQiOnsibmFtZSI6Im5vbmUifX19XV0=&macro_url=https%3A%2F%2Fraw.githubusercontent.com%2FdFoiler%2Fnotes%2Fmaster%2Fnir.tex
\[\begin{tikzcd}
	&& {\QQ_p(\zeta_p,\zeta_{N'})} \\
	& {\QQ_p(\zeta_p,\zeta_m)} && {\QQ_p(\zeta_{N'})} \\
	{\QQ_p(\zeta_p)} && {\QQ_p(\zeta_m)} \\
	& {\QQ_p}
	\arrow[no head, from=4-2, to=3-1]
	\arrow[no head, from=4-2, to=3-3]
	\arrow[no head, from=3-1, to=2-2]
	\arrow[no head, from=3-3, to=2-2]
	\arrow[no head, from=3-3, to=2-4]
	\arrow[no head, from=2-4, to=1-3]
	\arrow[no head, from=2-2, to=1-3]
\end{tikzcd}\]
Now, the reason we set $K_{\pi,\nu}=\QQ_p(\zeta_p)$ is that we can show that
\[\gamma\coloneqq(-p)^{1/(p-1)}\in\QQ_p(\zeta_p).\]
Indeed, we sneakily set $\pi=-p$ to be our uniformizer of $\QQ_p$ so that $\op N_{ML/L}(\gamma)=\gamma^{p-1}=-p$. Because it will be helpful for us shortly, we will actually give a construction of $(-p)^{1/(p-1)}$.
\begin{lemma} \label{lem:findrootofp}
	Let $p$ be a prime. Then we can find some $\gamma\coloneqq(-p)^{1/(p-1)}$ in $\QQ_p(\zeta_p)$. In fact, we can take $\gamma\equiv c\pi\pmod{\pi^2}$ for any $c\in\FF_p^\times$.
\end{lemma}
\begin{proof}[Sketch]
	We follow Professor Andrew Sutherland's \href{https://math.mit.edu/classes/18.785/2021fa/LectureNotes20.pdf\#theorem.2.5}{Lemma~20.5}. Set $\pi\coloneqq\zeta_p-1$ to be a uniformizer of $\QQ_p(\zeta_p)$. Now, the minimal polynomial of $\zeta_p$ is
	\[f(T)\coloneqq\frac{(T+1)^p-1}T,\]
	which is $p$-Eisenstein. To properly apply Hensel's lemma to solve $T^{p-1}+p$, we see that any solution should be divisible by $\pi$, so we divide out by this first. Note $v(\pi)=1/(p-1)$, so $u\coloneqq-\pi^{p-1}/p\in\mathcal O_{\QQ_p(\zeta_p)}^\times$. In fact, we can see from the polynomial $f$ that
	\[\pi^{p-1}+p\equiv0\pmod{p\pi},\]
	so $u\equiv-1\pmod\pi$. As such, we now note that $g(T)\coloneqq T^{p-1}-u$ has
	\[g(c)\equiv0\pmod\pi\qquad\text{and}\qquad g'(c)=(p-1)c\not\equiv0\pmod\pi,\]
	for any $c\in\FF_p^\times$, so we can lift $c$ to a root $\beta_c\in\mathcal O_{\QQ_p(\zeta_p)}$. From here, we see $(\pi/\beta_c)^{p-1}=\pi^{p-1}/u=-p$, so $\pi/\beta_c$ is our desired root. For the last statement, we see
	\[\pi/\beta_c\equiv c^{-1}\pi\pmod{\pi^2},\]
	so as $c\in\FF_p^\times$ varies, we do indeed get all equivalence classes.
\end{proof}
In light of \autoref{lem:findrootofp}, we will just take $\gamma$ to have $\gamma^{p-1}=-p$ with $\gamma\equiv c\pi\pmod{\pi^2}$. This satisfies $\op N_{ML/L}(\gamma)=-p$ as discussed above.

We start with the unramified side because it is easier. Namely, $\gamma\in\QQ_p(\zeta_p)$ is fixed by the Frobenius automorphism $\sigma_K$, so we may set $\eta_K\coloneqq1$ to have
\[\frac{\eta_K}{\sigma_K^f(\eta_K)}=1=\frac{\sigma_K(\gamma)}{\gamma}.\]
The corresponding $\alpha_0$ is thus
\[\boxed{\alpha_0=\gamma}.\]
We now deal with ramification. Observe $\op{Gal}(\QQ_p(\zeta_p)/\QQ_p)\simeq(\ZZ/p\ZZ)^\times$ is cyclic, but we must choose a generator nonetheless. Let $x\in(\ZZ/p\ZZ)^\times$ be a generator, and let $\sigma_x\colon\zeta_p\mapsto\zeta_p^x$ be the corresponding automorphism; namely, $\tau_1\coloneqq\sigma_x$. (Notably, this is not the automorphism generated by the Artin map; we will return to this point later.) Here is the corresponding computation.
\begin{lemma}
	Fix everything as above. Then $\zeta_{p-1}\coloneqq\sigma_x(\gamma)/\gamma$ is a primitive $p-1$st root of unity and in particular lies in $\QQ_p$. In fact, $\zeta_{p-1}\equiv x\pmod p$.
\end{lemma}
Note that we are defining $\zeta_{p-1}$ above, which is okay: in the worst case, we might have to adjust the definitions of $\zeta_{N'}$ and $\zeta_m$ to correspond with this particular $\zeta_{p-1}$, but otherwise $\zeta_{p-1}$ may be any fixed primitive $p-1$st root of unity.
\begin{proof}
	To see that $\zeta_{p-1}$ is a $p-1$st root of unity, we note that $\sigma_x(\gamma)=\zeta_{p-1}\cdot\gamma$, so an induction shows that
	\[\sigma_x^k(\gamma)=\zeta_{p-1}^k\cdot\gamma.\]
	Setting $k=p-1$ shows that $\zeta_{p-1}^{p-1}=1$, so $\zeta_{p-1}$ is a $p-1$st root of unity. To show that $\zeta_{p-1}$ is primitive, we know that $\zeta_{p-1}^k=1$ above would imply that $\sigma_x^k(\gamma)=\gamma$, but $\QQ_p(\gamma)=\QQ_p(\zeta_p)$ (we already know $\QQ_p(\gamma)\subseteq\QQ_p(\zeta_p)$, but both of these extensions have degree $p-1$), so in fact $\sigma_x^k=\id$. So $x\in(\ZZ/p\ZZ)^\times$ being a generator requires $p-1\mid k$. So indeed, the least positive integer $k$ with $\zeta_p^k=1$ is $k=p-1$.

	We now quickly note that $\QQ_p$ contains all $p-1$st roots of unity by Hensel's lemma because the polynomial $T^{p-1}-1\in\FF_p[T]$ fully splits into $p-1$ distinct factors; in particular, $\zeta_{p-1}\in\QQ_p$. In fact, Hensel's lemma tells us that the $p-1$st roots of unity of $\QQ_p$ fully represent $(\ZZ/p\ZZ)^\times$, so there is a chance for $\zeta_{p-1}\equiv x\pmod p$.

	Well, it remains to show $\zeta_{p-1}\equiv x\pmod p$. Let $\pi\coloneqq\zeta_p-1$ be a uniformizer of $\QQ_p(\zeta_p)$. Because $\zeta_{p-1},x\in\QQ_p$, it is enough for $v_\QQ(\zeta_{p-1}-x)>0$; as such, we will show that
	\[\zeta_{p-1}\stackrel?\equiv x\pmod\pi.\]
	To see this, recall $\gamma\equiv c\pi\pmod{\pi^2}$, so find $c_1\in\mathcal O_{\QQ_p(\zeta_p)}$ with $\gamma=c\pi+c_1\pi^2$. Now, observe $\sigma_x(\pi)=\zeta_p^x-1$ is another uniformizer and in particular divisible by $p$, so we may write
	\[\sigma_x(\gamma)=\left(\zeta_p^x-1\right)+c_2\pi^2\]
	for some $c_2\in\mathcal O_{\QQ_p(\zeta_p)}$. Thus, we note
	\[\frac{\zeta_p^x-1}{\zeta_p-1}\equiv1+\zeta_p+\cdots+\zeta_p^{x-1}\equiv\underbrace{1+\cdots+1}_x\equiv x\pmod\pi.\]
	It follows that
	\[\sigma_x(\gamma)\equiv\zeta_p^x-1\equiv x(\zeta_p-1)\equiv x\gamma\pmod{\pi^2},\]
	which is enough.
\end{proof}
We are almost able to compute $\eta_x\coloneqq\eta_1$. To do this, we pick up a quick lemma.
\begin{lemma}
	Let $p$ and $f$ be integers. Then
	\[\frac{p^{f(p-1)}-1}{(p-1)\left(p^f-1\right)}\in\ZZ.\]
\end{lemma}
\begin{proof}
	Observe
	\[\frac{p^{f(p-1)}-1}{p^f-1} = \sum_{k=0}^{p-1}p^{fk} \equiv \sum_{k=0}^{p-1}1=p-1\equiv0\pmod{p-1}.\]
	This finishes.
\end{proof}
In light of the above lemma, we define
\[z\coloneqq-\frac{p^{f(p-1)}-1}{(p-1)\left(p^f-1\right)}.\]
Note the sign here; it is very important! It follows that $\eta_x\coloneqq\zeta_{N'}^z$ will have
\begin{align*}
	\frac{\eta_x}{\sigma_K^f(\eta_x)} &= \frac{\zeta_{N'}^z}{\zeta_{N'}^{zp^f}} \\
	&= \zeta_{N'}^{-z\left(p^f-1\right)} \\
	&= \zeta_{N'}^{N'/(p-1)} \\
	&= \zeta_{p-1},
\end{align*}
which is indeed $\sigma_x(\gamma)/\gamma$. Notably, we have $\eta_x\in\QQ_p(\zeta_{N'})$, which is fixed by $\sigma_x$. Thus, the corresponding $\alpha_1$ is thus
\begin{align*}
	\alpha_1 &= \prod_{i=0}^{p-1}\sigma_x^i(\eta_i) \\
	&= \eta_i^{p-1} \\
	&= \zeta^{z(p-1)}_{N'} \\
	&= \zeta_{N'}^{-N'/\left(p^f-1\right)} \\
	\Aboxed{\alpha_1 &= \zeta_{p^f-1}^{-1}}.
\end{align*}
Lastly, we compute our $\beta_{10}$ as
\begin{align*}
	\beta_{10} &= \frac{\eta_K}{\sigma_x\eta_K}\cdot\frac{\sigma_K\eta_x}{\eta_x} \\
	&= \zeta_{N'}^{z(p-1)} \\
	\Aboxed{\beta_{10} &= \zeta_{p^f-1}^{-1}}.
\end{align*}
In total, we get the following nice result.
\begin{theorem}
	Let $p$ be an odd prime, and fix $K\coloneqq\QQ_p$ and $L\coloneqq\QQ_p(\zeta_p,\zeta_m)$, where $p\nmid m$. Further, set $L_0\coloneqq\QQ_p(\zeta_p)$ and $L_1\coloneqq\QQ_p(\zeta_m)$ so that $L=L_0L_1$ and $L_0\cap L_1=K$. Now, pick up the following data.
	\begin{itemize}
		\item Suppose the order of $p$ modulo $m$ is $f$.
		\item Let $\sigma_x\colon\zeta_p\mapsto\zeta_p^x$ be a generator of $\op{Gal}(\QQ_p(\zeta_p)/\QQ_p)$.
		\item Find $\gamma\in\QQ_p(\zeta_p)$ such that $\gamma^{p-1}+p=0$ and $\sigma_x(\gamma)/\gamma=\zeta_{p-1}$. (Equivalently, set $\zeta_{p-1}\coloneqq\sigma_x(\gamma)/\gamma$.)
	\end{itemize}
	Then the fundamental class $u_{L/K}\in H^2(\op{Gal}(L/K),L^\times)$ is represented by the triple
	\[(\alpha_0,\alpha_1,\beta_{10})=\left(\gamma,\zeta_{p^f-1}^{-1},\zeta_{p^f-1}^{-1}\right).\]
\end{theorem}
\begin{remark}
	We verify Artin reciprocity for $\QQ_p(\zeta_p)/\QQ_p$. Let $c\in Z^2(\op{Gal}(L/K),L^\times)$ represent the fundamental class. The explicit formula for $\alpha_1$ tells us that
	\[\alpha_1=\prod_{i=0}^{p-1}c\left(\sigma_x^i,\sigma_x\right)=[\sigma_x]\cup\op{Res}u_{L/\QQ_p}=[\sigma_x]\cup u_{L/\QQ_p(\zeta_m)}=\theta^{-1}_{L/\QQ_p(\zeta_m)}(\sigma_x).\]
	Taking norms down to $K^\times$, we see on one hand that
	\[\op N_{\QQ_p(\zeta_m)/\QQ_p}(\alpha_1)=\prod_{i=0}^{f-1}\zeta_{p^f-1}^{-p^i}=\zeta_{p^f-1}^{-\left(1+p+\cdots+p^{f-1}\right)}=\zeta_{p^f-1}^{-\left(p^f-1\right)/(p-1)}=\zeta_{p-1}^{-1}\equiv x^{-1}\pmod p.\]
	On the other hand,
	\[\op N_{\QQ_p(\zeta_m)/\QQ_p}\theta^{-1}_{L/\QQ_p(\zeta_m)}(\sigma_x)=\theta^{-1}_{L/\QQ_p}(\sigma_x)=\theta^{-1}_{\QQ_p(\zeta_p)/\QQ_p}(\sigma_x).\]
	So $\theta^{-1}_{\QQ_p(\zeta_p)/\QQ_p}$ sends $\sigma_x\colon\zeta_p\mapsto\zeta_p^x$ to $x^{-1}\pmod p$, as predicted by Lubin--Tate theory.
\end{remark}

\end{document}