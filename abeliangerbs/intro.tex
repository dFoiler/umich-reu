% !TEX root = ../abeliangerbs.tex

Given a Galois extension of fields of $L/K$ with Galois group $G$ and an algebraic torus $\mathbb T$, a Galois gerb $\mc E$ of $L/K$ bound by $\mathbb T$ is a group extension
\[0\to\mathbb T(L)\to\mc E\to G\to0.\]
Roughly speaking, the goal of the paper is to be able to describe such Galois gerbs---and group extensions in general---explicitly by giving $\mc E$ a group presentation.

More specifically, let $L/K$ be a finite Galois extension of global fields with Galois group $G$. In \cite{kottwitz}, Kottwitz defined three global gerbs $\mc E_1$, $\mc E_2$, and $\mc E_3$. The overall goal of this paper is to be able to provide a somewhat explicit description of the group law for $\mc E_3$ in the toy case of $L=\QQ(\zeta_q)$ and $K=\QQ$ for $q$ a prime-power. Along the way, we will develop various tools which suggest that the methods can be feasibly extended beyond this toy case.

To describe the approach, we quickly recall the definitions of $\mc E_1$, $\mc E_2$, and $\mc E_3$; more details are provided in \autoref{sec:globalsetup}. Let $V_F$ denote the set of places of a global field $F$. We begin with the following short exact sequences.
\begin{align*}
	0 \to \ZZ[V_L]_0 \to \ZZ[V_L] \to \ZZ \to 0 \tag{X} \label{eq:X} \\
	0 \to L^\times \to \AA_L^\times \to \AA_L^\times/L^\times \to 0 \tag{A} \label{eq:A}
\end{align*}
Selecting the global fundamental class $\alpha_1(L/K)=u_{L/K}\in\widehat H^2(G,\AA_L^\times/L^\times)$, we may construct the Galois gerb
\[0\to\op{Hom}_\ZZ(\ZZ,\AA_L^\times/L^\times)\to\mc E_1\to G\to0.\]
By gluing together local fundamental classes, we may construct a class $\alpha_2(L/K)\in\widehat H^2(G,\AA_L^\times)$ defining the Galois gerb
\[0\to\op{Hom}_\ZZ(\ZZ[V_L],\AA_L^\times)\to\mc E_2\to G\to0.\]
Constructing $\mc E_3$ is more difficult. Denote the set of morphisms of short exact sequences from \autoref{eq:X} to \autoref{eq:A} by $\op{Hom}_\ZZ(X,A)$. It turns out that
% https://q.uiver.app/?q=WzAsNCxbMCwwLCJIXjIoRyxcXG9we0hvbX0oWCxBKSkiXSxbMCwxLCJIXjIoRyxcXG9we0hvbX1fXFxaWihcXFpaLFxcQUFfTF5cXHRpbWVzL0xeXFx0aW1lcykpIl0sWzEsMCwiSF4yKEcsXFxvcHtIb219X1xcWlooXFxaWltWX0xdLFxcQUFfTF5cXHRpbWVzKSkiXSxbMSwxLCJIXjIoRyxcXG9we0hvbX1fXFxaWihcXFpaW1ZfTF0sXFxBQV9MXlxcdGltZXMvTF5cXHRpbWVzKSkiXSxbMCwxXSxbMSwzXSxbMCwyXSxbMiwzXV0=&macro_url=https%3A%2F%2Fraw.githubusercontent.com%2FdFoiler%2Fnotes%2Fmaster%2Fnir.tex
\begin{equation}
	\begin{tikzcd}
		{H^2(G,\op{Hom}_\ZZ(X,A))} & {H^2(G,\op{Hom}_\ZZ(\ZZ[V_L],\AA_L^\times))} \\
		{H^2(G,\op{Hom}_\ZZ(\ZZ,\AA_L^\times/L^\times))} & {H^2(G,\op{Hom}_\ZZ(\ZZ[V_L],\AA_L^\times/L^\times))}
		\arrow[from=1-1, to=2-1]
		\arrow[from=2-1, to=2-2]
		\arrow[from=1-1, to=1-2]
		\arrow[from=1-2, to=2-2]
	\end{tikzcd} \label{eq:magicalpullback}
\end{equation}
is a pull-back square, so we can check that we can construct a unique element $\alpha(L/K)\in H^2(G,\op{Hom}(X,A))$ by $\alpha_1(L/K)$ and $\alpha_2(L/K)$. Projecting $\alpha(L/K)$ to $H^2(G,\op{Hom}_\ZZ(\ZZ[V_L]_0,L^\times))$ yields $\alpha_3$ and hence the Galois gerb
\[0\to\op{Hom}_\ZZ(\ZZ[V_L]_0,L^\times)\to\mc E_3\to G\to 0.\]

Most of this definition can be turned directly into a computation. For example, a $2$-cocycle representing $\alpha_1(L/K)=u_{L/K}$ is not too hard to construct, especially in our toy case of $\QQ(\zeta_q)/\QQ$. Continuing, finding a representative for $\alpha_2(L/K)$ is simply a matter of constructing local fundamental classes and then gluing them together appropriately. However, it is harder to make the pull-back square of \autoref{eq:magicalpullback}. In short, this requires choosing a representative for $\alpha_2$ in such a way to appropriately cohere with our choice of representative for $\alpha_1$. This is by far the hardest part of this approach.

\subsection{Overview}
The layout of the paper is as follows. To write down the group law of a group extension of a group $G$ by a $G$-module $A$ requires being able to easily carry around $2$-cocycles in $Z^2(G,A)$. As such, \autoref{sec:crackpot} is interested in studying how one can, in general, encode cocycles. This section is rather pure homological algebra and is largely of separate interest.

The rest of the paper is interested in abelian groups $G$. In \autoref{sec:general}, we describe a natural way to give a group extension of $G$ by a $G$-module $A$ a group law and use this to provide a classification of group extensions. In \autoref{sec:tuplestudy}, we recast this theory in the abstract more machinery established in \autoref{sec:crackpot}.

Having established enough algebra, we turn to executing the above computation. In \autoref{sec:local}, we use the framework provided by \autoref{sec:general} to write down local fundamental classes of abelian extensions. Lastly, \autoref{sec:global} finishes the computation by gluing the local fundamental classes together appropriately to represent $\alpha_2(\QQ(\zeta_p)/\QQ)$ and so also $\alpha_3(\QQ(\zeta_p)/\QQ)$, in our toy case.

\subsection{Acknowledgements}
This research was conducted at the University of Michigan REU during the summer of 2022. The author would especially like to thank his advisors Alexander Bertoloni Meli, Patrick Daniels, and Peter Dillery for their eternal patience and guidance. Without their advice, this project would have been impossible. The author would also like to thank Maxwell Ye for a number of helpful conversations and consistent companionship. Without him, the author would have been left floating adrift and soulless.