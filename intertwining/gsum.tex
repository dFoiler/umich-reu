% !TEX root = ../intertwining.tex

\section{Computation of Matrix Gauss Sums} \label{sec:gsum}
As before, let $\FF_q$ denote the finite field with $q$ elements, where $q$ is an odd prime-power. For characters $\omega\colon\FF_q^\times\to\CC^\times$ and $\psi\colon\FF_q\to\CC$, we are interested in computing sums of the form
\[\sum_A\omega(\det A)\psi(\tr AT),\]
where $A$ and $T$ are possibly subject to certain constraints (e.g., symmetric or alternating). To be explicit, our sums will be done in three cases.
\begin{itemize}
    \item $\GL_n(\FF_q)$.
    \item $\Sym_n^\times(\FF_q)$, the set of invertible $n\times n$ symmetric matrices with coefficients in $\FF_q$.
    \item $\Alt_{2n}^\times(\FF_q)$, the set of invertible $2n\times 2n$ alternating matrices with coefficients in $\FF_q$. (Note that there are no invertible alternating matrices of odd dimension.)
\end{itemize}
For brevity, we will abbreviate $\FF_q$ from our notation as much as possible.

Note that the sum over $A\in\GL_n$ has already been considered by \cite{kim-gauss-sum} and many authors before; see \cite[Section 1]{kim-gauss-sum}. Additionally, the sum over symmetric matrices was considered in \cite{saito-sym-gauss-sum}, but the method there is based on a rather lengthy consideration with the Bruhat decomposition. We are under the impression that the sum over alternating matrices is new.

Our method is rather uniform over all kinds of sums considered. We will induct on the size of $A$ via an explicit row-reduction. As such, the arguments are essentially the same as the spirit of the arguments in \cite{kim-gauss-sum} in the case of $\GL_n$. However, we believe that there is gain to the case of sums of symmetric matrices because the arguments presented are somewhat more direct.

\subsection{Miscellaneous Computations}
We take a moment to discuss a few sums which will be used frequently in the sequel. For characters $\omega\colon\FF_q^\times\to\CC^\times$ and $\psi\colon\FF_q\to\CC^\times$, we denote the usual Gauss sum by
\[g(\omega,\psi)\coloneqq\sum_{a\in\FF_q^\times}\omega(a)\psi(a).\]
It will be helpful to have the following well-known fact about the quadratic Gauss sum. Because the proof is so quick, we include the proof.
\begin{prop} \label{prop:mag-gauss-sum}
    Let $\omega\colon\FF_q^\times\to\CC^\times$ and $\psi\colon\FF_q\to\CC^\times$ denote nontrivial characters. Then
    \[g(\omega,\psi)g\left(\omega^{-1},\psi^{-1}\right)=q.\]
    Thus, if $\chi\colon\FF_q^\times\to\CC^\times$ denotes the nontrivial quadratic character, then $g(\chi,\psi)^2=\chi(-1)q$.
\end{prop}
\begin{proof}
    For the first claim, we want to show
    \[\sum_{a,b\in\FF_q^\times}\omega(a/b)\psi(a-b)\stackrel?=q.\]
    Well, set $c\coloneqq a/b$ so that the sum is
    \[\sum_{c\in\FF_q^\times}\Bigg(\omega(c)\sum_{a\in\FF_q^\times}\psi(a-ac)\Bigg).\]
    If $c\ne1$, then the inner sum is $-\psi(0)+\sum_{a\in\FF_q}\psi(a-ac)=-1$. Otherwise, if $c=1$, then the inner sum is $q-1$. In total, we are left with
    \[(q-1)+\sum_{c\in\FF_q^\times\setminus\{1\}}-\omega(c)=q-\sum_{c\in\FF_q^\times}\omega(c)=q,\]
    which is what we wanted.

    For the second claim, we see
    \[g\left(\chi^{-1},\psi^{-1}\right)=\sum_{a\in\FF_q^\times}\chi(a)\psi(-a)=\chi(-1)\sum_{a\in\FF_q^\times}\chi(a)\psi(a)=\chi(-1)g(\chi,\psi),\]
    so the second claim follows from the first.
\end{proof}
The computation of the Gauss sums over $\Sym_n^\times$ will use the following fact.
\begin{prop} \label{prop:quad-twist-gauss-sum}
    Let $\omega\colon\FF_q^\times\to\CC^\times$ and $\psi\colon\FF_q\to\CC^\times$ be characters, and let $\chi\colon\FF_q^\times\to\CC^\times$ denote the nontrivial quadratic character. Then
    \[\omega(4)g(\omega,\psi)g(\omega\chi,\psi)=g\left(\omega^2,\psi\right)g(\chi,\psi).\]
\end{prop}
\begin{proof}
    Expanding out the Gauss sums, we are trying to show that
    \[\sum_{a,b\in\FF_q^\times}\omega(4ab)\chi(b)\psi(a+b)\stackrel?=\sum_{a,b\in\FF_q^\times}\omega\left(a^2\right)\chi(b)\psi(a+b).\]
    Fixing some $d\in\FF_q^\times$ and $t\in\FF_q$, it is enough to show that
    \begin{equation}
        \sum_{\substack{a+b=t\\4ab=d}}\chi(b) \stackrel?= \sum_{\substack{a+b=t\\a^2=d}}\chi(b) \label{eq:combo-quad-twist-gauss-sum}
    \end{equation}
    and then sum over all possible values of $d$ and $t$. At this point, the proof has become combinatorial number theory. For convenience, extend $\chi$ to $\FF_q$ by $\chi(0)\coloneqq0$, and allow $a,b\in\FF_q$ in the right-hand sum above; this will not change its value.

    For example, suppose that $d$ is not a square. Then the right-hand side of \eqref{eq:combo-quad-twist-gauss-sum} is empty and hence zero. On the other hand, we claim that the left-hand side is zero. Let $(a_1,b_1),\ldots,(a_m,b_m)$ denote the solutions to the system of equations $a+b=t$ and $4ab=d$. Because $d$ is not a square, $a_k\ne b_k$ for each $k$---in fact, if $a_k$ is a square, then $b_k$ is not a square (and vice versa). Thus, if $(a,b)$ is a solution, then $(b,a)$ is a distinct solution with $\{\chi(a),\chi(b)\}=\{1,-1\}$, so the two pairs $(a,b)$ and $(b,a)$ contribute $1-1=0$ to the left-hand side of \eqref{eq:combo-quad-twist-gauss-sum}. It follows that the left-hand side vanishes.

    Thus, in the rest of the proof, we may assume that $d=x^2$ where $x\in\FF_q^\times$, so the right-hand side of \eqref{eq:combo-quad-twist-gauss-sum} reads
    \[\chi(t+x)+\chi(t-x).\]
    To continue, observe that solving the system of equations $a+b=t$ and $4ab=d$ is equivalent to having $a=t-b$ and
    \[(2b-t)^2=t^2-d.\]
    As such, for our next case, suppose that $t^2-d$ fails to be a square. Then the left-hand side of \eqref{eq:combo-quad-twist-gauss-sum} is empty and hence vanishes, so we want to show that the right-hand side also vanishes. Well, $t^2-d=(t+x)(t-x)$ is then not a square, so both are nonzero, and one is a square while the other is not a square. Thus, $\chi(t+x)+\chi(t-x)=0$, as needed.

    Thus, in the rest of the proof, we may assume that $t^2-d=y^2$ for some $y\in\FF_q$. Quickly, we deal with the case where $y=0$. On one hand, we have $t^2=d$, so $t=\pm x$, so the right-hand side of \eqref{eq:combo-quad-twist-gauss-sum} is $\chi(2t)$. On the other hand, we see the left-hand side of \eqref{eq:combo-quad-twist-gauss-sum} is $\chi(t/2)$, so we finish by noting $\chi(2t)=\chi(t/2)$.

    At the current point, we can now say that $t^2=x^2+y^2$ where $x,y\in\FF_q^\times$, and the left-hand side of \eqref{eq:combo-quad-twist-gauss-sum} is $\chi\left(\frac{t+y}2\right)+\chi\left(\frac{t-y}2\right)$, so we are trying to show that
    \begin{equation}
        \chi\left(\frac{t+y}2\right)+\chi\left(\frac{t-y}2\right)\stackrel?=\chi(t+x)+\chi(t-x). \label{eq:last-combo-quad-twist-case}
    \end{equation}
    Because $(t-x)(t+x)=y^2$ and $\left(\frac{t+y}2\right)\left(\frac{t-y}2\right)=\frac14x^2$, we see that all values above are nonzero, and $\chi\left(\frac{t+y}2\right)=\chi\left(\frac{t-y}2\right)$ and $\chi(t+x)=\chi(t-x)$. Because, these values are in $\{\pm1\}$, we see that it is enough to show that $\chi(t+x)=1$ if and only if $\chi\left(\frac{t+y}2\right)=1$.

    The main claim, now, is that $\chi(t+x)=1$ implies that $\chi\left(\frac{t+y}2\right)=1$. This approximately boils down to the enumeration of Pythagorean triples. The above logic grants that $\chi(t+x)=\chi(t-x)=1$, so both $t+x$ and $t-x$ are squares; write $t+x=x_1^2$ and $t-x=x_2^2$ for $x_1,x_2\in\FF_q^\times$. Adjusting signs, we may assume that $y=x_1x_2$. Thus,
    \[\frac{t+y}2=\frac12\left(\frac{x_1^2+x_2^2}2+x_1x_2\right)=\left(\frac{x_1+x_2}2\right)^2\]
    is a square, and we know $\frac{t+y}2$ is nonzero from the above logic, so $\chi\left(\frac{t+y}2\right)=1$, as desired.

    To finish the proof, we must show the reverse implication: we claim that $\chi\left(\frac{t+y}2\right)=1$ implies $\chi(t+x)=1$. Well, we see that $\left(\frac x2\right)^2+\left(\frac y2\right)^2=\left(\frac t2\right)^2$, so the argument of the previous paragraph tells us that $\chi\left(\frac t2+\frac y2\right)=1$ implies
    \[\chi(t+x)=\chi\left(\frac{t+x}4\right)=\chi\left(\frac{\frac t2+\frac x2}2\right)=1,\]
    as desired.
\end{proof}
All of our computations will frequently sum over vectors in some way, so we pick up the following fact.
\begin{lemma} \label{lem:matrix-char-sum}
    Fix a character $\psi\colon\FF_q\to\CC^\times$ and some $A\in\FF_q^{n\times m}$. Then
    \[\sum_{B\in\FF_q^{m\times n}}\psi(\tr AB)=\begin{cases}
        0 & \text{if }A\ne0\text{ and }\psi\ne1, \\
        q^{mn} & \text{if }A=0\text{ or }\psi=1.
    \end{cases}\]
\end{lemma}
\begin{proof}
    Note that
    \[\tr AB=\sum_{j=1}^n(AB)_{jj}=\sum_{i=1}^m\sum_{j=1}^nA_{ji}B_{ij},\]
    so
    \[\sum_{B\in\FF_q^{m\times n}}\psi(\tr AB)=\prod_{i=1}^m\prod_{j=1}^n\sum_{B_{ij}}\psi(A_{ji}B_{ij}).\]
    If $A=0$ or $\psi=1$, then all terms are $1$, so we total to $q^{mn}$. Otherwise, say $A_{ij}\ne0$ for some given $(i,j)$. Then the corresponding term $\sum\psi(A_{ji}B_{ij})$ in the above product will vanish, as desired.
\end{proof}

\subsection{The Sum Over \texorpdfstring{$\GL_n$}{GL}}
For the purposes of this subsection, we define
\[g_n(\omega,\psi,T)\coloneqq\sum_{A\in\GL_n}\omega(\det A)\psi(\tr AT)\]
where $\omega\colon\FF_q^\times\to\CC^\times$ and $\psi\colon\FF_q\to\CC^\times$ are characters, and $T\in\GL_n$. Even though our method to compute $g_n(\omega,\psi,T)$ is essentially equivalent to the one presented in \cite{kim-gauss-sum}, we present it here because it provides a reasonable background to the approach.

The following general results will be helpful.
\begin{lemma} \label{lem:gsum-gl-basic}
    Fix characters $\omega\colon\FF_q^\times\to\CC^\times$ and $\psi\colon\FF_q\to\CC^\times$ and some $T\in\GL_n$.
    \begin{listalph}
        \item For any $g,h\in\GL_n$, we have
        \[g_n(\omega,\psi,gTh)=\omega(\det gh)^{-1}g_n(\omega,\psi,T).\]
        \item If $\psi=1$, then $g_n(\omega,\psi,T)=0$ unless $\omega=1$.
    \end{listalph}
\end{lemma}
\begin{proof}
    Here we go.
    \begin{listalph}
        \item We compute
        \begin{align*}
            g_n(\omega,\psi,gTh) &= \sum_{A\in\GL_n}\omega(\det A)\psi(\tr AgTh) \\
            &= \sum_{A\in\GL_n}\omega\left(\det h^{-1}Ag^{-1}\right)\psi(\tr AT) \\
            &= \omega(\det gh)^{-1}g_n(\omega,\psi,T).
        \end{align*}
        \item With $\psi=1$, we see that $g_n(\omega,\psi,T)=\sum_{A\in\GL_n}\omega(\det A)$ is the sum of a character $\omega$ over a group $\GL_n$, from which the result follows.
        \qedhere
    \end{listalph}
\end{proof}
Our explicit row-reduction is based on two cases: $A_{nn}\ne0$ and $A_{nn}=0$. We begin with the case $A_{nn}=0$ because it is easier.
\begin{lemma} \label{lem:gsum-gl-not-0}
    Fix characters $\omega\colon\FF_q^\times\to\CC^\times$ and $\psi\colon\FF_q\to\CC^\times$. Then
    \[\sum_{\substack{A\in\GL_{n+1}\\A_{n+1,n+1}\ne0}}\omega(\det A)\psi(\tr A)=\begin{cases}
        q^ng(\omega,\psi)g_n(\omega,\psi,1_n) & \text{if }\psi\ne1, \\
        q^{2n}(q-1)g_n(1,1,1_n) & \text{if }\omega=\psi=1.
    \end{cases}.\]
\end{lemma}
\begin{proof}
    The main point is that
    \[\arraycolsep=1.4pt\begin{array}{ccccccccc}
        \GL_n &\times& \FF_q^n &\times& \FF_q^n &\times& \FF_q^\times &\to& \GL_{n+1} \\
        (B &,& v &,& w &,& c) &\mapsto& \begin{bsmallmatrix}
            1_n & v \\
            & 1
        \end{bsmallmatrix}\begin{bsmallmatrix}
            B \\ & c
        \end{bsmallmatrix}\begin{bsmallmatrix}
            1_n \\
            w^\intercal & 1
        \end{bsmallmatrix}
    \end{array}\]
    is a bijection onto elements of $A\in\GL_{n+1}$ with nonzero entry $A_{n+1,n+1}$. Indeed, we can compute
    \[\begin{bmatrix}
        1_n & v \\ & 1
    \end{bmatrix}\begin{bmatrix}
        B \\ & c
    \end{bmatrix}\begin{bmatrix}
        1_n \\ w^\intercal & 1
    \end{bmatrix}=\begin{bmatrix}
        B+cvw^\intercal & cv \\
        cw^\intercal & c
    \end{bmatrix},\]
    so we can see that any $A\in\GL_{n+1}$ with nonzero $c\coloneqq A_{n+1,n+1}$ is uniquely given by some choice $(B,v,w)$. Thus, our total is
    \[\underbrace{\sum_{B\in\GL_n}\omega(\det B)\psi(\tr B)}_{g_n(\omega,\psi,1_n)}\sum_{c,v,w}\omega(c)\psi(c)\psi(\tr cvw^\intercal).\]
    If $(\omega,\psi)=(1,1)$, then our term is always $1$, so we total to $q^{2n}(q-1)g_n(1,1,1_n)$. Otherwise, $\psi=1$, so we can get cancellation by summing over $w$: this produces an internal sum of the form
    \[\sum_{w\in\FF_q^n}\psi(\tr v\cdot cw^\intercal)=\begin{cases}
        0 & \text{if }v\ne0, \\
        q^n & \text{if }v=0,
    \end{cases}\]
    by \Cref{lem:matrix-char-sum}. Thus, our total is
    \[q^ng_n(\omega,\psi,1_n)\underbrace{\sum_{c\in\FF_q^\times}\omega(c)\psi(c)}_{g(\omega,\psi)}\]
    after summing over $c$ as well.
\end{proof}
We now handle $A_{nn}=0$. It is mildly more technical because row-reduction still requires some nonzero term in the right column, so we want to rearrange the right column suitably.
\begin{lemma} \label{lem:gsum-gl-0}
    Fix characters $\omega\colon\FF_q^\times\to\CC^\times$ and $\psi\colon\FF_q\to\CC^\times$. Choosing some nonzero $\ov v,\ov w\in\FF_q^{n+2}$ such that $\ov v_{n+2}=\ov w_{n+2}=0$, we have
    \[\sum_{\substack{A\in\GL_{n+2}\\Ae_{n+2}=\ov v\\A^\intercal e_{n+2}=\ov w}}\omega(\det A)\psi(\tr A)=\begin{cases}
        0 & \text{if }\psi\ne1, \\
        q^{2n+1}g_n(1,1,1_n) & \text{if }(\psi,\omega)=(1,1).
    \end{cases}\]
\end{lemma}
\begin{proof}
    By allowing $T=1_{n+2}$ to be a permutation matrix, we argue that we may assume $\ov v_{n+1},\ov w_{n+1}\ne0$. We want to rearrange the coordinates of $\ov v$ and $\ov w$ so that $\ov v_{n+1},\ov w_{n+1}\ne0$ by mapping $A\mapsto\sigma A\tau$ for suitable permutation matrices $\sigma$ and $\tau$. (One can use $\tau$ to achieve $\ov v_{n+1}\ne0$ and use $\sigma$ to achieve $\ov w_{n+2}\ne0$.) This does not change $\det A$, but it transforms $\tr AT$ into $\tr A\sigma\tau$. For brevity, we rewrite $\sigma\tau$ as $\sigma$. Note the construction promises that $\sigma(n+2)=n+2$.

    The rest of the argument proceeds as before. Write $\ov v=(cv,c,0)$ and $\ov w=(dw,w,0)$ (as a column vector). Then the point is that
    \[\arraycolsep=1.4pt\begin{array}{ccccccccccccc}
        \GL_n &\times& \FF_q^n &\times& \FF_q^n &\times& \FF_q &\to& \GL_{n+1} \\
        (B &,& v' &,& w' &,& e) &\mapsto& \begin{bsmallmatrix}
            1_n & v & v' \\
            & 1 \\ && 1
        \end{bsmallmatrix}\begin{bsmallmatrix}
            B \\ & e & c \\ & d
        \end{bsmallmatrix}\begin{bsmallmatrix}
            1_n \\
            w^\intercal & 1 \\ (w')^\intercal && 1
        \end{bsmallmatrix}
    \end{array}\]
    is a bijection onto $A\in\GL_{n+2}$ satisfying $Ae_{n+2}=\ov v$. Indeed, we compute
    \[\begin{bmatrix}
        1_n & v & v' \\
        & 1 \\ && 1
    \end{bmatrix}\begin{bmatrix}
        B \\ & e & c \\ & d
    \end{bmatrix}\begin{bmatrix}
        1_n \\ w^\intercal & 1 \\
        (w')^\intercal && 1
    \end{bmatrix}=\begin{bmatrix}
        B+cv'w^\intercal+dv(w')^\intercal+ecdvw^\intercal & ev+cv' & cv \\
        ew^\intercal+d(w')^\intercal & e & c \\
        dw^\intercal & d & 0
    \end{bmatrix},\]
    so we can see that any $A\in\GL_{n+2}$ with $Ae_{n+2}=\ov v$ and $A^\intercal e_{n+2}=\ov w$ is uniquely given by some choice of $(B,v',w',e)$. Now, if $\omega=\psi=1$, then we are just counting, so we total to $q^{2n+1}g_n(1,1,1_n)$.

    Otherwise, we take $\psi\ne1$; we have two cases on $\sigma$.
    \begin{itemize}
        \item Suppose that $\sigma(n+1)=n+1$. Then we may write $\sigma$ as $\begin{bsmallmatrix}
            T_n \\ & 1_2
        \end{bsmallmatrix}$. Here, our sum looks like
        \[\sum_{B\in\GL_n}\omega(\det B)\psi(\tr AT_n)\sum_{v',w',d}\omega(-cd)\psi(\tr cv'w^\intercal T_n)\psi(\tr dv(w')^\intercal T_n)\psi(\tr ecdvw^\intercal T_n)\psi(e).\]
        By \Cref{lem:matrix-char-sum}, we see that the sum over $w'$ will only produce nonzero contribution if $v=0$. But in this case, the sum over $e$ is just $\sum\psi(e)=0$, so the total sum vanishes.
        \item Suppose $\sigma(n+1)<n+1$; say $\sigma(i_0)=n+1$. Here, we hold sum over $v'$ while holding all other variables fixed. The determinant does not depend on $v'$, so we are left summing over the $\psi$ terms. Only paying attention to $v'$, we see that we are computing
        \[\sum_{v'\in\FF_q^n}\prod_{i=1}^{n+2}\psi\left(e_i^\intercal \begin{bmatrix}
            cv'w^\intercal & ev+cv' & 0 \\
            0 & 0 & 0 \\
            0 & 0 & 0
        \end{bmatrix}e_{\sigma(i)}\right).\]
        We now hold sum over $v'_{i_0}$ and hold the remaining coordinates $v'_\bullet$ constant. Indeed, the only $i$ in the product which will ever use this coordinate is when $i=i_0$ so that $\sigma(i)=n+1$. This means that the sum over $v'_{i_0}$ contains the factor $\sum\psi(cv'_{i_0})$, which we see vanishes as $v'_{i_0}$ varies over $\FF_q$.
        \qedhere
    \end{itemize}
\end{proof}
We now synthesize our cases to evaluate our Gauss sums.
\begin{theorem} \label{thm:gsum-gl}
    Fix characters $\omega\colon\FF_q^\times\to\CC^\times$ and $\psi\colon\FF_q\to\CC^\times$ and some $T\in\GL_n$.
    \begin{listalph}
        \item Suppose $\psi\ne1$. Then
        \[g_n(\omega,\psi,T)=\frac{q^{n(n-1)/2}}{\omega(\det T)}\cdot g(\omega,\psi)^n.\]
        \item Suppose $\psi=1$ and $\omega=1$. Then
        \[g_n(\omega,\psi,T)=\prod_{i=0}^{n-1}\left(q^n-q^i\right).\]
    \end{listalph}For any $(\psi,\omega)$ not in the above list, the sum vanishes.
\end{theorem}
\begin{proof}
    Note the last sentence follows by \Cref{lem:gsum-gl-basic}. Quickly, we note that both (a) and (b) reduce to the case where $T=1_n$ by \Cref{lem:gsum-gl-basic}: both sides of the equalities in both statements are invariant under replacing $T$ by $gT$ for some $g\in\GL_n$, so taking $g=T^{-1}$ allows us to take $T=1_n$. We now proceed with our cases separately.
    \begin{listalph}
        \item For $n=0$, there is nothing to prove. Thus, by induction, it is enough to show that
        \[g_{n+1}(\omega,\psi,1_{n+1})\stackrel?=q^ng_n(\omega,\psi,T),\]
        which follows by summing \Cref{lem:gsum-gl-not-0,lem:gsum-gl-0}.
        \item For $n\in\{0,1\}$, there is not much to say. We now induct, noting that summing \Cref{lem:gsum-gl-not-0,lem:gsum-gl-0} produces
        \[g_{n+2}(1,1,1_{n+2})=q^{2n+2}(q-1)g_{n+1}(1,1,1_{n+1})+q^{2n+1}\left(q^{n+2}-1\right)^2g_n(1,1,1_n),\]
        so it is enough to show that our right-hand side also satisfies this recurrence relation. Well, the right-hand side of the above equation is
        \begin{align*}
            & q^{2n+2}(q-1)\prod_{i=0}^{n+1}\left(q^{n+1}-q^i\right)+q^{2n+1}\left(q^{n+2}-1\right)^2\prod_{i=0}^n\left(q^n-q^i\right) \\
            ={}& \left(q^{n+1}(q-1)\left(q^{n+2}-q\right)+q\left(q^{n+1}-1\right)^{2}\right)\prod_{i=2}^{n+1}\left(q^{n+2}-q^{i}\right) \\
            ={}& \left(q^{n+1}(q-1)+\left(q^{n+1}-1\right)\right)\prod_{i=1}^{n+1}\left(q^{n+2}-q^{i}\right) \\
            ={}& \prod_{i=0}^{n+1}\left(q^{n+2}-q^{i}\right),
        \end{align*}
        as required.
        \qedhere
    \end{listalph}
\end{proof}
We close this subsection with a combinatorial application; note there is a similar result in \cite[Theorem~6.2]{kim-gauss-sum}.
\begin{corollary}
    Let $n$ be a nonnegative integer, and fix some $T\in\GL_n$. Further, fix $d\in\FF_q^\times$ and $t\in\FF_q$. Then the number of $A\in\GL_n$ such that $\det A=d$ and $\tr AT=t$ is
    \begin{align*}
        &\frac1{q(q-1)}\left(\prod_{i=0}^{n-1}\left(q^n-q^i\right)-q^{n(n-1)/2}(q-1)^n\right) \\
        &+q^{n(n-1)/2}\cdot\#\left\{(y_1,\ldots,y_n):(y_1+\cdots+y_n)=t,\frac{y_1\cdots y_n}{\det T}=d\right\}.
    \end{align*}
\end{corollary}
\begin{proof}
    For any characters $\omega\colon\FF_q^\times\to\CC^\times$ and $\psi\colon\FF_q\to\CC^\times$, we claim that
    \begin{align*}
        g_n(\omega,\psi,T) &\stackrel?= \frac{q^{n(n-1)/2}}{\omega(\det T)}\cdot g(\omega,\psi)^n+\frac1{q(q-1)}\left(\prod_{i=0}^{n-1}\left(q^n-q^i\right)-q^{n(n-1)/2}(q-1)^n\right)\sum_{a\in\FF_q^\times,b\in\FF_q}\omega(a)\psi(b).
    \end{align*}
    If $\psi\ne1$, then this is (a) of \Cref{thm:gsum-gl}; if $\psi=1$, then both sides vanish unless $\omega=1$, in which case this is \Cref{thm:gsum-gl}. Now, we notice that full expansion gives
    \[\frac1{\omega(\det T)}\cdot g(\omega,\psi)^n=\sum_{y_1,\ldots,y_n\in\FF_q^\times}\omega\left(\frac{y_1\cdots y_n}{\det T}\right)\psi(y_1+\cdots+y_n),\]
    so get the result by summing appropriately over all $\omega$ and $\psi$.
\end{proof}

\subsection{The Sum Over \texorpdfstring{$\Sym_n^\times$}{ Sym}}
For the purposes of this subsection, we define
\[g_n(\omega,\psi,T)\coloneqq\sum_{A\in\Sym_n^\times}\omega(\det A)\psi(\tr AT)\]
where $\omega\colon\FF_q^\times\to\CC^\times$ and $\psi\colon\FF_q\to\CC^\times$ are characters, and $T\in\Sym_n^\times$. Additionally, throughout we let $\chi\colon\FF_q^\times\to\CC^\times$ denote the nontrivial quadratic character.

Anyway, we follow the outline of the previous subsection on $\GL_n$.
\begin{lemma} \label{lem:gsum-sym-basic}
    Fix characters $\omega\colon\FF_q^\times\to\CC^\times$ and $\psi\colon\FF_q\to\CC^\times$ and some $T\in\Sym_n^\times$.
    \begin{listalph}
        \item For any $g\in\GL_n$, we have
        \[g_n(\omega,\psi,gTg^\intercal)=\omega(\det g)^{-2}g_n(\omega,\psi,T).\]
        \item If $\psi=1$, then $g_n(\omega,\psi,T)=0$ unless $\omega^2=1$ and $n$ is even.
    \end{listalph}
\end{lemma}
\begin{proof}
    Here we go.
    \begin{listalph}
        \item We directly compute
        \begin{align*}
            g_n(\omega,\psi,gTg^\intercal) &= \sum_{A\in\Sym_n^\times}\omega(\det A)\psi(\tr AgTg^\intercal) \\
            &= \sum_{A\in\Sym_n^\times}\omega(\det A)\psi(\tr g^\intercal AgT) \\
            &= \sum_{A\in\Sym_n^\times}\omega(\det g^{-\intercal}Ag^{-1})\psi(\tr AT) \\
            &= \omega(\det g)^{-2}g_n(\omega,\psi,T).
        \end{align*}
        \item We have two cases.
        \begin{itemize}
            \item Suppose $\omega^2\ne1$. Then for any $g\in\GL_n$, we see that $A\in\Sym_n^\times$ if and only if $gAg^\intercal\in\Sym_n^\times$, so
            \[g_n(\omega,1,T)=\sum_{A\in\Sym_n^\times}\omega(\det A)=\sum_{A\in\Sym_n^\times}\omega(\det gAg^\intercal)=\omega(\det g)^2g_n(\omega,1,T).\]
            Thus, to conclude $g_n(\omega,1,T)=0$, it suffices to find $g\in\GL_n$ with $\omega(\det g)^2\ne1$. Well, $\omega^2\ne1$, so find $c\in\FF_q^\times$ such that $\omega(c)^2\ne1$ and then set $g\coloneqq\operatorname{diag}(c,1,\ldots,1)$.
            \item Suppose $n$ is odd. By the previous case, we may assume that $\omega^2=1$. Now, for any $c\in\FF_q^\times$, we see that $A\in\Sym_n^\times$ if and only if $cA\in\Sym_n^\times$, so
            \[g_n(\omega,1,T)=\sum_{A\in\Sym_n^\times}\omega(\det A)=\sum_{A\in\Sym_n^\times}\omega(c\det A)=\omega(c)^ng_n(\omega,1,T).\]
            Now, if we did have $g_n(\omega,1,T)\ne0$, then we would have $\omega(c)^n=1$ for all $c\in\FF_q^\times$ and hence $\omega^n=1$; however, $n$ is odd and $\omega^2=1$ already, so it would follow $\omega=1$. However, $\omega\ne1$ by hypothesis.
            \qedhere
        \end{itemize}
    \end{listalph}
\end{proof}
We now handle $A_{nn}\ne0$.
\begin{lemma} \label{lem:gsum-sym-ind-not-0}
    Fix characters $\omega\colon\FF_q^\times\to\CC^\times$ and $\psi\colon\FF_q\to\CC^\times$ and diagonal $T_{n+1}\in\Sym_{n+1}^\times$. Letting $T_n\coloneqq\op{diag}(T_{11},\ldots,T_{nn})$, we have
    \[\sum_{\substack{A\in\Sym_{n+1}^\times\\A_{n+1,n+1}\ne0}}\omega(\det A)\psi(\tr AT_{n+1}) = \begin{cases}
        g_n(\omega,\psi,T_n)\frac{\chi(\det T_n)\chi(T_{n+1,n+1})^n}{\omega(T_{n+1,n+1})}g(\omega\chi^n,\psi)g(\chi,\psi)^n & \text{if }\psi\ne1, \\
        0 & \text{if }(\omega,\psi)=(\chi,1), \\
        q^n(q-1)g_n(1,1,1_n) & \text{if }(\omega,\psi)=1.
    \end{cases}\]
\end{lemma}
\begin{proof}
    The main point is that
    \[\arraycolsep=1.4pt\begin{array}{ccccccc}
        \operatorname{Sym}_n^\times & \times & \FF_q^n & \times & \FF_q^\times & \to & \Sym_{n+1}^\times \\
        (B & , & v & , & c) & \mapsto & \displaystyle\begin{bsmallmatrix}
            1 & v \\
              & 1
        \end{bsmallmatrix}\begin{bsmallmatrix}
            B \\ & c
        \end{bsmallmatrix}\begin{bsmallmatrix}
            1 \\ v^\intercal & 1
        \end{bsmallmatrix}
    \end{array}\]
    is a bijection onto $A\in\Sym_{n+1}^\times$ with $A_{n+1,n+1}\ne0$. Indeed, we can compute
    \[\begin{bmatrix}
        1 & v \\
          & 1
    \end{bmatrix}\begin{bmatrix}
        B \\ & c
    \end{bmatrix}\begin{bmatrix}
        1 \\ v^\intercal & 1
    \end{bmatrix}=\begin{bmatrix}
        B+cvv^\intercal & cv \\
        cv^\intercal & c
    \end{bmatrix},\]
    from which the bijection follows. Thus, our sum in question is
    \[\underbrace{\sum_{B\in\Sym_n^\times}\omega(\det B)\psi(\tr BT_n)}_{g_n(\omega,\psi,T_n)}\sum_{v,c}\omega(c)\psi(\tr cvv^\intercal T_n)\psi(cT_{n+1,n+1}).\]
    If $\psi=1$ and $\omega=\chi$, then the right sum is $\sum_{v,c}\chi(c)=0$ because $\chi$ is nontrivial; if $\psi=1$ and $\omega=1$, then the right sum is $q^n(q-1)$ because all terms equal $1$.

    For the rest of the proof, we may assume that $\psi\ne1$. For brevity, we set $T\coloneqq\op{diag}(d_1,\ldots,d_{n+1})$. The main point is to compute our sum over $v$ and $c$, which we see is
    \[\sum_{c\in\FF_q^\times}\omega(c)\psi(cd_{n+1})\prod_{i=1}^n\Bigg(\sum_{a\in\FF_q}\psi\left(cd_ia^2\right)\Bigg)\]
    after some expansion (of $v\in\FF_q^n$). Quickly, we claim that
    \[\sum_{a\in\FF_q}\psi\left(cd_ka^2\right)\stackrel?=\sum_{a\in\FF_q}(1+\chi(cd_ka))\psi(a),\]
    where we have extended $\chi$ to $\FF_q$ by $\chi(0)\coloneqq0$. Indeed, for any $b\in\FF_q$, we see that $\psi(b)$ appears on the left-hand side $0$ times if $b$ does not have the form $cd_ka^2$, appears $1$ time if $b=0$, and appears $2$ times if $b$ is nonzero and has the form $cd_ka^2$; these values are exactly $1+\chi(cd_ka)$ in all cases. As such, the claim follows, and because $\psi$ is nontrivial, we actually have
    \[\sum_{a\in\FF_q}\psi\left(cd_ka^2\right)=\sum_{a\in\FF_q}\chi(cd_ka)\psi(a)=\chi(cd_k)g(\chi,\psi).\]
    Plugging this in, we see that our sum is
    \begin{align*}
        &\phantom{{}={}} g_n(\omega,\psi,T_n)\sum_{c\in\FF_q^\times}\omega(c)\chi(c)^n\psi(cd_{n+1})\chi(d_1\cdots d_n)g(\chi,\psi)^n \\
        &= g_n(\omega,\psi,T_n)\cdot\frac{\chi(d_1\cdots d_n)\chi(d_{n+1})^n}{\omega(d_{n+1})}\sum_{c\in\FF_q^\times}\omega(c)\chi(c)^n\psi(c)g(\chi,\psi)^n \\
        &= g_n(\omega,\psi,T_n)\cdot\frac{\chi(d_1\cdots d_n)\chi(d_{n+1})^n}{\omega(d_{n+1})}\cdot g\left(\omega\chi^n,\psi\right)g(\chi,\psi)^n,
    \end{align*}
    as desired.
\end{proof}
Next, we handle $A_{nn}=0$.
\begin{lemma} \label{lem:gsum-sym-ind-0}
    Fix characters $\omega\colon\FF_q^\times\to\CC^\times$ and $\psi\colon\FF_q\to\CC^\times$ and some diagonal $T_{n+2}\in\Sym^\times_{n+2}$, and let $T_n\coloneqq\op{diag}(T_{11},\ldots,T_{nn})$. Choosing some nonzero $\ov v\in\FF_q^{n+2}$ such that $\ov v_{n+2}=0$, we have
    \[\sum_{\substack{A\in\Sym_{n+2}^\times\\Ae_{n+2}=\ov v}}\omega(\det A)\psi(\tr AT)=\begin{cases}
        0 & \text{if }\psi\ne1, \\
        \omega(-1)q^{n+1}g_n(\omega,1,T_n) & \text{if }\psi=1\text{ and }\omega^2=1.
    \end{cases}\]
\end{lemma}
\begin{proof}
    We begin by arguing that we may assume $\ov v_{n+1}\ne0$. We want to rearrange the coordinates of $\ov v$ so that $v_{n+1}\ne0$ by mapping $A\mapsto\sigma^{-1}A\sigma$ for suitable permutation matrix $\sigma$. This does not change $\det A$, but it transforms $\tr AT$ into $\tr A\sigma T\sigma^{-1}$, effectively rearranging the rows and columns of $T$ into a different diagonal matrix. However, the conclusion is independent of $T$, so this rearrangement is legal.

    The rest of the argument proceeds as before. Write $\ov v=(cv,c,0)$ (as a column vector). Now, once again, the main point is that there is a bijection
    \[\arraycolsep=1.4pt\begin{array}{ccccccc}
        \operatorname{Sym}_{n}^\times & \times & \FF_q^{n} & \times & \FF_q & \to & \Sym_{n+2}^\times \\
        (B & , & w & , & d) & \mapsto &  \begin{bsmallmatrix}
            1 & v & w \\
                & 1 &   \\
                &   & 1
        \end{bsmallmatrix}\begin{bsmallmatrix}
            B \\
                & d & c \\
                & c
        \end{bsmallmatrix}\begin{bsmallmatrix}
            1 \\
            v^\intercal & 1 \\
            w^\intercal &   & 1
        \end{bsmallmatrix}
    \end{array}\]
    onto the set of $A\in\Sym_{n+2}^\times$ such that $Ae_{n+2}=\ov v$. To see that this is a bijection, we expand out the matrix product as
    \[\begin{bmatrix}
        B+dvv^\intercal+c(vw^\intercal+wv^\intercal) & cdv+cw & cv \\
        cdv^\intercal+cw^\intercal & d & c \\
        cv^\intercal & c & 0
    \end{bmatrix},\]
    from which the bijection follows. Thus, we see that our sum is
    \[\sum_{B\in\operatorname{Sym}_{n}^\times}\omega\left(-c^2\det B\right)\psi(\tr BT_n)\Bigg(\sum_{d\in\FF_q}\psi\left(dT_{nn}+d\tr vv^\intercal T_n\right)\sum_{w\in\FF_q^{n}}\psi(2c\tr vw^\intercal T_n)\Bigg).\]
    If $\psi\ne1$, then we see that the sum over $w$ vanishes by \Cref{lem:matrix-char-sum} unless $v=0$. But in the case where $v=0$, we see that the sum over $d$ will vanish, so the total sum continues to vanish. Otherwise, $\psi=1$ with $\omega^2=1$. Then our sums over $d$ and $w$ have all terms equal to $1$, so they produce contributions of $\omega(-1)q^{n+1}g_n(\omega,1,T_n)$, as required.
\end{proof}
We now synthesize our cases to evaluate our Gauss sums.
\begin{theorem} \label{thm:gsum-sym}
    Fix characters $\omega\colon\FF_q^\times\to\CC^\times$ and $\psi\colon\FF_q\to\CC^\times$ and some $T\in\Sym_n^\times$. Let $\chi\colon\FF_q^\times\to\CC^\times$ denote the quadratic character.
    \begin{listalph}
        \item Suppose $\psi\ne1$.
        \begin{itemize}
            \item If $n=2m$ is an even nonnegative integer, then
            \[g_{2m}(\omega,\psi,T)=\frac{\chi(-1)^m\chi(\det T)q^{m^2}}{\omega(4^m\det T)}\cdot g\left(\omega^2,\psi\right)^m.\]
            \item If $n=2m+1$ is an odd nonnegative integer, then
            \[g_{2m+1}(\omega,\psi,T)=\frac{q^{m(m+1)}}{\omega(4^m\det T)}\cdot g(\omega,\psi)g\left(\omega^2,\psi\right)^m.\]
        \end{itemize}
        \item Suppose $\psi=1$ and $\omega=\chi$. If $n=2m$ is even, then
        \[g_{2m}(\chi,1,T)=\chi(-1)^mq^{m^2}\prod_{k=0}^{m-1}\left(q^{2k+1}-1\right).\]
        \item Suppose $\psi=1$ and $\omega=1$.
        \begin{itemize}
            \item If $n=2m$ is even, then
            \[g_{2m}(1,1,T)=q^{m^2+m}\prod_{k=0}^{m-1}\left(q^{2k+1}-1\right).\]
            \item If $n=2m+1$ is odd, then
            \[g_{2m+1}(1,1,T)=q^{m^2+m}\prod_{k=0}^{m}\left(q^{2k+1}-1\right).\]
        \end{itemize}
    \end{listalph}
    For any $(\psi,\omega)$ not in the above list, the sum vanishes.
\end{theorem}
\begin{proof}
    Note the last sentence follows by \Cref{lem:gsum-sym-basic}. Anyway, we quickly reduce to the case where $T$ is diagonal. Indeed, by choosing an orthogonal basis for the symmetric bilinear form given by $T$, we receive some $g\in\GL_n$ such that $D\coloneqq gTg^\intercal$ is diagonal. As such, \Cref{lem:gsum-sym-basic} yields
    \[g_n(\omega,\psi,T) = \omega(\det g)^2g_n(\omega,\psi,D).\]
    Now, suppose we have proven the theorem for diagonal matrices. In this case, we see $g_n(\omega,\psi,D)=(\det D)^{-1}g_n(\omega,\psi,1)$, so $\det D=(\det g)^2(\det T)$ implies that
    \[g_n(\omega,\psi,T)=(\det T)^{-1}g_n(\omega,\psi,1),\]
    which is the theorem for $T$, as desired.

    Thus, we may assume that $T_n\coloneqq\op{diag}(d_1,\ldots,d_n)$; set $T_{n-1}\coloneqq\op{diag}(d_1,\ldots,d_{n-1})$ and define $T_{n-2}$ analogously. We now induct on $n$ in cases.
    \begin{listalph}
        \item We split our arguments by parity.
        \begin{itemize}
            \item Suppose that $n=2m$ is an even positive integer. In this case, \Cref{lem:gsum-sym-ind-not-0,lem:gsum-sym-ind-0} and induction yields
            \begin{align*}
                g_{2m}(\omega,\psi,T) &= g_{2m-1}(\omega,\psi,T_{2m-1})\cdot\frac{\chi(\det T)}{\omega(d_{n+1})}\cdot g(\omega\chi,\psi)g(\chi,\psi)^{2m-1} \\
                &= \frac{\chi(\det T)q^{(m-1)m}}{\omega(4^{m-1}\det T)}\cdot g(\omega,\psi)g\left(\omega^2,\psi\right)^{m-1}g(\omega\chi,\psi)g(\chi,\psi)^{2m-1}.
            \end{align*}
            By \Cref{prop:quad-twist-gauss-sum}, this is
            \[g_{2m}(\omega,\psi,T)=\frac{\chi(\det T)q^{m^2-m}}{\omega(4^m\det T)}\cdot g\left(\omega^2,\psi\right)^mg(\chi,\psi)^{2m}.\]
            Lastly, \Cref{prop:mag-gauss-sum} yields
            \[g_{2m}(\omega,\psi)=\frac{\chi(-1)^m\chi(\det T)q^{m^2}}{\omega(4^m\det T)}\cdot g\left(\omega^2,\psi\right)^m.\]
            \item Suppose $n=2m+1$ is an odd positive integer with $m\ge1$. In this case, \Cref{lem:gsum-sym-ind-not-0,lem:gsum-sym-ind-0} and induction yields
            \begin{align*}
                g_{2m+1}(\omega,\psi) &= g_{2m}(\omega,\psi)g(\omega,\psi)\cdot\frac{\chi(\det T_{2m-1})}{\omega(d_{n+1})}\cdot g(\chi,\psi)^{2m} \\
                &= \frac{\chi(-1)^mq^{m^2}}{\omega(4^m\det T)}\cdot g\left(\omega^2,\psi\right)^mg(\omega,\psi)g(\chi,\psi)^{2m}.
            \end{align*}
            From here, \Cref{prop:mag-gauss-sum} implies
            \[g_{2m+1}(\omega,\psi)=\frac{q^{m^2+m}}{\omega(4^m\det T)}\cdot g(\omega,\psi)g\left(\omega^2,\psi\right)^m.\]
        \end{itemize}
        \item Because $\psi=1$, the value of $T$ is irrelevant, so we will not adjust it in our notation. By induction, it is enough to show
        \[g_{2m+2}(\chi,1,T)\stackrel?=g_{2m}(\chi,1,T)\cdot\chi(-1)q^{2m+1}\left(q^{2m+1}-1\right),\]
        for all $m$, which is immediate upon summing \Cref{lem:gsum-sym-ind-not-0,lem:gsum-sym-ind-0}.
        \item Again, because $\psi=1$, the value of $T$ is irrelevant, so we will not adjust it in our notation. As with previous parts, we induct. For $n\in\{0,1\}$, there is not much to say. The main point is that
        \[g_{n+2}(1,1,T)=q^{n+1}(q-1)g_{n+1}(1,1,T_{n+1})+q^{n+1}\left(q^{n+1}-1\right)g_n(1,1,T_n)\]
        for any $n$, which we see follows from summing \Cref{lem:gsum-sym-ind-not-0,lem:gsum-sym-ind-0}. Now, to synthesize cases, we note that
        \[q^{m^2+m}\prod_{k=0}^{m}\left(q^{2k+1}-1\right)=q^{\frac12(2m+1)(2m+2)}\prod_{\substack{1\le k\le2m+1\\k\text{ odd}}}\left(1-\frac1{q^k}\right)\]
        and analogously for the even case. Thus, for our induction, we take $n\ge0$ and use the recursive formula to see $g_{n+2}(1,1,T)$ is
        \begin{align*}
            & q^{n+1}(q-1)g_{n+1}(1,1,T)+q^{n+1}\left(q^{n+1}-1\right)g_n(1,1,T) \\
            ={}& q^{\frac12(n+2)(n+1)}\left(q^{n+1}(q-1)\prod_{\substack{n<k\le n+1\\k\text{ odd}}}\left(1-\frac1{q^k}\right)+\left(q^{n+1}-1\right)\right)\prod_{\substack{1\le k\le n\\k\text{ odd}}}\left(1-\frac1{q^k}\right).
        \end{align*}
        If $n$ is odd, we have
        \[q^{\frac12(n+2)(n+1)}\left(q^{n+2}-1\right)\prod_{\substack{1\le k\le n\\k\text{ odd}}}\left(1-\frac1{q^k}\right),\]
        which simplifies correctly. If $n$ is even, we have
        \[q^{\frac12(n+2)(n+1)}\underbrace{\left(q^{n+1}(q-1)\left(1-\frac1{q^{n+1}}\right)+\left(q^{n+1}-1\right)\right)}_{q\left(q^{n+1}-1\right)}\prod_{\substack{1\le k\le n\\k\text{ odd}}}\left(1-\frac1{q^k}\right),\]
        which still simplifies correctly. This completes the induction.
        \qedhere
    \end{listalph}
\end{proof}
\begin{remark}
    Part (c) above has recovered the computation of the number of invertible symmetric matrices; part (b) even recovers the number of invertible symmetric matrices with square and non-square determinant. We have included an inductive proof above to show that our methods are capable of this.
\end{remark}
% \begin{remark}
%     The theorem implies that
%     \[\sum_{A\in\Sym_n^\times(\FF_q)}\omega\left(\det AT^{-1}\right)\psi(\tr A)=\sum_{B\in\Sym_n^\times(\FF_q)}\omega(\det B)\psi(\tr BT),\]
%     but this is not obvious: in particular, one cannot apply the variable change $B\coloneqq AT^{-1}$ because $AT^{-1}$ need not be symmetric! We would be interested in a more direct proof of the above equality.
% \end{remark}
We conclude this subsection with a combinatorial application.
\begin{corollary} \label{cor:count-sym}
    Let $n$ be a nonnegative integer, and fix some $T\in\Sym_n^\times$. Further, fix $d\in\FF_q^\times$ and $t\in\FF_q$.
    \begin{enumerate}[label=(\alph*)]
        \item Suppose that $n=2m+1$ is odd. Then the number of $A\in\Sym_{2m+1}^\times$ such that $\det A=d$ and $\tr AT=t$ is
        \begin{align*}
            &\frac{q^{m^2+m}}{q(q-1)}\Bigg(\prod_{k=0}^m\left(q^{2k+1}-1\right)-(q-1)^{m+1}\Bigg) \\
            &+q^{m^2+m}\cdot\#\left\{(x,y_1,\ldots,y_m):x+(y_1+\cdots+y_m)=t,\frac{x(y_1\cdots y_m)^2}{4^m\det T}=d\right\}.
        \end{align*}
        \item Suppose that $n=2m$ is even. Let $\chi\colon\FF_q^\times\to\CC^\times$ denote the nontrivial quadratic character. Then the number of $A\in\Sym_{2m}^\times$ such that $\det A=d$ and $\tr AT=t$ is
        \begin{align*}
            &\frac{q^{m^2}}{q(q-1)}\Bigg(\left(q^m+\chi(-1)^m\chi(d)\right)\prod_{k=0}^{m-1}\left(q^{2k+1}-1\right)-\chi(-1)^m\left(\chi(d)+\chi(\det T)\right)(q-1)^m\Bigg) \\
            &+\chi(-1)^m\chi(\det T)q^{m^2}\cdot\#\left\{(y_1,\ldots,y_m):y_1+\cdots+y_m=t,\frac{(y_1\cdots y_m)^2}{4^m\det T}=d\right\}.
        \end{align*}
    \end{enumerate}
\end{corollary}
\begin{proof}
    We prove these separately.
    \begin{enumerate}[label=(\alph*)]
        \item For any characters $\omega\colon\FF_q^\times\to\CC^\times$ and $\psi\colon\FF_q\to\CC^\times$, we claim that
        \begin{align*}
            g_n(\omega,\psi,T) &\stackrel?= \frac{q^{m(m+1)}}{\omega\left(4^m\det T\right)}\cdot g(\omega,\psi)g\left(\omega^2,\psi\right)^m \\
            &\quad+\frac{g_n(1,1,T)-q^{m(m+1)}(q-1)^{m+1}}{q(q-1)}\sum_{a\in\FF_q^\times,b\in\FF_q}\omega(a)\psi(b).
        \end{align*}
        This is by casework. If $\psi$ is nontrivial, the second sum on the right-hand side vanishes, so the claim follows from \Cref{thm:gsum-sym}. If $\psi$ is trivial and $\omega$ is nontrivial, then the right-hand side vanishes, and left-hand side vanishes by \Cref{thm:gsum-sym}. Lastly, if both $\psi$ and $\omega$ are trivial, then both sides are $g_n(1,1,T)$.

        Now, we notice that full expansion gives
        \[\frac1{\omega\left(4^m\det T\right)}\cdot g(\omega,\psi)g\left(\omega^2,\psi\right)^m=\sum_{x,y_1,\ldots,y_m\in\FF_q^\times}\omega\left(\frac{x(y_1\cdots y_m)^2}{4\det T}\right)\psi(x+(y_1+\cdots+y_m)),\]
        so by summing appropriately over all $\omega$ and $\psi$, we see that the number of $A\in\Sym_n^\times$ such that $\det A=d$ and $\tr AT=t$ is
        \begin{align*}
            &\frac{g_n(1,1,T)-q^{m(m+1)}(q-1)^{m+1}}{q(q+1)} \\
            &+q^{m^2+m}\cdot\#\left\{(x,y_1,\ldots,y_m):x+(y_1+\cdots+y_m)=t,\frac{x(y_1\cdots y_m)^2}{4^m\det T}=d\right\}.
        \end{align*}
        To finish, we note that we can simplify the first term with from \Cref{thm:gsum-sym}.

        \item For any characters $\omega\colon\FF_q^\times\to\CC^\times$ and $\psi\colon\FF_q\to\CC^\times$, we claim that
        \begin{align*}
            g_n(\omega,\psi,T) &\stackrel?= \frac{\chi(-1)^m\chi(\det T)q^{m^2}}{\omega\left(4^m\det T\right)}\cdot g\left(\omega^2,\psi\right)^m \\
            &\quad+\frac{g_n(\chi,1,T)-\chi(-1)^mq^{m^2}(q-1)^m}{q(q-1)}\sum_{a\in\FF_q^\times,b\in\FF_q}\chi(a)\omega(a)\psi(b) \\
            &\quad+\frac{g_n(1,1,T)-\chi(-1)^m\chi(\det T)q^{m^2}(q-1)^m}{q(q-1)}\sum_{a\in\FF_q^\times,b\in\FF_q}\omega(a)\psi(b).
        \end{align*}
        Again, this is by casework. If $\psi$ is nontrivial, this is \Cref{thm:gsum-sym}; otherwise, $\psi$ is trivial. Then if $\omega^2\ne1$ (i.e., $\omega\notin\{1,\chi\}$) the right-hand side vanishes, and the left-hand side vanishes by \Cref{thm:gsum-sym}. Lastly, if $\omega\in\{1,\chi\}$, then both sides are equal by construction.

        Now, as in (a), by full expansion of $\omega\left(4^m\det T\right)^{-1}g\left(\omega^2,\psi\right)^m$ and summing the claim over all $\omega$ and $\psi$ appropriately, we see that the number of $A\in\Sym_n^\times$ such that $\det A=d$ and $\tr AT=t$ is
        \begin{align*}
            &\frac{g_n(\chi,1,T)-\chi(-1)^mq^{m^2}(q-1)^m}{q(q-1)}\cdot\chi(d)+\frac{g_n(1,1,T)-\chi(-1)^m\chi(\det T)q^{m^2}(q-1)^m}{q(q-1)}\\
            &+\chi(-1)^m\chi(\det T)q^{m^2}\cdot\#\left\{(y_1,\ldots,y_m):y_1+\cdots+y_m=t,\frac{(y_1\cdots y_m)^2}{4^m\det T}=d\right\}.
        \end{align*}
        It remains to simplify the first two terms. On one hand, we note \Cref{thm:gsum-sym} gives
        \[\frac{g_n(\chi,1,T)-\chi(-1)^mq^{m^2}(q-1)^m}{q(q-1)}\cdot\chi(d)=\frac{\chi(-1)^mq^{m^2}}{q(q-1)}\left(\prod_{k=0}^{m-1}\left(q^{2k+1}-1\right)-(q-1)^m\right)\chi(d).\]
        On the other hand, \Cref{thm:gsum-sym} gives
        \[\frac{g_n(1,1,T)-\chi(-1)^m\chi(\det T)q^{m^2}(q-1)^m}{q(q-1)}=\frac{q^{m^2}}{q(q-1)}\left(q^m\prod_{k=0}^{m-1}\left(q^{2k+1}-1\right)-\chi(-1)^m\chi(\det T)(q-1)^m\right).\]
        Summing the above two equalities completes the simplification.
        \qedhere
    \end{enumerate}
\end{proof}

\subsection{The Sum Over \texorpdfstring{$\Alt_{2n}^\times$}{ Alt}}
For the purposes of this subsection, we define
\[g_{2n}(\omega,\psi,T)\coloneqq\sum_{A\in\Alt_{2n}^\times}\omega(\det A)\psi(\tr AT)\]
where $\omega\colon\FF_q^\times$ and $\psi\colon\FF_q\to\CC^\times$ are characters, and $T\in\Alt_{2n}^\times$. Additionally, throughout we let $J\coloneqq\begin{bsmallmatrix}
    & -1 \\ 1
\end{bsmallmatrix}$.

We would like to follow the outline established for $\GL_n$, but we can immediately tell that something will be different here because $A\in\Alt_{2n}^\times$ will automatically have $A_{2n,2n}=0$. Instead, our row-reduction will be based on subdividing $A$ into $2\times2$ minors. As such, our casework is based on $A_{2n,2n-1}=-A_{2n-1,2n}$.

Otherwise, our outline remains the same.
\begin{lemma} \label{lem:gsum-alt-basic}
    Fix characters $\omega\colon\FF_q^\times$ and $\psi\colon\FF_q\to\CC^\times$ and some $T\in\Alt_{2n}^\times$.
    \begin{listalph}
        \item For any $g\in\GL_n$, we have
        \[g_{2n}(\omega,\psi,gTg^\intercal)=\omega(\det g)^{-2}g_{2n}(\omega,\psi,T).\]
        \item If $\psi=1$, then $g_{2n}(\omega,\psi,T)=0$ unless $\omega^2=1$.
    \end{listalph}
\end{lemma}
\begin{proof}
    Here we go.
    \begin{listalph}
        \item We directly compute
        \begin{align*}
            g_{2n}(\omega,\psi,gTg^\intercal) &= \sum_{A\in\Alt_{2n}^\times}\omega(\det A)\psi(\tr AgTg^\intercal) \\
            &= \sum_{A\in\Alt_{2n}^\times}\omega(\det A)\psi(\tr g^\intercal AgT) \\
            &= \sum_{A\in\Alt_{2n}^\times}\omega(\det g^{-\intercal}Ag^{-1})\psi(\tr AT) \\
            &= \omega(\det g)^{-2}g_{2n}(\omega,\psi,T).
        \end{align*}
        \item For any $g\in\GL_n$, we see that $A\in\Alt_{2n}^\times$ if and only if $gAg^\intercal\in\Alt_{2n}^\times$, so
        \[g_{2n}(\omega,1,T)=\sum_{A\in\Alt_{2n}^\times}\omega(\det A)=\sum_{A\in\Alt_{2n}^\times}\omega(\det gAg^\intercal)=\omega(\det g)^2g_{2n}(\omega,1,T).\]
        Thus, to conclude $g_{2n}(\omega,1,T)=0$, it suffices to find $g\in\GL_n$ with $\omega(\det g)^2\ne1$. Well, $\omega^2\ne1$, so find $c\in\FF_q^\times$ such that $\omega(c)^2\ne1$ and then set $g\coloneqq\operatorname{diag}(cJ,J,\ldots,J)$.
        \qedhere
    \end{listalph}
\end{proof}
We now handle $A_{2n,2n-1}\ne0$.
\begin{lemma} \label{lem:gsum-alt-not-0}
    Fix characters $\omega\colon\FF_q^\times\to\CC^\times$ and $\psi\colon\FF_q\to\CC^\times$, and set $T_{2i}\coloneqq\op{diag}(J,\ldots,J)\in\Alt^\times_{2i}$ for each $i$. Then
    \[\sum_{\substack{A\in\Alt_{2n+2}^\times\\A_{2n,2n-1}\ne0}}\omega(\det A)\psi(\tr AT_{2n+2})=\begin{cases}
        q^{2n}g(\omega^2,\psi)g_{2n}(\omega,\psi,T_{2n}) & \text{if }\psi\ne1, \\
        q^{4n}(q-1)g_{2n}(1,1,T_n) & \text{if }\psi=1\text{ and }\omega^2=1.
    \end{cases}\]
\end{lemma}
\begin{proof}
    The point is that the bottom-right $2\times2$ minor of our considered $A\in\Alt_{2n+2}^\times$ is invertible. Thus, the main point is that
    \[\arraycolsep=1.4pt\begin{array}{ccccccccccc}
        \Alt_{2n}^\times &\times& \FF_q^{2n\times2} &\times& \FF_q^\times &\to& \Alt_{2n+2}^\times \\
        (B &,& V &,& c) &\mapsto& \begin{bsmallmatrix}
            1_{2n} & V \\ & 1_2
        \end{bsmallmatrix}\begin{bsmallmatrix}
            B \\ & cJ
        \end{bsmallmatrix}\begin{bsmallmatrix}
            1_{2n} \\ V^\intercal & 1_2
        \end{bsmallmatrix}
    \end{array}\]
    is a bijection onto $A\in\Alt_{2n+2}^\times$ with $A_{2n,2n-1}\ne0$. Indeed, letting $V=\begin{bsmallmatrix}
        v & w
    \end{bsmallmatrix}$, we can expand
    \[\begin{bmatrix}
        1_{2n} & v & w \\
        & 1 \\ && 1
    \end{bmatrix}\begin{bmatrix}
        B \\ && -c \\ & c
    \end{bmatrix}\begin{bmatrix}
        1_{2n} \\ v^\intercal & 1 \\ w^\intercal && 1
    \end{bmatrix}=\begin{bmatrix}
        B+cwv^\intercal-cvw^\intercal & cw & -cv \\
        -cw^\intercal && -c \\
        cv^\intercal & c
    \end{bmatrix},\]
    from which the bijection follows: $A\in\Alt_{2n+2}^\times$ with nonzero $c\coloneqq A_{2n,2n-1}$ is uniquely determined by a choice of $(B,v,w)$. Thus, the sum in question is
    \[\sum_{B,v,w,c}\omega(\det B)\omega\left(c^2\right)\psi(\tr BT_{2n})\psi(\tr(cwv^\intercal-cvw^\intercal)T_{2n})\psi(\underbrace{\tr cJJ}_{-2c}).\]
    If $\psi=1$ and $\omega^2=1$, then every term in the sum equals $1$, so we total to $q^{2n}(q-1)g_{2n}(1,1,T_{2n})$. Otherwise, we take $\psi\ne1$. In this case, we quickly compute $-\tr vw^\intercal T_{2n}=-\tr T_{2n}^\intercal wv^\intercal=\tr wv^\intercal T_{2n}$, so we may look at the sum
    \[\sum_{v,w}\psi(\tr(cwv^\intercal-cvw^\intercal)T_{2n})=\sum_{v,w}\psi(-2c\tr vw^\intercal T_{2n}).\]
    Fixing $v$ and summing over $w$, \Cref{lem:matrix-char-sum} tells us that we only get a nonzero contribution when $v=0$, where we see the sum will evaluate to $q^{2n}$. Thus, in this case, the sum in question compresses down to $q^{4n}g\left(\omega^2,\psi^2\right)g_{2n}(\omega,\psi,T_{2n})$.
\end{proof}
Next, we handle $A_{2n-1,2n}=0$.
\begin{lemma} \label{lem:gsum-alt-0}
    Take $n\ge1$. Fix characters $\omega\colon\FF_q^\times$ and $\psi\colon\FF_q\to\CC^\times$, and set $T_{2i}\coloneqq\op{diag}(J,\ldots,J)\in\Alt^\times_{2i}$ for each $i$. Choosing some nonzero vector $\ov v\in\FF_q^{2n+2}$ such that $\ov v_{2n+1}=\ov v_{2n+2}=0$, we have
    \[\sum_{\substack{A\in\Alt_{2n+2}^\times\\Ae_{n+2}=\ov v}}\omega(\det A)\psi(\tr AT_{2n+2})=\begin{cases}
        0 & \text{if }\psi=1, \\
        q^{2n}g_{2n}(\psi,\omega,T_{2n}) & \text{if }\psi=1\text{ and }\omega^2=1.
    \end{cases}\]
\end{lemma}
\begin{proof}
    We would like to use the same bijection as \Cref{lem:gsum-alt-not-0}, but of course something must change because the conclusion is different. Because $\ov v$ is nonzero, we may find an index $i_0\notin\{2n+1,2n+2\}$ such that $\ov v_{i_0}$ is nonzero. By mapping $A\mapsto\sigma A\sigma$ where $\sigma$ is the permutation matrix associated to some permutation of the form $(2i,2j)(2i+1,2j+1)$, we see that the sum will not change (because $\det\sigma A\sigma=\det A$ and $\sigma T_{2n+2}\sigma=T_{2n+2}$); thus, we may apply such a permutation to assume that $i_0\in\{2n-1,2n\}$. We now set $\sigma\coloneqq(i_0,2n+1)$ and apply $A\mapsto\sigma A\sigma$ to our sum, which does adjust $\ov v$ (so that $\ov v_{2n+1}\ne0$ while $\ov v_{i_0}=0$) as well as $T_{2n+2}$ so that
    \[\sigma T_{2n+2}\sigma\in\left\{\begin{bmatrix}
        \op{diag}(J,\ldots,J) \\
        &&&& -1 \\ &&& 1 \\ && -1 \\ & 1
    \end{bmatrix},\begin{bmatrix}
        \op{diag}(J,\ldots,J) \\
        &&& -1 \\ &&&& -1 \\ & 1 \\ && 1
    \end{bmatrix}\right\}\]
    (The left happens when $i_0=2n-1$, and the right happens when $i_0=2n$.) With our now adjusted $\ov v$, we write $\ov v=(-cv,-c,0)$ where $v\in\FF_q^{2n}$ and $c\in\FF_q^\times$.

    Now, as in \Cref{lem:gsum-alt-not-0}, the main point is that
    \[\arraycolsep=1.4pt\begin{array}{cccccc}
        \Alt_{2n}^\times &\times& \FF_q^n &\to& \Alt_{2n+2}^\times \\
        (B &,& w) &\mapsto& \begin{bsmallmatrix}
            1_{2n} & v & w \\
            & 1 \\ && 1
        \end{bsmallmatrix}\begin{bsmallmatrix}
            B \\ && -c \\ & c
        \end{bsmallmatrix}\begin{bsmallmatrix}
            1_{2n} \\ v^\intercal & 1 \\ w^\intercal && 1
        \end{bsmallmatrix}
    \end{array}\]
    is a bijection onto $A\in\Alt_{2n+2}^\times$ with $Ae_{2n+2}=(-cv,-c,0)$ because we can compute
    \[\begin{bmatrix}
        1_{2n} & v & w \\
        & 1 \\ && 1
    \end{bmatrix}\begin{bmatrix}
        B \\ && -c \\ & c
    \end{bmatrix}\begin{bmatrix}
        1_{2n} \\ v^\intercal & 1 \\ w^\intercal && 1
    \end{bmatrix}=\begin{bmatrix}
        B+cwv^\intercal-cvw^\intercal & cw & -cv \\
        -cw^\intercal && -c \\
        cv^\intercal & c
    \end{bmatrix}\]
    as before. Quickly, if $\psi=1$ and $\omega^2=1$, then the whole business with $\psi$ does not matter, so our sum looks like $\sum_{B,w}\omega(\det B)\omega\left(c^2\right)=q^{2n}g_{2n}(\psi,\omega,T_{2n})$, as required.

    Otherwise, we take $\psi\ne1$, and we need to produce cancellation in the sum. For this, we hold $B$ constant and let $w$ vary; in fact, we only need to let $w_{2n-1},w_{2n}\in\FF_q$ vary. In particular, after some rearrangement, we see that the sum in question contains the factor
    \[\sum_{w\in\FF_q^n}\psi\left(\tr\begin{bmatrix}
        wv^\intercal-cvw^\intercal & cw & -cv \\
        -cw^\intercal && -c \\
        cv^\intercal & c
    \end{bmatrix}\sigma T_{2n+2}\sigma\right),\]
    which we will show vanishes. In fact, we will look at a factor of this sum. Because $\sigma T_{2n+2}\sigma$ takes the form $\op{diag}(T_{2n-2},T')$ for some $T'\in\Alt_4^\times$, we see that the sum above contains the factor
    \[\sum_{w_{2n-1},w_{2n}\in\FF_q}\psi\left(\tr\begin{bmatrix}
        & cw_{2n-1}v_{2n}-cv_{2n-1}w_{2n} & cw_{2n-1} & -cv_{2n-1} \\
        cw_{2n}v_{2n-1}-cv_{2n}w_{2n-1} & & cw_{2n} & -cv_{2n} \\
        -cw_{2n-1} & -cw_{2n} & & -c \\
        cv_{2n-1} & cv_{2n} & c
    \end{bmatrix}T'\right)\]
    by using the bottom-right $4\times4$ minors of our matrices. We now compute this sum in two casework on $i_0$.
    \begin{itemize}
        \item If $i_0=2n-1$, our trace is
        \[\tr\begin{bmatrix}
            & cw_{2n-1}v_{2n}-cv_{2n-1}w_{2n} & cw_{2n-1} & -cv_{2n-1} \\
            cw_{2n}v_{2n-1}-cv_{2n}w_{2n-1} & & cw_{2n} & -cv_{2n} \\
            -cw_{2n-1} & -cw_{2n} & & -c \\
            cv_{2n-1} & cv_{2n} & c
        \end{bmatrix}\begin{bmatrix}
            &&& -1 \\ && 1 \\ & -1 \\ 1
        \end{bmatrix},\]
        so our sum is $\sum\psi(-2cv_{2n-1})\psi(-2cw_{2n})$, which vanishes as $w_{2n}\in\FF_q$ varies.
        \item If $i_0=2n$, our trace is
        \[\tr\begin{bmatrix}
            & cw_{2n-1}v_{2n}-cv_{2n-1}w_{2n} & cw_{2n-1} & -cv_{2n-1} \\
            cw_{2n}v_{2n-1}-cv_{2n}w_{2n-1} & & cw_{2n} & -cv_{2n} \\
            -cw_{2n-1} & -cw_{2n} & & -c \\
            cv_{2n-1} & cv_{2n} & c
        \end{bmatrix}\begin{bmatrix}
            && -1 \\ &&& -1 \\ 1 \\ & 1
        \end{bmatrix},\]
        so our sum is $\sum\psi(2cw_{2n-1}-2cv_{2n})$, which vanishes as $w_{2n-1}\in\FF_q$ varies.
        \qedhere
    \end{itemize}
\end{proof}
\begin{remark}
    In the case where $\psi\ne1$, it may appear suspicious that we induced cancellation in the sum of \Cref{lem:gsum-alt-0} but not in \Cref{lem:gsum-alt-not-0}. Namely, if $\ov v\in\FF_q^{2n+2}$ with $\ov v_{2n+1}=\ov v_{2n+2}=0$ can swap a coordinate with $\ov v_{2n+1}$ to cause the sum to vanish, then it looks like the sum should still vanish in the case where $\ov v_{2n+1}\ne0$. The way out of this apparent contradiction is to note that \Cref{lem:gsum-alt-not-0} shows, for fixed $v$ with $\ov v_{2n+1}\ne0$, the sum will in fact produce no contribution unless all the other coordinates vanish.
\end{remark}
We now synthesize our cases.
\begin{theorem} \label{thm:gsum-alt}
    Fix characters $\omega\colon\FF_q^\times\to\CC^\times$ and $\psi\colon\FF_q\to\CC^\times$ and some $T\in\Alt_{2n}^\times$.
    \begin{listalph}
        \item Suppose $\psi\ne1$. Then
        \[g_{2n}(\omega,\psi,T)=\frac{q^{n(n-1)}}{\omega(\det T)}\cdot g\left(\omega^2,\psi^2\right)^n.\]
        \item Suppose $\psi=1$ and $\omega^2=1$. Then
        \[g_{2n}(\omega,\psi,T)=q^{n(n-1)}\prod_{i=1}^{n}\left(q^{2i-1}-1\right).\]
    \end{listalph}
    For any $(\psi,\omega)$ not in the above list, the sum vanishes.
\end{theorem}
\begin{proof}
    Note the last sentence follows from \Cref{lem:gsum-alt-basic}. Anyway, we quickly reduce to the case where $T=\op{diag}(J,\ldots,J)$ using \Cref{lem:gsum-alt-basic}. For (b), $T$ does not even matter, so we may as well take this value of $T$. For (a), we note that we can find $g\in\GL_{2n}$ such that $\op{diag}(J,\ldots,J)=gTg^\intercal$. Now, \Cref{lem:gsum-alt-basic} and inspection tells us that both sides of our equalities adjust by the same factor of $\omega(\det g)^{-2}$ upon sending $T\mapsto gTg^\intercal$ in (a). Thus, we may as well assume $T=\op{diag}(J,\ldots,J)$.

    We now use induction to show both (a) and (b). Write $T_{2i}\coloneqq\op{diag}(J,\ldots,J)\in\GL_{2i}$ for each $i$.
    \begin{listalph}
        \item For $n=0$, there is nothing to say. Now, for our induction, we note that summing \Cref{lem:gsum-alt-not-0,lem:gsum-alt-0} implies
        \[g_{2n+2}(\omega,\psi,T_{2n+2})=q^{2n}g_{2n}(\omega,\psi,T_{2n}),\]
        so the result follows by induction.
        \item For $n\in\{0,1\}$, there is not much to say. Now, for our induction, we note that summing \Cref{lem:gsum-alt-not-0,lem:gsum-alt-0} implies
        \[g_{2n+2}(\omega,\psi,T)=\left(q^{4n}(q-1)+q^{2n}\left(q^{2n}-1\right)\right)g_{2n}(\omega,\psi,T)\]
        for $n\ge1$. Thus, it is enough to show that our right-hand side satisfies the same recurrence. Namely, we would like to show
        \[q^{\left(n+1\right)n}\prod_{i=1}^{n+1}\left(q^{2i-1}-1\right)\stackrel?=\left(q^{4n}\left(q-1\right)+q^{2n}\left(q^{2n}-1\right)\right)q^{n\left(n-1\right)}\prod_{i=1}^{n}\left(q^{2i-1}-1\right),\]
        which we note simplifies down to
        \[q^{2n}\left(q^{2n+1}-1\right)\stackrel?=\left(q^{4n}\left(q-1\right)+q^{2n}\left(q^{2n}-1\right)\right),\]
        which is true after a little more rearrangement.
        \qedhere
    \end{listalph}
\end{proof}
\begin{remark}
    A careful extraction of the row-reductions involved in our proof is able to show that any $T\in\Alt_{2n}^\times$ can be written as $g\op{diag}(J,\ldots,J)g^\intercal$ for some $g\in\GL_{2n}$.
\end{remark}
As usual, we close with a combinatorial application.
\begin{corollary} \label{cor:count-alt}
    Fix some even nonnegative integer $2n$ and some $T\in\Alt_{2n}^\times$. Further, fix $d\in\FF_q^{\times2}$ and $t\in\FF_q$. Then the number of $A\in\Alt_{2n}^\times$ such that $\det A=d$ and $\tr AT=t$ is
    \begin{align*}
        &\frac2{q(q-1)}\left(q^{n(n-1)}\prod_{i=1}^{n}\left(q^{2i-1}-1\right)-q^{n(n-1)}(q-1)^n\right) \\
        &+ q^{n(n-1)}\cdot\#\left\{(y_1,\ldots,y_n):2(y_1+\cdots+y_n)=t,\frac{(y_1\cdots y_n)^2}{\det T}=d\right\}.
    \end{align*}
\end{corollary}
\begin{proof}
    For any characters $\omega\colon\FF_q^\times$ and $\psi\colon\FF_q\to\CC^\times$, we claim that
    \begin{align*}
        g_{2n}(\omega,\psi,T) &\stackrel?= \frac{q^{n(n-1)}}{\omega(\det T)}\cdot g\left(\omega^2,\psi^2\right)^n \\
        &\qquad+\frac 2{q(q-1)}
        \left(q^{n(n-1)}\prod_{i=1}^{n}\left(q^{2i-1}-1\right)-q^{n(n-1)}(q-1)^n\right)\sum_{a\in\FF_q^{\times2},b\in\FF_q}\omega^2(a)\psi(b).
    \end{align*}
    If $\psi\ne1$, then this follows immediately from \Cref{thm:gsum-alt}. If $\psi=1$ and $\omega^2\ne1$, then everything vanishes. Lastly, if $\psi=1$ and $\omega^2=1$, then this follows from \Cref{thm:gsum-alt} again.

    We now note that a direct expansion implies
    \[\frac1{\omega(\det T)}\cdot g\left(\omega^2,\psi^2\right)=\sum_{y_1,\ldots,y_n}\omega\left(\frac{(y_1\cdot y_n)^2}{\det T}\right)\psi(2(y_1+\cdots+y_n)),\]
    so the result follows from summing the claim suitably over all $\omega$ and $\psi$.
\end{proof}

% \section{Computation of the Symmetric Gauss Sum} \label{sec:sym-gauss-sum}

% Let $\FF_q$ denote the finite field with $q$ elements, where $q$ is an odd prime-power, and let $\operatorname{Sym}_n^\times(\FF_q)$ denote the set of invertible symmetric $n\times n$ matrices with entries in $\FF_q$. The goal of the present section is to compute the ``symmetric'' Gauss sum
% \[g_n(\omega,\psi,T)\coloneqq\sum_{A\in\operatorname{Sym}_n^\times(\FF_q)}\omega(\det A)\psi(\tr AT)\]
% where $n\in\ZZ_{\ge0}$ is a nonnegative integer, $\omega\colon\FF_q^\times\to\CC^\times$ and $\psi\colon\FF_q\to\CC^\times$ are characters, and $T\in\Sym_n^\times(\FF_q)$. Here, $\operatorname{Sym}_0^\times$ is understood to consist of a single empty $0\times0$ matrix with trace $0$ and determinant $1$ so that $g_0(\omega,\psi,T)=1$. In the case where $\omega$ is a quadratic character, such sums were considered by \cite{walling-almost-gauss-sum}.

% In the following discussion, we will make use of many Gauss sums, so it will be helpful to have the notation
% \[g(\omega,\psi)\coloneqq\sum_{a\in\FF_q^\times}\omega(a)\psi(a),\]
% where $\omega$ and $\psi$ are as above. For example, $g_1(\omega,\psi,1)=g(\omega,\psi)$.

% We now state our main result.
% \begin{thm} \label{thm:sym-gauss-sum}
%     Let $\omega\colon\FF_q^\times\to\CC^\times$ and $\psi\colon\FF_q\to\CC^\times$ be characters, and let $\chi\colon\FF_q^\times\to\CC^\times$ denote the nontrivial quadratic character, and fix some $T\in\Sym_n^\times(\FF_q)$. Further, assume that $\psi$ is nontrivial.
%     \begin{itemize}
%         \item If $n=2m$ is an even nonnegative integer, then
%         \[g_{2m}(\omega,\psi,T)=\frac{\chi(-1)^m\chi(\det T)q^{m^2}}{\omega(4^m\det T)}\cdot g\left(\omega^2,\psi\right)^m.\]
%         \item If $n=2m+1$ is an odd nonnegative integer, then
%         \[g_{2m+1}(\omega,\psi,T)=\frac{q^{m(m+1)}}{\omega(4^m\det T)}\cdot g(\omega,\psi)g\left(\omega^2,\psi\right)^m.\]
%     \end{itemize}
% \end{thm}
% \begin{remark}
%     The theorem implies that
%     \[\sum_{A\in\Sym_n^\times(\FF_q)}\omega\left(\det AT^{-1}\right)\psi(\tr A)=\sum_{B\in\Sym_n^\times(\FF_q)}\omega(\det B)\psi(\tr BT),\]
%     but this is not obvious: in particular, one cannot apply the variable change $B\coloneqq AT^{-1}$ because $AT^{-1}$ need not be symmetric! We would be interested in a more direct proof of the above equality.
% \end{remark}
% \begin{remark} \label{rem:mag-of-sym-gauss-sum}
%     In the ``generic'' case $\omega^2\ne1$, all Gauss sums have magnitude $\sqrt q$ (see \Cref{prop:mag-gauss-sum}), so \Cref{thm:sym-gauss-sum} implies
%     \[\left|g_n(\omega,\psi,T)\right|=q^{n(n+1)/4}=q^{\frac12\binom{n+1}2}.\]
%     This is roughly what we expect to be true from ``square-root cancellation'': $\left|\Sym_n(\FF_q)\right|=q^{\binom{n+1}2}$.
% \end{remark}

% \subsection{The Main Computation}
% In this subsection, we prove \Cref{thm:sym-gauss-sum}. The key idea is to use an explicit row-reduction. This is akin to the application of the Bruhat decomposition to compute the relevant Gauss sums for $G\in\{\GL_{2n},\SL_{2n}\}$ done in \cite{kim-gauss-sum}; however, to handle sums of symmetric and alternating matrices, we make the row-reduction more explicit. As such, the hard work is done in the following lemma.
% \begin{lemma} \label{lem:sym-gauss-sum-ind}
%     Let $\omega\colon\FF_q^\times\to\CC^\times$ and $\psi\colon\FF_q\to\CC^\times$ be characters, and let $\chi\colon\FF_q^\times\to\CC^\times$ denote the nontrivial quadratic character. Further, assume that $\psi$ is nontrivial. For any positive integer $n$ and $d_1,\ldots,d_{n+1}\in\FF_q^\times$,
%     \[g_{n+1}(\omega,\psi,\operatorname{diag}(d_1,\ldots,d_{n+1}))=g_n(\omega,\psi,\operatorname{diag}(d_1,\ldots,d_n))\cdot\frac{\chi(d_1\cdots d_n)\chi(d_{n+1})^n}{\omega(d_{n+1})}\cdot g\left(\omega\chi^n,\psi\right)g(\chi,\psi)^n.\]
% \end{lemma}
% \begin{proof}
%     For brevity, set $T\coloneqq\operatorname{diag}(d_1,\ldots,d_{n+1})$. For a matrix square $A\in M_m(\FF_q)$, we use the notation $A_{k\ell}$ denote the entry of $A$ in the $k$th row and $\ell$th column. Now, for some $A\in\operatorname{Sym}_{n+1}^\times$, there are two cases.
%     \begin{itemize}
%         \item Suppose $A_{n+1,n+1}\ne0$; here, set $T'\coloneqq\operatorname{diag}(d_1,\ldots,d_n)$ for brevity. For our Gaussian elimination, we note that the map
%         \[\arraycolsep=1.4pt\begin{array}{rclclcl}
%             \operatorname{Sym}_n^\times(\FF_q) & \times & \FF_q^n & \times & \FF_q^\times & \to & \{A\in\operatorname{Sym}_{n+1}^\times(\FF_q):A_{n+1,n+1}\ne0\} \\
%             (A' & , & v & , & c) & \mapsto & \displaystyle\begin{bmatrix}
%                 1 & v \\
%                   & 1
%             \end{bmatrix}\begin{bmatrix}
%                 A' \\ & c
%             \end{bmatrix}\begin{bmatrix}
%                 1 \\ v^\intercal & 1
%             \end{bmatrix}=\begin{bmatrix}
%                 A'+cvv^\intercal & cv \\
%                 cv^\intercal & c
%             \end{bmatrix}
%         \end{array}\]
%         is a bijection. Indeed, $A_{n+1,n+1}$ uniquely determines $c$, the values $A_{k,n+1}$ for $1\le k\le n$ uniquely determine $v$, and then the rest of the matrix uniquely determines $A'$. Using this bijection, we see that
%         \begin{align*}
%             S_{\ne0} \coloneqq{}& \sum_{\substack{A\in\operatorname{Sym}_{n+1}^\times(\FF_q)\\A_{n+1,n+1}\ne0}}\omega(\det A)\psi(\tr AT) \\
%             ={}& \sum_{\substack{A\in\operatorname{Sym}_n^\times(\FF_q)\\v\in \FF_q^n,c\in\FF_q^\times}}\omega(c\det A')\psi\left(\tr A'T'+c\tr vv^\intercal T'+cd_{n+1}\right) \\
%             ={}& g_n(\omega,\psi,T')\sum_{c\in\FF_q^\times}\omega(c)\psi(cd_{n+1})\sum_{v\in\FF_q^n}\psi\left(c\tr vv^\intercal T'\right) \\
%             ={}& g_n(\omega,\psi,T')\sum_{c\in\FF_q^\times}\omega(c)\psi(cd_{n+1})\prod_{k=1}^n\Bigg(\sum_{a\in\FF_q}\psi\left(cd_ka^2\right)\Bigg).
%         \end{align*}
%         Quickly, we claim that
%         \[\sum_{a\in\FF_q}\psi\left(cd_ka^2\right)\stackrel?=\sum_{a\in\FF_q}(1+\chi(cd_ka))\psi(a),\]
%         where we have extended $\chi$ to $\FF_q$ by $\chi(0)\coloneqq0$. Indeed, for any $b\in\FF_q$, we see that $\psi(b)$ appears on the left-hand side $0$ times if $b$ does not have the form $cd_ka^2$, appears $1$ time if $b=0$, and appears $2$ times if $b$ is nonzero and has the form $cd_ka^2$; these values are exactly $1+\chi(cd_ka)$ in all cases. As such, the claim follows, and because $\psi$ is nontrivial, we actually have
%         \[\sum_{a\in\FF_q}\psi\left(cd_ka^2\right)=\sum_{a\in\FF_q}\chi(cd_ka)\psi(a)=\chi(cd_k)g(\chi,\psi).\]
%         Plugging this in, we see that
%         \begin{align*}
%             S_{\ne0} &= g_n(\omega,\psi,T')\sum_{c\in\FF_q^\times}\omega(c)\chi(c)^n\psi(cd_{n+1})\chi(d_1\cdots d_n)g(\chi,\psi)^n \\
%             &= g_n(\omega,\psi,T')\cdot\frac{\chi(d_1\cdots d_n)\chi(d_{n+1})^n}{\omega(d_{n+1})}\sum_{c\in\FF_q^\times}\omega(c)\chi(c)^n\psi(c)g(\chi,\psi)^n \\
%             &= g_n(\omega,\psi,T')\cdot\frac{\chi(d_1\cdots d_n)\chi(d_{n+1})^n}{\omega(d_{n+1})}\cdot g\left(\omega\chi^n,\psi\right)g(\chi,\psi)^n.
%         \end{align*}
%         \item Suppose $A_{n+1,n+1}=0$; here, set $T'\coloneqq\operatorname{diag}(d_1,\ldots,d_{n-1})$ for brevity. The computation in the previous case implies that we would like to show
%         \[\sum_{\substack{A\in\operatorname{Sym}_{n+1}^\times(\FF_q)\\A'_{n+1,n+1}=0}}\omega(\det A)\psi(\tr AT)\stackrel?=0.\]
%         In fact, let $e_{n+1}$ denote the $n$th basis vector, and for any $v\in k^{n-1}$ and $c\in k$, we claim
%         \[S(v,c)\coloneqq\sum_{\substack{A\in\operatorname{Sym}_{n+1}^\times(\FF_q)\\Ae_{n+1}=(v,c,0)}}\omega(\det A)\psi(\tr AT)\stackrel?=0.\]
%         To do Gaussian elimination, we would like to assume $c\ne0$. Well, because $A$ is invertible, we know that $A_{k,n+1}\ne0$ for some $1\le k\le n$ (recall $A_{n+1,n+1}=0$ already), so if the sum is to be nonempty, we may assume that $c\ne0$ or $v_k\ne0$ for some $k$. If $v_k\ne0$, then note swapping the $k$th row and column with the $n$th row and column (of both $A$ and $T$) will not affect the trace or determinant but does switch $v_k$ with $c$, which grants $c\ne0$.
        
%         We now do Gaussian elimination: note that there is a bijection
%         \[\arraycolsep=1.4pt\begin{array}{rclclcl}
%             \operatorname{Sym}_{n-1}^\times(\FF_q) & \times & \FF_q^{n-1} & \times & \FF_q & \to & \{A\in\operatorname{Sym}_{n+1}^\times(\FF_q):Ae_{n+1}=(v,c,0)\} \\
%             (A' & , & w & , & d) & \mapsto &  \begin{bmatrix}
%                 1 & \frac1cv & w \\
%                   & 1 &   \\
%                   &   & 1
%             \end{bmatrix}\begin{bmatrix}
%                 A' \\
%                   & d & c \\
%                   & c
%             \end{bmatrix}\begin{bmatrix}
%                 1 \\
%                 \frac1cv^\intercal & 1 \\
%                 w^\intercal &   & 1
%             \end{bmatrix}
%         \end{array}\]
%         (Here, $\operatorname{Sym}_0^\times$ is understood to consist of only the ``empty'' $0\times0$ matrix.) To see that this is a bijection, we expand out the matrix product as
%         \[\begin{bmatrix}
%             A'+\frac{d}{c^2}vv^\intercal+(vw^\intercal+wv^\intercal) & dv+cw & v \\
%             dv^\intercal+cw^\intercal & d & c \\
%             v^\intercal & c & 0
%         \end{bmatrix},\]
%         so we see that $A_{n,n}$ forces $d$, which then forces $w$ from $A_{k,n}$ as $1\le k\le n$; the rest of the data then forces $A'$. Thus,
%         \begin{align*}
%             S(v,c) &= \sum_{\substack{A'\in\operatorname{Sym}_{n-1}^\times(\FF_q)\\w\in\FF_q^{n-1},d\in\FF_q}}\omega\left(-c^2\det A'\right)\psi\left(\tr A'T'+\frac d{c^2}\tr vv^\intercal T'+2\tr vw^\intercal T'+dd_n\right) \\
%             &= \sum_{A'\in\operatorname{Sym}_{n-1}^\times(\FF_q)}\omega\left(-c^2\det A'\right)\psi(\tr AT')\Bigg(\sum_{d\in\FF_q}\psi\left(dd_n+\frac d{c^2}\tr vv^\intercal T'\right)\sum_{w\in\FF_q^{n-1}}\psi(2\tr vw^\intercal T')\Bigg).
%         \end{align*}
%         Beginning with the innermost sum, we see $\tr vw^\intercal T'=d_1v_1w_1+\cdots+d_{n-1}v_{n-1}w_{n-1}$, so this sum is
%         \[\sum_{w\in\FF_q^{n-1}}\psi(2\tr vw^\intercal T')=\prod_{k=1}^{n-1}\Bigg(\sum_{w_k\in\FF_q}\psi(2d_kv_kw_k)\Bigg).\]
%         In order for these inner sums to be nonzero, we note that we must have $v_k=0$ for each $k$ because $\psi$ is a nontrivial character. Thus, we may assume $v=0$, from which we see
%         \[S(0,c)=\Bigg(\sum_{A'\in\operatorname{Sym}_{n-1}^\times(\FF_q)}\omega(\det A')\psi(\tr A'T')\Bigg)\Bigg(\sum_{d\in\FF_q}\psi(d_nd)\Bigg)=0,\]
%         so we conclude in this case as well.
%     \end{itemize}
%     Summing the above two cases finishes the proof of \Cref{lem:sym-gauss-sum-ind}.
% \end{proof}
% We are now ready to prove \Cref{thm:sym-gauss-sum}.
% \begin{proof}[Proof of \Cref{thm:sym-gauss-sum}]
%     Quickly, we reduce to the case where $T$ is diagonal. Indeed, by choosing an orthogonal basis for the symmetric bilinear form given by $T$, we receive some $g\in\GL_n(\FF_q)$ such that $D\coloneqq gTg^\intercal$ is diagonal. As such, we compute
%     \begin{align*}
%         g_n(\omega,\psi,T) &= \sum_{A\in\Sym_n^\times(\FF_q)}\omega(\det A)\psi(\tr AT) \\
%         &= \sum_{A\in\Sym_n^\times(\FF_q)}\omega(\det A)\psi\left(\tr g^{-\intercal}Ag^{-1}D\right) \\
%         &= \sum_{A\in\Sym_n^\times(\FF_q)}\omega\left(\det g^\intercal Ag\right)\psi\left(\tr AD\right) \\
%         &= \omega(\det g)^2g_n(\omega,\psi,D).
%     \end{align*}
%     Now, suppose we have proven the theorem for diagonal matrices. In this case, we see $g_n(\omega,\psi,D)=(\det D)^{-1}g_n(\omega,\psi,1)$, so $\det D=(\det g)^2(\det T)$ implies that
%     \[g_n(\omega,\psi,T)=(\det T)^{-1}g_n(\omega,\psi,1),\]
%     which is the theorem for $T$, as desired.
    
%     Thus, we may assume that $T\coloneqq\operatorname{diag}(d_1,\ldots,d_n)$. At this point, we induct on $n$. For $n=0$ and $n=1$, there is nothing to say. For the induction, assume $n\ge2$, and we use \Cref{lem:sym-gauss-sum-ind}; for brevity, set $T'\coloneqq\operatorname{diag}(d_1,\ldots,d_{n-1})$. There are two cases.
%     \begin{itemize}
%         \item Suppose that $n=2m$ is an even positive integer. In this case, \Cref{lem:sym-gauss-sum-ind} and induction yields
%         \begin{align*}
%             g_{2m}(\omega,\psi) &= g_{2m-1}(\omega,\psi)\cdot\frac{\chi(\det T)}{\omega(d_{n+1})}\cdot g(\omega\chi,\psi)g(\chi,\psi)^{2m-1} \\
%             &= \frac{\chi(\det T)q^{(m-1)m}}{\omega(4^{m-1}\det T)}\cdot g(\omega,\psi)g\left(\omega^2,\psi\right)^{m-1}g(\omega\chi,\psi)g(\chi,\psi)^{2m-1}.
%         \end{align*}
%         By \Cref{prop:quad-twist-gauss-sum}, this is
%         \[g_{2m}(\omega,\psi)=\frac{\chi(\det T)q^{m^2-m}}{\omega(4^m\det T)}\cdot g\left(\omega^2,\psi\right)^mg(\chi,\psi)^{2m}.\]
%         Lastly, \Cref{prop:mag-gauss-sum} yields
%         \[g_{2m}(\omega,\psi)=\frac{\chi(-1)^m\chi(\det T)q^{m^2}}{\omega(4^m\det T)}\cdot g\left(\omega^2,\psi\right)^m.\]
%         \item Suppose $n=2m+1$ is an odd positive integer with $m\ge1$. In this case, \Cref{lem:sym-gauss-sum-ind} and induction yields
%         \begin{align*}
%             g_{2m+1}(\omega,\psi) &= g_{2m}(\omega,\psi)g(\omega,\psi)\cdot\frac{\chi(\det T')}{\omega(d_{n+1})}\cdot g(\chi,\psi)^{2m} \\
%             &= \frac{\chi(-1)^mq^{m^2}}{\omega(4^m\det T)}\cdot g\left(\omega^2,\psi\right)^mg(\omega,\psi)g(\chi,\psi)^{2m}.
%         \end{align*}
%         From here, \Cref{prop:mag-gauss-sum} implies
%         \[g_{2m+1}(\omega,\psi)=\frac{q^{m^2+m}}{\omega(4)^m}\cdot g(\omega,\psi)g\left(\omega^2,\psi\right)^m.\]
%     \end{itemize}
%     The above cases complete the induction.
% \end{proof}

% % \subsection{A Gamma Matrix Computation}
% % In this subsection, we use \Cref{thm:sym-gauss-sum} to compute the finite-field analogue of a $\gamma$-matrix attached to the prehomogeneous space $\Sym_n(\FF_q)$. For context, the $p$-adic analogue of \Cref{thm:sym-gauss-sum} is intimately related to zeta functions attached to prehomogeneous spaces; see \cite[Section~3]{kudla-degenerate-principal-series} or \cite[Section~2]{ikeda-siegel-fe}. We refer to \cite{sato-prehom-spaces} for the general theory of prehomogeneous spaces.

% % In our case, we note that $({\GL_n},{\Sym_n})$ is a prehomogeneous space, where the action is given by $g\cdot A\coloneqq gAg^\intercal$. In other words, there is a proper algebraic subset $S\subseteq\Sym_n(\overline k)$ such that $\Sym_n(\overline k)\setminus S$ is a single $\GL_n(\overline k)$-oribt. To see this, for any field $k$, we note that two invertible symmetric matrices $A,B\in\Sym_n(k)$ have some $g\in\GL_n(k)$ such that $g\cdot A=B$ if and only if $\det A$ and $\det B$ are the same element in $k^\times/k^{\times2}$; thus, when passing to the algebraic closure, $\Sym_n^\times(\overline k)$ is a Zariski open $\GL_n(\overline k)$-orbit in $\Sym_n(\overline k)$.

% % We now define our zeta function. Let the $\GL_n(\FF_q)$-orbits of $\Sym_n^\times(\FF_q)$ be denoted by $Y_1\sqcup Y_{-1}$, corresponding to if $A\in\Sym_n^\times(\FF_q)$ has square or non-square determinant, respectively. Now, because the proper algebraic subset $S\subseteq\Sym_n(\FF_q)$ is cut out by $\det$, our attached zeta functions are
% % \[Z_k(\omega,\varphi)\coloneqq\sum_{A\in Y_k}\omega(\det A)\varphi(A),\]
% % where $\omega\colon\FF_q^\times\to\CC^\times$ is a character and $\varphi\colon\Sym_n(\FF_q)\to\CC$ is some test function; let $S(\Sym_n(\FF_q))$ denote this space of test functions. Now, fix once and for all a nontrivial additive character $\psi\colon\FF_q\to\CC^\times$, so we may define the Fourier transform
% % \[\mathcal F_\psi\varphi(A)\coloneqq\sum_{B\in\Sym_n(\FF_q)}\varphi(B)\psi(\tr AB).\]
% % \begin{remark} \label{rem:sym-fourier-twice}
% %     To view $\mathcal F_\psi$ as a Fourier transform, we claim $\mathcal F_{\psi^{-1}}\circ\mathcal F_\psi=q^{\binom{n+1}2}$. It suffices to check this result on indicators $1_C$ where $C\in\Sym_n(\FF_q)$. Then we see $\mathcal F_\psi 1_C(B)=\psi(\tr BC)$ for any $B\in\Sym_n(\FF_q)$, so
% %     \[\left(\mathcal F_{\psi^{-1}}\mathcal F_\psi 1_C\right)(A)=\sum_{B\in\Sym_n(\FF_q)}\psi(\tr(C-A)B).\]
% %     If $A=C$, then the sum is $q^{\binom{n+1}2}$. Otherwise, $A'\coloneqq C-A\ne0$, and we need the sum to vanish. Well, if $A'_{k'\ell'}\ne0$ for some indices $k'$ and $\ell'$, then consider the matrix $B(k',\ell')$ by $B(k',\ell')_{k\ell}=1_{\{k,\ell\}=\{k',\ell'\}}$, which gives
% %     \[\sum_{b\in\FF_q}\psi(\tr A'bB(k',\ell'))=\sum_{b\in\FF_q}\sum_{k,\ell=1}^n\psi(bA'_{k\ell}B(k',\ell')_{\ell k})=\sum_{b\in\FF_q}\psi(2bA'_{k'\ell'})=0.\]
% %     Grouping the rest of the sum by $\Sym_n(\FF_q)/\FF_qB(k',\ell')$ shows that $\sum_{B\in\Sym_n(\FF_q)}\psi(\tr A'B)=0$, as needed.
% % \end{remark}
% % A functional equation of zeta functions attached to prehomogeneous spaces is typically a result relating $Z_\bullet(\omega,\varphi)$ to a dual version $Z_\bullet\left(\omega^{-1},\mathcal F_\psi\varphi\right)$; some such results exist in the literature \cite{denef-finite-field-prehom-space}, but we will prove an analogue here for completeness. To prove our analogue, we begin with the following multiplicity-two result.
% % \begin{prop} \label{prop:prehom-mult-two}
% %     Fix notation as above, and let $\omega\colon\FF_q^\times\to\CC^\times$ be a character.
% %     \begin{enumerate}[label=(\alph*)]
% %         \item For any $k\in\{\pm1\}$ and $g\in\GL_n(\FF_q)$ and $\varphi\in S(\Sym_n(\FF_q))$, we have
% %         \[Z_k(\omega,g\cdot\varphi)=\omega(\det g)^2Z_k(\omega,\varphi).\]
% %         \item The functionals $Z_1(\omega)$ and $Z_2(\omega)$ are a basis of the space
% %         \[\Hom_{\GL_n(\FF_q)}\left(S(\Sym_n(\FF_q))^\circ,\omega^2\circ\det\right),\]
% %         where $S(\Sym_n(\FF_q))^\circ$ denotes the functional on $\Sym_n(\FF_q)$ supported on $\Sym_n^\times(\FF_q)$.
% %     \end{enumerate}
% % \end{prop}
% % \begin{proof}
% %     Quickly, we recall that the $\GL_n(\FF_q)$-action on $\Sym_n(\FF_q)$ is given by $(g\cdot\varphi)(A)=\varphi\left(g^{-1}\cdot A\right)=\varphi\left(g^{-1}Ag^{-\intercal}\right)$. From this one can see that $S\left(\Sym_n^\times(\FF_q)\right)^\circ$ is in fact a $\GL_n(\FF_q)$-subrepresentation of $\Sym_n(\FF_q)$.

% %     To see (a), we directly compute
% %     \begin{align*}
% %         Z_k(\omega,g\cdot\varphi) &= \sum_{A\in Y_k}\omega(\det A)(g\cdot\varphi)(A) \\
% %         &= \sum_{A\in Y_k}\omega(\det A)\varphi\left(g^{-1}\cdot A\right) \\
% %         &= \sum_{A\in Y_k}\omega(\det g\cdot A)\varphi(A) \\
% %         &= \omega(\det g)^2\sum_{A\in Y_k}\omega(\det A)\varphi(A),
% %     \end{align*}
% %     which is what we wanted.
    
% %     Thus, we spend most of our time on (b). Fix representatives $A_1\in Y_1$ and $A_{-1}\in Y_{-1}$. Then we see that $Z_1(\omega)$ and $Z_2(\omega)$ are at least linearly independent as functionals on $S\left(\Sym_n(\FF_q)\right)^\circ$ because $Z_k(\omega,1_{A_\ell})=1_{k=\ell}\omega(\det A_\ell)$. 
    
% %     It remains to show that $Z_1$ and $Z_2$ span this eigenspace. The main point is that $\Sym_n^\times(\FF_q)$ has only two orbits, so any eigenvector $Z$ is essentially determined by two values. Rigorously, without loss of generality, we replace $Z$ with
% %     \[Z-\frac{Z(1_{A_1})}{\omega(\det A_1)}\cdot Z_1(\omega)-\frac{Z(1_{A_{-1}})}{\omega(\det A_{-1})}\cdot Z_{-1}(\omega)\]
% %     so that $Z(1_{A_1})=Z(1_{A_{-1}})=0$. We now claim that $Z=0$, which will complete the proof. It is enough to show that $Z(1_A)=0$ for any $A\in\Sym_n^\times(\FF_q)$.

% %     Well, $\Sym_n^\times(\FF_q)=Y_1\sqcup Y_{-1}$, so without loss of generality, take $A\in Y_1$. Then we may find $g\in\GL_n(\FF_q)$ so that $A=g\cdot A_1$, so
% %     \[1_A(B)=1_{g\cdot A_1}(B)=1_{A_1}\left(g^{-1}\cdot B\right)=(g\cdot1_{A_1})(B)\]
% %     for any $B\in\Sym_n(\FF_q)$. Thus, because $Z$ is an eigenvector,
% %     \[Z(1_A)=Z(g\cdot1_{A_1})=\omega(\det g)^2Z(1_{A_1})=0,\]
% %     as desired.
% % \end{proof}
% % \begin{remark} \label{rem:explicit-prehom-mult-two}
% %     In fact, for any eigenvector $Z$, the proof of \Cref{prop:prehom-mult-two} shows that
% %     \[Z(\varphi)=\frac{Z(1_{A_1})}{\omega(\det A_1)}\cdot Z_1(\omega,\varphi)+\frac{Z(1_{A_{-1}})}{\omega(\det A_{-1})}\cdot Z_{-1}(\omega,\varphi)\]
% %     for any $\varphi\in S(\Sym_n(\FF_q))^\circ$. Here, we recall $A_1\in Y_1$ and $A_{-1}\in Y_{-1}$ are any representatives.
% % \end{remark}
% % To use \Cref{prop:prehom-mult-two}, we thus want to show that $\varphi\mapsto Z\left(\omega^{-1},\mathcal F_\psi\varphi\right)$ is an eigenvector. This follows formally from the following lemma.
% % \begin{lemma} \label{lem:fourier-twists-g-action}
% %     For any $g\in\GL_n(\FF_q)$, the following diagram commutes.
% %     % https://q.uiver.app/#q=WzAsNCxbMCwwLCJTKFxcU3ltX24oXFxGRl9xKSkiXSxbMSwwLCJTKFxcU3ltX24oXFxGRl9xKSkiXSxbMCwxLCJTKFxcU3ltX24oXFxGRl9xKSkiXSxbMSwxLCJTKFxcU3ltX24oXFxGRl9xKSkiXSxbMCwxLCJcXG1hdGhjYWwgRl9cXHBzaSJdLFsyLDMsIlxcbWF0aGNhbCBGX1xccHNpIl0sWzAsMiwiZyIsMl0sWzEsMywiZ157LVxcaW50ZXJjYWx9Il1d&macro_url=https%3A%2F%2Fraw.githubusercontent.com%2FdFoiler%2Fnotes%2Fmaster%2Fnir.tex
% %     \[\begin{tikzcd}
% %     	{S(\Sym_n(\FF_q))} & {S(\Sym_n(\FF_q))} \\
% %     	{S(\Sym_n(\FF_q))} & {S(\Sym_n(\FF_q))}
% %     	\arrow["{\mathcal F_\psi}", from=1-1, to=1-2]
% %     	\arrow["{\mathcal F_\psi}", from=2-1, to=2-2]
% %     	\arrow["g"', from=1-1, to=2-1]
% %     	\arrow["{g^{-\intercal}}", from=1-2, to=2-2]
% %     \end{tikzcd}\]
% % \end{lemma}
% % \begin{proof}
% %     This is a direct computation. For any $g\in\GL_n(\FF_q)$ and $\varphi\in S(\Sym_n(\FF_q))$ and $A\in\Sym_n(\FF_q)$, we compute
% %     \begin{align*}
% %         (\mathcal F_\psi g\varphi)(A) &= \sum_{B\in\Sym_n(\FF_q)}(g\varphi)(B)\psi(\tr AB) \\
% %         &= \sum_{B\in\Sym_n(\FF_q)}\varphi\left(g^{-1}Bg^{-\intercal}\right)\psi(\tr AB) \\
% %         &= \sum_{B\in\Sym_n(\FF_q)}\varphi(B)\psi\left(\tr AgBg^\intercal\right) \\
% %         &= \sum_{B\in\Sym_n(\FF_q)}\varphi(B)\psi\left(\tr g^\intercal AgB\right) \\
% %         &= \mathcal F_\psi\varphi\left(g^\intercal\cdot A\right) \\
% %         &= \left(g^{-\intercal}\mathcal F_\psi\varphi\right)(A),
% %     \end{align*}
% %     which is what we wanted.
% % \end{proof}
% % \begin{thm} \label{thm:gamma-matrix-exists}
% %     Fix notation as above. Let $\omega\colon\FF_q^\times\to\CC^\times$ be a character. For any $k\in\{\pm1\}$, there exist unique constants $\gamma_{k,1}(\omega)$ and $\gamma_{k,-1}(\omega)$ such that
% %     \[Z_k\left(\omega^{-1},\mathcal F_\psi\varphi\right)=\gamma_{k,1}(\omega)Z_{1}(\omega,\varphi)+\gamma_{k,-1}(\omega)Z_{-1}(\omega,\varphi)\]
% %     for any $\varphi\in S(\Sym_n(\FF_q))$ supported on $\Sym_n^\times(\FF_q)$.
% % \end{thm}
% % \begin{proof}
% %     This follows formally from \Cref{prop:prehom-mult-two} and \Cref{lem:fourier-twists-g-action}. Indeed, it is enough to show that the functional $\varphi\mapsto Z_k\left(\omega^{-1},\mathcal F_\psi\varphi\right)$ on $S(\Sym_n(\FF_q))$ is a $\GL_n(\FF_q)$-eigenvector with eigenvalue $\omega^2\circ\det$. Well, for any $\varphi\in S(\Sym_n(\FF_q))$ and $g\in\GL_n(\FF_q)$, we use \Cref{lem:fourier-twists-g-action} to compute
% %     \begin{align*}
% %         Z_k\left(\omega^{-1},\mathcal F_\psi(g\varphi)\right) &= Z_k\left(\omega^{-1},g^{-\intercal}\mathcal F_\psi\varphi\right) \\
% %         &= \left(\omega^{-1}\left(\det g^{-\intercal}\right)\right)^2Z_k\left(\omega^{-1},\mathcal F_\psi\varphi\right) \\
% %         &= \omega(\det g)^2Z_k\left(\omega^{-1},\mathcal F_\psi\varphi\right),
% %     \end{align*}
% %     as desired.
% % \end{proof}
% % The main point of this subsection is to explicitly compute the constants $\gamma_{k,\ell}(\omega)$, which make up the ``change-of-basis'' $\gamma$-matrix. To this end, we have the following result.
% % \begin{thm} \label{thm:gamma-matrix-computation}
% %     Fix notation as above. Let $\omega\colon\FF_q^\times\to\CC^\times$ be a character, and let $\chi\colon\FF_q^\times\to\CC^\times$ be the nontrivial quadratic character. For any $k,\ell\in\{\pm1\}$, we have
% %     \begin{equation}
% %         c_{k,\ell}(\omega)=\frac{g_n\left(\omega^{-1},\psi,1\right)+k\ell g_n\left(\omega^{-1}\chi,\psi,1\right)}2. \label{eq:coef-by-sym-gauss-sum}
% %     \end{equation}
% %     In particular, we have the following.
% %     \begin{itemize}
% %         \item If $n=2m$ is an even nonnegative integer, then
% %         \[c_{k,\ell}(\omega)=\chi(-1)^m\omega(4)^mq^{m^2}g\left(\omega^{-2},\psi\right)^m1_{k=\ell}.\]
% %         \item If $n=2m+1$ is an odd nonnegative integer, then
% %         \[c_{k,\ell}(\omega)=\omega(4)^mq^{m(m+1)}g\left(\omega^{-2},\psi\right)^m\cdot\frac{g\left(\omega^{-1},\psi\right)+k\ell g\left(\omega^{-1}\chi,\psi\right)}2.\]
% %     \end{itemize}
% % \end{thm}
% % \begin{proof}
% %     The last computations follow from directly from plugging \eqref{eq:coef-by-sym-gauss-sum} into \Cref{thm:sym-gauss-sum}, so we will spend our time proving \eqref{eq:coef-by-sym-gauss-sum}. Using \Cref{rem:explicit-prehom-mult-two}, we see
% %     \[\gamma_{k,\ell}(\omega)=\frac{Z_k\left(\omega^{-1}\mathcal F_\psi1_{A_\ell}\right)}{\omega(\det A_\ell)}=\frac1{\omega(\det A_\ell)}\sum_{A\in Y_k}\omega^{-1}(\det A)\mathcal F_\psi 1_{A_\ell}(A),\]
% %     where $A_\ell\in Y_\ell$ is some representative. A direct computation shows $\mathcal F_\psi 1_{A_\ell}(A)=\psi(\tr AA_\ell)$, so
% %     \[\gamma_{k,\ell}(\omega)=\frac1{\omega(\det A_\ell)}\sum_{A\in Y_k}\omega^{-1}(\det A)\psi(\tr AA_\ell).\]
% %     To express this in terms of $g_n$s, we need to change the sum from over $A\in Y_k$ to over $A\in\Sym_n^\times(\FF_q)$. To this end, we note that $A\in Y_k$ if and only if $\chi(\det A)=k$ and is $-k$ otherwise, so a direct computation shows that $1_{Y_k}=\frac12(1+k\chi\circ\det)$. Thus,
% %     \begin{align*}
% %         \gamma_{k,\ell}(\omega) &= \frac1{\omega(\det A_\ell)}\sum_{A\in\Sym_n^\times}\omega^{-1}(\det A)\psi(\tr AA_\ell)\left(\frac{1+k\chi(\det A)}2\right) \\
% %         &= \frac{g_n\left(\omega^{-1},\psi,A_\ell\right)+kg_n\left(\omega^{-1}\chi,\psi,A_\ell\right)}{2\omega(\det A_\ell)}.
% %     \end{align*}
% %     To finish up, we note that \Cref{thm:sym-gauss-sum} implies that $g_n\left(\omega^{-1},\psi,A_\ell\right)=\omega(\det A_\ell)g_n\left(\omega^{-1},\psi,1\right)$ and
% %     \[g_n\left(\omega^{-1}\chi,\psi,A_\ell\right)=\omega(\det A_\ell)\chi(\det A_\ell)g_n\left(\omega^{-1}\chi,\psi,1\right)=\omega(\det A_\ell)\ell g_n\left(\omega^{-1}\chi,\psi,1\right),\]
% %     from which substitution completes the proof.
% % \end{proof}

% % \begin{corollary}
% %     Fix notation as above. Let $\omega\colon\FF_q^\times\to\CC^\times$ be a character. The functions $\varphi\mapsto Z_\bullet\left(\omega^{-1},\mathcal F_\psi\varphi\right)$ form a basis of the space
% %     \[\Hom_{\GL_n(\FF_q)}\left(S(\Sym_n(\FF_q))^\circ,\omega^2\circ\det\right),\]
% %     where $S(\Sym_n(\FF_q))^\circ$ denotes the functionals on $\Sym_n(\FF_q)$ supported on $\Sym_n^\times(\FF_q)$.
% % \end{corollary}
% % \begin{proof}
% %     For brevity, define $Z'_\bullet(\varphi)\coloneqq Z_\bullet\left(\omega^{-1},\mathcal F_\psi\varphi\right)$. Note that $Z'_\bullet$ is in fact an eigenvector by the proof of \Cref{thm:gamma-matrix-exists}, and this space has basis given by $Z_1(\omega)$ and $Z_2(\omega)$ by \Cref{prop:prehom-mult-two}. Now, the constants $(\gamma_{k,\ell})_{k,\ell\in\{\pm1\}}$ make a change-of-basis matrix from $\{Z_1(\omega),Z_2(\omega)\}$ to $\{Z'_1,Z'_2\}$, so it suffices to show that
% %     \[\det\begin{bmatrix}
% %         \gamma_{1,1} & \gamma_{1,-1} \\
% %         \gamma_{-1,1} & \gamma_{-1,-1}
% %     \end{bmatrix}\stackrel?\ne0.\]
% %     To use \Cref{thm:gamma-matrix-computation}, we set $g_+\coloneqq g_n\left(\omega^{-1},\psi,1\right)$ and $g_-\coloneqq g_n\left(\omega^{-1}\chi,\psi,1\right)$, from which we compute
% %     \[\det\begin{bmatrix}
% %         \gamma_{1,1} & \gamma_{1,-1} \\
% %         \gamma_{-1,1} & \gamma_{-1,-1}
% %     \end{bmatrix} = \det\frac12\begin{bmatrix}
% %         g_+ + g_- & g_+ - g_- \\
% %         g_+ - g_- & g_+ + g_-
% %     \end{bmatrix} = g_+g_-.\]
% %     Now, $g_+$ and $g_-$ are nonzero by \Cref{thm:sym-gauss-sum} (and \Cref{prop:mag-gauss-sum}), so we are done.
% % \end{proof}
% % \begin{remark}
% %     Combining the above computation with \Cref{rem:mag-of-sym-gauss-sum}, in the ``generic'' case $\omega^2\ne1$, we have
% %     \[\left|\det\begin{bmatrix}
% %         \gamma_{1,1} & \gamma_{1,-1} \\
% %         \gamma_{-1,1} & \gamma_{-1,-1}
% %     \end{bmatrix}\right|=q^{\binom{n+1}2}.\]
% %     If we were to normalize $\mathcal F_\psi$ to $\mathcal F_\psi^*\coloneqq q^{-\frac12\binom{n+1}2}\mathcal F_\psi$ and redefine everything with the normalized Fourier transform, then this determinant would have absolute value $1$. This normalization factor is desirable because \Cref{rem:sym-fourier-twice} implies $\mathcal F^*_{\psi^{-1}}\circ\mathcal F^*_\psi=1$.
% % \end{remark}

% \subsection{Combinatorics}
% In this subsection, we use \Cref{thm:sym-gauss-sum} to compute the number of symmetric invertible matrices over $\FF_q$ with specified trace and determinant. This requires a more complete understanding of the sums $g_n(\omega,\psi,T)$ than \Cref{thm:sym-gauss-sum} provides; in particular, we need to understand the case when $\psi$ is trivial. Nonetheless, the method of proof \Cref{thm:sym-gauss-sum} still applies.
% \begin{prop} \label{prop:most-sym-gauss-sum-psi-trivial}
%     Fix a nonnegative integer $n$ and some $T\in\Sym_n^\times(\FF_q)$.
%     \begin{enumerate}[label=(\alph*)]
%         \item Let $\omega\colon\FF_q^\times\to\CC^\times$ be a nontrivial character. If $n$ is odd or $\omega^2\ne1$, then $g_n(\omega,1,T)=0$.
%         \item Let $\chi\colon\FF_q^\times\to\CC^\times$ be the nontrivial quadratic character. If $n=2m$ is even, then
%         \[g_{2m}(\chi,1,T)=\chi(-1)^mq^{m^2}\prod_{k=0}^{m-1}\left(q^{2k+1}-1\right).\]
%     \end{enumerate}
% \end{prop}
% \begin{proof}
%     For the proof of (a), we have two cases.
%     \begin{itemize}
%         \item Suppose $\omega^2\ne1$. Then for any $g\in\GL_n(\FF_q)$, we see that $A\in\Sym_n^\times(\FF_q)$ if and only if $gAg^\intercal\in\Sym_n^\times(\FF_q)$, so
%         \[g_n(\omega,1,T)=\sum_{A\in\Sym_n^\times(\FF_q)}\omega(\det A)=\sum_{A\in\Sym_n^\times(\FF_q)}\omega(\det gAg^\intercal)=\omega(\det g)^2g_n(\omega,1,T).\]
%         Thus, to conclude $g_n(\omega,1,T)=0$, it suffices to find $g\in\GL_n(\FF_q)$ with $\omega(\det g)^2\ne1$. Well, $\omega^2\ne1$, so find $c\in\FF_q^\times$ such that $\omega(c)^2\ne1$ and then set $g\coloneqq\operatorname{diag}(c,1,\ldots,1)$.
%         \item Suppose $n$ is odd. By the previous case, we may assume that $\omega^2=1$. Now, for any $c\in\FF_q^\times$, we see that $A\in\Sym_n^\times(\FF_q)$ if and only if $cA\in\Sym_n^\times(\FF_q)$, so
%         \[g_n(\omega,1,T)=\sum_{A\in\Sym_n^\times(\FF_q)}\omega(\det A)=\sum_{A\in\Sym_n^\times(\FF_q)}\omega(c\det A)=\omega(c)^ng_n(\omega,1,T).\]
%         Now, if we did have $g_n(\omega,1,T)\ne0$, then we would have $\omega(c)^n=1$ for all $c\in\FF_q^\times$ and hence $\omega^n=1$; however, $n$ is odd and $\omega^2=1$ already, so it would follow $\omega=1$. However, $\omega\ne1$ by hypothesis.
%     \end{itemize}
%     For the proof of (b), we imitate the proof of \Cref{thm:sym-gauss-sum}. As an analogue of \Cref{lem:sym-gauss-sum-ind}, we claim that
%     \begin{equation}
%         g_{2m+2}(\chi,1,T)\stackrel?=g_{2m}(\chi,1,T)\cdot\chi(-1)q^{2m+1}\left(q^{2m+1}-1\right) \label{eq:ind-quad-gauss-sum}
%     \end{equation}
%     for any nonnegative integer $m$. Note that \eqref{eq:ind-quad-gauss-sum} will complete the proof of (b) by an induction.
    
%     Now, the proof of \eqref{eq:ind-quad-gauss-sum} is analogous to \Cref{lem:sym-gauss-sum-ind}; there are two cases. Set $n\coloneqq2m$ for brevity.
%     \begin{itemize}
%         \item We sum over $A\in\Sym_{n+2}^\times(\FF_q)$ with $A_{n+2,n+2}\ne0$. As in \Cref{lem:sym-gauss-sum-ind}, we have the following bijection.
%         \[\arraycolsep=1.4pt\begin{array}{rclclcl}
%             \operatorname{Sym}_{n+1}^\times(\FF_q) & \times & \FF_q^{n+1} & \times & \FF_q^\times & \to & \{A\in\operatorname{Sym}_{n+2}^\times(\FF_q):A_{n+2,n+2}\ne0\} \\
%             (A' & , & v & , & c) & \mapsto & \displaystyle\begin{bmatrix}
%                 1 & v \\
%                   & 1
%             \end{bmatrix}\begin{bmatrix}
%                 A' \\ & c
%             \end{bmatrix}\begin{bmatrix}
%                 1 \\ v^\intercal & 1
%             \end{bmatrix}
%         \end{array}\]
%         It follows that
%         \[\sum_{\substack{A\in\Sym_{n+2}^\times(\FF_q)\\A_{n+2,n+2}\ne0}}\chi(\det A) = \Bigg(\sum_{A'\in\Sym_{n+1}^\times(\FF_q)}\chi(\det A)\Bigg)\Bigg(\sum_{c\in\FF_q^\times,v\in\FF_q^{n+1}}\omega(c)\Bigg),\]
%         but the left sum vanishes by (a) because it is $g_{2m+1}(\chi,1,T)=0$. Thus, there is no contribution in this case.
%         \item We sum over $A\in\Sym_{n+2}^\times(\FF_q)$ with $A_{n+2,n+2}=0$. In light of the previous case, we expect all contribution from this case. Let $e_{n+2}$ denote the $(n+2)$nd basis vector. For any $v\in\FF_q^n$ and $c\in\FF_q$, we claim that
%         \[\sum_{\substack{A\in\Sym_{n+2}^\times(\FF_q)\\ Ae_{n+2}=(v,c,0)}}\chi(\det A)\stackrel?=g_{2m}(\chi,1,T)\cdot\chi(-1)q^{2m+1},\]
%         from which the claim will follow upon summing over all vectors $(v,c)\in\FF_q^{n+1}$ with at least one nonzero entry. Quickly, because some entry in $(v,c)\in\FF_q^{n+1}$, we note that we can rearrange the rows and columns of $A$ to allow us to assume that $c\ne0$.
        
%         Thus, as in \Cref{lem:sym-gauss-sum-ind}, we have the following bijection.
%         \[\arraycolsep=1.4pt\begin{array}{rclclcl}
%             \operatorname{Sym}_{n}^\times(\FF_q) & \times & \FF_q^n & \times & \FF_q & \to & \{A\in\operatorname{Sym}_{n+2}^\times(\FF_q):Ae_{n+2}=(v,c,0)\} \\
%             (A' & , & w & , & d) & \mapsto &  \begin{bmatrix}
%                 1 & \frac1cv & w \\
%                   & 1 &   \\
%                   &   & 1
%             \end{bmatrix}\begin{bmatrix}
%                 A' \\
%                   & d & c \\
%                   & c
%             \end{bmatrix}\begin{bmatrix}
%                 1 \\
%                 \frac1cv^\intercal & 1 \\
%                 w^\intercal &   & 1
%             \end{bmatrix}
%         \end{array}\]
%         It follows that
%         \[\sum_{\substack{A\in\Sym_{n+2}^\times(\FF_q)\\ Ae_{n+2}=(v,c,0)}}\chi(\det A)=\sum_{w\in\FF_q^n,d\in\FF_q}\Bigg(\sum_{A'\in\Sym_n^\times(\FF_q)}\chi\left(-c^2\det A\right)\Bigg),\]
%         which is what we wanted upon noting $\chi\left(-c^2\right)=\chi(-1)$ and collecting sums.
%     \end{itemize}
%     Combining the above casework completes the proof of (b).
% \end{proof}
% \begin{remark}
%     One can prove (b) by a combinatorial argument, directly counting the number of invertible symmetric matrices with square determinant; this is done in \cite[Theorem~4]{macwilliams-ortho-matrices}. We have included the above proof to emphasize the strength of the Gaussian elimination technique to compute these Gauss sums.
% \end{remark}
% The last sum $g_n(\omega,\psi,T)$ to consider is the case where $\omega$ and $\psi$ are both trivial. Equivalently, we are counting the number of invertible symmetric $n\times n$ matrices with entries in $\FF_q$. This result is well-known; for example, see \cite[Theorem~2]{macwilliams-ortho-matrices}. However, to emphasize the strength of our method (and for completeness), we will present a proof using Gaussian elimination, as done in \cite{br-rand-matrices} in the case of $\FF_p$.
% \begin{prop} \label{prop:count-sym-matrices}
%     Fix a nonnegative integer $n$ and some $T\in\Sym_n^\times(\FF_q)$.
%     \begin{enumerate}[label=(\alph*)]
%         \item If $n=2m$ is even, then
%         \[g_{2m}(1,1,T)=q^{m^2+m}\prod_{k=0}^{m-1}\left(q^{2k+1}-1\right).\]
%         \item If $n=2m+1$ is odd, then
%         \[g_{2m+1}(1,1,T)=q^{m^2+m}\prod_{k=0}^{m}\left(q^{2k+1}-1\right).\]
%     \end{enumerate}
% \end{prop}
% \begin{proof}
%     The proof will be by induction on $n$. In analogy to \Cref{lem:sym-gauss-sum-ind}, the main claim is that
%     \begin{equation}
%         g_{n+2}(1,1,T)\stackrel?=q^{n+1}(q-1)g_{n+1}(1,1,T)+q^{n+1}\left(q^{n+1}-1\right)g_n(1,1,T) \label{eq:ind-counting-sym}
%     \end{equation}
%     for any nonnegative integer $n$. The proof of \Cref{eq:ind-counting-sym} uses the typical Gaussian elimination technique.
%     \begin{itemize}
%         \item We sum over $A\in\Sym_{n+2}^\times(\FF_q)$ where $A_{n+2,n+2}\ne0$. As in \Cref{lem:sym-gauss-sum-ind}, we have the following bijection.
%         \[\arraycolsep=1.4pt\begin{array}{rclclcl}
%             \operatorname{Sym}_{n+1}^\times(\FF_q) & \times & \FF_q^{n+1} & \times & \FF_q^\times & \to & \{A\in\operatorname{Sym}_{n+2}^\times(\FF_q):A_{n+2,n+2}\ne0\} \\
%             (A' & , & v & , & c) & \mapsto & \displaystyle\begin{bmatrix}
%                 1 & v \\
%                   & 1
%             \end{bmatrix}\begin{bmatrix}
%                 A' \\ & c
%             \end{bmatrix}\begin{bmatrix}
%                 1 \\ v^\intercal & 1
%             \end{bmatrix}
%         \end{array}\]
%         It follows that the number of matrices in this case is $q^{n+1}(q-1)g_{n+1}(1,1,T)$.
%         \item We sum over $A\in\Sym_{n+2}^\times(\FF_q)$ where $A_{n+2,n+2}=0$. Let $e_{n+2}$ denote the $(n+2)$nd basis vector. For any $v\in\FF_q^n$ and $c\in\FF_q$, we claim that
%         \[\#\left\{A\in\Sym_{n+2}^\times(\FF_q):Ae_{n+2}=(v,c,0)\right\}\stackrel?=q^{n+1}g_n(1,1,T),\]
%         from which \eqref{eq:ind-counting-sym} will follow by summing over all $(v,c)\in\FF_q^{n+1}$ with at least one nonzero entry. Quickly, because some entry in $(v,c)\in\FF_q^{n+1}$, we note that we can rearrange the rows and columns of $A$ to allow us to assume that $c\ne0$.
        
%         Now, as in \Cref{lem:sym-gauss-sum-ind}, we have the following bijection.
%         \[\arraycolsep=1.4pt\begin{array}{rclclcl}
%             \operatorname{Sym}_{n}^\times(\FF_q) & \times & \FF_q^n & \times & \FF_q & \to & \{A\in\operatorname{Sym}_{n+2}^\times(\FF_q):Ae_{n+2}=(v,c,0)\} \\
%             (A' & , & w & , & d) & \mapsto &  \begin{bmatrix}
%                 1 & \frac1cv & w \\
%                   & 1 &   \\
%                   &   & 1
%             \end{bmatrix}\begin{bmatrix}
%                 A' \\
%                   & d & c \\
%                   & c
%             \end{bmatrix}\begin{bmatrix}
%                 1 \\
%                 \frac1cv^\intercal & 1 \\
%                 w^\intercal &   & 1
%             \end{bmatrix}
%         \end{array}\]
%         The desired equality follows.
%     \end{itemize}
%     Summing the above cases completes the proof of \eqref{eq:ind-counting-sym}.

%     We now complete the proof by an induction on $n$. For $n=0$ and $n=1$, there is nothing to say. Now, to synthesize cases, we note that
%     \[q^{m^2+m}\prod_{k=0}^{m}\left(q^{2k+1}-1\right)=q^{\frac12(2m+1)(2m+2)}\prod_{\substack{1\le k\le2m+1\\k\text{ odd}}}\left(1-\frac1{q^k}\right)\]
%     and analogously for the even case. Thus, for our induction, we take $n\ge0$ and use \eqref{eq:ind-counting-sym} to see $g_{n+2}(1,1,T)$ is
%     \begin{align*}
%         & q^{n+1}(q-1)g_{n+1}(1,1,T)+q^{n+1}\left(q^{n+1}-1\right)g_n(1,1,T) \\
%         ={}& q^{\frac12(n+2)(n+1)}\left(q^{n+1}(q-1)\prod_{\substack{n<k\le n+1\\k\text{ odd}}}\left(1-\frac1{q^k}\right)+\left(q^{n+1}-1\right)\right)\prod_{\substack{1\le k\le n\\k\text{ odd}}}\left(1-\frac1{q^k}\right).
%     \end{align*}
%     If $n$ is odd, we have
%     \[q^{\frac12(n+2)(n+1)}\left(q^{n+2}-1\right)\prod_{\substack{1\le k\le n\\k\text{ odd}}}\left(1-\frac1{q^k}\right),\]
%     which simplifies correctly. If $n$ is even, we have
%     \[q^{\frac12(n+2)(n+1)}\underbrace{\left(q^{n+1}(q-1)\left(1-\frac1{q^{n+1}}\right)+\left(q^{n+1}-1\right)\right)}_{q\left(q^{n+1}-1\right)}\prod_{\substack{1\le k\le n\\k\text{ odd}}}\left(1-\frac1{q^k}\right),\]
%     which still simplifies correctly. This completes the induction.
% \end{proof}

% We are now ready for our combinatorics.
