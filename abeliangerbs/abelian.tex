% !TEX root = ../abeliangerbs.tex

\subsection{Tuple Data}
The results from \autoref{sec:singlevar} are very focused on single elements, which are only enough when our group $G$ is cyclic. Our goal will be to be able to cover all abelian groups, so if we want to keep track of the fact our group is abelian, we should extract the elements of $A$ which can do so.
\begin{lemma}[{\cite[Lemma~1.2]{abelian-crossed}}] \label{lem:constructalphabeta}
	Let $A$ be a $ G$-module, and let 
	\[1\to A\to\mc E\stackrel\pi\to G\to1\]
	be a group extension. Further, fix some $F_1,F_2\in\mc E$ and define $\sigma_i\coloneqq\pi(F_i)$ for $i\in\{1,2\}$, and let $\sigma_i\in G$ have order $n_i$. Then, setting
	\[\alpha_i\coloneqq F_i^{n_i}\qquad\text{and}\qquad\beta\coloneqq F_1F_2F_1^{-1}F_2^{-1},\]
	we have the following.
	\begin{listalph}
		\item $\alpha_i\in A^{\langle\sigma_i\rangle}$ for $i\in\{1,2\}$ and $\beta\in A$.
		\item $N_1(\beta)=\alpha_1/\sigma_2(\alpha_1)$ and $N_2(\beta^{-1})=\alpha_2/\sigma_1(\alpha_2)$, where $N_i\coloneqq\sum_{p=0}^{n_i-1}\sigma_i^p$.
	\end{listalph}
\end{lemma}
\begin{proof}
	These checks are a matter of force. For brevity, we set $A_i\coloneqq A^{\langle\sigma_i\rangle}$ for $i\in\{1,2\}$.
	\begin{listalph}
		\item That $\alpha_i\in A_i$ follows from \autoref{lem:constructalpha}. Lastly, $\beta\in A$ follows from noting
		\[\pi(\beta)=\pi(F_1)\pi(F_2)\pi(F_1)^{-1}\pi(F_2)^{-1}=1,\]
		so $\beta\in\ker\pi=A$.
		\item We will check that $\op N_{L/L_1}(\beta)=\alpha_1/\sigma_2(\alpha_1)$; the other equality follows symmetrically after switching $1$s and $2$s because $\beta^{-1}=F_2F_1F_2^{-1}F_1^{-1}$. Well, we compute
		\begin{align*}
			N_1(\beta) &= \sigma_1^{-1}(\beta)\cdot\sigma_1^{-2}(\beta)\cdot\sigma^{-3}\cdot\ldots\cdot\sigma^{-n_1}(\beta) \\
			&= F_1^{-1}\left(F_1F_2F_1^{-1}F_2^{-1}\right)F_1 \\
			&\phantom{{}={}}\cdot F_1^{-2}\left(F_1F_2F_1^{-1}F_2^{-1}\right)F_1^2 \\
			&\phantom{{}={}}\cdot F_1^{-3}\left(F_1F_2F_1^{-1}F_2^{-1}\right)F_1^3\cdot\ldots \\
			&\phantom{{}={}}\cdot F_1^{-n_1}(F_1F_2F_1^{-1}F_2^{-1})F_1^{n_1} \\
			% &= F_2F_1^{-1}F_2^{-1} \\
			% &\phantom{{}={}}\cdot F_2F_1^{-1}F_2^{-1} \\
			% &\phantom{{}={}}\cdot F_2F_1^{-1}F_2^{-1}\cdot\ldots \\
			% &\phantom{{}={}}\cdot F_2F_1^{-1}F_2^{-1}F_1^{n_1} \\
			&= F_2F_1^{-1} \\
			&\phantom{{}={}}\cdot F_1^{-1} \\
			&\phantom{{}={}}\cdot F_1^{-1}\cdot\ldots \\
			&\phantom{{}={}}\cdot F_1^{-1}F_2^{-1}F_1^{n_1} \\
			&= F_2F_1^{-n_1}F_2^{-1}F_1^{n_1} \\
			&= \alpha_1/\sigma_2(\alpha_1).
		\end{align*}
	\end{listalph}
	The above computations finish the proof.
\end{proof}
The proof of (b) above might appear magical, but in fact it comes from a more general idea.
\begin{lemma}[{\cite[Lemma~1.1(b)]{abelian-crossed}}] \label{lem:switchtwo}
	Fix everything as in \autoref{lem:constructalphabeta}. Then, for $x,y\ge0$, we have
	\[F_1^xF_2^y=\prod_{k=0}^{x-1}\prod_{\ell=0}^{y-1}\sigma_1^k\sigma_2^\ell(\beta)F_2^yF_1^x.\]
\end{lemma}
\begin{proof}
	We induct. We take a moment to write out the case of $x=1$, for which we induct on $y$. To be explicit, we will prove
	\[F_1F_2^y=\prod_{\ell=0}^{y-1}\sigma_2^\ell(\beta)F_2^yF_1.\]
	For $y=0$, there is nothing to say. So suppose the statement for $y$ (and $x=1$), and we show $y+1$ (and $x=1$). Well, we compute
	\begin{align*}
		F_1F_2^{y+1} &= F_1F_2^y\cdot F_2 \\
		&= \prod_{\ell=0}^{y-1}\sigma_2^\ell(\beta)F_2^yF_1\cdot F_2 \\
		&= \prod_{\ell=0}^{y-1}\sigma_2^\ell(\beta)F_2^y\beta F_2F_1 \\
		&= \prod_{\ell=0}^{y-1}\sigma_2^\ell(\beta)\cdot \sigma_2^y(\beta)F_2^y\cdot F_2F_1 \\
		&= \prod_{\ell=0}^{(y+1)-1}\sigma_2^\ell(\beta)\cdot F_2^{y+1}F_1,
	\end{align*}
	which is what we wanted.
	
	We now move on to the general case. We will induct on $y$. Note that $y=0$ makes the product empty, leaving us with $F_1^x=F_1^x$, for any $x$. So suppose that the statement is true for some $y\ge0$, and we will show $y+1$. For this, we now turn to inducting on $x$. For $x=0$, we note that the product is once again empty, so we are left with showing $F_2^{y+1}=F_2^{y+1}$, which is true.
	
	To finish, we suppose the statement for $x$ and show the statement for $x+1$. Well, we compute
	\begin{align*}
		F_1^{x+1}F_2^{y+1} &= F_1\cdot F_1^xF_2^{y+1} \\
		&= F_1\cdot \prod_{k=0}^{x-1}\prod_{\ell=0}^{(y+1)-1}\sigma_1^k\sigma_2^\ell(\beta)\cdot F_2^{y+1}F_1^x \\
		&= \sigma_1\left(\prod_{k=0}^{x-1}\prod_{\ell=0}^{(y+1)-1}\sigma_1^k\sigma_2^\ell(\beta)\right)\cdot F_1F_2^{y+1}F_1^x \\
		&= \prod_{k=1}^{(x+1)-1}\prod_{\ell=0}^{(y+1)-1}\sigma_1^k\sigma_2^\ell(\beta)\cdot F_1F_2^{y+1}F_1^x \\
		&= \prod_{k=1}^{(x+1)-1}\prod_{\ell=0}^{(y+1)-1}\sigma_1^k\sigma_2^\ell(\beta)\cdot \prod_{\ell=0}^{(y+1)-1}\sigma_2^\ell(\beta)\cdot \sigma_2^y(\beta)\cdot F_2^{y+1}F_1\cdot F_1^x \\
		&= \prod_{k=0}^{(x+1)-1}\prod_{\ell=0}^{(y+1)-1}\sigma_1^k\sigma_2^\ell(\beta)F_2^{y+1}F_1^{x+1},
	\end{align*}
	which is what we wanted.
\end{proof}
\begin{remark} \label{rem:alphabetarelation}
	Setting $x=n_1$ and $y=1$ recovers $\op N_{L/L^{\langle\sigma_1\rangle}}(\beta)=\alpha_1/\sigma_2(\alpha_1)$.
\end{remark}
In particular, \autoref{rem:alphabetarelation} tells us that coherence of the group law in $\mc E$ should give rise to relations between our elements of $A$. Here is a more complex example.
\begin{lemma}[{\cite[Lemma~1.2]{abelian-crossed}}] \label{lem:betarelations}
	Let $A$ be a $ G$-module, and let 
	\[1\to A\to\mc E\stackrel\pi\to G\to1\]
	be a group extension. Further, fix some $F_1,F_2,F_3\in\mc E$ and define $\sigma_i\coloneqq\pi(F_i)$ for $i\in\{1,2,3\}$, and let $\sigma_i\in G$ have order $n_i$. Then, setting
	\[\beta_{ij}\coloneqq F_iF_jF_i^{-1}F_j^{-1}\]
	for each pair of indices $(i,j)$ with $i>j$. Then
	\[\frac{\sigma_2(\beta_{31})}{\beta_{31}}=\frac{\sigma_1(\beta_{32})}{\beta_{32}}\cdot\frac{\sigma_3(\beta_{21})}{\beta_{21}}.\]
\end{lemma}
\begin{proof}
	The point is to turn $F_3F_2F_1$ into $F_1F_2F_3$ in two different ways. On one hand,
	\begin{align*}
		(F_3F_2)F_1 &= \beta_{32}F_2F_3F_1 \\
		&= \beta_{32}F_2\beta_{31}F_1F_3 \\
		&= \beta_{32}\sigma_2(\beta_{31})(F_2F_1)F_3 \\
		&= \beta_{32}\sigma_2(\beta_{31})\beta_{21}F_1F_2F_3.
	\end{align*}
	On the other hand,
	\begin{align*}
		F_3(F_2F_1) &= F_3\beta_{21}F_1F_2 \\
		&= \sigma_3(\beta_{21})(F_3F_1)F_2 \\
		&= \sigma_3(\beta_{21})\beta_{31}F_1(F_3F_2) \\
		&= \sigma_3(\beta_{21})\beta_{31}F_1\beta_{32}F_2F_3 \\
		&= \sigma_3(\beta_{21})\beta_{31}\sigma_1(\beta_{32})F_1F_2F_3.
	\end{align*}
	Thus,
	\[\beta_{32}\sigma_2(\beta_{31})\beta_{21}=\sigma_3(\beta_{21})\beta_{31}\sigma_1(\beta_{32}),\]
	which rearranges into the desired equation.
\end{proof}
\begin{remark}
	The relation from \autoref{lem:betarelations} may look asymmetric in the $\beta_{ij}$, but this is because the definitions of the $\beta_{ij}$s themselves are asymmetric in $F_i$.
\end{remark}
\begin{remark}
	So far we have mostly been able to recover the results from \autoref{sec:singlevar} beyond working with just a single $\sigma\in G$ and $\alpha$s to be able to work with lots of group elements in $G$ and some $\beta$s. We have not built an analogue for \autoref{cor:alphaiscupproduct}, though we will explain that it is possible to do so much later in \autoref{rem:betaiscup}.
\end{remark}

\subsection{Tuples to Cocycles}
\subsubsection{The Set-Up}
The proceeding lemmas \autoref{lem:constructalphabeta} and \autoref{lem:betarelations} is intended to give intuition that the element $\beta$ is helping to specify the group law on $\mc E$.

More concretely, we will take the following set-up for the following results: fix a $ G$-module $A$, and let
\[1\to A\to\mc E\to G\to1\]
be a group extension. Once we choose elements $\{\sigma_i\}_{i=1}^m$ generating $ G$, we know by \autoref{prop:liftsgenerate} that we can generate $\mc E$ by $A$ and some arbitrarily chosen lifts $\{F_i\}_{i=1}^m$ of the $\{\sigma_i\}_{i=1}^m$. Then, letting $n_i$ be the order of $\sigma_i$, we set
\[\alpha_i\coloneqq F_i^{n_i}\]
for each index $i$ and
\[\beta_{ij}\coloneqq F_iF_jF_i^{-1}F_j^{-1}\]
for each index $1\le j,i\le m$. Notably, it suffices to only work with $j<i$: indeed, $\beta_{ii}=1$ and $\beta_{ij}=\beta_{ji}^{-1}$ for any $i$ and $j$. Setting $A_i\coloneqq A^{\langle\sigma_i\rangle}$ and $N_i\coloneqq\sum_{p=0}^{n_i-1}\sigma_i^p$, the story so far is that
\begin{equation}
	\alpha_i\in A_i\text{ for each }i\qquad\text{and}\qquad\beta_{ij}\in A\text{ for each }i>j \label{eq:tuplefields}
\end{equation}
and
\begin{equation}
	N_i(\beta_{ij})=\alpha_i/\sigma_j(\alpha_i)\qquad\text{and}\qquad N_j(\beta_{ij}^{-1})=\alpha_j/\sigma_i(\alpha_j)\qquad\text{ for each }i>j \label{eq:tuplerelations}
\end{equation}
by \autoref{lem:constructalphabeta}, and
\begin{equation}
	\frac{\sigma_j(\beta_{ik})}{\beta_{ik}}=\frac{\sigma_k(\beta_{ij})}{\beta_{ij}}\cdot\frac{\sigma_i(\beta_{jk})}{\beta_{jk}}\qquad\text{ for each }i>j>k \label{eq:betarelations}
\end{equation}
by \autoref{lem:betarelations}. This data is so important that we will give it a name.
\begin{definition}
	In the above set-up, the data of $(\{\alpha_i\},\{\beta_{ij}\})$ satisfying \autoref{eq:tuplefields} and \autoref{eq:tuplerelations} and \autoref{eq:betarelations} will be called a \textit{$\{\sigma_i\}_{i=1}^m$-tuple}. When understood, the $\{\sigma_i\}_{i=1}^m$ will be abbreviated. Once $G$ and $A$ are fixed, we will denote the set of $\{\sigma_i\}_{i=1}^m$-tuples by $\mathcal T(G,A)$.
\end{definition}
Note that this definition is independent of $\mc E$, but a choice of extension $\mc E$ and lifts $F_i$ give a $\{\sigma_i\}_{i=1}^m$-tuple as described above.
\begin{remark}
	The $\mathcal T(G,A)$ form a group under multiplication in $A$. Indeed, the conditions \autoref{eq:tuplefields} and \autoref{eq:tuplerelations} and \autoref{eq:betarelations} are closed under multiplication and inversion.
\end{remark}
We also know from \autoref{lem:switchtwo} that
\[F_i^xF_j^y=\prod_{k=0}^{x-1}\prod_{\ell=0}^{y-1}\sigma_i^k\sigma_j^\ell(\beta_{ij})F_j^yF_i^x\]
for $i>j$ and $x,y\ge0$. It will be helpful to have some notation for the residue term in $A$, so we define
\[\sigma^{(x)}\coloneqq\sum_{i=0}^{x-1}\sigma^i\]
so that we can write
\[\sigma_i^{(x)}\sigma_j^{(y)}\beta_{ij}=\prod_{k=0}^{x-1}\prod_{\ell=0}^{y-1}\sigma_i^k\sigma_j^\ell(\beta_{ij}).\]
Now, combined with the fact that $F_ix=\sigma_i(x)F_i$ for each $F_i$ and $x\in A$, we have been approximately told how the group operation works in $\mc E$. Namely, we could conceivably write any element of $\mc E$ in the form
\[xF_1^{a_1}\cdots F_m^{a_m}\]
for $x\in A$ and $a_i\in\ZZ/n_i\ZZ$ because we know how to make these elements commute and generate $\mc E$. Further, we can multiply out two terms of the form
\[xF_1^{a_1}\cdots F_m^{a_m}\cdot yF_1^{b_1}\cdots F_m^{b_m}\]
into a term of the form $zF_1^{c_1}\cdots F_m^{c_m}$. In fact, it will be helpful for us to see how to do this.
\begin{proposition} \label{prop:multiplytwoelements}
	Fix everything as in the set-up, except drop the assumption that $\{\sigma_i\}_{i=1}^m$ generate $ G$. Then, choosing $a_i,b_i\in\NN$ for each $i$, we have
	\[\left(\prod_{i=1}^mF_i^{a_i}\right)\left(\prod_{i=1}^mF_i^{b_i}\right)=\left[\prod_{1\le j<i\le m}\Bigg(\prod_{1\le k<j}\sigma_k^{a_k+b_k}\Bigg)\Bigg(\prod_{j\le k<i}\sigma_k^{a_k}\Bigg)\sigma_i^{(a_i)}\sigma_j^{(b_j)}\beta_{ij}\right]\left(\prod_{i=1}^mF_i^{a_i+b_i}\right).\]
\end{proposition}
\begin{proof}
	The reason that we dropped the assumption on $\{\sigma_i\}_{i=1}^m$ is so that we may induct directly on $m$. We start by showing that
	\[\left(\prod_{i=1}^mF_i^{a_i}\right)F_1^{b_1}=\left[\prod_{1<i\le m}\left(\prod_{1\le k<i}\sigma_k^{a_k}\right)\sigma_i^{(a_i)}\sigma_1^{(b_1)}\beta_{i1}\right]F_1^{a_1+b_1}\prod_{i=2}^mF_i^{a_i}.\]
	We do this by induction on $m$. When $m=0$ and even for $m=1$, there is nothing to say. For the inductive step, we assume
	\[\left(\prod_{i=1}^mF_i^{a_i}\right)F_1^{b_1}=\left[\prod_{1<i\le m}\left(\prod_{1\le k<i}\sigma_k^{a_k}\right)\sigma_i^{(a_i)}\sigma_1^{(b_1)}\beta_{i1}\right]F_1^{a_1+b_1}\prod_{i=2}^mF_i^{a_i}\]
	and compute
	\begin{align*}
		\left(\prod_{i=1}^{m+1}F_i^{a_i}\right)F_1^{b_1} &= \left(\prod_{i=1}^{m}F_i^{a_i}\right)F_{m+1}^{a_{m+1}}F_1^{b_1} \\
		&= \left(\prod_{i=1}^{m}F_i^{a_i}\right)\sigma_{m+1}^{(a_{m+1})}\sigma_1^{(b_1)}\beta_{m+1,1}F_1^{b_1}F_{m+1}^{a_{m+1}} \\
		&= \left[\left(\prod_{k=1}^m\sigma_k^{a_k}\right)\sigma_{m+1}^{(a_{m+1})}\sigma_1^{(b_1)}\beta_{m+1,1}\right]\left[\prod_{1<i\le m}\bigg(\prod_{1\le k<i}\sigma_k^{a_k}\bigg)\sigma_i^{(a_i)}\sigma_1^{(b_1)}\beta_{i1}\right]\cdot{} \\
		&\qquad F_1^{a_1+b_1}\left(\prod_{i=2}^mF_i^{a_i}\right)F_{m+1}^{a_{m+1}} \\
		&= \left[\prod_{1<i\le m+1}\left(\prod_{1\le k<i}\sigma_k^{a_k}\right)\sigma_i^{(a_i)}\sigma_1^{(b_1)}\beta_{i1}\right]F_1^{a_1+b_1}\left(\prod_{i=2}^{m+1}F_i^{a_i}\right),
	\end{align*}
	which completes our inductive step.

	We now attack the statement of the proposition directly, again inducting on $m$. For $m=0$ and even for $m=1$, there is again nothing to say. For the inductive step, take $m>1$, and we get to assume that
	\[\left(\prod_{i=2}^mF_i^{a_i}\right)\left(\prod_{i=2}^mF_i^{b_i}\right)=\left[\prod_{2\le j<i\le m}\Bigg(\prod_{2\le k<j}\sigma_k^{a_k+b_k}\Bigg)\Bigg(\prod_{j\le k<i}\sigma_k^{a_k}\Bigg)\sigma_i^{(a_i)}\sigma_j^{(b_j)}\beta_{ij}\right]\left(\prod_{i=2}^mF_i^{a_i+b_i}\right).\]
	From here, we can compute
	\begin{align*}
		\left(\prod_{i=1}^mF_i^{a_i}\right)\left(\prod_{i=1}^mF_i^{b_i}\right) &= \left(\prod_{i=1}^mF_i^{a_i}\right)F_1^{b_1}\left(\prod_{i=2}^mF_i^{b_i}\right) \\
		&= \left[\prod_{1<i\le m}\Bigg(\prod_{1\le k<i}\sigma_k^{a_k}\Bigg)\sigma_i^{(a_i)}\sigma_1^{(b_1)}\beta_{i1}\right]F_1^{a_1+b_1}\left(\prod_{i=2}^mF_i^{a_i}\right)\left(\prod_{i=2}^mF_i^{b_i}\right) \\
		&= \left[\prod_{1<i\le m}\Bigg(\prod_{1\le k<i}\sigma_k^{a_k}\Bigg)\sigma_i^{(a_i)}\sigma_1^{(b_1)}\beta_{i1}\right]F_1^{a_1+b_1}\cdot{} \\
		&\qquad\qquad\left[\prod_{2\le j<i\le m}\Bigg(\prod_{2\le k<j}\sigma_k^{a_k+b_k}\Bigg)\Bigg(\prod_{j\le k<i}\sigma_k^{a_k}\Bigg)\sigma_i^{(a_i)}\sigma_j^{(b_j)}\beta_{ij}\right]\left(\prod_{i=2}^mF_i^{a_i+b_i}\right) \\
		&= \left[\prod_{1<i\le m}\Bigg(\prod_{1\le k<i}\sigma_k^{a_k}\Bigg)\sigma_i^{(a_i)}\sigma_1^{(b_1)}\beta_{i1}\right]\cdot{} \\
		&\qquad\qquad \sigma_1^{a_1+b_1}\left[\prod_{2\le j<i\le m}\Bigg(\prod_{2\le k<j}\sigma_k^{a_k+b_k}\Bigg)\Bigg(\prod_{j\le k<i}\sigma_k^{a_k}\Bigg)\sigma_i^{(a_i)}\sigma_j^{(b_j)}\beta_{ij}\right]\left(\prod_{i=2}^mF_i^{a_i+b_i}\right).
	\end{align*}
	From here, a little rearrangement finishes the inductive step.
\end{proof}
The reason we exerted this pain upon ourselves is for the following result.
\begin{prop} \label{prop:writedowncocycle}
	Fix everything as in the set-up. Then, if well-defined, we can represent the cohomology class corresponding to $\mc E$ by the cocycle
	\[c(g,h)\coloneqq\left[\prod_{1\le j<i\le m}\Bigg(\prod_{1\le k<j}\sigma_k^{a_k+b_k}\Bigg)\Bigg(\prod_{j\le k<i}\sigma_k^{a_k}\Bigg)\sigma_i^{(a_i)}\sigma_j^{(b_j)}\beta_{ij}\right]\left[\prod_{i=1}^m\Bigg(\prod_{1\le k<i}\sigma_k^{a_k+b_k}\Bigg)\alpha_i^{\floor{\frac{a_i+b_i}{n_i}}}\right],\]
	where $g=\prod_i\sigma_i^{a_i}$ and $h=\prod_i\sigma_i^{b_i}$.
\end{prop}
Observe that \autoref{prop:writedowncocycle} has a fairly strong hypothesis that $c$ is well-defined; we will return to this later.
\begin{proof}
	Very quickly, we use the division algorithm to define
	\[a_i+b_i=n_iq_i+r_i\]
	where $q_i\in\{0,1\}$ and $0\le r_i<n_i$. In particular,
	\[gh=\prod_{i=1}^mF_i^{r_i}.\]
	Now, because the elements $\sigma_i$ generate $ G$, we see that the lifts $\sigma_i\mapsto F_i$ defines a section $s\colon G\to\mc E$. As such, we can compute a representing cocycle for our cohomology class as
	\begin{align*}
		c(g,h) &= s(g)s(h)s(gh)^{-1} \\
		&= \Bigg(\prod_{i=1}^mF_i^{a_i}\Bigg)\Bigg(\prod_{i=1}^mF_i^{b_i}\Bigg)\Bigg(\prod_{i=1}^mF_i^{r_i}\Bigg)^{-1} \\
		&= \left[\prod_{1\le j<i\le m}\Bigg(\prod_{1\le k<j}\sigma_k^{a_k+b_k}\Bigg)\Bigg(\prod_{j\le k<i}\sigma_k^{a_k}\Bigg)\sigma_i^{(a_i)}\sigma_j^{(b_j)}\beta_{ij}\right]\left(\prod_{i=1}^mF_i^{a_i+b_i}\right)\Bigg(\prod_{i=1}^mF_{m-i+1}^{-r_{m-i+1}}\Bigg).
	\end{align*}
	It remains to deal with the last products; we claim that it is equal to
	\[\left(\prod_{i=1}^mF_i^{a_i+b_i}\right)\Bigg(\prod_{i=1}^mF_{m-i+1}^{-r_{m-i+1}}\Bigg)=\prod_{i=1}^m\Bigg(\prod_{1\le k<i}\sigma_k^{a_k+b_k}\Bigg)\alpha_i^{q_i},\]
	which will finish the proof. We induct on $m$; for $m=0$ and $m=1$, there is nothing to say. For the inductive step, we assume that
	\[\left(\prod_{i=2}^mF_i^{a_i+b_i}\right)\Bigg(\prod_{i=1}^{m-1}F_{m-i+1}^{-r_{m-i+1}}\Bigg)=\prod_{i=2}^m\Bigg(\prod_{2\le k<i}\sigma_k^{a_k+b_k}\Bigg)\alpha_i^{q_i}\]
	and compute
	\begin{align*}
		\left(\prod_{i=1}^mF_i^{a_i+b_i}\right)\Bigg(\prod_{i=1}^mF_{m-i+1}^{-r_{m-i+1}}\Bigg) &= F_1^{a_1+b_1}\left(\prod_{i=2}^mF_i^{a_i+b_i}\right)\Bigg(\prod_{i=1}^{m-1}F_{m-i+1}^{-r_{m-i+1}}\Bigg)F_1^{-a_1-b_1}F_1^{a_1+b_1-r_1} \\
		&= F_1^{a_1+b_1}\left(\prod_{i=2}^m\Bigg(\prod_{2\le k<i}\sigma_k^{a_k+b_k}\Bigg)\alpha_i^{q_i}\right)F_1^{-a_1-b_1}\alpha_1^{q_1} \\
		&= \left(\prod_{i=2}^m\Bigg(\prod_{1\le k<i}\sigma_k^{a_k+b_k}\Bigg)\alpha_i^{q_i}\right)\alpha_1^{q_1} \\
		&= \prod_{i=1}^m\Bigg(\prod_{1\le k<i}\sigma_k^{a_k+b_k}\Bigg)\alpha_i^{q_i},
	\end{align*}
	finishing.
\end{proof}

\subsubsection{The Modified Set-Up}
A priori we have no reason to expect that the $c$ constructed in \autoref{prop:writedowncocycle} is actually a cocycle, especially if the $\sigma_i$ have nontrivial relations.

To account for this, we modify our set-up slightly. By the classification of finitely generated abelian groups, we may write
\[ G\simeq\bigoplus_{k=1}^m G_k,\]
where $ G_k\subseteq G$ with $ G_k\cong\ZZ/n_k\ZZ$ and $n_k>1$ for each $n_k$. As such, we let $\sigma_k$ be a generating element of $ G_k$ so that we still know that the $\sigma_k$ generate $ G$. In this case, we have the following result.
\begin{theorem}[{\cite[Theorem~1.3]{abelian-crossed}}] \label{thm:getcocycle}
	Fix everything as in the modified set-up, forgetting about the extension $\mc E$. Then a $\{\sigma_i\}_{i=1}^m$-tuple of $\{\alpha_i\}_{i=1}^m$ and $\{\beta_{ij}\}_{i>j}$ makes
	\[c(g,h)\coloneqq\left[\prod_{1\le j<i\le m}\Bigg(\prod_{1\le k<j}\sigma_k^{a_k+b_k}\Bigg)\Bigg(\prod_{j\le k<i}\sigma_k^{a_k}\Bigg)\sigma_i^{(a_i)}\sigma_j^{(b_j)}\beta_{ij}\right]\left[\prod_{i=1}^m\Bigg(\prod_{1\le k<i}\sigma_k^{a_k+b_k}\Bigg)\alpha_i^{\floor{\frac{a_i+b_i}{n_i}}}\right],\]
	where $g\coloneqq\prod_i\sigma_i^{a_i}$ with $h\coloneqq\prod_i\sigma_j^{a_j}$ and $0\le a_i,b_i<n_i$, into a cocycle in $Z^2( G,A)$.
\end{theorem}
\begin{proof}
	Note that $c$ is now surely well-defined because the elements $g$ and $h$ have unique representations as described. Anyway, we relegate the direct cocycle check to \autoref{sec:verifycocycle} because it is long, annoying, and unenlightening. We will also present an alternative proof in \autoref{thm:yesitisacocycle}, using more abstract theory.
\end{proof}
Observe that the above construction has now completely forgotten about $\mc E$! Namely, we have managed to go from tuples straight to cocycles; this is theoretically good because it will allow us to go fully in reverse: we will be able to start with a tuple, build the corresponding cocycle, from which the extension arises. However, equivalence classes of cocycles give the ``same'' extension, so we will also need to give equivalence classes for tuples as well.

\subsection{Building Tuples}
We continue in the modified set-up of the previous section. There is already an established way to get from a cocycle to an extension, which means that it should be possible to go straight from the cocycle to a $\{\sigma_i\}_{i=1}^m$-tuple. Again, it will be beneficial to write this out.
\begin{lemma} \label{lem:explicitalphabeta}
	Fix everything as in the modified set-up, but suppose that $\mc E=\mc E_c$ is the extension generated from a cocycle $c\in Z^2( G,A)$. Then, if $F_i=(x_i,\sigma_i)$ are our lifts, we have
	\[\alpha_i=N_i(x_i)\cdot\prod_{k=0}^{n_i-1}c\left(\sigma_i^k,\sigma_i\right)\qquad\text{and}\qquad\beta_{ij}=\frac{x_i}{\sigma_j(x_i)}\cdot\frac{\sigma_i(x_j)}{x_j}\cdot\frac{c(\sigma_i,\sigma_j)}{c(\sigma_j,\sigma_i)}\]
	for each $\alpha_i$ and $\beta_{ij}$.
\end{lemma}
\begin{proof}
	The equality for the $\alpha_i$ follow from \autoref{lem:explicitalpha}. For the equality about $\beta_{ij}$, we simply compute
	by brute force, writing
    \begin{align*}
        F_iF_j &= (x_i\cdot\sigma_ix_j\cdot c(\sigma_i,\sigma_j),\sigma_i\sigma_j) \\
        F_jF_i &= (x_j\cdot\sigma_jx_i\cdot c(\sigma_j,\sigma_i),\sigma_j\sigma_i) \\
        (F_jF_i)^{-1} &= \left((\sigma_j\sigma_i)^{-1}(x_j\cdot\sigma_jx_i\cdot c(\sigma_j,\sigma_i))^{-1},\sigma_i^{-1}\sigma_j^{-1}\right),
    \end{align*}
    which gives
    \begin{align*}
        \beta_{ij} &= (F_iF_j)(F_jF_i)^{-1} \\
        &= \left(\frac{x_i}{\sigma_jx_i}\cdot\frac{\sigma_ix_j}{x_j}\cdot\frac{c(\sigma_i,\sigma_j)}{c(\sigma_j,\sigma_i)},1\right),
    \end{align*}
	finishing.
\end{proof}
Here is a nice sanity check that we are doing things in the right setting: not only can we build tuples from extensions, but we can find an extension corresponding to any tuple.
\begin{cor} \label{cor:alltuplesfromextens}
	Fix everything as in the modified set-up, forgetting about the extension $\mc E$. For any $\{\sigma_i\}_{i=1}^m$-tuple of $\{\alpha_i\}_{i=1}^m$ and $\{\beta_{ij}\}_{i>j}$, there exists an extension $\mc E$ and lifts $F_i$ of the $\sigma_i$ so that
	\[\alpha_i=F_i^{n_i}\qquad\text{and}\qquad\beta_{ij}=F_iF_jF_i^{-1}F_j^{-1}.\]
\end{cor}
\begin{proof}
	From \autoref{thm:getcocycle}, we may build the cocycle $c\in Z^2( G,A)$ defined by
	\begin{equation}
		c(g,h)\coloneqq\left[\prod_{1\le j<i\le m}\Bigg(\prod_{1\le k<j}\sigma_k^{a_k+b_k}\Bigg)\Bigg(\prod_{j\le k<i}\sigma_k^{a_k}\Bigg)\sigma_i^{(a_i)}\sigma_j^{(b_j)}\beta_{ij}\right]\left[\prod_{i=1}^m\Bigg(\prod_{1\le k<i}\sigma_k^{a_k+b_k}\Bigg)\alpha_i^{\floor{\frac{a_i+b_i}{n_i}}}\right], \label{eq:uglycocycle}
	\end{equation}
	where $g\coloneqq\prod_iF_i^{a_i}$ and $h\coloneqq\prod_iF_j^{a_j}$ and $0\le a_i,b_i<n_i$. As such, we use $\mc E\coloneqq\mc E_c$ to be the corresponding extension and $F_i\coloneqq(1,\sigma_i)$ as our lifts. We have the following checks.
	\begin{itemize}
		\item To show $\alpha_i=F_i^{n_i}$, we use \autoref{lem:explicitalphabeta} to compute $F_i^{n_i}$, which means we want to compute
		\[\prod_{k=0}^{n_i-1}c\left(\sigma_i^k,\sigma_i\right).\]
		Well, plugging $c\left(\sigma_i^k,\sigma_i\right)$ into \autoref{eq:uglycocycle}, we note that all $\beta_{k\ell}^{(a_kb_\ell)}$ terms vanish (either $a_k=0$ or $b_\ell=0$ for each $k\ne\ell$), so the big left product completely vanishes.
		
		As for the right product, the only term we have to worry about is
		\[\Bigg(\prod_{1\le k<i}\sigma_k^{0+0}\Bigg)\alpha_i^{\floor{\frac{k+1}{n_i}}},\]
		which is equal to $1$ when $k\le n_i-1$ and $\alpha_i$ when $k=n_i-1$. As such, we do indeed have $\alpha_i=F_i^{n_i}$.
		\item To show $\beta_{ij}=F_iF_jF_i^{-1}F_j^{-1}$ for $i>j$, we again use \autoref{lem:explicitalphabeta} to compute $F_iF_jF_i^{-1}F_j^{-1}$, which means we want to compute
		\[\frac{c(\sigma_i,\sigma_j)}{c(\sigma_j,\sigma_i)}.\]
		Plugging into \autoref{eq:uglycocycle} once more, there is no way to make $\floor{(a_k+b_k)/n_k}$ nonzero (recall we set $n_k>1$ for each $k$) in either $c(\sigma_i,\sigma_j)$ or $c(\sigma_j,\sigma_i)$. As such, the right-hand product term disappears.

		As for the left product, we note that it still vanishes for $c(\sigma_j,\sigma_i)$ because $i>j$ implies that either $a_k=0$ or $b_\ell=0$ for each $k>\ell$. However, for $c(\sigma_i,\sigma_j)$, we do have $a_i=1$ and $b_j=1$ only, so we have to deal with exactly the term
		\[\Bigg(\prod_{1\le k<j}\sigma_k^{a_k+b_k}\Bigg)\Bigg(\prod_{j\le k<i}\sigma_k^{a_k}\Bigg)\beta_{ij}.\]
		With $i>j$ and $a_k=b_k=0$ for $k\notin\{i,j\}$, we see that the product of all the $\sigma_k$s will disappear, indeed only leaving us with $\beta_{ij}$.
	\end{itemize}
	The above computations complete the proof.
\end{proof}
And here is our first taste of (partial) classification.
\begin{cor} \label{cor:cocycletuplesection}
	Fix everything as in the modified set-up, forgetting about the extension $\mc E$. Then the formula of \autoref{thm:getcocycle} and the formulae of \autoref{lem:explicitalphabeta} (setting $x_i=1$ for each $i$) are homomorphisms of abelian groups between tuples in $\mathcal T(G,A)$ and cocycles in $Z^2( G,A)$. In fact, the formula of \autoref{thm:getcocycle} is a section of the formulae of \autoref{lem:explicitalphabeta}.
\end{cor}
\begin{proof}
	The formulae in \autoref{thm:getcocycle} and \autoref{lem:explicitalphabeta} are both large products in their inputs, so they are multiplicative (i.e., homomorphisms). It remains to check that we have a section. Well, starting with a $\{\sigma_i\}_{i=1}^m$-tuple and building the corresponding cocycle $c$ by \autoref{thm:getcocycle}, the proof of \autoref{cor:alltuplesfromextens} shows that the formulae of \autoref{lem:explicitalphabeta} recovers the correct $\{\sigma_i\}_{i=1}^m$-tuple.
\end{proof}

\subsection{Equivalence Classes of Tuples}
We continue in the modified set-up. We would like to make \autoref{cor:cocycletuplesection} into a proper isomorphism of abelian groups, but this is not feasible; for example, the cocycle $c$ generated by \autoref{thm:getcocycle} will always have $c(\sigma_j,\sigma_i)=1$ for $i>j$, which is not true of all cocycles in $Z^2( G,A)$.

However, we did have a notion that the data of a $\{\sigma_i\}_{i=1}^m$ should be enough to specify the group law of the extension that the tuple comes from, so we do expect to be able to define all extensions---and hence achieve all cohomology classes---from a specially chosen $\{\sigma_i\}_{i=1}^m$-tuple.

To make this precise, we want to define an equivalence relation on tuples which go to the same cohomology class and then show that the map \autoref{thm:getcocycle} is surjective on these equivalence classes. The correct equivalence relation is taken from \autoref{lem:explicitalphabeta}.
\begin{definition} \label{defi:tupleequiv}
	Fix everything as in the modified set-up. We say that two $\{\sigma_i\}_{i=1}^m$-tuples $(\{\alpha_i\},\{\beta_{ij}\})$ and $(\{\alpha_i'\},\{\beta_{ij}'\})$ are \textit{equivalent} if and only if there exist elements $x_1,\ldots,x_m\in A$ such that
	\[\alpha_i=N_i(x_i)\cdot\alpha_i'\qquad\text{and}\qquad\beta_{ij}=\frac{x_i}{\sigma_j(x_i)}\cdot\frac{\sigma_i(x_j)}{x_j}\cdot\beta_{ij}'\]
	for each $\alpha_i$ and $\beta_{ij}$. We may notate this by $(\{\alpha_i\},\{\beta_{ij}\})\sim(\{\alpha_i'\},\{\beta_{ij}'\})$.
\end{definition}
\begin{remark}
	It is not too hard to see directly from the definition that this is in fact an equivalence relation. In fact, the set of tuples equivalent to the ``trivial'' tuple of all $1$s is closed under multiplication (and inversion) and hence forms a subgroup of $\mathcal T(G,A)$. As such, the set of equivalence classes forms a quotient group of $\mathcal T(G,A)$. We will denote this quotient group by $\overline{\mathcal T}(G,A)$.
\end{remark}
This notion of equivalence can be seen to be the correct one in the sense that it correctly generalizes \autoref{prop:findallalpha}.
\begin{proposition}[{\cite[Theorem~1.4]{abelian-crossed}}] \label{prop:extenmakesaclass}
	Fix everything as in the modified set-up with an extension $\mc E$. As the lifts $F_i$ change, the corresponding values of
	\[\alpha_i\coloneqq F_i^{n_i}\qquad\text{and}\qquad\beta_{ij}\coloneqq F_iF_jF_i^{-1}F_j^{-1}\]
	go through a full equivalence class of $\{\sigma_i\}_{i=1}^m$-tuples.
\end{proposition}
\begin{proof}
	We proceed as in \autoref{prop:findallalpha}. Given an extension $\mc E'$, let $S_{\mc E'}$ be the set of $\{\sigma_i\}_{i=1}^m$-tuples generated as the lifts $F_i$ change. We start by showing that an isomorphism $\varphi\colon\mc E\cong\mc E'$ of extensions implies that $S_{\mc E}=S_{\mc E'}$; by symmetry, it will be enough for $S_{\mc E}\subseteq S_{\mc E'}$. The isomorphism induces the following diagram.
	% https://q.uiver.app/?q=WzAsMTAsWzAsMCwiMSJdLFsxLDAsIkxeXFx0aW1lcyJdLFsyLDAsIlxcbWMgRSJdLFszLDAsIlxcR2FtbWEiXSxbNCwwLCIxIl0sWzAsMSwiMSJdLFsxLDEsIkxeXFx0aW1lcyJdLFsyLDEsIlxcbWMgRSciXSxbMywxLCJcXEdhbW1hIl0sWzQsMSwiMSJdLFswLDFdLFsxLDJdLFsyLDMsIlxccGkiXSxbMyw0XSxbNSw2XSxbNiw3XSxbNyw4LCJcXHBpJyJdLFs4LDldLFsyLDcsIlxcdmFycGhpIl0sWzEsNiwiIiwxLHsibGV2ZWwiOjIsInN0eWxlIjp7ImhlYWQiOnsibmFtZSI6Im5vbmUifX19XSxbMyw4LCIiLDEseyJsZXZlbCI6Miwic3R5bGUiOnsiaGVhZCI6eyJuYW1lIjoibm9uZSJ9fX1dXQ==&macro_url=https%3A%2F%2Fraw.githubusercontent.com%2FdFoiler%2Fnotes%2Fmaster%2Fnir.tex
	\[\begin{tikzcd}
		1 & {A} & {\mc E} &  G & 1 \\
		1 & {A} & {\mc E'} &  G & 1
		\arrow[from=1-1, to=1-2]
		\arrow[from=1-2, to=1-3]
		\arrow["\pi", from=1-3, to=1-4]
		\arrow[from=1-4, to=1-5]
		\arrow[from=2-1, to=2-2]
		\arrow[from=2-2, to=2-3]
		\arrow["{\pi'}", from=2-3, to=2-4]
		\arrow[from=2-4, to=2-5]
		\arrow["\varphi", from=1-3, to=2-3]
		\arrow[Rightarrow, no head, from=1-2, to=2-2]
		\arrow[Rightarrow, no head, from=1-4, to=2-4]
	\end{tikzcd}\]
	To show that $S_{\mc E}\subseteq S_{\mc E'}$, pick up some $\{\sigma_i\}_{i=1}^m$-tuple $(\{\alpha_i\},\{\beta_{ij}\})$ generated from lifts $F_i\in\mc E$ (i.e., $\pi(F_i)=\sigma_i$), where
	\[\alpha_i\coloneqq F_i^{n_i}\qquad\text{and}\qquad\beta_{ij}\coloneqq F_iF_jF_i^{-1}F_j^{-1}.\]
	Now, we note that $F_i'\coloneqq\varphi(F_i)$ will have
	\[\pi(F_i')=\pi(\varphi(F_i))=\varphi(\pi(F_i))=\sigma_i\]
	by the commutativity of the diagram, so the $F_i'$ are lifts of the $\sigma_i$. Further, we see that
	\[(F_i')^{n_i}=\varphi(F_i)^{n_i}=\varphi\left(F_i^{n_i}\right)=\varphi(\alpha_i)=\alpha_i\]
	for each $i$, and
	\[F_i'F_j'(F_i')^{-1}(F_j')^{-1}=\varphi\left(F_iF_jF_i^{-1}F_j^{-1}\right)=\varphi(\beta_{ij})=\beta_{ij}\]
	for each $i>j$. Thus, $(\{\alpha_i\},\{\beta_{ij}\})$ is a $\{\sigma_i\}_{i=1}^m$-tuple generated by lifts from $\mc E'$, implying that $(\{\alpha_i\},\{\beta_{ij}\})\in S_{\mc E'}$.

	It now suffices to show the statement in the proposition for a specific extension isomorphic to $\mc E$. Well, the isomorphism class of $\mc E$ corresponds to some cohomology class in $H^2( G,A)$, for which we let $c$ be a representative; then $\mc E\simeq\mc E_c$, so we may show the statement for $\mc E\coloneqq\mc E_c$. Indeed, as the lifts $F_i=(x_i,\sigma_i)$ change, we know by \autoref{lem:explicitalphabeta} that
	\[\alpha_i=N_i(x_i)\cdot\prod_{k=0}^{n_i-1}c\left(\sigma_i^k,\sigma_i\right)\qquad\text{and}\qquad\beta_{ij}=\frac{x_i}{\sigma_j(x_i)}\cdot\frac{\sigma_i(x_j)}{x_j}\cdot\frac{c(\sigma_i,\sigma_j)}{c(\sigma_j,\sigma_i)}\]
	for each $\alpha_i$ and $\beta_{ij}$. All of these live in the same equivalence class by definition of the equivalence, and as the $x_i$ are allowed to vary over all of $A$, they will fill up that equivalence class fully. This finishes.
\end{proof}
We are now ready to upgrade our section.
\begin{cor} \label{cor:cohomologymakesaclass}
	Fix everything as in the modified set-up, forgetting about the extension $\mc E$. Fixing a cohomology class $[c]\in H^2( G,A)$, the set of $\{\sigma_i\}_{i=1}^m$-tuples which correspond to $[c]$ (via \autoref{thm:getcocycle}) forms exactly one equivalence class.
\end{cor}
\begin{proof}
	We show that two tuples are equivalent if and only if their corresponding cocycles (via \autoref{thm:getcocycle}) to the same cohomology class, which will be enough.
	
	In one direction, suppose $(\{\alpha_i\},\{\beta_{ij}\})\sim(\{\alpha_i'\},\{\beta_{ij}'\})$. By \autoref{cor:alltuplesfromextens}, we can find an extension $\mc E$ which gives $(\{\alpha_i\},\{\beta_{ij}\})$ by choosing an appropriate set of lifts. By \autoref{prop:extenmakesaclass}, we see that $(\{\alpha_i'\},\{\beta_{ij}'\})$ must also come from choosing an appropriate set of lifts in $\mc E$. However, the cocycles in $Z^2( G,A)$ generated by \autoref{thm:getcocycle} from our two tuples now both represent the isomorphism class of $\mc E$ by \autoref{prop:writedowncocycle}, so these cocycles belong to the same cohomology class.

	In the other direction, name the cocycles corresponding to $(\{\alpha_i\},\{\beta_{ij}\})$ and $(\{\alpha_i'\},\{\beta_{ij}'\})$ by $c$ and $c'$ respectively, and suppose $[c]=[c']$. Then $\mc E_c\cong\mc E_{c'}$ as extensions, but we know by the proof of \autoref{cor:alltuplesfromextens} that $(\{\alpha_i\},\{\beta_{ij}\})$ comes from choosing lifts of $\mc E_c$ and similar for $(\{\alpha_i'\},\{\beta_{ij}'\})$. In particular, because $\mc E_c\cong\mc E_{c'}$, we know that $(\{\alpha_i'\},\{\beta_{ij}'\})$ will also come from choosing some lifts in $\mc E_c$ (recall the proof of \autoref{prop:extenmakesaclass}), so $(\{\alpha_i\},\{\beta_{ij}\})\sim(\{\alpha_i'\},\{\beta_{ij}'\})$ follows.
\end{proof}
\begin{theorem} \label{thm:classisomorphism}
	The maps described in \autoref{cor:cocycletuplesection} descend to an isomorphism of abelian groups between the equivalence classes in $\overline{\mathcal T}(G,A)$ and cohomology classes in $H^2( G,A)$.
\end{theorem}
\begin{proof}
	The fact that the maps are well-defined (in both directions) and hence injective is \autoref{cor:cohomologymakesaclass}. The fact that we had a section from tuples to cocycles implies that the map from cocycles to tuples was also surjective. Thus, we have a bona fide isomorphism.
\end{proof}

\subsection{Classification of Extensions}
We remark that we are now able to classify all extensions up to isomorphism, in some sense. At a high level, an isomorphism class of extensions corresponds to a particular cohomology class in $H^2( G,A)$, so choosing a $\{\sigma_i\}_{i=1}^m$-tuple $(\{\alpha_i\},\{\beta_{ij}\})$ corresponding to this class, we can write out a representative of this cocycle by \autoref{thm:getcocycle}, properly corresponding to the original extension by \autoref{prop:writedowncocycle}.

In fact, the cocycle in \autoref{prop:writedowncocycle} is generated by the description of the group law in \autoref{prop:multiplytwoelements}, and the entire computation only needed to use the following relations, for the appropriate choice of lifts $F_i$.
\begin{listalph}
	\item $F_ix=\sigma_i(x)F_i$ for each $i$ and $x\in A$.
	\item $F_i^{n_i}=\alpha_i$ for each $i$.
	\item $F_iF_jF_i^{-1}F_j^{-1}=\beta_{ij}$ for each $i>j$; i.e., $F_iF_j=\beta_{ij}F_jF_i$.
\end{listalph}
As such, the above relations fully describe the extension because they also specify the cocycle, and we know that this cocycle is well-defined. We summarize this discussion into the following theorem.
\begin{theorem}
	Fix everything as in the modified set-up, forgetting about the extension $\mc E$. Further, fix a $\{\sigma_i\}_{i=1}^m$-tuple $(\{\alpha_i\},\{\beta_{ij}\})$, and define the group $\mc E(\{\alpha_i\},\{\beta_{ij}\})$ as being generated by $A$ and elements $\{F_i\}_{i=1}^n$ having the following relations.
	\begin{listalph}
		\item $F_ix=\sigma_i(x)F_i$ for each $i$ and $x\in A$.
		\item $F_i^{n_i}=\alpha_i$ for each $i$.
		\item $F_iF_j=\beta_{ij}F_jF_i$ for each $i>j$.
	\end{listalph}
	Then the natural embedding $A\into\mc E(\{\alpha_i\},\{\beta_{ij}\})$ and projection $\pi\colon\mc E(\{\alpha_i\},\{\beta_{ij}\})\onto G$ by $F_i\mapsto\sigma_i$ makes $\mc E(\{\alpha_i\},\{\beta_{ij}\})$ into an extension. In fact, all extensions are isomorphic to some $\mc E(\{\alpha_i\},\{\beta_{ij}\})$.
\end{theorem}
\begin{proof}
	This follows from the preceding discussion, though we will provide a few more words in this proof. The exactness of
	\[1\to A\to\mc E(\{\alpha_i\},\{\beta_{ij}\})\stackrel\pi\to G\to1\]
	follows quickly. Further, the action of conjugation of $\mc E$ on $A$ corresponds correctly to the $ G$-action by (a). So we do indeed have an extension.

	It remains to show that all extensions are isomorphic to one of this type. Well, note that \autoref{prop:multiplytwoelements} and \autoref{prop:writedowncocycle} use only the above relations to write down a cocycle representing the isomorphism class of $\mc E(\{\alpha_i\},\{\beta_{ij}\})$, and it is the cocycle corresponding to the $\{\sigma_i\}_{i=1}^m$-tuple $(\{\alpha_i\},\{\beta_{ij}\})$ itself as described in \autoref{thm:getcocycle}.

	However, we know that as the equivalence class of $(\{\alpha_i\},\{\beta_{ij}\})$ changes, we will hit all cohomology classes in $H^2( G,A)$ by \autoref{thm:classisomorphism}. Thus, because every extension is represented by some cohomology class, every extension will be isomorphic to some $\mc E(\{\alpha_i\},\{\beta_{ij}\})$. This completes the proof.
\end{proof}

\subsection{Change of Group}
We continue in the modified set-up, but we will no longer need access to an extension $\mc E$. In this subsection, we are interested in what happens to tuples when the cocycle operations of $\op{Inf}\colon H^2\left(G/H,A^H\right)\to H^2(G,A)$ and $\op{Res}\colon H^2(G,A)\to H^2(H,A)$ are applied, where $H\subseteq G$ is some subgroup.

In general, this is difficult because the structure of a subgroup $H\subseteq G$ might not be particularly amenable to forming a tuple from a tuple in $G$. More concretely, $H$ might have generators which look very different from those of $G$. However, it will be enough for our purposes to restrict our attention to the subgroups of the form
\[H=\langle\sigma_1^{d_1},\ldots,\sigma_m^{d_m}\rangle,\]
where the $\{d_i\}_{i=1}^m$ are some positive integers with $d_i\mid n_i$ for each $i$. With that said, here are our computations. We begin with inflation.
\begin{lemma} \label{lem:tupleinflation}
	Fix everything as in the modified set-up, forgetting about the extension $\mc E$. Further, let $H\coloneqq\langle\sigma_1^{d_1},\ldots,\sigma_m^{d_m}\rangle$ be a subgroup with $d_\bullet\mid n_\bullet$, and let $\overline\sigma_i$ be the image of $\sigma_i$ in $G/H$. Consider the inflation map $\op{Inf}\colon H^2\left(G/H,A^H\right)\to H^2(G,A)$.
	
	If the cocycle $\overline c\in Z^2\left(G/H,A^H\right)$ gives the $\left\{\overline\sigma_i\right\}_{i=1}^m$-tuple $(\{\overline\alpha_i\},\{\overline\beta_{ij}\})$ (by \autoref{cor:cocycletuplesection}), then the cocycle $\op{Inf}\overline c\in Z^2(G,A)$ gives the $\{\sigma_i\}_{i=1}^m$-tuple
	\[\op{Inf}(\{\overline\alpha_i\},\{\overline\beta_{ij}\})\coloneqq(\{\alpha_i\},\{\beta_{ij}\})=\left(\left\{\overline\alpha_i^{n_i/d_i}\right\},\{\overline\beta_{ij}\}\right).\]
\end{lemma}
\begin{proof}
	The point is to use the explicit formulae for the $\alpha_i$ and $\beta_{ij}$ of \autoref{lem:explicitalphabeta}.
	
	More explicitly, the map of \autoref{cor:cocycletuplesection} tells us that we can compute the tuple for $\op{Inf}\overline c$ by using our explicit formulae for $\alpha_i$ and $\beta_{ij}$ on the $2$-cocycle $\op{Inf}\overline c\in Z^2(G,A)$. For some $\alpha_i$, the computation is
	\begin{align*}
		\alpha_i &= \prod_{k=0}^{n_i-1}(\op{Inf}\overline c)\left(\sigma_i^k,\sigma_i\right) \\
		&= \prod_{k=0}^{n_i-1}\overline c\left(\overline\sigma_i^k,\overline\sigma_i\right) \\
		&= \Bigg(\prod_{k=0}^{d_i-1}\overline c\left(\overline\sigma_i^k,\overline\sigma_i\right)\Bigg)^{n_i/d_i}
	\end{align*}
	where the last equality is because $\overline\sigma_i^{d_i}=1$ in $G/H$. In fact, $d_i$ is the order of $\overline\sigma_i$, so the product is just $\overline\alpha_i$ by \autoref{lem:explicitalphabeta} and how we defined $\overline\alpha_i$. It follows
	\[\alpha_i=\overline\alpha_i^{n_i/d_i}.\]
	Continuing, for some $\beta_{ij}$, we have
	\begin{align*}
		\beta_{ij} &= \frac{(\op{Inf}\overline c)(\sigma_i,\sigma_j)}{(\op{Inf}\overline c)(\sigma_j,\sigma_i)} \\
		&= \frac{\overline c(\overline\sigma_i,\overline\sigma_j)}{\overline c(\overline\sigma_j,\overline\sigma_i)} \\
		&= \overline\beta_{ij},
	\end{align*}
	where the last equality is by how we defined $\overline\beta_{ij}$. These computations complete the proof.
\end{proof}
\begin{remark} \label{rem:tupleinflationcommutativediagram}
	We can also the statement of \autoref{lem:tupleinflation} as asserting that the diagram
	% https://q.uiver.app/?q=WzAsNCxbMCwwLCJcXG1hdGhjYWwgVFxcbGVmdChHL0gsQV5IXFxyaWdodCkiXSxbMSwwLCJcXG1hdGhjYWwgVChHLEEpIl0sWzAsMSwiWl4yXFxsZWZ0KEcvSCxBXkhcXHJpZ2h0KSJdLFsxLDEsIlpeMihHLEEpIl0sWzAsMSwiXFxvcHtJbmZ9Il0sWzAsMiwiIiwyLHsic3R5bGUiOnsiaGVhZCI6eyJuYW1lIjoiZXBpIn19fV0sWzEsMywiIiwwLHsic3R5bGUiOnsiaGVhZCI6eyJuYW1lIjoiZXBpIn19fV0sWzIsMywiXFxvcHtJbmZ9IiwyXV0=&macro_url=https%3A%2F%2Fraw.githubusercontent.com%2FdFoiler%2Fnotes%2Fmaster%2Fnir.tex
	\[\begin{tikzcd}
		{Z^2\left(G/H,A^H\right)} & {Z^2(G,A)} \\
		{\mathcal T\left(G/H,A^H\right)} & {\mathcal T(G,A)}
		\arrow["{\op{Inf}}", from=1-1, to=1-2]
		\arrow[two heads, from=1-1, to=2-1]
		\arrow[two heads, from=1-2, to=2-2]
		\arrow["{\op{Inf}}", from=2-1, to=2-2]
	\end{tikzcd}\]
	commutes, where the vertical morphisms are from \autoref{cor:cocycletuplesection}.
\end{remark}
\begin{remark} \label{rem:inflationclasses}
	In light of the fact that the cohomology class of some $\op{Inf}\overline c\in Z^2(G,A)$ is only defined up to the cohomology class of $\overline c\in Z^2\left(G/H,A^H\right)$, changing an input tuple $(\{\overline\alpha_i\},\{\overline\beta_{ij}\})\in\mathcal T\left(G/H,A^H\right)$ up to equivalence will not change the cohomology class of the associated cocycle in $\overline c\in Z^2\left(G/H,A^H\right)$ and hence will not change the cohomology class of $\op{Inf}\overline c$ nor the equivalence class of $\op{Inf}(\{\overline\alpha_i\},\{\overline\beta_{ij}\})\in\mathcal T\left(G,A\right)$. All this is to say that we have a well-defined map
	\[\op{Inf}\colon\overline{\mathcal T}\left(G/H,A^H\right)\to\overline{\mathcal T}(G,A)\]
	and commutative diagram
	% https://q.uiver.app/?q=WzAsNCxbMCwwLCJcXG92ZXJsaW5le1xcbWF0aGNhbCBUfVxcbGVmdChHL0gsQV5IXFxyaWdodCkiXSxbMSwwLCJcXG92ZXJsaW5le1xcbWF0aGNhbCBUfShHLEEpIl0sWzAsMSwiSF4yXFxsZWZ0KEcvSCxBXkhcXHJpZ2h0KSJdLFsxLDEsIkheMihHLEEpIl0sWzAsMSwiXFxvcHtJbmZ9Il0sWzAsMiwiIiwyLHsic3R5bGUiOnsidGFpbCI6eyJuYW1lIjoiaG9vayIsInNpZGUiOiJ0b3AifSwiaGVhZCI6eyJuYW1lIjoiZXBpIn19fV0sWzEsMywiIiwwLHsic3R5bGUiOnsidGFpbCI6eyJuYW1lIjoiaG9vayIsInNpZGUiOiJ0b3AifSwiaGVhZCI6eyJuYW1lIjoiZXBpIn19fV0sWzIsMywiXFxvcHtJbmZ9IiwyXV0=&macro_url=https%3A%2F%2Fraw.githubusercontent.com%2FdFoiler%2Fnotes%2Fmaster%2Fnir.tex
	\[\begin{tikzcd}
		{\overline{\mathcal T}\left(G/H,A^H\right)} & {\overline{\mathcal T}(G,A)} \\
		{H^2\left(G/H,A^H\right)} & {H^2(G,A)}
		\arrow["{\op{Inf}}", from=1-1, to=1-2]
		\arrow[hook, two heads, from=1-1, to=2-1]
		\arrow[hook, two heads, from=1-2, to=2-2]
		\arrow["{\op{Inf}}", from=2-1, to=2-2]
	\end{tikzcd}\]
	induced by modding out from \autoref{rem:tupleinflationcommutativediagram}.
\end{remark}
Restriction is similar.
\begin{lemma} \label{lem:restricttuple}
	Fix everything as in the modified set-up, forgetting about the extension $\mc E$. Further, let $H\coloneqq\langle\sigma_1^{d_1},\ldots,\sigma_m^{d_m}\rangle$ be a subgroup with $d_\bullet\mid n_\bullet$. Consider the restriction map $\op{Res}\colon H^2\left(G,A\right)\to H^2(H,A)$.
	
	If the cohomology class $[c]\in H^2\left(G,A\right)$ is represented by the $\left\{\sigma_i\right\}_{i=1}^m$-tuple $(\{\alpha_i\},\{\beta_{ij}\})$, then the cohomology class $[\op{Res}\overline c]$ is represented by the $\{\sigma_i^{d_i}\}_{i=1}^m$-tuple
	\[(\{\overline\alpha_i\},\{\overline\beta_{ij}\})=\left(\left\{\alpha_i^{1_{d_i=n_i}}\right\},\left\{\sigma_i^{(d_i1_{n_i=d_i})}\sigma_j^{(d_j1_{n_j=d_j})}\beta_{ij}\right\}\right).\]
\end{lemma}
\begin{proof}
	As in the previous proof, we will simply define $c$ by \autoref{thm:getcocycle}, and we will use the formulae of \autoref{lem:explicitalphabeta} to retrieve the $\{\sigma_i^{d_i}\}$-tuple for $\op{Res}c$. Indeed, we compute
	\begin{align*}
		\overline\alpha_i &= \prod_{k=0}^{n_i/d_i-1}(\op{Res}c)\left(\sigma_i^{d_ik},\sigma_i^{d_i}\right) \\
		&= \prod_{k=0}^{n_i/d_i-1}c\left(\sigma_i^{d_ik},\sigma_i^{d_i}\right) \\
		&= \prod_{k=0}^{n_i/d_i-1}\alpha_i^{\floor{d_i(k+1)/n_i}},
	\end{align*}
	where in the last equality we have used the construction of $c$. Now, if $n_i=d_i$, and the product is empty, and we get $1$; otherwise, the last term of the product $k=n_i/d_i-1$ is the only term which does not return $1$, and it returns $\alpha_i$. So this matches the claimed $\alpha_i^{1_{n_i=d_i}}$.

	Continuing, we compute
	\begin{align*}
		\overline\beta_{ij} &= \frac{(\op{Res}c)\left(\sigma_i^{d_i},\sigma_j^{d_j}\right)}{(\op{Res}c)\left(\sigma_j^{d_j},\sigma_i^{d_i}\right)} \\
		&= \frac{c\left(\sigma_i^{d_i},\sigma_j^{d_j}\right)}{c\left(\sigma_j^{d_j},\sigma_i^{d_i}\right)} \\
		&= c\left(\sigma_i^{d_i},\sigma_j^{d_j}\right),
	\end{align*}
	where in the last step we have used the construction of $c$. Now, if $n_i= d_i$ or $n_i= d_j$, then we are computing $c\left(1,\sigma_j^{d_j}\right)$ or $c\left(\sigma_i^{d_i},1\right)$, which are both $1$, as needed. Otherwise, $d_i<n_i$ and $d_j<n_j$, so
	\[\overline\beta_{ij}=\beta_{ij}^{(d_id_j)},\]
	which again is as claimed.
\end{proof}
Thankfully, we will really only care about inflation in the following discussion, but we will say that there are analogues of \autoref{rem:tupleinflationcommutativediagram} and \autoref{rem:inflationclasses}.

\subsection{Profinite Groups}
In this subsection, we will use our results on change of group to extend our results a little to allow profinite groups. As such, we will want to slightly modify our set-up; we will call the following set-up the ``profinite set-up.''

Let $\mathcal I$ be a poset category such that any pair of elements has an upper bound (i.e., a directed set), and let the functor $G_\bullet\colon\mathcal I\opp\to\op{FinAbGrp}$ be an inverse system of finite abelian groups. These will create a profinite group
\[G\coloneqq\limit_{i\in\mathcal I}G_i.\]
In order to be able to apply our theory, we will assume that $G$ is a finite direct sum of procyclic groups as
\[G\simeq\bigoplus_{k=1}^m\overline{\langle\sigma_k\rangle}\]
for some elements $\{\sigma_k\}_{k=1}^m\subseteq G$. Further, we will require that the kernel $N_i$ of the map $G\onto G_i$ to take the form
\[N_i\coloneqq\overline{\left\langle\sigma_1^{d_{i,1}},\ldots,\sigma_m^{d_{i,m}}\right\rangle}\]
in such a way that $\langle\sigma_kN_i\rangle$ has order $d_{i,k}$. In short, our restriction on the $N_i$ will allow our inflation maps to be computable in the sense of \autoref{lem:tupleinflation}. We quickly remark that, because the topology on $G$ is the coarsest one making the projections $G\onto G_i$ continuous, the subsets $\{N_i\}_{i\in\mathcal I}$ give a fundamental system of open neighborhoods around the identity.
\begin{remark}
	Of course, one could also start with $G$ being a finite direct sum of procyclic groups and then define the $N_i$ and $G_i$ accordingly. We have chosen the above approach because in application one might only have access to select $G_i$s, and it is not obvious how to choose these from such a ``top-down'' approach.
\end{remark}
\begin{example}
	To show that we are still allowing interesting groups, we can set 
	\[G_{m,\nu}\coloneqq\op{Gal}\left(\QQ_p(\zeta_{p^m-1})\QQ_p(\zeta_{p^\nu})/\QQ_p\right)\simeq\op{Gal}\left(\QQ_p(\zeta_{p^m-1})/\QQ_p\right)\oplus\op{Gal}\left(\QQ_p(\zeta_{p^\nu})/\QQ_p\right),\]
	which becomes $G=\op{Gal}\left(\QQ_p^{\op{ab}}/\QQ_p\right)\simeq\widehat\ZZ\oplus\ZZ_p^\times$ upon taking the inverse limit. It is not very hard to check that the kernels are generated correctly; for example, when $p$ is odd, we have $\ZZ_p^\times\cong\ZZ/(p-1)\ZZ\oplus\ZZ_p$, and under our isomorphisms, we will have
	\[\op{Gal}\left(\QQ(\zeta_{p^\nu})/\QQ_p\right)\simeq\ZZ/(p-1)\ZZ\oplus\ZZ_p/p^{\nu-1}\ZZ_p,\]
	so the kernel of $G\onto G_{m,\nu}$ is $m\widehat\ZZ\oplus(\ZZ/(p-1)\ZZ)^{1_{\nu=0}}\oplus p^{\nu-1}\ZZ_p$.
	% let $K$ be a local field with $\op{char}K=0$, and set $G_{\pi,n,m}\coloneqq\op{Gal}(K_{\pi,n}K_m/K)$, where $\{K_{\pi,n}\}$ is an ascending chain of Lubin--Tate extension and $K_m$ is the unramified extension of degree $m$. Then
	% \begin{align*}
	% 	\colimit_{n,m}\op{Gal}(K_{\pi,n}K_m/K) &\cong \op{Gal}(K^{\mathrm{ab}}/K) \\
	% 	&\cong \op{Gal}(K^{\op{unr}}/K)\oplus\op{Gal}(K_\pi) \\
	% 	&\cong \overline{\langle\op{Frob}_K\rangle}\oplus\mathcal O_K^\times \\
	% 	&\cong \widehat\ZZ\oplus\FF_\mf p^\times\oplus(1+\mf p) \\
	% 	&\cong \widehat\ZZ\oplus\FF_\mf p^\times\oplus\ZZ/p^a\ZZ\oplus\ZZ_p^{[K:\QQ_p]}
	% \end{align*}
	% for some sufficiently large $a\in\NN$; here the last isomorphism is by the logarithm map, which exists because $\op{char}K=0$. (For details, see \cite{neukirch-alg-nt}, Proposition II.5.7.)
\end{example}
\begin{remark}
	I'm not sure if such an explicit construction can be extended to other local fields $K$ (say, via Lubin--Tate theory). Because $K^\times$ is not topologically finitely generated when $K$ is in positive characteristic (see for example \cite[Proposition~II.5.7]{neukirch-alg-nt}) such a construction must do something subtle.
\end{remark}
Let $A$ be a discrete $G$-module. The main goal of this subsection is to be able to provide a notion of a ``compatible system'' of tuples from each individual $H^2(G_i,A)$ to be able to exactly describe an element of $H^2(G,A)$. To effect this, we have the following somewhat annoying checks.
% \begin{lemma}
% 	Fix everything as in the profinite set-up, and let $N\subseteq G$ be an open normal subgroup. If $\sigma\in G$ is such that $[\sigma]_N\in G/N$ has finite order $n_\sigma$, then there exists some $i\in\mathcal I$ such that the order of $[\sigma]_{N_i}\in G/N_i=G_i$ has order divisible by $n_\sigma$.
% \end{lemma}
% \begin{proof}
% 	We proceed in steps. Let $1$ denote the identity of $G$.
% 	\begin{enumerate}
% 		\item Suppose that $p\coloneqq n_\sigma$ is prime. We proceed by contraposition. Namely, suppose there is no $N_i$ such that $\sigma^p\in N_i$, and we will show that $[\sigma]_N\in G/N$ cannot have order $p$. We may assume that $\sigma^p\in N$, which means that we are actually interested in showing $\sigma\in N$.

% 		Well, we claim that $\sigma$ is a limit point of $\langle\sigma^p\rangle$. To see this, we need to show that any open neighborhood $U$ around $\sigma$ has nontrivial intersection with $\langle\sigma^p\rangle$. Indeed, $\sigma^{-1}U$ is an open set containing the identity, but because the $\{N_i\}_{i\in\mathcal I}$ form a fundamental system of open neighborhoods around the identity, we have $N_i\subseteq\sigma^{-1}U$ for some $i\in\mathcal I$. Thus, it suffices to show that 
% 		\[\sigma N_i\cap\langle\sigma^p\rangle\ne\emp.\]
% 		Now, the order of $\sigma N_i$ is not divisible by $p$, so $\langle\sigma^pN_i\rangle=\langle\sigma N_i\rangle$ and in particular $\langle\sigma^pN_i\rangle$ contains $\sigma N_i$. Concretely, let's say $\sigma^{pk}N_i=\sigma N_i$; then $\sigma^{pk}\in\sigma N_i\cap\langle\sigma^p\rangle$, finishing.

% 		In total, because $N$ is an open subgroup and hence closed, we see that $\sigma^p\in N$ must also contain $\langle\sigma^p\rangle$ and hence must contain the limit point $\sigma$. This finishes.

% 		\item Next suppose that $n_\sigma=p^\nu$ is a power of a prime. We proceed by induction on $\nu$. When $\nu=0$, there is nothing to say because the order of a group element is always divisible by $1$; when $\nu=1$, this is the previous step. Otherwise, when $\nu>1$, we note that $\left[\sigma^p\right]_N=[\sigma]_N^p$ has order $p^{\nu-1}$: certainly $[\sigma]_N^{p\cdot p^{\nu-1}}$ vanishes, so the order divides $p^{\nu-1}$, but no smaller of $p$ will do because this would make the order of $[\sigma]_N$ too small.

% 		Thus, by the inductive hypothesis, there exists some $i\in\mathcal I$ such that the order of $[\sigma^p]_{N_i}$ has order divisible by $p^{\nu-1}$. We claim $[\sigma]_{N_i}$ has order divisible by $p^\nu$. Indeed, if not, then there exists some $k$ with $p\nmid k$ such that
% 		\[[\sigma]_{N_i}^{p^{\nu-1}k}=[1]_{N_i},\]
% 		from which we conclude that the order of $[\sigma^p]$ divides $p^{\nu-2}k$, which is not divisible by $p^{\nu-1}$.

% 		\item To finish, we show a version of multiplicativity: if the order of $[\sigma]_{N_p}$ is divisible by $n_p$, and the order of $[\sigma]_{N_q}$ is divisible by $n_q$ with $\gcd(n_p,n_q)=1$, then there exists $r\in\mathcal I$ such that the order of $[\sigma]_{N_r}$ is divisible by $n_pn_q$.

% 		Indeed, because $\mathcal I$ is a directed set, there exists some $r\in\mathcal I$ with morphisms $i\to r$ and $j\to r$. These correspond to having morphisms $G_r\to G_i$ and $G_r\to G_j$, and the fact that these morphisms are well-defined requires $N_r\subseteq N_i,N_j$.

% 		Now, let's say that the order of $[\sigma]_{N_r}$ is $n_r$; we want to show $n_pn_q\mid n_r$. Because $\gcd(n_p,n_q)=1$, it suffices (by symmetry) to show $n_p\mid n_r$. Well, $\sigma^{n_r}\in N_r\subseteq N_p$, so
% 		\[[\sigma]_{N_p}^{n_r}=[1]_{N_p},\]
% 		so the order of $[\sigma]_{N_p}$ divides $n_r$. In particular, $n_p\mid n_r$.
% 	\end{enumerate}
% 	We now note that, for the general case of $n\in\NN$, we can prime-factor $n$ into coprime factors, use step 2 to create a list of $N_i$, one for each prime factor, and then use step 3 to glue them all together. This completes the proof.
% \end{proof}
% \begin{lemma}
% 	Fix everything as in the profinite set-up. Then, for any open normal subgroup $N\subseteq G$, there exists $i\in\mathcal I$ so that $N$ contains $N_i$.
% \end{lemma}
% \begin{proof}
% 	This follows directly from the fact that the collection $\{N_i\}_{i\in\mathcal I}$ is a fundamental system of open neighborhoods around the identity of $G$. In particular, $N$ contains the identity and is open.
% 	% Because $G$ is compact (it's profinite), we have $[G:N]<\infty$. In particular, for any $\sigma\in G$, we must have $\sigma^{[G:N]}\in N$.
% 	% Thus, it will roughly speaking be enough to show that, for any $\sigma_k$ and $t\in\NN$, we have $\sigma_k^t\in N_i$ for some $i\in\mathcal I$. We have two cases.
% 	% \begin{itemize}
% 	% 	\item Suppose that $\sigma_k\in G$ has infinite order. 
% 	% \end{itemize}
% \end{proof}
\begin{lemma} \label{lem:colimitfiltered}
	Suppose that $\mathcal P$ is a directed set, and let $\mathcal P'\subseteq\mathcal P$ be a subcategory such that any $x\in\mathcal P$ has some $x'\in\mathcal P'$ such that $x\le x'$. Then, given a functor $F\colon\mathcal P\to\mathcal C$, we have
	\[\colimit_\mathcal PF\simeq\colimit_{\mathcal P'}F,\]
	provided that both colimits exist.
\end{lemma}
\begin{proof}
	For concreteness, if $x\le y$ in $\mathcal P$, we will let $f_{yx}\colon x\to y$ be the corresponding morphism; in particular, $x\le y\le z$ has $f_{zx}=f_{zy}f_{yx}$. Now, for brevity, set
	\[X\coloneqq\colimit_\mathcal PF\qquad\text{and}\qquad X'\coloneqq\colimit_{\mathcal P'}F.\]
	By the Yoneda lemma, it suffices to fix some object $Y\in\mathcal C$ and show that $\op{Mor}_\mathcal C(X,Y)\simeq\op{Mor}_\mathcal C(X',Y)$. Well, morphsims $X\to Y$ are in (natural) bijection with cones under $F$ with nadir $Y$, and morphisms $X'\to Y$ are in (natural) bijection with cones under $F'\coloneqq F|_{\mathcal P'}$ with nadir $Y$.

	Thus, it suffices to give a natural bijection between cones under $F$ with nadir $Y$ and cones under $F'$ with nadir $Y$. Well, given a cone under $F$ with nadir $Y$, we can simply restrict it to $\mathcal P'$ to get a cone under $F'$. In the other direction, given a cone under $F'$ with nadir $Y$, we can build a cone under $F$ with nadir $Y$ as follows; let $\varphi_{x'}\colon F(x')\to Y$ for $x'\in\mathcal P'$ be the corresponding morphisms in our cone.
	
	For any $x\in\mathcal P$, find $x'\in\mathcal P'$ such that $x\le x'$. Then set
	\[\varphi_x\coloneqq\varphi_{x'}\circ f_{x'x}\]
	Note that $\varphi_x$ is in fact independent of our choice of $x'$: if $x\le x_1'$ and $x\le x_2'$, then because $\mathcal P$ is a directed set, we can find $y\in\mathcal P$ such that $x_1',x_2'\le y$ and then $y'\in\mathcal P'$ with $y\le y'$. Then
	\begin{align*}
		\varphi_{x_\bullet'}\circ f_{x_\bullet'x} &= \varphi_{y'}\circ f_{y'x_\bullet'}\circ f_{x_\bullet'x} \\
		&= \varphi_{y'}\circ f_{y'x}
	\end{align*}
	for $x_\bullet'\in\{x_1',x_2'\}$. Anyway, we can check that the morphisms $\varphi$ do assemble to a cone under $F'$: if $x\le y$ in $\mathcal P$, then find $y'\in\mathcal P$ with $x\le y\le y'$, and we compute
	\begin{align*}
		\varphi_y\circ f_{yx} &= \varphi_{y'}\circ f_{y'y}\circ f_{yx} \\
		&= \varphi_{y'}\circ f_{y'x} \\
		&= \varphi_x.
	\end{align*}
	Thus, we do have a natural, well-defined map sending cones under $F'$ with nadir $Y$ to cones under $F$ with nadir $Y$. It is not too hard to see that these maps are inverse to each other (for example, the cone under $F'$, extended to $F$, does indeed restrict back to $F'$ properly), which completes the proof.
\end{proof}
\begin{remark}
	One can remove the hypothesis that the colimits exist and use essentially the same proof.
\end{remark}
\begin{proposition} \label{prop:bettercohomlimit}
	Fix everything as in the profinite set-up. Then, given a discrete $G$-module $A$,
	\[H^2(G,A)\simeq\colimit_{i\in\mathcal I}H^2\left(G_i,A^{N_i}\right).\]
	Here, the morphisms between the collection of $H^2\left(G_i,A^{N_i}\right)$ are induced by inflation: if $i\to j$ in $\mathcal I$, then $G_j\to G_i$ in $\mathrm{FinAbGrp}$, giving an inflation map $\op{Inf}\colon H^2\left(G_i,A^{N_i}\right)\to H^2\left(G_j,A^{N_j}\right)$.
\end{proposition}
\begin{proof}
	Let $\mathcal N$ be the poset category of open normal subgroups of $G$, reverse ordered under inclusion; i.e., $N_1\subseteq N_2$ in $G$ induces a map $N_2\to N_1$. Then it is already known that
	\[H^2(G,A)\simeq\colimit_{N\in\mathcal N}H^2\left(G/N,A^N\right).\]
	On the other hand, observe that $i\le j$ in $\mathcal I$ induces $G_j\to G_i$, so $N_j\subseteq N_i$. In other words, $i\mapsto N_i$ will define a functor $\mathcal I\to\mathcal N$; functoriality follows because $\mathcal I$ and $\mathcal N$ are poset categories. Letting $\mathcal N'$ denote the image of $\mathcal I$ in $\mathcal N$, we see
	\[\colimit_{i\in\mathcal I}H^2\left(G_i,A^{N_i}\right)\simeq\colimit_{N\in\mathcal N'}H^2\left(G/N,A^N\right).\]
	Notably, the inflation maps $\op{Inf}\colon H^2\left(G_i,A^{N_i}\right)\to H^2\left(G_j,A^{N_j}\right)$ when $i\le j$ become the inflation maps $\op{Inf}\colon H^2\left(G/N,A^N\right)\to H^2\left(G/N',A^{N'}\right)$ when $N'\subseteq N$. So if we let $F\colon\mathcal N\to\op{AbGrp}$ be the functor taking $N$ to $H^2\left(G/N,A^N\right)$ (and $N\subseteq N'$ to the inflation map), we are trying to show
	\[\colimit_{\mathcal N}F=\colimit_{\mathcal N'}F.\]
	For this, we use \autoref{lem:colimitfiltered}. Indeed, for a given open normal subgroup $N\in\mathcal N$, we need to find some $N'\in\mathcal N'$ such that $N\le N'$, which means $N'\subseteq N$.
	
	However, the elements of $\mathcal N'$ are the collection $\{N_i\}_{i\in\mathcal I}$, which form a fundamental system of open neighborhoods around the identity. Thus, the fact that $N$ is an open set containing the identity implies there is some $N_i\in\mathcal N'$ such that $N_i\subseteq N$. This finishes the proof.
\end{proof}
Observe that the above proofs did not use the extra hypotheses on $G$ nor $N_i$ to be products of procyclic groups. We use these hypotheses now.
% \begin{definition}
% 	Fix everything as in the profinite set-up, and let $A$ be a discrete $G$-module. Then a \textit{compatible system of $\{\sigma_p\}_{p=1}^m$-tuples} is an indexed set
% 	\[\left(\{\alpha_{i,p}\},\{\beta_{i,pq}\}\right)_{i\in\mathcal I}\]
% 	such that $\left(\{\alpha_{i,p}\},\{\beta_{i,pq}\}\right)$ is a $\{\sigma_pN_i\}_{p=1}^m$-tuple (corresponding to a class in $H^2\left(G_i,A^{N_i}\right)$) and
% 	\[\op{Inf}\left(\{\alpha_{i,p}\},\{\beta_{i,pq}\}\right)\sim\left(\{\alpha_{j,p}\},\{\beta_{j,pq}\}\right)\]
% 	as tuples corresponding to $H^2\left(G_j,A^{N_j}\right)$, whenever $i\le j$ in $\mathcal I$. We also define the relation $\sim$ of equivalence between compatible systems if and only if they are pointwise equivalent.
% \end{definition}
% The precise definition above is one of technical convenience, as we will shortly see.
To work more concretely, we note that any $i\in\mathcal I$ has
\[G_i\simeq\frac G{N_i}\simeq\bigoplus_{p=1}^m\overline{\langle\sigma_p\rangle}/\overline{\langle\sigma_p^{d_{i,p}}\rangle}\simeq\bigoplus_{p=1}^m\langle\sigma_p\rangle/\langle\sigma_p^{d_{i,p}}\rangle\subseteq\bigoplus_{p=1}^m\ZZ/d_{i,p}\ZZ\]
is a finite abelian group generated by the elements $\sigma_pN_i$. By choosing the $d_{i,p}$ appropriately, recall that we also forced the order of $\sigma_pN_i$ to be $d_{i,p}$.

Regardless, the main point is that, given a discrete $G$-module $A$, we can consider the $\{\sigma_pN_i\}_{p=1}^m$-tuples $\mathcal T\left(G_i,A^{N_i}\right)$. Now, as discussed above, $i\le j$ in $\mathcal I$ induces a quotient map $G_j\simeq G/N_j\onto G/N_i\simeq G_i$. From this, we have the following coherence check.
\begin{lemma} \label{lem:tupleinflationcommutes}
	Fix everything as in the profinite set-up, and let $A$ be a discrete $G$-module. Then, given $i\le j\le k$ in $\mathcal I$, the diagram
	% https://q.uiver.app/?q=WzAsMyxbMCwwLCJcXG1hdGhjYWwgVFxcbGVmdChHX2ksQV9pXntOX2l9XFxyaWdodCkiXSxbMSwwLCJcXG1hdGhjYWwgVFxcbGVmdChHX2osQV9qXntOX2p9XFxyaWdodCkiXSxbMSwxLCJcXG1hdGhjYWwgVFxcbGVmdChHX2ssQV9rXntOX2t9XFxyaWdodCkiXSxbMCwxLCJcXG9we0luZn0iXSxbMSwyLCJcXG9we0luZn0iXSxbMCwyLCJcXG9we0luZn0iLDJdXQ==&macro_url=https%3A%2F%2Fraw.githubusercontent.com%2FdFoiler%2Fnotes%2Fmaster%2Fnir.tex
	\[\begin{tikzcd}
		{\mathcal T\left(G_i,A^{N_i}\right)} & {\mathcal T\left(G_j,A^{N_j}\right)} \\
		& {\mathcal T\left(G_k,A^{N_k}\right)}
		\arrow["{\op{Inf}}", from=1-1, to=1-2]
		\arrow["{\op{Inf}}", from=1-2, to=2-2]
		\arrow["{\op{Inf}}"', from=1-1, to=2-2]
	\end{tikzcd}\]
	commutes. Here, the $\op{Inf}$ maps are defined as in \autoref{lem:tupleinflation}.
\end{lemma}
\begin{proof}
	For each $i\in\mathcal I$, we let $n_{i,p}$ denote the order of $\sigma_pN_i\in G_i$. Using the definition of $\op{Inf}$ from \autoref{lem:tupleinflation}, we just pick up some $\{\sigma_pN_p\}_{p=1}^m$-tuple $(\{\alpha_p\},\{\beta_{pq}\})$-tuple in $\mathcal T\left(G_i,A^{N_i}\right)$ and track through the diagram as follows.
	% https://q.uiver.app/?q=WzAsNCxbMCwwLCIoXFx7XFxhbHBoYV9wXFx9LFxce1xcYmV0YV97cHF9XFx9KSJdLFsxLDAsIlxcbGVmdChcXGJpZ1xce1xcYWxwaGFfcF57bl97aixwfS9uX3tpLHB9fVxcYmlnXFx9LFxce1xcYmV0YV97cHF9XFx9XFxyaWdodCkiXSxbMSwxLCJcXGxlZnQoXFxiaWdcXHtcXGFscGhhX3BeeyhuX3tqLHB9L25fe2kscH0pKG5fe2sscH0vbl97aixwfSl9XFxiaWdcXH0sXFx7XFxiZXRhX3twcX1cXH1cXHJpZ2h0KSJdLFswLDEsIlxcbGVmdChcXGJpZ1xce1xcYWxwaGFfcF57bl97ayxwfS9uX3tpLHB9fVxcYmlnXFx9LFxce1xcYmV0YV97cHF9XFx9XFxyaWdodCkiXSxbMCwxLCJcXG9we0luZn0iXSxbMSwyLCJcXG9we0luZn0iXSxbMCwzLCJcXG9we0luZn0iLDJdLFszLDIsIiIsMix7ImxldmVsIjoyLCJzdHlsZSI6eyJoZWFkIjp7Im5hbWUiOiJub25lIn19fV1d&macro_url=https%3A%2F%2Fraw.githubusercontent.com%2FdFoiler%2Fnotes%2Fmaster%2Fnir.tex
	\[\begin{tikzcd}
		{(\{\alpha_p\},\{\beta_{pq}\})} & {\left(\big\{\alpha_p^{d_{j,p}/d_{i,p}}\big\},\{\beta_{pq}\}\right)} \\
		{\left(\big\{\alpha_p^{1_{d_{k,p}=d_{i,p}}d_{k,p}/d_{i,p}}\big\},\{\beta_{pq}\}\right)} & {\left(\big\{\alpha_p^{(1_{d_{j,p}=d_{i,p}}d_{j,p}/d_{i,p})(1_{d_{k,p}=d_{j,p}}d_{k,p}/d_{j,p})}\big\},\{\beta_{pq}\}\right)}
		\arrow["{\op{Inf}}", from=1-1, to=1-2]
		\arrow["{\op{Inf}}", from=1-2, to=2-2]
		\arrow["{\op{Inf}}"', from=1-1, to=2-1]
		\arrow[Rightarrow, no head, from=2-1, to=2-2]
	\end{tikzcd}\]
	Notably, $d_{k,p}=d_{i,p}$ implies that these are both equal to $d_{j,p}$ because $i\le j\le k$ upon tracking the order of $\sigma_p$ through our morphisms $G_k\to G_j\to G_i$. This completes the proof.
\end{proof}
% Now, if we let $n_{i,p}$ denote the actual order of $\sigma_{i,p}N_i\in G_i$, then we may compute the inflation map $\op{Inf}\colon H^2\left(G_i,A^{N_i}\right)\to H^2\left(G_j,A^{N_j}\right)$ by
% \[\op{Inf}\left(\{\alpha_{i,p}\},\{\beta_{i,pq}\}\right)=\left(\{\alpha_{i,p}^{n_{j,p}}\},\{\beta_{i,pq}\}\right),\]
% so we are asking for
% \[\left(\{\alpha_{i,p}^{n_{j,p}}\},\{\beta_{i,pq}\}\right)\sim\left(\{\alpha_{j,p}\},\{\beta_{j,pq}\}\right)\]
% in the coherence condition for a compatible tuple.
% \begin{remark}
% 	From the above description, we can see why we ``have'' to allow the equivalence relation into our notion of compatibility. For example, if one of the $G_i$ is the trivial group, and $A^G$ is trivial, then we would be requiring all the $\beta_{i,pq}$ elements to be trivial for all $i\in\mathcal I$. This is not good.
% \end{remark}
And here is the result.
\begin{theorem}
	Fix everything as in the profinite set-up, and let $A$ be a discrete $G$-module. Then the isomorphisms of \autoref{thm:classisomorphism} upgrade into an isomorphism
	\[H^2(G,A)\simeq\colimit_{i\in\mathcal I}\overline{\mathcal T}\left(G_i,A^{N_i}\right).\]
	Here the morphisms between the $\overline{\mathcal T}\left(G_i,A^{N_i}\right)$ are inflation maps of \autoref{lem:tupleinflation}.
\end{theorem}
\begin{proof}
	Note that the objects $\overline{\mathcal T}\left(G_i,A^{N_i}\right)$ do make a directed system over $\mathcal I$ because of the commutativity of \autoref{lem:tupleinflationcommutes}. Namely, the lemma checks that $\mathcal I\to\op{AbGrp}$ by $i\mapsto\overline{\mathcal T}\left(G_i,A^{N_i}\right)$ is actually functorial; technically we must also check that the maps $\overline{\mathcal T}\left(G_i,A^{N_i}\right)\to\overline{\mathcal T}\left(G_i,A^{N_i}\right)$ are the identity, but this follows from the definition.

	Now, by \autoref{prop:bettercohomlimit}, we have
	\[H^2(G,A)\simeq\colimit_{i\in\mathcal I}H^2\left(G_i,A^{N_i}\right),\]
	but now the natural isomorphism induced by \autoref{rem:inflationclasses} induces an isomorphism of direct limits
	\[\colimit_{i\in\mathcal I}H^2\left(G_i,A^{N_i}\right)\simeq\colimit_{i\in\mathcal I}\overline{\mathcal T}\left(G_i,A^{N_i}\right)\]
	given by the isomorphism of \autoref{thm:classisomorphism} acting pointwise. This completes the proof.
\end{proof}
Because there are reasonably explicit descriptions of direct limits of abelian groups, and we already have an explicit description of each $\overline{\mathcal T}\left(G_i,A^{N_i}\right)$ term in addition to a description of the inflation maps between them, we will be content with our sufficiently explicit description of $H^2(G,A)$. So we call it done here.