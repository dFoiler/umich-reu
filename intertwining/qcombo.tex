% !TEX root = ../intertwining.tex

\section{\texorpdfstring{$q$}{q}-Combinatorial Inputs} \label{sec:qcombo}
In this section, we discuss some combinatorial inputs into our main results. We forget about all the notation we set in the previous section in the first few subsections. Instead, $q$ will be treated as a free variable until stated otherwise.

\subsection{A Couple \texorpdfstring{$q$}{q}-Identities}
In this quick subsection, we pick up a couple $q$-identities which will be useful in the sequel. Throughout, we freely use the packages \texttt{qZeil} and \texttt{qMultiSum} developed by Axel Riese; see \cite{riese-zeil,riese-multisum} for a description of these packages. The following identity is used for the symplectic and orthogonal groups.
\begin{prop}
    For any nonnegative integers $m,n\in\ZZ$, we have
    \[q^{\binom{m-n}2}\sum_{i=0}^{\floor{n/2}}\frac{\left(q;q\right)_n}{\left(q^{2};q^{2}\right)_i\left(q;q\right)_{n-2i}}q^{\binom{n-m-2i}2}=\sum_{j=0}^{m}\left(-1\right)^{m-j}\frac{\left(q^{2};q^{2}\right)_m}{\left(q;q\right)_j\left(q^{2};q^{2}\right)_{m-j}}q^{\binom{j-n}2}.\]
\end{prop}
\begin{proof}
    For technical reasons, set
    \begin{align*}
        L_{m,n}(q)&\coloneqq q^{\frac{1}{2}\left(n^{2}+2m^{2}-4mn-n\right)}\sum_{i=0}^{\floor{n/2}}\frac{\left(q;q\right)_n}{\left(q^{2};q^{2}\right)_i\left(q;q\right)_{n-2i}}q^{2i^{2}-2ni+i}\left(q^{m}\right)^{2i} \\
        R_{m,,n}(q)&\coloneqq\sum_{j=0}^{m}\left(-1\right)^{m-j}\frac{\left(q^{2};q^{2}\right)_m}{\left(q;q\right)_j\left(q^{2};q^{2}\right)_{m-j}}q^{\frac{1}{2}\left(j^{2}-j\right)}\left(q^{n}\right)^{-j}
    \end{align*}
    so that we want to show that $L_{m,n}(q)=R_{m,n}(q)$. We will show that $L_{m,n}(q)$ and $R_{m,n}(q)$ satisfy the same recurrence in $m$ and then check that $L_{m,n,}(q)=R_{m,n}(q)$ for some small $m$. With this outline in mind, we have the following steps.
    \begin{enumerate}
        \item The package \texttt{qZeil} shows that $m\ge2$ has
        \[R_{m,n}(q)=q^{-2-n}\left(q^{2m}+q^{1+2m}-q^{2+n}\right)R_{m-1,n}(q)+q^{-3+2m-2n}\left(1-q^{-2+2m}\right)R_{m-2,n}(q).\]
        We would like to show that $L_{m,n}(q)$ satisfies the same recurrence in $m$. Well, define $\widetilde L_{m,n}(q)$ to be
        \[L_{m,n}(q)-q^{-2-n}\left(q^{2m}+q^{1+2m}-q^{2+n}\right)L_{m-1,n}(q)-q^{-3+2m-2n}\left(1-q^{-2+2m}\right)L_{m-2,n}(q),\]
        and we want to show that $\widetilde L_{m,n}(q)$ vanishes. Some elementary rearrangement shows that $\widetilde L_{m,n}(q)$ is
        \begin{align*}
            \sum_{i=0}^{\floor{n/2}}&q^{-1+2m\left(-1+i\right)+2i^{2}-i\left(3+2n\right)} \\
            &\cdot \big(q^{1+2m+4i}+q^{2\left(m+n\right)}+q^{2\left(1+i+n\right)}-q^{2m+2i+n}-q^{1+2m+2i+n}-q^{2+2n}\big) \\
            &\cdot\frac{\left(q;q\right)_n}{\left(q;q\right)_{n-2i}\left(q^{2};q^{2}\right)_i}
        \end{align*}
        up to some powers of $q$ that we have divided out. The package \texttt{qZeil} is able to show that this sum vanishes.
        \item It remains to check that $L_{m,n}(q)=R_{m,n}(q)$ for $m\in\{0,1\}$. For $m=0$, \texttt{qZeil} shows that $L_{0,n}(q)=1$, which agrees with $R_{0,n}(q)$. For $m=1$, \texttt{qZeil} shows that
        \[L_{1,n}(q)=\frac{1+q-q^n}{q+q^2-q^n}L_{1,n-1}(q)\]
        and checks that $R_{1,n}$ satisfies the same recurrence. So it is enough to check that $L_{1,0}(q)=R_{1,0}(q)=q$.
        \qedhere
    \end{enumerate}
\end{proof}
The following identity is used for the linear groups.
\begin{prop}
    For any nonnegative integers $m,n\in\ZZ$, we have
    \begin{align*}
        &q^{-nm+\binom m2}\sum_{i=0}^{n}\frac{\left(q;q\right)_n^{2}}{\left(q;q\right)_i^{2}\left(q;q\right)_{n-i}}q^{i^{2}-im} \\
        ={}&\sum_{0\le j_2\le j_1\le m}\left(-1\right)^{m-j_{1}}\frac{\left(q;q\right)_m}{\left(q;q\right)_{m-j_{1}}\left(q;q\right)_{j_{1}-j_{2}}\left(q;q\right)_{j_{2}}}q^{\frac{1}{2}\left(j_{1}^2-j_1+2j_2^2-2j_1j_2-2nj_1\right)}.
    \end{align*}
\end{prop}
\begin{proof}
    Let the left-hand side be $L_{m,n}(q)$ and the right-hand side be $R_{m,n}(q)$ so that we want to show that $L_{m,n}(q)=R_{m,n}(q)$. As with the last proof, we will show that $L_{m,n}(q)$ and $R_{m,n}(q)$ satisfy the same recurrence in $m$ and then check that $L_{m,n,}(q)=R_{m,n}(q)$ for some small $m$. With this outline in mind, we have the following steps.
    \begin{enumerate}
        \item After some rearranging, \texttt{qMultiSum} shows that $m\ge2$ has
        \[R_{m,n}(q)=q^{-n}\left(2q^{\left(1+m-2\right)}-q^{n}\right)R_{m-1,n}(q)-q^{m-2n-2}\left(q^{m-1}-1\right)R_{m-2,n}(q).\]
        We would like to show that $L_{m,n}(q)$ satisfies the same recurrence in $m$. Well, define $\widetilde L_{m,n}(q)$ to be
        \[\widetilde L_{m,n}(q)\coloneqq L_{m,n}(q)-q^{-n}\left(2q^{\left(1+m-2\right)}-q^{n}\right)L_{m-1,n}(q)+q^{m-2n-2}\left(q^{m-1}-1\right)L_{m-2,n}(q),\]
        Some elementary rearranging shows that $\widetilde L_{m,n}(q)$ is
        \[\sum_{i=0}^{n}\frac{\left(q;q\right)_n^{2}}{\left(q;q\right)_i^{2}p\left(q;q\right)_{n-iover}}q^{i^{2}-im}\left(-1+qq^{m}-2q^{1-i}q^{m}+q^{n-i}+q^{1-2i}q^{m}\right)\]
        up to some factors that we have divided out. Now, \texttt{qZeil} provides the recurrence
        \[\widetilde L_{m,n}(q)=-q^{-2-m}\left(-q^{2n}-q^{2+m}-q^{3+m}+2q^{2+n+m}\right)\widetilde L_{m,n-1}(q)-q\left(1-q^{n-1}\right)^2\widetilde L_{m,n-2}(q),\]
        so it suffices to show that $\widetilde L_{m,0}(q)=\widetilde L_{m,1}(q)=0$. Both of these small cases are readily checked by hand.
        \item It remains to check that $L_{m,n}(q)=R_{m,n}(q)$ for $m\in\{0,1\}$. For $m=0$, \texttt{qZeil} shows $L_{0,n}(q)=1$, which agrees with $R_{0,n}(q)$. For $m=1$, \texttt{qZeil} shows
        \[L_{1,n}(q)=\frac{q^n-2}{q\left(q^{n-1}-2\right)}L_{1,n-1}(q)\]
        and checks that $R_{1,n}(q)$ satisfies the same recurrence. So it is enough to check that $L_{1,0}(q)=R_{1,0}(q)=1$.
        \qedhere
    \end{enumerate}
\end{proof}

\subsection{Some Antitriangular Matrices}
In this subsection, we discuss the eigenvalues of some antitriangular matrices. Essentially the only method in the literature to access the eigenvalues of an antitriangular matrix is to do some educated guessing in order to make the give matrix upper-triangular. See \cite{britnell-antitriangular} for a thorough discussion of a special case; the work in this subsection can be seen as a $q$-analogue for some of their results.\todo{}
\begin{theorem} \label{thm:eigens-gl}
    Eigenvalues for $G\in\{\GL_{2n},\SL_{2n}\}$.
\end{theorem}

% diagonalize the generic matrices
% diagonalize the matrices from previous

% \appendix
% \section{Computation of \texorpdfstring{$c(1,I_6,\psi)$}{c(1,I6,psi)}}
% Throughout this section, $\FF_q$ denotes a finite field with $q$ elements, where $q$ is an odd prime-power, and $\Sym_3^\times$ denotes the set of invertible $3\times3$ matrices with entries in $\FF_q$. The goal of the present section is to prove the following result.
% \begin{thm} \label{thm:symmetric-gauss-sum-easy}
%     We have
%     \[\sum_{A\in\Sym_3^\times}\psi(\tr A)=q^2.\]
% \end{thm}
% We will prove this by using combinatorics and elementary number theory in order to compute the number of invertible symmetric matrices with given diagonal entries. For brevity, given $(a_{11},a_{22},a_{33})\in\FF_q^3$, we say that $A\in\Sym_3^\times$ has ``type $(a_{11},a_{22},a_{33})$ if and only if
% \[A=\begin{bmatrix}
%     a_{11} & a_{12} & a_{13} \\
%     a_{12} & a_{22} & a_{23} \\
%     a_{13} & a_{23} & a_{33}
% \end{bmatrix}.\]
% We now examine each type individually, in ascending levels of difficulty.
% \begin{lemma} \label{lem:all-zero-type}
%     There are $(q-1)^3$ matrices in $\Sym_3^\times$ of type $(0,0,0)$.
% \end{lemma}
% \begin{proof}
%     Our matrices take the form
%     \[A\coloneqq\begin{bmatrix}
%         0 & a_{12} & a_{13} \\
%         a_{12} & 0 & a_{23} \\
%         a_{13} & a_{23} & 0
%     \end{bmatrix},\]
%     which has determinant $\det A=2a_{12}a_{13}a_{23}$. As such, this matrix is invertible if and only if each $a_{12},a_{13},a_{23}$ is nonzero, totaling to $(q-1)^3$ matrices.
% \end{proof}
% \begin{lemma} \label{lem:two-zero-type}
%     For any $a\in\FF_q^\times$, there are $q^3-q^2-(q-1)^2$ matrices in $\Sym_3^\times$ of type $(a,0,0)$. The same statement holds for permutations of $(a,0,0)$.
% \end{lemma}
% \begin{proof}
%     Our matrices take the form
%     \[A\coloneqq\begin{bmatrix}
%         a & a_{12} & a_{13} \\
%         a_{12} & 0 & a_{23} \\
%         a_{13} & a_{23} & 0
%     \end{bmatrix},\]
%     which has determinant $\det A=2a_{12}a_{13}a_{23}-aa_{23}^2$. By counting the complement, we would like to show that there are $q^2+(q-1)^2$ solutions $(x,y,z)\in\FF_q^3$ to $2xyz-ax^2=0$. There are two cases.
%     \begin{itemize}
%         \item If $x=0$, then any $(y,z)\in\FF_q^2$ will work, totaling to $q^2$ matrices here.
%         \item If $x\ne0$, then we see $2yz=ax\ne0$. Thus, there are $q-1$ choices for $y\in\FF_q^\times$, from which $z$ is forced. Counting over all $x\in\FF_q^\times$, there are $(q-1)^2$ matrices here.
%     \end{itemize}
%     Summing completes the proof.
% \end{proof}
% \begin{lemma} \label{lem:one-zero-type}
%     For any $a,b\in\FF_q^\times$, there are $q^3-q-(q-1)^2$ matrices in $\Sym_3^\times$ of type $(a,b,0)$. The statement holds for permutations of $(a,b,0)$.
% \end{lemma}
% \begin{proof}
%     Our matrices take the form
%     \[A\coloneqq\begin{bmatrix}
%         a & a_{12} & a_{13} \\
%         a_{12} & b & a_{23} \\
%         a_{13} & a_{23} & 0
%     \end{bmatrix},\]
%     which has determinant $\det A=2a_{12}a_{13}a_{23} - aa_{23}^2-ba_{13}^2$. By counting the complement, we would like to show that there are $q+(q-1)^2$ solutions $(x,y,z)\in\FF_q^3$ to $2xyz=ax^2+by^2$. We have two cases.
%     \begin{itemize}
%         \item If $x=0$, then we must have $y=0$, from which any $z\in\FF_q$ will do. There are $q$ matrices here.
%         \item If $x\ne0$ and $y\ne0$, then $z\coloneqq\left(ax^2+by^2\right)/(2xy)$ is forced. Totaling, there are $(q-1)^2$ matrices here.
%     \end{itemize}
%     Summing completes the proof.
% \end{proof}
% \begin{lemma} \label{lem:nonzero-type}
%     For any $a,b,c\in\FF_q^\times$, there are $q^3-\left(q^2+1\right)$ matrices in $\Sym_3^\times$ of type $(a,b,c)$.
% \end{lemma}
% \begin{proof}
%     Our matrices take the form
%     \[A\coloneqq\begin{bmatrix}
%         a & a_{12} & a_{13} \\
%         a_{12} & b & a_{23} \\
%         a_{13} & a_{23} & c
%     \end{bmatrix}.\]
%     Scaling will not change invertibility, so for psychological reasons we replace $A$ with $a^{-1}A$ so that we may assume $a=1$. Then $\det A=2a_{12}a_{13}a_{23}-ba_{12}^2-ca_{12}^2 - a_{23}^2+bc$. By counting the complement, we would like to show that there are $q^2+1$ solutions $(x,y,z)\in\FF_q^3$ to $x^2-2xyz=-cy^2-bz^2+bc$. Sending $x\mapsto x+yz$, we are counting solutions to
%     \[x^2=y^2z^2-cy^2-bz^2+bc=\left(y^2-b\right)\left(z^2-c\right).\]
%     We now do casework on what elements on the right-hand side are squares. This requires the following lemma.
%     \begin{lemma}
%         Fix $a\in\FF_q^\times$. The number of $x\in\FF_q$ such that $x^2-a$ is a square is
%         \[\begin{cases}
%             \frac{q-1}2 & \text{if }a\text{ is not a square}, \\
%             \frac{q+1}2 & \text{if }a\text{ is a square}.
%         \end{cases}\]
%     \end{lemma}
%     \begin{proof}
%         We are counting the number of $x\in\FF_q$ for which there is a solution $y\in\FF_q$ to the equation $x^2-a=y^2$. This rearranges to
%         \[(x+y)(x-y)=a.\]
%         Setting $s\coloneqq\frac{x+y}2$ and $d\coloneqq\frac{x-y}2$, we see that $sd=a/4$, so it is necessary and sufficient to have $x=s+\frac a{4s}$ for some $s\in\FF_q^\times$. In other words, we are currently counting the size of the image of the map $x\colon\FF_q^\times\to\FF_q$ given by
%         \[x\colon s\mapsto s+\frac a{4s}.\]
%         Now, $x(s_1)=x(s_2)$ if and only if $s_1+\frac a{4s_1}=s_2+\frac a{4s_2}$, which upon clearing fractions and rearranging is equivalent to
%         \[(4s_1s_2-a)(s_1-s_2)=0.\]
%         This is now equivalent to $s_1=s_2$ or $s_1=\frac a{4s_2}$. Thus, for each $s\in\FF_q^\times$, we see that $x^{-1}(\{x(s)\})=\{s,a/(4s)\}$, a set which has size $2$ unless $a$ is a square and $s$ is a square root of $a/4$.

%         To finish, we see that if $a$ is not a square, there are $\frac{q-1}2$ values of $x$. Otherwise, $a$ is a square, and there are two fibers with exactly one element, totaling to $\frac{q-3}2+2=\frac{q+1}2$ values of $x$. This completes the proof.
%     \end{proof}
%     We now have the following cases on $b$ and $c$.
%     \begin{itemize}
%         \item Suppose $b$ and $c$ are not squares. Then $y^2-b$ and $z^2-c$ are always nonzero, so for $\left(y^2-b\right)\left(z^2-c\right)$ to be a square, either both are squares or neither are squares. Each such pair $(y,z)$ produces two valid values of $x$, so we have counted
%         \[2\left(\left(\frac{q-1}2\right)^2+\left(\frac{q+1}2\right)^2\right)=q^2+1\]
%         triples $(x,y,z)$ in this case.
%         \item Suppose exactly one of $b$ or $c$ is a square; without loss of generality, say that $b$ is a square. There are two values of $y$ for which $y^2-b$ vanishes, from which $z$ has any value and $x=0$, totaling to $2q$ solutions here.

%         Continuing, there are $\frac{q-3}2$ additional values of $y$ for which $y^2-b$ is a nonzero square; here, $z^2-c$ must be a (nonzero) square, giving
%         \[2\left(\frac{q-3}2\right)\left(\frac{q-1}2\right)=\frac{q^2-4q+3}2\]
%         additional solutions.

%         Lastly, there are $\frac{q-1}2$ values of $y$ for which $y^2-b$ is not a square; here $z^2-c$ must not be a square, giving
%         \[2\left(\frac{q-1}2\right)\left(\frac{q+1}2\right)=\frac{q^2-1}2\]
%         more solutions. Summing all three cases gives $2q+\frac12\left(q^2-4q+3\right)+\frac12\left(q^2-1\right)=q^2+1$ solutions.
%         \item Suppose that both $b$ and $c$ are squares. There are two values of $y$ for which $y^2-b$ from which $z$ has any value and $x=0$, totaling to $2q$ solutions. There are two values for $z$ for which $z^2-c$ vanishes again, which adds $2q-4$ more solutions.

%         In the remaining cases, both $y^2-b$ and $z^2-c$ must be nonzero. For their product to be a square, either both are squares or neither is a square, so we have counted
%         \[2\left(\left(\frac{q-3}2\right)^2+\left(\frac{q-1}2\right)^2\right)=q^2-4q+5\]
%         more solutions. In total, there are $2q-4+q^2-4q+5=q^2+1$ solutions.
%     \end{itemize}
%     The above casework completes the proof of \Cref{lem:nonzero-type}.
% \end{proof}
% We are now ready to prove \Cref{thm:symmetric-gauss-sum-easy}.
% \begin{proof}[Proof of \Cref{thm:symmetric-gauss-sum-easy}]
%     For given $t\in\FF_q$, we will count $A\in\Sym_3^\times(\FF_q)$ such that $\tr A=t$. We have two cases.
%     \begin{itemize}
%         \item Suppose $t=0$. Then the type of any $A\in\Sym_3^\times(\FF_q)$ has one of the following forms.
%         \begin{itemize}
%             \item Type $(0,0,0)$: there are $(q-1)^3$ matrices here.
%             \item Permutations of type $(0,a,-a)$ for given $a\in\FF_q^\times$: there are $q^3-q-(q-1)^2$ matrices.
%             \item Type $(a,b,-a-b)$ for given $a,b,-a-b\in\FF_q^\times$: there are $q^3-\left(q^2+1\right)$ matrices.
%         \end{itemize}
%         Totaling all cases, we have
%         \[\left(q-1\right)^{3}+3\left(q-1\right)\left(q^{3}-q-\left(q-1\right)^{2}\right)+\left(q-1\right)\left(q-2\right)\left(q^{3}-q^{2}-1\right)\]
%         matrices. Simplifying, this is $q^5-q^4$.
%         \item Suppose $t\ne0$. Then the type of any $A\in\Sym_3^\times(\FF_q)$ has one of the following forms.
%         \begin{itemize}
%             \item Permutations of type $(t,0,0)$: there are $q^3-q^2-(q-1)^2$ matrices here.
%             \item Permutations of type $(a,t-a,0)$ for given $a,t-a\in\FF_q^\times$: there are $q^3-q-(q-1)^2$ matrices.
%             \item Type $(a,b,t-a-b)$ for given $a,b,t-a-b\in\FF_q^\times$: there are $q^3-\left(q^2+1\right)$ matrices.

%             Quickly, note that $a\notin\{0,t\}$ requires $b\notin\{0,t-a\}$ and hence $q-2$ options for $b$; otherwise $a=t$ requires $b\ne0$ and hence $q-1$ options for $b$.
%         \end{itemize}
%         Totaling all cases, we have
%         \[3\left(q^3-q^2-(q-1)^2\right)+3(q-2)\left(q^3-q-(q-1)^2\right)+((q-2)(q-2)+(q-1))\left(q^3-\left(q^2+1\right)\right)\]
%         matrices. Simplifying, this is $q^{5}-q^{4}-q^{2}$.
%     \end{itemize}
%     Combining cases, we see
%     \begin{align*}
%         \sum_{A\in\Sym_3^\times(\FF_q)}\psi(\tr A) &= \sum_{t\in\FF_q}\#\left\{A\in\Sym_3^\times(\FF_q):\tr A=t\right\}\psi(t) \\
%         &= q^2\psi(0)+\sum_{t\in\FF_q}\left(q^5-q^4-q^2\right)\psi(t) \\
%         &= q^2,
%     \end{align*}
%     which is what we wanted.
% \end{proof}