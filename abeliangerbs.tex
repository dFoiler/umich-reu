\documentclass{article}
\usepackage[utf8]{inputenc}

\newcommand{\nirpdftitle}{Abelian Extensions}
\usepackage{import}
\inputfrom{../notes}{nir}
\numberwithin{equation}{section}

\pagestyle{contentpage}

\title{Classifying Extensions of Abelian Groups}
\author{Nir Elber}
\date{\today}
\usepackage{graphicx}
\lhead{}
\rhead{\textit{ABELIAN EXTENSIONS}}

\begin{document}

\maketitle

\begin{abstract}
	\noindent We use group cohomology to provide some general theory to classify all group extensions of a $ G$-module $A$ in the case of an abelian group $ G$. The main idea is to provide a group presentation of the extension using specially chosen elements of $A$.
\end{abstract}

\setcounter{tocdepth}{4}
\tableofcontents

\section{General Group Extensions} \label{sec:general}
Throughout this section, $ G$ will be a finite group and $A$ will be a $ G$-module; we will write the group operation of $A$ and the group action of $ G$ on $A$ multiplicatively. To sketch the idea here, begin with an extension
\[1\to A\to\mc E\stackrel\pi\to G\to1.\]
We know that we can abstractly represent $\mc E$ as the set $A\times G$ with some group law dictated by a $2$-cocycle in $H^2(G,A)$, so we expect that $\mc E$ can be presented by $A$ and a choice of lifts from $ G$, with some specially chosen relations.

Here are some basic observations realizing this idea. We start by lifting a single element of $ G$.
\begin{lemma} \label{lem:constructalpha}
	Let $A$ be a $ G$-module, and let 
	\[1\to A\to\mc E\stackrel\pi\to G\to1\]
	denote a group extension. Further, fix some $\sigma\in G$ of order $n_\sigma$, and find $F\in\mc E$ such that $\sigma\coloneqq\pi(F)$. Then
	\[\alpha\coloneqq F^{n_\sigma}\]
	has $\alpha\in A^{\langle\sigma\rangle}$.
\end{lemma}
\begin{proof}
	A priori, we only know that $\alpha\in\mc E$, so we compute
	\[\pi(\alpha)=\pi\left(F^{n_\sigma}\right)=\sigma^{n_\sigma}=1,\]
	so $\alpha\in\ker\pi=A$. Thus, we may say that
	\[\sigma(\alpha)=F\alpha F^{-1}=F^{n_\sigma}=\alpha,\]
	so $\alpha\in A^{\langle\sigma\rangle}$, as desired.
\end{proof}
We can make the above proof more explicit by specifying the group law of $\mc E$.
\begin{lemma} \label{lem:explicitalpha}
	Let $A$ be a $ G$-module. Picking up some $2$-cocycle $c\in Z^2( G,A)$, let
	\[1\to A\to\mc E\stackrel\pi\to G\to1\]
	be the corresponding extension. Fixing $\sigma\in G$ of order $n_\sigma$, let $F\coloneqq(m,\sigma)\in\mc E$ be a lift. Then
	\[\alpha\coloneqq F^{n_\sigma}=N_\sigma(m)\prod_{i=0}^{n_\sigma-1}c\left(\sigma^i,\sigma\right),\]
	where $N_\sigma\coloneqq\sum_{i=0}^{n_\sigma-1}\sigma^i$.
\end{lemma}
\begin{proof}
	This is a direct computation. By induction, we have that
	\[F^k=\left(\prod_{i=0}^{k-1}\sigma^i(m)c\left(\sigma^i,\sigma\right),\sigma^k\right)\]
	for $k\in\NN$. Indeed, there is nothing to say for $k=0$, and the inductive step merely expands out $F^k\cdot F$.

	It follows that
	\[\alpha=F^{n_\sigma}=\left(\prod_{i=0}^{n_\sigma-1}\sigma^i(m)\cdot\prod_{i=0}^{n_\sigma-1}c\left(\sigma^i,\sigma\right),1\right),\]
	which is what we wanted.
\end{proof}
Having this explicit formula lets us say how $\alpha$ changes as we vary the lift.
\begin{prop} \label{prop:findallalpha}
	Let $A$ be a $ G$-module. Fixing a cohomology class $u\in H^2( G,A)$, let 
	\[1\to A\to\mc E\stackrel\pi\to G\to1\]
	be a group extension whose isomorphism class corresponds to $u$. Further, fix some $\sigma\in G$ of order $n_\sigma$, and let $A_\sigma\coloneqq A^{\langle\sigma\rangle}$ be the fixed submodule. Then the set
	\[S_{\mc E,\sigma}\coloneqq\left\{F^{n_\sigma}:\pi(F)=\sigma\right\}\]
	is an equivalence class in $A_\sigma/N_\sigma(A)$, independent of the choice of $\mc E$. Again, $N_\sigma\coloneqq\sum_{i=1}^{n_\sigma-1}\sigma^i$.
\end{prop}
\begin{proof}
	Note that $S_{\mc E,\sigma}\subseteq A_\sigma$ already from \autoref{lem:constructalpha}.
	
	The point is to use \autoref{lem:explicitalpha}. Note the extension $\mc E$ corresponds to the equivalence class $u\in H^2( G,A)$, so let $c\in Z^2( G,A)$ be a representative. Letting $\mc E_c$ be the extension constructed from $c$, we are promised an isomorphism $\varphi\colon\mc E\simeq\mc E_c$ making the following diagram commute.
	% https://q.uiver.app/?q=WzAsMTAsWzAsMCwiMSJdLFsxLDAsIkxeXFx0aW1lcyJdLFsyLDAsIlxcbWMgRSJdLFszLDAsIlxcR2FtbWEiXSxbNCwwLCIxIl0sWzAsMSwiMSJdLFsxLDEsIkxeXFx0aW1lcyJdLFsyLDEsIlxcbWMgRV9jIl0sWzMsMSwiXFxHYW1tYSJdLFs0LDEsIjEiXSxbMCwxXSxbMSwyXSxbMiwzLCJcXHBpIl0sWzMsNF0sWzUsNl0sWzYsN10sWzcsOCwiXFxwaV9jIl0sWzgsOV0sWzIsNywiXFx2YXJwaGkiXSxbMSw2LCIiLDEseyJsZXZlbCI6Miwic3R5bGUiOnsiaGVhZCI6eyJuYW1lIjoibm9uZSJ9fX1dLFszLDgsIiIsMSx7ImxldmVsIjoyLCJzdHlsZSI6eyJoZWFkIjp7Im5hbWUiOiJub25lIn19fV1d&macro_url=https%3A%2F%2Fraw.githubusercontent.com%2FdFoiler%2Fnotes%2Fmaster%2Fnir.tex
	\[\begin{tikzcd}
		1 & {A} & {\mc E} &  G & 1 \\
		1 & {A} & {\mc E_c} &  G & 1
		\arrow[from=1-1, to=1-2]
		\arrow[from=1-2, to=1-3]
		\arrow["\pi", from=1-3, to=1-4]
		\arrow[from=1-4, to=1-5]
		\arrow[from=2-1, to=2-2]
		\arrow[from=2-2, to=2-3]
		\arrow["{\pi_c}", from=2-3, to=2-4]
		\arrow[from=2-4, to=2-5]
		\arrow["\varphi", from=1-3, to=2-3]
		\arrow[Rightarrow, no head, from=1-2, to=2-2]
		\arrow[Rightarrow, no head, from=1-4, to=2-4]
	\end{tikzcd}\]
	We start by claiming that $S_{\mc E,\sigma}=S_{\mc E_c,\sigma}$, which will show that $S_{\mc E,\sigma}$ is independent of the choice of representative $\mc E$. To show $S_{\mc E,\sigma}\subseteq S_{\mc E_c,\sigma}$, note that $\alpha\in S_{\mc E,\sigma}$ has $F\in\mc E$ with $\pi(F)=\sigma$ and $\alpha=F^{n_\sigma}$. Pushing this through $\varphi$, we see $\varphi(F)\in\mc E_c$ has
	\[\pi_c(\varphi(F))=\varphi(\pi(F))=\sigma\qquad\text{and}\qquad\varphi(F)^{n_\sigma}=\varphi(F^{n_\sigma})=\alpha,\]
	so $\alpha\in S_{\mc E_c,\sigma}$ follows. An analogous argument with $\varphi^{-1}$ shows the other needed inclusion.

	It thus suffices to show that $S_{\mc E_c,\sigma}$ is an equivalence class in $A_\sigma/N_\sigma(A)$. However, this is exactly what \autoref{lem:explicitalpha} says as we let the possible lifts $F=(m,\sigma)\in\mc E_c$ of $\sigma$ vary over $m\in A$.
\end{proof}
The fact that we are taking elements of $ G$ to equivalence classes in $A_\sigma^\times/N_\sigma\left(A\right)$ is reminiscent of the (inverse) Artin reciprocity map, and indeed that is exactly what is going on.
\begin{cor} \label{cor:alphaiscupproduct}
	Work in the context of \autoref{prop:findallalpha}. Then
	\[S_\sigma\coloneqq S_{\mc E,\sigma}=[\sigma]\cup[c],\]
	where $\cup\colon\widehat H^{-2}( G,A)\times\widehat H^2( G,A)\to\widehat H^0( G,A)$ is the cup product in Tate cohomology.
\end{cor}
\begin{proof}
	Using notation as in the proof of \autoref{prop:findallalpha}, we recall that $S_\sigma=S_{\mc E_c,\sigma}$, so it suffices to prove the result for $\mc E_c$. Well, by \autoref{lem:explicitalpha}, $S_\sigma$ is represented by
	\[\prod_{i=0}^{n_\sigma-1}c\left(\sigma^i,\sigma\right).\]
	However, this product is exactly the cup product $[\sigma]\cup[c]$.
\end{proof}
\begin{cor}
	Let $L/K$ be a finite Galois extension of local fields with Galois group $ G\coloneqq\op{Gal}(L/K)$. Further, let
	\[1\to L^\times\to\mc E\stackrel\pi\to G\to1\]
	be an $L/K$-gerb bound by $\mathbb G_m$ whose isomorphism class corresponds to the fundamental class $u_{L/K}\in H^2( G,L^\times)$. Further, fix some $\sigma\in G$ of order $n_\sigma$, and let $L_\sigma\coloneqq L^{\langle\sigma\rangle}$ be the fixed field. Then
	\[\theta_{L/L_\sigma}^{-1}(\sigma)=\left\{F^{n_\sigma}:\pi(F)=\sigma\right\}.\]
\end{cor}
\begin{proof}
	Recalling $\theta_{L/L_\sigma}^{-1}$ is a cup product map, note that $\theta_{L/L_\sigma}^{-1}(\sigma)$ is given by $[\sigma]\cup u_{L/K}$. So we are done by \autoref{cor:alphaiscupproduct}.
\end{proof}
The above results are all interested in lifting single elements of $ G$ and studying how they behave on their own. In the discussion that follows, we will need to study how the lifts interact with each other, but for now, we will justify why lifts are adequate to study as follows.
\begin{proposition} \label{prop:liftsgenerate}
	Let $A$ be a $ G$-module. Further, let
	\[1\to A\to\mc E\stackrel\pi\to G\to1\]
	be a group extension. Given elements $\{\sigma_i\}_{i=1}^m$ which generate $ G$, then $\mc E$ is generated by $A$ and a set of lifts $\{F_i\}_{i=1}^m$ with $\pi(F_i)=\sigma_i$ for each $i$.
\end{proposition}
\begin{proof}
	Fix some element $e\in\mc E$, which we need to exhibit as a product of elements in $A$ and $F_i$s. Well, because the $\sigma_i$ generate $ G$, we know that $\pi(e)\in G$ can be written as
	\[\pi(e)=\prod_{i=1}^m\sigma_i^{a_i}\]
	for some sequence of integers $\{a_i\}_{i=1}^m$. It follows that
	\[\pi\left(\frac e{\prod_{i=1}^mF_i^{a_i}}\right)=1,\]
	so $\frac e{\prod_{i=1}^mF_i^{a_i}}=\ker\pi=A$. Thus, we can find some $x\in A$ such that
	\[e=x\cdot\prod_{i=1}^mF_i^{a_i},\]
	which is what we wanted.
\end{proof}

\section{Abelian Group Extensions} \label{sec:abelian}

\subsection{Extensions to Tuples}
The above proofs technically don't even require that the group $ G$ is abelian. If we want to keep track of the fact our group is abelian, we should extract the elements of $A$ which can do so.
\begin{lemma} \label{lem:constructalphabeta}
	Let $A$ be a $ G$-module, and let 
	\[1\to A\to\mc E\stackrel\pi\to G\to1\]
	be a group extension. Further, fix some $F_1,F_2\in\mc E$ and define $\sigma_i\coloneqq\pi(F_i)$ for $i\in\{1,2\}$, and let $\sigma_i\in G$ have order $n_i$. Then, setting
	\[\alpha_i\coloneqq F_i^{n_i}\qquad\text{and}\qquad\beta\coloneqq F_1F_2F_1^{-1}F_2^{-1},\]
	we have the following.
	\begin{listalph}
		\item $\alpha_i\in A^{\langle\sigma_i\rangle}$ for $i\in\{1,2\}$ and $\beta\in A$.
		\item $N_1(\beta)=\alpha_1/\sigma_2(\alpha_1)$ and $N_2(\beta^{-1})=\alpha_2/\sigma_1(\alpha_2)$, where $N_i\coloneqq\sum_{p=0}^{n_i-1}\sigma_i^p$.
	\end{listalph}
\end{lemma}
\begin{proof}
	These checks are a matter of force. For brevity, we set $A_i\coloneqq A^{\langle\sigma_i\rangle}$ for $i\in\{1,2\}$.
	\begin{listalph}
		\item That $\alpha_i\in A_i$ follows from \autoref{lem:constructalpha}. Lastly, $\beta\in A$ follows from noting
		\[\pi(\beta)=\pi(F_1)\pi(F_2)\pi(F_1)^{-1}\pi(F_2)^{-1}=1,\]
		so $\beta\in\ker\pi=A$.
		\item We will check that $\op N_{L/L_1}(\beta)=\alpha_1/\sigma_2(\alpha_1)$; the other equality follows symmetrically after switching $1$s and $2$s because $\beta^{-1}=F_2F_1F_2^{-1}F_1^{-1}$. Well, we compute
		\begin{align*}
			N_1(\beta) &= \sigma_1^{-1}(\beta)\cdot\sigma_1^{-2}(\beta)\cdot\sigma^{-3}\cdot\ldots\cdot\sigma^{-n_1}(\beta) \\
			&= F_1^{-1}\left(F_1F_2F_1^{-1}F_2^{-1}\right)F_1 \\
			&\phantom{{}={}}\cdot F_1^{-2}\left(F_1F_2F_1^{-1}F_2^{-1}\right)F_1^2 \\
			&\phantom{{}={}}\cdot F_1^{-3}\left(F_1F_2F_1^{-1}F_2^{-1}\right)F_1^3\cdot\ldots \\
			&\phantom{{}={}}\cdot F_1^{-n_1}(F_1F_2F_1^{-1}F_2^{-1})F_1^{n_1} \\
			% &= F_2F_1^{-1}F_2^{-1} \\
			% &\phantom{{}={}}\cdot F_2F_1^{-1}F_2^{-1} \\
			% &\phantom{{}={}}\cdot F_2F_1^{-1}F_2^{-1}\cdot\ldots \\
			% &\phantom{{}={}}\cdot F_2F_1^{-1}F_2^{-1}F_1^{n_1} \\
			&= F_2F_1^{-1} \\
			&\phantom{{}={}}\cdot F_1^{-1} \\
			&\phantom{{}={}}\cdot F_1^{-1}\cdot\ldots \\
			&\phantom{{}={}}\cdot F_1^{-1}F_2^{-1}F_1^{n_1} \\
			&= F_2F_1^{-n_1}F_2^{-1}F_1^{n_1} \\
			&= \alpha_1/\sigma_2(\alpha_1).
		\end{align*}
	\end{listalph}
	The above computations finish the proof.
\end{proof}
The proof of (b) above might appear magical, but in fact it comes from a more general idea.
\begin{lemma} \label{lem:switchtwo}
	Fix everything as in \autoref{lem:constructalphabeta}. Then, for $x,y\ge0$, we have
	\[F_1^xF_2^y=\prod_{k=0}^{x-1}\prod_{\ell=0}^{y-1}\sigma_1^k\sigma_2^\ell(\beta)F_2^yF_1^x.\]
\end{lemma}
\begin{proof}
	We induct. We take a moment to write out the case of $x=1$, for which we induct on $y$. To be explicit, we will prove
	\[F_1F_2^y=\prod_{\ell=0}^{y-1}\sigma_2^\ell(\beta)F_2^yF_1.\]
	For $y=0$, there is nothing to say. So suppose the statement for $y$ (and $x=1$), and we show $y+1$ (and $x=1$). Well, we compute
	\begin{align*}
		F_1F_2^{y+1} &= F_1F_2^y\cdot F_2 \\
		&= \prod_{\ell=0}^{y-1}\sigma_2^\ell(\beta)F_2^yF_1\cdot F_2 \\
		&= \prod_{\ell=0}^{y-1}\sigma_2^\ell(\beta)F_2^y\beta F_2F_1 \\
		&= \prod_{\ell=0}^{y-1}\sigma_2^\ell(\beta)\cdot \sigma_2^y(\beta)F_2^y\cdot F_2F_1 \\
		&= \prod_{\ell=0}^{(y+1)-1}\sigma_2^\ell(\beta)\cdot F_2^{y+1}F_1,
	\end{align*}
	which is what we wanted.
	
	We now move on to the general case. We will induct on $y$. Note that $y=0$ makes the product empty, leaving us with $F_1^x=F_1^x$, for any $x$. So suppose that the statement is true for some $y\ge0$, and we will show $y+1$. For this, we now turn to inducting on $x$. For $x=0$, we note that the product is once again empty, so we are left with showing $F_2^{y+1}=F_2^{y+1}$, which is true.
	
	To finish, we suppose the statement for $x$ and show the statement for $x+1$. Well, we compute
	\begin{align*}
		F_1^{x+1}F_2^{y+1} &= F_1\cdot F_1^xF_2^{y+1} \\
		&= F_1\cdot \prod_{k=0}^{x-1}\prod_{\ell=0}^{(y+1)-1}\sigma_1^k\sigma_2^\ell(\beta)\cdot F_2^{y+1}F_1^x \\
		&= \sigma_1\left(\prod_{k=0}^{x-1}\prod_{\ell=0}^{(y+1)-1}\sigma_1^k\sigma_2^\ell(\beta)\right)\cdot F_1F_2^{y+1}F_1^x \\
		&= \prod_{k=1}^{(x+1)-1}\prod_{\ell=0}^{(y+1)-1}\sigma_1^k\sigma_2^\ell(\beta)\cdot F_1F_2^{y+1}F_1^x \\
		&= \prod_{k=1}^{(x+1)-1}\prod_{\ell=0}^{(y+1)-1}\sigma_1^k\sigma_2^\ell(\beta)\cdot \prod_{\ell=0}^{(y+1)-1}\sigma_2^\ell(\beta)\cdot \sigma_2^y(\beta)\cdot F_2^{y+1}F_1\cdot F_1^x \\
		&= \prod_{k=0}^{(x+1)-1}\prod_{\ell=0}^{(y+1)-1}\sigma_1^k\sigma_2^\ell(\beta)F_2^{y+1}F_1^{x+1},
	\end{align*}
	which is what we wanted.
\end{proof}
\begin{remark} \label{rem:alphabetarelation}
	Setting $x=n_1$ and $y=1$ recovers $\op N_{L/L^{\langle\sigma_1\rangle}}(\beta)=\alpha_1/\sigma_2(\alpha_1)$.
\end{remark}
In particular, \autoref{rem:alphabetarelation} tells us that coherence of the group law in $\mc E$ should give rise to relations between our elements of $A$. Here is a more complex example.
\begin{lemma} \label{lem:betarelations}
	Let $A$ be a $ G$-module, and let 
	\[1\to A\to\mc E\stackrel\pi\to G\to1\]
	be a group extension. Further, fix some $F_1,F_2,F_3\in\mc E$ and define $\sigma_i\coloneqq\pi(F_i)$ for $i\in\{1,2,3\}$, and let $\sigma_i\in G$ have order $n_i$. Then, setting
	\[\beta_{ij}\coloneqq F_iF_jF_i^{-1}F_j^{-1}\]
	for each pair of indices $(i,j)$ with $i>j$. Then
	\[\frac{\sigma_2(\beta_{31})}{\beta_{31}}=\frac{\sigma_1(\beta_{32})}{\beta_{32}}\cdot\frac{\sigma_3(\beta_{21})}{\beta_{21}}.\]
\end{lemma}
\begin{proof}
	The point is to turn $F_3F_2F_1$ into $F_1F_2F_3$ in two different ways. On one hand,
	\begin{align*}
		(F_3F_2)F_1 &= \beta_{32}F_2F_3F_1 \\
		&= \beta_{32}F_2\beta_{31}F_1F_3 \\
		&= \beta_{32}\sigma_2(\beta_{31})(F_2F_1)F_3 \\
		&= \beta_{32}\sigma_2(\beta_{31})\beta_{21}F_1F_2F_3.
	\end{align*}
	On the other hand,
	\begin{align*}
		F_3(F_2F_1) &= F_3\beta_{21}F_1F_2 \\
		&= \sigma_3(\beta_{21})(F_3F_1)F_2 \\
		&= \sigma_3(\beta_{21})\beta_{31}F_1(F_3F_2) \\
		&= \sigma_3(\beta_{21})\beta_{31}F_1\beta_{32}F_2F_3 \\
		&= \sigma_3(\beta_{21})\beta_{31}\sigma_1(\beta_{32})F_1F_2F_3.
	\end{align*}
	Thus,
	\[\beta_{32}\sigma_2(\beta_{31})\beta_{21}=\sigma_3(\beta_{21})\beta_{31}\sigma_1(\beta_{32}),\]
	which rearranges into the desired equation.
\end{proof}
\begin{remark}
	The relation from \autoref{lem:betarelations} may look asymmetric in the $\beta_{ij}$, but this is because the definitions of the $\beta_{ij}$s themselves are asymmetric in $F_i$.
\end{remark}

\subsection{Tuples to Cocycles}
\subsubsection{The Set-Up}
The proceeding lemma is intended to give intuition that the element $\beta$ is helping to specify the group law on $\mc E$.

More concretely, we will take the following set-up for the following results: fix a $ G$-module $A$, and let
\[1\to A\to\mc E\to G\to1\]
be a group extension. Once we choose elements $\{\sigma_i\}_{i=1}^m$ generating $ G$, we know by \autoref{prop:liftsgenerate} that we can generate $\mc E$ by $A$ and some arbitrarily chosen lifts $\{F_i\}_{i=1}^m$ of the $\{\sigma_i\}_{i=1}^m$. Then, letting $n_i$ be the order of $\sigma_i$, we set
\[\alpha_i\coloneqq F_i^{n_i}\]
for each index $i$ and
\[\beta_{ij}\coloneqq F_iF_jF_i^{-1}F_j^{-1}\]
for each index $1\le j<i\le m$. Notably, we will not need more $\beta$s: indeed, $\beta_{ii}=1$ and $\beta_{ij}=\beta_{ji}^{-1}$ for any $i$ and $j$. Setting $A_i\coloneqq A^{\langle\sigma_i\rangle}$ and $N_i\coloneqq\sum_{p=0}^{n_i-1}\sigma_i^p$, the story so far is that
\begin{equation}
	\alpha_i\in A_i\text{ for each }i\qquad\text{and}\qquad\beta_{ij}\in A\text{ for each }i>j \label{eq:tuplefields}
\end{equation}
and
\begin{equation}
	N_i(\beta_{ij})=\alpha_i/\sigma_j(\alpha_i)\qquad\text{and}\qquad N_j(\beta_{ij}^{-1})=\alpha_j/\sigma_i(\alpha_j)\qquad\text{ for each }i>j \label{eq:tuplerelations}
\end{equation}
by \autoref{lem:constructalphabeta}, and
\begin{equation}
	\frac{\sigma_j(\beta_{ik})}{\beta_{ik}}=\frac{\sigma_k(\beta_{ij})}{\beta_{ij}}\cdot\frac{\sigma_i(\beta_{jk})}{\beta_{jk}}\qquad\text{ for each }i>j>k \label{eq:betarelations}
\end{equation}
by \autoref{lem:betarelations}. This data is so important that we will give it a name.
\begin{definition}
	In the above set-up, the data of $(\{\alpha_i\},\{\beta_{ij}\})$ satisfying \autoref{eq:tuplefields} and \autoref{eq:tuplerelations} and \autoref{eq:betarelations} will be called a \textit{$\{\sigma_i\}_{i=1}^m$-tuple}. When understood, the $\{\sigma_i\}_{i=1}^m$ will be abbreviated.
\end{definition}
Note that this definition is independent of $\mc E$, but a choice of extension $\mc E$ and lifts $F_i$ give a $\{\sigma_i\}_{i=1}^m$-tuple as described above.
\begin{remark}
	The set of $\{\sigma_i\}_{i=1}^m$-tuples form a group under multiplication in $A$. Indeed, the conditions \autoref{eq:tuplefields} and \autoref{eq:tuplerelations} and \autoref{eq:betarelations} are closed under multiplication and inversion.
\end{remark}
We also know from \autoref{lem:switchtwo} that
\[F_i^xF_j^y=\prod_{k=0}^{x-1}\prod_{\ell=0}^{y-1}\sigma_i^k\sigma_j^\ell(\beta_{ij})F_j^yF_i^x\]
for $i>j$ and $x,y\ge0$. It will be helpful to have some notation for the residue term in $A$, so we define
\[\beta_{ij}^{(k\ell)}\coloneqq\prod_{k=0}^{x-1}\prod_{\ell=0}^{y-1}\sigma_i^k\sigma_j^\ell(\beta_{ij}).\]
Now, combined with the fact that $F_ix=\sigma_i(x)F_i$ for each $F_i$ and $x\in A$, we have been approximately told how the group operation works in $\mc E$. Namely, we could conceivably write any element of $\mc E$ in the form
\[xF_1^{a_1}\cdots F_m^{a_m}\]
for $x\in A$ and $a_i\in\ZZ/n_i\ZZ$ because we know how to make these elements commute and generate $\mc E$. Further, we can multiply out two terms of the form
\[xF_1^{a_1}\cdots F_m^{a_m}\cdot yF_1^{b_1}\cdots F_m^{b_m}\]
into a term of the form $zF_1^{c_1}\cdots F_m^{c_m}$. In fact, it will be helpful for us to see how to do this.
\begin{proposition} \label{prop:multiplytwoelements}
	Fix everything as in the set-up, except drop the assumption that $\{\sigma_i\}_{i=1}^m$ generate $ G$. Then, choosing $a_i,b_i\in\NN$ for each $i$, we have
	\[\left(\prod_{i=1}^mF_i^{a_i}\right)\left(\prod_{i=1}^mF_i^{b_i}\right)=\left[\prod_{1\le j<i\le m}\Bigg(\prod_{1\le k<j}\sigma_k^{a_k+b_k}\Bigg)\Bigg(\prod_{j\le k<i}\sigma_k^{a_k}\Bigg)\beta_{ij}^{(a_ib_j)}\right]\left(\prod_{i=1}^mF_i^{a_i+b_i}\right).\]
\end{proposition}
\begin{proof}
	The reason that we dropped the assumption on $\{\sigma_i\}_{i=1}^m$ is so that we may induct directly on $m$. We start by showing that
	\[\left(\prod_{i=1}^mF_i^{a_i}\right)F_1^{b_1}=\left[\prod_{1<i\le m}\left(\prod_{1\le k<i}\sigma_k^{a_k}\right)\beta_{i1}^{(a_ib_1)}\right]F_1^{a_1+b_1}\prod_{i=2}^mF_i^{a_i}.\]
	We do this by induction on $m$. When $m=0$ and even for $m=1$, there is nothing to say. For the inductive step, we assume
	\[\left(\prod_{i=1}^mF_i^{a_i}\right)F_1^{b_1}=\left[\prod_{1<i\le m}\left(\prod_{1\le k<i}\sigma_k^{a_k}\right)\beta_{i1}^{(a_ib_1)}\right]F_1^{a_1+b_1}\prod_{i=2}^mF_i^{a_i}\]
	and compute
	\begin{align*}
		\left(\prod_{i=1}^{m+1}F_i^{a_i}\right)F_1^{b_1} &= \left(\prod_{i=1}^{m}F_i^{a_i}\right)F_{m+1}^{a_{m+1}}F_1^{b_1} \\
		&= \left(\prod_{i=1}^{m}F_i^{a_i}\right)\beta_{m+1,1}^{(a_{m+1}b_1)}F_1^{b_1}F_{m+1}^{a_{m+1}} \\
		&= \left[\left(\prod_{k=1}^m\sigma_k^{a_k}\right)\beta_{m+1,1}^{(a_{m+1}b_1)}\right]\left[\prod_{1<i\le m}\left(\prod_{1\le k<i}\sigma_k^{a_k}\right)\beta_{i1}^{(a_ib_1)}\right]F_1^{a_1+b_1}\left(\prod_{i=2}^mF_i^{a_i}\right)F_{m+1}^{a_{m+1}} \\
		&= \left[\prod_{1<i\le m+1}\left(\prod_{1\le k<i}\sigma_k^{a_k}\right)\beta_{i1}^{(a_ib_1)}\right]F_1^{a_1+b_1}\left(\prod_{i=2}^{m+1}F_i^{a_i}\right),
	\end{align*}
	which completes our inductive step.

	We now attack the statement of the proposition directly, again inducting on $m$. For $m=0$ and even for $m=1$, there is again nothing to say. For the inductive step, take $m>1$, and we get to assume that
	\[\left(\prod_{i=2}^mF_i^{a_i}\right)\left(\prod_{i=2}^mF_i^{b_i}\right)=\left[\prod_{2\le j<i\le m}\Bigg(\prod_{2\le k<j}\sigma_k^{a_k+b_k}\Bigg)\Bigg(\prod_{j\le k<i}\sigma_k^{a_k}\Bigg)\beta_{ij}^{(a_ib_j)}\right]\left(\prod_{i=2}^mF_i^{a_i+b_i}\right).\]
	From here, we can compute
	\begin{align*}
		\left(\prod_{i=1}^mF_i^{a_i}\right)\left(\prod_{i=1}^mF_i^{b_i}\right) &= \left(\prod_{i=1}^mF_i^{a_i}\right)F_1^{b_1}\left(\prod_{i=2}^mF_i^{b_i}\right) \\
		&= \left[\prod_{1<i\le m}\Bigg(\prod_{1\le k<i}\sigma_k^{a_k}\Bigg)\beta_{i1}^{(a_ib_1)}\right]F_1^{a_1+b_1}\left(\prod_{i=2}^mF_i^{a_i}\right)\left(\prod_{i=2}^mF_i^{b_i}\right) \\
		&= \left[\prod_{1<i\le m}\Bigg(\prod_{1\le k<i}\sigma_k^{a_k}\Bigg)\beta_{i1}^{(a_ib_1)}\right]F_1^{a_1+b_1}\cdot \\
		&\qquad\qquad\left[\prod_{2\le j<i\le m}\Bigg(\prod_{2\le k<j}\sigma_k^{a_k+b_k}\Bigg)\Bigg(\prod_{j\le k<i}\sigma_k^{a_k}\Bigg)\beta_{ij}^{(a_ib_j)}\right]\left(\prod_{i=2}^mF_i^{a_i+b_i}\right) \\
		&= \left[\prod_{1<i\le m}\Bigg(\prod_{1\le k<i}\sigma_k^{a_k}\Bigg)\beta_{i1}^{(a_ib_1)}\right]\cdot \\
		&\qquad\qquad \sigma_1^{a_1+b_1}\left[\prod_{2\le j<i\le m}\Bigg(\prod_{2\le k<j}\sigma_k^{a_k+b_k}\Bigg)\Bigg(\prod_{j\le k<i}\sigma_k^{a_k}\Bigg)\beta_{ij}^{(a_ib_j)}\right]\left(\prod_{i=2}^mF_i^{a_i+b_i}\right).
	\end{align*}
	From here, a little rearrangement finishes the inductive step.
\end{proof}
The reason we exerted this pain upon ourselves is for the following result.
\begin{prop} \label{prop:writedowncocycle}
	Fix everything as in the set-up. Then, if well-defined, we can represent the cohomology class corresponding to $\mc E$ by the cocycle
	\[c(g,h)\coloneqq\left[\prod_{1\le j<i\le m}\Bigg(\prod_{1\le k<j}\sigma_k^{a_k+b_k}\Bigg)\Bigg(\prod_{j\le k<i}\sigma_k^{a_k}\Bigg)\beta_{ij}^{(a_ib_j)}\right]\left[\prod_{i=1}^m\Bigg(\prod_{1\le k<i}\sigma_k^{a_k+b_k}\Bigg)\alpha_i^{\floor{\frac{a_i+b_i}{n_i}}}\right],\]
	where $g=\prod_i\sigma_i^{a_i}$ and $h=\prod_i\sigma_i^{b_i}$.
\end{prop}
Observe that \autoref{prop:writedowncocycle} has a fairly strong hypothesis that $c$ is well-defined; we will return to this later.
\begin{proof}
	Very quickly, we use the division algorithm to define
	\[a_i+b_i=n_iq_i+r_i\]
	where $q_\in\{0,1\}$ and $0\le r_i<n_i$. In particular,
	\[gh=\prod_{i=1}^mF_i^{r_i}.\]
	Now, because the elements $\sigma_i$ generate $ G$, we see that the lifts $\sigma_i\mapsto F_i$ defines a section $s\colon G\to\mc E$. As such, we can compute a representing cocycle for our cohomology class as
	\begin{align*}
		c(g,h) &= s(g)s(h)s(gh)^{-1} \\
		&= \Bigg(\prod_{i=1}^mF_i^{a_i}\Bigg)\Bigg(\prod_{i=1}^mF_i^{b_i}\Bigg)\Bigg(\prod_{i=1}^mF_i^{r_i}\Bigg)^{-1} \\
		&= \left[\prod_{1\le j<i\le m}\Bigg(\prod_{1\le k<j}\sigma_k^{a_k+b_k}\Bigg)\Bigg(\prod_{j\le k<i}\sigma_k^{a_k}\Bigg)\beta_{ij}^{(a_ib_j)}\right]\left(\prod_{i=1}^mF_i^{a_i+b_i}\right)\Bigg(\prod_{i=1}^mF_{m-i+1}^{-r_{m-i+1}}\Bigg).
	\end{align*}
	It remains to deal with the last products; we claim that it is equal to
	\[\left(\prod_{i=1}^mF_i^{a_i+b_i}\right)\Bigg(\prod_{i=1}^mF_{m-i+1}^{-r_{m-i+1}}\Bigg)=\prod_{i=1}^m\Bigg(\prod_{1\le k<i}\sigma_k^{a_k+b_k}\Bigg)\alpha_i^{q_i},\]
	which will finish the proof. We induct on $m$; for $m=0$ and $m=1$, there is nothing to say. For the inductive step, we assume that
	\[\left(\prod_{i=2}^mF_i^{a_i+b_i}\right)\Bigg(\prod_{i=1}^{m-1}F_{m-i+1}^{-r_{m-i+1}}\Bigg)=\prod_{i=2}^m\Bigg(\prod_{2\le k<i}\sigma_k^{a_k+b_k}\Bigg)\alpha_i^{q_i}\]
	and compute
	\begin{align*}
		\left(\prod_{i=1}^mF_i^{a_i+b_i}\right)\Bigg(\prod_{i=1}^mF_{m-i+1}^{-r_{m-i+1}}\Bigg) &= F_1^{a_1+b_1}\left(\prod_{i=2}^mF_i^{a_i+b_i}\right)\Bigg(\prod_{i=1}^{m-1}F_{m-i+1}^{-r_{m-i+1}}\Bigg)F_1^{-a_1-b_1}F_1^{a_1+b_1-r_1} \\
		&= F_1^{a_1+b_1}\left(\prod_{i=2}^m\Bigg(\prod_{2\le k<i}\sigma_k^{a_k+b_k}\Bigg)\alpha_i^{q_i}\right)F_1^{-a_1-b_1}\alpha_1^{q_1} \\
		&= \left(\prod_{i=2}^m\Bigg(\prod_{1\le k<i}\sigma_k^{a_k+b_k}\Bigg)\alpha_i^{q_i}\right)\alpha_1^{q_1} \\
		&= \prod_{i=1}^m\Bigg(\prod_{1\le k<i}\sigma_k^{a_k+b_k}\Bigg)\alpha_i^{q_i},
	\end{align*}
	finishing.
\end{proof}

\subsubsection{The Modified Set-Up}
A priori we have no reason to expect that the $c$ constructed in \autoref{prop:writedowncocycle} is actually a cocycle, especially if the $\sigma_i$ have nontrivial relations.

To account for this, we modify our set-up slightly. By the classification of finitely generated abelian groups, we may write
\[ G\simeq\bigoplus_{k=1}^m G_k,\]
where $ G_k\subseteq G$ with $ G_k\cong\ZZ/n_k\ZZ$ and $n_k>1$ for each $n_k$. As such, we let $\sigma_k$ be a generating element of $ G_k$ so that we still know that the $\sigma_k$ generate $ G$. In this case, we have the following result.
\begin{theorem} \label{thm:getcocycle}
	Fix everything as in the modified set-up, forgetting about the extension $\mc E$. Then a $\{\sigma_i\}_{i=1}^m$-tuple of $\{\alpha_i\}_{i=1}^m$ and $\{\beta_{ij}\}_{i>j}$ makes
	\[c(g,h)\coloneqq\left[\prod_{1\le j<i\le m}\Bigg(\prod_{1\le k<j}\sigma_k^{a_k+b_k}\Bigg)\Bigg(\prod_{j\le k<i}\sigma_k^{a_k}\Bigg)\beta_{ij}^{(a_ib_j)}\right]\left[\prod_{i=1}^m\Bigg(\prod_{1\le k<i}\sigma_k^{a_k+b_k}\Bigg)\alpha_i^{\floor{\frac{a_i+b_i}{n_i}}}\right],\]
	where $g\coloneqq\prod_i\sigma_i^{a_i}$ with $h\coloneqq\prod_i\sigma_j^{a_j}$ and $0\le a_i,b_i<n_i$, into a cocycle in $Z^2( G,A)$.
\end{theorem}
\begin{proof}
	Note that $c$ is now surely well-defined because the elements $g$ and $h$ have unique representations as described. Anyway, we relegate the direct cocycle check to \autoref{sec:verifycocycle} because it is long, annoying, and unenlightening. We will also present an alternative proof in \autoref{sec:tuplestudy}, using more abstract theory.
\end{proof}
Observe that the above construction has now completely forgotten about $\mc E$! Namely, we have managed to go from tuples straight to cocycles; this is theoretically good because it will allow us to go fully in reverse: we will be able to start with a tuple, build the corresponding cocycle, from which the extension arises. However, equivalence classes of cocycles give the ``same'' extension, so we will also need to give equivalence classes for tuples as well.

\subsection{Building Tuples}
We continue in the modified set-up of the previous section. There is already an established way to get from a cocycle to an extension, which means that it should be possible to go straight from the cocycle to a $\{\sigma_i\}_{i=1}^m$-tuple. Again, it will be beneficial to write this out.
\begin{lemma} \label{lem:explicitalphabeta}
	Fix everything as in the modified set-up, but suppose that $\mc E=\mc E_c$ is the extension generated from a cocycle $c\in Z^2( G,A)$. Then, if $F_i=(x_i,\sigma_i)$ are our lifts, we have
	\[\alpha_i=N_i(x_i)\cdot\prod_{k=0}^{n_i-1}c\left(\sigma_i^k,\sigma_i\right)\qquad\text{and}\qquad\beta_{ij}=\frac{x_i}{\sigma_j(x_i)}\cdot\frac{\sigma_i(x_j)}{x_j}\cdot\frac{c(\sigma_i,\sigma_j)}{c(\sigma_j,\sigma_i)}\]
	for each $\alpha_i$ and $\beta_{ij}$.
\end{lemma}
\begin{proof}
	The equality for the $\alpha_i$ follow from \autoref{lem:explicitalpha}. For the equality about $\beta_{ij}$, we simply compute
	by brute force, writing
    \begin{align*}
        F_iF_j &= (x_i\cdot\sigma_ix_j\cdot c(\sigma_i,\sigma_j),\sigma_i\sigma_j) \\
        F_jF_i &= (x_j\cdot\sigma_jx_i\cdot c(\sigma_j,\sigma_i),\sigma_j\sigma_i) \\
        (F_jF_i)^{-1} &= \left((\sigma_j\sigma_i)^{-1}(x_j\cdot\sigma_jx_i\cdot c(\sigma_j,\sigma_i))^{-1},\sigma_i^{-1}\sigma_j^{-1}\right),
    \end{align*}
    which gives
    \begin{align*}
        \beta_{ij} &= (F_iF_j)(F_jF_i)^{-1} \\
        &= \left(\frac{x_i}{\sigma_jx_i}\cdot\frac{\sigma_ix_j}{x_j}\cdot\frac{c(\sigma_i,\sigma_j)}{c(\sigma_j,\sigma_i)},1\right),
    \end{align*}
	finishing.
\end{proof}
Here is a nice sanity check that we are doing things in the right setting: not only can we build tuples from extensions, but we can find an extension corresponding to any tuple.
\begin{cor} \label{cor:alltuplesfromextens}
	Fix everything as in the modified set-up, forgetting about the extension $\mc E$. For any $\{\sigma_i\}_{i=1}^m$-tuple of $\{\alpha_i\}_{i=1}^m$ and $\{\beta_{ij}\}_{i>j}$, there exists an extension $\mc E$ and lifts $F_i$ of the $\sigma_i$ so that
	\[\alpha_i=F_i^{n_i}\qquad\text{and}\qquad\beta_{ij}=F_iF_jF_i^{-1}F_j^{-1}.\]
\end{cor}
\begin{proof}
	From \autoref{thm:getcocycle}, we may build the cocycle $c\in Z^2( G,A)$ defined by
	\begin{equation}
		c(g,h)\coloneqq\left[\prod_{1\le j<i\le m}\Bigg(\prod_{1\le k<j}\sigma_k^{a_k+b_k}\Bigg)\Bigg(\prod_{j\le k<i}\sigma_k^{a_k}\Bigg)\beta_{ij}^{(a_ib_j)}\right]\left[\prod_{i=1}^m\Bigg(\prod_{1\le k<i}\sigma_k^{a_k+b_k}\Bigg)\alpha_i^{\floor{\frac{a_i+b_i}{n_i}}}\right], \label{eq:uglycocycle}
	\end{equation}
	where $g\coloneqq\prod_iF_i^{a_i}$ and $h\coloneqq\prod_iF_j^{a_j}$ and $0\le a_i,b_i<n_i$. As such, we use $\mc E\coloneqq\mc E_c$ to be the corresponding extension and $F_i\coloneqq(1,\sigma_i)$ as our lifts. We have the following checks.
	\begin{itemize}
		\item To show $\alpha_i=F_i^{n_i}$, we use \autoref{lem:explicitalphabeta} to compute $F_i^{n_i}$, which means we want to compute
		\[\prod_{k=0}^{n_i-1}c\left(\sigma_i^k,\sigma_i\right).\]
		Well, plugging $c\left(\sigma_i^k,\sigma_i\right)$ into \autoref{eq:uglycocycle}, we note that all $\beta_{k\ell}^{(a_kb_\ell)}$ terms vanish (either $a_k=0$ or $b_\ell=0$ for each $k\ne\ell$), so the big left product completely vanishes.
		
		As for the right product, the only term we have to worry about is
		\[\Bigg(\prod_{1\le k<i}\sigma_k^{0+0}\Bigg)\alpha_i^{\floor{\frac{k+1}{n_i}}},\]
		which is equal to $1$ when $k\le n_i-1$ and $\alpha_i$ when $k=n_i-1$. As such, we do indeed have $\alpha_i=F_i^{n_i}$.
		\item To show $\beta_{ij}=F_iF_jF_i^{-1}F_j^{-1}$ for $i>j$, we again use \autoref{lem:explicitalphabeta} to compute $F_iF_jF_i^{-1}F_j^{-1}$, which means we want to compute
		\[\frac{c(\sigma_i,\sigma_j)}{c(\sigma_j,\sigma_i)}.\]
		Plugging into \autoref{eq:uglycocycle} once more, there is no way to make $\floor{(a_k+b_k)/n_k}$ nonzero (recall we set $n_k>1$ for each $k$) in either $c(\sigma_i,\sigma_j)$ or $c(\sigma_j,\sigma_i)$. As such, the right-hand product term disappears.

		As for the left product, we note that it still vanishes for $c(\sigma_j,\sigma_i)$ because $i>j$ implies that either $a_k=0$ or $b_\ell=0$ for each $k>\ell$. However, for $c(\sigma_i,\sigma_j)$, we do have $a_i=1$ and $b_j=1$ only, so we have to deal with exactly the term
		\[\Bigg(\prod_{1\le k<j}\sigma_k^{a_k+b_k}\Bigg)\Bigg(\prod_{j\le k<i}\sigma_k^{a_k}\Bigg)\beta_{ij}.\]
		With $i>j$ and $a_k=b_k=0$ for $k\notin\{i,j\}$, we see that the product of all the $\sigma_k$s will disappear, indeed only leaving us with $\beta_{ij}$.
	\end{itemize}
	The above computations complete the proof.
\end{proof}
And here is our first taste of (partial) classification.
\begin{cor} \label{cor:cocycletuplesection}
	Fix everything as in the modified set-up, forgetting about the extension $\mc E$. Then the formula of \autoref{thm:getcocycle} and the formulae of \autoref{lem:explicitalphabeta} (setting $x_i=1$ for each $i$) are homomorphisms of abelian groups between the set of $\{\sigma_i\}_{i=1}^m$-tuples and cocycles in $Z^2( G,A)$. In fact, the formula of \autoref{thm:getcocycle} is a section of the formulae of \autoref{lem:explicitalphabeta}.
\end{cor}
\begin{proof}
	The formulae in \autoref{thm:getcocycle} and \autoref{lem:explicitalphabeta} are both large products in their inputs, so they are multiplicative (i.e., homomorphisms). It remains to check that we have a section. Well, starting with a $\{\sigma_i\}_{i=1}^m$-tuple and building the corresponding cocycle $c$ by \autoref{thm:getcocycle}, the proof of \autoref{cor:alltuplesfromextens} shows that the formulae of \autoref{lem:explicitalphabeta} recovers the correct $\{\sigma_i\}_{i=1}^m$-tuple.
\end{proof}

\subsection{Equivalence Classes of Tuples}
We continue in the modified set-up. We would like to make \autoref{cor:cocycletuplesection} into a proper isomorphism of abelian groups, but this is not feasible; for example, the cocycle $c$ generated by \autoref{thm:getcocycle} will always have $c(\sigma_j,\sigma_i)=1$ for $i>j$, which is not true of all cocycles in $Z^2( G,A)$.

However, we did have a notion that the data of a $\{\sigma_i\}_{i=1}^m$ should be enough to specify the group law of the extension that the tuple comes from, so we do expect to be able to define all extensions---and hence achieve all cohomology classes---from a specially chosen $\{\sigma_i\}_{i=1}^m$-tuple.

To make this precise, we want to define an equivalence relation on tuples which go to the same cohomology class and then show that the map \autoref{thm:getcocycle} is surjective on these equivalence classes. The correct equivalence relation is taken from \autoref{lem:explicitalphabeta}.
\begin{definition}
	Fix everything as in the modified set-up. We say that two $\{\sigma_i\}_{i=1}^m$-tuples $(\{\alpha_i\},\{\beta_{ij}\})$ and $(\{\alpha_i'\},\{\beta_{ij}'\})$ are \textit{equivalent} if and only if there exist elements $x_1,\ldots,x_m\in A$ such that
	\[\alpha_i=N_i(x_i)\cdot\prod_{k=0}^{n_i-1}c\left(\sigma_i^k,\sigma_i\right)\qquad\text{and}\qquad\beta_{ij}=\frac{x_i}{\sigma_j(x_i)}\cdot\frac{\sigma_i(x_j)}{x_j}\cdot\frac{c(\sigma_i,\sigma_j)}{c(\sigma_j,\sigma_i)}\]
	for each $\alpha_i$ and $\beta_{ij}$. We may notate this by $(\{\alpha_i\},\{\beta_{ij}\})\sim(\{\alpha_i'\},\{\beta_{ij}'\})$.
\end{definition}
This notion of equivalence can be seen to be the correct one in the sense that it correctly generalizes \autoref{prop:findallalpha}.
\begin{proposition} \label{prop:extenmakesaclass}
	Fix everything as in the modified set-up with an extension $\mc E$. As the lifts $F_i$ change, the corresponding values of
	\[\alpha_i\coloneqq F_i^{n_i}\qquad\text{and}\qquad\beta_{ij}\coloneqq F_iF_jF_i^{-1}F_j^{-1}\]
	go through a full equivalence class of $\{\sigma_i\}_{i=1}^m$-tuples.
\end{proposition}
\begin{proof}
	We proceed as in \autoref{prop:findallalpha}. Given an extension $\mc E'$, let $S_{\mc E'}$ be the set of $\{\sigma_i\}_{i=1}^m$-tuples generated as the lifts $F_i$ change. We start by showing that an isomorphism $\varphi\colon\mc E\simeq\mc E'$ of extensions implies that $S_{\mc E}=S_{\mc E'}$; by symmetry, it will be enough for $S_{\mc E}\subseteq S_{\mc E'}$. The isomorphism induces the following diagram.
	% https://q.uiver.app/?q=WzAsMTAsWzAsMCwiMSJdLFsxLDAsIkxeXFx0aW1lcyJdLFsyLDAsIlxcbWMgRSJdLFszLDAsIlxcR2FtbWEiXSxbNCwwLCIxIl0sWzAsMSwiMSJdLFsxLDEsIkxeXFx0aW1lcyJdLFsyLDEsIlxcbWMgRSciXSxbMywxLCJcXEdhbW1hIl0sWzQsMSwiMSJdLFswLDFdLFsxLDJdLFsyLDMsIlxccGkiXSxbMyw0XSxbNSw2XSxbNiw3XSxbNyw4LCJcXHBpJyJdLFs4LDldLFsyLDcsIlxcdmFycGhpIl0sWzEsNiwiIiwxLHsibGV2ZWwiOjIsInN0eWxlIjp7ImhlYWQiOnsibmFtZSI6Im5vbmUifX19XSxbMyw4LCIiLDEseyJsZXZlbCI6Miwic3R5bGUiOnsiaGVhZCI6eyJuYW1lIjoibm9uZSJ9fX1dXQ==&macro_url=https%3A%2F%2Fraw.githubusercontent.com%2FdFoiler%2Fnotes%2Fmaster%2Fnir.tex
	\[\begin{tikzcd}
		1 & {A} & {\mc E} &  G & 1 \\
		1 & {A} & {\mc E'} &  G & 1
		\arrow[from=1-1, to=1-2]
		\arrow[from=1-2, to=1-3]
		\arrow["\pi", from=1-3, to=1-4]
		\arrow[from=1-4, to=1-5]
		\arrow[from=2-1, to=2-2]
		\arrow[from=2-2, to=2-3]
		\arrow["{\pi'}", from=2-3, to=2-4]
		\arrow[from=2-4, to=2-5]
		\arrow["\varphi", from=1-3, to=2-3]
		\arrow[Rightarrow, no head, from=1-2, to=2-2]
		\arrow[Rightarrow, no head, from=1-4, to=2-4]
	\end{tikzcd}\]
	To show that $S_{\mc E}\subseteq S_{\mc E'}$, pick up some $\{\sigma_i\}_{i=1}^m$-tuple $(\{\alpha_i\},\{\beta_{ij}\})$ generated from lifts $F_i\in\mc E$ (i.e., $\pi(F_i)=\sigma_i$), where
	\[\alpha_i\coloneqq F_i^{n_i}\qquad\text{and}\qquad\beta_{ij}\coloneqq F_iF_jF_i^{-1}F_j^{-1}.\]
	Now, we note that $F_i'\coloneqq\varphi(F_i)$ will have
	\[\pi(F_i')=\pi(\varphi(F_i))=\varphi(\pi(F_i))=\sigma_i\]
	by the commutativity of the diagram, so the $F_i'$ are lifts of the $\sigma_i$. Further, we see that
	\[(F_i')^{n_i}=\varphi(F_i)^{n_i}=\varphi\left(F_i^{n_i}\right)=\varphi(\alpha_i)=\alpha_i\]
	for each $i$, and
	\[F_i'F_j'(F_i')^{-1}(F_j')^{-1}=\varphi\left(F_iF_jF_i^{-1}F_j^{-1}\right)=\varphi(\beta_{ij})=\beta_{ij}\]
	for each $i>j$. Thus, $(\{\alpha_i\},\{\beta_{ij}\})$ is a $\{\sigma_i\}_{i=1}^m$-tuple generated by lifts from $\mc E'$, implying that $(\{\alpha_i\},\{\beta_{ij}\})\in S_{\mc E'}$.

	It now suffices to show the statement in the proposition for a specific extension isomorphic to $\mc E$. Well, the isomorphism class of $\mc E$ corresponds to some cohomology class in $H^2( G,A)$, for which we let $c$ be a representative; then $\mc E\simeq\mc E_c$, so we may show the statement for $\mc E\coloneqq\mc E_c$. Indeed, as the lifts $F_i=(x_i,\sigma_i)$ change, we know by \autoref{lem:explicitalphabeta} that
	\[\alpha_i=N_i(x_i)\cdot\prod_{k=0}^{n_i-1}c\left(\sigma_i^k,\sigma_i\right)\qquad\text{and}\qquad\beta_{ij}=\frac{x_i}{\sigma_j(x_i)}\cdot\frac{\sigma_i(x_j)}{x_j}\cdot\frac{c(\sigma_i,\sigma_j)}{c(\sigma_j,\sigma_i)}\]
	for each $\alpha_i$ and $\beta_{ij}$. All of these live in the same equivalence class by definition of the equivalence, and as the $x_i$ are allowed to vary over all of $A$, they will fill up that equivalence class fully. This finishes.
\end{proof}
We are now ready to upgrade our section.
\begin{cor} \label{cor:cohomologymakesaclass}
	Fix everything as in the modified set-up, forgetting about the extension $\mc E$. Fixing a cohomology class $[c]\in H^2( G,A)$, the set of $\{\sigma_i\}_{i=1}^m$ which correspond to $[c]$ (via \autoref{thm:getcocycle}) forms exactly one equivalence class.
\end{cor}
\begin{proof}
	We show that two tuples are equivalent if and only if their corresponding cocycles (via \autoref{thm:getcocycle}) to the same cohomology class, which will be enough.
	
	In one direction, suppose $(\{\alpha_i\},\{\beta_{ij}\})\sim(\{\alpha_i'\},\{\beta_{ij}'\})$. By \autoref{cor:alltuplesfromextens}, we can find an extension $\mc E$ which gives $(\{\alpha_i\},\{\beta_{ij}\})$ by choosing an appropriate set of lifts. By \autoref{prop:extenmakesaclass}, we see that $(\{\alpha_i'\},\{\beta_{ij}'\})$ must also come from choosing an appropriate set of lifts in $\mc E$. However, the cocycles in $Z^2( G,A)$ generated by \autoref{thm:getcocycle} from our two tuples now both represent the isomorphism class of $\mc E$ by \autoref{prop:writedowncocycle}, so these cocycles belong to the same cohomology class.

	In the other direction, name the cocycles corresponding to $(\{\alpha_i\},\{\beta_{ij}\})$ and $(\{\alpha_i'\},\{\beta_{ij}'\})$ by $c$ and $c'$ respectively, and suppose $[c]=[c']$. Then $\mc E_c\simeq\mc E_{c'}$ as extensions, but we know by the proof of \autoref{cor:alltuplesfromextens} that $(\{\alpha_i\},\{\beta_{ij}\})$ comes from choosing lifts of $\mc E_c$ and similar for $(\{\alpha_i'\},\{\beta_{ij}'\})$. In particular, because $\mc E_c\simeq\mc E_{c'}$, we know that $(\{\alpha_i'\},\{\beta_{ij}'\})$ will also come from choosing some lifts in $\mc E_c$ (recall the proof of \autoref{prop:extenmakesaclass}), so $(\{\alpha_i\},\{\beta_{ij}\})\sim(\{\alpha_i'\},\{\beta_{ij}'\})$ follows.
\end{proof}
\begin{theorem} \label{thm:classisomorphism}
	The maps described in \autoref{cor:cocycletuplesection} descend to an isomorphism of abelian groups between the equivalence classes of $\{\sigma_i\}_{i=1}^m$-tuples and cohomology classes in $H^2( G,A)$.
\end{theorem}
\begin{proof}
	The fact that the maps are well-defined (in both directions) and hence injective is \autoref{cor:cohomologymakesaclass}. The fact that we had a section from tuples to cocycles implies that the map from cocycles to tuples was also surjective. Thus, we have a bona fide isomorphism.
\end{proof}

\subsection{Classification of Extensions}
We remark that we are now able to classify all extensions up to isomorphism, in some sense. At a high level, an isomorphism class of extensions corresponds to a particular cohomology class in $H^2( G,A)$, so choosing a $\{\sigma_i\}_{i=1}^m$-tuple $(\{\alpha_i\},\{\beta_{ij}\})$ corresponding to this class, we can write out a representative of this cocycle by \autoref{thm:getcocycle}, properly corresponding to the original extension by \autoref{prop:writedowncocycle}.

In fact, the cocycle in \autoref{prop:writedowncocycle} is generated by the description of the group law in \autoref{prop:multiplytwoelements}, and the entire computation only needed to use the following relations, for the appropriate choice of lifts $F_i$.
\begin{listalph}
	\item $F_ix=\sigma_i(x)F_i$ for each $i$ and $x\in A$.
	\item $F_i^{n_i}=\alpha_i$ for each $i$.
	\item $F_iF_jF_i^{-1}F_j^{-1}=\beta_{ij}$ for each $i>j$; i.e., $F_iF_j=\beta_{ij}F_jF_i$.
\end{listalph}
As such, the above relations fully describe the extension because they also specify the cocycle, and we know that this cocycle is well-defined. We summarize this discussion into the following theorem.
\begin{theorem}
	Fix everything as in the modified set-up, forgetting about the extension $\mc E$. Given a $\{\sigma_i\}_{i=1}^m$-tuple $(\{\alpha_i\},\{\beta_{ij}\})$, define the group $\mc E(\{\alpha_i\},\{\beta_{ij}\})$ as being generated by $A$ and elements $\{F_i\}_{i=1}^n$ having the following relations.
	\begin{listalph}
		\item $F_ix=\sigma_i(x)F_i$ for each $i$ and $x\in A$.
		\item $F_i^{n_i}=\alpha_i$ for each $i$.
		\item $F_iF_j=\beta_{ij}F_jF_i$ for each $i>j$.
	\end{listalph}
	Then the natural embedding $A\into\mc E(\{\alpha_i\},\{\beta_{ij}\})$ and projection $\pi\colon\mc E(\{\alpha_i\},\{\beta_{ij}\})\onto G$ by $F_i\mapsto\sigma_i$ makes $\mc E(\{\alpha_i\},\{\beta_{ij}\})$ into an extension. In fact, all extensions are isomorphic to some $\mc E(\{\alpha_i\},\{\beta_{ij}\})$.
\end{theorem}
\begin{proof}
	This follows from the preceding discussion, though we will provide a few more words in this proof. The exactness of
	\[1\to A\to\mc E(\{\alpha_i\},\{\beta_{ij}\})\stackrel\pi\to G\to1\]
	follows quickly. Further, the action of conjugation of $\mc E$ on $A$ corresponds correctly to the $ G$-action by (a). So we do indeed have an extension.

	It remains to show that all extensions are isomorphic to one of this type. Well, note that \autoref{prop:multiplytwoelements} and \autoref{prop:writedowncocycle} use only the above relations to write down a cocycle representing the isomorphism class of $\mc E(\{\alpha_i\},\{\beta_{ij}\})$, and it is the cocycle corresponding to the $\{\sigma_i\}_{i=1}^m$-tuple $(\{\alpha_i\},\{\beta_{ij}\})$ itself as described in \autoref{thm:getcocycle}.

	However, we know that as the equivalence class of $(\{\alpha_i\},\{\beta_{ij}\})$ changes, we will hit all cohomology classes in $H^2( G,A)$ by \autoref{thm:classisomorphism}. Thus, because every extension is represented by some cohomology class, every extension will be isomorphic to some $\mc E(\{\alpha_i\},\{\beta_{ij}\})$. This completes the proof.
\end{proof}

\section{Studying Tuples} \label{sec:tuplestudy}
The story so far has been able to generalize the one-variable results from \autoref{sec:general} to results using all generators of an abelian group in \autoref{sec:abelian}. It remains to prove \autoref{thm:getcocycle}, which is the main goal of this section.

\subsection{Set-Up and Overview} \label{sec:overview}
The approach here will be to attempt to abstract our data away from the $ G$-module $A$ as much as possible. To set up our discussion, we continue with
\[G\simeq\bigoplus_{i=1}^mG_i,\]
where $G_i=\langle\sigma_i\rangle\subseteq G$ and $\sigma_i$ has order $n_k$. These variables allow us to define
\[T_i\coloneqq(\sigma_i-1)\qquad\text{and}\qquad N_i\coloneqq\sum_{p=0}^{n_i-1}\sigma_i^p\]
for each index $i$. In fact, it will be helpful to also have notation
\[\sigma^{(a)}\coloneqq\sum_{p=0}^{a-1}\sigma^p\]
for any $\sigma\in G$ and nonnegative integer $a\ge0$; in particular, $\sigma^{(0)}=0$ and $\sigma_i^{(n_i)}=N_i$. The main benefits to this notation will be the facts that
\[\sigma^{(a+b)}=\sigma^{(a)}+\sigma^a\sigma^{(b)}\qquad\text{and}\qquad\sigma_i^a=T_i\sigma_i^{(a)}+1,\]
which can be seen by direct expansion. Given $g\in\prod_{p=1}^n\sigma_p^{a_p}$, we will also define the notation
\[g_i\coloneqq\prod_{p=1}^{i-1}\sigma_p^{a_p}\]
for $i\ge0$. In particular $g_0=g_1=1$ and $g_{n+1}=g$.

Now, our tool in the proof of \autoref{thm:getcocycle} will be the magical map $\mathcal F\colon\ZZ[G]^m\times\ZZ[G]^{\binom m2}\to\ZZ[G]^m$ defined by
\[\mathcal F\colon\big((x_i)_{i=1}^m,(y_{ij})_{i>j}\big)\mapsto\Bigg(x_iN_i-\sum_{j=1}^{i-1}y_{ij}T_j+\sum_{j=i+1}^my_{ji}T_j\Bigg)_{i=1}^m.\]
This is of course as $G$-module homomorphism. We will go ahead and state the main results we will prove. Roughly speaking, $\mathcal F$ is manufactured to make the following result true.
\begin{prop} \label{prop:manufacturedcocycle}
	Fix everything as in the set-up. Then the function
	\[\overline c(g)\coloneqq\left(g_i\sigma_i^{(a_i)}\right)_{i=1}^m,\]
	where $g\coloneqq\prod_{i=1}^m\sigma_i^{a_i}$, is a $1$-cocycle in $Z^1(G,\coker\mathcal F)$.
\end{prop}
The reason we care about this cocycle is that we can pass it through a boundary morphism induced by the short exact sequence
\[0\to\underbrace{\frac{\ZZ[G]^m\times\ZZ[G]^{\binom m2}}{\ker\mathcal F}}_{X\coloneqq}\stackrel{\mathcal F}\to\ZZ[G]^m\to\coker\mathcal F\to0,\]
so we have a $2$-cocycle $\delta(\overline c)\in Z^2(G,X)$; in fact, we will be able to explicitly compute $\delta(\overline c)$ as a result of the proof of \autoref{prop:manufacturedcocycle}.

Only now will we bring in tuples. The first result provides an alternate description of tuples.
\begin{restatable}{prop}{propalternativetuple} \label{prop:alternativetuple}
	Fix everything as in the set-up, and now let $A$ be a $G$-module. Then $\{\sigma_i\}_{i=1}^m$-tuples are canonically isomorphic to $\op{Hom}_{\ZZ[G]}(X,A)=H^0(G,\op{Hom}_\ZZ(X,A))$.
\end{restatable}
\noindent The second result brings in the last ingredient, the cup product.
\begin{restatable}{theorem}{thmyesitisacocycle} \label{thm:yesitisacocycle}
	Fix everything as in the set-up. Further, fix a $G$-module $A$ and a $\{\sigma_i\}_{i=1}^m$-tuple $\left(\{\alpha_i\},\{\beta_{ij}\}\right)$. Then observe there is a natural cup product map
	\[\cup\colon H^2(G,X)\times H^0(G,\op{Hom}_\ZZ(X,A))\to H^2(G,A)\]
	by using the evaluation map $X\otimes_\ZZ\op{Hom}_\ZZ(X,A)\to A$. Then, using the isomorphism of \autoref{prop:alternativetuple}, the cocycle defined in \autoref{thm:getcocycle} is simply the output of $\delta(\overline c)\cup\left(\{\alpha_i\},\{\beta_{ij}\}\right)$ on cocycles.
\end{restatable}
\noindent Because we know that the cup product sends cocycles to cocycles, this will show that the cocycle of \autoref{thm:getcocycle} is in fact well-defined.

% it might be worth stating the main results we are going to prove here, but they are somewhat notation-heavy

\subsection{Classification of 1-Cocycles}
We continue in the set-up of the previous subsection. The goal of this subsection is to prove \autoref{prop:manufacturedcocycle}. In fact, we will show the following stronger result.
\begin{proposition} \label{prop:allmanufacturedcocycles}
	Fix everything as in the set-up. Then $H^1(G,\coker\mathcal F)$ is cyclic generated by the class $[\overline c]$ represented by $\overline c$, where
	\[\overline c(g)\coloneqq\left(g_i\sigma_i^{(a_i)}\right)_{i=1}^m,\]
	with $g\coloneqq\prod_{i=1}^m\sigma_i^{a_i}$
\end{proposition}
Before jumping into the proof, we define some (more) notation which will be useful later on as well. First, in $\ZZ[G]^m\times\ZZ[G]^{\binom m2}$, we define
\[\kappa_p\coloneqq\big((1_{i=p})_i,(0)_{i>j}\big)\in X\qquad\text{and}\qquad\lambda_{pq}\coloneqq\big((0)_i,(1_{(i,j)=(p,q)})_{i>j}\big)\]
for all relevant indices $p$ and $q$ so that the $\kappa_p$ and $\lambda_{pq}$ are a basis for $\ZZ[G]^m\times\ZZ[G]^{\binom m2}$ as a $\ZZ[G]$-module. Secondly, we define
\[\varepsilon_p\coloneqq(1_{i=p})_{i=1}^m\]
for all indices $p$, again giving a basis for $\ZZ[G]^m$ as a $\ZZ[G]$-module. For example, this notation lets us write
\begin{equation}
	\mathcal F\left(\sum_{i=1}^mx_i\kappa_i+\sum_{i>j}y_{ij}\lambda_{ij}\right)=\sum_{i=1}^mx_iN_i\varepsilon_i+\sum_{i>j}y_{ij}(T_i\varepsilon_j-T_j\varepsilon_i), \label{eq:betterf}
\end{equation}
and
\[\overline c(g)=\sum_{i=1}^mg_i\sigma_i^{(a_i)}\varepsilon_i\]
where $g\coloneqq\prod_{i=1}^m\sigma_i^{a_i}$.

Additionally, so that we do not need to interrupt our discussion later, we establish a few lemmas which will aide our proof of \autoref{prop:allmanufacturedcocycles}.
\begin{lemma} \label{lem:separatenijs}
	Fix everything as in the set-up. Then, for any set of distinct indices $(i_1,\ldots,i_k)$, we have
	\[\bigcap_{p=1}^k\im N_{i_p}=\im\prod_{p=1}^kN_{i_p},\]
	where we are identifying $x\in\ZZ[G]$ with its associated multiplication map $x\colon\ZZ[G]\to\ZZ[G]$.
\end{lemma}
\begin{proof}
	The point is that the elements of $\bigcap_{p=1}^k\im N_{i_p}$ and $\im\prod_{p=1}^kN_{i_p}$ are both simply the elements whose expansion in the form $\sum_gc_gg\in\ZZ[G]$ have $c_j$ ``constant in $\sigma_p$ and $\sigma_q$.'' More explicitly, of course, $\prod_{p=1}^kN_{i_p}\in\bigcap_{p=1}^k\im N_{i_p}$, so
	\[\im\prod_{p=1}^kN_{i_p}\in\bigcap_{p=1}^k\im N_{i_p}.\]
	In the other direction, suppose that we have some element
	\[z\coloneqq\sum_{(a_i)_i}c_{(a_i)_i}\sigma_1^{a_1}\cdots\sigma_m^{a_m}\in\bigcap_{p=1}^k\im N_{i_p},\]
	the sum is over sequences $(a_i)_{i=1}^m$ such that $0\le a_i<n_i$ for each index $i$. We will show $z\in\im\prod_{p=1}^kN_{i_p}$.
	
	Now, $z\in\im N_r$ for $r\in\{p,q\}$ is equivalent to $z\in\ker T_r$, but upon multiplying by $(\sigma_r-1)$ we see that we are asking for
	\[\sum_{(a_i)_i}c_{(a_i)_i}\sigma_1^{a_1}\cdots\sigma_{r-1}^{a_{r-1}}\sigma_r^{a_r}\sigma_{r+1}^{a_{r+1}}\cdots\sigma_n^{a_n}=\sum_{(a_i)_i}c_{(a_i)_i}\sigma_1^{a_1}\cdots\sigma_{r-1}^{a_{r-1}}\sigma_r^{a_r+1}\sigma_{r+1}^{a_{r+1}}\cdots\sigma_n^{a_n}.\]
	In other words, this is asking for $c_{(a_i)_i}=c_{(a_i)_i+(1_{i=r})_i}$, or more succinctly just that $c$ is constant in the $i=r$ coordinate.

	Thus, $c$ is constant in all the $i=i_p$ coordinates for each index $i_p$. Thus, we let $d_{(a_i)_{i\notin\{i_p\}}}$ be the restricted function equal to $c_{(a_i)_i}$ but forgetting the information input from any of the $a_{i_p}$. This allows us to write
	\begin{align*}
		z &= \sum_{(a_i)_i}c_{(a_i)_i}\sigma_1^{a_1}\cdots\sigma_m^{a_m} \\
		&= \sum_{(a_i)_{i\notin\{i_p\}}}\sum_{a_{i_1}=0}^{n_{i_1}-1}\cdots\sum_{a_{i_k}=0}^{n_{i_k}-1}d_{(a_i)_{i\notin\{i_p\}}}\sigma_1^{a_1}\cdots\sigma_m^{a_m} \\
		&= \Bigg(\sum_{(a_i)_{i\notin\{i_p\}}}d_{(a_i)_{i\notin\{i_p\}}}\prod_{\substack{i=0\\i\notin\{i_p\}}}^m\sigma_i^{a_i}\Bigg)\Bigg(\sum_{a_{i_1}=0}^{n_{i_1}-1}\sigma_{i_1}^{a_{i_1}}\Bigg)\cdots\Bigg(\sum_{a_{i_k}=0}^{n_{i_k}-1}\sigma_{i_k}^{a_{i_k}}\Bigg),
	\end{align*}
	which is now manifestly in $\im\prod_{p=1}^kN_{i_p}$.
\end{proof}
\begin{lemma} \label{lem:expandgi}
	Fix everything as in the set-up. Then, given $g\coloneqq\prod_{i=1}^m\sigma_i^{a_i}$, we have
	\[g_i=1+\sum_{p=1}^{i-1}g_p\sigma_p^{(a_p)}T_p\]
	for $i\ge1$.
\end{lemma}
\begin{proof}
	This is by induction. For $i=1$, there is nothing to say. For the inductive step, we take $i>1$ where we may assume the statement for $i-1$. Via some relabeling, we may make our inductive hypothesis assert
	\[\prod_{p=2}^{i-1}\sigma_p^{a_p}=1+\sum_{p=2}^{i-1}\Bigg(\prod_{q=2}^{p-1}\sigma_q^{a_q}\Bigg)\sigma_p^{(a_p)}T_p.\]
	In particular, multiplying through by $\sigma_1^{a_1}$ yields
	\begin{align*}
		g_i &= \sigma_1^{a_1}\cdot\prod_{p=2}^{i-1}\sigma_p^{a_p} \\
		&= \sigma_1^{a_1}+\sigma_1^{a_1}\sum_{p=2}^{i-1}\Bigg(\prod_{q=2}^{p-1}\sigma_q^{a_q}\Bigg)\sigma_p^{(a_p)}T_p \\
		&= \sigma_1^{a_1}+\sum_{p=2}^{i-1}g_p\sigma_p^{(a_p)}T_p \\
		&= 1+\sigma_1^{(a_1)}T_1+\sum_{p=2}^{i-1}g_p\sigma_p^{(a_p)}T_p,
	\end{align*}
	which is exactly what we wanted, after a little more rearrangement.
\end{proof}
\begin{lemma} \label{lem:sessplits}
	Fix everything as in the set-up. Then consider $\ZZ$-module map $\rho\colon\ZZ[G]^m\to\ZZ[G]^m$ defined by
	\[\rho(g\varepsilon_i)\coloneqq g_i\big(\sigma_i^{a_i}-N_i1_{a_i=n_i-1}\big)\varepsilon_i+\sum_{j=i+1}^mg_j\sigma_j^{(a_j)}T_i\varepsilon_j,\]
	where $g\coloneqq\prod_{i=1}^m\sigma_i^{a_i}$ with $0\le a_i<n_i$. Then $\rho$ descends to a map $\overline\rho\colon\coker\mathcal F\to\ZZ[G]^m$ witnessing the splitting of the short exact sequence
	\[0\to X\to\ZZ[G]^m\to\coker\mathcal F\to0\]
	over $\ZZ$.
\end{lemma}
\begin{proof}
	Observe that we have a well-defined map $\rho\colon\ZZ[G]^m\to\ZZ[G]^m$ because $\ZZ[G]^m$ is a free abelian group generated by $g\varepsilon_i$ for $g\in G$ and indices $i$. It remains to show that $\im\mathcal F\subseteq\ker\rho$ to get a map $\overline\rho\colon\coker\mathcal F\to\ZZ[G]^m$ and then to show that $\rho(z)\equiv z\pmod{\im\mathcal F}$ to get the splitting. We show these individually.

	To show that $\im\mathcal F\subseteq\ker\rho$, we note from \autoref{eq:betterf} that $\im\mathcal F$ is generated over $\ZZ[G]$ by the elements $N_i\varepsilon_i$ and $T_i\varepsilon_j-T_j\varepsilon_i$ for relevant indices $i$ and $j$. Thus, $\im\mathcal F$ is generated over $\ZZ$ by the elements $gN_i\varepsilon_i$ and $gT_i\varepsilon_j-gT_j\varepsilon_i$ for relevant indices $i$ and $j$. Thus, we fix any $g\coloneqq\prod_{i=1}^n\sigma_i^{a_i}$ and show that $gN_i\varepsilon_i\in\ker\rho$ and $gT_i\varepsilon_j-gT_j\varepsilon_i\in\ker\rho$ for relevant indices $i$ and $j$.
	\begin{itemize}
		\item We show $gN_i\varepsilon_i\in\ker\rho$ for any $i$. Because $gN_i=g\sigma_iN_i$, we may as well as assume that $a_i=0$. Then
		\[\rho\left(g\sigma_i^a\varepsilon_i\right)=g_i\big(\sigma_i^{a}-N_i1_{a=n_i-1}\big)\varepsilon_i+\sum_{j=i+1}^mg_j\sigma_i^a\sigma_j^{(a_j)}T_i\varepsilon_j.\]
		As $a$ varies from $0$ to $n_i-1$, we note that the term $g_i\big(\sigma_i^{a}-N_i1_{a=n_i-1}\big)\varepsilon_i$ will only get the $-N_i$ contribution exactly once at $a=n_i-1$. Summing, we thus see that
		\[\rho(gN_i\varepsilon_i)=g_i\Bigg(-N_i+\sum_{a=0}^{n_i-1}\sigma_i^{a}\Bigg)\varepsilon_i+\sum_{a=0}^{n_i-1}\sum_{j=i+1}^mg_j\sigma_i^a\sigma_j^{(a_j)}T_i\varepsilon_j.\]
		The left term vanishes because $N_i=\sum_{a=0}^{n_i-1}\sigma_i^a$. Additionally, the right term vanishes because we can factor $T_i\sum_{a=0}^{n_i-1}\sigma_i^a=T_iN_i=0$. So $gN_i\varepsilon_i\in\ker\rho$.
		\item We show $gT_p\varepsilon_q-gT_q\varepsilon_p\in\ker\rho$ for any $p>q$. Equivalently, we will show that $\rho(g\sigma_p\varepsilon_q)-\rho(g\varepsilon_q)=\rho(g\sigma_q\varepsilon_p)-\rho(g\varepsilon_p)$. On one hand, note
		\begin{align*}
			\rho(g\sigma_p\varepsilon_q) &= g_q\big(\sigma_q^{a_q}-N_i1_{a_q=n_q-1}\big)\varepsilon_q \\
			&\qquad\qquad+\sum_{j=q+1}^{p-1}g_j\sigma_j^{(a_j)}T_q\varepsilon_j \\
			&\qquad\qquad+g_p\left(\sigma_p^{(a_p+1)}-N_p1_{a_p=n_p-1}\right)T_q\varepsilon_p \\
			&\qquad\qquad+\sum_{j=p+1}^m\sigma_pg_j\sigma_j^{(a_j)}T_q\varepsilon_j
		\end{align*}
		because $g_j$ doesn't ``see'' the extra $\sigma_p$ term until $j>p$. (For the $j=p$ term, we would like to write $\sigma_p^{(a_p+1)}$ above, but when $a_p=n_p-1$, we actually end up with $\sigma_p^{(0)}=0$ and hence have to subtract out $\sigma_p^{(n_p)}=N_p$.) Thus,
		\[\rho(g\sigma_p\varepsilon_q)-\rho(g\varepsilon_q) = g_p\left(\sigma_p^{a_p}-N_p1_{a_p=n_p-1}\right)T_q\varepsilon_p+\sum_{j=p+1}^mg_j\sigma_j^{(a_j)}T_pT_q\varepsilon_j.\]
		On the other hand, we have
		\[\rho(g\sigma_q\varepsilon_p) = \sigma_qg_p\big(\sigma_p^{a_p}-N_p1_{a_p=n_p-1}\big)\varepsilon_p+\sum_{j=p+1}^m\sigma_qg_j\sigma_j^{(a_j)}T_p\varepsilon_j\]
		where this time all $j>p$ also have $j>q$ and so $(\sigma_qg)_j=\sigma_qg_j$. Thus,
		\[\rho(g\sigma_q\varepsilon_p)-\rho(g\varepsilon_p) = g_p\left(\sigma_p^{a_p}-N_p1_{a_p=n_p-1}\right)T_q\varepsilon_p+\sum_{j=p+1}^mg_j\sigma_j^{(a_j)}T_pT_q\varepsilon_j,\]
		as desired.
	\end{itemize}
	We now check the splitting. For this, we simply need to check that $\rho(g\varepsilon_i)\equiv g\varepsilon_i\pmod{\im\mathcal F}$, and we will get the result for all elements of $\ZZ[G]^m$ by additivity of $\rho$. Well, using \autoref{lem:expandgi}, we write
	\begin{align*}
		g\varepsilon_i &= g_i\sigma_i^{a_i}\Bigg(\prod_{j=i+1}^m\sigma_j^{a_j}\Bigg)\varepsilon_i \\
		&= g_i\sigma_i^{a_i}\Bigg(1+\sum_{j=i+1}^m\Bigg(\prod_{q=i+1}^{j-1}\sigma_q^{a_q}\Bigg)\sigma_j^{(a_j)}T_j\Bigg)\varepsilon_i \\
		&= g_i\sigma_i^{a_i}\varepsilon_i+\sum_{j=i+1}^mg_i\sigma_i^{a_i}\Bigg(\prod_{q=i+1}^{j-1}\sigma_q^{a_q}\Bigg)\sigma_j^{(a_j)}T_j\varepsilon_i \\
		&\equiv g_i\sigma_i^{a_i}\varepsilon_i+\sum_{j=i+1}^mg_j\sigma_j^{(a_j)}T_i\varepsilon_j,
	\end{align*}
	where in the last step we have used the fact that $T_j\varepsilon_i\equiv T_j\varepsilon_i\pmod{\im\mathcal F}$. Lastly, we note that $hN_i\varepsilon_i\equiv h\varepsilon_i\pmod{\im\mathcal F}$ for any $h\in G$, so in fact
	\[g\varepsilon_i\equiv g_i\left(\sigma_i^{a_i}-N_i1_{a_i=n_i-1}\right)\varepsilon_i+\sum_{j=i+1}^mg_j\sigma_j^{(a_j)}T_i\varepsilon_j,\]
	and now the right-hand side is $\rho(g\varepsilon_i)$.
\end{proof}
\begin{remark}
	The purpose of \autoref{lem:sessplits} is to give an injective map from $\coker\mathcal F$ to a more controlled setting. In particular, it is somewhat annoying to check if an element $z\in\ZZ[G]^m$ lives in $\im\mathcal F$, but it is easier to check the equivalent condition $\overline\rho(z)=0$.
\end{remark}
We are now ready to more directly attack the proof of \autoref{prop:allmanufacturedcocycles}. We begin by reducing the amount of data we have to carry around in a cocycle.
\begin{lemma} \label{lem:compresscocycle}
	Fix everything as in the set-up, and let $A$ be a $G$-module. Then, if $f\in Z^1(G,A)$ is a cocycle, then
	\[f(g)=\sum_{i=1}^mg_i\sigma_i^{(a_i)}f(\sigma_i),\]
	where $g\coloneqq\prod_{i=1}^m\sigma_i^{a_i}$ with $a_i\ge0$.
\end{lemma}
\begin{proof}
	Unsurprisingly, this is by induction. To begin, we claim that
	\[f\left(\sigma^a\right)=\sigma^{(a)}f(\sigma)\]
	by induction on $a$. When $a=0$, we are showing that $f(1)=0$, for which we note that the $1$-cocycle condition implies $f(1)=f(1)+f(1)$ and so $f(1)=0$. Then for the inductive step, we assume $f(\sigma^a)=\sigma^{(a)}f(\sigma)$ and note
	\[f\left(\sigma^{a+1}\right)=\sigma f\left(\sigma^a\right)+f(\sigma)=\left(1+\sigma\sigma^{(a)}\right)f(\sigma)=\sigma^{(a+1)}f(\sigma),\]
	finishing.

	We now show the original statement by an induction on $m$. For $m=0$, this is asserting $f(1)=0$, which is true. Then for the inductive step, we assume for $m-1$ and note that $m>1$ has
	\[f\left(g_m\sigma_m^{a_m}\right)=f(g_m)+g_mf\left(\sigma_m^{a_m}\right)=\sum_{i=1}^{m-1}g_i\sigma_i^{(a_i)}f(\sigma_i)+g_m\sigma_m^{(a_m)}f(\sigma_m),\]
	which is what we wanted.
\end{proof}
Thus, to build a $1$-cocycle, we only have to specify $f(\sigma_i)$ for indices $i$ and then check the $1$-cocycle condition to make sure we are okay.

As such, we now run through what the $1$-cocycle check requires.
\begin{lemma} \label{lem:cocycleforcecoord}
	Fix everything as in the set-up. Further, fix some $z\in\ZZ[G]^m$. Then $N_iz\in\im\mathcal F$ if and only if $[z]\in\coker\mathcal F$ has a representative of the form $a_i\varepsilon_i\in\ZZ[G]^m$ where $a_i\in\ZZ[G]$.
\end{lemma}
\begin{proof}
	In one direction, if $z\equiv a_i\varepsilon_i\pmod{\im\mathcal F}$, then
	\[N_iz\equiv a_i\cdot N_i\varepsilon_i\equiv a_i\cdot0\equiv0\pmod{\im\mathcal F}\]
	because $N_i\varepsilon_i\in\im\mathcal F$.

	In the other direction, we pass through $\overline\rho$ of \autoref{lem:sessplits}. By possibly rearranging our $\sigma$s, we may set $i=1$. As such, suppose $N_1z\in\im\mathcal F$, and write
	\[z\coloneqq\sum_{i=1}^mz_i\varepsilon_i\]
	where $z_i\in\ZZ[G]$. By using the fact that $T_i\varepsilon_1\equiv T_1\varepsilon_i\pmod{\im\mathcal F}$ for any index $i$, we can find a representative for $z$ in $\ZZ[G]^m$ such that $z_i$ has no $\sigma_1$ powers for each $i>1$; without loss of generality, replace $z$ with this representative.
	
	We thus claim that $w\coloneqq z-z_1\varepsilon_1\in\im\mathcal F$, which means that $z$ is represented by $z_1\varepsilon_1$; to show this, we already know that $N_1w=N_1(z-z_1\varepsilon_1)\in\im\mathcal F$, so we pass through $\overline\rho$. In other words, it suffices to show that $\rho(w)=0$ from $\rho(N_1w)=0$ and the fact that $w$ features no $\sigma_1$ nor $\varepsilon_1$ terms.
	
	Well, because $w$ features no $\sigma_1$ nor $\varepsilon_1$ terms, the only terms we care about have $g\varepsilon_i$ where $g$ has no $\sigma_1$ and $i>1$; in this case,
	\[\rho\left(g\sigma_1^a\varepsilon_i\right)\coloneqq\sigma_1^ag_i\big(\sigma_i^{a_i}-N_i1_{a_i=n_i-1}\big)\varepsilon_i+\sum_{j=i+1}^m\sigma_1^ag_j\sigma_j^{(a_j)}T_i\varepsilon_j=\sigma_1^a\rho(g\varepsilon_i),\]
	where $g\coloneqq\prod_{i=2}^m\sigma_i^{a_i}$ with $0\le a_i<n_i$. Looping over all possible $g$ and $\varepsilon_i$, we see $\rho(\sigma_1^aw)=\sigma_1^a\rho(w)$, so
	\[N_1\rho(w)=\rho(N_1w)=0.\]
	Thus, $\rho(w)\in\im T_1$, so say $\rho(w)=(\sigma_1-1)w'$. However, because $w$ has no $\varepsilon_1$ terms nor any term with a $\sigma_1$, we can see from the expansion of $\rho(w)$ that $\rho(w)$ will have no $\sigma_1$ terms. It follows that $\rho(w)\in\ZZ[G]^m$ is preserved upon applying $\sigma_1\mapsto1$, but then $(\sigma_1-1)w'$ gets sent to $0$, so it follows $\rho(w)=0$. This finishes.
\end{proof}
\begin{lemma} \label{lem:cocycleforcecohere}
	Fix everything as in the set-up. Suppose we have $\{z_i\}_{i=1}^m\subseteq\ZZ[G]$ such that
	\[T_iz_j\varepsilon_j=T_jz_i\varepsilon_i\]
	in $\coker\mathcal F$, for any pair of indices $(i,j)$. Then there exists $z\in\ZZ[G]$ such that $z\varepsilon_i=z\varepsilon_i$ (in $\coker\mathcal F$) for each index $i$.
\end{lemma}
\begin{proof}
	We proceed by induction on $m$. For $m=1$, we simply set $z\coloneqq z_1$. For the inductive step, take $m>1$, and we are given elements $\{z_i\}_{i=1}^m\subseteq\ZZ[G]$ such that
	\[T_iz_j\varepsilon_j=T_jz_i\varepsilon_i\]
	for any pair of indices $(i,j)$. By the inductive hypothesis, we may use the equations with indices less than $m$ to conjure some $z\in\ZZ[G]$ such that
	\[z\varepsilon_i\equiv z_i\varepsilon_i\pmod{\im\mathcal F}\]
	for each $i<m$. It remains to deal with the equations which have $m$ as an index; namely, for each $i<m$, we have an equation
	\[T_iz_m\varepsilon_m\equiv T_mz_i\varepsilon_i\equiv T_mz\varepsilon_i\pmod{\im\mathcal F}.\]
	Now, $T_m\varepsilon_i\equiv T_i\varepsilon_m\pmod{\im\mathcal F}$, so this is equivalent to asserting
	\[T_i(z_m-z)\varepsilon_m\equiv0\pmod{\im\mathcal F}\]
	for each index $i<m$. Thus, $T_i(z_m-z)\varepsilon_m\in\im\mathcal F$ for each $i$, which we will use by passing through the $\rho$ of \autoref{lem:sessplits}: this is equivalent to $\rho(T_i(z_m-z)\varepsilon_m)=0$ for each $i<m$. Now, we note that any $g=\prod_{j=1}^m\sigma_j^{a_j}\sigma\in G$ and $i<m$ will have
	\[\rho(\sigma_ig\varepsilon_m)=\sigma_ig_m\big(\sigma_m^{a_m}-N_i1_{a_m=n_m-1}\big)\varepsilon_m=\sigma_i\rho(g\varepsilon_m),\]
	where in particular the sum in $\rho$ vanished because $m$ is the largest index. (Also, we note $(\sigma_ig)_m=\sigma_ig_m$ because $i<m$.) Extending this linearly over all $g\in G$, we see that
	\[0=\rho(T_i(z_m-z)\varepsilon_m)=T_i\rho((z_m-z)\varepsilon_m)\]
	for each $i<m$. In particular, letting $\rho((z_m-z)\varepsilon_m)=r\varepsilon_m$, we see$r\in\im N_i$ for each $i<m$, so it follows from \autoref{lem:separatenijs} that $r\in\im N_1\cdots N_{m-1}$, so we can find $w\in\ZZ[G]$ such that
	\[\rho((z_m-z)\varepsilon_m)=N_1\cdots N_{m-1}w\varepsilon_m.\]
	Now, for technical reasons we note that any $g=\prod_{j=1}^m\sigma_j^{a_j}$ gives
	\[\rho(g\varepsilon_m)=g_m\big(\sigma_m^{a_m}-N_i1_{a_m=n_m-1}\big)\varepsilon_m,\]
	which can have no $\sigma_m^{n_m-1}$ term in it because this would have to come from $\big(\sigma_m^{a_m}-N_i1_{a_m=n_m-1}\big)$, which manually kills all such terms. As such, $N_1\cdots N_{m-1}w$ should have no $\sigma_m^{n_m-1}$ terms, which means $w$ itself should have no such terms.

	With this in mind, we set $z'\coloneqq z+N_1\cdots N_{m-1}w$. To check that we haven't broken anything, we note that any $i<m$ has
	\[z'\varepsilon_i=z\varepsilon_i+N_1\cdots N_{m-1}w\varepsilon_i\equiv z\varepsilon_i\equiv z_i\varepsilon_i\pmod{\im\mathcal F}\]
	where we note that $N_i\varepsilon_i\equiv0\pmod{\im\mathcal F}$. It remains to deal with $i=m$. Because $w$ features no $\sigma_m^{a_m-1}$ terms, we can check that any $g=\prod_{j=1}^m\sigma_j^{a_j}$ with $a_m<n_m-1$ has
	\[\rho(g\varepsilon_m)=g_m\big(\sigma_m^{a_m}-N_i1_{a_m=n_m-1}\big)\varepsilon_m=g_m\sigma_m^{a_m}\varepsilon_m=g\varepsilon_m,\]
	so $\rho$ will just act as the identity on $w$! Extending this linearly, we see that
	\begin{align*}
		\rho((z_m-z')\varepsilon_m) &= \rho((z_m-z)\varepsilon_m)-\rho(N_1\cdots N_{m-1}w\varepsilon_m) \\
		&= N_1\cdots N_{m-1}w\varepsilon_m-N_1\cdots N_{m-1}w\varepsilon_m \\
		&= 0.
	\end{align*}
	Thus, $(z_m-z')\varepsilon_m\in\im\mathcal F$, so $z_m\varepsilon_m\equiv z\varepsilon_m\pmod{\im\mathcal F}$ as well.
\end{proof}
We are now ready to classify our $1$-cocycles.
\begin{proposition} \label{prop:cocycleclassify}
	Fix everything as in the set-up. If $f\in Z^1(G,\coker\mathcal F)$ is a $1$-cocycle, then there exists $z\in\ZZ[G]$ such that $f(\sigma_i)=z\varepsilon_i$ for each index $i$. Combined with the formula in \autoref{lem:compresscocycle}, this fully determines $f$.
\end{proposition}
\begin{proof}
	We start by noting that each index $i$ has
	\[0=f(1)=f\left(\sigma_i^{n_i}\right)=\sigma_i^{(n_i)}f(\sigma_i)=N_i(f(\sigma_i))\]
	by plugging in $\sigma_i^{n_i}$ into \autoref{lem:compresscocycle}. Thus, \autoref{lem:cocycleforcecoord} grants us some $z_i\in\ZZ[G]$ such that $f(\sigma_i)=z_i\varepsilon_i$ for each index $i$.

	Continuing, we note that each pair of indices $(i,j)$ has
	\[\sigma_if(\sigma_j)+f(\sigma_i)=f(\sigma_i\sigma_j)=f(\sigma_j\sigma_i)=\sigma_jf(\sigma_i)+f(\sigma_j),\]
	so
	\[T_iz_j\varepsilon_j=T_if(\sigma_j)=T_jf(\sigma_i)=T_jz_i\varepsilon_i.\]
	Thus, we know from \autoref{lem:cocycleforcecohere} that there exists $z\in\ZZ[G]$ such that $f(\sigma_i)=z_i\varepsilon_i=z\varepsilon_i$ for each index $i$. This completes the proof.
\end{proof}
Note that \autoref{prop:cocycleclassify} does not say that all the conjured $1$-cocycles are actually $1$-cocycles. It will be beneficial for us to show this by hand, so we postpone it to the next subsection.

\subsection{Verification of 1-Cocycles}
The point of this subsection is to verify that all the $1$-cocycles of \autoref{prop:cocycleclassify} are indeed $1$-cocycles. The main step is in showing that the $1$-cochain $\overline c\in C^1(G,\coker\mathcal F)$ defined by
\[\overline c(g)=\sum_{i=1}^mg_i\sigma_i^{(a_i)}\varepsilon_i\]
where $g\coloneqq\prod_{i=1}^m\sigma_i^{a_i}$ is actually a $1$-cocycle. It will be beneficial for us to do this by hand, which is a matter of brute force. Set $c\in C^1\left(G,\ZZ[G]^m\right)$ defined by
\[c(g)\coloneqq\left(g_i\sigma_i^{(a_i)}\right)^m_{i=1},\]
where $g\coloneqq\prod_{i=1}^m\sigma_i^{a_i}$. We will show that $\im dc\subseteq\im\mathcal F$, which we will mean that $\im\overline{dc}=\im d\overline c=0$, where $f\mapsto\overline f$ is the map $C^\bullet\left(G,\ZZ[G]^m\right)\onto C^\bullet\left(G,\coker\mathcal F\right)$ induced by modding out.

As such, we set $g\coloneqq\prod_{i=1}^m\sigma_i^{a_i}$ and $h\coloneqq\prod_{i=1}^m\sigma_i^{b_i}$ with $0\le a_i,b_i<n_i$ for each $i$. Then, using the division algorithm, write
\[a_i+b_i=n_iq_i+r_i\]
where $q_i\in\{0,1\}$ and $0\le r_i<n_i$ for each $i$. Now, we want to show $dc(g,h)\in\im\mathcal F$, so we begin by writing
\begin{align}
	dc(g,h) &= gc(h)-c(gh)+c(g) \notag \\
	&= g\left(h_i\sigma_i^{(b_i)}\right)_{i=1}^m-\Bigg(\prod_{p=0}^{i-1}\sigma_p^{r_p}\cdot\sigma_i^{(r_i)}\Bigg)_{i=1}^m+\left(g_i\sigma_i^{(a_i)}\right)_{i=1}^m \notag \\
	&= \left(gh_i\sigma_i^{(b_i)}\right)_{i=1}^m-\left(g_ih_i\sigma_i^{(r_i)}\right)_{i=1}^m+\left(g_i\sigma_i^{(a_i)}\right)_{i=1}^m. \label{eq:expandedcocycle}
\end{align}
We now go term-by-term in \autoref{eq:expandedcocycle}. The easiest is the middle term of \autoref{eq:expandedcocycle}, for which we write
\begin{align*}
	g_ih_i\sigma_i^{(r_i)} &= g_ih_i\sigma_i^{(a_i+b_i)}-g_ih_i\sigma_i^{r_i}\sigma_i^{(n_iq_i)} \\
	&= g_ih_i\sigma_i^{(a_i+b_i)}-g_ih_i\sigma_i^{a_i+b_i}\cdot q_iN_i \\
	&= g_ih_i\sigma_i^{(a_i+b_i)}-g_ih_i\cdot q_iN_i,
\end{align*}
where the last equality is because $\sigma_iN_i=N_i$. Thus,
\begin{align*}
	-\left(g_ih_i\sigma_i^{(r_i)}\right)_{i=1}^m &= -\left(g_ih_i\sigma_i^{(a_i+b_i)}\right)_{i=1}^m+\left(g_ih_i\cdot q_iN_i\right)_{i=1}^m \\
	&= -\left(g_ih_i\sigma_i^{(a_i+b_i)}\right)_{i=1}^m+\mathcal F\big((g_ih_iq_i)_i,(0)_{i>j}\big).
\end{align*}
Now, using \autoref{lem:expandgi}, the $i$th coordinate of the left term of \autoref{eq:expandedcocycle} is
\begin{align*}
	gh_i\sigma_i^{(b_i)} &= g_i\sigma_i^{a_i}\Bigg(\prod_{ j=i+1}^{m}\sigma_j^{a_j}\Bigg)h_i\sigma_i^{(b_i)} \\
	&= g_i\Bigg(1+\sum_{j=i+1}^{m}\Bigg(\prod_{q=i+1}^{j-1}\sigma_q^{a_q}\Bigg)\sigma_j^{(a_j)}T_j\Bigg)h_i\sigma_i^{a_i}\sigma_i^{(b_i)} \\
	&= g_ih_i\sigma_i^{a_i}\sigma_i^{(b_i)}+\sum_{j=i+1}^{m}\Bigg(g_i\sigma_i^{a_i}\prod_{q=i+1}^{j-1}\sigma_q^{a_q}\Bigg)h_i\sigma_j^{(a_j)}\sigma_i^{(b_i)}T_j \\
	&= g_ih_i\sigma_i^{a_i}\sigma_i^{(b_i)}+\sum_{j=i+1}^{m}g_jh_i\sigma_j^{(a_j)}\sigma_i^{(b_i)}T_j.
\end{align*}
And lastly, for the right term of \autoref{eq:expandedcocycle}, the $i$th coordinate is
\begin{align*}
	g_i\sigma_i^{(a_i)} &= g_i\Bigg(h_i-\sum_{j=1}^{i-1}h_j\sigma_j^{(b_j)}T_j\Bigg)\sigma_i^{(a_i)} \\
	&= g_ih_i\sigma_i^{(a_i)}-\sum_{j=1}^{i-1}g_ih_j\sigma_i^{(a_i)}\sigma_j^{(b_j)}T_j.
\end{align*}
So to finish, we continue from \autoref{eq:expandedcocycle}, which gives
\begin{align*}
	dc(g,h)-\mathcal F\big((g_ih_iq_i)_i,(0)_{i>j}\big) &= \left(g_ih_i\sigma_i^{a_i}\sigma_i^{(b_i)}\right)_{i=1}^m-\left(g_ih_i\sigma_i^{(a_i+b_i)}\right)_{i=1}^m+\left(g_ih_i\sigma_i^{(a_i)}\right)_{i=1}^m \\
	&\qquad\qquad+\Bigg(\sum_{j=i+1}^{m}g_jh_i\sigma_j^{(a_j)}\sigma_i^{(b_i)}T_j-\sum_{j=1}^{i-1}g_ih_j\sigma_i^{(a_i)}\sigma_j^{(b_j)}T_j\Bigg)_{i=1}^m \\
	&= \Bigg(-\sum_{j=1}^{i-1}g_ih_j\sigma_i^{(a_i)}\sigma_j^{(b_j)}T_j+\sum_{j=i+1}^{m}g_jh_i\sigma_j^{(a_j)}\sigma_i^{(b_i)}T_j\Bigg)_{i=1}^m \\
	&= \mathcal F\left((0)_i,\big(g_ih_j\sigma_i^{(a_i)}\sigma_j^{(b_j)}\big)_{i>j}\right).
\end{align*}
Thus,
\begin{equation}
	dc(g,h) = \mathcal F\left((g_ih_iq_i)_i,\big(g_ih_j\sigma_i^{(a_i)}\sigma_j^{(b_j)}\big)_{i>j}\right)\in\im\mathcal F. \label{eq:computedelta}
\end{equation}
This completes the proof of \autoref{prop:manufacturedcocycle}.

In fact, the above proof has found an explicit element $z$ so that $\mathcal F(z)=dc(g,h)$ for each $g,h\in G$. As such, we recall that we set
\[X\coloneqq\frac{\ZZ[G]^m\times\ZZ[G]^{\binom m2}}{\ker\mathcal F}\]
to give the short exact sequence
\[0\to X\stackrel{\mathcal F}\to\ZZ[G]^m\to\coker\mathcal F\to0.\]
In particular, we can track $\overline c\in Z^1(G,\coker\mathcal F)$ through a boundary morphism: we already have a chosen lift $c\in Z^1(G,\ZZ[G]^m)$ for $\overline c$, and we have also computed $\mathcal F^{-1}\circ dc$ from the above work. This gives the following result.
\begin{cor} \label{cor:deltaccomputation}
	Fix everything as in the set-up. Then the $\overline c$ of \autoref{prop:manufacturedcocycle} has
	\[\delta(c)(g,h)\coloneqq\left((g_ih_iq_i)_i,\big(g_ih_j\sigma_i^{(a_i)}\sigma_j^{(b_j)}\big)_{i>j}\right)\in Z^2(G,X)\]
	where $\delta$ is induced by
	\[0\to X\stackrel{\mathcal F}\to\ZZ[G]^m\to\coker\mathcal F\to0.\]
\end{cor}
\begin{proof}
	This follows from tracking how $\delta$ behaves, using \autoref{eq:computedelta}.
\end{proof}
\begin{remark}
	In some sense, this $\delta(c)$ is exactly the cocycle of \autoref{thm:getcocycle}, where we have abstracted away everything about $A$. We will rigorize this notion in our proof of \autoref{thm:yesitisacocycle}.
\end{remark}
We are now ready to complete the proof of \autoref{prop:allmanufacturedcocycles}. In fact, we show the following stronger result.
\begin{proposition} \label{prop:computeh1cokerF}
	Fix everything as in the set-up. Further, let $\varepsilon\colon\ZZ[G]\to\ZZ$ be the augmentation map sending $\sigma_i\mapsto1$ for each $i$. Then the following are true.
	\begin{listalph}
		\item Given any $z\in\ZZ[G]$, the formula
		\[f(g)=\sum_{i=1}^mg_i\sigma_i^{(a_i)}\cdot z\varepsilon_i=(z\cdot\overline c)(g)\]
		for $g\coloneqq\prod_{i=1}^m\sigma_i^{a_i}$ defines a $1$-cocycle in $Z^1(G,\coker\mathcal F)$. These are all the $1$-cocycles.
		\item If $f\in Z^1(G,\coker\mathcal F)$ is a $1$-cocycle, then $[f]=[\varepsilon(z)\cdot\overline c]$ in $H^1(G,\coker\mathcal F)$, for the $z\in\ZZ[G]$ of \autoref{prop:cocycleclassify}. In particular, $H^1(G,\coker\mathcal F)$ is a cyclic abelian group generated by $[\overline c]$.
	\end{listalph}
\end{proposition}
\begin{proof}
	We proceed one at a time.
	\begin{listalph}
		\item Given $z\in\ZZ[G]$, to see that $f$ is a $1$-cocycle, note that $f=z\cdot\overline c$. Thus, for the $1$-cocycle check, we just note that any $g,h\in G$ have
		\begin{align*}
			f(gh) &= z\cdot\overline c(gh) \\
			&= z\cdot(g\overline c(h)+\overline c(g)) \\
			&= gf(h)+f(g)
		\end{align*}
		because we already know that $\overline c\in Z^1(G,\coker\mathcal F)$.

		To see that these are all the $1$-cocycles, let $f\in Z^1(G,\coker\mathcal F)$ be any $1$-cocycle. Then \autoref{prop:cocycleclassify} promises $z\in\ZZ[G]$ such that $f(\sigma_i)=z\varepsilon_i$ for each index $i$, for which \autoref{lem:compresscocycle} tells us that
		\[f(g)=\sum_{i=1}^mg_i\sigma_i^{(a_i)}f(\sigma_i)=\sum_{i=1}^mg_i\sigma_i^{(a_i)}\cdot z\varepsilon_i\]
		for $g\coloneqq\prod_{i=1}^m\sigma_i^{a_i}$. So $f$ does have the desired form.

		\item Fix $f\in Z^1(G,\coker\mathcal F)$, and conjure the corresponding $z\in\ZZ[G]$ of \autoref{prop:cocycleclassify}. We note in part (a) that $f=z\cdot\overline c$, so it remains to show that $[z\cdot\overline c]=[\varepsilon(z)\cdot\overline c]$ in $H^1(G,\coker\mathcal F)$.
		
		By linearity of $\ZZ[G]$, it suffices to show that $[g\cdot\overline c]=[\overline c]$ for each $g\in G$. By induction on the number of generators $\sigma_i$ appearing in $g\in G$, it suffices to show that $[\sigma_i\cdot\overline c]=[\overline c]$ for each index $i$. Lastly, by rearranging the $\sigma_i$, it suffices to show that $[\sigma_1\cdot\overline c]=[\overline c]$.

		Well, for any $\sigma_i$, we note that
		\[(\sigma_1\overline c-\overline c)(\sigma_i)=\sigma_1\varepsilon_i-\varepsilon_i=T_1\varepsilon_i=T_i\varepsilon_1,\]
		where in the last equality we have used that we're living in $\coker\mathcal F$. Letting $d\colon C^0(G,\coker\mathcal F)\to B^1(G,\coker\mathcal F)$ denote the corresponding differential, we see
		\[(\sigma_1\overline c-\overline c-d\varepsilon_1)(\sigma_i)=T_i\varepsilon_1-(\sigma_i-1)\varepsilon_1=0\]
		for each index $i$. Thus, $\sigma_1\overline c-\overline c-d\varepsilon_1)\in Z^1(G,\coker\mathcal F)$ vanishes on all $\sigma_i$, so \autoref{lem:compresscocycle} tells us that it vanishes on all $g\in G$. It follows $[\sigma_1\overline c-\overline c]=[0]$, which finishes.
	\end{listalph}
	The above parts complete the proof.
\end{proof}
\begin{cor} \label{cor:computeh2x}
	Fix everything as in the set-up. Then $H^2(G,X)$ is a cyclic abelian group generated by $[\delta(\overline c)]$, where $\delta$ is induced by
	\[0\to X\stackrel{\mathcal F}\to\ZZ[G]^m\to\coker\mathcal F\to0.\]
\end{cor}
\begin{proof}
	From the long exact sequence of cohomology, we see that
	\[\delta\colon H^1(G,\coker\mathcal F)\to H^2(G,X)\]
	is an isomorphism because $\ZZ[G]^m$ is projective and hence acyclic. Thus, this follows from (b) of \autoref{prop:computeh1cokerF}.
\end{proof}

\subsection{Tuples via Cohomology}
We continue in the set-up of the previous subsection. The goal of this subsection is to prove \autoref{prop:alternativetuple}. The main idea is that we will be able to finitely generate $\ker\mathcal F$ essentially using the relations of a $\{\sigma_i\}_{i=1}^m$-tuple.

We start with the following basic result.
\begin{lemma} \label{lem:getgens}
	Fix everything as in the set-up. Then $\ker\mathcal F$ contains the following elements.
	\begin{listalph}
		\item $T_p\kappa_p$ for any index $p$.
		\item $N_pN_q\lambda_{pq}$ for any pair of indices $(p,q)$ with $p>q$.
		\item $T_q\kappa_p+N_p\lambda_{pq}$ for any pair of indices $(p,q)$ with $p>q$.
		\item $T_p\kappa_q-N_q\lambda_{pq}$ for any pair of indices $(p,q)$ with $p>q$.
		\item $T_q\lambda_{pr}-T_r\lambda_{pq}-T_p\lambda_{qr}$ for any triplet of indices $(p,q,r)$ with $p>q>r$.
	\end{listalph}
\end{lemma}
\begin{proof}
	We start by showing that all the listed elements are in fact in $\ker\mathcal F$.
	\begin{listalph}
		\item Note that $\mathcal F$ only ever takes the $x_i$ term to $x_iN_i$, so if $x_i=T_i$, then the effect of $x_i$ vanishes.
		\item Similarly, note that $\mathcal F$ only ever takes the $y_{ij}$ term to $y_{ij}T_i$ or $y_{ij}T_j$. As such, if $y_{ij}=N_iN_j$, then the effect of $y_{ij}$ vanishes again.
		\item The only relevant terms are at indices $p$ and $q$. Here, $i=p$ has $\mathcal F$ output
		\[T_qN_p-N_pT_q+0=0.\]
		For $i=q$, we have no $x_q$ term, so we are left with $N_pT_p=0$.
		\item Again, the only relevant terms are at indices $p$ and $q$. This time the interesting term is at $i=q$, where we have
		\[T_pN_q-0+(-N_q)T_p=0.\]
		Then at $i=p$, we simply have $0N_p-(-N_q)T_q+0=0$.
		\item The relevant terms, as usual, are for $i\in\{p,q,r\}$.
		\begin{itemize}
			\item At $i=p$, we have $0-(T_qT_r+(-T_r)T_q)+0=0.$
			\item At $i=q$, we have $0-(-(T_p)T_r)+((-T_r)T_p)=0$.
			\item At $i=r$, we have $0-0+(T_qT_p+(-T_p)T_q)=0$.
		\end{itemize}
	\end{listalph}
	The above checks complete this part of the proof.
\end{proof}
\begin{remark}
	The above elements are intended to encode the relations to be a $\{\sigma_i\}_{i=1}^n$-tuple. We will see this made rigorous in the proof of \autoref{prop:alternativetuple}.
\end{remark}
In fact, the following is true.
\begin{lemma} \label{lem:havegens}
	Fix everything as in the set-up. Then the elements (a)--(e) of \autoref{lem:getgens}, with (b) removed, generate $\ker\mathcal F$.
\end{lemma}
\begin{proof}
	We remark that we callously removed (b) because it is implied (c): $T_q\kappa_p+N_p\lambda_{pq}\in\ker\mathcal F$ implies that
	\[N_q\cdot(T_q\kappa_p+N_p\lambda_{pq})=N_pN_q\lambda_{pq}\]
	is also in $\ker\mathcal F$. Anyway, this proof is long and annoying and hence relegated to \autoref{sec:havegensproof}.
\end{proof}
Here is the payoff for the hard work in \autoref{lem:havegens}.
\propalternativetuple*
\begin{proof}
	Let $\mathcal T$ denote the set of $\{\sigma_i\}_{i=1}^m$-tuples. We now define the map $\varphi\colon\op{Hom}_{\ZZ[G]}(X,A)\to\mathcal T$ by
	\[\varphi\colon f\mapsto\Big(\big(f(\kappa_i)\big)_i,\big(f(\lambda_{ij})\big)_{i>j}\Big).\]
	In other words, we simply read off the values of $f$ from indicators on the coordinates of $X$. It's not hard to see that $\varphi$ is in fact a $G$-module homomorphism, but we will have to check that $\varphi$ is well-defined, for which we have to check the conditions on being a $\{\sigma_i\}_{i=1}^m$-tuple.
	\begin{lemma} \label{lem:kernelisrelations}
		Fix everything as in the set-up, and let $A$ be a $G$-module. Then, given $f\colon\ZZ[G]^m\times\ZZ[G]^{\binom m2}$, we have $\ker\mathcal F\subseteq\ker f$ if and only if
		\[\Big(\big(f(\kappa_i)\big)_i,\big(f(\lambda_{ij})\big)_{i>j}\Big)\]
		is a $\{\sigma_i\}_{i=1}^m$-tuple.
	\end{lemma}
	\begin{proof}
		By \autoref{lem:havegens}, we see $\ker\mathcal F\subseteq\ker f$ if and only if $f$ vanishes on the elements given in \autoref{lem:getgens}. As such, we now run the following checks.
		\begin{enumerate}
			\item We discuss \autoref{eq:tuplefields}. For one, note that $f(\lambda_{ij})\in A$ essentially for free. Now, we note
			\begin{align*}
				f(\kappa_i)\in A^{\langle\sigma_i\rangle} &\iff T_if(\kappa_i)=0 \\
				&\iff f(T_i\kappa_i)=0 \\
				&\iff T_i\kappa_i\in\ker f.
			\end{align*}
			\item We discuss \autoref{eq:tuplerelations}. On one hand, note that $i>j$ has
			\begin{align*}
				N_if(\lambda_{ij})=-T_jf(\lambda_i) &\iff f(N_i\lambda_{ij}+T_j\lambda_i) \\
				&\iff N_i\lambda_{ij}+T_j\lambda_i\in\ker f.
			\end{align*}
			On the other hand,
			\begin{align*}
				-N_jf(\lambda_{ij})=-T_if(\lambda_j) &\iff f(N_j\lambda_{ij}+T_i\lambda_j)=0 \\
				&\iff N_j\lambda_{ij}+T_i\lambda_j\in\ker f.
			\end{align*}
			\item We discuss \autoref{eq:betarelations}. Simply note indices $i>j>k$ have
			\begin{align*}
				T_jf(\lambda_{ik})=T_kf(\lambda_{ij})+T_if(\lambda_{jk}) &\iff f(T_j\lambda_{ik}-T_k\lambda_{ij}-T_i\lambda_{jk})=0 \\
				&\iff T_j\lambda_{ik}-T_k\lambda_{ij}-T_i\lambda_{jk}\in\ker f.
			\end{align*}
		\end{enumerate}
		In total, we see that satisfying the relations to be a $\{\sigma_i\}_{i=1}^m$-tuple exactly encodes the data of having the generators of $\ker\mathcal F$ live in $\ker f$.
	\end{proof}
	So indeed, given $f\colon X\to A$, the above lemma applied to the composite
	\[\ZZ[G]^m\times\ZZ[G]^{\binom m2}\onto X\stackrel{f}\to A\]
	shows that $\varphi(f)\in\mathcal T$.

	To show that $\varphi$ is an isomorphism, we exhibit its inverse; fix some $(\{\alpha_i\},\{\beta_{ij}\}_{i>j})\in\mathcal T$. Well, $\ZZ[G]\times\ZZ[G]^{\binom m2}$ has as a basis the $\kappa_i$ and $\lambda_{ij}$, so we can uniquely define a $G$-module homomorphism $f\colon X\to A$ by
	\[f(\kappa_i)\coloneqq\alpha_i\qquad\text{and}\qquad f(\lambda_{ij})\coloneqq\beta_{ij}\]
	for all relevant indices $i,j$, and in fact the map $\mathcal T\to\op{Hom}_\ZZ\left(\ZZ[G]^m\times\ZZ[G]^{\binom m2},A\right)$ we can see to be a $G$-module homomorphism. However, because these outputs are a $\{\sigma_i\}_{i=1}^m$-tuple, we can read \autoref{lem:kernelisrelations} backward to say that $f$ has kernel containing $\ker\mathcal F$, so in fact we induce a map $\overline f\colon X\to A$.
	
	So in total, we get a $G$-module homomorphism $\psi\colon\mathcal T\to\op{Hom}_{\ZZ[G]}(X,A)$ by
	\[\psi\colon(\{\alpha_i\},\{\beta_{ij}\}_{i>j})\mapsto\overline f,\]
	where $\overline f$ is defined on the basis elements above. Further, $\psi$ is the inverse of $\varphi$ essentially because the $\{\kappa_i\}_i\cup\{\lambda_{ij}\}_{i>j}$ form a basis of $\ZZ[G]^m\times\ZZ[G]^{\binom m2}$. This completes the proof.
\end{proof}
And now because it is so easy, we might as well prove \autoref{thm:yesitisacocycle}.
\thmyesitisacocycle*
\begin{proof}
	The main point is that we have a computation of $\delta(\overline c)$ from \autoref{cor:deltaccomputation}, which we merely need to track through. In particular, fix a $\{\sigma_i\}_{i=1}^m$-tuple $(\{\alpha_i\}_i,\{\beta_{ij}\}_{i>j})$, and let $f\in H^0(G,\op{Hom}_\ZZ(X,A))$ be the corresponding morphism. As such, we may compute
	\[\delta(\overline c)\cup f\colon(g,h)\mapsto\delta(\overline c)(g,h)\otimes_\ZZ gh\cdot f=\delta(\overline c)(g,h)\otimes_\ZZ f.\]
	To pass through evaluation, we set $g\coloneqq\prod_i\sigma_i^{a_i}$ and $h\coloneqq\prod_i\sigma_i^{b_i}$, from which we get
	\begin{align*}
		f(\delta(\overline c)(g,h)) &= f\left((g_ih_iq_i)_i,\big(g_ih_j\sigma_i^{(a_i)}\sigma_j^{(b_j)}\big)_{i>j}\right) \\
		&= \sum_{i=1}^mg_ih_i\floor{\frac{a_i+b_i}{n_i}}\cdot\alpha_i+\sum_{\substack{i,j=1\\i>j}}^mg_ih_j\sigma_i^{(a_i)}\sigma_j^{(b_j)}\cdot\beta_{ij} \\
		&= \sum_{\substack{i,j=1\\i>j}}^m\Bigg(\prod_{p<i}\sigma_p^{a_p}\Bigg)\Bigg(\prod_{q<j}\sigma_q^{b_q}\Bigg)\sigma_i^{(a_i)}\sigma_j^{(b_j)}\beta_{ij}+\sum_{i=1}^mg_ih_i\alpha_i^{\floor{\frac{a_i+b_i}{n_i}}}.
	\end{align*}
	Doing a little more rearrangement and writing this multiplicatively exactly recovers the cocycle of \autoref{thm:getcocycle}. This completes the proof.
\end{proof}

\subsection{Some Loose Ends}
We continue in the set-up and notation of the previous subsection. Though we have proven everything we set out to do in \autoref{sec:overview}, there is more to discuss with our alternate description of tuples. The main goal of this subsection is to prove the following extension of \autoref{prop:alternativetuple}.
\begin{proposition}
	Fix everything as in the set-up, and now let $L/K$ be an extension of local fields with $G=\op{Gal}(L/K)$. Then the isomorphism of \autoref{prop:alternativetuple} descends to an isomorphism between equivalence classes of $\{\sigma_i\}_{i=1}^m$-tuples are canonically isomorphic to $\widehat H^0(G,\op{Hom}_\ZZ(X,L^\times))$.
\end{proposition}
\begin{proof}
	The main point is that the cup product with $\delta(\overline c)$ will induce an isomorphism
	\[\widehat H^0(G,\op{Hom}_\ZZ(X,L^\times))\to H^2(G,L^\times).\]
	Indeed, note that $\mathcal F\colon X\to\ZZ[G]^m$ is an embedding of $\ZZ$-modules, so $X$ is a free abelian group because $\ZZ[G]^m$ is. It follows that $X$ is the character group of an algebraic torus $\mathcal T=\op{Hom}_\ZZ(X,\mathbb G_m)$, so we write $X=X^*(\mathcal T)$. Now, the main point is that we can realize the cup-product map of \autoref{thm:yesitisacocycle} in Tate cohomology as
	\[\cup\colon\widehat H^0(G,\mathcal T(L))\times\widehat H^2(G,X^*(\mathcal T))\to H^2(G,L^\times).\]
	However, by Tate--Nakayama duality, we know that this pairing is non-degenerate. In particular, because $\delta(\overline c)$ generates $H^2(G,X^*(\mathcal T))=H^2(G,X)$ by \autoref{cor:computeh2x}, we know that the map
	\[\delta(\overline c)\cup-\colon\widehat H^0(G,\op{Hom}_\ZZ(X,L^\times))\to H^2(G,L^\times)\]
	must be injective. On the other hand, by taking a cohomology class $[c]\in H^2(G,L^\times)$, lifting to a representative $\{\sigma_i\}_{i=1}^m$-tuple (as in \autoref{thm:classisomorphism}) gives an input to the above cup product map hitting $[c]$. Thus, the above cup product map we already know to be surjective, so it is an isomorphism.

	We now attack the statement directly. Let $\mathcal T$ denote the set of $\{\sigma_i\}_{i=1}^m$-tuples and $\mathcal T_0$ denote the set (in fact, equivalence class) of tuples corresponding to the trivial cohomology class in $H^2(G,L^\times)$. Then we draw the following diagram, which we claim commutes and is made of isomorphisms.
	% https://q.uiver.app/?q=WzAsMyxbMCwwLCJcXG1hdGhjYWwgVC9cXG1hdGhjYWwgVF8wIl0sWzAsMSwiXFx3aWRlaGF0IEheMChHLFxcb3B7SG9tfV9cXFpaKFgsTF5cXHRpbWVzKSkiXSxbMSwxLCJcXHdpZGVoYXQgSF4yKEcsTF5cXHRpbWVzKSJdLFswLDJdLFswLDFdLFsxLDJdXQ==&macro_url=https%3A%2F%2Fraw.githubusercontent.com%2FdFoiler%2Fnotes%2Fmaster%2Fnir.tex
	\[\begin{tikzcd}
		{\mathcal T/\mathcal T_0} \\
		{\widehat H^0(G,\op{Hom}_\ZZ(X,L^\times))} & {\widehat H^2(G,L^\times)}
		\arrow[from=1-1, to=2-2]
		\arrow[from=1-1, to=2-1, dashed]
		\arrow[from=2-1, to=2-2]
	\end{tikzcd}\]
	Namely, the map $\mathcal T/\mathcal T_0\to\widehat H^2(G,L^\times)$ sends an equivalence class of tuples to its cocycle, and it is an isomorphism by \autoref{thm:classisomorphism}. Further, the map $\widehat H^0(G,\op{Hom}_\ZZ(X,L^\times))\to\widehat H^2(G,L^\times)$ is the cup product with $\delta(\overline c)$, and it is an isomorphism as described above.
	
	Lastly, $\mathcal T/\mathcal T_0\to\widehat H^0(G,\op{Hom}_\ZZ(X,L^\times))$ is descended from the morphism of \autoref{prop:alternativetuple}, so the diagram does indeed commute by \autoref{thm:yesitisacocycle}. In particular, this vertical map is well-defined and in fact an isomorphism by the commutativity of the diagram. This completes the proof.
\end{proof}
Another loose end we have to tie up is that we showed $H^2(G,X)$ is cyclic generated by $[\delta(\overline c)]$, but we do not actually know the order. Tracking through Tate--Nakayama duality in the proof will tell us that the order is $\#G$, but this requires $G$ to be a Galois group. Thankfully, we are able to work this out for general $G$ using the rest of the theory that we have built.
\begin{lemma} \label{lem:zivanish}
	Fix everything as in the set-up. If $z\in\ZZ[G]$ has $z\varepsilon_i=0$ in $\coker\mathcal F$, then $z\in\im N_i$.
\end{lemma}
\begin{proof}
	The point is to pass through $\rho$ of \autoref{lem:sessplits}. By possibly rearranging the $\sigma_i$, we may assume that $i=m$. Then, for any $g\coloneqq\prod_{i=1}^m\sigma_i^{a_i}$, we see
	\[\rho(g\varepsilon_m)=g_m\big(\sigma_m^{a_m}-N_m1_{a_m=n_m-1}\big)\varepsilon_m=g\varepsilon_m-g_m1_{a_m=n_m-1}\cdot N_m\varepsilon_m.\]
	Namely, $\rho(g\varepsilon_m)-g\varepsilon_m=N_mz_g\varepsilon_m$ for some $z_g\in\ZZ[G]$.
	
	Extending this linearly, we see that
	\[\rho(z\varepsilon_m)-z\varepsilon_m=w\cdot N_m\varepsilon_m\]
	for some $w\in\ZZ[G]$, but $z\varepsilon_m=0$ in $\coker\mathcal F$ makes this say $z\varepsilon_m=-w\cdot N_m\varepsilon_m$. Because this is now an equality in $\ZZ[G]^m$, we conclude $z=-w\cdot N_m\in N_m$.
\end{proof}
\begin{lemma} \label{lem:computeordc}
	Fix everything as in the set-up. Then $z\cdot\overline c=0$ in $Z^1(G,\coker\mathcal F)$ if and only if $z\in\im N_G$, where $N_G=\sum_{g\in G}g$.
\end{lemma}
\begin{proof}
	In one direction, if $z=N_Gw$, then
	\[z\varepsilon_i=N_Gw\varepsilon_i\equiv0\pmod{\im\mathcal F}\]
	for each index $i$, so it follows that $(z\cdot\overline c)(\sigma_i)=z\varepsilon_i=0$ for each $\sigma_i$. Thus, using \autoref{lem:compresscocycle}, we conclude that $z\cdot\overline c=0$.

	The other direction is more difficult. Suppose that $z\cdot\overline c=0$. In particular, it follows that $(z\cdot\overline c)(\sigma_i)=z\varepsilon_i$ must equal $0$ for each index $i$. In particular, by \autoref{lem:zivanish}, we conclude that $z\in\im N_i$ for each index $i$, which by \autoref{lem:separatenijs} tells us that
	\[z\in\im N_1\cdots N_m=\im N_G.\]
	This completes the proof.
\end{proof}
\begin{prop} \label{prop:finishh1cokerFcomputation}
	Fix everything as in the set-up. Then $H^1(G,\coker\mathcal F)$ is cyclic of order $\#G$, generated by $[\overline c]$.
\end{prop}
\begin{proof}
	To help us use \autoref{prop:computeh1cokerF}, let $\varepsilon\colon\ZZ[G]\to\ZZ$ denote the augmentation map.
	
	Note that we already know $H^1(G,\coker\mathcal F)$ is cyclic generated by $[\overline c]$ by \autoref{prop:computeh1cokerF}, so it only remains to compute the order of $[\overline c]$. On one hand, we have an upper bound on the order of $[\overline c]$ because $H^1(G,\coker\mathcal F)$ is $\#G$-toresion, but we can also see this directly: note that \autoref{lem:computeordc} tells us that
	\[[0]=[N_G\cdot\overline c].\]
	However, $[N_G\cdot\overline c]=[\varepsilon(N_G)\cdot\overline c]=[\#G\cdot\overline c]$ by \autoref{prop:computeh1cokerF}, so we do see that $\#G\cdot\overline c=0$.
	
	It remains to show that $[\overline c]$ has order at least $\#G$. As such, it suffices to show that if $n$ has $[n\cdot\overline c]=[0]$, then $\#G\mid n$. In particular, $n\cdot\overline c$ is a coboundary, so letting $d\colon C^0(G,\coker\mathcal F)\to B^1(G,\coker\mathcal F)$ denote the corresponding differential, we have
	\[n\cdot\overline c=d\left(\sum_{i=1}^mb_i\varepsilon_i\right)=\sum_{i=1}^mb_i(d\varepsilon_i)\]
	for some $\{b_i\}_{i=1}^m\subseteq\ZZ[G]$. Now, $(d\varepsilon_i)(\sigma_j)=T_j\varepsilon_i=T_i\varepsilon_j$ for any pair of indices $(i,j)$, so by the uniqueness of the extension in \autoref{lem:compresscocycle}, we conclude $d\varepsilon_i=T_i\overline c$. Thus, we set
	\[z\coloneqq n-\sum_{i=1}^mb_iT_i\]
	so that $\varepsilon(z)=n$ and $z\cdot\overline c=0$.

	To finish, we note \autoref{lem:computeordc} now tells us that $z\in\im N_G$, so letting $z=N_Gw$, we see that
	\[n=\varepsilon(z)=\varepsilon(N_G)\varepsilon(w)=\#G\cdot\varepsilon(w),\]
	so $\#G\mid n$. This completes the proof.
\end{proof}
\begin{cor}
	Fix everything as in the set-up. Then $H^1(G,\coker\mathcal F)$ is cyclic of order $\#G$, generated by $[\delta(\overline c)]$, where $\delta$ is induced by
	\[0\to X\stackrel{\mathcal F}\to\ZZ[G]^m\to\coker\mathcal F\to0.\]
\end{cor}
\begin{proof}
	As in the proof of \autoref{cor:computeh2x}, we note $\delta\colon H^1(G,\coker\mathcal F)\to H^2(G,X)$ is an isomorphism, so this follows from \autoref{prop:finishh1cokerFcomputation}.
\end{proof}

\newpage
\appendix
\section{Verification of the Cocycle} \label{sec:verifycocycle}
In this section, we verify \autoref{thm:getcocycle}. As such, in this section, we will work under the modified set-up, forgetting about the extension $\mc E$ but letting $(\{\alpha_i\},\{\beta_{ij}\})$ be some $\{\sigma_i\}_{i=1}^m$-tuple.

Here the formula looks like
\[c(g,g')\coloneqq\left[\prod_{1\le j<i\le m}\Bigg(\prod_{1\le k<j}\sigma_k^{a_k+b_k}\Bigg)\Bigg(\prod_{j\le k<i}\sigma_k^{a_k}\Bigg)\beta_{ij}^{(a_ib_j)}\right]\left[\prod_{i=1}^m\Bigg(\prod_{1\le k<i}\sigma_k^{a_k+b_k}\Bigg)\alpha_i^{\floor{\frac{a_i+b_i}{n_i}}}\right],\]
where $g=\prod_i\sigma_i^{a_i}$ and $g'=\prod_i\sigma_i^{b_i}$ with $0\le a_i,b_i<n_i$ and $q_i\coloneqq\floor{(a_i+b_i)/n_i}$. To make this more digestible, we define
\[g_i\coloneqq\prod_{1\le k<i}\sigma_k^{a_k}\]
for any $g=\prod_i\sigma_i^{a_i}\in G$, so we can write down our formula as
\[c(g,g')\coloneqq\left[\prod_{1\le j<i\le m}g_ig'_j\beta_{ij}^{(a_ib_j)}\right]\left[\prod_{i=1}^mg_ig'_i\alpha_i^{\floor{\frac{a_i+b_i}{n_i}}}\right].\]
Now, given $g,g',g''\in G$, we would like to check
\[gc(g',g'')\cdot c(g,g'g'')\stackrel?=c(gg',g'')\cdot c(g,g'),\]
where $g=\prod_i\sigma_i^{a_i}$ and $g'=\prod_i\sigma_i^{b_i}$ and $g''=\prod_i\sigma_i^{c_i}$ with $0\le a_i,b_i,c_i<n_i$.

\subsection{Carries}
We will begin our verification by dealing with carries; we start with the following lemma, intended to beef up our relation \autoref{eq:tuplerelations}.
\begin{lemma}
	Given indices $i>j$ with $a_i,a_j,q_i,q_j\ge0$, we have
	\[\beta_{ij}^{(a_ia_j)}=\beta_{ij}^{(a_i+q_in_i,a_j)}\left(\frac{\sigma_j^{a_j}(\alpha_i)}{\alpha_i}\right)^{q_i}\qquad\text{and}\qquad\beta_{ij}^{(a_ia_j)}=\beta_{ij}^{(a_i,a_j+q_jn_j)}\left(\frac{\alpha_j}{\sigma_i^{a_i}(\alpha_j)}\right)^{q_j}.\]
\end{lemma}
\begin{proof}
	This is a matter of force. For one, we compute
	\begin{align*}
		\beta_{ij}^{(a_i+n_iq_i,a_j)} &= \prod_{p=0}^{a_i+n_iq_i-1}\prod_{q=0}^{a_j-1}\sigma_i^p\sigma_j^q\beta_{ij} \\
		&= \left(\prod_{p=0}^{a_i-1}\prod_{q=0}^{a_j-1}\sigma_i^p\sigma_j^q\beta_{ij}\right)\left(\prod_{q=0}^{a_j-1}\prod_{p=a_i}^{a_i+n_iq_i-1}\sigma_i^p\sigma_j^q\beta_{ij}\right) \\
		&= \beta_{ij}^{(a_ia_j)}\left(\prod_{q=0}^{a_j-1}\sigma_j^q\op N_{L/L_i}(\beta_{ij})\right)^{q_i}.
	\end{align*}
	Now, using the relation $\op N_{L/L_i}(\beta_{ij})=\alpha_i/\sigma_j(\alpha_i)$ from \autoref{eq:tuplerelations}, this becomes
	\begin{align*}
		\beta_{ij}^{(a_i+n_iq_i,a_j)} &= \beta_{ij}^{(a_ia_j)}\left(\prod_{q=0}^{a_j-1}\frac{\sigma_j^q\alpha_i}{\sigma^{j+1}\alpha_i}\right)^{q_i} \\
		&= \beta_{ij}^{(a_ia_j)}\left(\frac{\alpha_i}{\sigma^{a_j}\alpha_i}\right)^{q_i},
	\end{align*}
	which rearranges into what we wanted.

	For the other, we again just compute
	\begin{align*}
		\beta_{ij}^{(a_i,a_j+n_jq_j)} &= \prod_{p=0}^{a_i-1}\prod_{q=0}^{a_j+n_jq_j-1}\sigma_i^p\sigma_j^q\beta_{ij} \\
		&= \left(\prod_{p=0}^{a_i-1}\prod_{q=0}^{a_j-1}\sigma_i^p\sigma_j^q\beta_{ij}\right)\left(\prod_{p=0}^{a_i-1}\prod_{q=q_j}^{a_j+n_jq_j-1}\sigma_i^p\sigma_j^q\beta_{ij}\right) \\
		&= \beta_{ij}^{(a_ia_j)}\left(\prod_{p=0}^{a_i-1}\sigma_i^p\op N_{L/L_q}(\beta_{ij})\right)^{q_i}.
	\end{align*}
	This time, we use the relation $\op N_{L/L_j}(\beta_{ij})=\sigma_i(\alpha_j)/\alpha_j$, which gives
	\begin{align*}
		\beta_{ij}^{(a_i,a_j+n_jq_j)} &= \beta_{ij}^{(a_ia_j)}\left(\prod_{p=0}^{a_i-1}\frac{\sigma_i^{p+1}(\alpha_j)}{\sigma_i^p(\alpha_j)}\right)^{q_i} \\
		&= \beta_{ij}^{(a_ia_j)}\left(\frac{\sigma_i^{a_j}(\alpha_j)}{\alpha_j}\right)^{q_i},
	\end{align*}
	which again rearranges into the desired.
\end{proof}
We are now ready to begin the computation, dealing with carries to start. Use the division algorithm to write
\[a_i+b_i=n_iu_i+x_i\qquad\text{and}\qquad b_i+c_i=n_iv_i+y_i,\]
where $u_i,v_i\in\{0,1\}$ and $0\le x_i,y_i<n_i$ for each $i$. We start by collecting remainder terms on the side of $gc(g',g'')\cdot c(g,g'g'')$.
\begin{enumerate}
	\item Note
	\begin{align*}
		gc(g',g'') &= g\left[\prod_{1\le j<i\le m}g_i'g''_j\beta_{ij}^{(b_ic_j)}\right]\cdot g\left[\prod_{i=1}^mg'_ig''_i\alpha_i^{v_i}\right],
	\end{align*}
	so we set
	\[R_1\coloneqq\prod_{i=1}^mgg'_ig''_i\alpha_i^{v_i}\]
	to be our remainder term.
	\item Note
	\begin{align*}
		c(g,g'g'') &= \left[\prod_{1\le j<i\le m}g_ig'_jg''_j\beta_{ij}^{(a_iy_j)}\right]\left[\prod_{i=1}^mg_ig'_ig''_i\alpha_i^{\floor{\frac{a_i+y_i}{n_i}}}\right] \\
		&= \left[\prod_{1\le j<i\le m}g_ig'_jg''_j\beta_{ij}^{(a_i,b_j+c_j)}\cdot g_ig'_jg''_j\left(\frac{\alpha_j}{\sigma_i^{a_i}\alpha_j}\right)^{v_i}\right]\left[\prod_{i=1}^mg_ig'_ig''_i\alpha_i^{\floor{\frac{a_i+y_i}{n_i}}}\right] \\
		&= \left[\prod_{1\le j<i\le m}g_ig'_jg''_j\beta_{ij}^{(a_i,b_j+c_j)}\right]\left[\prod_{1\le j<i\le m}g_ig'_jg''_j\left(\frac{\alpha_j}{\sigma_i^{a_i}\alpha_j}\right)^{v_i}\right]\left[\prod_{i=1}^mg_ig'_ig''_i\alpha_i^{\floor{\frac{a_i+y_i}{n_i}}}\right],
	\end{align*}
	so we set
	\[R_2\coloneqq\left[\prod_{1\le j<i\le m}g_ig'_jg''_j\left(\frac{\alpha_j}{\sigma_i^{a_i}\alpha_j}\right)^{v_i}\right]\left[\prod_{i=1}^mg_ig'_ig''_i\alpha_i^{\floor{\frac{a_i+y_i}{n_i}}}\right]\]
	to be our remainder term.
	\item Lastly, we collect our remainders. Observe
	\begin{align*}
		R_2 &= \left[\prod_{j=1}^mg'_jg''_j\Bigg(\prod_{i=j+1}^mg_i\cdot\frac{\alpha_j}{\sigma_i^{a_i}\alpha_j}\Bigg)^{v_i}\right]\left[\prod_{i=1}^mg_ig'_ig''_i\alpha_i^{\floor{\frac{a_i+y_i}{n_i}}}\right] \\
		&= \left[\prod_{j=1}^mg'_jg''_j\Bigg(\prod_{i=j+1}^m\frac{(\sigma_1^{a_1}\cdots\sigma_{i-1}^{a_{i-1}})\alpha_j}{(\sigma_1^{a_1}\cdots\sigma_{i-1}^{a_{i-1}})\sigma_i^{a_i}\alpha_j}\Bigg)^{v_i}\right]\left[\prod_{i=1}^mg_ig'_ig''_i\alpha_i^{\floor{\frac{a_i+y_i}{n_i}}}\right] \\
		&= \left[\prod_{j=1}^mg'_jg''_j\Bigg(\prod_{i=j+1}^m\frac{g_i\alpha_j}{g_{i+1}\alpha_j}\Bigg)^{v_i}\right]\left[\prod_{i=1}^mg_ig'_ig''_i\alpha_i^{\floor{\frac{a_i+y_i}{n_i}}}\right] \\
		&= \left[\prod_{j=1}^mg'_jg''_j\cdot\frac{g_{j+1}\alpha_j^{v_j}}{g\alpha_j^{v_j}}\right]\left[\prod_{i=1}^mg_ig'_ig''_i\alpha_i^{\floor{\frac{a_i+y_i}{n_i}}}\right].
	\end{align*}
	We now note that $g_{j+1}\alpha_j=g_j\alpha_j$ because $\alpha_j$ is fixed by $\sigma_j$. As such,
	\begin{align*}
		R_1R_2 &= \left[\prod_{i=1}^mgg'_ig''_i\alpha_i^{v_i}\right]\left[\prod_{i=1}^mg'_ig''_i\cdot\frac{g_i\alpha_i^{v_i}}{g\alpha_i^{v_i}}\right]\left[\prod_{i=1}^mg_ig'_ig''_i\alpha_i^{\floor{\frac{a_i+y_i}{n_i}}}\right] \\
		&= \prod_{i=1}^mg_ig'_ig''_i\alpha_i^{v_i+\floor{\frac{a_i+y_i}{n_i}}},
	\end{align*}
	which is nice enough for us now.
\end{enumerate}
Now, we collect remainder terms from $c(gg',g'')\cdot c(g,g')$.
\begin{enumerate}
	\item Note
	\begin{align*}
		c(gg',g'') &= \left[\prod_{1\le j<i\le m}g_ig'_ig''_j\beta_{ij}^{(x_ic_j)}\right]\left[\prod_{i=1}^mg_ig'_ig''_i\alpha_i^{\floor{\frac{x_i+c_i}{n_i}}}\right] \\
		&= \left[\prod_{1\le j<i\le m}g_ig'_ig''_j\beta_{ij}^{(a_i+b_i,c_j)}\cdot g_ig'_ig''_j\left(\frac{\sigma_j^{c_j}\alpha_i}{\alpha_i}\right)^{u_i}\right]\left[\prod_{i=1}^mg_ig'_ig''_i\alpha_i^{\floor{\frac{x_i+c_i}{n_i}}}\right] \\
		&= \left[\prod_{1\le j<i\le m}g_ig'_ig''_j\beta_{ij}^{(a_i+b_i,c_j)}\right]\left[\prod_{1\le j<i\le m}g_ig'_ig''_j\left(\frac{\sigma_j^{c_j}\alpha_i}{\alpha_i}\right)^{u_i}\right]\left[\prod_{i=1}^mg_ig'_ig''_i\alpha_i^{\floor{\frac{x_i+c_i}{n_i}}}\right],
	\end{align*}
	so we set
	\[R_3\coloneqq\left[\prod_{1\le j<i\le m}g_ig'_ig''_j\left(\frac{\sigma_j^{c_j}\alpha_i}{\alpha_i}\right)^{u_i}\right]\left[\prod_{i=1}^mg_ig'_ig''_i\alpha_i^{\floor{\frac{x_i+c_i}{n_i}}}\right].\]
	\item Note
	\[c(g,g') = \left[\prod_{1\le j<i\le m}g_ig'_j\beta_{ij}^{(a_ib_j)}\right]\left[\prod_{i=1}^mg_ig'_i\alpha_i^{u_i}\right],\]
	so we set
	\[R_4\coloneqq\left[\prod_{i=1}^mg_ig'_i\alpha_i^{u_i}\right].\]
	\item Lastly, we collect our remainder terms. Observe
	\begin{align*}
		R_3 &= \left[\prod_{i=1}^mg_ig'_i\Bigg(\prod_{j=1}^{i-1}g''_j\cdot\frac{\sigma_j^{c_j}\alpha_i}{\alpha_i}\Bigg)^{u_i}\right]\left[\prod_{i=1}^mg_ig'_ig''_i\alpha_i^{\floor{\frac{x_i+c_i}{n_i}}}\right] \\
		&= \left[\prod_{i=1}^mg_ig'_i\Bigg(\prod_{j=1}^{i-1}\frac{(\sigma_1^{c_1}\cdots\sigma_{j-1}^{c_{j-1}})\sigma_j^{c_j}\alpha_i}{(\sigma_1^{c_1}\cdots\sigma_{j-1}^{c_{j-1}})\alpha_i}\Bigg)^{u_i}\right]\left[\prod_{i=1}^mg_ig'_ig''_i\alpha_i^{\floor{\frac{x_i+c_i}{n_i}}}\right] \\
		&= \left[\prod_{i=1}^mg_ig'_i\Bigg(\prod_{j=1}^{i-1}\frac{g''_{j+1}\alpha_i}{g''_j\alpha_i}\Bigg)^{u_i}\right]\left[\prod_{i=1}^mg_ig'_ig''_i\alpha_i^{\floor{\frac{x_i+c_i}{n_i}}}\right] \\
		&= \left[\prod_{i=1}^mg_ig'_i\cdot\frac{g''_i\alpha_i^{u_i}}{\alpha_i^{u_i}}\right]\left[\prod_{i=1}^mg_ig'_ig''_i\alpha_i^{\floor{\frac{x_i+c_i}{n_i}}}\right].
	\end{align*}
	Thus,
	\begin{align*}
		R_3R_4 &= \left[\prod_{i=1}^mg_ig'_i\cdot\frac{g''_i\alpha_i^{u_i}}{\alpha_i^{u_i}}\right]\left[\prod_{i=1}^mg_ig'_ig''_i\alpha_i^{\floor{\frac{x_i+c_i}{n_i}}}\right]\left[\prod_{i=1}^mg_ig'_i\alpha_i^{u_i}\right] \\
		&= \prod_{i=1}^mg_ig'_ig''_i\alpha_i^{u_i+\floor{\frac{x_i+c_i}{n_i}}},
	\end{align*}
	which is again simple enough for our purposes.
\end{enumerate}
We now note that, for each $i$,
\[u_i+\floor{\frac{x_i+c_i}{n_i}}=\floor{\frac{a_i+b_i+c_i}{n_i}}=v_i+\floor{\frac{a_i+y_i}{n_i}}\]
by how carried addition behaves. It follows that
\[R_1R_2=\prod_{i=1}^mg_ig'_ig''_i\alpha_i^{v_i+\floor{\frac{a_i+y_i}{n_i}}}=\prod_{i=1}^mg_ig'_ig''_i\alpha_i^{u_i+\floor{\frac{x_i+c_i}{n_i}}}=R_3R_4.\]
Thus, it suffices to show that
\[\frac{gc(g',g'')}{R_1}\cdot\frac{c(g,g'')}{R_2}\stackrel?=\frac{c(gg',g'')}{R_3}\cdot\frac{c(g,g')}{R_4},\]
which is equivalent to
\[g\left[\prod_{1\le j<i\le m}g_i'g''_j\beta_{ij}^{(b_ic_j)}\right]\cdot\left[\prod_{1\le j<i\le m}g_ig'_jg''_j\beta_{ij}^{(a_i,b_j+c_j)}\right]\stackrel?=\left[\prod_{1\le j<i\le m}g_ig'_ig''_j\beta_{ij}^{(a_i+b_i,c_j)}\right]\cdot\left[\prod_{1\le j<i\le m}g_ig'_j\beta_{ij}^{(a_ib_j)}\right]\]
by the work above.

\subsection{Finishing}
We need to verify that
\[g\left[\prod_{1\le j<i\le m}g_i'g''_j\beta_{ij}^{(b_ic_j)}\right]\cdot\left[\prod_{1\le j<i\le m}g_ig'_jg''_j\beta_{ij}^{(a_i,b_j+c_j)}\right]\stackrel?=\left[\prod_{1\le j<i\le m}g_ig'_ig''_j\beta_{ij}^{(a_i+b_i,c_j)}\right]\cdot\left[\prod_{1\le j<i\le m}g_ig'_j\beta_{ij}^{(a_ib_j)}\right]\]
as discussed in the previous subsection.

Before beginning the check, we recall the relations on the $\beta$s from \autoref{eq:betarelations} can be written as
\[\frac{\sigma_2(\beta_{31})}{\beta_{31}}=\frac{\sigma_1(\beta_{32})}{\beta_{32}}\cdot\frac{\sigma_3(\beta_{21})}{\beta_{21}},\]
because we only have one triple $(i,j,k)$ of indices with $i>j>k$. This is somewhat difficult to deal with directly, so we quickly show a more general version.
\begin{lemma} \label{lem:betterbetarelation}
	Fix indices with $i>j>k$, and let $a_i,a_j,a_k\ge0$. Then
	\[\frac{\sigma_j^{a_j}\beta_{ik}^{(a_ia_k)}}{\beta_{ik}^{(a_ia_k)}}=\frac{\sigma_k^{a_k}\beta_{ij}^{(a_ia_j)}}{\beta_{ij}^{(a_ia_j)}}\cdot\frac{\sigma_i^{a_i}\beta_{jk}^{(a_ja_k)}}{\beta_{jk}^{(a_ja_k)}}.\]
\end{lemma}
\begin{proof}
	We simply compute
	\begin{align*}
		\frac{\sigma_i^{a_i}\beta_{jk}^{(a_ja_k)}}{\beta_{jk}^{(a_ja_k)}}\cdot\frac{\sigma_k^{a_k}\beta_{ij}^{(a_ia_j)}}{\beta_{ij}^{(a_ia_j)}} &= \prod_{r=0}^{a_i-1}\frac{\sigma_i^{r+1}\beta_{jk}^{(a_ja_k)}}{\sigma_i^r\beta_{jk}^{(a_ja_k)}}\cdot\prod_{p=0}^{a_k-1}\frac{\sigma_k^{p+1}\beta_{ij}^{(a_ia_j)}}{\sigma_k^p\beta_{ij}^{(a_ia_j)}} \\
		&= \prod_{p=0}^{a_k-1}\prod_{q=0}^{a_j-1}\prod_{r=0}^{a_i-1}\left(\frac{\sigma_k^p\sigma_j^q\sigma_i^{r+1}\beta_{jk}}{\sigma_k^p\sigma_j^q\sigma_i^r\beta_{jk}}\cdot\frac{\sigma_k^{p+1}\sigma_j^q\sigma_i^r\beta_{ij}}{\sigma_k^p\sigma_j^q\sigma_i^r\beta_{ij}}\right) \\
		&= \prod_{p=0}^{a_k-1}\prod_{q=0}^{a_j-1}\prod_{r=0}^{a_i-1}\sigma_k^p\sigma_j^q\sigma_i^r\left(\frac{\sigma_i\beta_{jk}}{\beta_{jk}}\cdot\frac{\sigma_k\beta_{ij}}{\beta_{ij}}\right) \\
		&= \prod_{p=0}^{a_k-1}\prod_{q=0}^{a_j-1}\prod_{r=0}^{a_i-1}\sigma_k^p\sigma_j^q\sigma_i^r\left(\frac{\sigma_j\beta_{ik}}{\beta_{ik}}\right),
	\end{align*}
	where in the last equality we have use the relation on the $\beta$s. Continuing,
	\begin{align*}
		\frac{\sigma_i^{a_i}\beta_{jk}^{(a_ja_k)}}{\beta_{jk}^{(a_ja_k)}}\cdot\frac{\sigma_k^{a_k}\beta_{ij}^{(a_ia_j)}}{\beta_{ij}^{(a_ia_j)}} &= \prod_{q=0}^{a_j-1}\left(\prod_{p=0}^{a_k-1}\prod_{r=0}^{a_i-1}\frac{\sigma_j^{q+1}\sigma_k^p\sigma_i^r\beta_{ik}}{\sigma_j^q\sigma_k^p\sigma_i^r\beta_{ik}}\right) \\
		&= \prod_{q=0}^{a_j-1}\frac{\sigma_j^{q+1}\beta_{ik}^{(a_ia_k)}}{\sigma_j^q\beta_{ik}^{(a_ia_k)}} \\
		&= \frac{\sigma_j^{a_j}\beta_{ik}^{(a_ia_k)}}{\beta_{ik}^{(a_ia_k)}},
	\end{align*}
	which is what we wanted.
\end{proof}
We now proceed with the check, by induction. More precisely, we claim that any $m'\le m$ gives
\[g_{m'+1}\left[\prod_{j<i\le m'}g_i'g''_j\beta_{ij}^{(b_ic_j)}\right]\left[\prod_{j<i\le m'}g_ig'_jg''_j\beta_{ij}^{(a_i,b_j+c_j)}\right]\stackrel?=\left[\prod_{j<i\le m'}g_ig'_ig''_j\beta_{ij}^{(a_i+b_i,c_j)}\right]\left[\prod_{j<i\le m'}g_ig'_j\beta_{ij}^{(a_ib_j)}\right]\]
which we will show by induction on $m'$. For $m'=1$, there is nothing to say because there are no indices $i>j$.

So now suppose we have equality for $m'<m$, and we give equality for $m''\coloneqq m'+1$. That is, we want to show that
\[g_{m'+2}\prod_{j<i\le m'+1}g_i'g''_j\beta_{ij}^{(b_ic_j)}\cdot\prod_{j<i\le m'+1}g_ig'_jg''_j\beta_{ij}^{(a_i,b_j+c_j)}\stackrel?=\prod_{j<i\le m'+1}g_ig'_ig''_j\beta_{ij}^{(a_i+b_i,c_j)}\cdot\prod_{j<i\le m'+1}g_ig'_j\beta_{ij}^{(a_ib_j)}\]
but by the inductive hypothesis it suffices for
\[\frac{\displaystyle g_{m''+1}\prod_{j<i\le m'+1}g_i'g''_j\beta_{ij}^{(b_ic_j)}}{\displaystyle g_{m'+1}\prod_{j<i\le m'}g_i'g''_j\beta_{ij}^{(b_ic_j)}}\cdot
\frac{\displaystyle\prod_{j<i\le m'+1}g_ig'_jg''_j\beta_{ij}^{(a_i,b_j+c_j)}}{\displaystyle\prod_{j<i\le m'}g_ig'_jg''_j\beta_{ij}^{(a_i,b_j+c_j)}}
\stackrel?=
\frac{\displaystyle\prod_{j<i\le m'+1}g_ig'_ig''_j\beta_{ij}^{(a_i+b_i,c_j)}}{\displaystyle\prod_{j<i\le m'}g_ig'_ig''_j\beta_{ij}^{(a_i+b_i,c_j)}}\cdot
\frac{\displaystyle\prod_{j<i\le m'+1}g_ig'_j\beta_{ij}^{(a_ib_j)}}{\displaystyle\prod_{j<i\le m'}g_ig'_j\beta_{ij}^{(a_ib_j)}}\]
which is collapses to
\[\frac{\displaystyle g_{m''+1}\prod_{j<i\le m'+1}g_i'g''_j\beta_{ij}^{(b_ic_j)}}{\displaystyle g_{m'+1}\prod_{j<i\le m'}g_i'g''_j\beta_{ij}^{(b_ic_j)}}\cdot
\prod_{j\le m'}g_{m''}g'_jg''_j\beta_{m''j}^{(a_{m''},b_j+c_j)}
\stackrel?=
\prod_{j\le m'}g_{m''}g'_{m''}g''_j\beta_{m''j}^{(a_{m''}+b_{m''},c_j)}\cdot
\displaystyle\prod_{j\le m'}g_{m''}g'_j\beta_{ij}^{(a_{m''}b_j)}\]
because the terms with $i<m''=m'+1$ got cancelled in the rightmost three products. Rearranging, this is the same as
\[\frac{\displaystyle g_{m''+1}\prod_{j<i\le m'+1}g_i'g''_j\beta_{ij}^{(b_ic_j)}}{\displaystyle g_{m'+1}\prod_{j<i\le m'}g_i'g''_j\beta_{ij}^{(b_ic_j)}}
\stackrel?=
\frac{\displaystyle\prod_{j<m''}g_{m''}g'_{m''}g''_j\beta_{m''j}^{(a_{m''}+b_{m''},c_j)}\cdot
\displaystyle\prod_{j<m''}g_{m''}g'_j\beta_{m''j}^{(a_{m''}b_j)}}
{\displaystyle\prod_{j<m''}g_{m''}g'_jg''_j\beta_{m''j}^{(a_{m''},b_j+c_j)}}.\]
Peeling off the $i=m''=m'+1$ terms from the left-hand side numerator, we're showing
\[\frac{\displaystyle g_{m''+1}\prod_{j<i\le m'}g_i'g''_j\beta_{ij}^{(b_ic_j)}}{\displaystyle g_{m'+1}\prod_{j<i\le m'}g_i'g''_j\beta_{ij}^{(b_ic_j)}}
\stackrel?=
\frac{\displaystyle\prod_{j<m''}g_{m''}g'_{m''}g''_j\beta_{m''j}^{(a_{m''}+b_{m''},c_j)}\cdot
\displaystyle\prod_{j<m''}g_{m''}g'_j\beta_{m''j}^{(a_{m''}b_j)}}
{\displaystyle\prod_{j<m''}g_{m''+1}g_{m''}'g''_j\beta_{m''j}^{(b_{m''}c_j)}\cdot
\prod_{j<m''}g_{m''}g'_jg''_j\beta_{m''j}^{(a_{m''},b_j+c_j)}}.\]
We take a moment to simplify the left-hand side with \autoref{lem:betterbetarelation} by writing
\begin{align*}
	g_{m'+1}\prod_{j<i\le m'}g_i'g''_j\left(\frac{\sigma_{m''}^{a_{m''}}\beta_{ij}^{(b_ic_j)}}{\beta_{ij}^{(b_ic_j)}}\right) &= g_{m''}\prod_{j<i\le m'}g_i'g''_j\left(\frac{\sigma_i^{b_i}\beta_{m''j}^{(a_{m''}c_j)}}{\beta_{m''j}^{(a_{m''}c_j)}}\cdot\frac{\beta_{m''i}^{(a_{m''}b_i)}}{\sigma_j^{c_j}\beta_{m''i}^{(a_{m''}b_i)}}\right) \\
	&= g_{m''}\left[\prod_{j=1}^{m'}g''_j\prod_{i=j+1}^{m'}g_i'\left(\frac{\sigma_i^{b_i}\beta_{m''j}^{(a_{m''}c_j)}}{\beta_{m''j}^{(a_{m''}c_j)}}\right)\cdot
	\prod_{i=1}^{m'}g_i'\prod_{j=1}^{i-1}g_j''\left(\frac{\beta_{m''i}^{(a_{m''}b_i)}}{\sigma_j^{c_j}\beta_{m''i}^{(a_{m''}b_i)}}\right)\right] \\
	&= g_{m''}\left[\prod_{j=1}^{m'}\frac{g'_{m'+1}g''_j\beta_{m''j}^{(a_{m''}c_j)}}{g'_{j+1}g''_j\beta_{m''j}^{(a_{m''}c_j)}}\cdot
	\prod_{i=1}^{m'}\frac{g_i'\beta_{m''i}^{(a_{m''}b_i)}}{g_i'g_i''\beta_{m''i}^{(a_{m''}b_i)}}\right] \\
	&= g_{m''}\left[\prod_{j<m''}\frac{g'_{m''}g''_j\beta_{m''j}^{(a_{m''}c_j)}}{g'_{j+1}g''_j\beta_{m''j}^{(a_{m''}c_j)}}\cdot
	\prod_{j<m''}\frac{g_j'\beta_{m''j}^{(a_{m''}b_j)}}{g_j'g_j''\beta_{m''j}^{(a_{m''}b_j)}}\right]
\end{align*}
after doing a lot of telescoping. Now, we can remove $g_{m''}$ everywhere to give
\[\prod_{j<m''}\frac{g'_{m''}g''_j\beta_{m''j}^{(a_{m''}c_j)}}{g'_{j+1}g''_j\beta_{m''j}^{(a_{m''}c_j)}}\cdot
\prod_{j<m''}\frac{g_j'\beta_{m''j}^{(a_{m''}b_j)}}{g_j'g_j''\beta_{m''j}^{(a_{m''}b_j)}}
\stackrel?=
\frac{\displaystyle\prod_{j<m''}g'_{m''}g''_j\beta_{m''j}^{(a_{m''}+b_{m''},c_j)}\cdot
\displaystyle\prod_{j<m''}g'_j\beta_{m''j}^{(a_{m''}b_j)}}
{\displaystyle\prod_{j<m''}g'_{m''+1}g''_j\beta_{m''j}^{(b_{m''}c_j)}\cdot
\prod_{j<m''}g'_jg''_j\beta_{m''j}^{(a_{m''},b_j+c_j)}},\]
or
\[\prod_{j<m''}\frac{g'_{m''}g''_j\beta_{m''j}^{(a_{m''}c_j)}}{g'_{j+1}g''_j\beta_{m''j}^{(a_{m''}c_j)}}
\stackrel?=
\frac{\displaystyle\prod_{j<m''}g'_{m''}g''_j\beta_{m''j}^{(a_{m''}+b_{m''},c_j)}\cdot
\displaystyle\prod_{j<m''}g'_jg''_j\beta_{m''j}^{(a_{m''}b_j)}}
{\displaystyle\prod_{j<m''}g'_{m''+1}g''_j\beta_{m''j}^{(b_{m''}c_j)}\cdot
\prod_{j<m''}g'_jg''_j\beta_{m''j}^{(a_{m''},b_j+c_j)}}.\]
Rearranging, we want
\[\prod_{j<m''}
\frac{g'_jg''_j\beta_{m''j}^{(a_{m''},b_j+c_j)}}
{g'_jg''_j\beta_{m''j}^{(a_{m''}b_j)}\cdot
g'_{j+1}g''_j\beta_{m''j}^{(a_{m''}c_j)}}
\stackrel?=\prod_{j<m''}
\frac{g'_{m''}g''_j\beta_{m''j}^{(a_{m''}+b_{m''},c_j)}}
{g'_{m''}g''_j\beta_{m''j}^{(a_{m''}c_j)}\cdot
g'_{m''+1}g''_j\beta_{m''j}^{(b_{m''}c_j)}},\]
which is
\[\prod_{j<m''}
g'_jg''_j\left(\frac{\beta_{m''j}^{(a_{m''},b_j+c_j)}}
{\beta_{m''j}^{(a_{m''}b_j)}\cdot
\sigma_j^{b_j}\beta_{m''j}^{(a_{m''}c_j)}}\right)
\stackrel?=\prod_{j<m''}
g'_{m''}g''_j\left(\frac{\beta_{m''j}^{(a_{m''}+b_{m''},c_j)}}
{\beta_{m''j}^{(a_{m''}c_j)}\cdot
\sigma_{m''}^{a_{m''}}\beta_{m''j}^{(b_{m''}c_j)}}\right).\]
However, by definition of the $\beta_{ij}^{(xy)}$, we see that
\[\frac{\beta_{m''j}^{(a_{m''},b_j+c_j)}}
{\beta_{m''j}^{(a_{m''}b_j)}\cdot
\sigma_j^{b_j}\beta_{m''j}^{(a_{m''}c_j)}}=\frac{\beta_{m''j}^{(a_{m''}+b_{m''},c_j)}}
{\beta_{m''j}^{(a_{m''}c_j)}\cdot
\sigma_{m''}^{a_{m''}}\beta_{m''j}^{(b_{m''}c_j)}}=1,\]
so everything does indeed cancel out properly. This completes the check.

\section{Computation of \texorpdfstring{$\ker\mathcal F$}{ker F}} \label{sec:havegensproof}
In this section we give a proof of \autoref{lem:havegens}. As such, we will use all the context from the statement and proceed directly with the proof; as mentioned earlier, we may add (b) back to our list of generators because it is induced by (c). Pick up some $z\coloneqq((x_i)_i,(y_{ij})_{i>j})\in\ker\mathcal F$, which is equivalent to saying
\[x_iN_i-\sum_{j=1}^{i-1}y_{ij}T_j+\sum_{j=i+1}^my_{ji}T_j=0\]
for each index $i$. We want to write $z$ as a $\ZZ[G]$-linear combination of the elements from (a)--(e). The main idea will be to slowly subtract out $\ZZ[G]$-linear combinations of the above elements (which does not affect $z\in\ker\mathcal F$) until we can prove that we have $0$ left over. We start with the $x_i$ terms, which we do in two steps.
\begin{enumerate}
	\item We begin by dealing with the $x_i$ terms. Fix some index $p$, and we will subtract out a suitable $\ZZ[G]$-linear combination of the above generators to set $x_p=0$ while not changing the other $x_i$ terms. Well, using the element
	\[\kappa_pT_p,\tag{a}\]
	we may assume that $x_p$ has no $\sigma_p$ terms because $\sigma_p\equiv1\pmod{T_p}$. Then for each $q<p$, we can subtract out a suitable multiple of
	\[T_q\kappa_p+N_p\lambda_{pq}\tag{c}\]
	to make it so that we may assume $x_p$ has no $\sigma_q$ terms because $\sigma_q\equiv1\pmod{T_q}$. Similarly, for each $q>p$, we can subtract out a suitable multiple of
	\[T_q\kappa_p-N_p\lambda_{pq}\tag{d}\]
	to make it so that we may assume $x_p$ has no $\sigma_q$ terms because $\sigma_q\equiv1\pmod{T_q}$.

	\item Thus, the above process allows us to assume that $x_p\in\ZZ$, and the above linear combinations have not affected any $x_i$ for $i\ne p$. We now use the fact that $z\in\ker\mathcal F$. Indeed, we know that
	\[x_pN_p-\sum_{j=1}^{p-1}y_{pj}T_j+\sum_{j=p+1}^my_{jp}T_j=0.\]
	Applying the augmentation map $\varepsilon\colon\ZZ[G]\to\ZZ$, sending $\varepsilon\colon\sigma_i\mapsto1$ for each index $i$, we see that $x_p\in\ZZ$ implying that $x_p$ remains fixed. On the other hand $\varepsilon\colon T_j\mapsto0$ for each index $j$ and $\varepsilon\colon N_p\mapsto n_p$, so we are left with
	\[n_px_p=0.\]
	Because $n_p\ne0$ (it's the order of $\sigma_p$), we conclude that $x_p=0$. Applying this argument to the other $x_i$ terms, we conclude that we may assume $x_i=0$ for each $i$.
\end{enumerate}
It remains to deal with the $y_{ij}$ terms, which is a little more involved. For reference, we are showing that
\[-\sum_{j=1}^{i-1}y_{ij}T_j+\sum_{j=i+1}^my_{ji}T_j=0\]
for each index $i$ implies that $z=((0)_i,(y_{ij})_{i>j})$ is a $\ZZ[G]$-linear combination of the terms from (b) and (e).

We will now more or less proceed with the $y_{ij}$ by induction on $m$, allowing the group $G$ (in its number of generators $m$) to be changed in the process. For $m=1$, there is nothing to say because there is no $y_{ij}$ term at all. For a taste of how we will use \autoref{lem:separatenijs}, we also work out $m=2$: our equations read
\[\underbrace{-y_{21}T_1=0}_{i=1}\qquad\text{and}\qquad\underbrace{y_{21}T_2=0}_{i=2}.\]
Thus, $y_{21}\in(\ker T_1)\cap(\ker T_2)=(\im N_1)\cap(\im N_2)$, which is $\im N_1N_2$ by \autoref{lem:separatenijs}.

We now proceed with the general case; take $m>2$. Let $G'\coloneqq\langle\sigma_2,\ldots,\sigma_m\rangle$, which has $m-1$ generators. By the inductive hypothesis, we may assume the statement for $G'$. Explicitly, we will assume that, if $(y_{ij}')_{i>j\ge2}\in\ZZ[G']^{\binom{m-1}2}$ are variables satisfying
\[-\sum_{j=2}^{i-1}y_{ij}'T_j+\sum_{j=i+1}^my_{ji}'T_j=0\]
for each index $i\ge2$, then $y_{ij}'$ are a linear combination of terms from the elements from (b) and (e) above, only using indices at least $2$.

We will again proceed in steps, for clarity.
\begin{enumerate}
	\item To apply the inductive hypothesis, we need to force $y_{pq}\in\ZZ[G']$ for each pair of indices $(p,q)$ with $p>q\ge2$. Well, we use the relation (e) so that we can subtract multiples of
	\[T_q\lambda_{p1}-T_1\lambda_{pq}-T_p\lambda_{q1}.\]
	In particular, this element will subtract out $T_1$ from $y_{pq}$ while only introducing chaos to the elements $y_{p1}$ and $y_{q1}$ in the process. Thus, subtracting a suitable multiple allows us to assume that $y_{pq}$ has no $\sigma_1$ terms while not affecting any other $y_{ij}$ with $i>j\ge2$.

	Applying this process to all $y_{ij}$ with $i>j\ge2$, we do indeed get $y_{ij}\in\ZZ[G']$ for each $i>j\ge2$.

	\item We are now ready to apply the inductive hypothesis. For each index $i\ge2$, we have the equation
	\[-y_{i1}T_1-\sum_{j=2}^{i-1}y_{ij}T_j+\sum_{j=i+1}^my_{ji}T_j=0.\]
	Because each $y_{pq}$ term with $p>q\ge2$ features no $\sigma_1$, applying the transformation $\sigma_1\mapsto1$ will affect no term in the sums while causing $y_{i1}T_1$ to vanish. Thus, we have the equations
	\[-\sum_{j=2}^{i-1}y_{ij}T_j+\sum_{j=i+1}^my_{ji}T_j=0\]
	for each index $i\ge2$. Because $y_{ij}\in\ZZ[G']$ for $i>j\ge2$ already, we see that we may apply the inductive hypothesis to assert that the $y_{ij}$ are $\ZZ[G']$-linear combinations of terms from (b) and (e) (only using indices at least $2$).
	
	Subtracting these linear combinations out, we may assume $y_{ij}=0$ for each $i>j\ge2$.

	\item To take stock, our equations for $i\ge2$ now read
	\[-y_{i1}T_1=0,\]
	which simply tells us that $y_{i1}\in\im N_1$ for each $i\ge2$. As such, we pick up $w_i\in\ZZ[G]$ so that $y_{i1}=w_iN_1$ for each $i\ge2$; because $\sigma_1N_1=N_1$, we may assume that $w_i\in\ZZ[G']$ for each $i\ge2$.

	Now the equation for $i=1$ reads
	\[\sum_{j=2}^my_{j1}T_j=0,\]
	or
	\[\sum_{i=2}^mw_iN_1T_i=0.\]
	Sending $\sigma_1\mapsto1$, we see that $w_i$ and $T_i$ are both fixed because they feature no $\sigma_1$s, so we merely have
	\[n_1\sum_{i=2}^mw_iT_i=0.\]
	Dividing out by $n_1$, we are left with
	\[\sum_{i=2}^mw_iT_i=0.\]

	\item At this point, we may appear stuck, but we have one final trick: taking indices $p>q\ge2$, subtracting out multiples of
	\[\big(T_q\lambda_{p1}-T_1\lambda_{pq}-T_p\lambda_{q1}\big)\cdot N_1\]
	will not affect the $y_{pq}$ term because $T_1N_1$. Indeed, subtracting this term out looks like
	\[T_qN_1\lambda_{p1}-T_pN_1\lambda_{q1},\]
	which after factoring out $N_1$ takes $w_p\mapsto w_p-T_q$ and $w_q\mapsto w_q+T_p$.

	In particular, fixing any $q\ge2$ and then applying this trick for all $p>q$, we may assume that $w_q$ does not feature any $\sigma_p$ terms for $p>q$. Thus, looking at our equation
	\[\sum_{i=2}^mw_iT_i=0,\]
	we are now able to show that $w_i\in\ker T_i=\im N_i$ for each $i\ge2$, which will finish because it shows $y_{i1}\in N_iN_1$. Indeed, starting with $i=2$, we see that $w_2$ features no $\sigma_p$ for $p>2$, so we may take $\sigma_p\mapsto1$ for each $p>2$ safely, giving the equation
	\[w_2T_2=0,\]
	finishing for $w_2$. Thus, we are left with the equation
	\[\sum_{i=3}^mw_iT_i=0,\]
	from which we see we can induct downwards (this has fewer variables) to finish.
\end{enumerate}
The above steps complete the proof, as advertised.

\end{document}