In this section, we work through \autoref{cor:fundtriple} very explicitly in a basic case. Let $p$ be an odd prime because the following discussion has no content in the case of $p=2$. Set $K\coloneqq\QQ_p$ and $K_m\coloneqq\QQ_p(\zeta_m)$ with $f\coloneqq[\QQ_p(\zeta_m):\QQ_p]$.

The main simplification we will make which allows explicit computation is that we will set $K_{\pi,\nu}\coloneqq\QQ_p(\zeta_p)$. Continuing with the set-up, we see $L=\QQ_p(\zeta_p,\zeta_m)$ with $n\coloneqq (p-1)\cdot f$; as such, set $N'\coloneqq p^n-1$ so that $M=\QQ_p(\zeta_{N'})$. Here is the diagram of our fields.
% https://q.uiver.app/?q=WzAsNixbMSwzLCJcXFFRX3AiXSxbMCwyLCJcXFFRX3AoXFx6ZXRhX3ApIl0sWzIsMiwiXFxRUV9wKFxcemV0YV9tKSJdLFsxLDEsIlxcUVFfcChcXHpldGFfcCxcXHpldGFfbSkiXSxbMywxLCJcXFFRX3AoXFx6ZXRhX3tOJ30pIl0sWzIsMCwiXFxRUV9wKFxcemV0YV9wLFxcemV0YV97Tid9KSJdLFswLDEsIiIsMCx7InN0eWxlIjp7ImhlYWQiOnsibmFtZSI6Im5vbmUifX19XSxbMCwyLCIiLDIseyJzdHlsZSI6eyJoZWFkIjp7Im5hbWUiOiJub25lIn19fV0sWzEsMywiIiwwLHsic3R5bGUiOnsiaGVhZCI6eyJuYW1lIjoibm9uZSJ9fX1dLFsyLDMsIiIsMix7InN0eWxlIjp7ImhlYWQiOnsibmFtZSI6Im5vbmUifX19XSxbMiw0LCIiLDIseyJzdHlsZSI6eyJoZWFkIjp7Im5hbWUiOiJub25lIn19fV0sWzQsNSwiIiwyLHsic3R5bGUiOnsiaGVhZCI6eyJuYW1lIjoibm9uZSJ9fX1dLFszLDUsIiIsMix7InN0eWxlIjp7ImhlYWQiOnsibmFtZSI6Im5vbmUifX19XV0=&macro_url=https%3A%2F%2Fraw.githubusercontent.com%2FdFoiler%2Fnotes%2Fmaster%2Fnir.tex
\[\begin{tikzcd}
	&& {\QQ_p(\zeta_p,\zeta_{N'})} \\
	& {\QQ_p(\zeta_p,\zeta_m)} && {\QQ_p(\zeta_{N'})} \\
	{\QQ_p(\zeta_p)} && {\QQ_p(\zeta_m)} \\
	& {\QQ_p}
	\arrow[no head, from=4-2, to=3-1]
	\arrow[no head, from=4-2, to=3-3]
	\arrow[no head, from=3-1, to=2-2]
	\arrow[no head, from=3-3, to=2-2]
	\arrow[no head, from=3-3, to=2-4]
	\arrow[no head, from=2-4, to=1-3]
	\arrow[no head, from=2-2, to=1-3]
\end{tikzcd}\]
So that we are able to isolate our set-up, we note that
\[\op{Gal}(\QQ(\zeta_p)/\QQ)\simeq(\ZZ/p\ZZ)^\times\]
is cyclic, so we choose some $x\in(\ZZ/p\ZZ)^\times$ to generate, which corresponds to the automorphism $\sigma_x\colon\zeta_p\mapsto\zeta_p^x$. Namely, we may set $\tau_1\coloneqq\sigma_x$.

Now, the reason we set $K_{\pi,\nu}=\QQ_p(\zeta_p)$ is that we will be able to set
\[\gamma\coloneqq(-p)^{1/(p-1)}\in\QQ_p(\zeta_p).\]
Indeed, we sneakily set $\pi=-p$ to be our uniformizer of $\QQ_p$ so that $\op N_{ML/L}(\gamma)=\gamma^{p-1}=-p$. Because it will be helpful for us shortly, we will actually give a construction of $(-p)^{1/(p-1)}$, for completeness.
\begin{lemma} \label{lem:findrootofp}
	Let $p$ be a prime. Then we can find some $\gamma\coloneqq(-p)^{1/(p-1)}$ in $\QQ_p(\zeta_p)$. In fact, we can take $\gamma\equiv c\varpi\pmod{\varpi^2}$ for any $c\in\FF_p^\times$, where $\varpi\coloneqq\zeta_p-1$ is a uniformizer.
\end{lemma}
\begin{proof}
	That a root $\gamma$ exists is well-known. The factorization
	\[x^{p-1}-1\equiv\prod_{c\in\FF_p^\times}(x-c)\pmod p\]
	lifts to a factorization in $\ZZ_p$ by \cite[Lemma~II.4.6]{neukirch-alg-nt}. As such, as soon as we have one root $\gamma$ of $x^{p-1}+p$, observe that $|\gamma|=p^{1/(p-1)}=|\varpi|$, so $\gamma$ is a uniformizer as well, meaning that the $c$ in
	\[\zeta_{p-1}\gamma\equiv c\varpi\pmod{\varpi^2}\]
	is nonzero and will vary across all representatives in $\FF_p^\times$ as we exchange the root $\gamma$ with $\zeta_{p-1}\gamma$ for various $\zeta_{p-1}$.
	% We follow Professor Andrew Sutherland's \href{https://math.mit.edu/classes/18.785/2021fa/LectureNotes20.pdf\#theorem.2.5}{Lemma~20.5}. Set $\pi\coloneqq\zeta_p-1$ to be a uniformizer of $\QQ_p(\zeta_p)$. Now, the minimal polynomial of $\zeta_p$ is
	% \[f(T)\coloneqq\frac{(T+1)^p-1}T,\]
	% which is $p$-Eisenstein. To properly apply Hensel's lemma to solve $T^{p-1}+p$, we see that any solution should be divisible by $\pi$, so we divide out by this first. Note $v(\pi)=1/(p-1)$, so $u\coloneqq-\pi^{p-1}/p\in\mathcal O_{\QQ_p(\zeta_p)}^\times$. In fact, we can see from the polynomial $f$ that
	% \[\pi^{p-1}+p\equiv0\pmod{p\pi},\]
	% so $u\equiv-1\pmod\pi$. As such, we now note that $g(T)\coloneqq T^{p-1}-u$ has
	% \[g(c)\equiv0\pmod\pi\qquad\text{and}\qquad g'(c)=(p-1)c\not\equiv0\pmod\pi,\]
	% for any $c\in\FF_p^\times$, so we can lift $c$ to a root $\beta_c\in\mathcal O_{\QQ_p(\zeta_p)}$. From here, we see $(\pi/\beta_c)^{p-1}=\pi^{p-1}/u=-p$, so $\pi/\beta_c$ is our desired root. For the last statement, we see
	% \[\pi/\beta_c\equiv c^{-1}\pi\pmod{\pi^2},\]
	% so as $c\in\FF_p^\times$ varies, we do indeed get all equivalence classes.
\end{proof}
In light of \autoref{lem:findrootofp}, we will just take $\gamma$ to have $\gamma^{p-1}=-p$ with $\gamma\equiv c\pi\pmod{\pi^2}$ for any particular $c\in\FF_p^\times$. This satisfies $\op N_{ML/L}(\gamma)=-p$ as discussed above.

We will now compute the tuple. We start with the unramified side because it is easier. Namely, $\gamma\in\QQ_p(\zeta_p)$ is fixed by the Frobenius automorphism $\sigma_K$, so we may set $\eta_K\coloneqq1$ to have
\[\frac{\eta_K}{\sigma_K^f(\eta_K)}=1=\frac{\sigma_K(\gamma)}{\gamma}.\]
The corresponding $\alpha_0$ is thus
\[\boxed{\alpha_0=\gamma}.\]
We now deal with ramification.
% Observe $\op{Gal}(\QQ_p(\zeta_p)/\QQ_p)\simeq(\ZZ/p\ZZ)^\times$ is cyclic, but we must choose a generator nonetheless. Let $x\in(\ZZ/p\ZZ)^\times$ be a generator, and let $\sigma_x\colon\zeta_p\mapsto\zeta_p^x$ be the corresponding automorphism; namely, $\tau_1\coloneqq\sigma_x$. (Notably, this is not the automorphism generated by the Artin map; we will return to this point later.)
We begin with a computational lemma, tying in what we have with Teichm\"uller lifts.
\begin{lemma}
	Fix everything as above. Then $\zeta_{p-1}\coloneqq\sigma_x(\gamma)/\gamma$ is a primitive $(p-1)$st root of unity and in particular lies in $\QQ_p$. In fact, $\zeta_{p-1}\equiv x\pmod p$.
\end{lemma}
Note that we are defining $\zeta_{p-1}$ above, which is okay: in the worst case, we might have to adjust the definitions of $\zeta_{N'}$ and $\zeta_m$ to correspond with this particular $\zeta_{p-1}$, but otherwise $\zeta_{p-1}$ may be any fixed primitive $(p-1)$st root of unity.
\begin{proof}
	To see that $\zeta_{p-1}$ is a $(p-1)$st root of unity, we note that $\sigma_x(\gamma)=\zeta_{p-1}\cdot\gamma$, so an induction shows that
	\[\sigma_x^k(\gamma)=\zeta_{p-1}^k\cdot\gamma.\]
	Setting $k=p-1$ shows that $\zeta_{p-1}^{p-1}=1$, so $\zeta_{p-1}$ is a $(p-1)$st root of unity.
	% To show that $\zeta_{p-1}$ is primitive, we know that $\zeta_{p-1}^k=1$ above would imply that $\sigma_x^k(\gamma)=\gamma$, but $\QQ_p(\gamma)=\QQ_p(\zeta_p)$ (we already know $\QQ_p(\gamma)\subseteq\QQ_p(\zeta_p)$, but both of these extensions have degree $p-1$), so in fact $\sigma_x^k=\id$. So $x\in(\ZZ/p\ZZ)^\times$ being a generator requires $p-1\mid k$. So indeed, the least positive integer $k$ with $\zeta_p^k=1$ is $k=p-1$.

	% We now quickly note that $\QQ_p$ contains all $(p-1)$st roots of unity by Hensel's lemma because the polynomial $T^{p-1}-1\in\FF_p[T]$ fully splits into $p-1$ distinct factors; in particular, $\zeta_{p-1}\in\QQ_p$. In fact, Hensel's lemma tells us that the $p-1$st roots of unity of $\QQ_p$ fully represent $(\ZZ/p\ZZ)^\times$, so there is a chance for $\zeta_{p-1}\equiv x\pmod p$.

	We next show $\zeta_{p-1}\equiv x\pmod p$; this will automatically imply that $\zeta_{p-1}$ is primitive because it will force $\zeta_{p-1}$ to have at least the order of $x\pmod p$, which is $p-1$. Let $\varpi\coloneqq\zeta_p-1$ be a uniformizer of $\QQ_p(\zeta_p)$. Because $\zeta_{p-1},x\in\QQ_p$, it is enough for $v_{\QQ_p}(\zeta_{p-1}-x)>0$; as such, we will show that
	\[\zeta_{p-1}\stackrel?\equiv x\pmod\varpi.\]
	To see this, recall $\gamma\equiv c\varpi\pmod{\varpi^2}$, so
	\[\zeta_{p-1}=\frac{\sigma_x(\gamma)}{\gamma}\equiv\frac{c\cdot\sigma_x(\varpi)}{c\cdot\varpi}\equiv\frac{\sigma_x(\varpi)}{\varpi}\pmod\varpi.\]
	However, $\sigma_x(\varpi)=\zeta_p^x-1$, so
	\[\frac{\sigma_x(\varpi)}\varpi=\frac{\zeta_p^x-1}{\zeta_p-1}\equiv1+\zeta_p+\cdots+\zeta_p^{x-1}\equiv\underbrace{1+\cdots+1}_x\equiv x\pmod\varpi,\]
	finishing.
\end{proof}
We are almost able to compute $\eta_x\coloneqq\eta_1$. To do this, we pick up a quick lemma.
\begin{lemma}
	Let $p$ and $f$ be integers. Then
	\[\frac{p^{f(p-1)}-1}{(p-1)\left(p^f-1\right)}\in\ZZ.\]
\end{lemma}
\begin{proof}
	Observe
	\[\frac{p^{f(p-1)}-1}{p^f-1} = \sum_{k=0}^{p-1}p^{fk} \equiv \sum_{k=0}^{p-1}1=p-1\equiv0\pmod{p-1}.\]
	This finishes.
\end{proof}
In light of the above lemma, we define
\[z\coloneqq-\frac{p^{f(p-1)}-1}{(p-1)\left(p^f-1\right)}.\]
Note the sign here: it is very important! It follows that $\eta_x\coloneqq\zeta_{N'}^z$ will have
\begin{align*}
	\frac{\eta_x}{\sigma_K^f(\eta_x)} &= \frac{\zeta_{N'}^z}{\zeta_{N'}^{zp^f}} = \zeta_{N'}^{-z\left(p^f-1\right)} = \zeta_{N'}^{N'/(p-1)} = \zeta_{p-1},
\end{align*}
which is indeed $\sigma_x(\gamma)/\gamma$. Thus, the corresponding $\alpha_1$ is
\begin{align*}
	\alpha_1 &= \prod_{i=0}^{p-1}\sigma_x^i(\eta_i) \\
	&= \eta_i^{p-1} \\
	&= \zeta^{z(p-1)}_{N'} \\
	&= \zeta_{N'}^{-N'/\left(p^f-1\right)} \\
	\Aboxed{\alpha_1 &= \zeta_{p^f-1}^{-1}}.
\end{align*}
Lastly, we compute our $\beta_{10}$ as
\begin{align*}
	\beta_{10} &= \frac{\eta_K}{\sigma_x\eta_K}\cdot\frac{\sigma_K\eta_x}{\eta_x} \\
	&= \zeta_{N'}^{z(p-1)} \\
	\Aboxed{\beta_{10} &= \zeta_{p^f-1}^{-1}}.
\end{align*}
In total, we get the following nice result.
\begin{theorem}
	Let $p$ be an odd prime, and fix $K\coloneqq\QQ_p$ and $L\coloneqq\QQ_p(\zeta_p,\zeta_m)$, where $p\nmid m$. Further, set $L_0\coloneqq\QQ_p(\zeta_p)$ and $L_1\coloneqq\QQ_p(\zeta_m)$ so that $L=L_0L_1$ and $L_0\cap L_1=K$. Now, pick up the following data.
	\begin{itemize}
		\item Suppose the order of $p$ modulo $m$ is $f$.
		\item Let $\sigma_x\colon\zeta_p\mapsto\zeta_p^x$ be a generator of $\op{Gal}(\QQ_p(\zeta_p)/\QQ_p)$.
		\item Find $\gamma\in\QQ_p(\zeta_p)$ such that $\gamma^{p-1}+p=0$ and $\sigma_x(\gamma)/\gamma=\zeta_{p-1}$. (Equivalently, set $\zeta_{p-1}\coloneqq\sigma_x(\gamma)/\gamma$.)
	\end{itemize}
	Then the fundamental class $u_{L/K}\in H^2(\op{Gal}(L/K),L^\times)$ is represented by the triple
	\[(\alpha_0,\alpha_1,\beta_{10})=\left(\gamma,\zeta_{p^f-1}^{-1},\zeta_{p^f-1}^{-1}\right).\]
\end{theorem}
\begin{remark}
	We verify Artin reciprocity for $\QQ_p(\zeta_p)/\QQ_p$. Let $c\in Z^2(\op{Gal}(L/K),L^\times)$ represent the fundamental class. The explicit formula for $\alpha_1$ tells us that
	\[\alpha_1=\prod_{i=0}^{p-1}c\left(\sigma_x^i,\sigma_x\right)=[\sigma_x]\cup\op{Res}u_{L/\QQ_p}=[\sigma_x]\cup u_{L/\QQ_p(\zeta_m)}=\theta^{-1}_{L/\QQ_p(\zeta_m)}(\sigma_x).\]
	Taking norms down to $K^\times$, we see on one hand that
	\[\op N_{\QQ_p(\zeta_m)/\QQ_p}(\alpha_1)=\prod_{i=0}^{f-1}\zeta_{p^f-1}^{-p^i}=\zeta_{p^f-1}^{-\left(1+p+\cdots+p^{f-1}\right)}=\zeta_{p^f-1}^{-\left(p^f-1\right)/(p-1)}=\zeta_{p-1}^{-1}\equiv x^{-1}\pmod p.\]
	On the other hand,
	\[\op N_{\QQ_p(\zeta_m)/\QQ_p}\theta^{-1}_{L/\QQ_p(\zeta_m)}(\sigma_x)=\theta^{-1}_{L/\QQ_p}(\sigma_x)=\theta^{-1}_{\QQ_p(\zeta_p)/\QQ_p}(\sigma_x).\]
	So $\theta^{-1}_{\QQ_p(\zeta_p)/\QQ_p}$ sends $\sigma_x\colon\zeta_p\mapsto\zeta_p^x$ to $x^{-1}\pmod p$, as predicted by Lubin--Tate theory.
\end{remark}