% !TEX root = ../abeliangerbs.tex

The story so far has been able to generalize the one-variable results from \autoref{sec:general} to results using all generators of an abelian group in \autoref{sec:abelian}. It remains to prove \autoref{thm:getcocycle}, which is the main goal of this section.

\subsection{Set-Up and Overview} \label{sec:overview}
The approach here will be to attempt to abstract our data away from the $ G$-module $A$ as much as possible. To set up our discussion, we continue with
\[G\simeq\bigoplus_{i=1}^mG_i,\]
where $G_i=\langle\sigma_i\rangle\subseteq G$ and $\sigma_i$ has order $n_k$. These variables allow us to define
\[T_i\coloneqq(\sigma_i-1)\qquad\text{and}\qquad N_i\coloneqq\sum_{p=0}^{n_i-1}\sigma_i^p\]
for each index $i$. In fact, it will be helpful to also have notation
\[\sigma^{(a)}\coloneqq\sum_{p=0}^{a-1}\sigma^p\]
for any $\sigma\in G$ and nonnegative integer $a\ge0$; in particular, $\sigma^{(0)}=0$ and $\sigma_i^{(n_i)}=N_i$. The main benefits to this notation will be the facts that
\[\sigma^{(a+b)}=\sigma^{(a)}+\sigma^a\sigma^{(b)}\qquad\text{and}\qquad\sigma_i^a=T_i\sigma_i^{(a)}+1,\]
which can be seen by direct expansion. Given $g\in\prod_{p=1}^n\sigma_p^{a_p}$, we will also define the notation
\[g_i\coloneqq\prod_{p=1}^{i-1}\sigma_p^{a_p}\]
for $i\ge0$. In particular $g_0=g_1=1$ and $g_{n+1}=g$.

Now, our tool in the proof of \autoref{thm:getcocycle} will be the magical map $\mathcal F\colon\ZZ[G]^m\times\ZZ[G]^{\binom m2}\to\ZZ[G]^m$ defined by
\[\mathcal F\colon\big((x_i)_{i=1}^m,(y_{ij})_{i>j}\big)\mapsto\Bigg(x_iN_i-\sum_{j=1}^{i-1}y_{ij}T_j+\sum_{j=i+1}^my_{ji}T_j\Bigg)_{i=1}^m.\]
This is of course a $G$-module homomorphism. We will go ahead and state the main results we will prove. Roughly speaking, $\mathcal F$ is manufactured to make the following result true.
\begin{prop} \label{prop:manufacturedcocycle}
	Fix everything as in the set-up. Then the function
	\[\overline c(g)\coloneqq\left(g_i\sigma_i^{(a_i)}\right)_{i=1}^m,\]
	where $g\coloneqq\prod_{i=1}^m\sigma_i^{a_i}$, is a $1$-cocycle in $Z^1(G,\coker\mathcal F)$.
\end{prop}
The reason we care about this cocycle is that we can pass it through a boundary morphism induced by the short exact sequence
\[0\to\underbrace{\frac{\ZZ[G]^m\times\ZZ[G]^{\binom m2}}{\ker\mathcal F}}_{X\coloneqq}\stackrel{\mathcal F}\to\ZZ[G]^m\to\coker\mathcal F\to0,\]
so we have a $2$-cocycle $\delta(\overline c)\in Z^2(G,X)$; in fact, we will be able to explicitly compute $\delta(\overline c)$ as a result of the proof of \autoref{prop:manufacturedcocycle}.

Only now will we bring in tuples. The first result provides an alternate description of tuples.
\begin{restatable}{prop}{propalternativetuple} \label{prop:alternativetuple}
	Fix everything as in the set-up, and now let $A$ be a $G$-module. Then $\{\sigma_i\}_{i=1}^m$-tuples are canonically isomorphic to $\op{Hom}_{\ZZ[G]}(X,A)=H^0(G,\op{Hom}_\ZZ(X,A))$.
\end{restatable}
\noindent The second result brings in the last ingredient, the cup product.
\begin{restatable}{theorem}{thmyesitisacocycle} \label{thm:yesitisacocycle}
	Fix everything as in the set-up. Further, fix a $G$-module $A$ and a $\{\sigma_i\}_{i=1}^m$-tuple $\left(\{\alpha_i\},\{\beta_{ij}\}\right)$. Then observe there is a natural cup product map
	\[\cup\colon H^2(G,X)\times H^0(G,\op{Hom}_\ZZ(X,A))\to H^2(G,A)\]
	by using the evaluation map $X\otimes_\ZZ\op{Hom}_\ZZ(X,A)\to A$. Then, using the isomorphism of \autoref{prop:alternativetuple}, the cocycle defined in \autoref{thm:getcocycle} is simply the output of $\delta(\overline c)\cup\left(\{\alpha_i\},\{\beta_{ij}\}\right)$ on cocycles.
\end{restatable}
\noindent Because we know that the cup product sends cocycles to cocycles, this will show that the cocycle of \autoref{thm:getcocycle} is in fact well-defined.

% it might be worth stating the main results we are going to prove here, but they are somewhat notation-heavy

\subsection{Preliminary Work}
We continue in the set-up of the previous subsection.
% The goal of this subsection is to prove \autoref{prop:manufacturedcocycle}. In fact, we will show the following stronger result.
% \begin{proposition} \label{prop:allmanufacturedcocycles}
% 	Fix everything as in the set-up. Then $H^1(G,\coker\mathcal F)$ is cyclic generated by the class $[\overline c]$ represented by $\overline c$, where
% 	\[\overline c(g)\coloneqq\left(g_i\sigma_i^{(a_i)}\right)_{i=1}^m,\]
% 	with $g\coloneqq\prod_{i=1}^m\sigma_i^{a_i}$
% \end{proposition}
Before jumping into any hard logic, we define some (more) notation which will be useful later on as well. First, in $\ZZ[G]^m\times\ZZ[G]^{\binom m2}$, we define
\[\kappa_p\coloneqq\big((1_{i=p})_i,(0)_{i>j}\big)\in X\qquad\text{and}\qquad\lambda_{pq}\coloneqq\big((0)_i,(1_{(i,j)=(p,q)})_{i>j}\big)\]
for all relevant indices $p$ and $q$ so that the $\kappa_p$ and $\lambda_{pq}$ are a basis for $\ZZ[G]^m\times\ZZ[G]^{\binom m2}$ as a $\ZZ[G]$-module. Secondly, we define
\[\varepsilon_p\coloneqq(1_{i=p})_{i=1}^m\]
for all indices $p$, again giving a basis for $\ZZ[G]^m$ as a $\ZZ[G]$-module. For example, this notation lets us write
\begin{equation}
	\mathcal F\left(\sum_{i=1}^mx_i\kappa_i+\sum_{i>j}y_{ij}\lambda_{ij}\right)=\sum_{i=1}^mx_iN_i\varepsilon_i+\sum_{i>j}y_{ij}(T_i\varepsilon_j-T_j\varepsilon_i), \label{eq:betterf}
\end{equation}
and
\[\overline c(g)=\sum_{i=1}^mg_i\sigma_i^{(a_i)}\varepsilon_i\]
where $g\coloneqq\prod_{i=1}^m\sigma_i^{a_i}$.

Additionally, so that we do not need to interrupt our discussion later, we establish a few lemmas which will aide our proof of \autoref{prop:manufacturedcocycle}.
\begin{lemma} \label{lem:separatenijs}
	Fix everything as in the set-up. Then, for any set of distinct indices $(i_1,\ldots,i_k)$, we have
	\[\bigcap_{p=1}^k\im N_{i_p}=\im\prod_{p=1}^kN_{i_p},\]
	where we are identifying $x\in\ZZ[G]$ with its associated multiplication map $x\colon\ZZ[G]\to\ZZ[G]$.
\end{lemma}
\begin{proof}
	The point is that the elements of $\bigcap_{p=1}^k\im N_{i_p}$ and $\im\prod_{p=1}^kN_{i_p}$ are both simply the elements whose expansion in the form $\sum_gc_gg\in\ZZ[G]$ have $c_j$ ``constant in $\sigma_p$ and $\sigma_q$.'' More explicitly, of course, $\prod_{p=1}^kN_{i_p}\in\bigcap_{p=1}^k\im N_{i_p}$, so
	\[\im\prod_{p=1}^kN_{i_p}\subseteq\bigcap_{p=1}^k\im N_{i_p}.\]
	In the other direction, suppose that we have some element
	\[z\coloneqq\sum_{(a_i)_i}c_{(a_i)_i}\sigma_1^{a_1}\cdots\sigma_m^{a_m}\in\bigcap_{p=1}^k\im N_{i_p},\]
	the sum is over sequences $(a_i)_{i=1}^m$ such that $0\le a_i<n_i$ for each index $i$. We will show $z\in\im\prod_{p=1}^kN_{i_p}$.
	
	Now, $z\in\im N_r$ for $r\in\{p,q\}$ is equivalent to $z\in\ker T_r$, but upon multiplying by $(\sigma_r-1)$ we see that we are asking for
	\[\sum_{(a_i)_i}c_{(a_i)_i}\sigma_1^{a_1}\cdots\sigma_{r-1}^{a_{r-1}}\sigma_r^{a_r}\sigma_{r+1}^{a_{r+1}}\cdots\sigma_n^{a_n}=\sum_{(a_i)_i}c_{(a_i)_i}\sigma_1^{a_1}\cdots\sigma_{r-1}^{a_{r-1}}\sigma_r^{a_r+1}\sigma_{r+1}^{a_{r+1}}\cdots\sigma_n^{a_n}.\]
	In other words, this is asking for $c_{(a_i)_i}=c_{(a_i)_i+(1_{i=r})_i}$, or more succinctly just that $c$ is constant in the $i=r$ coordinate.

	Thus, $c$ is constant in all the $i=i_p$ coordinates for each index $i_p$. Thus, we let $d_{(a_i)_{i\notin\{i_p\}}}$ be the restricted function equal to $c_{(a_i)_i}$ but forgetting the information input from any of the $a_{i_p}$. This allows us to write
	\begin{align*}
		z &= \sum_{(a_i)_i}c_{(a_i)_i}\sigma_1^{a_1}\cdots\sigma_m^{a_m} \\
		&= \sum_{(a_i)_{i\notin\{i_p\}}}\sum_{a_{i_1}=0}^{n_{i_1}-1}\cdots\sum_{a_{i_k}=0}^{n_{i_k}-1}d_{(a_i)_{i\notin\{i_p\}}}\sigma_1^{a_1}\cdots\sigma_m^{a_m} \\
		&= \Bigg(\sum_{(a_i)_{i\notin\{i_p\}}}d_{(a_i)_{i\notin\{i_p\}}}\prod_{\substack{i=0\\i\notin\{i_p\}}}^m\sigma_i^{a_i}\Bigg)\Bigg(\sum_{a_{i_1}=0}^{n_{i_1}-1}\sigma_{i_1}^{a_{i_1}}\Bigg)\cdots\Bigg(\sum_{a_{i_k}=0}^{n_{i_k}-1}\sigma_{i_k}^{a_{i_k}}\Bigg),
	\end{align*}
	which is now manifestly in $\im\prod_{p=1}^kN_{i_p}$.
\end{proof}
\begin{lemma} \label{lem:expandgi}
	Fix everything as in the set-up. Then, given $g\coloneqq\prod_{i=1}^m\sigma_i^{a_i}$, we have
	\[g_i=1+\sum_{p=1}^{i-1}g_p\sigma_p^{(a_p)}T_p\]
	for $i\ge1$.
\end{lemma}
\begin{proof}
	This is by induction. For $i=1$, there is nothing to say. For the inductive step, we take $i>1$ where we may assume the statement for $i-1$. Via some relabeling, we may make our inductive hypothesis assert
	\[\prod_{p=2}^{i-1}\sigma_p^{a_p}=1+\sum_{p=2}^{i-1}\Bigg(\prod_{q=2}^{p-1}\sigma_q^{a_q}\Bigg)\sigma_p^{(a_p)}T_p.\]
	In particular, multiplying through by $\sigma_1^{a_1}$ yields
	\begin{align*}
		g_i &= \sigma_1^{a_1}\cdot\prod_{p=2}^{i-1}\sigma_p^{a_p} \\
		&= \sigma_1^{a_1}+\sigma_1^{a_1}\sum_{p=2}^{i-1}\Bigg(\prod_{q=2}^{p-1}\sigma_q^{a_q}\Bigg)\sigma_p^{(a_p)}T_p \\
		&= \sigma_1^{a_1}+\sum_{p=2}^{i-1}g_p\sigma_p^{(a_p)}T_p \\
		&= 1+\sigma_1^{(a_1)}T_1+\sum_{p=2}^{i-1}g_p\sigma_p^{(a_p)}T_p,
	\end{align*}
	which is exactly what we wanted, after a little more rearrangement.
\end{proof}
And mostly because we can, we show that our main short exact sequence splits.
\begin{lemma} \label{lem:sessplits}
	Fix everything as in the set-up. Then consider $\ZZ$-module map $\rho\colon\ZZ[G]^m\to\ZZ[G]^m$ defined by
	\[\rho(g\varepsilon_i)\coloneqq g_i\big(\sigma_i^{a_i}-N_i1_{a_i=n_i-1}\big)\varepsilon_i+\sum_{j=i+1}^mg_j\sigma_j^{(a_j)}T_i\varepsilon_j,\]
	where $g\coloneqq\prod_{i=1}^m\sigma_i^{a_i}$ with $0\le a_i<n_i$. Then $\rho$ descends to a map $\overline\rho\colon\coker\mathcal F\to\ZZ[G]^m$ witnessing the splitting of the short exact sequence
	\[0\to X\to\ZZ[G]^m\to\coker\mathcal F\to0\]
	over $\ZZ$.
\end{lemma}
\begin{proof}
	Observe that we have a well-defined map $\rho\colon\ZZ[G]^m\to\ZZ[G]^m$ because $\ZZ[G]^m$ is a free abelian group generated by $g\varepsilon_i$ for $g\in G$ and indices $i$. It remains to show that $\im\mathcal F\subseteq\ker\rho$ to get a map $\overline\rho\colon\coker\mathcal F\to\ZZ[G]^m$ and then to show that $\rho(z)\equiv z\pmod{\im\mathcal F}$ to get the splitting. We show these individually.

	To show that $\im\mathcal F\subseteq\ker\rho$, we note from \autoref{eq:betterf} that $\im\mathcal F$ is generated over $\ZZ[G]$ by the elements $N_i\varepsilon_i$ and $T_i\varepsilon_j-T_j\varepsilon_i$ for relevant indices $i$ and $j$. Thus, $\im\mathcal F$ is generated over $\ZZ$ by the elements $gN_i\varepsilon_i$ and $gT_i\varepsilon_j-gT_j\varepsilon_i$ for relevant indices $i$ and $j$. Thus, we fix any $g\coloneqq\prod_{i=1}^n\sigma_i^{a_i}$ and show that $gN_i\varepsilon_i\in\ker\rho$ and $gT_i\varepsilon_j-gT_j\varepsilon_i\in\ker\rho$ for relevant indices $i$ and $j$.
	\begin{itemize}
		\item We show $gN_i\varepsilon_i\in\ker\rho$ for any $i$. Because $gN_i=g\sigma_iN_i$, we may as well as assume that $a_i=0$. Then
		\[\rho\left(g\sigma_i^a\varepsilon_i\right)=g_i\big(\sigma_i^{a}-N_i1_{a=n_i-1}\big)\varepsilon_i+\sum_{j=i+1}^mg_j\sigma_i^a\sigma_j^{(a_j)}T_i\varepsilon_j.\]
		As $a$ varies from $0$ to $n_i-1$, we note that the term $g_i\big(\sigma_i^{a}-N_i1_{a=n_i-1}\big)\varepsilon_i$ will only get the $-N_i$ contribution exactly once at $a=n_i-1$. Summing, we thus see that
		\[\rho(gN_i\varepsilon_i)=g_i\Bigg(-N_i+\sum_{a=0}^{n_i-1}\sigma_i^{a}\Bigg)\varepsilon_i+\sum_{a=0}^{n_i-1}\sum_{j=i+1}^mg_j\sigma_i^a\sigma_j^{(a_j)}T_i\varepsilon_j.\]
		The left term vanishes because $N_i=\sum_{a=0}^{n_i-1}\sigma_i^a$. Additionally, the right term vanishes because we can factor $T_i\sum_{a=0}^{n_i-1}\sigma_i^a=T_iN_i=0$. So $gN_i\varepsilon_i\in\ker\rho$.
		\item We show $gT_p\varepsilon_q-gT_q\varepsilon_p\in\ker\rho$ for any $p>q$. Equivalently, we will show that $\rho(g\sigma_p\varepsilon_q)-\rho(g\varepsilon_q)=\rho(g\sigma_q\varepsilon_p)-\rho(g\varepsilon_p)$. On one hand, note
		\begin{align*}
			\rho(g\sigma_p\varepsilon_q) &= g_q\big(\sigma_q^{a_q}-N_i1_{a_q=n_q-1}\big)\varepsilon_q \\
			&\qquad\qquad+\sum_{j=q+1}^{p-1}g_j\sigma_j^{(a_j)}T_q\varepsilon_j \\
			&\qquad\qquad+g_p\left(\sigma_p^{(a_p+1)}-N_p1_{a_p=n_p-1}\right)T_q\varepsilon_p \\
			&\qquad\qquad+\sum_{j=p+1}^m\sigma_pg_j\sigma_j^{(a_j)}T_q\varepsilon_j
		\end{align*}
		because $g_j$ doesn't ``see'' the extra $\sigma_p$ term until $j>p$. (For the $j=p$ term, we would like to write $\sigma_p^{(a_p+1)}$ above, but when $a_p=n_p-1$, we actually end up with $\sigma_p^{(0)}=0$ and hence have to subtract out $\sigma_p^{(n_p)}=N_p$.) Thus,
		\[\rho(g\sigma_p\varepsilon_q)-\rho(g\varepsilon_q) = g_p\left(\sigma_p^{a_p}-N_p1_{a_p=n_p-1}\right)T_q\varepsilon_p+\sum_{j=p+1}^mg_j\sigma_j^{(a_j)}T_pT_q\varepsilon_j.\]
		On the other hand, we have
		\[\rho(g\sigma_q\varepsilon_p) = \sigma_qg_p\big(\sigma_p^{a_p}-N_p1_{a_p=n_p-1}\big)\varepsilon_p+\sum_{j=p+1}^m\sigma_qg_j\sigma_j^{(a_j)}T_p\varepsilon_j\]
		where this time all $j>p$ also have $j>q$ and so $(\sigma_qg)_j=\sigma_qg_j$. Thus,
		\[\rho(g\sigma_q\varepsilon_p)-\rho(g\varepsilon_p) = g_p\left(\sigma_p^{a_p}-N_p1_{a_p=n_p-1}\right)T_q\varepsilon_p+\sum_{j=p+1}^mg_j\sigma_j^{(a_j)}T_pT_q\varepsilon_j,\]
		as desired.
	\end{itemize}
	We now check the splitting. For this, we simply need to check that $\rho(g\varepsilon_i)\equiv g\varepsilon_i\pmod{\im\mathcal F}$, and we will get the result for all elements of $\ZZ[G]^m$ by additivity of $\rho$. Well, using \autoref{lem:expandgi}, we write
	\begin{align*}
		g\varepsilon_i &= g_i\sigma_i^{a_i}\Bigg(\prod_{j=i+1}^m\sigma_j^{a_j}\Bigg)\varepsilon_i \\
		&= g_i\sigma_i^{a_i}\Bigg(1+\sum_{j=i+1}^m\Bigg(\prod_{q=i+1}^{j-1}\sigma_q^{a_q}\Bigg)\sigma_j^{(a_j)}T_j\Bigg)\varepsilon_i \\
		&= g_i\sigma_i^{a_i}\varepsilon_i+\sum_{j=i+1}^mg_i\sigma_i^{a_i}\Bigg(\prod_{q=i+1}^{j-1}\sigma_q^{a_q}\Bigg)\sigma_j^{(a_j)}T_j\varepsilon_i \\
		&\equiv g_i\sigma_i^{a_i}\varepsilon_i+\sum_{j=i+1}^mg_j\sigma_j^{(a_j)}T_i\varepsilon_j,
	\end{align*}
	where in the last step we have used the fact that $T_j\varepsilon_i\equiv T_j\varepsilon_i\pmod{\im\mathcal F}$. Lastly, we note that $hN_i\varepsilon_i\equiv h\varepsilon_i\pmod{\im\mathcal F}$ for any $h\in G$, so in fact
	\[g\varepsilon_i\equiv g_i\left(\sigma_i^{a_i}-N_i1_{a_i=n_i-1}\right)\varepsilon_i+\sum_{j=i+1}^mg_j\sigma_j^{(a_j)}T_i\varepsilon_j,\]
	and now the right-hand side is $\rho(g\varepsilon_i)$.
\end{proof}
% \begin{remark}
% 	The purpose of \autoref{lem:sessplits} is to give an injective map from $\coker\mathcal F$ to a more controlled setting. In particular, it is somewhat annoying to check if an element $z\in\ZZ[G]^m$ lives in $\im\mathcal F$, but it is easier to check the equivalent condition $\overline\rho(z)=0$.
% \end{remark}
% We are now ready to more directly attack the proof of \autoref{prop:allmanufacturedcocycles}. We begin by reducing the amount of data we have to carry around in a cocycle.
% \begin{lemma} \label{lem:compresscocycle}
% 	Fix everything as in the set-up, and let $A$ be a $G$-module. Then, if $f\in Z^1(G,A)$ is a cocycle, then
% 	\[f(g)=\sum_{i=1}^mg_i\sigma_i^{(a_i)}f(\sigma_i),\]
% 	where $g\coloneqq\prod_{i=1}^m\sigma_i^{a_i}$ with $a_i\ge0$.
% \end{lemma}
% \begin{proof}
% 	Unsurprisingly, this is by induction. To begin, we claim that
% 	\[f\left(\sigma^a\right)=\sigma^{(a)}f(\sigma)\]
% 	by induction on $a$. When $a=0$, we are showing that $f(1)=0$, for which we note that the $1$-cocycle condition implies $f(1)=f(1)+f(1)$ and so $f(1)=0$. Then for the inductive step, we assume $f(\sigma^a)=\sigma^{(a)}f(\sigma)$ and note
% 	\[f\left(\sigma^{a+1}\right)=\sigma f\left(\sigma^a\right)+f(\sigma)=\left(1+\sigma\sigma^{(a)}\right)f(\sigma)=\sigma^{(a+1)}f(\sigma),\]
% 	finishing.

% 	We now show the original statement by an induction on $m$. For $m=0$, this is asserting $f(1)=0$, which is true. Then for the inductive step, we assume for $m-1$ and note that $m>1$ has
% 	\[f\left(g_m\sigma_m^{a_m}\right)=f(g_m)+g_mf\left(\sigma_m^{a_m}\right)=\sum_{i=1}^{m-1}g_i\sigma_i^{(a_i)}f(\sigma_i)+g_m\sigma_m^{(a_m)}f(\sigma_m),\]
% 	which is what we wanted.
% \end{proof}
% Thus, to build a $1$-cocycle, we only have to specify $f(\sigma_i)$ for indices $i$ and then check the $1$-cocycle condition to make sure we are okay.

% As such, we now run through what the $1$-cocycle check requires.
% \begin{lemma} \label{lem:cocycleforcecoord}
% 	Fix everything as in the set-up. Further, fix some $z\in\ZZ[G]^m$. Then $N_iz\in\im\mathcal F$ if and only if $[z]\in\coker\mathcal F$ has a representative of the form $a_i\varepsilon_i\in\ZZ[G]^m$ where $a_i\in\ZZ[G]$.
% \end{lemma}
% \begin{proof}
% 	In one direction, if $z\equiv a_i\varepsilon_i\pmod{\im\mathcal F}$, then
% 	\[N_iz\equiv a_i\cdot N_i\varepsilon_i\equiv a_i\cdot0\equiv0\pmod{\im\mathcal F}\]
% 	because $N_i\varepsilon_i\in\im\mathcal F$.

% 	In the other direction, we pass through $\overline\rho$ of \autoref{lem:sessplits}. By possibly rearranging our $\sigma$s, we may set $i=1$. As such, suppose $N_1z\in\im\mathcal F$, and write
% 	\[z\coloneqq\sum_{i=1}^mz_i\varepsilon_i\]
% 	where $z_i\in\ZZ[G]$. By using the fact that $T_i\varepsilon_1\equiv T_1\varepsilon_i\pmod{\im\mathcal F}$ for any index $i$, we can find a representative for $z$ in $\ZZ[G]^m$ such that $z_i$ has no $\sigma_1$ powers for each $i>1$; without loss of generality, replace $z$ with this representative.
	
% 	We thus claim that $w\coloneqq z-z_1\varepsilon_1\in\im\mathcal F$, which means that $z$ is represented by $z_1\varepsilon_1$; to show this, we already know that $N_1w=N_1(z-z_1\varepsilon_1)\in\im\mathcal F$, so we pass through $\overline\rho$. In other words, it suffices to show that $\rho(w)=0$ from $\rho(N_1w)=0$ and the fact that $w$ features no $\sigma_1$ nor $\varepsilon_1$ terms.
	
% 	Well, because $w$ features no $\sigma_1$ nor $\varepsilon_1$ terms, the only terms we care about have $g\varepsilon_i$ where $g$ has no $\sigma_1$ and $i>1$; in this case,
% 	\[\rho\left(g\sigma_1^a\varepsilon_i\right)\coloneqq\sigma_1^ag_i\big(\sigma_i^{a_i}-N_i1_{a_i=n_i-1}\big)\varepsilon_i+\sum_{j=i+1}^m\sigma_1^ag_j\sigma_j^{(a_j)}T_i\varepsilon_j=\sigma_1^a\rho(g\varepsilon_i),\]
% 	where $g\coloneqq\prod_{i=2}^m\sigma_i^{a_i}$ with $0\le a_i<n_i$. Looping over all possible $g$ and $\varepsilon_i$, we see $\rho(\sigma_1^aw)=\sigma_1^a\rho(w)$, so
% 	\[N_1\rho(w)=\rho(N_1w)=0.\]
% 	Thus, $\rho(w)\in\im T_1$, so say $\rho(w)=(\sigma_1-1)w'$. However, because $w$ has no $\varepsilon_1$ terms nor any term with a $\sigma_1$, we can see from the expansion of $\rho(w)$ that $\rho(w)$ will have no $\sigma_1$ terms. It follows that $\rho(w)\in\ZZ[G]^m$ is preserved upon applying $\sigma_1\mapsto1$, but then $(\sigma_1-1)w'$ gets sent to $0$, so it follows $\rho(w)=0$. This finishes.
% \end{proof}
% \begin{lemma} \label{lem:cocycleforcecohere}
% 	Fix everything as in the set-up. Suppose we have $\{z_i\}_{i=1}^m\subseteq\ZZ[G]$ such that
% 	\[T_iz_j\varepsilon_j=T_jz_i\varepsilon_i\]
% 	in $\coker\mathcal F$, for any pair of indices $(i,j)$. Then there exists $z\in\ZZ[G]$ such that $z\varepsilon_i=z\varepsilon_i$ (in $\coker\mathcal F$) for each index $i$.
% \end{lemma}
% \begin{proof}
% 	We proceed by induction on $m$. For $m=1$, we simply set $z\coloneqq z_1$. For the inductive step, take $m>1$, and we are given elements $\{z_i\}_{i=1}^m\subseteq\ZZ[G]$ such that
% 	\[T_iz_j\varepsilon_j=T_jz_i\varepsilon_i\]
% 	for any pair of indices $(i,j)$. By the inductive hypothesis, we may use the equations with indices less than $m$ to conjure some $z\in\ZZ[G]$ such that
% 	\[z\varepsilon_i\equiv z_i\varepsilon_i\pmod{\im\mathcal F}\]
% 	for each $i<m$. It remains to deal with the equations which have $m$ as an index; namely, for each $i<m$, we have an equation
% 	\[T_iz_m\varepsilon_m\equiv T_mz_i\varepsilon_i\equiv T_mz\varepsilon_i\pmod{\im\mathcal F}.\]
% 	Now, $T_m\varepsilon_i\equiv T_i\varepsilon_m\pmod{\im\mathcal F}$, so this is equivalent to asserting
% 	\[T_i(z_m-z)\varepsilon_m\equiv0\pmod{\im\mathcal F}\]
% 	for each index $i<m$. Thus, $T_i(z_m-z)\varepsilon_m\in\im\mathcal F$ for each $i$, which we will use by passing through the $\rho$ of \autoref{lem:sessplits}: this is equivalent to $\rho(T_i(z_m-z)\varepsilon_m)=0$ for each $i<m$. Now, we note that any $g=\prod_{j=1}^m\sigma_j^{a_j}\sigma\in G$ and $i<m$ will have
% 	\[\rho(\sigma_ig\varepsilon_m)=\sigma_ig_m\big(\sigma_m^{a_m}-N_i1_{a_m=n_m-1}\big)\varepsilon_m=\sigma_i\rho(g\varepsilon_m),\]
% 	where in particular the sum in $\rho$ vanished because $m$ is the largest index. (Also, we note $(\sigma_ig)_m=\sigma_ig_m$ because $i<m$.) Extending this linearly over all $g\in G$, we see that
% 	\[0=\rho(T_i(z_m-z)\varepsilon_m)=T_i\rho((z_m-z)\varepsilon_m)\]
% 	for each $i<m$. In particular, letting $\rho((z_m-z)\varepsilon_m)=r\varepsilon_m$, we see$r\in\im N_i$ for each $i<m$, so it follows from \autoref{lem:separatenijs} that $r\in\im N_1\cdots N_{m-1}$, so we can find $w\in\ZZ[G]$ such that
% 	\[\rho((z_m-z)\varepsilon_m)=N_1\cdots N_{m-1}w\varepsilon_m.\]
% 	Now, for technical reasons we note that any $g=\prod_{j=1}^m\sigma_j^{a_j}$ gives
% 	\[\rho(g\varepsilon_m)=g_m\big(\sigma_m^{a_m}-N_i1_{a_m=n_m-1}\big)\varepsilon_m,\]
% 	which can have no $\sigma_m^{n_m-1}$ term in it because this would have to come from $\big(\sigma_m^{a_m}-N_i1_{a_m=n_m-1}\big)$, which manually kills all such terms. As such, $N_1\cdots N_{m-1}w$ should have no $\sigma_m^{n_m-1}$ terms, which means $w$ itself should have no such terms.

% 	With this in mind, we set $z'\coloneqq z+N_1\cdots N_{m-1}w$. To check that we haven't broken anything, we note that any $i<m$ has
% 	\[z'\varepsilon_i=z\varepsilon_i+N_1\cdots N_{m-1}w\varepsilon_i\equiv z\varepsilon_i\equiv z_i\varepsilon_i\pmod{\im\mathcal F}\]
% 	where we note that $N_i\varepsilon_i\equiv0\pmod{\im\mathcal F}$. It remains to deal with $i=m$. Because $w$ features no $\sigma_m^{a_m-1}$ terms, we can check that any $g=\prod_{j=1}^m\sigma_j^{a_j}$ with $a_m<n_m-1$ has
% 	\[\rho(g\varepsilon_m)=g_m\big(\sigma_m^{a_m}-N_i1_{a_m=n_m-1}\big)\varepsilon_m=g_m\sigma_m^{a_m}\varepsilon_m=g\varepsilon_m,\]
% 	so $\rho$ will just act as the identity on $w$! Extending this linearly, we see that
% 	\begin{align*}
% 		\rho((z_m-z')\varepsilon_m) &= \rho((z_m-z)\varepsilon_m)-\rho(N_1\cdots N_{m-1}w\varepsilon_m) \\
% 		&= N_1\cdots N_{m-1}w\varepsilon_m-N_1\cdots N_{m-1}w\varepsilon_m \\
% 		&= 0.
% 	\end{align*}
% 	Thus, $(z_m-z')\varepsilon_m\in\im\mathcal F$, so $z_m\varepsilon_m\equiv z\varepsilon_m\pmod{\im\mathcal F}$ as well.
% \end{proof}
% We are now ready to classify our $1$-cocycles.
% \begin{proposition} \label{prop:cocycleclassify}
% 	Fix everything as in the set-up. If $f\in Z^1(G,\coker\mathcal F)$ is a $1$-cocycle, then there exists $z\in\ZZ[G]$ such that $f(\sigma_i)=z\varepsilon_i$ for each index $i$. Combined with the formula in \autoref{lem:compresscocycle}, this fully determines $f$.
% \end{proposition}
% \begin{proof}
% 	We start by noting that each index $i$ has
% 	\[0=f(1)=f\left(\sigma_i^{n_i}\right)=\sigma_i^{(n_i)}f(\sigma_i)=N_i(f(\sigma_i))\]
% 	by plugging in $\sigma_i^{n_i}$ into \autoref{lem:compresscocycle}. Thus, \autoref{lem:cocycleforcecoord} grants us some $z_i\in\ZZ[G]$ such that $f(\sigma_i)=z_i\varepsilon_i$ for each index $i$.

% 	Continuing, we note that each pair of indices $(i,j)$ has
% 	\[\sigma_if(\sigma_j)+f(\sigma_i)=f(\sigma_i\sigma_j)=f(\sigma_j\sigma_i)=\sigma_jf(\sigma_i)+f(\sigma_j),\]
% 	so
% 	\[T_iz_j\varepsilon_j=T_if(\sigma_j)=T_jf(\sigma_i)=T_jz_i\varepsilon_i.\]
% 	Thus, we know from \autoref{lem:cocycleforcecohere} that there exists $z\in\ZZ[G]$ such that $f(\sigma_i)=z_i\varepsilon_i=z\varepsilon_i$ for each index $i$. This completes the proof.
% \end{proof}
% Note that \autoref{prop:cocycleclassify} does not say that all the conjured $1$-cocycles are actually $1$-cocycles. It will be beneficial for us to show this by hand, so we postpone it to the next subsection.

\subsection{Verification of 1-Cocycles}
Here we prove \autoref{prop:manufacturedcocycle}.
% verify that all the $1$-cocycles of \autoref{prop:cocycleclassify} are indeed $1$-cocycles.
Namely, we show that the $1$-cochain $\overline c\in C^1(G,\coker\mathcal F)$ defined by
\[\overline c(g)=\sum_{i=1}^mg_i\sigma_i^{(a_i)}\varepsilon_i\]
where $g\coloneqq\prod_{i=1}^m\sigma_i^{a_i}$ is actually a $1$-cocycle. It will be beneficial for us to do this by hand, which is a matter of brute force. Set $c\in C^1\left(G,\ZZ[G]^m\right)$ defined by
\[c(g)\coloneqq\left(g_i\sigma_i^{(a_i)}\right)^m_{i=1},\]
where $g\coloneqq\prod_{i=1}^m\sigma_i^{a_i}$. We will show that $\im dc\subseteq\im\mathcal F$, which we will mean that $\im\overline{dc}=\im d\overline c=0$, where $f\mapsto\overline f$ is the map $C^\bullet\left(G,\ZZ[G]^m\right)\onto C^\bullet\left(G,\coker\mathcal F\right)$ induced by modding out.

As such, we set $g\coloneqq\prod_{i=1}^m\sigma_i^{a_i}$ and $h\coloneqq\prod_{i=1}^m\sigma_i^{b_i}$ with $0\le a_i,b_i<n_i$ for each $i$. Then, using the division algorithm, write
\[a_i+b_i=n_iq_i+r_i\]
where $q_i\in\{0,1\}$ and $0\le r_i<n_i$ for each $i$. Now, we want to show $dc(g,h)\in\im\mathcal F$, so we begin by writing
\begin{align}
	dc(g,h) &= gc(h)-c(gh)+c(g) \notag \\
	&= g\left(h_i\sigma_i^{(b_i)}\right)_{i=1}^m-\Bigg(\prod_{p=0}^{i-1}\sigma_p^{r_p}\cdot\sigma_i^{(r_i)}\Bigg)_{i=1}^m+\left(g_i\sigma_i^{(a_i)}\right)_{i=1}^m \notag \\
	&= \left(gh_i\sigma_i^{(b_i)}\right)_{i=1}^m-\left(g_ih_i\sigma_i^{(r_i)}\right)_{i=1}^m+\left(g_i\sigma_i^{(a_i)}\right)_{i=1}^m. \label{eq:expandedcocycle}
\end{align}
We now go term-by-term in \autoref{eq:expandedcocycle}. The easiest is the middle term of \autoref{eq:expandedcocycle}, for which we write
\begin{align*}
	g_ih_i\sigma_i^{(r_i)} &= g_ih_i\sigma_i^{(a_i+b_i)}-g_ih_i\sigma_i^{r_i}\sigma_i^{(n_iq_i)} \\
	&= g_ih_i\sigma_i^{(a_i+b_i)}-g_ih_i\sigma_i^{a_i+b_i}\cdot q_iN_i \\
	&= g_ih_i\sigma_i^{(a_i+b_i)}-g_ih_i\cdot q_iN_i,
\end{align*}
where the last equality is because $\sigma_iN_i=N_i$. Thus,
\begin{align*}
	-\left(g_ih_i\sigma_i^{(r_i)}\right)_{i=1}^m &= -\left(g_ih_i\sigma_i^{(a_i+b_i)}\right)_{i=1}^m+\left(g_ih_i\cdot q_iN_i\right)_{i=1}^m \\
	&= -\left(g_ih_i\sigma_i^{(a_i+b_i)}\right)_{i=1}^m+\mathcal F\big((g_ih_iq_i)_i,(0)_{i>j}\big).
\end{align*}
Now, using \autoref{lem:expandgi}, the $i$th coordinate of the left term of \autoref{eq:expandedcocycle} is
\begin{align*}
	gh_i\sigma_i^{(b_i)} &= g_i\sigma_i^{a_i}\Bigg(\prod_{ j=i+1}^{m}\sigma_j^{a_j}\Bigg)h_i\sigma_i^{(b_i)} \\
	&= g_i\Bigg(1+\sum_{j=i+1}^{m}\Bigg(\prod_{q=i+1}^{j-1}\sigma_q^{a_q}\Bigg)\sigma_j^{(a_j)}T_j\Bigg)h_i\sigma_i^{a_i}\sigma_i^{(b_i)} \\
	&= g_ih_i\sigma_i^{a_i}\sigma_i^{(b_i)}+\sum_{j=i+1}^{m}\Bigg(g_i\sigma_i^{a_i}\prod_{q=i+1}^{j-1}\sigma_q^{a_q}\Bigg)h_i\sigma_j^{(a_j)}\sigma_i^{(b_i)}T_j \\
	&= g_ih_i\sigma_i^{a_i}\sigma_i^{(b_i)}+\sum_{j=i+1}^{m}g_jh_i\sigma_j^{(a_j)}\sigma_i^{(b_i)}T_j.
\end{align*}
And lastly, for the right term of \autoref{eq:expandedcocycle}, the $i$th coordinate is
\begin{align*}
	g_i\sigma_i^{(a_i)} &= g_i\Bigg(h_i-\sum_{j=1}^{i-1}h_j\sigma_j^{(b_j)}T_j\Bigg)\sigma_i^{(a_i)} \\
	&= g_ih_i\sigma_i^{(a_i)}-\sum_{j=1}^{i-1}g_ih_j\sigma_i^{(a_i)}\sigma_j^{(b_j)}T_j.
\end{align*}
So to finish, we continue from \autoref{eq:expandedcocycle}, which gives
\begin{align*}
	dc(g,h)-\mathcal F\big((g_ih_iq_i)_i,(0)_{i>j}\big) &= \left(g_ih_i\sigma_i^{a_i}\sigma_i^{(b_i)}\right)_{i=1}^m-\left(g_ih_i\sigma_i^{(a_i+b_i)}\right)_{i=1}^m+\left(g_ih_i\sigma_i^{(a_i)}\right)_{i=1}^m \\
	&\qquad\qquad+\Bigg(\sum_{j=i+1}^{m}g_jh_i\sigma_j^{(a_j)}\sigma_i^{(b_i)}T_j-\sum_{j=1}^{i-1}g_ih_j\sigma_i^{(a_i)}\sigma_j^{(b_j)}T_j\Bigg)_{i=1}^m \\
	&= \Bigg(-\sum_{j=1}^{i-1}g_ih_j\sigma_i^{(a_i)}\sigma_j^{(b_j)}T_j+\sum_{j=i+1}^{m}g_jh_i\sigma_j^{(a_j)}\sigma_i^{(b_i)}T_j\Bigg)_{i=1}^m \\
	&= \mathcal F\left((0)_i,\big(g_ih_j\sigma_i^{(a_i)}\sigma_j^{(b_j)}\big)_{i>j}\right).
\end{align*}
Thus,
\begin{equation}
	dc(g,h) = \mathcal F\left((g_ih_iq_i)_i,\big(g_ih_j\sigma_i^{(a_i)}\sigma_j^{(b_j)}\big)_{i>j}\right)\in\im\mathcal F. \label{eq:computedelta}
\end{equation}
This completes the proof of \autoref{prop:manufacturedcocycle}.

In fact, the above proof has found an explicit element $z$ so that $\mathcal F(z)=dc(g,h)$ for each $g,h\in G$. As such, we recall that we set
\[X\coloneqq\frac{\ZZ[G]^m\times\ZZ[G]^{\binom m2}}{\ker\mathcal F}\]
to give the short exact sequence
\[0\to X\stackrel{\mathcal F}\to\ZZ[G]^m\to\coker\mathcal F\to0.\]
In particular, we can track $\overline c\in Z^1(G,\coker\mathcal F)$ through a boundary morphism: we already have a chosen lift $c\in Z^1(G,\ZZ[G]^m)$ for $\overline c$, and we have also computed $\mathcal F^{-1}\circ dc$ from the above work. This gives the following result.
\begin{cor} \label{cor:deltaccomputation}
	Fix everything as in the set-up. Then the $\overline c$ of \autoref{prop:manufacturedcocycle} has
	\[\delta(c)(g,h)\coloneqq\left((g_ih_iq_i)_i,\big(g_ih_j\sigma_i^{(a_i)}\sigma_j^{(b_j)}\big)_{i>j}\right)\in Z^2(G,X)\]
	where $\delta$ is induced by
	\[0\to X\stackrel{\mathcal F}\to\ZZ[G]^m\to\coker\mathcal F\to0.\]
\end{cor}
\begin{proof}
	This follows from tracking how $\delta$ behaves, using \autoref{eq:computedelta}.
\end{proof}
\begin{remark}
	In some sense, this $\delta(c)$ is exactly the cocycle of \autoref{thm:getcocycle}, where we have abstracted away everything about $A$. We will rigorize this notion in our proof of \autoref{thm:yesitisacocycle}.
\end{remark}
% We are now ready to complete the proof of \autoref{prop:allmanufacturedcocycles}. In fact, we show the following stronger result.
% \begin{proposition} \label{prop:computeh1cokerF}
% 	Fix everything as in the set-up. Further, let $\varepsilon\colon\ZZ[G]\to\ZZ$ be the augmentation map sending $\sigma_i\mapsto1$ for each $i$. Then the following are true.
% 	\begin{listalph}
% 		\item Given any $z\in\ZZ[G]$, the formula
% 		\[f(g)=\sum_{i=1}^mg_i\sigma_i^{(a_i)}\cdot z\varepsilon_i=(z\cdot\overline c)(g)\]
% 		for $g\coloneqq\prod_{i=1}^m\sigma_i^{a_i}$ defines a $1$-cocycle in $Z^1(G,\coker\mathcal F)$. These are all the $1$-cocycles.
% 		\item If $f\in Z^1(G,\coker\mathcal F)$ is a $1$-cocycle, then $[f]=[\varepsilon(z)\cdot\overline c]$ in $H^1(G,\coker\mathcal F)$, for the $z\in\ZZ[G]$ of \autoref{prop:cocycleclassify}. In particular, $H^1(G,\coker\mathcal F)$ is a cyclic abelian group generated by $[\overline c]$.
% 	\end{listalph}
% \end{proposition}
% \begin{proof}
% 	We proceed one at a time.
% 	\begin{listalph}
% 		\item Given $z\in\ZZ[G]$, to see that $f$ is a $1$-cocycle, note that $f=z\cdot\overline c$. Thus, for the $1$-cocycle check, we just note that any $g,h\in G$ have
% 		\begin{align*}
% 			f(gh) &= z\cdot\overline c(gh) \\
% 			&= z\cdot(g\overline c(h)+\overline c(g)) \\
% 			&= gf(h)+f(g)
% 		\end{align*}
% 		because we already know that $\overline c\in Z^1(G,\coker\mathcal F)$.

% 		To see that these are all the $1$-cocycles, let $f\in Z^1(G,\coker\mathcal F)$ be any $1$-cocycle. Then \autoref{prop:cocycleclassify} promises $z\in\ZZ[G]$ such that $f(\sigma_i)=z\varepsilon_i$ for each index $i$, for which \autoref{lem:compresscocycle} tells us that
% 		\[f(g)=\sum_{i=1}^mg_i\sigma_i^{(a_i)}f(\sigma_i)=\sum_{i=1}^mg_i\sigma_i^{(a_i)}\cdot z\varepsilon_i\]
% 		for $g\coloneqq\prod_{i=1}^m\sigma_i^{a_i}$. So $f$ does have the desired form.

% 		\item Fix $f\in Z^1(G,\coker\mathcal F)$, and conjure the corresponding $z\in\ZZ[G]$ of \autoref{prop:cocycleclassify}. We note in part (a) that $f=z\cdot\overline c$, so it remains to show that $[z\cdot\overline c]=[\varepsilon(z)\cdot\overline c]$ in $H^1(G,\coker\mathcal F)$.
		
% 		By linearity of $\ZZ[G]$, it suffices to show that $[g\cdot\overline c]=[\overline c]$ for each $g\in G$. By induction on the number of generators $\sigma_i$ appearing in $g\in G$, it suffices to show that $[\sigma_i\cdot\overline c]=[\overline c]$ for each index $i$. Lastly, by rearranging the $\sigma_i$, it suffices to show that $[\sigma_1\cdot\overline c]=[\overline c]$.

% 		Well, for any $\sigma_i$, we note that
% 		\[(\sigma_1\overline c-\overline c)(\sigma_i)=\sigma_1\varepsilon_i-\varepsilon_i=T_1\varepsilon_i=T_i\varepsilon_1,\]
% 		where in the last equality we have used that we're living in $\coker\mathcal F$. Letting $d\colon C^0(G,\coker\mathcal F)\to B^1(G,\coker\mathcal F)$ denote the corresponding differential, we see
% 		\[(\sigma_1\overline c-\overline c-d\varepsilon_1)(\sigma_i)=T_i\varepsilon_1-(\sigma_i-1)\varepsilon_1=0\]
% 		for each index $i$. Thus, $\sigma_1\overline c-\overline c-d\varepsilon_1)\in Z^1(G,\coker\mathcal F)$ vanishes on all $\sigma_i$, so \autoref{lem:compresscocycle} tells us that it vanishes on all $g\in G$. It follows $[\sigma_1\overline c-\overline c]=[0]$, which finishes.
% 	\end{listalph}
% 	The above parts complete the proof.
% \end{proof}
% \begin{cor} \label{cor:computeh2x}
% 	Fix everything as in the set-up. Then $H^2(G,X)$ is a cyclic abelian group generated by $[\delta(\overline c)]$, where $\delta$ is induced by
% 	\[0\to X\stackrel{\mathcal F}\to\ZZ[G]^m\to\coker\mathcal F\to0.\]
% \end{cor}
% \begin{proof}
% 	From the long exact sequence of cohomology, we see that
% 	\[\delta\colon H^1(G,\coker\mathcal F)\to H^2(G,X)\]
% 	is an isomorphism because $\ZZ[G]^m$ is projective and hence acyclic. Thus, this follows from (b) of \autoref{prop:computeh1cokerF}.
% \end{proof}

\subsection{Tuples via Cohomology}
We continue in the set-up of the previous subsection. The goal of this subsection is to prove \autoref{prop:alternativetuple}. The main idea is that we will be able to finitely generate $\ker\mathcal F$ essentially using the relations of a $\{\sigma_i\}_{i=1}^m$-tuple.

We start with the following basic result.
\begin{lemma} \label{lem:getgens}
	Fix everything as in the set-up. Then $\ker\mathcal F$ contains the following elements.
	\begin{listalph}
		\item $T_p\kappa_p$ for any index $p$.
		\item $N_pN_q\lambda_{pq}$ for any pair of indices $(p,q)$ with $p>q$.
		\item $T_q\kappa_p+N_p\lambda_{pq}$ for any pair of indices $(p,q)$ with $p>q$.
		\item $T_p\kappa_q-N_q\lambda_{pq}$ for any pair of indices $(p,q)$ with $p>q$.
		\item $T_q\lambda_{pr}-T_r\lambda_{pq}-T_p\lambda_{qr}$ for any triplet of indices $(p,q,r)$ with $p>q>r$.
	\end{listalph}
\end{lemma}
\begin{proof}
	We start by showing that all the listed elements are in fact in $\ker\mathcal F$.
	\begin{listalph}
		\item Note that $\mathcal F$ only ever takes the $x_i$ term to $x_iN_i$, so if $x_i=T_i$, then the effect of $x_i$ vanishes.
		\item Similarly, note that $\mathcal F$ only ever takes the $y_{ij}$ term to $y_{ij}T_i$ or $y_{ij}T_j$. As such, if $y_{ij}=N_iN_j$, then the effect of $y_{ij}$ vanishes again.
		\item The only relevant terms are at indices $p$ and $q$. Here, $i=p$ has $\mathcal F$ output
		\[T_qN_p-N_pT_q+0=0.\]
		For $i=q$, we have no $x_q$ term, so we are left with $N_pT_p=0$.
		\item Again, the only relevant terms are at indices $p$ and $q$. This time the interesting term is at $i=q$, where we have
		\[T_pN_q-0+(-N_q)T_p=0.\]
		Then at $i=p$, we simply have $0N_p-(-N_q)T_q+0=0$.
		\item The relevant terms, as usual, are for $i\in\{p,q,r\}$.
		\begin{itemize}
			\item At $i=p$, we have $0-(T_qT_r+(-T_r)T_q)+0=0.$
			\item At $i=q$, we have $0-(-(T_p)T_r)+((-T_r)T_p)=0$.
			\item At $i=r$, we have $0-0+(T_qT_p+(-T_p)T_q)=0$.
		\end{itemize}
	\end{listalph}
	The above checks complete this part of the proof.
\end{proof}
\begin{remark}
	The above elements are intended to encode the relations to be a $\{\sigma_i\}_{i=1}^n$-tuple. We will see this made rigorous in the proof of \autoref{prop:alternativetuple}.
\end{remark}
In fact, the following is true.
\begin{lemma} \label{lem:havegens}
	Fix everything as in the set-up. Then the elements (a)--(e) of \autoref{lem:getgens}, with (b) removed, generate $\ker\mathcal F$.
\end{lemma}
\begin{proof}
	We remark that we callously removed (b) because it is implied (c): $T_q\kappa_p+N_p\lambda_{pq}\in\ker\mathcal F$ implies that
	\[N_q\cdot(T_q\kappa_p+N_p\lambda_{pq})=N_pN_q\lambda_{pq}\]
	is also in $\ker\mathcal F$. Anyway, this proof is long and annoying and hence relegated to \autoref{sec:havegensproof}.
\end{proof}
Here is the payoff for the hard work in \autoref{lem:havegens}.
\propalternativetuple*
\begin{proof}
	Let $\mathcal T$ denote the set of $\{\sigma_i\}_{i=1}^m$-tuples. We now define the map $\varphi\colon\op{Hom}_{\ZZ[G]}(X,A)\to\mathcal T$ by
	\[\varphi\colon f\mapsto\Big(\big(f(\kappa_i)\big)_i,\big(f(\lambda_{ij})\big)_{i>j}\Big).\]
	In other words, we simply read off the values of $f$ from indicators on the coordinates of $X$. It's not hard to see that $\varphi$ is in fact a $G$-module homomorphism, but we will have to check that $\varphi$ is well-defined, for which we have to check the conditions on being a $\{\sigma_i\}_{i=1}^m$-tuple.
	\begin{lemma} \label{lem:kernelisrelations}
		Fix everything as in the set-up, and let $A$ be a $G$-module. Then, given $f\colon\ZZ[G]^m\times\ZZ[G]^{\binom m2}$, we have $\ker\mathcal F\subseteq\ker f$ if and only if
		\[\Big(\big(f(\kappa_i)\big)_i,\big(f(\lambda_{ij})\big)_{i>j}\Big)\]
		is a $\{\sigma_i\}_{i=1}^m$-tuple.
	\end{lemma}
	\begin{proof}
		By \autoref{lem:havegens}, we see $\ker\mathcal F\subseteq\ker f$ if and only if $f$ vanishes on the elements given in \autoref{lem:getgens}. As such, we now run the following checks.
		\begin{enumerate}
			\item We discuss \autoref{eq:tuplefields}. For one, note that $f(\lambda_{ij})\in A$ essentially for free. Now, we note
			\begin{align*}
				f(\kappa_i)\in A^{\langle\sigma_i\rangle} &\iff T_if(\kappa_i)=0 \\
				&\iff f(T_i\kappa_i)=0 \\
				&\iff T_i\kappa_i\in\ker f.
			\end{align*}
			\item We discuss \autoref{eq:tuplerelations}. On one hand, note that $i>j$ has
			\begin{align*}
				N_if(\lambda_{ij})=-T_jf(\lambda_i) &\iff f(N_i\lambda_{ij}+T_j\lambda_i) \\
				&\iff N_i\lambda_{ij}+T_j\lambda_i\in\ker f.
			\end{align*}
			On the other hand,
			\begin{align*}
				-N_jf(\lambda_{ij})=-T_if(\lambda_j) &\iff f(N_j\lambda_{ij}+T_i\lambda_j)=0 \\
				&\iff N_j\lambda_{ij}+T_i\lambda_j\in\ker f.
			\end{align*}
			\item We discuss \autoref{eq:betarelations}. Simply note indices $i>j>k$ have
			\begin{align*}
				T_jf(\lambda_{ik})=T_kf(\lambda_{ij})+T_if(\lambda_{jk}) &\iff f(T_j\lambda_{ik}-T_k\lambda_{ij}-T_i\lambda_{jk})=0 \\
				&\iff T_j\lambda_{ik}-T_k\lambda_{ij}-T_i\lambda_{jk}\in\ker f.
			\end{align*}
		\end{enumerate}
		In total, we see that satisfying the relations to be a $\{\sigma_i\}_{i=1}^m$-tuple exactly encodes the data of having the generators of $\ker\mathcal F$ live in $\ker f$.
	\end{proof}
	So indeed, given $f\colon X\to A$, the above lemma applied to the composite
	\[\ZZ[G]^m\times\ZZ[G]^{\binom m2}\onto X\stackrel{f}\to A\]
	shows that $\varphi(f)\in\mathcal T$.

	To show that $\varphi$ is an isomorphism, we exhibit its inverse; fix some $(\{\alpha_i\},\{\beta_{ij}\}_{i>j})\in\mathcal T$. Well, $\ZZ[G]\times\ZZ[G]^{\binom m2}$ has as a basis the $\kappa_i$ and $\lambda_{ij}$, so we can uniquely define a $G$-module homomorphism $f\colon X\to A$ by
	\[f(\kappa_i)\coloneqq\alpha_i\qquad\text{and}\qquad f(\lambda_{ij})\coloneqq\beta_{ij}\]
	for all relevant indices $i,j$, and in fact the map $\mathcal T\to\op{Hom}_\ZZ\left(\ZZ[G]^m\times\ZZ[G]^{\binom m2},A\right)$ we can see to be a $G$-module homomorphism. However, because these outputs are a $\{\sigma_i\}_{i=1}^m$-tuple, we can read \autoref{lem:kernelisrelations} backward to say that $f$ has kernel containing $\ker\mathcal F$, so in fact we induce a map $\overline f\colon X\to A$.
	
	So in total, we get a $G$-module homomorphism $\psi\colon\mathcal T\to\op{Hom}_{\ZZ[G]}(X,A)$ by
	\[\psi\colon(\{\alpha_i\},\{\beta_{ij}\}_{i>j})\mapsto\overline f,\]
	where $\overline f$ is defined on the basis elements above. Further, $\psi$ is the inverse of $\varphi$ essentially because the $\{\kappa_i\}_i\cup\{\lambda_{ij}\}_{i>j}$ form a basis of $\ZZ[G]^m\times\ZZ[G]^{\binom m2}$. This completes the proof.
\end{proof}
And now because it is so easy, we might as well prove \autoref{thm:yesitisacocycle}.
\thmyesitisacocycle*
\begin{proof}
	The main point is that we have a computation of $\delta(\overline c)$ from \autoref{cor:deltaccomputation}, which we merely need to track through. In particular, fix a $\{\sigma_i\}_{i=1}^m$-tuple $(\{\alpha_i\}_i,\{\beta_{ij}\}_{i>j})$, and let $f\in H^0(G,\op{Hom}_\ZZ(X,A))$ be the corresponding morphism. As such, we may compute
	\[\delta(\overline c)\cup f\colon(g,h)\mapsto\delta(\overline c)(g,h)\otimes_\ZZ gh\cdot f=\delta(\overline c)(g,h)\otimes_\ZZ f.\]
	To pass through evaluation, we set $g\coloneqq\prod_i\sigma_i^{a_i}$ and $h\coloneqq\prod_i\sigma_i^{b_i}$, from which we get
	\begin{align*}
		f(\delta(\overline c)(g,h)) &= f\left((g_ih_iq_i)_i,\big(g_ih_j\sigma_i^{(a_i)}\sigma_j^{(b_j)}\big)_{i>j}\right) \\
		&= \sum_{i=1}^mg_ih_i\floor{\frac{a_i+b_i}{n_i}}\cdot\alpha_i+\sum_{\substack{i,j=1\\i>j}}^mg_ih_j\sigma_i^{(a_i)}\sigma_j^{(b_j)}\cdot\beta_{ij} \\
		&= \sum_{\substack{i,j=1\\i>j}}^m\Bigg(\prod_{p<i}\sigma_p^{a_p}\Bigg)\Bigg(\prod_{q<j}\sigma_q^{b_q}\Bigg)\sigma_i^{(a_i)}\sigma_j^{(b_j)}\beta_{ij}+\sum_{i=1}^mg_ih_i\alpha_i^{\floor{\frac{a_i+b_i}{n_i}}}.
	\end{align*}
	Doing a little more rearrangement and writing this multiplicatively exactly recovers the cocycle of \autoref{thm:getcocycle}. This completes the proof.
\end{proof}

% \subsection{Some Loose Ends}
% We continue in the set-up and notation of the previous subsection.
Though we have proven everything we set out to do in \autoref{sec:overview}, there is more to discuss with our alternate description of tuples. As a taste, we prove the following extension of \autoref{prop:alternativetuple}.
\begin{proposition} \label{prop:alternativetupleclass}
	Fix everything as in the set-up, and let $A$ be a $G$-module. Then the isomorphism of \autoref{prop:alternativetuple} descends to an isomorphism between equivalence classes of $\{\sigma_i\}_{i=1}^m$-tuples are canonically isomorphic to $\widehat H^0(G,\op{Hom}_\ZZ(X,A))$.
\end{proposition}
\begin{proof}
	Recall that the short exact sequence
	\[0\to X\stackrel{\mathcal F}\to\ZZ[G]^m\to\coker\mathcal F\to 0\]
	of $G$-modules splits as $\ZZ$-modules by \autoref{lem:sessplits}, so we have a short exact sequence
	\[0\to\op{Hom}_\ZZ(\coker\mathcal F,A)\to\op{Hom}_\ZZ(\ZZ[G]^m,A)\stackrel{-\circ\mathcal F}\to\op{Hom}_\ZZ(X,A)\to 0.\]
	Now, the key trick will be to compare regular group cohomology with Tate cohomology. To begin, we note that our cohomology theories give the following commutative diagram with exact rows.
	% https://q.uiver.app/?q=WzAsNixbMCwwLCJIXjAoRyxcXG9we0hvbX1fXFxaWihcXFpaW0ddXm0sQSkpIl0sWzEsMCwiSF4wKEcsXFxvcHtIb219X1xcWlooWCxBKSkiXSxbMSwxLCJcXHdpZGVoYXQgSF4wKEcsXFxvcHtIb219X1xcWlooWCxBKSkiXSxbMiwwLCJIXjEoRyxcXG9we0hvbX1fXFxaWihcXGNva2VyXFxtYXRoY2FsIEYsQSkpIl0sWzIsMSwiXFx3aWRlaGF0IEheMShHLFxcb3B7SG9tfV9cXFpaKFxcY29rZXJcXG1hdGhjYWwgRixBKSkiXSxbMCwxLCIwIl0sWzEsMiwiIiwwLHsic3R5bGUiOnsiaGVhZCI6eyJuYW1lIjoiZXBpIn19fV0sWzMsNCwiIiwwLHsibGV2ZWwiOjIsInN0eWxlIjp7ImhlYWQiOnsibmFtZSI6Im5vbmUifX19XSxbMSwzXSxbMiw0XSxbNSwyXSxbMCwxLCItXFxjaXJjXFxtYXRoY2FsIEYiXV0=&macro_url=https%3A%2F%2Fraw.githubusercontent.com%2FdFoiler%2Fnotes%2Fmaster%2Fnir.tex
	\begin{equation}
		\begin{tikzcd}
			{H^0(G,\op{Hom}_\ZZ(\ZZ[G]^m,A))} & {H^0(G,\op{Hom}_\ZZ(X,A))} & {H^1(G,\op{Hom}_\ZZ(\coker\mathcal F,A))} \\
			0 & {\widehat H^0(G,\op{Hom}_\ZZ(X,A))} & {\widehat H^1(G,\op{Hom}_\ZZ(\coker\mathcal F,A))}
			\arrow[two heads, from=1-2, to=2-2]
			\arrow[Rightarrow, no head, from=1-3, to=2-3]
			\arrow[from=1-2, to=1-3]
			\arrow[from=2-2, to=2-3]
			\arrow[from=2-1, to=2-2]
			\arrow["{-\circ\mathcal F}", from=1-1, to=1-2]
		\end{tikzcd} \label{eq:crazycohomology}
	\end{equation}
	Here, the middle vertical map is reduction modulo $\im N_G$. The rows are exact from the long exact sequences, and the square commutes by construction of Tate cohomology. Now, the point is that the diagram induces the isomorphism
	\begin{equation}
		\frac{H^0(G,\op{Hom}_\ZZ(X,A))}{\im(-\circ\mathcal F)}\simeq\widehat H^0(G,\op{Hom}_\ZZ(X,A)), \label{eq:crazyinducediso}
	\end{equation}
	which simply sends $[f]\mapsto[f]$.

	Thus, the main content here will be to track through the image of $-\circ\mathcal F$ in \autoref{eq:crazycohomology}. Let $\mathcal T$ denote the set of $\{\sigma_i\}_{i=1}^m$-triples of $A$, and let $\mathcal T_0$ denote the set (in fact, equivalence class) of triples corresponding to $[0]\in H^2(G,A)$. Letting $\varphi\colon H^0(G,\op{Hom}_\ZZ(X,A))\to\mathcal T$ be defined by
	\[\varphi\colon f\mapsto\Big(\big(f(\kappa_i)\big)_i,\big(f(\lambda_{ij})\big)_{i>j}\Big)\]
	be the isomorphism of \autoref{prop:alternativetuple}, we claim that the image of $-\circ\mathcal F$ in $H^0(G,\op{Hom}_\ZZ(X,A))$ corresponds under $\varphi$ to exactly $\mathcal T_0$.

	Indeed, we take a $G$-module homomorphism $f\colon\ZZ[G]^m\to A$ to the $G$-module homomorphism $(f\circ\mathcal F)\colon X\to A$. Then we compute
	\begin{align*}
		(f\circ\mathcal F)(\kappa_i) &= f(N_i\varepsilon_i) \\
		&= N_if(\varepsilon_i) \\
		(f\circ\mathcal F)(\lambda_{ij}) &= f(T_i\varepsilon_j-T_j\varepsilon_i) \\
		&= T_if(\varepsilon_j)-T_jf(\varepsilon_i)
	\end{align*}
	for all relevant indices $i$ and $j$. Thus,
	\[\varphi(f\circ\mathcal F)=\left(\big(N_if(\varepsilon_i)\big)_{i},\big(T_if(\varepsilon_j)-T_jf(\varepsilon_i)\big)_{i>j}\right),\]
	which we can see lives in $\mathcal T_0$ by definition of our equivalence relation (upon using multiplicative notation). In fact, as $f$ varies, we see that the values of $f(\varepsilon_i)$ may vary over all $A$, so the image of $f\mapsto\varphi(f\circ\mathcal F)$ is exactly all of $\mathcal T_0$. Thus, $\varphi$ induces an isomorphism
	\[\overline\varphi\colon\frac{H^0(G,\op{Hom}_\ZZ(X,A))}{\im(-\circ\mathcal F)}\simeq\frac{\mathcal T}{\mathcal T_0}.\]
	Composing this with the ``identity'' map \autoref{eq:crazyinducediso} finishes the proof.
	% The main point is that the cup product with $\delta(\overline c)$ will induce an isomorphism
	% \[\widehat H^0(G,\op{Hom}_\ZZ(X,L^\times))\to H^2(G,L^\times).\]
	% Indeed, note that $\mathcal F\colon X\to\ZZ[G]^m$ is an embedding of $\ZZ$-modules, so $X$ is a free abelian group because $\ZZ[G]^m$ is. It follows that $X$ is the character group of an algebraic torus $\mathcal T=\op{Hom}_\ZZ(X,\mathbb G_m)$, so we write $X=X^*(\mathcal T)$. Now, the main point is that we can realize the cup-product map of \autoref{thm:yesitisacocycle} in Tate cohomology as
	% \[\cup\colon\widehat H^0(G,\mathcal T(L))\times\widehat H^2(G,X^*(\mathcal T))\to H^2(G,L^\times).\]
	% However, by Tate--Nakayama duality, we know that this pairing is non-degenerate. In particular, because $\delta(\overline c)$ generates $H^2(G,X^*(\mathcal T))=H^2(G,X)$ by \autoref{cor:computeh2x}, we know that the map
	% \[\delta(\overline c)\cup-\colon\widehat H^0(G,\op{Hom}_\ZZ(X,L^\times))\to H^2(G,L^\times)\]
	% must be injective. On the other hand, by taking a cohomology class $[c]\in H^2(G,L^\times)$, lifting to a representative $\{\sigma_i\}_{i=1}^m$-tuple (as in \autoref{thm:classisomorphism}) gives an input to the above cup product map hitting $[c]$. Thus, the above cup product map we already know to be surjective, so it is an isomorphism.
	% We now attack the statement directly. Let $\mathcal T$ denote the set of $\{\sigma_i\}_{i=1}^m$-tuples and $\mathcal T_0$ denote the set (in fact, equivalence class) of tuples corresponding to the trivial cohomology class in $H^2(G,L^\times)$. Then we draw the following diagram, which we claim commutes and is made of isomorphisms.
	% % https://q.uiver.app/?q=WzAsMyxbMCwwLCJcXG1hdGhjYWwgVC9cXG1hdGhjYWwgVF8wIl0sWzAsMSwiXFx3aWRlaGF0IEheMChHLFxcb3B7SG9tfV9cXFpaKFgsTF5cXHRpbWVzKSkiXSxbMSwxLCJcXHdpZGVoYXQgSF4yKEcsTF5cXHRpbWVzKSJdLFswLDJdLFswLDFdLFsxLDJdXQ==&macro_url=https%3A%2F%2Fraw.githubusercontent.com%2FdFoiler%2Fnotes%2Fmaster%2Fnir.tex
	% \[\begin{tikzcd}
	% 	{\mathcal T/\mathcal T_0} \\
	% 	{\widehat H^0(G,\op{Hom}_\ZZ(X,L^\times))} & {\widehat H^2(G,L^\times)}
	% 	\arrow[from=1-1, to=2-2]
	% 	\arrow[from=1-1, to=2-1, dashed]
	% 	\arrow[from=2-1, to=2-2]
	% \end{tikzcd}\]
	% Namely, the map $\mathcal T/\mathcal T_0\to\widehat H^2(G,L^\times)$ sends an equivalence class of tuples to its cocycle, and it is an isomorphism by \autoref{thm:classisomorphism}. Further, the map $\widehat H^0(G,\op{Hom}_\ZZ(X,L^\times))\to\widehat H^2(G,L^\times)$ is the cup product with $\delta(\overline c)$, and it is an isomorphism as described above.
	% Lastly, $\mathcal T/\mathcal T_0\to\widehat H^0(G,\op{Hom}_\ZZ(X,L^\times))$ is descended from the morphism of \autoref{prop:alternativetuple}, so the diagram does indeed commute by \autoref{thm:yesitisacocycle}. In particular, this vertical map is well-defined and in fact an isomorphism by the commutativity of the diagram. This completes the proof.
\end{proof}
\begin{remark}
	This proof feels more motivated coming from the perspective that $X$ ``should'' be a $2$-encoding module (for example, $\coker\mathcal F$ ``should'' be a $1$-encoding module, allowing us to use \autoref{prop:encodingses}), so actually the equivalence relation on the tuples from \autoref{defi:tupleequiv} can be seen as falling out of the quotient
	\[H^0(G,\op{Hom}_\ZZ(X,-))\Rightarrow\widehat H^0(G,\op{Hom}_\ZZ(X,-)).\]
	Indeed, the equivalence relations had better match up anyway.
\end{remark}
% Another loose end we have to tie up is that we showed $H^2(G,X)$ is cyclic generated by $[\delta(\overline c)]$, but we do not actually know the order. Tracking through Tate--Nakayama duality in the proof will tell us that the order is $\#G$, but this requires $G$ to be a Galois group. Thankfully, we are able to work this out for general $G$ using the rest of the theory that we have built.
% \begin{lemma} \label{lem:zivanish}
% 	Fix everything as in the set-up. If $z\in\ZZ[G]$ has $z\varepsilon_i=0$ in $\coker\mathcal F$, then $z\in\im N_i$.
% \end{lemma}
% \begin{proof}
% 	The point is to pass through $\rho$ of \autoref{lem:sessplits}. By possibly rearranging the $\sigma_i$, we may assume that $i=m$. Then, for any $g\coloneqq\prod_{i=1}^m\sigma_i^{a_i}$, we see
% 	\[\rho(g\varepsilon_m)=g_m\big(\sigma_m^{a_m}-N_m1_{a_m=n_m-1}\big)\varepsilon_m=g\varepsilon_m-g_m1_{a_m=n_m-1}\cdot N_m\varepsilon_m.\]
% 	Namely, $\rho(g\varepsilon_m)-g\varepsilon_m=N_mz_g\varepsilon_m$ for some $z_g\in\ZZ[G]$.
	
% 	Extending this linearly, we see that
% 	\[\rho(z\varepsilon_m)-z\varepsilon_m=w\cdot N_m\varepsilon_m\]
% 	for some $w\in\ZZ[G]$, but $z\varepsilon_m=0$ in $\coker\mathcal F$ makes this say $z\varepsilon_m=-w\cdot N_m\varepsilon_m$. Because this is now an equality in $\ZZ[G]^m$, we conclude $z=-w\cdot N_m\in N_m$.
% \end{proof}
% \begin{lemma} \label{lem:computeordc}
% 	Fix everything as in the set-up. Then $z\cdot\overline c=0$ in $Z^1(G,\coker\mathcal F)$ if and only if $z\in\im N_G$, where $N_G=\sum_{g\in G}g$.
% \end{lemma}
% \begin{proof}
% 	In one direction, if $z=N_Gw$, then
% 	\[z\varepsilon_i=N_Gw\varepsilon_i\equiv0\pmod{\im\mathcal F}\]
% 	for each index $i$, so it follows that $(z\cdot\overline c)(\sigma_i)=z\varepsilon_i=0$ for each $\sigma_i$. Thus, using \autoref{lem:compresscocycle}, we conclude that $z\cdot\overline c=0$.

% 	The other direction is more difficult. Suppose that $z\cdot\overline c=0$. In particular, it follows that $(z\cdot\overline c)(\sigma_i)=z\varepsilon_i$ must equal $0$ for each index $i$. In particular, by \autoref{lem:zivanish}, we conclude that $z\in\im N_i$ for each index $i$, which by \autoref{lem:separatenijs} tells us that
% 	\[z\in\im N_1\cdots N_m=\im N_G.\]
% 	This completes the proof.
% \end{proof}
% \begin{prop} \label{prop:finishh1cokerFcomputation}
% 	Fix everything as in the set-up. Then $H^1(G,\coker\mathcal F)$ is cyclic of order $\#G$, generated by $[\overline c]$.
% \end{prop}
% \begin{proof}
% 	To help us use \autoref{prop:computeh1cokerF}, let $\varepsilon\colon\ZZ[G]\to\ZZ$ denote the augmentation map.
	
% 	Note that we already know $H^1(G,\coker\mathcal F)$ is cyclic generated by $[\overline c]$ by \autoref{prop:computeh1cokerF}, so it only remains to compute the order of $[\overline c]$. On one hand, we have an upper bound on the order of $[\overline c]$ because $H^1(G,\coker\mathcal F)$ is $\#G$-torsion, but we can also see this directly: note that \autoref{lem:computeordc} tells us that
% 	\[[0]=[N_G\cdot\overline c].\]
% 	However, $[N_G\cdot\overline c]=[\varepsilon(N_G)\cdot\overline c]=[\#G\cdot\overline c]$ by \autoref{prop:computeh1cokerF}, so we do see that $\#G\cdot\overline c=0$.
	
% 	It remains to show that $[\overline c]$ has order at least $\#G$. As such, it suffices to show that if $n$ has $[n\cdot\overline c]=[0]$, then $\#G\mid n$. In particular, $n\cdot\overline c$ is a coboundary, so letting $d\colon C^0(G,\coker\mathcal F)\to B^1(G,\coker\mathcal F)$ denote the corresponding differential, we have
% 	\[n\cdot\overline c=d\left(\sum_{i=1}^mb_i\varepsilon_i\right)=\sum_{i=1}^mb_i(d\varepsilon_i)\]
% 	for some $\{b_i\}_{i=1}^m\subseteq\ZZ[G]$. Now, $(d\varepsilon_i)(\sigma_j)=T_j\varepsilon_i=T_i\varepsilon_j$ for any pair of indices $(i,j)$, so by the uniqueness of the extension in \autoref{lem:compresscocycle}, we conclude $d\varepsilon_i=T_i\overline c$. Thus, we set
% 	\[z\coloneqq n-\sum_{i=1}^mb_iT_i\]
% 	so that $\varepsilon(z)=n$ and $z\cdot\overline c=0$.

% 	To finish, we note \autoref{lem:computeordc} now tells us that $z\in\im N_G$, so letting $z=N_Gw$, we see that
% 	\[n=\varepsilon(z)=\varepsilon(N_G)\varepsilon(w)=\#G\cdot\varepsilon(w),\]
% 	so $\#G\mid n$. This completes the proof.
% \end{proof}
% \begin{cor}
% 	Fix everything as in the set-up. Then $H^1(G,\coker\mathcal F)$ is cyclic of order $\#G$, generated by $[\delta(\overline c)]$, where $\delta$ is induced by
% 	\[0\to X\stackrel{\mathcal F}\to\ZZ[G]^m\to\coker\mathcal F\to0.\]
% \end{cor}
% \begin{proof}
% 	As in the proof of \autoref{cor:computeh2x}, we note $\delta\colon H^1(G,\coker\mathcal F)\to H^2(G,X)$ is an isomorphism, so this follows from \autoref{prop:finishh1cokerFcomputation}.
% \end{proof}

% \subsection{Some Cup Product Computations}
% We take a brief intermission to establish a little theory on cup products. In this section, we let $G$ denote a generic finite group (not necessarily assumed to be abelian) and $A$ a $G$-module.
% \begin{lemma}[\cite{bonn-lectures}, Proposition~I.5.3] \label{lem:cupproductmorphism}
% 	Let $G$ be a finite group. Given any $G$-modules $A,B,C$ with a $G$-module homomorphism $\varphi\colon B\to C$, the following diagram commutes for any $p,q\in\ZZ$ and $[a]\in\widehat H^p(G,A)$.
% 	% https://q.uiver.app/?q=WzAsNCxbMCwwLCJcXHdpZGVoYXQgSF5wKEcsQikiXSxbMSwwLCJcXHdpZGVoYXQgSF5wKEcsQykiXSxbMCwxLCJcXHdpZGVoYXQgSF57cCtxfShHLEFcXG90aW1lc19cXFpaIEIpIl0sWzEsMSwiXFx3aWRlaGF0IEhee3ArcX0oRyxBXFxvdGltZXNfXFxaWiBDKSJdLFswLDEsIlxcdmFycGhpIl0sWzIsMywiXFx2YXJwaGkiXSxbMCwyLCJhXFxjdXAgLSIsMl0sWzEsMywiYVxcY3VwIC0iXV0=&macro_url=https%3A%2F%2Fraw.githubusercontent.com%2FdFoiler%2Fnotes%2Fmaster%2Fnir.tex
% 	\[\begin{tikzcd}
% 		{\widehat H^q(G,B)} & {\widehat H^q(G,C)} \\
% 		{\widehat H^{p+q}(G,A\otimes_\ZZ B)} & {\widehat H^{p+q}(G,A\otimes_\ZZ C)}
% 		\arrow["\varphi", from=1-1, to=1-2]
% 		\arrow["\id_A\otimes\varphi", from=2-1, to=2-2]
% 		\arrow["{[a]\cup -}"', from=1-1, to=2-1]
% 		\arrow["{[a]\cup -}", from=1-2, to=2-2]
% 	\end{tikzcd}\]
% \end{lemma}
% \begin{proof}
% 	When $p,q\ge0$, we can argue directly. Indeed, we claim that the diagram commutes on the level of homogeneous cochains: let $[a]\in\widehat H^p(G,A)$ and $[b]\in\widehat H^q(G,B)$ be cohomology classes represented by the homogeneous cochains $a\in[a]$ and $b\in[b]$. Tracking along the top of the diagram, we see
% 	\begin{align*}
% 		(a\cup\varphi(b))(g_0,\ldots,g_{p+q}) &= a(g_0,\ldots,g_p)\otimes\varphi(b)(g_p,\ldots,g_{p+1}) \\
% 		&= a(g_0,\ldots,g_p)\otimes\varphi(b(g_p,\ldots,g_{p+1})).
% 	\end{align*}
% 	Tracking along the bottom of the diagram, we see
% 	\begin{align*}
% 		(\id_A\otimes\varphi)(a\cup b)(g_0,\ldots,g_{p+q}) &= (\id_A\otimes\varphi)(a(g_0,\ldots,g_p)\otimes b(g_p,\ldots,g_{p+q})) \\
% 		&= a(g_0,\ldots,g_p)\otimes\varphi(b(g_p,\ldots,g_{p+q})),
% 	\end{align*}
% 	which is equal. This completes the proof in the case of $p,q\ge0$.

% 	We will only need the case of $p,q\ge0$ in the application, but we will go ahead and do the general case now; we dimension-shift $p$ and $q$ downwards. For example, to shift $p$ downwards, we note that the (split) short exact sequence
% 	\begin{equation}
% 		0\to A\otimes_\ZZ I_G\to A\otimes_\ZZ\ZZ[G]\to A\to0 \label{eq:standardashift}
% 	\end{equation}
% 	induces the isomorphism $\delta\colon\widehat H^{p-1}(G,A)\to\widehat H^p(G,I_G\otimes_\ZZ A)$. As such, given $a\in\widehat H^{p-1}(G,A)$, the inductive hypothesis reassures that the following diagram commutes.
% 	% https://q.uiver.app/?q=WzAsNCxbMCwwLCJcXHdpZGVoYXQgSF5wKEcsQikiXSxbMSwwLCJcXHdpZGVoYXQgSF5wKEcsQykiXSxbMCwxLCJcXHdpZGVoYXQgSF57cCtxfShHLElfR1xcb3RpbWVzX1xcWlogQVxcb3RpbWVzX1xcWlogQikiXSxbMSwxLCJcXHdpZGVoYXQgSF57cCtxfShHLElfR1xcb3RpbWVzX1xcWlogQVxcb3RpbWVzX1xcWlogQykiXSxbMCwxLCJcXHZhcnBoaSJdLFsyLDMsIlxcdmFycGhpIl0sWzAsMiwiXFxkZWx0YShhKVxcY3VwIC0iLDJdLFsxLDMsIlxcZGVsdGEoYSlcXGN1cCAtIl1d&macro_url=https%3A%2F%2Fraw.githubusercontent.com%2FdFoiler%2Fnotes%2Fmaster%2Fnir.tex
% 	\[\begin{tikzcd}
% 		{\widehat H^q(G,B)} & {\widehat H^q(G,C)} \\
% 		{\widehat H^{p+q}(G,I_G\otimes_\ZZ A\otimes_\ZZ B)} & {\widehat H^{p+q}(G,I_G\otimes_\ZZ A\otimes_\ZZ C)}
% 		\arrow["\varphi", from=1-1, to=1-2]
% 		\arrow["\varphi", from=2-1, to=2-2]
% 		\arrow["{\delta(a)\cup -}"', from=1-1, to=2-1]
% 		\arrow["{\delta(a)\cup -}", from=1-2, to=2-2]
% 	\end{tikzcd}\]
% 	In other words, all $b\in\widehat H^q(G,B)$ have $\varphi(\delta(a)\cup b)=\delta(a)\cup\varphi(b)$.
	
% 	Now, because \autoref{eq:standardashift} is split, we can hit it with $-\otimes_\ZZ B$ and $-\otimes_\ZZ C$ to induce the following commutative diagram with exact rows.
% 	% https://q.uiver.app/?q=WzAsMTAsWzAsMCwiMCJdLFsxLDAsIkFcXG90aW1lc19cXFpaIElfR1xcb3RpbWVzX1xcWlogQiJdLFsyLDAsIkFcXG90aW1lc19cXFpaXFxaWltHXVxcb3RpbWVzX1xcWlogQiJdLFszLDAsIkFcXG90aW1lc19cXFpaIEIiXSxbNCwwLCIwIl0sWzEsMSwiQVxcb3RpbWVzX1xcWlogSV9HXFxvdGltZXNfXFxaWiBDIl0sWzIsMSwiQVxcb3RpbWVzX1xcWlpcXFpaW0ddXFxvdGltZXNfXFxaWiBDIl0sWzMsMSwiQVxcb3RpbWVzX1xcWlogQyJdLFswLDEsIjAiXSxbNCwxLCIwIl0sWzAsMV0sWzEsMl0sWzIsM10sWzMsNF0sWzgsNV0sWzUsNl0sWzYsN10sWzcsOV0sWzEsNSwiXFx2YXJwaGkiLDJdLFsyLDYsIlxcdmFycGhpIiwyXSxbMyw3LCJcXHZhcnBoaSIsMl1d&macro_url=https%3A%2F%2Fraw.githubusercontent.com%2FdFoiler%2Fnotes%2Fmaster%2Fnir.tex
% 	\[\begin{tikzcd}
% 		0 & {A\otimes_\ZZ I_G\otimes_\ZZ B} & {A\otimes_\ZZ\ZZ[G]\otimes_\ZZ B} & {A\otimes_\ZZ B} & 0 \\
% 		0 & {A\otimes_\ZZ I_G\otimes_\ZZ C} & {A\otimes_\ZZ\ZZ[G]\otimes_\ZZ C} & {A\otimes_\ZZ C} & 0
% 		\arrow[from=1-1, to=1-2]
% 		\arrow[from=1-2, to=1-3]
% 		\arrow[from=1-3, to=1-4]
% 		\arrow[from=1-4, to=1-5]
% 		\arrow[from=2-1, to=2-2]
% 		\arrow[from=2-2, to=2-3]
% 		\arrow[from=2-3, to=2-4]
% 		\arrow[from=2-4, to=2-5]
% 		\arrow["\varphi"', from=1-2, to=2-2]
% 		\arrow["\varphi"', from=1-3, to=2-3]
% 		\arrow["\varphi"', from=1-4, to=2-4]
% 	\end{tikzcd}\]
% 	Letting $\delta_B\colon\widehat H^{p-1}(A\otimes_\ZZ B)\to\widehat H^p(I_G\otimes_\ZZ A\otimes_\ZZ B)$ and $\delta_C\colon\widehat H^{p-1}(A\otimes_\ZZ B)\to\widehat H^p(I_G\otimes_\ZZ A\otimes_\ZZ B)$ denote the corresponding isomorphisms (note that the middle terms are induced and hence acyclic), we note that the functoriality of boundary morphisms tells us that $\varphi\delta_B=\delta_C\varphi$. In total, it follows that $b\in\widehat H^q(G,B)$ will have
% 	\[\delta_C(\varphi(a\cup b))=\varphi(\delta_B(a\cup b))=\varphi(\delta(a)\cup b)\stackrel*=\delta(a)\cup\varphi(b)=\delta_C(a\cup\varphi(b)),\]
% 	where we have used the inductive hypothesis at $\stackrel*=$. Because $\delta_C$ is an isomorphism, this completes the step to shift $p$ downwards to $p-1$.

% 	Shifting $q$ downwards is similar. This time we start with the following commutative diagram whose rows are (split) short exact sequences.
% 	% https://q.uiver.app/?q=WzAsMTAsWzAsMCwiMCJdLFsxLDAsIklfR1xcb3RpbWVzX1xcWlogQiJdLFsyLDAsIlxcWlpbR11cXG90aW1lc19cXFpaIEIiXSxbMywwLCJCIl0sWzQsMCwiMCJdLFsxLDEsIklfR1xcb3RpbWVzX1xcWlogQyJdLFsyLDEsIlxcWlpbR11cXG90aW1lc19cXFpaIEMiXSxbMywxLCJDIl0sWzAsMSwiMCJdLFs0LDEsIjAiXSxbMCwxXSxbMSwyXSxbMiwzXSxbMyw0XSxbOCw1XSxbNSw2XSxbNiw3XSxbNyw5XSxbMSw1LCJcXHZhcnBoaSIsMl0sWzIsNiwiXFx2YXJwaGkiLDJdLFszLDcsIlxcdmFycGhpIiwyXV0=&macro_url=https%3A%2F%2Fraw.githubusercontent.com%2FdFoiler%2Fnotes%2Fmaster%2Fnir.tex
% 	\[\begin{tikzcd}
% 		0 & {I_G\otimes_\ZZ B} & {\ZZ[G]\otimes_\ZZ B} & B & 0 \\
% 		0 & {I_G\otimes_\ZZ C} & {\ZZ[G]\otimes_\ZZ C} & C & 0
% 		\arrow[from=1-1, to=1-2]
% 		\arrow[from=1-2, to=1-3]
% 		\arrow[from=1-3, to=1-4]
% 		\arrow[from=1-4, to=1-5]
% 		\arrow[from=2-1, to=2-2]
% 		\arrow[from=2-2, to=2-3]
% 		\arrow[from=2-3, to=2-4]
% 		\arrow[from=2-4, to=2-5]
% 		\arrow["\varphi"', from=1-2, to=2-2]
% 		\arrow["\varphi"', from=1-3, to=2-3]
% 		\arrow["\varphi"', from=1-4, to=2-4]
% 	\end{tikzcd}\]
% 	In particular, we let $\delta_B'\colon\widehat H^{q-1}(G,B)\to\widehat H^q(G,I_G\otimes_\ZZ B)$ and $\delta_C'\colon\widehat H^{q-1}(G,B)\to\widehat H^q(G,I_G\otimes_\ZZ C)$ denote the induced isomorphisms, and again functoriality of the boundary morphisms tells us that $\varphi\delta_B=\delta_C\varphi$. Now, the inductive hypothesis tells us that the following diagram commutes for any $a\in\widehat H^p(G,A)$.
% 	% https://q.uiver.app/?q=WzAsNCxbMCwwLCJcXHdpZGVoYXQgSF5wKEcsSV9HXFxvdGltZXNfXFxaWiBCKSJdLFsxLDAsIlxcd2lkZWhhdCBIXnAoRyxJX0dcXG90aW1lc19cXFpaIEMpIl0sWzAsMSwiXFx3aWRlaGF0IEhee3ArcX0oRyxBXFxvdGltZXNfXFxaWiBJX0dcXG90aW1lc19cXFpaIEIpIl0sWzEsMSwiXFx3aWRlaGF0IEhee3ArcX0oRyxBXFxvdGltZXNfXFxaWiBJX0dcXG90aW1lc19cXFpaIEMpIl0sWzAsMSwiXFx2YXJwaGkiXSxbMiwzLCJcXHZhcnBoaSJdLFswLDIsImFcXGN1cCAtIiwyXSxbMSwzLCJhXFxjdXAgLSJdXQ==&macro_url=https%3A%2F%2Fraw.githubusercontent.com%2FdFoiler%2Fnotes%2Fmaster%2Fnir.tex
% 	\[\begin{tikzcd}
% 		{\widehat H^q(G,I_G\otimes_\ZZ B)} & {\widehat H^q(G,I_G\otimes_\ZZ C)} \\
% 		{\widehat H^{p+q}(G,A\otimes_\ZZ I_G\otimes_\ZZ B)} & {\widehat H^{p+q}(G,A\otimes_\ZZ I_G\otimes_\ZZ C)}
% 		\arrow["\varphi", from=1-1, to=1-2]
% 		\arrow["\varphi", from=2-1, to=2-2]
% 		\arrow["{a\cup -}"', from=1-1, to=2-1]
% 		\arrow["{a\cup -}", from=1-2, to=2-2]
% 	\end{tikzcd}\]
% 	Namely, any $b\in\widehat H^{p-1}(G,B)$ has
% 	\begin{align*}
% 		\delta_C'(a\cup\varphi(b)) &= (-1)^p\big(a\cup\delta_C'(\varphi(b))\big) \\
% 		&= (-1)^p\big(a\cup\varphi(\delta_B'(b))\big) \\
% 		&\stackrel*= (-1)^p\varphi(a\cup\delta_B'(b)) \\
% 		&= (-1)^p\cdot(-1)^p\varphi(\delta_B'(a\cup b)) \\
% 		&= \delta_C'(\varphi(a\cup b)),
% 	\end{align*}
% 	where we've applied the inductive hypothesis at $\stackrel*=$. Because $\delta_C'$ is an isomorphism, this completes shifting $q$ downwards to $q-1$.
% \end{proof}
% \begin{remark}
% 	An analogous argument shows that a $G$-module homomorphism $\psi\colon A\to B$ induces the following commutative diagram, for any $p,q\in\ZZ$ and $c\in\widehat H^q(G,C)$.
% 	% https://q.uiver.app/?q=WzAsNCxbMCwwLCJcXHdpZGVoYXQgSF5wKEcsQSkiXSxbMSwwLCJcXHdpZGVoYXQgSF5wKEcsQikiXSxbMCwxLCJcXHdpZGVoYXQgSF57cCtxfShHLEFcXG90aW1lc19cXFpaIEMpIl0sWzEsMSwiXFx3aWRlaGF0IEhee3ArcX0oRyxCXFxvdGltZXNfXFxaWiBDKSJdLFswLDEsIlxccHNpIl0sWzIsMywiXFxwc2kiXSxbMCwyLCItXFxjdXAgYyIsMl0sWzEsMywiLVxcY3VwIGMiLDJdXQ==&macro_url=https%3A%2F%2Fraw.githubusercontent.com%2FdFoiler%2Fnotes%2Fmaster%2Fnir.tex
% 	\[\begin{tikzcd}
% 		{\widehat H^p(G,A)} & {\widehat H^p(G,B)} \\
% 		{\widehat H^{p+q}(G,A\otimes_\ZZ C)} & {\widehat H^{p+q}(G,B\otimes_\ZZ C)}
% 		\arrow["\psi", from=1-1, to=1-2]
% 		\arrow["\psi", from=2-1, to=2-2]
% 		\arrow["{-\cup c}"', from=1-1, to=2-1]
% 		\arrow["{-\cup c}"', from=1-2, to=2-2]
% 	\end{tikzcd}\]
% \end{remark}
% And here are some corollaries, tying back into our theory.
% \begin{cor} \label{cor:xhasallcupisos}
% 	Fix notation as in \autoref{sec:overview}. Then, for any $G$-module $A$ and index $i\in\ZZ$, the cup-product map
% 	\[[\delta(\overline c)]\cup-\colon\widehat H^i(G,\op{Hom}_\ZZ(X,A))\to\widehat H^{i+2}(G,A)\]
% 	is an isomorphism.
% \end{cor}
% \begin{proof}
% 	Set $p=2$ and $q=0$ and $c$ to $[\delta(\overline c)]$ in \autoref{prop:dimshiftcupisos}; the hypothesis is satisfied by combining the cup-product map of \autoref{thm:yesitisacocycle} with \autoref{prop:alternativetupleclass}. (Namely, the cup-product map is sending an equivalence class of tuples to the corresponding cohomology class, which is an isomorphism by \autoref{thm:classisomorphism}.) Anyway, \autoref{prop:dimshiftcupisos} does indeed give the result.
% \end{proof}

% \section{Torus Reciprocity}
% In this section we will apply the theory we have built to the specific case where $G$ is a Galois group of an extension of local fields $L/K$.

% In particular, we keep the notation as in \autoref{sec:tuplestudy} while asserting that $G=\op{Gal}(L/K)$ for some finite abelian extension of local fields $L/K$. Now, we note that the embedding
% \[X\stackrel{\mathcal F}\into\ZZ[G]^m\]
% tells us that $X$ embeds into a free abelian group and hence must be a free abelian group. In particular, because $X$ has a $G$-action---which extends to a $\op{Gal}(K^{\op{sep}}/K)$-action by taking quotients---we are promised a $K$-torus $T$ such that
% \[X^*(T)=X.\]
% By dualizing again, we see that $T\simeq\op{Hom}_\ZZ(X,\mathbb G_m-)$. As such, we just set $T\coloneqq\op{Hom}_\ZZ(X,\mathbb G_m-)$, which gives the following result.
% \begin{cor} \label{cor:torustupledescription}
% 	Fix notation as above. Then $H^0(L/K,T(L))$ is in natural bijection with $\{\sigma_i\}_{i=1}^m$-tuples, and $\widehat H^0(L/K,T(L))$ is in natural bijection with equivalence classes of $\{\sigma_i\}_{i=1}^m$-tuples.
% \end{cor}
% \begin{proof}
% 	Because $T(L)=\op{Hom}_\ZZ(X^*(T),L^\times)$, we may plug into \autoref{prop:alternativetuple} and \autoref{prop:alternativetupleclass}.
% \end{proof}
% To continue our discussion, we recall the following generalization of Artin reciprocity.
% \begin{theorem} \label{thm:torusreciprocity}
% 	Let $K$ be a local field and $T$ a $K$-torus. Suppose that $L/K$ is an extension of fields such that the base change $T_L$ is a split torus. Then cup product with the fundamental class $u_{L/K}\in H^2(L/K,L^\times)$ induces an isomorphism
% 	\[-\cup u_{L/K}\colon\widehat H^n(L/K,X_*(T))\to\widehat H^{n+2}(L/K,T(L))\]
% 	for all integers $n$.
% \end{theorem}
% \begin{proof}
% 	Omitted.
% \end{proof}
% Importantly, our torus $T$ is split over $L$ because (say) all characters in $X^*(T)$ are defined over $L$.\todo{I'm not sure if this makes sense.}

% In light of \autoref{cor:torustupledescription}, we are particularly interested in the case of $n=-2$ in \autoref{thm:torusreciprocity}, giving an isomorphism
% \[-\cup u_{L/K}\colon\widehat H^{-2}(L/K,X_*(T))\to\widehat H^0(L/K,T(L)).\]
% For example, we may use \autoref{thm:yesitisacocycle} to create the following diagram.
% % https://q.uiver.app/?q=WzAsMyxbMCwwLCJcXHdpZGVoYXQgSF57LTJ9KEwvSyxYXyooVCkpIl0sWzEsMCwiXFx3aWRlaGF0IEheMChML0ssVChMKSkiXSxbMSwxLCJcXHdpZGVoYXQgSF4yKEwvSyxMXlxcdGltZXMpIl0sWzAsMSwiLVxcY3VwIHVfe0wvS30iXSxbMSwyLCJcXGRlbHRhKFxcb3ZlcmxpbmUgYylcXGN1cC0iXV0=&macro_url=https%3A%2F%2Fraw.githubusercontent.com%2FdFoiler%2Fnotes%2Fmaster%2Fnir.tex
% \[\begin{tikzcd}
% 	{\widehat H^{-2}(L/K,X_*(T))} & {\widehat H^0(L/K,T(L))} \\
% 	& {\widehat H^2(L/K,L^\times)}
% 	\arrow["{-\cup u_{L/K}}", from=1-1, to=1-2]
% 	\arrow["{[\delta(\overline c)]\cup-}", from=1-2, to=2-2]
% \end{tikzcd}\]
% In particular, the vertical arrow is now an isomorphism because we know that equivalence classes of tuples uniquely correspond to cocycles from \autoref{thm:classisomorphism}.

% To complete the above suggestive diagram, we have the following lemma.
% \begin{lemma} \label{lem:torusdiagram}
% 	Let $L/K$ be an extension of local fields, and let $T$ be a $K$-torus which splits over $L$. Then, given arbitrary classes $u\in H^2(L/K,L^\times)$ and $c\in H^2(L/K,X^*(T))$, the following diagram commutes.
% 	% https://q.uiver.app/?q=WzAsNCxbMCwwLCJcXHdpZGVoYXQgSF57LTJ9KEwvSyxYXyooVCkpIl0sWzEsMCwiXFx3aWRlaGF0IEheMChML0ssVChMKSkiXSxbMSwxLCJcXHdpZGVoYXQgSF4yKEwvSyxMXlxcdGltZXMpIl0sWzAsMSwiXFx3aWRlaGF0IEheMChML0ssXFxaWikiXSxbMCwxLCItXFxjdXAgdSJdLFsxLDIsIlxcZGVsdGEoXFxvdmVybGluZSBjKVxcY3VwLSJdLFszLDIsIi1cXGN1cCB1IiwyXSxbMCwzLCJcXGRlbHRhKFxcb3ZlcmxpbmUgYylcXGN1cC0iLDJdXQ==&macro_url=https%3A%2F%2Fraw.githubusercontent.com%2FdFoiler%2Fnotes%2Fmaster%2Fnir.tex
% 	\[\begin{tikzcd}
% 		{\widehat H^{-2}(L/K,X_*(T))} & {\widehat H^0(L/K,T(L))} \\
% 		{\widehat H^0(L/K,\ZZ)} & {\widehat H^2(L/K,L^\times)}
% 		\arrow["{-\cup u}", from=1-1, to=1-2]
% 		\arrow["{c\cup-}", from=1-2, to=2-2]
% 		\arrow["{-\cup u}"', from=2-1, to=2-2]
% 		\arrow["{c\cup-}"', from=1-1, to=2-1]
% 	\end{tikzcd}\]
% \end{lemma}
% \begin{proof}
% 	At a high level, this should follow from the associativity of the cup product, but some care is required because the cup product maps are augmented in various ways throughout the diagram. For peace of mind, we will actually track through these maps.
	
% 	We use the standard resolution by homogeneous cochains; let $G\coloneqq\op{Gal}(L/K)$. We can represent a class $[x]\in\widehat H^{-2}(L/K,X_*(T))$ by an element $x\in\op{Hom}_G(P_{-2},X_*(T))$, where $P_{-2}\coloneqq\op{Hom}_\ZZ(P_1,\ZZ)=\op{Hom}_\ZZ(\ZZ[G]^2,\ZZ)$. To compute our cup products, we also need to define the notation
% 	\[(g_1^*,\ldots,g_q^*)\in P_{-q}\coloneqq\op{Hom}_\ZZ(P_{q-1},\ZZ)=\op{Hom}_\ZZ(\ZZ[G]^q,\ZZ)\]
% 	which behaves as the indicator function for $(g_1,\ldots,g_q)$ on $G^q$. We are now ready to track through our diagram
% 	\begin{itemize}
% 		\item Along the bottom, we start with $c\cup x\in\widehat H^0(L/K,X^*(T)\otimes_\ZZ X_*(T))$, which is
% 		\[(c\cup x)(g)=\sum_{s_1,s_2\in G}c(g,s_1,s_2)\otimes x(s_2^*,s_1^*).\]
% 		Now, passing through $X^*(T)\otimes_\ZZ X_*(T)\to\op{Hom}(\mathbb G_m,\mathbb G_m)$ by $(f\otimes g)\mapsto(f\circ g)$, we get the map
% 		\[(c\cup x)(g)=\prod_{s_1,s_2\in G}c(g,s_1,s_2)\circ x(s_2^*,s_1^*).\]
% 		In particular, the right-hand side is an algebraic map $L^\times\to L^\times$, which we know must take the form $z\mapsto z^{r_g}$ for some $r_g\in\ZZ$. The mapping $g\mapsto r_g$ is the $0$-cocycle in $\widehat H^0(L/K,\ZZ)$.

% 		Next we must compute $(c\cup x)\cup u\in\widehat H^2(L/K,\ZZ\otimes_\ZZ L^\times)$, which is
% 		\[((c\cup x)\cup u)(g_0,g_1,g_2)=r_{g_0}\otimes u(g_0,g_1,g_2).\]
% 		Passing through $\ZZ\otimes_\ZZ L^\times\to L^\times$ by $r\otimes z\mapsto z^r$, we see that we get
% 		\begin{align}
% 			((c\cup x)\cup u)(g_0,g_1,g_2) &= u(g_0,g_1,g_2)^{r_{g_0}} \notag \\
% 			&= \prod_{s_1,s_2\in G}\big(c(g_0,s_1,s_2)\circ x(s_2^*,s_1^*)\big)\big(u(g_0,g_1,g_2)\big), \label{eq:cupsleft}
% 		\end{align}
% 		where in the last equality we applied the definition of $r_{g_0}$.

% 		\item We now track along the top. Starting with $x\cup u\in\widehat H^0(L/K,X_*(T)\otimes_\ZZ L^\times)$, we have
% 		\[(x\cup u)(g)=\sum_{s_1,s_2\in G}x(s_1^*,s_2^*)\otimes u(s_2,s_1,g).\]
% 		Passing through $X_*(T)\otimes_\ZZ L^\times\to T(L)$ by $(f\otimes z)\mapsto f(z)$, we get
% 		\[(x\cup u)(g)=\prod_{s_1,s_2\in G}x(s_1^*,s_2^*)\big(u(s_2,s_1,g)\big).\]
% 		Continuing, we compute $(c\cup(x\cup u))\in\widehat H^2(L/K,X*(T)\otimes_\ZZ T(L))$ as
% 		\[(c\cup(x\cup u))(g_0,g_1,g_2)=c(g_0,g_1,g_2)\otimes\prod_{s_1,s_2\in G}x(s_1^*,s_2^*)\big(u(s_2,s_1,g_2)\big).\]
% 		And to finish, we pass through $X^*(T)\otimes_\ZZ T(L)\to L^\times$ by $(f\otimes z)\mapsto f(z)$, which gives
% 		\begin{equation}
% 			(c\cup(x\cup u))(g_0,g_1,g_2)=\prod_{s_1,s_2\in G}\big(c(g_0,g_1,g_2)\circ x(s_1^*,s_2^*)\big)\big(u(s_2,s_1,g_2)\big). \label{eq:cupsright}
% 		\end{equation}
% 	\end{itemize}
% 	At this point, it might look like we're in trouble because \autoref{eq:cupsleft} and \autoref{eq:cupsright} look different from each other. However, this is just an outcome of how the cup product is defined. Indeed, we know abstractly that we must have $(c\cup x)\cup u=c\cup(x\cup u)$ in $\widehat H^2(L/K,X^*(T)\otimes_\ZZ X_*(T)\otimes_\ZZ L^\times)$, but these are
% 	\begin{align*}
% 		((c\cup x)\cup u)(g_0,g_1,g_2) &= (c\cup x)(g_0)\otimes u(g_0,g_1,g_2) \\
% 		&= \sum_{s_1,s_2\in G}c(g_0,s_1,s_2)\otimes x(s_2^*,s_1^*)\otimes u(g_0,g_1,g_2),
% 	\end{align*}
% 	and
% 	\begin{align*}
% 		(c\cup(x\cup u))(g_0,g_1,g_2) &= c(g_0,g_1,g_2)\otimes(x\cup u)(g_0) \\
% 		&= \sum_{s_1,s_2\in G}c(g_0,g_1,g_2)\otimes x(s_1^*,s_2^*)\otimes u(s_2,s_1,g_0).
% 	\end{align*}
% 	The above two cocycles need to be equal up to a coboundary in $B^2(G,X^*(T)\otimes_\ZZ X_*(T)\otimes_\ZZ L^\times)$, so we have that the map sending $(g_0,g_1,g_2)$ to
% 	\[\sum_{s_1,s_2\in G}c(g_0,s_1,s_2)\otimes x(s_2^*,s_1^*)\otimes u(g_0,g_1,g_2)-\sum_{s_1,s_2\in G}c(g_0,g_1,g_2)\otimes x(s_1^*,s_2^*)\otimes u(s_2,s_1,g_0)\]
% 	lives in $B^2(G,X^*(T)\otimes_\ZZ X_*(T)\otimes_\ZZ L^\times)$. Passing this all through the evaluation map $X^*(T)\otimes_\ZZ X_*(T)\otimes_\ZZ L^\times\to L^\times$ by $(f\otimes g\otimes z)\mapsto(f\circ g)(z)$, we see that the map sending $(g_0,g_1,g_2)$ to
% 	\[\prod_{s_1,s_2\in G}\big(c(g_0,s_1,s_2)\circ x(s_2^*,s_1^*)\big)\big(u(g_0,g_1,g_2)\big)\bigg/\prod_{s_1,s_2\in G}\big(c(g_0,g_1,g_2)\circ x(s_1^*,s_2^*)\big)\big(u(s_2,s_1,g_2)\big)\]
% 	is a coboundary in $B^2(G,L^\times)$. (Namely, homomorphisms $M\to M'$ induce homomorphisms $B^\bullet(G,M)\to B^\bullet(G,M)$.) Thus, \autoref{eq:cupsleft} and \autoref{eq:cupsright} are indeed in the same cocycle class.
% \end{proof}
% The point of \autoref{lem:torusdiagram} is the following result.
% \begin{theorem} \label{thm:goodcuppairing}
% 	Let $L/K$ be an extension of local fields, and let $T$ be a $K$-torus which splits over $L$. Further, suppose there is some $c\in H^2(L/K,X^*(T))$ such that
% 	\[c\cup-\colon\widehat H^0(L/K,T(L))\to\widehat H^2(L/K,L^\times)\]
% 	is an isomorphism. Now, if $u,u'\in H^2(L/K,L^\times)$ has
% 	\[x\cup u=x\cup u'\in\widehat H^0(L/K,T(L))\]
% 	for all $x\in\widehat H^{-2}(L/K,X_*(T))$, then $u=u'$.
% \end{theorem}
% \begin{proof}
% 	In the language of \autoref{lem:torusdiagram}, we are being told that $v=u$ and $v=u'$ are inducing the same top row of the following commutative diagram.
% 	\begin{equation}
% 		\begin{tikzcd}
% 			{\widehat H^{-2}(L/K,X_*(T))} & {\widehat H^0(L/K,T(L))} \\
% 			{\widehat H^0(L/K,\ZZ)} & {\widehat H^2(L/K,L^\times)}
% 			\arrow["{-\cup v}", from=1-1, to=1-2]
% 			\arrow["{c\cup-}", from=1-2, to=2-2]
% 			\arrow["{-\cup v}"', from=2-1, to=2-2]
% 			\arrow["{c\cup-}"', from=1-1, to=2-1]
% 		\end{tikzcd} \label{eq:torusdiagram}
% 	\end{equation}
% 	Quickly, note that $v=u_{L/K}$ makes the top row an isomorphism by \autoref{thm:torusreciprocity} as well as the bottom row an isomorphism (e.g., the generator $[1]\in\widehat H^0(L/K,\ZZ)$ maps to the generator $[1]\cup u_{L/K}=u^1_{L/K}=u_{L/K}$). Additionally, the right arrow is an isomorphism by hypothesis on $c$. Thus, the commutativity of the diagram tells us that the left arrow is also an isomorphism.

% 	Now, because $-\cup u,-\cup u'\colon H^{-2}\colon\widehat H^{-2}(L/K,X_*(T))\to\widehat H^0(L/K,T(L))$ induce the same map along the top row of \autoref{eq:torusdiagram}, and the vertical arrows of \autoref{eq:torusdiagram} are isomorphisms, we see that
% 	\[-\cup u,-\cup u'\colon\widehat H^0(L/K,\ZZ)\to\widehat H^2(L/K,L^\times)\]
% 	must also be the same map (along the bottom row). However, for any $v\in H^2(L/K,L^\times)$, we see that
% 	\[[1]\cup v=v^1=v,\]
% 	so it follows $u=[1]\cup u=[1]\cup u'=u'$. This finishes.
% \end{proof}
% \begin{cor}
% 	Fix notation and in particular the torus $T$ as above. If $u\in H^2(L/K,L^\times)$ induces the same isomorphism of \autoref{thm:torusreciprocity} via the cup-product map $-\cup u$, then $u=u_{L/K}$.
% \end{cor}
% \begin{proof}
% 	We apply \autoref{thm:goodcuppairing} with $c$ set to be $[\delta(\overline c)]$; notably,
% 	\[[\delta(\overline c)]\cup-\colon\widehat H^0(L/K,T(L))\to\widehat H^2(L/K,L^\times)\]
% 	is an isomorphism by combining \autoref{prop:alternativetupleclass} with \autoref{thm:yesitisacocycle}. Thus, \autoref{thm:goodcuppairing} applies and achieves the result upon setting $u'\coloneqq u_{L/K}$.
% \end{proof}
% \begin{remark}
% 	The above corollary is false when $T=\mathbb G_m$ and $G$ is non-cyclic. What makes our $T$ special is that we have some $[\delta(\overline c)]\in H^2(L/K,X^*(T))$ such that
% 	\[[\delta(\overline c)]\cup-\colon\widehat H^0(L/K,T(L))\to\widehat H^2(L/K,L^\times)\]
% 	is an isomorphism. No such element exists when $T=\mathbb G_m$ and $G$ is non-cyclic because the above groups need not even be isomorphic!
% \end{remark}

\subsection{Algebraic Corollaries}
We continue in the set-up of the previous subsection. Observe that \autoref{prop:alternativetupleclass} combined with \autoref{thm:classisomorphism} tells us that we have isomorphisms
\[\delta(\overline c)\cup-\colon\widehat H^0(G,\op{Hom}_\ZZ(X,A))\to\widehat H^2(G,A).\]
In fact, \autoref{lem:cuppingisnatural} tells us that these isomorphisms assemble into a natural isomorphism, so we have the following result.
\begin{theorem}
	Fix everything as in the set-up. Then $X$ is a $2$-encoding module.
\end{theorem}
\begin{proof}
	This follows from the above discussion.
\end{proof}
\begin{remark}
	It is perhaps useful to note that we can show that $X$ is a $2$-encoding module, without the need to digress to tuples as done in \autoref{prop:alternativetupleclass}. Indeed, we recall that
	\[0\to X\to\ZZ[G]^m\to\coker\mathcal F\to0\]
	splits by \autoref{lem:sessplits}, so because $\ZZ[G]^m\cong\ZZ[G]\otimes_\ZZ\ZZ^m$ is induced, it suffices to show that $\coker\mathcal F$ is a $1$-encoding module by \autoref{prop:encodingses}. For this, we can use \autoref{prop:intdualelement} and manually give $x$ and $x^\lor$; here, $x=[\overline c]$ will work, and one can solve for $x^\lor$. (One could even solve for $[\delta(\overline c)]^\lor$ explicitly, though this is harder.)
\end{remark}
Now that we have a $2$-encoding module, we can apply all the theory we built in \autoref{sec:crackpot}. For example, it might have felt like magic that the isomorphism sending a tuple to its cohomology class was induced by a cup product, but in fact this must have been true all along by \autoref{cor:encodingsarecups}.

Here are some other results.
\begin{cor}
	Fix everything as in the set-up. Then $X$ is cohomologically equivalent to $I_G\otimes_\ZZ I_G$.
\end{cor}
\begin{proof}
	We know that $I_G\otimes_\ZZ I_G$ is a $2$-encoding module by \autoref{ex:igisencoding}, from which \autoref{cor:encodingmodules} finishes.
\end{proof}
\begin{cor}
	Fix everything as in the set-up. Then, for any $i\in\ZZ$ and subgroup $H\subseteq G$, we have natural isomorphisms
	\[\op{Res}[\delta(\overline c)]\cup-\colon\widehat H^i(H,\op{Hom}_\ZZ(X,A))\to\widehat H^{i+2}(H,A).\]
\end{cor}
\begin{proof}
	Follow the proof of \autoref{cor:betterencodingdef} to see that we can set $x=[\delta(\overline c)]$ there. This gives the result for $H=G$, and we get general subgroups by appealing to \autoref{cor:encodingsubgroups}.
\end{proof}
\begin{remark}
	Even though we have some notion of restriction, writing a ``tuple'' in $\widehat H^0(H,\op{Hom}_\ZZ(X,A))$ seems somewhat difficult in general. For example, it is not clear how to (in general) write $X$ as $\ZZ[H]^m/M$ for an $H$-module $M$. In simple cases, we have worked this out in \autoref{lem:restricttuple}.
\end{remark}
\begin{cor}
	Fix everything as in the set-up. Then $\widehat H^2(G,X)$ is cyclic of order $\#G$ generated by $[\delta(\overline c)]$.
\end{cor}
\begin{proof}
	This follows from \autoref{cor:h2xcomputation}.
\end{proof}
\begin{remark}
	Fix notation as in \autoref{sec:overview}, and take $m=2$. Then there are natural transformations
	\[\widehat H^2(G,-)\stackrel{[\delta(\overline c)]^\lor\cup-}\Rightarrow\widehat H^0(G,\op{Hom}_\ZZ(X,-))\Rightarrow\widehat H^{-1}(G,-)\]
	sending a $2$-cocycle to its $\{\sigma_i\}_{i=1}^m$-tuple and then to the (class of) $\beta_{10}$. (It turns out that, because $G$ is bicyclic, the equivalence relation on $\beta_{10}$ is exactly what we need to form a class of $\widehat H^{-1}$.) Now, applying \autoref{cor:encodingsarecups}, we see that the right natural transformation must be a cup-product map, so by associativity of the cup product, the entire natural transformation is a cup-product map.

	Thus, analogously to what \autoref{cor:alphaiscupproduct} says for $\alpha$s, we can describe the projection from $2$-cocycles to $\beta$s purely via (restricted) cup products.
\end{remark}
\begin{remark}
	Noting that $\mathcal F\colon X\into\ZZ[G]^m$ implies that $X$ is $\ZZ$-free, there is a torus $T\coloneqq\op{Hom}_\ZZ(X,\mathbb G_m)$. It is conceivable that one could realize the approach of \autoref{rem:artinreciptaste} for our torus $T$.
\end{remark}
% \begin{remark}
% 	One might be surprised that the equivalence class for tuples appeared when we moved from $H^0(G,\op{Hom}_\ZZ(X,-))$ to $\widehat H^0(G,\op{Hom}_\ZZ(X,-))$. In fact, one could see this without magic: with some effort, one can show that $\widehat H^1(G,\coker\mathcal F)=\ZZ/\#G\ZZ$ by direct classification of $1$-cocycles, yielding
% 	\[\widehat H^2(G,X)=\ZZ/\#G\ZZ.\]
% 	From this it follows that a generator $x\in\widehat H^2(G,X)$ must give rise to an isomorphism
% 	\[x\cup-\colon\widehat H^0(G,\op{Hom}_\ZZ(X,-))\Rightarrow\widehat H^2(G,-).\]
% \end{remark}