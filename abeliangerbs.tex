\documentclass{article}
\usepackage[utf8]{inputenc}

\newcommand{\nirpdftitle}{Abelian Extensions}
\usepackage{import}
\inputfrom{.}{nir}
\numberwithin{equation}{section}
\usepackage[backend=biber,
    style=alphabetic,
    sorting=ynt
]{biblatex}
\addbibresource{bib.bib}

\pagestyle{contentpage}

\title{Encoding Cohomology and Classifying Extensions}
\author{Nir Elber}
\date{\today}
\usepackage{graphicx}
\setlength{\headheight}{12.0pt}
\lhead{}
\rhead{\textit{ABELIAN EXTENSIONS}}

\begin{document}

\maketitle

\begin{abstract}
	\noindent We use group cohomology to provide some general theory to classify all group extensions of a $ G$-module $A$ in the case of an abelian group $ G$. The main idea is to use a group presentation of $G$ provide a group presentation of the extension using specially chosen elements of $A$. It turns out that this ``encoding'' of the extension into elements of $A$ enjoys a number of homological niceties.
\end{abstract}

\setcounter{tocdepth}{4}
\tableofcontents

\section{Generalized Periodic Cohomology} \label{sec:crackpot}
% !TEX root = ../abeliangerbs.tex

The goal of this section is to separate out what we can, a priori, expect from our cohomology-encoding modules from what is a special property of the specific cohomology-encoding module we study in the rest of the paper.

Throughout this section, $G$ will be a finite group. To motivate where we are going, we will go ahead and say that a $p$-encoding $G$-module $X$ is a $G$-module equipped with a natural isomorphism
\[\widehat H^i(G,\op{Hom}_\ZZ(X,-))\Rightarrow\widehat H^{i+p}(G,-).\]
The idea is that, in the case of $i=0$ for a specific $G$-modulee $A$, we are taking cohomology of $\widehat H^p(G,A)$ and encoding this data as
\[\widehat H^0(G,\op{Hom}_\ZZ(X,A))=\frac{\op{Hom}_{\ZZ[G]}(X,A)}{N_G\op{Hom}_\ZZ(X,A)}.\]
If $X$ is finitely generated, we can write $X=\ZZ[G]^m/M$ for some $m\ge0$ and $G$-module $M$, so this object essentially picks out $m$ elements of $A$ and encodes some relations among them. In other words, an $m$-tuple of elements in $A$ (satisfying some special relations) is able to encode cohomology.

When we may take $X=\ZZ$, we are essentially studying groups with periodic cohomology, so some results in this section will mimic these results. However, periodic cohomology requires somewhat stringent conditions on the group itself, and allowing this ``free parameter'' $X$ will permit general groups at the cost of a perhaps more complex $X$. For example, when $p\ge0$, we can take $X=I_G^{\otimes p}$, though this $G$-module is quite rough to handle.

\subsection{Shiftable Functors}
The main point of this section is to set up some theory around what we call shiftable functors.
\begin{definition}
	Let $G$ be a finite group. Then a functor $F\colon\op{Mod}_G\to\op{Mod}_G$ is a \textit{shiftable functor} if and only if $F$ is both additive and sends induced modules to induced modules.
\end{definition}
The main point to shiftable functors $F$ is that the dimension-shifting short exact sequences
\[\arraycolsep=1.4pt\begin{array}{ccccccccc}
	0 &\to& I_G\otimes_\ZZ A &\to& \ZZ[G]\otimes_\ZZ A &\to& A &\to& 0 \\
	0 &\to& A &\to& \op{Hom}_\ZZ(\ZZ[G],A) &\to& \op{Hom}_\ZZ(I_G,A) &\to& 0
\end{array}\]
will remain exact upon applying $F$ (because $F$ is additive, and these short exact sequences are $\ZZ$-split), and the middle term will remain induced.

Our key example of a shiftable functor will be $\op{Hom}_\ZZ(X,-)$ for $G$-modules $X$.
\begin{lemma} \label{lem:hompreservesinduced}
	Let $G$ be a finite group and $X$ a $G$-module. Then $\op{Hom}_\ZZ(X,-)$ is a shiftable functor.
\end{lemma}
\begin{proof}
	It is known that $\op{Hom}_\ZZ(X,-)$ is an additive functor, so we just need to check that it sends induced modules to induced modules. Let $M$ be an induced module, and we want to show that $\op{Hom}_\ZZ(X,M)$ is also induced. By definition, we can write $M\coloneqq\op{Hom}_\ZZ(\ZZ[G],A)$ for some $G$-module $A$, where $A$ has perhaps trivial $G$-action. Now, we claim that
	\[\arraycolsep=1.4pt\begin{array}{cccc}
		\varphi\colon& \op{Hom}_\ZZ(X,\op{Hom}_\ZZ(\ZZ[G],A)) &\simeq& \op{Hom}_\ZZ(\ZZ[G],\op{Hom}_\ZZ(X,A)) \\
		\varphi\colon& f &\mapsto& \big(z\mapsto(x\mapsto f(x)(z))\big)
	\end{array}\]
	is an isomorphism of $G$-modules. This will finish because the right-hand $G$-module is induced.
	
	Now, $\varphi$ s a homomorphism of abelian groups because
	\[\varphi(f+f')(z)(x)=(f+g)(x)(z)=\varphi(f)(z)(x)+\varphi(f')(z)(x)\]
	for any $x$ and $z$ and $f,f'\in\op{Hom}_\ZZ(X,\op{Hom}_\ZZ(\ZZ[G],A))$. This is a $G$-module homomorphism because any $g\in G$ and $f\in\op{Hom}_\ZZ(X,\op{Hom}_\ZZ(\ZZ[G],A))$ has
	\begin{align*}
		\varphi(gf)(z)(x) &= \big(g\cdot\varphi(f)(g^{-1}z)\big)(x) \\
		&= g\cdot\varphi(f)(g^{-1}z)(g^{-1}x) \\
		&= g\cdot f(g^{-1}x)(g^{-1}z) \\
		&= \big(g\cdot f(g^{-1}x)\big)(z) \\
		&= (gf)(x)(z) \\
		&= \varphi(gf)(x)(z)
	\end{align*}
	for each $x$ and $z$.

	Now, we define
	\[\arraycolsep=1.4pt\begin{array}{cccc}
		\psi\colon& \op{Hom}_\ZZ(\ZZ[G],\op{Hom}_\ZZ(X,A)) &\simeq& \op{Hom}_\ZZ(X,\op{Hom}_\ZZ(\ZZ[G],A)) \\
		\psi\colon& f &\mapsto& \big(x\mapsto(z\mapsto f(z)(x))\big)
	\end{array}\]
	to be the inverse morphism. The exact same checks show that this is a $G$-module homomorphism, and it is not hard to see that
	\[\varphi\psi(f)(z)(x)=\psi(f)(z)(x)=f(x)(z),\]
	so $\varphi\circ\psi$ is the identity; similarly, $\psi\circ\varphi$ is the identity.
\end{proof}
With that said, we also remark that shifting functors are rather expansive, and we will need a little more freedom in applications.
\begin{lemma} \label{lem:contravariantshiftable}
	Let $G$ be a finite group and $X$ a $G$-module. Then $\op{Hom}_\ZZ(-,X)$ is a (contravariant) shiftable functor.
\end{lemma}
\begin{proof}
	As usual, we already know that our functor is additive, so the main check is that we send induced modules to induced modules. Well, without loss of generality, let $\ZZ[G]\otimes_\ZZ M$ be our induced module. Then the tensor--hom adjunction gives
	\[\op{Hom}_\ZZ(\ZZ[G]\otimes_\ZZ M,X)\simeq\op{Hom}_\ZZ(\ZZ[G],\op{Hom}_\ZZ(M,X)),\]
	which is also a $G$-module isomorphism. This finishes.
\end{proof}
\begin{lemma}
	Let $G$ be a finite group and $X$ a $G$-module. Then $X\otimes_\ZZ-$ is a shiftable functor.
\end{lemma}
\begin{proof}
	Again, $X\otimes_\ZZ-$ is additive, so we just need to check that it sends induced modules to induced modules. Well, suppose $M\coloneqq\ZZ[G]\otimes_\ZZ A$ is an induced module. Then we note the isomorphisms
	\[X\otimes_\ZZ M=X\otimes_\ZZ\ZZ[G]\otimes_\ZZ A\simeq\ZZ[G]\otimes_\ZZ(X\otimes_\ZZ A)\]
	are all also isomorphisms of $G$-modules. Because $\ZZ[G]\otimes_\ZZ(X\otimes_\ZZ A)$ is induced, we are done.
\end{proof}
\begin{lemma}
	Let $G$ be a finite group. If $F$ and $F'$ are shiftable functors, then $F\circ F'$ is a shiftable functor.
\end{lemma}
\begin{proof}
	This follows directly from the definition.
\end{proof}
\begin{example}
	The functor
	\[A\mapsto\op{Hom}_\ZZ(I_G,I_G\otimes_\ZZ A)\]
	is a shiftable functor.
\end{example}

\subsection{Shifting by Cup Products}
A key property of shiftable functors is how we will be able to relate them to each other via cup products. With this in mind, we have the following definition.
\begin{definition}
	Let $G$ be a finite group. Then we define a \textit{shifting pair} $(F,F',X,\eta)$ to be a pair of shiftable functors $F$ and $F'$ equipped with a natural transformation
	\[\eta_\bullet\colon X\otimes_\ZZ F\Rightarrow F'.\]
\end{definition}
\begin{example} \label{ex:shiftingpair}
	Given $G$-modules $X$ and $X'$, there is a canonical pre-composition map
	\[\eta_\bullet\colon\op{Hom}_\ZZ(X',X)\otimes_\ZZ\op{Hom}_\ZZ(X,-)\Rightarrow\op{Hom}_\ZZ(X',-),\]
	so $(\op{Hom}_\ZZ(X,-),\op{Hom}_\ZZ(X',-),\op{Hom}_\ZZ(X',X),\eta_\bullet)$ is a shifting pair.
\end{example}
\begin{lemma} \label{lem:cuppingisnatural}
	Let $G$ be a finite group, and let $(F,F',X,\eta)$ be a shifting pair. Then, given indices $p,q\in\ZZ$ and $c\in\widehat H^p(G,X)$, the cup-product maps
	\[(c\cup-)\colon\widehat H^q(G,F-)\Rightarrow\widehat H^{p+q}(G,F'-)\]
	make a natural transformation of cohomology functors.
\end{lemma}
\begin{proof}
	Given a $G$-module $A$, we note that our cup-product map is defined by
	\[\widehat H^q(G,FA)\stackrel{c\cup-}\to\widehat H^{p+q}(G,X\otimes_\ZZ FA)\stackrel{\eta_A}\to\widehat H^{p+q}(G,F'A).\]
	So, to check naturality, we pick up a $G$-module homomorphism $\varphi\colon A\to B$ and draw the following diagram.
	% https://q.uiver.app/?q=WzAsNixbMCwwLCJcXHdpZGVoYXQgSF5xKEcsRkEpIl0sWzEsMCwiXFx3aWRlaGF0IEhee3ArcX0oRyxYXFxvdGltZXNfXFxaWiBGQSkiXSxbMiwwLCJcXHdpZGVoYXQgSF57cCtxfShHLEYnQSkiXSxbMCwxLCJcXHdpZGVoYXQgSF5xKEcsRkIpIl0sWzEsMSwiXFx3aWRlaGF0IEhee3ArcX0oRyxYXFxvdGltZXNfXFxaWiBGQikiXSxbMiwxLCJcXHdpZGVoYXQgSF57cCtxfShHLEYnQikiXSxbMCwxLCJjXFxjdXAtIl0sWzEsMiwiXFxldGFfQSJdLFs0LDUsIlxcZXRhX0IiXSxbMSw0LCJmIiwyXSxbMiw1LCJmIiwyXSxbMCwzLCJmIiwyXSxbMyw0LCJjXFxjdXAtIl1d&macro_url=https%3A%2F%2Fraw.githubusercontent.com%2FdFoiler%2Fnotes%2Fmaster%2Fnir.tex
	\[\begin{tikzcd}
		{\widehat H^q(G,FA)} & {\widehat H^{p+q}(G,X\otimes_\ZZ FA)} & {\widehat H^{p+q}(G,F'A)} \\
		{\widehat H^q(G,FB)} & {\widehat H^{p+q}(G,X\otimes_\ZZ FB)} & {\widehat H^{p+q}(G,F'B)}
		\arrow["{c\cup-}", from=1-1, to=1-2]
		\arrow["{\eta_A}", from=1-2, to=1-3]
		\arrow["{\eta_B}", from=2-2, to=2-3]
		\arrow["f"', from=1-2, to=2-2]
		\arrow["f"', from=1-3, to=2-3]
		\arrow["f"', from=1-1, to=2-1]
		\arrow["{c\cup-}", from=2-1, to=2-2]
	\end{tikzcd}\]
	The left square commutes by functoriality of cup products (see \cite{bonn-lectures}, Proposition~I.5.3), and the right square commutes by the naturality of $\eta$ and functoriality of $\widehat H^{p+q}(G,-)$.
\end{proof}
Let's start with a key result on shiftable functors, which gives a taste for why our hypotheses are so specially chosen.
\begin{proposition} \label{prop:dimshiftcupisos}
	Let $G$ be a finite group, and let $(F,F',X,\eta)$ be a shifting pair. If we have indices $p,q\in\ZZ$ and $c\in H^p(G,X)$ such that the cup-product map
	\[c\cup-\colon\widehat H^q(G,F-)\Rightarrow\widehat H^{p+q}(G,F'-)\]
	is a natural isomorphism, then the cup-product map
	\[c\cup-\colon\widehat H^j(G,F-)\Rightarrow\widehat H^{p+j}(G,F'-)\]
	is a natural isomorphism and indices $j\in\ZZ$.
\end{proposition}
\begin{proof}
	This proof is by dimension-shifting on $q$. Note that it suffices by \autoref{lem:cuppingisnatural} to only worry about the component morphisms being isomorphisms.
	
	To shift downwards, we suppose that the cup-product map is always an isomorphism for $j$, and we show that it is always an isomorphism $j-1$. Namely, fix a $G$-module $A$, and we are interested in showing that the cup-product map
	\[c\cup-\colon\widehat H^{j-1}(G,FA)\to\widehat H^{p+j-1}(G,F'A)\]
	is an isomorphism. To do so, we note the short exact sequence
	\begin{equation}
		0\to I_G\to\ZZ[G]\to\ZZ\to0 \label{eq:shiftingses}
	\end{equation}
	which splits over $\ZZ$ and thus gives us the short exact sequences
	% https://q.uiver.app/?q=WzAsMTUsWzAsMCwiMCJdLFsxLDAsIlxcb3B7SG9tfV9cXFpaKFgsSV9HXFxvdGltZXNfXFxaWiBBKSJdLFsyLDAsIlxcb3B7SG9tfV9cXFpaKFgsXFxaWltHXVxcb3RpbWVzX1xcWlogQSkiXSxbMywwLCJcXG9we0hvbX1fXFxaWihYLEEpIl0sWzQsMCwiMCJdLFswLDEsIjAiXSxbNCwxLCIwIl0sWzEsMSwiWFxcb3RpbWVzX1xcWlpcXG9we0hvbX1fXFxaWihYLElfR1xcb3RpbWVzX1xcWlogQSkiXSxbMiwxLCJYXFxvdGltZXNfXFxaWlxcb3B7SG9tfV9cXFpaKFgsXFxaWltHXVxcb3RpbWVzX1xcWlogQSkiXSxbMywxLCJYXFxvdGltZXNfXFxaWlxcb3B7SG9tfV9cXFpaKFgsQSkiXSxbMCwyLCIwIl0sWzEsMiwiSV9HXFxvdGltZXNfXFxaWiBBIl0sWzIsMiwiXFxaWltHXVxcb3RpbWVzX1xcWlogQSJdLFszLDIsIkEiXSxbNCwyLCIwIl0sWzAsMV0sWzEsMl0sWzIsM10sWzMsNF0sWzUsN10sWzcsOF0sWzgsOV0sWzksNl0sWzEwLDExXSxbMTEsMTJdLFsxMiwxM10sWzEzLDE0XSxbNywxMSwiXFxldGFfe0lfR30iLDJdLFs4LDEyLCJcXGV0YV97XFxaWltHXX0iLDJdLFs5LDEzLCJcXGV0YV9BIiwyXV0=&macro_url=https%3A%2F%2Fraw.githubusercontent.com%2FdFoiler%2Fnotes%2Fmaster%2Fnir.tex
	\[\begin{tikzcd}
		0 & {F(I_G\otimes_\ZZ A)} & {F(\ZZ[G]\otimes_\ZZ A)} & {FA} & 0 \\
		0 & {X\otimes_\ZZ F(I_G\otimes_\ZZ A)} & {X\otimes_\ZZ F(\ZZ[G]\otimes_\ZZ A)} & {X\otimes_\ZZ FA} & 0 \\
		0 & {F'(I_G\otimes_\ZZ A)} & {F'(\ZZ[G]\otimes_\ZZ A)} & {F'A} & 0
		\arrow[from=1-1, to=1-2]
		\arrow[from=1-2, to=1-3]
		\arrow[from=1-3, to=1-4]
		\arrow[from=1-4, to=1-5]
		\arrow[from=2-1, to=2-2]
		\arrow[from=2-2, to=2-3]
		\arrow[from=2-3, to=2-4]
		\arrow[from=2-4, to=2-5]
		\arrow[from=3-1, to=3-2]
		\arrow[from=3-2, to=3-3]
		\arrow[from=3-3, to=3-4]
		\arrow[from=3-4, to=3-5]
		\arrow["{\eta_{I_G}}"', from=2-2, to=3-2]
		\arrow["{\eta_{\ZZ[G]}}"', from=2-3, to=3-3]
		\arrow["{\eta_A}"', from=2-4, to=3-4]
	\end{tikzcd}\]
	where the bottom two rows commute by definition of $\eta$ and thus give a morphism of short exact sequences. These short exact sequences give us boundary morphisms
	\[\arraycolsep=1.4pt\begin{array}{rlcl}
		\delta\colon& \widehat H^{p+j-1}(G,F'A) &\to& \widehat H^{p+j}(G,F'(I_G\otimes_\ZZ A)) \\
		\delta_h\colon& \widehat H^{j-1}(G,FA) &\to& \widehat H^j(G,F(I_G\otimes_\ZZ A)) \\
		\delta_t\colon& \widehat H^{p+j-1}(G,X\otimes_\ZZ FA) &\to& \widehat H^{p+j}(G,X\otimes_\ZZ F(I_G\otimes_\ZZ A)).
	\end{array}\]
	Notably, all these $\delta$ morphisms because their short exact sequences have induced middle terms: in particular, $F$, $X\otimes_\ZZ F$, and $F'$ are all shiftable functors.
	
	Now, the key to this dimension-shifting is claiming that the diagram
	% https://q.uiver.app/?q=WzAsNCxbMCwwLCJcXHdpZGVoYXQgSF57ai0xfShHLFxcb3B7SG9tfV9cXFpaKFgsQSkpIl0sWzEsMCwiXFx3aWRlaGF0IEhee3Arai0xfShHLEEpIl0sWzAsMSwiXFx3aWRlaGF0IEhee2p9KEcsXFxvcHtIb219X1xcWlooWCxJX0dcXG90aW1lc19cXFpaIEEpKSJdLFsxLDEsIlxcd2lkZWhhdCBIXntwK2p9KEcsSV9HXFxvdGltZXNfXFxaWiBBKSJdLFswLDEsImNcXGN1cC0iXSxbMiwzLCJjXFxjdXAtIl0sWzAsMiwiXFxkZWx0YV9oIiwyXSxbMSwzLCJcXGRlbHRhIiwyXV0=&macro_url=https%3A%2F%2Fraw.githubusercontent.com%2FdFoiler%2Fnotes%2Fmaster%2Fnir.tex
	\[\begin{tikzcd}
		{\widehat H^{j-1}(G,FA)} & {\widehat H^{p+j-1}(G,F'A)} \\
		{\widehat H^{j}(G,F(I_G\otimes_\ZZ A))} & {\widehat H^{p+j}(G,F'(I_G\otimes_\ZZ A))}
		\arrow["{c\cup-}", from=1-1, to=1-2]
		\arrow["{c\cup-}", from=2-1, to=2-2]
		\arrow["{\delta_h}"', from=1-1, to=2-1]
		\arrow["(-1)^p\delta"', from=1-2, to=2-2]
	\end{tikzcd}\]
	commutes. Indeed, this will be enough because the bottom row is an isomorphism by the inductive hypothesis, and the left and morphisms are isomorphisms as discussed above, which makes the top row into an isomorphism. Well, to see that the diagram commutes, we expand the diagram as follows.
	% https://q.uiver.app/?q=WzAsNixbMCwwLCJcXHdpZGVoYXQgSF57ai0xfShHLFxcb3B7SG9tfV9cXFpaKFgsQSkpIl0sWzEsMCwiXFx3aWRlaGF0IEhee3Aran0oRyxYXFxvdGltZXNfXFxaWlxcb3B7SG9tfV9cXFpaKEEpKSJdLFswLDEsIlxcd2lkZWhhdCBIXntqfShHLFxcb3B7SG9tfV9cXFpaKFgsSV9HXFxvdGltZXNfXFxaWiBBKSkiXSxbMSwxLCJcXHdpZGVoYXQgSF57cCtqfShHLFhcXG90aW1lc19cXFpaXFxvcHtIb219X1xcWlooSV9HXFxvdGltZXNfXFxaWiBBKSkiXSxbMiwwLCJcXHdpZGVoYXQgSF57cCtqLTF9KEcsQSkiXSxbMiwxLCJcXHdpZGVoYXQgSF57cCtqfShHLElfR1xcb3RpbWVzX1xcWlogQSkiXSxbMCwxLCJjXFxjdXAtIl0sWzIsMywiY1xcY3VwLSJdLFswLDIsIlxcZGVsdGFfaCIsMl0sWzEsMywiXFxkZWx0YV90IiwyXSxbMSw0LCJcXGV0YV9BIl0sWzMsNSwiXFxldGFfe0lfR30iXSxbNCw1LCJcXGRlbHRhIiwyXV0=&macro_url=https%3A%2F%2Fraw.githubusercontent.com%2FdFoiler%2Fnotes%2Fmaster%2Fnir.tex
	\[\begin{tikzcd}
		{\widehat H^{j-1}(G,FA)} & {\widehat H^{p+j-1}(G,X\otimes_\ZZ FA)} & {\widehat H^{p+j-1}(G,F'A)} \\
		{\widehat H^{j}(G,F(I_G\otimes_\ZZ A))} & {\widehat H^{p+j}(G,X\otimes_\ZZ F(I_G\otimes_\ZZ A))} & {\widehat H^{p+j}(G,F'(I_G\otimes_\ZZ A))}
		\arrow["{c\cup-}", from=1-1, to=1-2]
		\arrow["{c\cup-}", from=2-1, to=2-2]
		\arrow["{\delta_h}"', from=1-1, to=2-1]
		\arrow["{(-1)^p\delta_t}"', from=1-2, to=2-2]
		\arrow["{\eta_A}", from=1-2, to=1-3]
		\arrow["{\eta_{I_G}}", from=2-2, to=2-3]
		\arrow["(-1)^p\delta"', from=1-3, to=2-3]
	\end{tikzcd}\]
	The left square commutes because cup products commute with boundary morphisms; the right square commutes by functoriality of boundary morphisms.

	Shifting upwards is similar. Suppose that the cup-product in question is always an isomorphism for $j$, and we show that it is always an isomorphism for $j+1$. Namely, fix a $G$-module $A$, and we are interested in showing that the cup-product map
	\[c\cup-\colon\widehat H^{j+1}(G,FA)\to\widehat H^{p+j+1}(G,F'A)\]
	is an isomorphism. As before, we use \autoref{eq:shiftingses} to induce the short exact sequences
	% https://q.uiver.app/?q=WzAsMTUsWzAsMCwiMCJdLFsxLDAsIlxcb3B7SG9tfV9cXFpaKFgsQSkiXSxbMiwwLCJcXG9we0hvbX1fXFxaWihYLFxcb3B7SG9tfV9cXFpaKFxcWlpbR10sQSkpIl0sWzMsMCwiXFxvcHtIb219X1xcWlooWCxcXG9we0hvbX1fXFxaWihJX0csQSkpIl0sWzQsMCwiMCJdLFswLDEsIjAiXSxbNCwxLCIwIl0sWzEsMSwiWFxcb3RpbWVzX1xcWlpcXG9we0hvbX1fXFxaWihYLEEpIl0sWzIsMSwiWFxcb3RpbWVzX1xcWlpcXG9we0hvbX1fXFxaWihYLFxcb3B7SG9tfV9cXFpaKFxcWlpbR10sQSkpIl0sWzMsMSwiWFxcb3RpbWVzX1xcWlpcXG9we0hvbX1fXFxaWihYLFxcb3B7SG9tfV9cXFpaKElfRyxBKSkiXSxbMCwyLCIwIl0sWzEsMiwiQSJdLFsyLDIsIlxcb3B7SG9tfV9cXFpaKFxcWlpbR10sQSkiXSxbMywyLCJcXG9we0hvbX1fXFxaWihJX0csQSkiXSxbNCwyLCIwIl0sWzAsMV0sWzEsMl0sWzIsM10sWzMsNF0sWzUsN10sWzcsOF0sWzgsOV0sWzksNl0sWzEwLDExXSxbMTEsMTJdLFsxMiwxM10sWzEzLDE0XSxbNywxMSwiXFxldGFfQSIsMl0sWzgsMTIsIlxcZXRhX3tcXFpaW0ddfSIsMl0sWzksMTMsIlxcZXRhX3tJX0d9IiwyXV0=&macro_url=https%3A%2F%2Fraw.githubusercontent.com%2FdFoiler%2Fnotes%2Fmaster%2Fnir.tex
	\[\begin{tikzcd}
		0 & {FA} & {F(\op{Hom}_\ZZ(\ZZ[G],A))} & {F(\op{Hom}_\ZZ(I_G,A))} & 0 \\
		0 & {X\otimes_\ZZ FA} & {X\otimes_\ZZ F(\op{Hom}_\ZZ(\ZZ[G],A))} & {X\otimes_\ZZ F(\op{Hom}_\ZZ(I_G,A))} & 0 \\
		0 & F'A & {F'(\op{Hom}_\ZZ(\ZZ[G],A))} & {F'(\op{Hom}_\ZZ(I_G,A))} & 0
		\arrow[from=1-1, to=1-2]
		\arrow[from=1-2, to=1-3]
		\arrow[from=1-3, to=1-4]
		\arrow[from=1-4, to=1-5]
		\arrow[from=2-1, to=2-2]
		\arrow[from=2-2, to=2-3]
		\arrow[from=2-3, to=2-4]
		\arrow[from=2-4, to=2-5]
		\arrow[from=3-1, to=3-2]
		\arrow[from=3-2, to=3-3]
		\arrow[from=3-3, to=3-4]
		\arrow[from=3-4, to=3-5]
		\arrow["{\eta_A}"', from=2-2, to=3-2]
		\arrow["{\eta_{\ZZ[G]}}"', from=2-3, to=3-3]
		\arrow["{\eta_{I_G}}"', from=2-4, to=3-4]
	\end{tikzcd}\]
	where again the bottom rows commute by definition of $\eta$. As before, we have the boundary morphisms
	\[\arraycolsep=1.4pt\begin{array}{rlcl}
		\delta\colon& \widehat H^{p+j}(G,F'(\op{Hom}_\ZZ(I_G,A))) &\to& \widehat H^{p+j+1}(G,F'A) \\
		\delta_h\colon& \widehat H^{j}(G,F(\op{Hom}_\ZZ(I_G,A))) &\to& \widehat H^{j+1}(G,FA) \\
		\delta_t\colon& \widehat H^{p+j}(G,X\otimes_\ZZ F(\op{Hom}_\ZZ(I_G,A))) &\to& \widehat H^{p+j+1}(G,X\otimes_\ZZ FA).
	\end{array}\]
	We again note that all $\delta$ are isomorphisms because the middle terms of our short exact sequences are induced: all of $F$ and $X\otimes_\ZZ F$ and $F'$ are shiftable functors.

	Once more, the key to the dimension-shifting will be the claim that the diagram
	% https://q.uiver.app/?q=WzAsNCxbMCwwLCJcXHdpZGVoYXQgSF57an0oRyxcXG9we0hvbX1fXFxaWihYLFxcb3B7SG9tfV9cXFpaKElfRyxBKSkpIl0sWzAsMSwiXFx3aWRlaGF0IEhee2orMX0oRyxcXG9we0hvbX1fXFxaWihYLEEpKSJdLFsxLDAsIlxcd2lkZWhhdCBIXntwK2p9KEcsXFxvcHtIb219X1xcWlooSV9HLEEpKSJdLFsxLDEsIlxcd2lkZWhhdCBIXntwK2orMX0oRyxBKSJdLFswLDEsIlxcZGVsdGFfaCIsMl0sWzIsMywiXFxkZWx0YSIsMl0sWzAsMiwiY1xcY3VwLSJdLFsxLDMsImNcXGN1cC0iXV0=&macro_url=https%3A%2F%2Fraw.githubusercontent.com%2FdFoiler%2Fnotes%2Fmaster%2Fnir.tex
	\[\begin{tikzcd}
		{\widehat H^{j}(G, F(\op{Hom}_\ZZ(I_G,A)))} & {\widehat H^{p+j}(G,F'(\op{Hom}_\ZZ(I_G,A)))} \\
		{\widehat H^{j+1}(G,FA)} & {\widehat H^{p+j+1}(G,F'A)}
		\arrow["{\delta_h}"', from=1-1, to=2-1]
		\arrow["(-1)^p\delta"', from=1-2, to=2-2]
		\arrow["{c\cup-}", from=1-1, to=1-2]
		\arrow["{c\cup-}", from=2-1, to=2-2]
	\end{tikzcd}\]
	commutes. This will be enough because the top arrow is an isomorphism by the inductive hypothesis, and the left and right arrows are isomorphisms as discussed above, thus making the bottom arrow also an isomorphism. Now, to see that the diagram commutes, we expand out our cup products as follows.
	% https://q.uiver.app/?q=WzAsNixbMCwwLCJcXHdpZGVoYXQgSF57an0oRyxcXG9we0hvbX1fXFxaWihYLFxcb3B7SG9tfV9cXFpaKElfRyxBKSkpIl0sWzAsMSwiXFx3aWRlaGF0IEhee2orMX0oRyxcXG9we0hvbX1fXFxaWihYLEEpKSJdLFsxLDAsIlxcd2lkZWhhdCBIXntwK2p9KEcsWFxcb3RpbWVzX1xcWlpcXG9we0hvbX1fXFxaWihJX0csQSkpIl0sWzEsMSwiXFx3aWRlaGF0IEhee3AraisxfShHLFhcXG90aW1lc19cXFpaXFxvcHtIb219X1xcWlooWCxBKSkiXSxbMiwxLCJcXHdpZGVoYXQgSF57cCtqKzF9KEcsQSkiXSxbMiwwLCJcXHdpZGVoYXQgSF57cCtqfShHLFxcb3B7SG9tfV9cXFpaKElfRyxBKSkiXSxbMCwxLCJcXGRlbHRhX2giLDJdLFsyLDMsIigtMSlecFxcZGVsdGFfdCIsMl0sWzAsMiwiY1xcY3VwLSJdLFsxLDMsImNcXGN1cC0iXSxbMiw1LCJcXGV0YV97SV9HfSJdLFszLDQsIlxcZXRhX0EiXSxbNSw0LCIoLTEpXnBcXGRlbHRhIiwyXV0=&macro_url=https%3A%2F%2Fraw.githubusercontent.com%2FdFoiler%2Fnotes%2Fmaster%2Fnir.tex
	\[\begin{tikzcd}
		{\widehat H^{j}(G,F(\op{Hom}_\ZZ(I_G,A)))} & {\widehat H^{p+j}(G,X\otimes_\ZZ F(\op{Hom}_\ZZ(I_G,A)))} & {\widehat H^{p+j}(G,F'(\op{Hom}_\ZZ(I_G,A)))} \\
		{\widehat H^{j+1}(G,FA)} & {\widehat H^{p+j+1}(G,X\otimes_\ZZ FA)} & {\widehat H^{p+j+1}(G,F'A)}
		\arrow["{\delta_h}"', from=1-1, to=2-1]
		\arrow["{(-1)^p\delta_t}"', from=1-2, to=2-2]
		\arrow["{c\cup-}", from=1-1, to=1-2]
		\arrow["{c\cup-}", from=2-1, to=2-2]
		\arrow["{\eta_{I_G}}", from=1-2, to=1-3]
		\arrow["{\eta_A}", from=2-2, to=2-3]
		\arrow["{(-1)^p\delta}"', from=1-3, to=2-3]
	\end{tikzcd}\]
	The left square commutes because cup products commute with boundary morphisms, and the right square commutes by functoriality of boundary morphisms. This finishes.
\end{proof}
Here are some applications.
\begin{cor} \label{cor:cupup}
	Let $G$ be a finite group. There exists $c\in\widehat H^1(G,I_G)$ such that, for any $G$-module $X$,
	\[c\cup-\colon\widehat H^i(G,\op{Hom}_\ZZ(X,-))\Rightarrow\widehat H^{i+1}(G,\op{Hom}_\ZZ(X,I_G\otimes_\ZZ-))\]
	is a natural isomorphism for any $i\in\ZZ$.
\end{cor}
\begin{proof}
	Here, we are using the shifting pair $(\op{Hom}_\ZZ(X,-),\op{Hom}_\ZZ(X,I_G\otimes_\ZZ-),I_G,\eta)$, where
	\[\eta_A\colon I_G\otimes_\ZZ\op{Hom}_\ZZ(X,A)\to\op{Hom}_\ZZ(X,I_G\otimes_\ZZ A)\]
	is the canonical map sending $z\otimes f$ to $x\mapsto z\otimes f(x)$.

	Now, in light of \autoref{prop:dimshiftcupisos}, we merely have to find $c\in\widehat H^1(G,I_G)$ and show that we have a natural isomorphism at $i=0$. Because we already have a natural transformation by \autoref{lem:cuppingisnatural}, we are only worried about making the component morphisms
	\[\widehat H^0(G,\op{Hom}_\ZZ(X,A))\to\widehat H^1(G,\op{Hom}_\ZZ(X,I_G\otimes_\ZZ A))\]
	isomorphisms for all $G$-modules $A$. Well, we note that we have the ($\ZZ$-split) short exact sequence
	\[0\to\op{Hom}_\ZZ(X,I_G\otimes_\ZZ A)\to\op{Hom}_\ZZ(X,\ZZ[G]\otimes_\ZZ A)\to\op{Hom}_\ZZ(X,I_G\otimes_\ZZ A)\to0\]
	which will induce a $\delta$ morphism between the correct modules. In fact, because $\op{Hom}_\ZZ(X,-)$ is a shiftable functor, the middle term here is induced, so the $\delta$ morphism
	\[\delta\colon\widehat H^0(G,\op{Hom}_\ZZ(X,A))\to\widehat H^1(G,\op{Hom}_\ZZ(X,I_G\otimes_\ZZ A))\]
	is an isomorphism.

	To finish, we claim that this $\delta$ morphism arises as a cup product. We simply show this by hand by tracking through the $\delta$ morphism. Given $[f]\in\widehat H^0(G,\op{Hom}_\ZZ(X,A))$ where $f\colon X\to A$ is a $G$-module homomorphism, we can pull this back to the $0$-chain $\widetilde f\colon X\to\ZZ[G]\otimes_\ZZ A$ by
	\[\widetilde f\colon x\mapsto 1\otimes f(x).\]
	Applying the differential, we get the $1$-cocycle $d\widetilde f\in B^1(G,\op{Hom}_\ZZ(X,\ZZ[G]\otimes_\ZZ A))$ defined by
	\begin{align*}
		(d\widetilde f)(g)(x) &= (g\widetilde f)(x)-\widetilde f(x) \\
		&= g\left(1\otimes f(g^{-1}x)\right)-(1\otimes f(x)) \\
		&= (g-1)\otimes f(x),
	\end{align*}
	which we know must be the $1$-cocycle $\delta([f])\in C^1(G,\op{Hom}_\ZZ(X,I_G\otimes_\ZZ A))$.

	Thus, we see that we should set $c\in\widehat H^1(G,I_G)$ to be represented by $g\mapsto(g-1)$. This will work as long as $g\mapsto(g-1)$ is a $1$-cocycle in $\widehat H^1(G,I_G)$. Well, take $X=A=\ZZ$ and $f=\id_\ZZ$ in the above argument so that $\delta(f)$ is exactly $g\mapsto(x\mapsto(g-1)\otimes x)$, which is $g\mapsto(g-1)$ after applying $\op{Hom}_\ZZ(\ZZ,I_G)\simeq I_G$.
\end{proof}
\begin{remark}
	Essentially the same proof will work when $\op{Hom}_\ZZ(X,-)$ is replaced by $X\otimes_\ZZ-$, or any composite of these. There isn't an analogue for arbitrary shiftable functors because, for example, there is no way obvious way to construct $\eta$ in general. Regardless, we will not need to work in these levels of generality.
\end{remark}
\begin{cor} \label{cor:cupdown}
	Let $G$ be a finite group. There exists $c\in\widehat H^1(G,I_G)$ such that, for any $G$-module $X$,
	\[c\cup-\colon\widehat H^i(G,\op{Hom}_\ZZ(X,\op{Hom}_\ZZ(I_G,-)))\Rightarrow\widehat H^{i+1}(G,\op{Hom}_\ZZ(X,-))\]
	is a natural isomorphism for any $i\in\ZZ$.
\end{cor}
\begin{proof}
	Similar to before, we are using the shifting pair $(\op{Hom}_\ZZ(X,\op{Hom}_\ZZ(I_G,-)),\op{Hom}_\ZZ(X,-),I_G,\eta)$, where
	\[\eta_A\colon I_G\otimes_\ZZ\op{Hom}_\ZZ(X,\op{Hom}_\ZZ(I_G,A))\Rightarrow\op{Hom}_\ZZ(X,-)\]
	is the canonical map sending $z\otimes f$ to $x\mapsto f(x)(z)$.

	Using \autoref{prop:dimshiftcupisos} and \autoref{lem:cuppingisnatural} again, it will suffice to find $c\in\widehat H^1(G,I_G)$ such that we have isomorphisms
	\[c\cup-\colon\widehat H^0(G,\op{Hom}_\ZZ(X,\op{Hom}_\ZZ(I_G,A)))\to\widehat H^1(G,\op{Hom}_\ZZ(X,A))\]
	for all $G$-modules $A$. This time around we use the ($\ZZ$-split) short exact sequence
	\[0\to\op{Hom}_\ZZ(X,A)\to\op{Hom}_\ZZ(X,\op{Hom}_\ZZ(\ZZ[G],A))\to\op{Hom}_\ZZ(X,\op{Hom}_\ZZ(I_G,A))\to0\]
	which will induce a boundary morphism
	\[\delta\colon\widehat H^0(G,\op{Hom}_\ZZ(X,\op{Hom}_\ZZ(I_G,A)))\to\widehat H^1(G,\op{Hom}_\ZZ(X,A)).\]
	In fact, this is an isomorphism because our middle term $\op{Hom}_\ZZ(X,\op{Hom}_\ZZ(\ZZ[G],A))$ is induced.

	We now show that $\delta$ is a cup product by hand. We start with some $[f]\in\widehat H^0(G,\op{Hom}_\ZZ(X,\op{Hom}_\ZZ(I_G,A)))$ where $f\colon X\to\op{Hom}_\ZZ(I_G,A)$ is a $G$-module homomorphism. This pulls back to the $0$-cochain
	\[\widetilde f\colon x\mapsto\big(z\mapsto f(x)(z-\varepsilon(z))\big).\]
	Applying the differential, we compute
	\begin{align*}
		(d\widetilde f)(g)(x)(z) &= (g\widetilde f-\widetilde f)(x)(z) \\
		&= (g\widetilde f)(x)(z)-\widetilde f(x)(z) \\
		&= \left(g\cdot\widetilde f\left(g^{-1}x\right)\right)(z)-\widetilde f(x)(z) \\
		&= g\cdot\widetilde f\left(g^{-1}x\right)\left(g^{-1}z\right)-\widetilde f(x)(z) \\
		&= g\cdot f\left(g^{-1}x\right)\left(g^{-1}z-\varepsilon(z)\right)-f(x)(z-\varepsilon(z)) \\
		&= g\cdot \left(g^{-1}f\left(x\right)\right)\left(g^{-1}z-\varepsilon(z)\right)-f(x)(z-\varepsilon(z)) \\
		&= f(x)\left(z-g\varepsilon(z)\right)-f(x)(z-\varepsilon(z)) \\
		&= \varepsilon(z)f(x)\left(1-g\right).
	\end{align*}
	Thus, this pulls back to the $1$-cocycle $g\mapsto(x\mapsto f(x)(1-g))$ in $\widehat H^1(G,\op{Hom}_\ZZ(X,A))$.
	
	In particular, we see that we should take $c$ represented by $g\mapsto(1-g)$, which will work as soon as we know that $g\mapsto(1-g)$ is a $1$-cocycle. Well, this is the negation of $g\mapsto(g-1)$ from the previous corollary. We close by remarking that we can actually take $c$ represented by $g\mapsto(g-1)$ because negating $c$ does not change the fact that the cup product gives an isomorphism.
\end{proof}
The point of the above two results is that have a somewhat general version of dimension-shifting granted by cup products. In fact, we see that we can use the same $c\in\widehat H^1(G,I_G)$ represented by $g\mapsto(g-1)$ for both shifting isomorphisms.

\subsection{Shifting Natural Transformations}
Observe that a natural transformation $F\Rightarrow F'$ of shiftable functors will induce natural transformations in cohomology
\[\widehat H^i(G,F-)\Rightarrow\widehat H^i(G,F'-)\]
It will turn out that, when $F=\op{Hom}_\ZZ(X,-)$ and $F'=\op{Hom}_\ZZ(X',-)$, we will be able to force all natural transformations in cohomology will come from natural transformations $F\Rightarrow F'$.

To begin, we show this result for $i=0$.
\begin{lemma} \label{lem:naturaltransiscupping}
	Let $G$ be a finite group, and let $X$ and $X'$ be $G$-modules. Suppose that, for given index $p\in\ZZ$, there is a natural transformation
	\[\Phi_\bullet\colon\widehat H^0(G,\op{Hom}_\ZZ(X,-))\Rightarrow\widehat H^p(G,\op{Hom}_\ZZ(X',-)).\]
	Then there exists $x\in\widehat H^p(G,\op{Hom}_\ZZ(X',X))$ such that $\Phi_\bullet=(x\cup-)$, where the cup product is induced by the shifting pair of \autoref{ex:shiftingpair}.
	% Then there exists a $G$-module homomorphism $\varphi\colon X'\to X$ such that $\Phi_A([f])=(-\circ\varphi)([f])$ for any $f\in\op{Hom}_{\ZZ[G]}(X,A)$.
\end{lemma}
\begin{proof}
	This is essentially the Yoneda lemma. As such, set $[x]\coloneqq\Phi_X([\id_X])$. The point is to fix some $G$-module $A$ and $[\overline f]\in\widehat H^0(G,\op{Hom}_\ZZ(X,A))$ in order to track through the commutativity of the following diagram.
	% https://q.uiver.app/?q=WzAsNCxbMCwwLCJcXHdpZGVoYXQgSF4wKEcsXFxvcHtIb219X1xcWlooWCxYKSkiXSxbMSwwLCJcXHdpZGVoYXQgSF4wKEcsXFxvcHtIb219X1xcWlooWCcsWCkpIl0sWzAsMSwiXFx3aWRlaGF0IEheMChHLFxcb3B7SG9tfV9cXFpaKFgsQSkpIl0sWzEsMSwiXFx3aWRlaGF0IEheMChHLFxcb3B7SG9tfV9cXFpaKFgnLEEpKSJdLFswLDEsIlxcUGhpX1giXSxbMCwyLCJmIiwyXSxbMSwzLCJmIiwyXSxbMiwzLCJcXFBoaV9BIl1d&macro_url=https%3A%2F%2Fraw.githubusercontent.com%2FdFoiler%2Fnotes%2Fmaster%2Fnir.tex
	\begin{equation}
		\begin{tikzcd}
			{\widehat H^0(G,\op{Hom}_\ZZ(X,X))} & {\widehat H^p(G,\op{Hom}_\ZZ(X',X))} \\
			{\widehat H^0(G,\op{Hom}_\ZZ(X,A))} & {\widehat H^p(G,\op{Hom}_\ZZ(X',A))}
			\arrow["{\Phi_X}", from=1-1, to=1-2]
			\arrow["\overline f"', from=1-1, to=2-1]
			\arrow["\overline f"', from=1-2, to=2-2]
			\arrow["{\Phi_A}", from=2-1, to=2-2]
		\end{tikzcd} \label{eq:cohomologicalyoneda}
	\end{equation}
	Because we will need to deal with the cup products with negative indices, we will use the standard resolution of \cite{cassels-frolich}. For example, we interpret $f\in[\overline f]\in\widehat H^0(G,\op{Hom}_\ZZ(X,A))$ as a constant function $f\in\op{Hom}_G(\ZZ[G],\op{Hom}_\ZZ(X,A))$ outputting $\overline f$, which means that $f(z)$ is the same $G$-module homomorphism for each $z\in\ZZ[G]$.

	As such, we can track the left arrow of \autoref{eq:cohomologicalyoneda} as
	\[\arraycolsep=1.4pt\begin{array}{cccc}
		\overline f\colon& \widehat H^0(G,\op{Hom}_\ZZ(X,X)) &\to& \widehat H^0(G,\op{Hom}_\ZZ(X,A)) \\
		& {[z\mapsto\id_X]} &\mapsto& {[z\mapsto f(z)\circ\id_X]=[\overline f]}.
	\end{array}\]
	So, along the bottom of \autoref{eq:cohomologicalyoneda}, we are evaluating $\Phi_A([\overline f])$.

	Along the top of \autoref{eq:cohomologicalyoneda}, we immediately send $[z\mapsto\id_X]$ to $\Phi_X([z\mapsto\id_X])=[x]$, so to finish the proof, we need to show that
	\[\overline f([x])\stackrel?=[x]\cup[\overline f],\]
	which will be enough by the commutativity of \autoref{eq:cohomologicalyoneda}. We have two similar cases to appropriately deal with the cup product.
	\begin{itemize}
		\item Suppose that $p\ge0$ so that we can interpret $x$ as an element of $\op{Hom}_{\ZZ[G]}\left(\ZZ[G^{p+1}],X\right)$, using the standard resolution. As such, we compute
		\[(x\cup f)(g_0,\ldots,g_p) = x(g_0,\ldots,g_p) \otimes f(g_p),\]
		where our output is in $\op{Hom}_\ZZ(X',X)\otimes_\ZZ\op{Hom}_\ZZ(X,A)$. Applying evaluation, the cup product is outputting
		\[(g_0,\ldots,g_p)\mapsto f(g_p)\circ x(g_0,\ldots,g_p)\]
		as our element of $\op{Hom}_{\ZZ[G]}(\ZZ[G^{p+1}],\op{Hom}_\ZZ(X',A))$. Indeed, this morphism represents $\overline f([x])$.
		\item Analogously, suppose that $p<0$ so that we interpret $x$ as an element of $\op{Hom}_{\ZZ[G]}\left(\op{Hom}_\ZZ(\ZZ[G]^p,\ZZ),X\right)$. To decrease headaches, we let $g^*\colon\ZZ[G]\to\ZZ$ denote the $G$-module homomorphism sending $g\mapsto1$ and other group elements to $0$. Then $p$-tuples $(g_1^*,\ldots,g_p^*)$ form a $\ZZ$-basis of $\op{Hom}_\ZZ(\ZZ[G]^p,\ZZ)$, so it's enough to specify
		\[(x\cup f)(g_1^*,\ldots,g_p^*) = x(g_1^*,\ldots,g_p^*)\otimes f(g_p),\]
		where the output is in $\op{Hom}_\ZZ(X',X)\otimes_\ZZ\op{Hom}_\ZZ(X,A)$. Applying evaluation, the cup product is outputting
		\[(g_1^*,\ldots,g_p^*)\mapsto f(g_p)\circ x(g_1^*,\ldots,g_p^*)\]
		as an element of $\op{Hom}_{\ZZ[G]}\left(\op{Hom}_\ZZ(\ZZ[G]^p,\ZZ),\op{Hom}_\ZZ(X',A)\right)$. Indeed, this represents $\overline f([x])$.
	\end{itemize}
	The above cases finish tracking through \autoref{eq:cohomologicalyoneda} and hence finish the proof.
	% This is essentially the Yoneda lemma. Choose $\varphi\colon X'\to X$ to represent $[\varphi]=\Phi_X([\id_X])$. Then tracking through the commutativity of
	% % https://q.uiver.app/?q=WzAsNCxbMCwwLCJcXHdpZGVoYXQgSF4wKEcsXFxvcHtIb219X1xcWlooWCxYKSkiXSxbMSwwLCJcXHdpZGVoYXQgSF4wKEcsXFxvcHtIb219X1xcWlooWCcsWCkpIl0sWzAsMSwiXFx3aWRlaGF0IEheMChHLFxcb3B7SG9tfV9cXFpaKFgsQSkpIl0sWzEsMSwiXFx3aWRlaGF0IEheMChHLFxcb3B7SG9tfV9cXFpaKFgnLEEpKSJdLFswLDEsIlxcUGhpX1giXSxbMCwyLCJmIiwyXSxbMSwzLCJmIiwyXSxbMiwzLCJcXFBoaV9BIl1d&macro_url=https%3A%2F%2Fraw.githubusercontent.com%2FdFoiler%2Fnotes%2Fmaster%2Fnir.tex
	% \[\begin{tikzcd}
	% 	{\widehat H^0(G,\op{Hom}_\ZZ(X,X))} & {\widehat H^0(G,\op{Hom}_\ZZ(X',X))} \\
	% 	{\widehat H^0(G,\op{Hom}_\ZZ(X,A))} & {\widehat H^0(G,\op{Hom}_\ZZ(X',A))}
	% 	\arrow["{\Phi_X}", from=1-1, to=1-2]
	% 	\arrow["f"', from=1-1, to=2-1]
	% 	\arrow["f"', from=1-2, to=2-2]
	% 	\arrow["{\Phi_A}", from=2-1, to=2-2]
	% \end{tikzcd}\]
	% reveals that
	% % https://q.uiver.app/?q=WzAsNCxbMCwwLCJbXFxpZF9YXSJdLFsxLDAsIltcXHZhcnBoaV0iXSxbMCwxLCJbZl0iXSxbMSwxLCJcXFBoaV9BKFtmXSk9W2ZcXGNpcmNcXHZhcnBoaV0iXSxbMCwxLCJcXFBoaV9YIiwwLHsic3R5bGUiOnsidGFpbCI6eyJuYW1lIjoibWFwcyB0byJ9fX1dLFswLDIsImYiLDIseyJzdHlsZSI6eyJ0YWlsIjp7Im5hbWUiOiJtYXBzIHRvIn19fV0sWzEsMywiZiIsMix7InN0eWxlIjp7InRhaWwiOnsibmFtZSI6Im1hcHMgdG8ifX19XSxbMiwzLCJcXFBoaV9BIiwwLHsic3R5bGUiOnsidGFpbCI6eyJuYW1lIjoibWFwcyB0byJ9fX1dXQ==&macro_url=https%3A%2F%2Fraw.githubusercontent.com%2FdFoiler%2Fnotes%2Fmaster%2Fnir.tex
	% \[\begin{tikzcd}
	% 	{[\id_X]} & {[\varphi]} \\
	% 	{[f]} & {\Phi_A([f])=[f\circ\varphi]}
	% 	\arrow["{\Phi_X}", maps to, from=1-1, to=1-2]
	% 	\arrow["f"', maps to, from=1-1, to=2-1]
	% 	\arrow["f"', maps to, from=1-2, to=2-2]
	% 	\arrow["{\Phi_A}", maps to, from=2-1, to=2-2]
	% \end{tikzcd}\]
	% after plugging in. To finish, we note that $\varphi([f])=[f\circ\varphi]$ under the induced map $\varphi\colon\op{Hom}_\ZZ(X,A)\to\op{Hom}_\ZZ(X',A)$. This finishes.
\end{proof}
The case of $p=0$ will be particularly interesting to us, so we note that the above proof gives it a more concrete description.
\begin{cor} \label{cor:cupiscomp}
	Let $G$ be a finite group, and let $X$ and $X'$ be $G$-modules. Then, given a $G$-module morphism $\varphi\colon X'\to X$, the maps $(-\circ\varphi)$ and $[\varphi]\cup-$ on
	\[\widehat H^i(G,\op{Hom}_\ZZ(X,-))\Rightarrow\widehat H^i(G,\op{Hom}_\ZZ(X',-))\]
	assemble into the same natural transformation.
\end{cor}
\begin{proof}
	This follows from unpacking the definitions.
	
	We already know that $[\varphi]\cup-$ is a natural transformation by \autoref{lem:cuppingisnatural}, so it suffices to show that the two maps agree on components. (Namely, naturality of $(-\circ\varphi)$ will immediately follow.) To see this, we note that the proof of \autoref{lem:naturaltransiscupping} above immediately computed for us that, given a $G$-module $A$, $[f]\in\widehat H^0(G,\op{Hom}_\ZZ(X,A))$ got sent to
	\[[f\circ\varphi]=f([\varphi])=[\varphi]\cup[f],\]
	which is what we wanted.
\end{proof}
We now get the main result by dimension-shifting.
\begin{prop} \label{prop:allnaturaltransarecups}
	Let $G$ be a finite group, and let $X$ and $X'$ be $G$-modules. Then, given indices $p,q\in\ZZ$, any natural transformation
	\[\Phi_\bullet^{(q)}\colon\widehat H^q(G,\op{Hom}_\ZZ(X,-))\Rightarrow\widehat H^{p+q}(G,\op{Hom}_\ZZ(X',-)),\]
	is $\Phi_\bullet^{(q)}=x\cup-$ for some $x\in\widehat H^p(G,\op{Hom}_\ZZ(X',X))$.
	% are natural transformations (respectively, isomorphisms)
	% \[\Phi_\bullet^{(i)}\colon\widehat H^i(G,F-)\Rightarrow\widehat H^i(G,F'-),\]
	% for all $i\in\ZZ$.
\end{prop}
% \begin{proof}
% 	This argument is by dimension-shifting the $p$ upwards and downwards. Namely, we show the conclusion of the statement by induction on $i$; for $i=p$, there is nothing to say. We will show how to induct downwards to $i\le p$ in detail, and inducting upwards is similar.

% 	To induct downwards, suppose the statement is true for $i+1$, and we show $i$, so fix a natural transformation
% 	\[\Phi_\bullet^{(i+1)}\colon\widehat H^i(G,\op{Hom}_\ZZ(X,-))\Rightarrow\widehat H^i(G,\op{Hom}_\ZZ(X',-)),\]
% 	which we would like to know arises as a cup product. The main idea is to use $\Phi_\bullet^{(i+1)}$ in order to construct $\Phi_\bullet^{(i)}$. Well, for any $G$-module $A$, we note that the two $\ZZ$-split short exact sequences
% 	\begin{equation}
% 		\arraycolsep=1.4pt\begin{array}{ccccccccc}
% 			1 &\to& F(I_G\otimes_\ZZ A) &\to& F(\ZZ[G]\otimes_\ZZ A) &\to& FA &\to& 1 \\
% 			1 &\to& F'(I_G\otimes_\ZZ A) &\to& F'(\ZZ[G]\otimes_\ZZ A) &\to& F'A &\to& 1
% 		\end{array} \label{eq:usualhomshiftingses}
% 	\end{equation}
% 	induce $\delta$ morphisms
% 	\[\arraycolsep=1.4pt\begin{array}{ccccccccc}
% 		\delta_{tA}\colon& \widehat H^i(G,FA) &\to& \widehat H^{i+1}(G,F'(I_G\otimes_\ZZ A)) \\
% 		\delta_{tA}'\colon& \widehat H^i(G,F'A) &\to& \widehat H^{i+1}(G,F'(I_G\otimes_\ZZ A))
% 	\end{array}\]
% 	which are in fact isomorphisms because the modules $\op{Hom}_\ZZ(X,\ZZ[G]\otimes_\ZZ A)$ and $\op{Hom}_\ZZ(X,\ZZ[G]\otimes_\ZZ A)$ are induced by \autoref{lem:hompreservesinduced}. As such, we have the diagram
% 	% https://q.uiver.app/?q=WzAsNCxbMCwwLCJcXHdpZGVoYXQgSF5pKEcsXFxvcHtIb219X1xcWlooWCxBKSkiXSxbMSwwLCJcXHdpZGVoYXQgSF57aSsxfShHLFxcb3B7SG9tfV9cXFpaKFgsSV9HXFxvdGltZXNfXFxaWiBBKSkiXSxbMCwxLCJcXHdpZGVoYXQgSF5pKEcsXFxvcHtIb219X1xcWlooWCcsQSkpIl0sWzEsMSwiXFx3aWRlaGF0IEhee2krMX0oRyxcXG9we0hvbX1fXFxaWihYJyxJX0dcXG90aW1lc19cXFpaIEEpKSJdLFswLDEsIlxcZGVsdGFfWCJdLFsxLDMsIlxcUGhpXnsoaSsxKX1fe0lfR1xcb3RpbWVzX1xcWlogQX0iXSxbMiwzLCJcXGRlbHRhX3tYJ30iXSxbMCwyLCIiLDAseyJzdHlsZSI6eyJib2R5Ijp7Im5hbWUiOiJkYXNoZWQifX19XV0=&macro_url=https%3A%2F%2Fraw.githubusercontent.com%2FdFoiler%2Fnotes%2Fmaster%2Fnir.tex
% 	\[\begin{tikzcd}
% 		{\widehat H^i(G,FA)} & {\widehat H^{i+1}(G,F(I_G\otimes_\ZZ A))} \\
% 		{\widehat H^i(G,F'A)} & {\widehat H^{i+1}(G,F'(I_G\otimes_\ZZ A))}
% 		\arrow["{\delta_{tA}}", from=1-1, to=1-2]
% 		\arrow["{\Phi^{(i+1)}_{I_G\otimes_\ZZ A}}", from=1-2, to=2-2]
% 		\arrow["{\delta_{tA}'}", from=2-1, to=2-2]
% 		\arrow[dashed, from=1-1, to=2-1]
% 	\end{tikzcd}\]
% 	where the horizontal arrows are isomorphisms. Thus, we induce a morphism
% 	\[\Phi_A^{(i)}\coloneqq(\delta_{tA}')^{-1}\circ\Phi^{(i+1)}_{I_G\otimes_\ZZ A}\circ\delta_{tA}.\]
% 	We claim that $\Phi_\bullet^{(i)}$ assembles into a natural transformation $\widehat H^i(G,F-)\Rightarrow\widehat H^i(G,F'-)$. For this, we must check naturality. Suppose that we have a morphism $f\colon A\to B$. This gives rise to the following diagram. 
% 	% https://q.uiver.app/?q=WzAsOCxbMSwwLCJcXHdpZGVoYXQgSF5pKEcsXFxvcHtIb219X1xcWlooWCxBKSkiXSxbMywwLCJcXHdpZGVoYXQgSF57aSsxfShHLFxcb3B7SG9tfV9cXFpaKFgsSV9HXFxvdGltZXNfXFxaWiBBKSkiXSxbMywyLCJcXHdpZGVoYXQgSF57aSsxfShHLFxcb3B7SG9tfV9cXFpaKFgnLElfR1xcb3RpbWVzX1xcWlogQSkpIl0sWzEsMiwiXFx3aWRlaGF0IEheaShHLFxcb3B7SG9tfV9cXFpaKFgnLEEpKSJdLFswLDEsIlxcd2lkZWhhdCBIXmkoRyxcXG9we0hvbX1fXFxaWihYLEIpKSJdLFsyLDEsIlxcd2lkZWhhdCBIXntpKzF9KEcsXFxvcHtIb219X1xcWlooWCxJX0dcXG90aW1lc19cXFpaIEIpKSJdLFswLDMsIlxcd2lkZWhhdCBIXmkoRyxcXG9we0hvbX1fXFxaWihYJyxCKSkiXSxbMiwzLCJcXHdpZGVoYXQgSF57aSsxfShHLFxcb3B7SG9tfV9cXFpaKFgnLElfR1xcb3RpbWVzX1xcWlogQikpIl0sWzAsMSwiXFxkZWx0YV97aEF9IiwwLHsibGFiZWxfcG9zaXRpb24iOjIwfV0sWzMsMiwiXFxkZWx0YV97aEF9JyIsMCx7ImxhYmVsX3Bvc2l0aW9uIjoyMH1dLFs0LDUsIlxcZGVsdGFfe2hCfSIsMSx7ImxhYmVsX3Bvc2l0aW9uIjoyMH1dLFs2LDcsIlxcZGVsdGFfe2hCfSciLDAseyJsYWJlbF9wb3NpdGlvbiI6MjB9XSxbMCwzLCJcXFBoaV9BXnsoaSl9IiwxLHsibGFiZWxfcG9zaXRpb24iOjIwLCJzdHlsZSI6eyJib2R5Ijp7Im5hbWUiOiJkYXNoZWQifX19XSxbNCw2LCJcXFBoaV9CXnsoaSl9IiwxLHsibGFiZWxfcG9zaXRpb24iOjIwLCJzdHlsZSI6eyJib2R5Ijp7Im5hbWUiOiJkYXNoZWQifX19XSxbMCw0LCJmIiwxXSxbMSw1LCJmIiwxXSxbMiw3LCJmIiwxXSxbNSw3LCJcXFBoaV9CXnsoaSsxKX0iLDEseyJsYWJlbF9wb3NpdGlvbiI6MjB9XSxbMSwyLCJcXFBoaV9BXnsoaSsxKX0iLDEseyJsYWJlbF9wb3NpdGlvbiI6MjB9XSxbMyw2LCJmIiwxXV0=&macro_url=https%3A%2F%2Fraw.githubusercontent.com%2FdFoiler%2Fnotes%2Fmaster%2Fnir.tex
% 	\[\begin{tikzcd}[column sep={3cm,between origins}]
% 		& {\widehat H^i(G,FA)} && {\widehat H^{i+1}(G,F(I_G\otimes_\ZZ A))} \\
% 		{\widehat H^i(G,FB)} && {\widehat H^{i+1}(G,F(I_G\otimes_\ZZ B))} \\
% 		& {\widehat H^i(G,F'A)} && {\widehat H^{i+1}(G,F'(I_G\otimes_\ZZ A))} \\
% 		{\widehat H^i(G,F'B)} && {\widehat H^{i+1}(G,F'(I_G\otimes_\ZZ B))}
% 		\arrow["{\delta_{tA}}"{description, pos=0.2}, from=1-2, to=1-4]
% 		\arrow["{\delta_{tA}'}"{description, pos=0.2}, from=3-2, to=3-4]
% 		\arrow["{\delta_{tB}}"{description, pos=0.2}, from=2-1, to=2-3]
% 		\arrow["{\delta_{tB}'}"{description, pos=0.2}, from=4-1, to=4-3]
% 		\arrow["{\Phi_A^{(i)}}"{description, pos=0.2}, dashed, from=1-2, to=3-2]
% 		\arrow["{\Phi_B^{(i)}}"{description, pos=0.2}, dashed, from=2-1, to=4-1]
% 		\arrow["Ff"{description}, from=1-2, to=2-1]
% 		\arrow["Ff"{description}, from=1-4, to=2-3]
% 		\arrow["F'f"{description}, from=3-4, to=4-3]
% 		\arrow["{\Phi_{I_G\otimes_\ZZ B}^{(i+1)}}"{description, pos=0.2}, from=2-3, to=4-3]
% 		\arrow["{\Phi_{I_G\otimes_\ZZ A}^{(i+1)}}"{description, pos=0.2}, from=1-4, to=3-4]
% 		\arrow["F'f"{description}, from=3-2, to=4-1]
% 	\end{tikzcd}\]
% 	We want to show that the left face commutes. For this, we note that all the horizontal arrows are isomorphisms (they're the $\delta$s from before), so it suffices to show that the rest of the cube commutes.
% 	\begin{itemize}
% 		\item The top face commutes by functoriality of $\delta$ morphsims applied to the following morphism of short exact sequences.
% 		% https://q.uiver.app/?q=WzAsMTAsWzAsMCwiMCJdLFsxLDAsIlxcb3B7SG9tfV9cXFpaKFgsSV9HXFxvdGltZXNfXFxaWiBBKSJdLFsyLDAsIlxcb3B7SG9tfV9cXFpaKFgsXFxaWltHXVxcb3RpbWVzX1xcWlogQSkiXSxbMywwLCJcXG9we0hvbX1fXFxaWihYLEEpIl0sWzQsMCwiMCJdLFsxLDEsIlxcb3B7SG9tfV9cXFpaKFgsSV9HXFxvdGltZXNfXFxaWiBCKSJdLFsyLDEsIlxcb3B7SG9tfV9cXFpaKFgsXFxaWltHXVxcb3RpbWVzX1xcWlogQikiXSxbMywxLCJcXG9we0hvbX1fXFxaWihYLEIpIl0sWzQsMSwiMCJdLFswLDEsIjAiXSxbMCwxXSxbOSw1XSxbMSwyXSxbMiwzXSxbMyw0XSxbNSw2XSxbNiw3XSxbNyw4XSxbMSw1LCJmIl0sWzIsNiwiZiJdLFszLDcsImYiXV0=&macro_url=https%3A%2F%2Fraw.githubusercontent.com%2FdFoiler%2Fnotes%2Fmaster%2Fnir.tex
% 		\[\begin{tikzcd}
% 			0 & {F(I_G\otimes_\ZZ A)} & {F(\ZZ[G]\otimes_\ZZ A)} & {FA} & 0 \\
% 			0 & {F(I_G\otimes_\ZZ B)} & {F(\ZZ[G]\otimes_\ZZ B)} & {FB} & 0
% 			\arrow[from=1-1, to=1-2]
% 			\arrow[from=2-1, to=2-2]
% 			\arrow[from=1-2, to=1-3]
% 			\arrow[from=1-3, to=1-4]
% 			\arrow[from=1-4, to=1-5]
% 			\arrow[from=2-2, to=2-3]
% 			\arrow[from=2-3, to=2-4]
% 			\arrow[from=2-4, to=2-5]
% 			\arrow["Ff", from=1-2, to=2-2]
% 			\arrow["Ff", from=1-3, to=2-3]
% 			\arrow["Ff", from=1-4, to=2-4]
% 		\end{tikzcd}\]
% 		The bottom face commutes for the same reason, replacing $F$s with $F'$s in the above morphism of short exact sequences.
% 		\item The front and back faces commute by definition of the morphisms $\Phi_\bullet^{(i)}$.
% 		\item The right face commutes by naturality of $\Phi^{(i+1)}_\bullet$ applied to the induced morphism $f\colon I_G\otimes_\ZZ A\to I_G\otimes_\ZZ B$.
% 	\end{itemize}
% 	The above commutativity checks complete the proof that $\Phi_\bullet^{(i)}$ makes a natural transformation. To finish, we note that, if $\Phi_\bullet^{(i+1)}$ is a natural isomorphism, then $\Phi_\bullet^{(i)}$ is a natural isomorphism as well by its construction. This completes the induction downwards.

% 	We will not give detail for the induction upwards from $i-1$ to $i$, except to say that the short exact sequences \autoref{eq:usualhomshiftingses} are replaced with the following.
% 	\[\arraycolsep=1.4pt\begin{array}{ccccccccc}
% 		1 &\to& FA &\to& F(\op{Hom}_\ZZ(\ZZ[G],A)) &\to& F(\op{Hom}_\ZZ(I_G,A)) &\to& 1 \\
% 		1 &\to& F'A &\to& F'(\op{Hom}_\ZZ(\ZZ[G],A)) &\to& F'(\op{Hom}_\ZZ(I_G,A)) &\to& 1
% 	\end{array}\]
% 	The rest of the approach essentially goes through verbatim, constructing $\Phi_\bullet^{(i)}$ from a given $\Phi_\bullet^{(i-1)}$.
% \end{proof}
\begin{proof}
	This argument is by dimension-shifting the $q$ upwards and downwards. Namely, we show the conclusion of the statement by induction on $i$; for $i=q$, there is nothing to say. We will show how to induct upwards to $i\ge q$ in detail, and inducting downwards is similar. For brevity, we set $F\coloneqq\op{Hom}_\ZZ(X,-)$ and $F'\coloneqq\op{Hom}_\ZZ(X',-)$.

	To induct upwards, suppose the statement is true for $i$, and we show $i+1$, so fix a natural transformation
	\[\Phi_\bullet^{(i+1)}\colon\widehat H^{i+1}(G,F-)\Rightarrow\widehat H^{p+i+1}(G,F'-),\]
	which we would like to know arises as $x\cup-$ for some $x\in\widehat H^p(G,\op{Hom}_\ZZ(X',X))$. The main idea is to use $\Phi_\bullet^{(i+1)}$ in order to construct $\Phi_\bullet^{(i)}$. Well, using \autoref{cor:cupup}, we have some $c\in\widehat H^1(G,I_G)$ given by $g\mapsto(g-1)$ yielding the following isomorphisms for any $G$-module $A$.
	% Well, for any $G$-module $A$, we note that the two $\ZZ$-split short exact sequences
	% \begin{equation}
	% 	\arraycolsep=1.4pt\begin{array}{ccccccccc}
	% 		1 &\to& F(I_G\otimes_\ZZ A) &\to& F(\ZZ[G]\otimes_\ZZ A) &\to& FA &\to& 1 \\
	% 		1 &\to& F'(I_G\otimes_\ZZ A) &\to& F'(\ZZ[G]\otimes_\ZZ A) &\to& F'A &\to& 1
	% 	\end{array} \label{eq:usualhomshiftingses}
	% \end{equation}
	% induce $\delta$ morphisms
	\[\arraycolsep=1.4pt\begin{array}{ccccccccc}
		(c\cup-)_d\colon& \widehat H^i(G,FA) &\to& \widehat H^{i+1}(G,F(I_G\otimes_\ZZ A)) \\
		(c\cup-)'_d\colon& \widehat H^{p+i}(G,F'A) &\to& \widehat H^{p+i+1}(G,F'(I_G\otimes_\ZZ A))
	\end{array}\]
	As such, we have the diagram
	% https://q.uiver.app/?q=WzAsNCxbMCwwLCJcXHdpZGVoYXQgSF5pKEcsXFxvcHtIb219X1xcWlooWCxBKSkiXSxbMSwwLCJcXHdpZGVoYXQgSF57aSsxfShHLFxcb3B7SG9tfV9cXFpaKFgsSV9HXFxvdGltZXNfXFxaWiBBKSkiXSxbMCwxLCJcXHdpZGVoYXQgSF5pKEcsXFxvcHtIb219X1xcWlooWCcsQSkpIl0sWzEsMSwiXFx3aWRlaGF0IEhee2krMX0oRyxcXG9we0hvbX1fXFxaWihYJyxJX0dcXG90aW1lc19cXFpaIEEpKSJdLFswLDEsIlxcZGVsdGFfWCJdLFsxLDMsIlxcUGhpXnsoaSsxKX1fe0lfR1xcb3RpbWVzX1xcWlogQX0iXSxbMiwzLCJcXGRlbHRhX3tYJ30iXSxbMCwyLCIiLDAseyJzdHlsZSI6eyJib2R5Ijp7Im5hbWUiOiJkYXNoZWQifX19XV0=&macro_url=https%3A%2F%2Fraw.githubusercontent.com%2FdFoiler%2Fnotes%2Fmaster%2Fnir.tex
	\[\begin{tikzcd}
		{\widehat H^i(G,FA)} & {\widehat H^{i+1}(G,F(I_G\otimes_\ZZ A))} \\
		{\widehat H^{p+i}(G,F'A)} & {\widehat H^{p+i+1}(G,F'(I_G\otimes_\ZZ A))}
		\arrow["{(c\cup-)_d}", from=1-1, to=1-2]
		\arrow["{\Phi^{(i+1)}_{I_G\otimes_\ZZ A}}", from=1-2, to=2-2]
		\arrow["{(c\cup-)'_d}", from=2-1, to=2-2]
		\arrow[dashed, from=1-1, to=2-1]
	\end{tikzcd}\]
	where the horizontal arrows are isomorphisms. Thus, we induce a morphism
	\[\Phi_A^{(i)}\coloneqq((c\cup-)'_d)^{-1}\circ\Phi^{(i+1)}_{I_G\otimes_\ZZ A}\circ(c\cup-)_d.\]
	Note that $\Phi_\bullet^{(i)}$ is the composition of natural transformations (the cup product is a natural transformation by construction) and therefore is a natural transformation.
	% We claim that $\Phi_\bullet^{(i)}$ assembles into a natural transformation $\widehat H^i(G,F-)\Rightarrow\widehat H^i(G,F'-)$. For this, we must check naturality. Suppose that we have a morphism $f\colon A\to B$. This gives rise to the following diagram.
	% % https://q.uiver.app/?q=WzAsOCxbMSwwLCJcXHdpZGVoYXQgSF5pKEcsXFxvcHtIb219X1xcWlooWCxBKSkiXSxbMywwLCJcXHdpZGVoYXQgSF57aSsxfShHLFxcb3B7SG9tfV9cXFpaKFgsSV9HXFxvdGltZXNfXFxaWiBBKSkiXSxbMywyLCJcXHdpZGVoYXQgSF57aSsxfShHLFxcb3B7SG9tfV9cXFpaKFgnLElfR1xcb3RpbWVzX1xcWlogQSkpIl0sWzEsMiwiXFx3aWRlaGF0IEheaShHLFxcb3B7SG9tfV9cXFpaKFgnLEEpKSJdLFswLDEsIlxcd2lkZWhhdCBIXmkoRyxcXG9we0hvbX1fXFxaWihYLEIpKSJdLFsyLDEsIlxcd2lkZWhhdCBIXntpKzF9KEcsXFxvcHtIb219X1xcWlooWCxJX0dcXG90aW1lc19cXFpaIEIpKSJdLFswLDMsIlxcd2lkZWhhdCBIXmkoRyxcXG9we0hvbX1fXFxaWihYJyxCKSkiXSxbMiwzLCJcXHdpZGVoYXQgSF57aSsxfShHLFxcb3B7SG9tfV9cXFpaKFgnLElfR1xcb3RpbWVzX1xcWlogQikpIl0sWzAsMSwiXFxkZWx0YV97aEF9IiwwLHsibGFiZWxfcG9zaXRpb24iOjIwfV0sWzMsMiwiXFxkZWx0YV97aEF9JyIsMCx7ImxhYmVsX3Bvc2l0aW9uIjoyMH1dLFs0LDUsIlxcZGVsdGFfe2hCfSIsMSx7ImxhYmVsX3Bvc2l0aW9uIjoyMH1dLFs2LDcsIlxcZGVsdGFfe2hCfSciLDAseyJsYWJlbF9wb3NpdGlvbiI6MjB9XSxbMCwzLCJcXFBoaV9BXnsoaSl9IiwxLHsibGFiZWxfcG9zaXRpb24iOjIwLCJzdHlsZSI6eyJib2R5Ijp7Im5hbWUiOiJkYXNoZWQifX19XSxbNCw2LCJcXFBoaV9CXnsoaSl9IiwxLHsibGFiZWxfcG9zaXRpb24iOjIwLCJzdHlsZSI6eyJib2R5Ijp7Im5hbWUiOiJkYXNoZWQifX19XSxbMCw0LCJmIiwxXSxbMSw1LCJmIiwxXSxbMiw3LCJmIiwxXSxbNSw3LCJcXFBoaV9CXnsoaSsxKX0iLDEseyJsYWJlbF9wb3NpdGlvbiI6MjB9XSxbMSwyLCJcXFBoaV9BXnsoaSsxKX0iLDEseyJsYWJlbF9wb3NpdGlvbiI6MjB9XSxbMyw2LCJmIiwxXV0=&macro_url=https%3A%2F%2Fraw.githubusercontent.com%2FdFoiler%2Fnotes%2Fmaster%2Fnir.tex
	% \[\begin{tikzcd}[column sep={3cm,between origins}]
	% 	& {\widehat H^i(G,FA)} && {\widehat H^{i+1}(G,F(I_G\otimes_\ZZ A))} \\
	% 	{\widehat H^i(G,FB)} && {\widehat H^{i+1}(G,F(I_G\otimes_\ZZ B))} \\
	% 	& {\widehat H^i(G,F'A)} && {\widehat H^{i+1}(G,F'(I_G\otimes_\ZZ A))} \\
	% 	{\widehat H^i(G,F'B)} && {\widehat H^{i+1}(G,F'(I_G\otimes_\ZZ B))}
	% 	\arrow["{\delta_{tA}}"{description, pos=0.2}, from=1-2, to=1-4]
	% 	\arrow["{\delta_{tA}'}"{description, pos=0.2}, from=3-2, to=3-4]
	% 	\arrow["{\delta_{tB}}"{description, pos=0.2}, from=2-1, to=2-3]
	% 	\arrow["{\delta_{tB}'}"{description, pos=0.2}, from=4-1, to=4-3]
	% 	\arrow["{\Phi_A^{(i)}}"{description, pos=0.2}, dashed, from=1-2, to=3-2]
	% 	\arrow["{\Phi_B^{(i)}}"{description, pos=0.2}, dashed, from=2-1, to=4-1]
	% 	\arrow["Ff"{description}, from=1-2, to=2-1]
	% 	\arrow["Ff"{description}, from=1-4, to=2-3]
	% 	\arrow["F'f"{description}, from=3-4, to=4-3]
	% 	\arrow["{\Phi_{I_G\otimes_\ZZ B}^{(i+1)}}"{description, pos=0.2}, from=2-3, to=4-3]
	% 	\arrow["{\Phi_{I_G\otimes_\ZZ A}^{(i+1)}}"{description, pos=0.2}, from=1-4, to=3-4]
	% 	\arrow["F'f"{description}, from=3-2, to=4-1]
	% \end{tikzcd}\]
	% We want to show that the left face commutes. For this, we note that all the horizontal arrows are isomorphisms (they're the $\delta$s from before), so it suffices to show that the rest of the cube commutes.
	% \begin{itemize}
	% 	\item The top face commutes by functoriality of $\delta$ morphsims applied to the following morphism of short exact sequences.
	% 	% https://q.uiver.app/?q=WzAsMTAsWzAsMCwiMCJdLFsxLDAsIlxcb3B7SG9tfV9cXFpaKFgsSV9HXFxvdGltZXNfXFxaWiBBKSJdLFsyLDAsIlxcb3B7SG9tfV9cXFpaKFgsXFxaWltHXVxcb3RpbWVzX1xcWlogQSkiXSxbMywwLCJcXG9we0hvbX1fXFxaWihYLEEpIl0sWzQsMCwiMCJdLFsxLDEsIlxcb3B7SG9tfV9cXFpaKFgsSV9HXFxvdGltZXNfXFxaWiBCKSJdLFsyLDEsIlxcb3B7SG9tfV9cXFpaKFgsXFxaWltHXVxcb3RpbWVzX1xcWlogQikiXSxbMywxLCJcXG9we0hvbX1fXFxaWihYLEIpIl0sWzQsMSwiMCJdLFswLDEsIjAiXSxbMCwxXSxbOSw1XSxbMSwyXSxbMiwzXSxbMyw0XSxbNSw2XSxbNiw3XSxbNyw4XSxbMSw1LCJmIl0sWzIsNiwiZiJdLFszLDcsImYiXV0=&macro_url=https%3A%2F%2Fraw.githubusercontent.com%2FdFoiler%2Fnotes%2Fmaster%2Fnir.tex
	% 	\[\begin{tikzcd}
	% 		0 & {F(I_G\otimes_\ZZ A)} & {F(\ZZ[G]\otimes_\ZZ A)} & {FA} & 0 \\
	% 		0 & {F(I_G\otimes_\ZZ B)} & {F(\ZZ[G]\otimes_\ZZ B)} & {FB} & 0
	% 		\arrow[from=1-1, to=1-2]
	% 		\arrow[from=2-1, to=2-2]
	% 		\arrow[from=1-2, to=1-3]
	% 		\arrow[from=1-3, to=1-4]
	% 		\arrow[from=1-4, to=1-5]
	% 		\arrow[from=2-2, to=2-3]
	% 		\arrow[from=2-3, to=2-4]
	% 		\arrow[from=2-4, to=2-5]
	% 		\arrow["Ff", from=1-2, to=2-2]
	% 		\arrow["Ff", from=1-3, to=2-3]
	% 		\arrow["Ff", from=1-4, to=2-4]
	% 	\end{tikzcd}\]
	% 	The bottom face commutes for the same reason, replacing $F$s with $F'$s in the above morphism of short exact sequences.
	% 	\item The front and back faces commute by definition of the morphisms $\Phi_\bullet^{(i)}$.
	% 	\item The right face commutes by naturality of $\Phi^{(i+1)}_\bullet$ applied to the induced morphism $f\colon I_G\otimes_\ZZ A\to I_G\otimes_\ZZ B$.
	% \end{itemize}
	% The above commutativity checks complete the proof that $\Phi_\bullet^{(i)}$ makes a natural transformation. To finish, we note that, if $\Phi_\bullet^{(i+1)}$ is a natural isomorphism, then $\Phi_\bullet^{(i)}$ is a natural isomorphism as well by its construction. This completes the induction downwards.

	Thus, the inductive hypothesis now tells us that $\Phi_\bullet^{(i)}=(x\cup-)$ for some $x\in\widehat H^p(G,\op{Hom}_\ZZ(X',X))$. We now need to turn this around on $\Phi_\bullet^{(i+1)}$, which essentially means we need to shift back in the other direction. As such, we use \autoref{cor:cupdown} to give the following isomorphisms for any $G$-module $A$.
	\[\arraycolsep=1.4pt\begin{array}{ccccccccc}
		(c\cup-)_u\colon&\widehat H^i(G,F(\op{Hom}_\ZZ(I_G,A)))\to\widehat H^{i+1}(G,FA) \\
		(c\cup-)_u'\colon&\widehat H^{p+i}(G,F(\op{Hom}_\ZZ(I_G,A)))\to\widehat H^{p+i+1}(G,FA)
	\end{array}\]
	Now, to deal with $\Phi_\bullet^{(i+1)}$, we note that associativity and commutativity of cup products implies $\left((-1)^px\cup-\right)$ can be used to make the right arrow in the diagram
	% https://q.uiver.app/?q=WzAsNCxbMCwwLCJcXHdpZGVoYXQgSF5pKEcsXFxvcHtIb219X1xcWlooWCxcXG9we0hvbX1fXFxaWihJX0csQSkpKSJdLFsxLDAsIlxcd2lkZWhhdCBIXntpKzF9KEcsXFxvcHtIb219X1xcWlooWCxBKSkiXSxbMCwxLCJcXHdpZGVoYXQgSF5pKEcsXFxvcHtIb219X1xcWlooWCcsXFxvcHtIb219X1xcWlooSV9HLEEpKSkiXSxbMSwxLCJcXHdpZGVoYXQgSF57aSsxfShHLFxcb3B7SG9tfV9cXFpaKFgnLEEpKSJdLFswLDEsIihjXFxjdXAtKV91Il0sWzIsMywiKGNcXGN1cC0pX3UnIl0sWzAsMiwiKFxcdmFycGhpXFxjaXJjLSkiLDJdLFsxLDMsIiIsMSx7InN0eWxlIjp7ImJvZHkiOnsibmFtZSI6ImRhc2hlZCJ9fX1dXQ==&macro_url=https%3A%2F%2Fraw.githubusercontent.com%2FdFoiler%2Fnotes%2Fmaster%2Fnir.tex
	\[\begin{tikzcd}
		{\widehat H^i(G,F(\op{Hom}_\ZZ(I_G,A)))} & {\widehat H^{i+1}(G,FA)} \\
		{\widehat H^{p+i}(G,F'(\op{Hom}_\ZZ(I_G,A)))} & {\widehat H^{p+i+1}(G,F'A)}
		\arrow["{(c\cup-)_u}", from=1-1, to=1-2]
		\arrow["{(c\cup-)_u'}", from=2-1, to=2-2]
		\arrow["{(x\cup-)}"', from=1-1, to=2-1]
		\arrow[dashed, from=1-2, to=2-2]
	\end{tikzcd}\]
	commute; technically, we ought to expand out this diagram to use the associativity and commutativity of the cup product for this diagram to commute, but we won't bother.
	
	Now, this right arrow is unique because the horizontal arrows are isomorphisms, so we will be done if we can show that we can place $\Phi_A^{(i+1)}$ in the right arrow to also make the diagram commute. For this, we draw the following very large diagram.
	% https://q.uiver.app/?q=WzAsNixbMCwwLCJcXHdpZGVoYXQgSF5pKEcsRihcXG9we0hvbX1fXFxaWihJX0csQSkpKSJdLFsxLDEsIlxcd2lkZWhhdCBIXntpKzF9KEcsRihJX0dcXG90aW1lc19cXFpaXFxvcHtIb219X1xcWlooSV9HLEEpKSkiXSxbMiwwLCJcXHdpZGVoYXQgSF57aSsxfShHLEZBKSJdLFswLDMsIlxcd2lkZWhhdCBIXmkoRyxGJyhcXG9we0hvbX1fXFxaWihJX0csQSkpKSJdLFsxLDQsIlxcd2lkZWhhdCBIXntpKzF9KEcsRicoSV9HXFxvdGltZXNfXFxaWlxcb3B7SG9tfV9cXFpaKElfRyxBKSkpIl0sWzIsMywiXFx3aWRlaGF0IEhee2krMX0oRyxGJ0EpIl0sWzAsMiwiKGNcXGN1cC0pX3UiLDEseyJsYWJlbF9wb3NpdGlvbiI6MjB9XSxbMCwxLCIoY1xcY3VwLSlfZCIsMV0sWzEsMiwiZiIsMV0sWzAsMywiKC1cXGNpcmNcXHZhcnBoaSkiLDEseyJsYWJlbF9wb3NpdGlvbiI6MzB9XSxbMSw0LCJcXFBoaV57KGkrMSl9X3tJX0dcXG90aW1lc19cXFpaXFxvcHtIb219X1xcWlooSV9HLEEpfSIsMSx7ImxhYmVsX3Bvc2l0aW9uIjozMH1dLFsyLDUsIlxcUGhpXnsoaSsxKX1fQSIsMSx7ImxhYmVsX3Bvc2l0aW9uIjozMH1dLFszLDUsIihjXFxjdXAtKV91JyIsMSx7ImxhYmVsX3Bvc2l0aW9uIjoyMH1dLFszLDQsIihjXFxjdXAtKV9kJyIsMV0sWzQsNSwiZiIsMV1d&macro_url=https%3A%2F%2Fraw.githubusercontent.com%2FdFoiler%2Fnotes%2Fmaster%2Fnir.tex
	\[\begin{tikzcd}
		{\widehat H^i(G,F(\op{Hom}_\ZZ(I_G,A)))} && {\widehat H^{i+1}(G,FA)} \\
		& {\widehat H^{i+1}(G,F(I_G\otimes_\ZZ\op{Hom}_\ZZ(I_G,A)))} \\
		\\
		{\widehat H^{p+i}(G,F'(\op{Hom}_\ZZ(I_G,A)))} && {\widehat H^{p+i+1}(G,F'A)} \\
		& {\widehat H^{p+i+1}(G,F'(I_G\otimes_\ZZ\op{Hom}_\ZZ(I_G,A)))}
		\arrow["{(c\cup-)_u}"{description, pos=0.2}, from=1-1, to=1-3]
		\arrow["{(c\cup-)_d}"{description}, from=1-1, to=2-2]
		\arrow["f"{description}, from=2-2, to=1-3]
		\arrow["{(x\cup-)}"{description, pos=0.3}, from=1-1, to=4-1]
		\arrow["{\Phi^{(i+1)}_{I_G\otimes_\ZZ\op{Hom}_\ZZ(I_G,A)}}"{description, pos=0.3}, from=2-2, to=5-2]
		\arrow["{\Phi^{(i+1)}_A}"{description, pos=0.3}, from=1-3, to=4-3]
		\arrow["{(c\cup-)_u'}"{description, pos=0.2}, from=4-1, to=4-3]
		\arrow["{(c\cup-)_d'}"{description}, from=4-1, to=5-2]
		\arrow["f"{description}, from=5-2, to=4-3]
	\end{tikzcd}\]
	Here, the $f$ maps are induced by the evaluation map
	\[f\colon I_G\otimes_\ZZ\op{Hom}_\ZZ(I_G,A)\to A.\]
	We want the outer rectangle to commute, for which it suffices to show that each parallelogram and the small top and bottom triangles to commute.
	\begin{itemize}
		\item The left parallelogram commutes by definition of $\Phi_A^{(i)}$.
		\item The right parallelogram commutes by naturality of $\Phi^{(i+1)}_\bullet$.
		\item Showing that the bottom triangle commutes will be analogous to showing that the top triangle commutes, so we will only show the top. Unwinding \autoref{cor:cupup} and \autoref{cor:cupdown}, we see that this triangle is actually induced by the following diagram.
		% https://q.uiver.app/?q=WzAsNCxbMCwwLCJcXHdpZGVoYXQgSF5pKEcsRihcXG9we0hvbX1fXFxaWihJX0csQSkpKSJdLFsxLDAsIlxcd2lkZWhhdCBIXmkoRyxJX0dcXG90aW1lc19cXFpaIEYoXFxvcHtIb219X1xcWlooSV9HLEEpKSkiXSxbMSwxLCJcXHdpZGVoYXQgSF5pKEcsRihJX0dcXG90aW1lc19cXFpaXFxvcHtIb219X1xcWlooSV9HLEEpKSkiXSxbMiwwLCJcXHdpZGVoYXQgSF5pKEcsRkEpIl0sWzAsMSwiY1xcY3VwLSJdLFsyLDMsImYiLDJdLFsxLDIsIlxcZXRhX2QiLDJdLFsxLDMsIlxcZXRhX3UiXV0=&macro_url=https%3A%2F%2Fraw.githubusercontent.com%2FdFoiler%2Fnotes%2Fmaster%2Fnir.tex
		\[\begin{tikzcd}
			{\widehat H^i(G,F(\op{Hom}_\ZZ(I_G,A)))} & {\widehat H^{i+1}(G,I_G\otimes_\ZZ F(\op{Hom}_\ZZ(I_G,A)))} & {\widehat H^{i+1}(G,FA)} \\
			& {\widehat H^{i+1}(G,F(I_G\otimes_\ZZ\op{Hom}_\ZZ(I_G,A)))}
			\arrow["{c\cup-}", from=1-1, to=1-2]
			\arrow["f"', from=2-2, to=1-3]
			\arrow["{\eta_d}"', from=1-2, to=2-2]
			\arrow["{\eta_u}", from=1-2, to=1-3]
		\end{tikzcd}\]
		Here, $\eta_u\colon I_G\otimes_\ZZ\op{Hom}_\ZZ(X,\op{Hom}_\ZZ(I_G,A))\to\op{Hom}_\ZZ(X,A)$ behaves as
		\[\eta_u\colon z\otimes f\mapsto\big(x\mapsto f(z)(x)\big),\]
		and $\eta_d\colon I_G\otimes_\ZZ\op{Hom}_\ZZ(X,\op{Hom}_\ZZ(I_G,A))\to \op{Hom}_\ZZ(X,I_G\otimes_\ZZ\op{Hom}_\ZZ(I_G,A))$ behaves as
		\[\eta_d\colon z\otimes f\mapsto\big(x\mapsto z\otimes f(x)\big).\]
		Now, to check our commutativity, it suffices to show that the triangle
		% https://q.uiver.app/?q=WzAsMyxbMCwwLCJJX0dcXG90aW1lc19cXFpaXFxvcHtIb219X1xcWlooWCxcXG9we0hvbX1fXFxaWihJX0csQSkpIl0sWzAsMSwiXFxvcHtIb219X1xcWlooWCxJX0dcXG90aW1lc19cXFpaXFxvcHtIb219X1xcWlooSV9HLEEpKSJdLFsxLDAsIlxcb3B7SG9tfV9cXFpaKFgsQSkiXSxbMSwyLCJmIiwyXSxbMCwxLCJcXGV0YV9kIiwyXSxbMCwyLCJcXGV0YV91Il1d&macro_url=https%3A%2F%2Fraw.githubusercontent.com%2FdFoiler%2Fnotes%2Fmaster%2Fnir.tex
		\[\begin{tikzcd}
			{I_G\otimes_\ZZ\op{Hom}_\ZZ(X,\op{Hom}_\ZZ(I_G,A))} & {\op{Hom}_\ZZ(X,A)} \\
			{\op{Hom}_\ZZ(X,I_G\otimes_\ZZ\op{Hom}_\ZZ(I_G,A))}
			\arrow["f"', from=2-1, to=1-2]
			\arrow["{\eta_d}"', from=1-1, to=2-1]
			\arrow["{\eta_u}", from=1-1, to=1-2]
		\end{tikzcd}\]
		commutes. Well, we can simply track through the diagram as follows.
		% https://q.uiver.app/?q=WzAsMyxbMCwwLCJ6XFxvdGltZXMgZiJdLFswLDEsIlxcYmlnKHhcXG1hcHN0byB6XFxvdGltZXMgZih4KVxcYmlnKSJdLFsxLDAsIlxcYmlnKHhcXG1hcHN0byBmKHgpKHopXFxiaWcpIl0sWzAsMiwiIiwyLHsic3R5bGUiOnsidGFpbCI6eyJuYW1lIjoibWFwcyB0byJ9fX1dLFswLDEsIiIsMCx7InN0eWxlIjp7InRhaWwiOnsibmFtZSI6Im1hcHMgdG8ifX19XSxbMSwyLCIiLDAseyJzdHlsZSI6eyJ0YWlsIjp7Im5hbWUiOiJtYXBzIHRvIn19fV1d&macro_url=https%3A%2F%2Fraw.githubusercontent.com%2FdFoiler%2Fnotes%2Fmaster%2Fnir.tex
		\[\begin{tikzcd}
			{z\otimes f} & {\big(x\mapsto f(x)(z)\big)} \\
			{\big(x\mapsto z\otimes f(x)\big)}
			\arrow[maps to, from=1-1, to=1-2]
			\arrow[maps to, from=1-1, to=2-1]
			\arrow[maps to, from=2-1, to=1-2]
		\end{tikzcd}\]
	\end{itemize}
	The above commutativity checks finish the induction upwards.

	We will not give detail for the induction downwards from $i-1$ to $i$, except to say that we reverse the applications of \autoref{cor:cupup} and \autoref{cor:cupdown}. The rest of the approach essentially goes through verbatim, constructing $\Phi_\bullet^{(i)}$ from a given $\Phi_\bullet^{(i-1)}$, applying the inducting hypothesis to $\Phi_\bullet^{(i)}$, and then finishing by shifting back to $\Phi_\bullet^{(i-1)}$.
\end{proof}
\begin{remark}
	Essentially the same proof can show that, for any pair of shiftable functors $F,F'\colon\mathrm{Mod}_G\to\mathrm{Mod}_G$, a natural transformation (respectively, isomorphism)
	\[\Phi_\bullet^{(i)}\colon\widehat H^i(G,F-)\Rightarrow\widehat H^i(G,F'-),\]
	at $i=p$ induces natural transformations (respectively, isomorphisms) at all $i\in\ZZ$. Instead of using \autoref{cor:cupup} and \autoref{cor:cupdown}, we must instead dimension-shifting using the usual short exact sequences.
\end{remark}
\begin{cor}
	Let $G$ be a finite group, and let $X$ and $X'$ be $G$-modules. Then, given indices $q\in\ZZ$, any natural transformation
	\[\Phi_\bullet^{(q)}\colon\widehat H^q(G,\op{Hom}_\ZZ(X,-))\Rightarrow\widehat H^{q}(G,\op{Hom}_\ZZ(X',-)),\]
	is $\Phi_\bullet^{(q)}=(-\circ\varphi)$ for some $G$-module morphism $\varphi\colon X'\to X$.
\end{cor}
\begin{proof}
	\autoref{prop:allnaturaltransarecups} tells us that the natural transformation takes the form $[\varphi]\cup-$ for some $G$-module morphism $\varphi\colon X'\to X$. Then $[\varphi]\cup-$ is simply $(-\circ\varphi)$ by \autoref{cor:cupiscomp}.
\end{proof}

\subsection{Cohomological Equivalence}
It might be the case that ``many'' different shiftable functors give the same cohomology groups. Because we are mostly interested in the case of $\op{Hom}_\ZZ(X,-)$, we now have the tools to talk fairly concretely about what this means. We have the following definition.
\begin{definition}
	Let $G$ be a finite group. We say that two $G$-modules $X,X'$ are \textit{cohomologically equivalent} if and only if there exist morphisms $[\varphi]\in\widehat H^0(G,\op{Hom}_\ZZ(X',X))$ and $[\varphi']\in\widehat H^0(G,\op{Hom}_\ZZ(X,X'))$ such that
	\[[\varphi\circ\varphi']=[\id_X]\in\widehat H^0(G,\op{Hom}_\ZZ(X,X))\qquad\text{and}\qquad[\varphi'\circ\varphi]=[\id_{X'}]\in\widehat H^0(G,\op{Hom}_\ZZ(X',X')).\]
\end{definition}
\begin{example}
	All induced modules $X$ are cohomologically equivalent to $0$. To see this, we set $\varphi\colon 0\to X$ and $\varphi'\colon X\to0$ equal to the zero maps (which are our only options). Then note that $\op{Hom}_\ZZ(X,X)$ is induced by \autoref{lem:hompreservesinduced} and $\op{Hom}_\ZZ(0,0)=0$, so
	\[\widehat H^0(G,\op{Hom}_\ZZ(X,X))=\widehat H^0(G,\op{Hom}_\ZZ(X',X'))=0,\]
	making the checks on $\varphi$ and $\varphi'$ both trivial.
\end{example}
More concretely, $X$ and $X'$ are cohomologically equivalent if and only if we have two $G$-module morphisms $\varphi\colon X'\to X$ and $\varphi'\colon X\to X'$ and two $\ZZ$-module morphisms $f\colon X\to X$ and $f'\colon X'\to X'$ such that
\[\varphi\circ\varphi'=\id_X+N_Gf\qquad\text{and}\qquad\varphi'\circ\varphi=\id_{X'}+N_Gf'.\]
As a quick sanity check that this is a reasonable notion of equivalence of modules, we have the following.
\begin{lemma} \label{lem:monoidequiv}
	Let $G$ be a finite group. If the $G$-modules $X$ and $X'$ are equivalent and $Y$ and $Y'$ are equivalent, then $X\oplus Y$ is equivalent to $X'\oplus Y'$.
\end{lemma}
\begin{proof}
	We are promised the morphisms
	\begin{itemize}
		\item $\varphi\colon X'\to X$ and $\varphi'\colon X\to X'$ (as morphisms of $G$-modules),
		\item $f\colon X\to X$ and $f'\colon X'\to X'$ (as morphisms of $\ZZ$-modules),
		\item $\psi\colon Y'\to Y$ and $\psi'\colon Y\to Y'$ (as morphisms of $G$-modules),
		\item $g\colon Y\to Y$ and $g'\colon Y'\to Y'$ (as morphisms of $\ZZ$-modules),
	\end{itemize}
	which are required to satisfy
	\begin{align*}
		\varphi\circ\varphi'={\id_X}+N_Gf\qquad&\text{and}\qquad\varphi'\circ\varphi={\id_{X'}}+N_Gf', \\
		\psi\circ\psi'={\id_Y}+N_Gg\qquad&\text{and}\qquad\psi'\circ\psi={\id_{Y'}}+N_Gg'.
	\end{align*}
	Summing everywhere, we get the $G$-module homomorphisms $\varphi\oplus\psi\colon X\oplus Y\to X'\oplus Y'$ and $\varphi'\oplus\psi'\colon X'\oplus Y'\to X\oplus Y$ satisfying
	\begin{align*}
		(\varphi\oplus\psi)\circ(\varphi'\oplus\psi') &= (\varphi\circ\varphi')\oplus(\psi\circ\psi') \\
		&= ({\id_X}+N_Gf)\oplus({\id_Y}+N_Gg) \\
		&= {\id_X}\oplus{\id_Y}+N_G(f\oplus g).
	\end{align*}
	The other check is analogous, switching primed and unprimed variables.
\end{proof}
We now show that this notion of equivalence correctly translates to shiftable functors.
\begin{proposition} \label{prop:cohomologicaldef}
	Let $G$ be a finite group, and let $X$ and $X'$ be $G$-modules. Then $X$ and $X'$ are cohomologically equivalent if and only if there is a natural isomorphism
	\[\Phi_\bullet\colon\widehat H^0(G,\op{Hom}_\ZZ(X,-))\Rightarrow\widehat H^0(G,\op{Hom}_\ZZ(X',-)).\]
\end{proposition}
\begin{proof}
	In the forward direction, suppose $X$ and $X'$ are cohomologically equivalent so that we have $[\varphi]\in\widehat H^0(G,\op{Hom}_\ZZ(X',X))$ and $[\varphi']\in\widehat H^0(G,\op{Hom}_\ZZ(X,X'))$ such that
	\[[\varphi]\cup[\varphi']=[\varphi\circ\varphi']=[\id_X]\qquad\text{and}\qquad[\varphi']\cup[\varphi]=[\varphi'\circ\varphi]=[\id_{X'}],\]
	where we are using the canonical evaluation maps for the cup products. Now, we note that, for any $G$-module $A$, we have inverse morphisms
	\begin{equation}
		\arraycolsep=1.4pt\begin{array}{ccc}
			\widehat H^0(G,\op{Hom}_\ZZ(X,A)) &\simeq& \widehat H^0(G,\op{Hom}_\ZZ(X,A)) \\
			{[f]} &\mapsto& {[f\circ\varphi]} \\
			{[f'\circ\varphi']} & \mapsfrom & {[f']}.
		\end{array} \label{eq:makenaturaliso}
	\end{equation}
	Indeed, these are mutually inverse because
	\[[f\circ\varphi\circ\varphi']=[f].\]
	To finish, we note that the isomorphisms \autoref{eq:makenaturaliso} assemble into a natural isomorphism by \autoref{lem:cuppingisnatural} and \autoref{cor:cupiscomp}.

	We now show the backwards direction. Suppose we have a natural isomorphism $\Phi_\bullet$. Then \autoref{lem:naturaltransiscupping} promises us $[\varphi]\in\widehat H^0(G,\op{Hom}_\ZZ(X',X))$ and $[\varphi']\in\widehat H^0(G,\op{Hom}_\ZZ(X,X'))$ such that the morphisms
	\[\arraycolsep=1.4pt\begin{array}{cccc}
		\Phi_\bullet\colon& \widehat H^0(G,\op{Hom}_\ZZ(X,-)) &\simeq& \widehat H^0(G,\op{Hom}_\ZZ(X,-)) \\
		&{[f]} &\mapsto& {[f\circ\varphi]} \\
		&{[f'\circ\varphi']} & \mapsfrom & {[f']}
	\end{array}\]
	are mutually inverse. In particular, we see that
	\[[\id_X]=[{\id_X}\circ\varphi\circ\varphi']=[\varphi\circ\varphi'],\]
	so $[\varphi\circ\varphi']=[\id_X]$. Swapping primed and unprimed variables, we see $[\varphi'\circ\varphi]=[\id_{X'}]$ as well.
\end{proof}
\begin{remark}
	The above result makes it fairly clear that cohomological equivalence actually makes an equivalence relation. In particular, we can invert and compose natural isomorphisms, which gives symmetry and transitivity of cohomological equivalence respectively.
\end{remark}
This alternate definition also provides us with a way to multiply.
\begin{cor}
	Let $G$ be a finite group. If $X$ and $X'$ are equivalent and $Y$ and $Y'$ are equivalent, then $X\otimes_\ZZ X'$ is equivalent to $Y\otimes_\ZZ Y'$.
\end{cor}
\begin{proof}
	We are granted natural isomorphisms as follows.
	\[\arraycolsep=1.4pt\begin{array}{cccc}
		\Phi_\bullet\colon &\widehat H^0(G,\op{Hom}_\ZZ(X,-))&\Rightarrow&\widehat H^0(G,\op{Hom}_\ZZ(X',-)) \\
		\Psi_\bullet\colon &\widehat H^0(G,\op{Hom}_\ZZ(Y,-))&\Rightarrow&\widehat H^0(G,\op{Hom}_\ZZ(Y',-))
	\end{array}\]
	Now, repeatedly using the hom--tensor adjunction, we can chain together natural isomorphisms
	\begin{align*}
		\widehat H^0(G,\op{Hom}_\ZZ(X\otimes_\ZZ Y,-)) &\simeq \widehat H^0(G,\op{Hom}_\ZZ(X,\op{Hom}_\ZZ(Y,-))) \\
		&\stackrel{\Phi\op{Hom}(Y,-)}\simeq \widehat H^0(G,\op{Hom}_\ZZ(X',\op{Hom}_\ZZ(Y,-))) \\
		&\simeq \widehat H^0(G,\op{Hom}_\ZZ(X'\otimes_\ZZ Y,-)) \\
		&\simeq \widehat H^0(G,\op{Hom}_\ZZ(Y\otimes_\ZZ X',-)) \\
		&\simeq \widehat H^0(G,\op{Hom}_\ZZ(Y,\op{Hom}_\ZZ( X',-))) \\
		&\stackrel{\Psi\op{Hom}_\ZZ(X',-)}\simeq \widehat H^0(G,\op{Hom}_\ZZ(Y',\op{Hom}_\ZZ(X',-))) \\
		&\simeq \widehat H^0(G,\op{Hom}_\ZZ(Y'\otimes_\ZZ X',-)) \\
		&\simeq \widehat H^0(G,\op{Hom}_\ZZ(X'\otimes_\ZZ Y',-)),
	\end{align*}
	which is what we wanted.
\end{proof}
One might hope that we can get more information by using indices away from $0$, but in fact we cannot.
\begin{proposition} \label{prop:betterocohomdef}
	Let $G$ be a finite group, and let $X$ and $X'$ be $G$-modules. Then the following are equivalent.
	\begin{listalph}
		\item $X$ and $X'$ are cohomologically equivalent.
		\item For some $p\in\ZZ$, there is a natural isomorphism
		\[\Phi_\bullet^{(p)}\colon\widehat H^p(G,\op{Hom}_\ZZ(X,-))\Rightarrow\widehat H^p(G,\op{Hom}_\ZZ(X',-)).\]
		\item There is a $G$-module homomorphism $\varphi\colon X'\to X$ such that the induced maps
		\[(-\circ\varphi)\colon\widehat H^i(G,\op{Hom}_\ZZ(X,-))\Rightarrow\widehat H^i(G,\op{Hom}_\ZZ(X',-))\]
		are natural isomorphisms for all $i\in\ZZ$.
	\end{listalph}
\end{proposition}
\begin{proof}
	Note that (a) implies (b) by taking $p=0$ and applying \autoref{prop:cohomologicaldef}. Also, (c) implies (a) by taking $i=0$ and again applying \autoref{prop:cohomologicaldef}. Lastly, to show (b) implies (c), we note that \autoref{prop:allnaturaltransarecups} promises us $\varphi\colon X'\to X$ such that
	\[\Phi_\bullet^{(p)}=(-\circ\varphi).\]
	We would like to use \autoref{prop:dimshiftcupisos}. Let our shifting pair be $(\op{Hom}_\ZZ(X,-),\op{Hom}_\ZZ(X',-),\op{Hom}_\ZZ(X',X),\eta)$, where $\eta_\bullet$ is the canonical pre-composition map
	\[\eta_\bullet\colon\op{Hom}_\ZZ(X',X)\otimes_\ZZ\op{Hom}_\ZZ(X,-)\to\op{Hom}_\ZZ(X',-).\]
	Then we take $p=p$ and $q=0$ and $c=[\varphi]$ as above so that the cup-product natural transformation
	\[[\varphi]\cup-\colon\widehat H^i(G,\op{Hom}_\ZZ(X,-))\Rightarrow\widehat H^i(G,\op{Hom}_\ZZ(X',-))\]
	is simply induced by $(-\circ\varphi)$ for any $i\in\ZZ$ by \autoref{cor:cupiscomp}. So we are given that this is a natural isomorphism at $i=p$, so \autoref{prop:dimshiftcupisos} gives us this isomorphism at all $i\in\ZZ$, which proves (c).
	% Thus, (b) implies (c) is the interesting part. Observe that we already know from \autoref{lem:dimshiftnaturaltrans} that we have a natural isomorphism
	% \[\Phi^{(0)}_\bullet\colon\widehat H^0(G,\op{Hom}_\ZZ(X,-))\Rightarrow\widehat H^0(G,\op{Hom}_\ZZ(X',-)).\]
	% Now, applying \autoref{lem:naturaltransiscupping}, we see that this must be induced by some $\varphi\colon X'\to X$, so we know that we have a natural isomorphism
	% \begin{equation}
	% 	(-\circ\varphi)\colon\widehat H^i(G,\op{Hom}_\ZZ(X,A))\to\widehat H^i(G,\op{Hom}_\ZZ(X',A)) \label{eq:cupshifting}
	% \end{equation}
	% is an isomorphism for all $G$-modules $A$ at $i=0$. To prove (c), we will shift this isomorphism up and down from $0$. To shift downwards, we suppose that we have an isomorphism always at $i$, and we show that we have an isomorphism always at $i-1$. Well, for any $G$-module $A$, we note the morphism of ($\ZZ$-split) short exact sequences
	% % https://q.uiver.app/?q=WzAsMTAsWzAsMSwiMCJdLFsxLDEsIlxcb3B7SG9tfV9cXFpaKFgnLElfR1xcb3RpbWVzX1xcWlogQSkiXSxbMiwxLCJcXG9we0hvbX1fXFxaWihYJyxcXFpaW0ddXFxvdGltZXNfXFxaWiBBKSJdLFszLDEsIlxcb3B7SG9tfV9cXFpaKFgnLEEpIl0sWzQsMSwiMCJdLFsxLDAsIlxcb3B7SG9tfV9cXFpaKFgsSV9HXFxvdGltZXNfXFxaWiBBKSJdLFsyLDAsIlxcb3B7SG9tfV9cXFpaKFgsXFxaWltHXVxcb3RpbWVzX1xcWlogQSkiXSxbMywwLCJcXG9we0hvbX1fXFxaWihYLEEpIl0sWzAsMCwiMCJdLFs0LDAsIjAiXSxbOCw1XSxbNSw2XSxbNiw3XSxbNyw5XSxbMCwxXSxbMSwyXSxbMiwzXSxbMyw0XSxbNSwxLCIoLVxcY2lyY1xcdmFycGhpKSIsMl0sWzYsMiwiKC1cXGNpcmNcXHZhcnBoaSkiLDJdLFs3LDMsIigtXFxjaXJjXFx2YXJwaGkpIiwyXV0=&macro_url=https%3A%2F%2Fraw.githubusercontent.com%2FdFoiler%2Fnotes%2Fmaster%2Fnir.tex
	% \begin{equation}
	% 	\begin{tikzcd}[column sep=10pt]
	% 		0 & {\op{Hom}_\ZZ(X,I_G\otimes_\ZZ A)} & {\op{Hom}_\ZZ(X,\ZZ[G]\otimes_\ZZ A)} & {\op{Hom}_\ZZ(X,A)} & 0 \\
	% 		0 & {\op{Hom}_\ZZ(X',I_G\otimes_\ZZ A)} & {\op{Hom}_\ZZ(X',\ZZ[G]\otimes_\ZZ A)} & {\op{Hom}_\ZZ(X',A)} & 0
	% 		\arrow[from=1-1, to=1-2]
	% 		\arrow[from=1-2, to=1-3]
	% 		\arrow[from=1-3, to=1-4]
	% 		\arrow[from=1-4, to=1-5]
	% 		\arrow[from=2-1, to=2-2]
	% 		\arrow[from=2-2, to=2-3]
	% 		\arrow[from=2-3, to=2-4]
	% 		\arrow[from=2-4, to=2-5]
	% 		\arrow["{(-\circ\varphi)}"', from=1-2, to=2-2]
	% 		\arrow["{(-\circ\varphi)}"', from=1-3, to=2-3]
	% 		\arrow["{(-\circ\varphi)}"', from=1-4, to=2-4]
	% 	\end{tikzcd} \label{eq:somesesgoingdown}
	% \end{equation}
	% whose boundary morphisms induce the following commutative square.
	% % https://q.uiver.app/?q=WzAsNCxbMSwwLCJcXHdpZGVoYXQgSF57aS0xfShHLFxcb3B7SG9tfV9cXFpaKFgnLEEpKSJdLFswLDAsIlxcd2lkZWhhdCBIXntpLTF9KEcsXFxvcHtIb219X1xcWlooWCxBKSkiXSxbMCwxLCJcXHdpZGVoYXQgSF4wKEcsXFxvcHtIb219X1xcWlooWCxJX0dcXG90aW1lc19cXFpaIEEpIl0sWzEsMSwiXFx3aWRlaGF0IEheMChHLFxcb3B7SG9tfV9cXFpaKFgnLElfR1xcb3RpbWVzX1xcWlogQSkiXSxbMSwwLCJcXHZhcnBoaVxcY3VwLSJdLFsyLDMsIlxcdmFycGhpXFxjdXAtIl0sWzEsMiwiXFxkZWx0YSIsMl0sWzAsMywiXFxkZWx0YSIsMl1d&macro_url=https%3A%2F%2Fraw.githubusercontent.com%2FdFoiler%2Fnotes%2Fmaster%2Fnir.tex
	% \[\begin{tikzcd}
	% 	{\widehat H^{i-1}(G,\op{Hom}_\ZZ(X,A))} & {\widehat H^{i-1}(G,\op{Hom}_\ZZ(X',A))} \\
	% 	{\widehat H^i(G,\op{Hom}_\ZZ(X,I_G\otimes_\ZZ A))} & {\widehat H^i(G,\op{Hom}_\ZZ(X',I_G\otimes_\ZZ A))}
	% 	\arrow["{(-\circ\varphi)}", from=1-1, to=1-2]
	% 	\arrow["{(-\circ\varphi)}", from=2-1, to=2-2]
	% 	\arrow["\delta"', from=1-1, to=2-1]
	% 	\arrow["\delta"', from=1-2, to=2-2]
	% \end{tikzcd}\]
	% In particular, the inductive hypothesis tells us that the bottom row is an isomorphism, and the fact that both middle terms of \autoref{eq:somesesgoingdown} are induced by \autoref{lem:hompreservesinduced} implies that the $\delta$s on either side are also isomorphisms. So the top row is an isomorphism, finishing.
	%
	% Similarly, to shift upwards, we suppose that \autoref{eq:cupshifting} is always an isomorphism at $i$, and we show that we have an isomorphism always at $i+1$. Well, for any $G$-module $A$, we note the ($\ZZ$-split) short exact sequences
	% % https://q.uiver.app/?q=WzAsMTAsWzAsMSwiMCJdLFsxLDEsIlxcb3B7SG9tfV9cXFpaKFgnLEEpIl0sWzIsMSwiXFxvcHtIb219X1xcWlooWCcsXFxvcHtIb219X1xcWlooXFxaWltHXSxBKSkiXSxbMywxLCJcXG9we0hvbX1fXFxaWihYJyxcXG9we0hvbX1fXFxaWihJX0csQSkpIl0sWzQsMSwiMCJdLFsxLDAsIlxcb3B7SG9tfV9cXFpaKFgsQSkiXSxbMiwwLCJcXG9we0hvbX1fXFxaWihYLFxcb3B7SG9tfV9cXFpaKFxcWlpbR10sQSkpIl0sWzMsMCwiXFxvcHtIb219X1xcWlooWCxcXG9we0hvbX1fXFxaWihJX0csQSkpIl0sWzAsMCwiMCJdLFs0LDAsIjAiXSxbOCw1XSxbNSw2XSxbNiw3XSxbNyw5XSxbMCwxXSxbMSwyXSxbMiwzXSxbMyw0XSxbNSwxLCIoLVxcY2lyY1xcdmFycGhpKSIsMl0sWzYsMiwiKC1cXGNpcmNcXHZhcnBoaSkiLDJdLFs3LDMsIigtXFxjaXJjXFx2YXJwaGkpIiwyXV0=&macro_url=https%3A%2F%2Fraw.githubusercontent.com%2FdFoiler%2Fnotes%2Fmaster%2Fnir.tex
	% \begin{equation}
	% 	\begin{tikzcd}[column sep=10pt]
	% 		0 & {\op{Hom}_\ZZ(X,A)} & {\op{Hom}_\ZZ(X,\op{Hom}_\ZZ(\ZZ[G],A))} & {\op{Hom}_\ZZ(X,\op{Hom}_\ZZ(I_G,A))} & 0 \\
	% 		0 & {\op{Hom}_\ZZ(X',A)} & {\op{Hom}_\ZZ(X',\op{Hom}_\ZZ(\ZZ[G],A))} & {\op{Hom}_\ZZ(X',\op{Hom}_\ZZ(I_G,A))} & 0
	% 		\arrow[from=1-1, to=1-2]
	% 		\arrow[from=1-2, to=1-3]
	% 		\arrow[from=1-3, to=1-4]
	% 		\arrow[from=1-4, to=1-5]
	% 		\arrow[from=2-1, to=2-2]
	% 		\arrow[from=2-2, to=2-3]
	% 		\arrow[from=2-3, to=2-4]
	% 		\arrow[from=2-4, to=2-5]
	% 		\arrow["{(-\circ\varphi)}"', from=1-2, to=2-2]
	% 		\arrow["{(-\circ\varphi)}"', from=1-3, to=2-3]
	% 		\arrow["{(-\circ\varphi)}"', from=1-4, to=2-4]
	% 	\end{tikzcd} \label{eq:somesesgoingup}
	% \end{equation}
	% whose boundary morphisms induce the following commutative square.
	% % https://q.uiver.app/?q=WzAsNCxbMSwwLCJcXHdpZGVoYXQgSF57aX0oRyxcXG9we0hvbX1fXFxaWihYJyxcXG9we0hvbX1fXFxaWihJX0csQSkpKSJdLFswLDAsIlxcd2lkZWhhdCBIXntpfShHLFxcb3B7SG9tfV9cXFpaKFgsXFxvcHtIb219X1xcWlooSV9HLEEpKSkiXSxbMCwxLCJcXHdpZGVoYXQgSF57aSsxfShHLFxcb3B7SG9tfV9cXFpaKFgsQSkiXSxbMSwxLCJcXHdpZGVoYXQgSF57aSsxfShHLFxcb3B7SG9tfV9cXFpaKFgnLEEpIl0sWzEsMCwiXFx2YXJwaGlcXGN1cC0iXSxbMiwzLCJcXHZhcnBoaVxcY3VwLSJdLFsxLDIsIlxcZGVsdGEiLDJdLFswLDMsIlxcZGVsdGEiLDJdXQ==&macro_url=https%3A%2F%2Fraw.githubusercontent.com%2FdFoiler%2Fnotes%2Fmaster%2Fnir.tex
	% \[\begin{tikzcd}
	% 	{\widehat H^{i}(G,\op{Hom}_\ZZ(X,\op{Hom}_\ZZ(I_G,A)))} & {\widehat H^{i}(G,\op{Hom}_\ZZ(X',\op{Hom}_\ZZ(I_G,A)))} \\
	% 	{\widehat H^{i+1}(G,\op{Hom}_\ZZ(X,A))} & {\widehat H^{i+1}(G,\op{Hom}_\ZZ(X',A))}
	% 	\arrow["{(-\circ\varphi)}", from=1-1, to=1-2]
	% 	\arrow["{(-\circ\varphi)}", from=2-1, to=2-2]
	% 	\arrow["\delta"', from=1-1, to=2-1]
	% 	\arrow["\delta"', from=1-2, to=2-2]
	% \end{tikzcd}\]
	% This time around, the top row is an isomorphism by the inductive hypothesis, and the left and row arrows are isomorphisms because the middle terms of \autoref{eq:somesesgoingup} are induced by \autoref{lem:hompreservesinduced}. So the bottom row is an isomorphism as well, finishing.
\end{proof}

\subsection{Encoding Modules}
Lastly, we arrive at the application we care about: encoding cohomology.
\begin{definition}
	Let $G$ be a finite group and $p\in\ZZ$ be an index. Then a $G$-module $X$ is a \textit{$p$-encoding $G$-module} if and only if there is a natural isomorphism
	\[\Phi_\bullet\colon\widehat H^i(G,\op{Hom}_\ZZ(X,-))\Rightarrow\widehat H^{i+p}(G,-)\]
	for some $i\in\ZZ$.
\end{definition}
Cohomological equivalence is exactly what we need to talk about uniqueness.
\begin{cor} \label{cor:encodingmodules}
	Let $G$ be a finite group, and let $p,q\in\ZZ$ be indices. Then the set of $G$-module $X$ with a natural isomorphism
	\[\Phi_\bullet\colon\widehat H^p(G,\op{Hom}_\ZZ(X,-))\Rightarrow\widehat H^q(G,-)\]
	make up exactly one cohomological equivalence class.
\end{cor}
\begin{proof}
	Fix some $G$-module $X$ with such a natural isomorphism
	\[\Psi_\bullet\colon\widehat H^p(G,\op{Hom}_\ZZ(X,-))\Rightarrow\widehat H^q(G,-).\]
	We would like to show that a $G$-module $X$ has a natural isomorphism $\Phi_\bullet$ between the same functors if and only if $X$ and $X'$ are cohomologically equivalent.

	% Noting that the cup product is natural in both arguments (see, for example, \autoref{}), we see that the combination of \autoref{thm:yesitisacocycle} and \autoref{prop:alternativetupleclass} tells us that we have a natural isomorphism
	% \[\Psi_\bullet\colon\widehat H^0(G,\op{Hom}_\ZZ(X,-))\Rightarrow\widehat H^2(G,-).\]
	% We now proceed with the proof.
	If $X$ and $X'$ are cohomologically equivalent, then we can compose the promised natural isomorphism of \autoref{prop:betterocohomdef} (c) with $\Psi_\bullet$, giving a natural isomorphism
	\[\widehat H^p(G,\op{Hom}_\ZZ(X',-))\Rightarrow\widehat H^p(G,\op{Hom}_\ZZ(X,-))\stackrel{\Psi_\bullet}\Rightarrow\widehat H^q(G,-).\]
	In the other direction, if we have a natural isomorphism
	\[\Phi_\bullet\colon\widehat H^p(G,\op{Hom}_\ZZ(X',-))\Rightarrow\widehat H^q(G,-),\]
	then we can compose with $\Psi_\bullet^{-1}$ to build a natural isomorphism
	\[\widehat H^p(G,\op{Hom}_\ZZ(X',-))\stackrel{\Phi_\bullet}\Rightarrow\widehat H^q(G,-)\stackrel{\Psi^{-1}_\bullet}\Rightarrow\widehat H^p(G,\op{Hom}_\ZZ(X,-)),\]
	from which it follows that $X$ and $X'$ are cohomologically equivalent by \autoref{prop:cohomologicaldef} (b).
\end{proof}
% \begin{remark}
% 	In fact, we can see that the natural isomorphism
% 	\[\Phi_\bullet\colon\widehat H^0(G,\op{Hom}_\ZZ(X',-))\Rightarrow\widehat H^2(G,-)\]
% 	must be a cup-product map with an element in $\widehat H^2(G,X')$. Namely, we note that $\Phi_\bullet$ is equal to the composite
% 	\[\widehat H^0(G,\op{Hom}_\ZZ(X',-))\stackrel{\Phi_\bullet}\Rightarrow\widehat H^2(G,-)\stackrel{\Psi^{-1}_\bullet}\Rightarrow\widehat H^0(G,\op{Hom}_\ZZ(X,-))\stackrel{\Psi_\bullet}\Rightarrow\widehat H^2(G,-).\]
% 	However, $\Psi_\bullet$ is a cup-product map with an element in $\widehat H^2(G,X)$ by its construction, and
% 	\[\widehat H^0(G,\op{Hom}_\ZZ(X',-))\stackrel{\Phi_\bullet}\Rightarrow\widehat H^2(G,-)\stackrel{\Psi^{-1}_\bullet}\Rightarrow\widehat H^0(G,\op{Hom}_\ZZ(X,-))\]
% 	is a cup-product map with an element in $\widehat H^0(G,\op{Hom}_\ZZ(X,X'))$ by \autoref{lem:naturaltransiscupping}. Composing our cup-product maps makes a cup-product map with an element in $\widehat H^2(G,X')$.
% \end{remark}
% \begin{example}
% 	Suppose that a $G$-module $X$ has a natural isomorphism
% 	\[\widehat H^0(\]
% 	If $M$ is any induced module, then we know $M$ is cohomologically equivalent to $0$. So \autoref{lem:monoidequiv} reassures us that $X\oplus M$ is cohomologically equivalent to $X\oplus0\simeq X$.
% \end{example}
\begin{example} \label{ex:igisencoding}
	Take $q\ge p$. Dimension-shifting iteratively with the short exact sequence
	\[0\to I_G\otimes_\ZZ A\to\ZZ[G]\otimes_\ZZ A\to A\to0\]
	shows that
	\[\widehat H^q(G,A)\simeq\widehat H^p\left(G,\op{Hom}_\ZZ(I_G^{\otimes (q-p)},A)\right),\]
	and in fact these isomorphisms are natural by the functoriality of boundary morphisms. So the equivalence class of \autoref{cor:encodingmodules} is represented by $I_G^{\otimes(q-p)}$.
\end{example}
\begin{example} \label{ex:notalltorsionfree}
	Not all $p$-encoding modules are $\ZZ$-torsion-free. For example, if $M$ is a $p$-encoding module, and $A$ is induced, then $M\oplus A$ is cohomologically equivalent to $M$, so $M\oplus A$ is a $p$-encoding module. However, not all induced modules $A$ are $\ZZ$-torsion-free.
\end{example}
% The above two examples should give a feeling for why this uniqueness problem is difficult.
In fact, akin to the classification of natural transformations from \autoref{prop:allnaturaltransarecups}, we can show that these encoding maps must be cup products.
% \begin{lemma}
% 	Let $G$ be a finite group, and let $p\ge0$ be an index. Then there exists some $x_p\in\widehat H^p\left(G,I_G^{\otimes p}\right)$ such that
% 	\[x_p\cup-\colon\widehat H^0\left(G,\op{Hom}_\ZZ(I_G^{\otimes p},-)\right)\Rightarrow H^p(G,-)\]
% 	is a natural isomorphism.
% \end{lemma}
% \begin{proof}
% 	We quickly remark that (akin to \autoref{}), by \autoref{} and functoriality of $\widehat H^p(G,-)$, any $x_p\in\widehat H^p(G,I_G^{\otimes p})$ will at least create a natural transformation. So the main point is to make the cup-product into an isomorphism. For $p=0$, we take $x_0\coloneqq[1]\in\widehat H^p(G,\ZZ)$ so that
% 	\[[1]\cup-\colon\widehat H^0(G,\op{Hom}_\ZZ(\ZZ,-))\Rightarrow\widehat H^0(G,-)\]
% 	is simply the map taking a $0$-cocycle $c$ to $[1]\cup c=c(1)$, which is the isomorphism $\op{Hom}_\ZZ(\ZZ,A)\simeq A$ for each $A$ anyway.

% 	We will also show $p=1$ by hand. The point here is that, for any $G$-module $A$, we already have (natural) isomorphisms
% 	\[\widehat H^0(G,\op{Hom}_\ZZ(I_G,A))\simeq\widehat H^1(G,A)\]
% 	by dimension-shifting, so we just have to track these through. Namely, we have the ($\ZZ$-split) short exact sequence
% 	\[0\to A\to\op{Hom}_\ZZ(\ZZ[G],A)\to\op{Hom}_\ZZ(I_G,A)\to 0,\]
% 	so given $[f]\in\widehat H^0(G,\op{Hom}_\ZZ(I_G,A))$ so that $f\in\op{Hom}_{\ZZ[G]}(I_G,A)$, this gets pulled back to the $0$-cochain $c\in\op{Hom}_\ZZ(\ZZ[G],A)$ defined by
% 	\[c\colon z\mapsto f(z-\varepsilon(z)).\]
% 	Pushing this down to $Z^1(G,\op{Hom}_\ZZ(\ZZ[G],A))$, we compute
% 	\begin{align*}
% 		(dc)(g)(z) &= (gc)(z)-c(z) \\
% 		&= g\cdot c\left(g^{-1}z\right) - c(z) \\
% 		&= g\cdot f\left(g^{-1}z-\varepsilon(g^{-1}z)\right) - f(z-\varepsilon(z)) \\
% 		&= f(z-g\varepsilon(z))-f(z-\varepsilon(z)) \\
% 		&= \varepsilon(z)\cdot f(1-g),
% 	\end{align*}
% 	so we pull back to the $1$-cochain $g\mapsto f(1-g)$ in $H^1(G,A)$. To see this as a cup product, we just note that running $A=I_G$ and $f=\id_{I_G}$ through this argument would reveal that $g\mapsto(1-g)$ is a $1$-cochain in $H^1(G,I_G)$, which we denote by $x_1$. Then returning to a general $G$-module $A$ with $[f]\in\widehat H^1(G,\op{Hom}_\ZZ(I_G,A))$, we see
% 	\[(x_1\cup[f])\colon g\mapsto(f\circ x_1)(g)=f(1-g)\]
% 	by construction of the cup product as evaluation.
% \end{proof}
\begin{cor} \label{cor:encodingsarecups}
	Let $G$ be a finite group, and let $p\in\ZZ$ be an index. Suppose we have a $G$-module $X$ and index $i\in\ZZ$ with a natural transformation
	\[\Phi_\bullet\colon\widehat H^i(G,\op{Hom}_\ZZ(X,-))\Rightarrow\widehat H^{i+p}(G,-).\]
	Then there exists $[x]\in\widehat H^p(G,X)$ such that $\Phi_\bullet$ is the cup-product map $[x]\cup-$.
\end{cor}
\begin{proof}
	The point is to set $X'=\ZZ$ in \autoref{prop:allnaturaltransarecups}. Indeed, $\Phi_\bullet$ will induce a natural transformation
	\[\widehat H^0(G,\op{Hom}_\ZZ(X,-))\stackrel{\Phi_\bullet}\Rightarrow\widehat H^{i+p}(G,-)\Rightarrow\widehat H^p(G,\op{Hom}_\ZZ(\ZZ,-)),\]
	where the last natural transformation is induced by the natural isomorphism $\eta\colon{\id}\simeq\op{Hom}_\ZZ(\ZZ,-)$. By \autoref{prop:allnaturaltransarecups}, we are promised $[x]\in\widehat H^0(G,\op{Hom}_\ZZ(\ZZ,X))$ such that this composite is $[x]\cup-$. Without being too detailed, we'll just say that passing everything through $\eta^{-1}$ shows that $\Phi_\bullet$ is
	\[\left[\eta^{-1}_Xx\right]\cup-\colon\widehat H^i(G,\op{Hom}_\ZZ(X,-))\Rightarrow\widehat H^{i+p}(G,-).\]
	One should check that all the evaluation maps correctly align, but they morally should because we're just doing pre-composition.
\end{proof}
\begin{example}
	For $p\ge0$, standard dimension-shifting arguments give natural isomorphisms
	\[\widehat H^0\left(G,\op{Hom}_\ZZ(I_G^{\otimes p},-)\right)\Rightarrow\widehat H^p(G,-),\]
	so \autoref{cor:encodingsarecups} implies that these isomorphisms are cup products with an element of $\widehat H^p(G,I_G^{\otimes p})$. For example, when $p=0$, we have $[1]\in\widehat H^0(G,\ZZ)$; and when $p=1$, we have $g\mapsto(1-g)$ in $\widehat H^1(G,I_G)$. Observe that we could also see this by inductively dimension-shifting with \autoref{cor:cupdown}.
\end{example}
Because cup products are better-behaved than just general natural transformations, we get the following nice statement.
\begin{cor} \label{cor:betterencodingdef}
	Let $G$ be a finite group, and let $p\in\ZZ$ an index. Then a $p$-encoding module $X$ has $x\in\widehat H^p(G,X)$ such that
	\[x\cup-\colon\widehat H^i(G,\op{Hom}_\ZZ(X,-))\Rightarrow\widehat H^{i+p}(G,-)\]
	is a natural isomorphism for all $i\in\ZZ$.
\end{cor}
\begin{proof}
	By definition of $X$, we know that there is some $i\in\ZZ$ such that we have a natural isomorphism
	\[\Phi_\bullet\colon\widehat H^i(G,\op{Hom}_\ZZ(X,-))\Rightarrow\widehat H^{i+p}(G,-).\]
	Then \autoref{cor:encodingsarecups} tells us that this natural isomorphism arises as $x\cup-$ for some $x\in\widehat H^p(G,X)$.
	
	To finish, we extend $x\cup-$ being a natural isomorphism from a single $i$ to all $i\in\ZZ$ by using \autoref{prop:dimshiftcupisos}. Indeed, take $F=\op{Hom}_\ZZ(X,-)$ and $F'=\mathrm{id}$ and $X=X$ and $\eta\colon X\otimes_\ZZ\op{Hom}_\ZZ(X,-)\Rightarrow\mathrm{id}$ to be the canonical evaluation maps. This finishes.
\end{proof}
\begin{remark}
	Taking $X=\ZZ$ above, we are asserting that, if $G$ is a group such that all $G$-modules admit period-$p$ cohomology which is natural in some sense at a single index $i$, then this periodicity extends to all indices and arises from a cup product with an element of $\widehat H^p(G,\ZZ)$.
	
	Observe that the naturality in the isomorphisms is important: letting $G\coloneqq\ZZ/p\ZZ$ act on $A\coloneqq\ZZ/p\ZZ$ trivially,
	\[\widehat H^{-1}(G,A)=\frac{\ZZ/p\ZZ}{0}\simeq\widehat H^0(G,A),\]
	but this does not extend to all $G$-modules. For example,
	\[\widehat H^{-1}(G,\ZZ)=0\not\cong\frac{\ZZ}{p\ZZ}=\widehat H^0(G,\ZZ).\]
\end{remark}

\subsection{Encoding Is Unique}
Fix a $p$-encoding module $X$. As a brief intermission, we will show that there is essentially one way to do the encoding
\[\widehat H^i(G,\op{Hom}_\ZZ(X,-))\Rightarrow\widehat H^{i+p}(G,-).\]
Namely, we know from \autoref{cor:encodingsarecups}, that this natural isomorphism must come from a cup-product with an element $x\in\widehat H^p(G,X)$, so we might wonder how unique this element $x$ is. The answer to this, roughly speaking, will be that $\widehat H^p(G,X)$ is cyclic of order $\#G$ generated by $x$.

Anyway, the main idea will be the following duality result.
\begin{prop}[\cite{cartan-eilenberg}, Corollary~XII.6.5] \label{prop:ceduality}
	Let $G$ be a finite group and $A$ be any $G$-module. Then the cup-product pairing induces an isomorphism
	\[\widehat H^{i-1}(G,\op{Hom}_\ZZ(A,\QQ/\ZZ))\to\op{Hom}_\ZZ\left(\widehat H^{-i}(G,A),\widehat H^{-1}(G,\QQ/\ZZ)\right)\]
	for all $i\in\ZZ$. Indeed, this is a duality upon embedding $\widehat H^{-1}(G,\QQ/\ZZ)$ into $\QQ/\ZZ$.
\end{prop}
And here is our computation.
\begin{cor} \label{cor:h2xcomputation}
	Let $G$ be a finite group and $X$ a $p$-encoding module. Then $\widehat H^p(G,X)\simeq\ZZ/\#G\ZZ$, generated by $x$, where $x\in\widehat H^p(G,X)$ is conjured from \autoref{cor:betterencodingdef}.
\end{cor}
\begin{proof}
	For brevity, set $n\coloneqq\#G$. By \autoref{cor:betterencodingdef}, we have the isomorphism
	\[x\cup-\colon\widehat H^{-p-1}(G,\op{Hom}_\ZZ(X,\QQ/\ZZ))\to\widehat H^{-1}(G,\QQ/\ZZ)=\textstyle\frac1n\ZZ/\ZZ.\]
	In particular, $\widehat H^{-p-1}(G,\op{Hom}_\ZZ(X,\ZZ))\simeq\ZZ/n\ZZ$, generated by some element $x^\lor$ such that $x\cup x^\lor=[1/n]$.

	Now, we apply \autoref{prop:ceduality} to say that the cup-product pairing induces an isomorphism
	\[{\textstyle\frac1n\ZZ/n\ZZ}\simeq\widehat H^{-p-1}(G,\op{Hom}_\ZZ(X,\QQ/\ZZ))\to\op{Hom}_\ZZ\left(\widehat H^p(G,X),\widehat H^{-1}(G,\QQ/\ZZ)\right)\simeq\op{Hom}_\ZZ\left(\widehat H^p(G,X),\textstyle\frac1n\ZZ/\ZZ\right).\]
	Because $\widehat H^p(G,X)$ is $n$-torsion, homomorphisms $\widehat H^2(G,X)\to\QQ/\ZZ$ must have image in $\frac1n\ZZ/\ZZ$, so in fact the rightmost group is the dual of $\widehat H^p(G,X)$. Because an abelian group is isomorphic to its dual, we see that $\widehat H^p(G,X)$ is in fact cyclic of order $n$.

	It remains to show that $x$ is a generator; for this, we show that $x$ has order at least $n$, which will be enough because $H^2(G,X)$ is cyclic of order $n$. Well, if we have $k\in\ZZ$ such that $kx=0$, then
	\[[k/n]=k\big(x\cup x^\lor\big)=kx\cup x^\lor=[0]\cup x^\lor=[0]\]
	in $\widehat H^{-1}(G,\QQ/\ZZ)\simeq\frac1n\ZZ/\ZZ$, so $n\mid k$. This finishes.
\end{proof}
\begin{remark}
	Conversely, if $x\in\widehat H^p(G,X)$ is any generator, then
	\[x\cup-\colon\widehat H^i(G,\op{Hom}_\ZZ(X,-))\Rightarrow\widehat H^{i+p}(G,-)\]
	is a natural isomorphism. Indeed, certainly some generator $x_0\in\widehat H^p(G,X)$ conjured from \autoref{cor:betterencodingdef} suffices, but then $x=kx_0$ for some $k\in(\ZZ/n\ZZ)^\times$, so we have the equality
	\[(x\cup-)=(kx_0)\cup-=k(x_0\cup-)\]
	of natural transformations. But multiplication by $k$ is a natural isomorphism $\widehat H^\bullet(G,-)\Rightarrow\widehat H^\bullet(G,-)$ because these cohomology groups are $\#G$-torsion, so we conclude $(x\cup-)=k(x_0\cup-)$ is a natural isomorphism.
\end{remark}
\begin{cor}
	Let $G$ be a finite group, and let $X$ be a $p$-encoding module. Then, given $i\in\ZZ$ and two natural isomorphisms
	\[\Phi_\bullet,\Phi_\bullet'\colon\widehat H^i(G,\op{Hom}_\ZZ(X,-))\Rightarrow\widehat H^{i+p}(G,-),\]
	there exists a unique $k\in(\ZZ/\#G\ZZ)^\times$ such that $\Phi_\bullet'=k\Phi_\bullet$.
\end{cor}
\begin{proof}
	Note that we are allowed to interpret $k\pmod n$ because these cohomology groups are $\#G$-torsion, so $\#G\cdot\Phi_\bullet=0$.

	Anyway, by \autoref{cor:encodingsarecups}, we know that there are $x,x'\in\widehat H^p(G,X)$ such that
	\[\Phi_\bullet=(x\cup-)\qquad\text{and}\qquad\Phi_\bullet'=(x'\cup-).\]
	However, by \autoref{cor:h2xcomputation}, we see that $\widehat H^p(G,X)$ is cyclic generated by $x$ of order $\#G$, so we can write $x'=kx$ for a unique $k\in\ZZ/\#G\ZZ$; because $x'$ must also be a generator, we see that $k\in(\ZZ/\#G\ZZ)^\times$ is forced. Namely, we can find $\ell\in\ZZ/\#G\ZZ$ such that $x=\ell x'$ as well.

	It remains to show that $\Phi_\bullet'=k\Phi_\bullet$. Well, for any $G$-module $A$ and $c\in\widehat H^i(G,\op{Hom}_\ZZ(X,A))$, we observe that
	\[\Phi_A'(c)=x'\cup c=kx\cup c=k(x\cup c)=k\Phi_A(c).\]
	It follows that $\Phi_\bullet'=k\Phi_\bullet$.
\end{proof}

\subsection{The Dual Element}
Let $X$ be a $p$-encoding module, and conjure $x\in\widehat H^p(G,X)$ from \autoref{cor:betterencodingdef}. Then note that our proof of \autoref{cor:h2xcomputation} found $x^\lor\in\widehat H^{-p-1}(G,\op{Hom}_\ZZ(X,\QQ/\ZZ))$ such that
\[x\cup x^\lor=[1/n]\in\widehat H^{-1}(G,\QQ/\ZZ).\]
This is fairly close to saying that the operation of $x\cup-$ can be inverted with the correct $x^\lor\cup-$ operation (and maybe a sign), but cupping with $[1/n]$ would then not necessarily by the identity transformation.

In particular, we would like to actually be in $\widehat H^0(G,\ZZ)$, whose cup products are well-behaved. As such, we have the following.
\begin{prop} \label{prop:intdualelement}
	Let $G$ be a finite group, and let $X$ be a $G$-module with index $p\in\ZZ$. The following are equivalent.
	\begin{listalph}
		\item $X$ is a $p$-encoding module.
		\item There are $x\in\widehat H^p(G,X)$ and $x^\lor\in\widehat H^{-p}(G,\op{Hom}_\ZZ(X,\ZZ))$ such that
		\[x\cup x^\lor=[1]\in\widehat H^0(G,\ZZ)\qquad\text{and}\qquad x^\lor\cup x=[{\id_X}]\in\widehat H^0(G,\op{Hom}_\ZZ(X,X)).\]
	\end{listalph}
\end{prop}
\begin{proof}
	Set $n\coloneqq\#G$.
	
	We start by showing (a) implies (b). By \autoref{cor:betterencodingdef}, we can find a generator $x\in\widehat H^p(G,X)$ yielding the isomorphism
	\[x\cup-\colon\widehat H^{-p}(G,\op{Hom}_\ZZ(X,\ZZ))\to\widehat H^0(G,\ZZ)=\ZZ/n\ZZ.\]
	As such, we can find a unique $x^\lor\in\widehat H^{-p}(G,\op{Hom}_\ZZ(X,\ZZ))$ such that $x\cup x^\lor=[1]$. It remains to show that $x^\lor\cup x=[{\id_X}]$.
	
	Note that $x\cup-$ and $x^\lor\cup-$ induce morphisms
	\[\arraycolsep=1.4pt\begin{array}{rccc}
		x\cup-\colon& \widehat H^0(G,\op{Hom}_\ZZ(X,X)) &\to& \widehat H^p(G,X) \\
		x^\lor\cup-\colon& \widehat H^{p}(G,X) &\to& \widehat H^0(G,\op{Hom}_\ZZ(X,X))
	\end{array}\]
	We claim that these are inverse. Because $x\cup-$ is already an isomorphism, it suffices to show that we have an inverse on one side. Also, $\widehat H^p(G,X)$ is cyclic generated by $x$, so it suffices to note that
	\[\big((x\cup-)\circ(x^\lor\cup-)\big)(kx)=x\cup x^\lor\cup kx=x\cup[k]=kx.\]
	Formally, we are computing the cup product by tracking $x^\lor$ through the commutativity of the following diagram.
	% https://q.uiver.app/?q=WzAsOSxbMCwwLCJcXHdpZGVoYXQgSF57LXB9KEcsXFxvcHtIb219X1xcWlooWCxcXFpaKSkiXSxbMSwwLCJcXHdpZGVoYXQgSF4wKEcsXFxvcHtIb219X1xcWlooWCxcXFpaKVxcb3RpbWVzX1xcWlogWCkiXSxbMCwxLCJcXHdpZGVoYXQgSF4wKEcsWFxcb3RpbWVzX1xcWlpcXG9we0hvbX1fXFxaWihYLFxcWlopKSJdLFsxLDEsIlxcd2lkZWhhdCBIXnAoRyxYXFxvdGltZXNfXFxaWlxcb3B7SG9tfV9cXFpaKFgsXFxaWilcXG90aW1lc19cXFpaIFgpIl0sWzAsMiwiXFx3aWRlaGF0IEheMChHLFxcWlopIl0sWzEsMiwiXFx3aWRlaGF0IEhecChHLFxcWlpcXG90aW1lc19cXFpaIFgpIl0sWzIsMiwiXFx3aWRlaGF0IEhecChHLFgpIl0sWzIsMSwiXFx3aWRlaGF0IEhecChHLFhcXG90aW1lc19cXFpaXFxvcHtIb219X1xcWlooWCxYKSkiXSxbMiwwLCJcXHdpZGVoYXQgSF4wKEcsXFxvcHtIb219X1xcWlooWCxYKSkiXSxbMCwxLCItXFxjdXAga3giXSxbMCwyLCJ4XFxjdXAtIiwyXSxbMiwzLCItXFxjdXAga3giXSxbMSwzLCJ4XFxjdXAtIiwyXSxbMiw0XSxbMyw1XSxbNCw1LCItXFxjdXAga3giXSxbMyw3XSxbNyw2XSxbNSw2XSxbMSw4XSxbOCw3LCJ4XFxjdXAtIiwyXV0=&macro_url=https%3A%2F%2Fraw.githubusercontent.com%2FdFoiler%2Fnotes%2Fmaster%2Fnir.tex
	\[\begin{tikzcd}
		{\widehat H^{-p}(G,\op{Hom}_\ZZ(X,\ZZ))} & {\widehat H^0(G,\op{Hom}_\ZZ(X,\ZZ)\otimes_\ZZ X)} & {\widehat H^0(G,\op{Hom}_\ZZ(X,X))} \\
		{\widehat H^0(G,X\otimes_\ZZ\op{Hom}_\ZZ(X,\ZZ))} & {\widehat H^p(G,X\otimes_\ZZ\op{Hom}_\ZZ(X,\ZZ)\otimes_\ZZ X)} & {\widehat H^p(G,X\otimes_\ZZ\op{Hom}_\ZZ(X,X))} \\
		{\widehat H^0(G,\ZZ)} & {\widehat H^p(G,\ZZ\otimes_\ZZ X)} & {\widehat H^p(G,X)}
		\arrow["{-\cup kx}", from=1-1, to=1-2]
		\arrow["{x\cup-}"', from=1-1, to=2-1]
		\arrow["{-\cup kx}", from=2-1, to=2-2]
		\arrow["{x\cup-}"', from=1-2, to=2-2]
		\arrow[from=2-1, to=3-1]
		\arrow[from=2-2, to=3-2]
		\arrow["{-\cup kx}", from=3-1, to=3-2]
		\arrow[from=2-2, to=2-3]
		\arrow[from=2-3, to=3-3]
		\arrow[from=3-2, to=3-3]
		\arrow[from=1-2, to=1-3]
		\arrow["{x\cup-}"', from=1-3, to=2-3]
	\end{tikzcd}\]
	We have written out this diagram because the bottom-right square requires some attention. Anyway, we now see that we have inverse morphisms, so
	\[x\cup[{\id_X}]=\id_X(x)=x\]
	implies that $x^\lor\cup x=[{\id_X}]$, finishing.

	We now show (b) implies (a). The main point is that
	\[\arraycolsep=1.4pt\begin{array}{rccc}
		x\cup-\colon& \widehat H^0(G,\op{Hom}_\ZZ(X,-)) &\Rightarrow& \widehat H^p(G,-) \\
		x^\lor\cup-\colon& \widehat H^{p}(G,-) &\Rightarrow& \widehat H^0(G,\op{Hom}_\ZZ(X,-))
	\end{array}\]
	ought to be inverse natural transformations; indeed, we know that they are natural by \autoref{lem:cuppingisnatural}, and it suffices to show that $x\cup-$ above is a natural isomorphism to finish.

	We already know that $x\cup-$ is a natural transformation, so it suffices to show that its component morphisms
	\[x\cup-\colon \widehat H^0(G,\op{Hom}_\ZZ(X,A))\to\widehat H^p(G,A)\]
	are isomorphisms for each $G$-module $A$. In fact, we claim that the corresponding map
	\[x^\lor\cup-\colon\widehat H^p(G,A)\to\widehat H^0(G,\op{Hom}_\ZZ(X,A))\]
	is the inverse morphism. In one direction, we note that any $a\in\widehat H^p(G,A)$ has
	\[\big((x\cup-)\circ(x^\lor\cup-)\big)(a)=x\cup x^\lor\cup a=[1]\cup a=a\]
	by tracking $x^\lor$ through the following commutative diagram.
	% https://q.uiver.app/?q=WzAsOSxbMCwwLCJcXHdpZGVoYXQgSF57LXB9KEcsXFxvcHtIb219X1xcWlooWCxcXFpaKSkiXSxbMSwwLCJcXHdpZGVoYXQgSF4wKEcsXFxvcHtIb219X1xcWlooWCxcXFpaKVxcb3RpbWVzX1xcWlogQSkiXSxbMCwxLCJcXHdpZGVoYXQgSF4wKEcsWFxcb3RpbWVzX1xcWlpcXG9we0hvbX1fXFxaWihYLFxcWlopKSJdLFsxLDEsIlxcd2lkZWhhdCBIXnAoRyxYXFxvdGltZXNfXFxaWlxcb3B7SG9tfV9cXFpaKFgsXFxaWilcXG90aW1lc19cXFpaIEEpIl0sWzAsMiwiXFx3aWRlaGF0IEheMChHLFxcWlopIl0sWzEsMiwiXFx3aWRlaGF0IEhecChHLFxcWlpcXG90aW1lc19cXFpaIEEpIl0sWzIsMiwiXFx3aWRlaGF0IEhecChHLEEpIl0sWzIsMSwiXFx3aWRlaGF0IEhecChHLFhcXG90aW1lc19cXFpaXFxvcHtIb219X1xcWlooWCxBKSkiXSxbMiwwLCJcXHdpZGVoYXQgSF4wKEcsXFxvcHtIb219X1xcWlooWCxBKSkiXSxbMCwxLCItXFxjdXAgYSJdLFswLDIsInhcXGN1cC0iLDJdLFsyLDMsIi1cXGN1cCBhIl0sWzEsMywieFxcY3VwLSIsMl0sWzIsNF0sWzMsNV0sWzQsNSwiLVxcY3VwIGEiXSxbMyw3XSxbNyw2XSxbNSw2XSxbMSw4XSxbOCw3LCJ4XFxjdXAtIiwyXV0=&macro_url=https%3A%2F%2Fraw.githubusercontent.com%2FdFoiler%2Fnotes%2Fmaster%2Fnir.tex
	\[\begin{tikzcd}
		{\widehat H^{-p}(G,\op{Hom}_\ZZ(X,\ZZ))} & {\widehat H^0(G,\op{Hom}_\ZZ(X,\ZZ)\otimes_\ZZ A)} & {\widehat H^0(G,\op{Hom}_\ZZ(X,A))} \\
		{\widehat H^0(G,X\otimes_\ZZ\op{Hom}_\ZZ(X,\ZZ))} & {\widehat H^p(G,X\otimes_\ZZ\op{Hom}_\ZZ(X,\ZZ)\otimes_\ZZ A)} & {\widehat H^p(G,X\otimes_\ZZ\op{Hom}_\ZZ(X,A))} \\
		{\widehat H^0(G,\ZZ)} & {\widehat H^p(G,\ZZ\otimes_\ZZ A)} & {\widehat H^p(G,A)}
		\arrow["{-\cup a}", from=1-1, to=1-2]
		\arrow["{x\cup-}"', from=1-1, to=2-1]
		\arrow["{-\cup a}", from=2-1, to=2-2]
		\arrow["{x\cup-}"', from=1-2, to=2-2]
		\arrow[from=2-1, to=3-1]
		\arrow[from=2-2, to=3-2]
		\arrow["{-\cup a}", from=3-1, to=3-2]
		\arrow[from=2-2, to=2-3]
		\arrow[from=2-3, to=3-3]
		\arrow[from=3-2, to=3-3]
		\arrow[from=1-2, to=1-3]
		\arrow["{x\cup-}"', from=1-3, to=2-3]
	\end{tikzcd}\]
	And in the other direction, we note that $a^\lor\in\widehat H^0(G,\op{Hom}_\ZZ(X,A))$ will have
	\[\big((x^\lor\cup-)\circ(x\cup-)\big)(a^\lor)=x^\lor\cup x\cup a=[\id_X]\cup a=\id_X(a)=a\]
	by tracking $x$ through the following commutative diagram.
	% https://q.uiver.app/?q=WzAsOSxbMCwwLCJcXHdpZGVoYXQgSF57cH0oRyxYKSJdLFsxLDAsIlxcd2lkZWhhdCBIXjAoRyxYXFxvdGltZXNfXFxaWlxcb3B7SG9tfV9cXFpaKFgsQSkpIl0sWzAsMSwiXFx3aWRlaGF0IEheMChHLFxcb3B7SG9tfV9cXFpaKFgsXFxaWilcXG90aW1lc19cXFpaIFgpIl0sWzEsMSwiXFx3aWRlaGF0IEhecChHLFxcb3B7SG9tfV9cXFpaKFgsXFxaWilcXG90aW1lc19cXFpaIFhcXG90aW1lc19cXFpaXFxvcHtIb219X1xcWlooWCxBKSkiXSxbMCwyLCJcXHdpZGVoYXQgSF4wKEcsXFxvcHtIb219X1xcWlooWCxYKSkiXSxbMSwyLCJcXHdpZGVoYXQgSF5wKEcsXFxvcHtIb219X1xcWlooWCxYKVxcb3RpbWVzX1xcWlpcXG9we0hvbX1fXFxaWihYLEEpKSJdLFsyLDIsIlxcd2lkZWhhdCBIXnAoRyxcXG9we0hvbX1fXFxaWihYLEEpKSJdLFsyLDEsIlxcd2lkZWhhdCBIXnAoRyxcXG9we0hvbX1fXFxaWihYLFxcWlopXFxvdGltZXNfXFxaWiBBKSJdLFsyLDAsIlxcd2lkZWhhdCBIXjAoRyxBKSJdLFswLDEsIi1cXGN1cCBhXlxcbG9yIl0sWzAsMiwieF5cXGxvclxcY3VwLSIsMl0sWzIsMywiLVxcY3VwIGFeXFxsb3IiXSxbMSwzLCJ4XlxcbG9yXFxjdXAtIiwyXSxbMiw0XSxbMyw1XSxbNCw1LCItXFxjdXAgYV5cXGxvciJdLFszLDddLFs3LDZdLFs1LDZdLFsxLDhdLFs4LDcsInheXFxsb3JcXGN1cC0iLDJdXQ==&macro_url=https%3A%2F%2Fraw.githubusercontent.com%2FdFoiler%2Fnotes%2Fmaster%2Fnir.tex
	\[\begin{tikzcd}[column sep=12pt]
		{\widehat H^{p}(G,X)} & {\widehat H^0(G,X\otimes_\ZZ\op{Hom}_\ZZ(X,A))} & {\widehat H^0(G,A)} \\
		{\widehat H^0(G,\op{Hom}_\ZZ(X,\ZZ)\otimes_\ZZ X)} & {\widehat H^p(G,\op{Hom}_\ZZ(X,\ZZ)\otimes_\ZZ X\otimes_\ZZ\op{Hom}_\ZZ(X,A))} & {\widehat H^p(G,\op{Hom}_\ZZ(X,\ZZ)\otimes_\ZZ A)} \\
		{\widehat H^0(G,\op{Hom}_\ZZ(X,X))} & {\widehat H^p(G,\op{Hom}_\ZZ(X,X)\otimes_\ZZ\op{Hom}_\ZZ(X,A))} & {\widehat H^p(G,\op{Hom}_\ZZ(X,A))}
		\arrow["{-\cup a^\lor}", from=1-1, to=1-2]
		\arrow["{x^\lor\cup-}"', from=1-1, to=2-1]
		\arrow["{-\cup a^\lor}", from=2-1, to=2-2]
		\arrow["{x^\lor\cup-}"', from=1-2, to=2-2]
		\arrow[from=2-1, to=3-1]
		\arrow[from=2-2, to=3-2]
		\arrow["{-\cup a^\lor}", from=3-1, to=3-2]
		\arrow[from=2-2, to=2-3]
		\arrow[from=2-3, to=3-3]
		\arrow[from=3-2, to=3-3]
		\arrow[from=1-2, to=1-3]
		\arrow["{x^\lor\cup-}"', from=1-3, to=2-3]
	\end{tikzcd}\]
	This finishes the proof.
\end{proof}
Here is an amusing corollary we get from this.
\begin{cor} \label{cor:encodingsubgroups}
	Let $G$ be a finite group, and let $p\in2\ZZ$ be even. Letting $X$ be a $p$-encoding module, and construct $x\in\widehat H^p(G,X)$ from \autoref{cor:betterencodingdef}. Then, for any subgroup $H\subseteq G$ and index $i\in\ZZ$, we have a natural isomorphism
	\[(\op{Res}x)\cup-\colon\widehat H^i(H,\op{Hom}_\ZZ(X,-))\Rightarrow\widehat H^{i+p}(H,-).\]
\end{cor}
\begin{proof}
	The point is that restriction commutes with cup products, so we may use \autoref{prop:intdualelement}. Indeed, we are given $x\in\widehat H^p(G,X)$ and $x^\lor\in\widehat H^{-p}(G,\op{Hom}_\ZZ(X,X))$ such that
	\[x\cup x^\lor=[1]\in\widehat H^0(G,\ZZ)\qquad\text{and}\qquad x^\lor\cup x=[{\id_X}]\in\widehat H^0(G,\op{Hom}_\ZZ(X,X)).\]
	Applying restriction to $H$ everywhere, we see
	\begin{align*}
		\op{Res}x\cup\op{Res}x^\lor &= \op{Res}(x\cup x^\lor) \\
		&= \op{Res}[1] \\
		&= [1]\in\widehat H^0(H,\ZZ),
	\end{align*}
	and
	\begin{align*}
		\op{Res}x^\lor\cup\op{Res}x &= \op{Res}(x^\lor\cup x) \\
		&= \op{Res}[{\id_X}] \\
		&= [{\id_X}]\in\widehat H^0(H,\op{Hom}_\ZZ(X,X)),
	\end{align*}
	which is enough by \autoref{prop:intdualelement}.
\end{proof}
% \begin{remark}
% 	The constraint that $p$ be even is not too strict. Namely, if $X$ is a $p$-encoding module, then we have natural isomorphisms
% 	\[\widehat H^i(G,\op{Hom}_\ZZ(X\otimes_\ZZ X,-))\simeq\widehat H^i(G,\op{Hom}_\ZZ(X,\op{Hom}_\ZZ(X,-)))\simeq\widehat H^{i+p}(G,\op{Hom}_\ZZ(X,-))\simeq\widehat H^{i+2p}(G,-),\]
% 	so $X\otimes_\ZZ X$ is a $2p$-encoding module.
% \end{remark}
\begin{remark}
	Essentially the same proof should hold for inflation.
\end{remark}
\begin{example}
	It is not true that, if $X$ is a $p$-encoding $G_q$-module for all Sylow $q$-subgroups $G_q\subseteq G$, then $X$ is a $p$-encoding $G$-module. Indeed, take $X=\ZZ$ and $G=S_3$: all Sylow $q$-subgroups of $S_3$ are cyclic, so $\ZZ$ is a $2$-encoding module for all these subgroups. However, $S_3$ is not cyclic, so
	\[\widehat H^{-2}(G,\op{Hom}_\ZZ(X,\ZZ))\simeq\widehat H^{-2}(G,\ZZ)\simeq S_3/[S_3,S_3]\not\cong\ZZ/6\ZZ=\widehat H^0(G,\ZZ).\]
\end{example}
We close by noting that the proof of \autoref{prop:intdualelement} actually managed to conjure the inverse natural transformation to $x\cup-$.
\begin{cor}
	Let $G$ be a finite group, and let $X$ be a $p$-encoding module. Constructing $x\in\widehat H^p(G,X)$ and $x^\lor\in\widehat H^{-p}(G,\op{Hom}_\ZZ(X,\ZZ))$ from \autoref{prop:intdualelement}, the natural transformations
	\[\arraycolsep=1.4pt\begin{array}{rccc}
		x\cup-\colon& \widehat H^i(G,\op{Hom}_\ZZ(X,-)) &\to& \widehat H^{i+p}(G,-) \\
		x^\lor\cup-\colon& \widehat H^{i+p}(G,-) &\to& \widehat H^i(G,\op{Hom}_\ZZ(X,-))
	\end{array}\]
	are inverse for each $i\in\ZZ$.
\end{cor}
\begin{proof}
	The case of $i=0$ follows directly from the proof of \autoref{prop:intdualelement}. The general case follows by literally shifting all the indices in the proof of \autoref{prop:intdualelement} away from $i=0$ to a general $i\in\ZZ$; nothing in the logic changes.
\end{proof}

\subsection{Using Duality}
\autoref{prop:ceduality} was able to provide us with some duality for $\widehat H^p(G,X)$, which enabled us to prove that $\widehat H^p(G,X)\simeq\ZZ/\#G\ZZ$. However, we saw in \autoref{prop:intdualelement}, that somehow the correct dual element is supposed to live in $\widehat H^{-p}(G,\op{Hom}_\ZZ(X,\ZZ))$.

Thus, we might hope that if we can go find $x\in\widehat H^p(G,X)$ and its dual element $x^\lor\in\widehat H^p(G,\op{Hom}_\ZZ(X,\ZZ))$, then we could actually recover the fact that $X$ is a $p$-encoding module. To this end, we pick up the following ``integral'' duality statement.
\begin{proposition} \label{prop:abstractintegralduality}
	Let $G$ be a finite group, and let $X$ be a finitely generated $\ZZ$-free $G$-module. Then the cup-product pairing induces an isomorphism
	\[\widehat H^i(G,\op{Hom}_\ZZ(X,\ZZ))\to\op{Hom}_\ZZ\left(\widehat H^{-i}(G,X),\widehat H^0(G,\ZZ)\right)\]
	for all $i\in\ZZ$. Indeed, this is a duality upon identifying $\widehat H^0(G,\ZZ)$ with $\frac1{\#G}\ZZ/\ZZ\subseteq\QQ/\ZZ$.
\end{proposition}
\begin{proof}
	This proof is analogous to \cite{cartan-eilenberg}, Theorem~XII.6.6. The key to the proof is the short exact sequence
	\begin{equation}
		0\to\ZZ\to\QQ\to\QQ/\ZZ\to0. \label{eq:divisibleses}
	\end{equation}
	The main point is that $X$ being finitely generated and $\ZZ$-free implies that $X$ is projective (as an abelian group), so we can apply $\op{Hom}_\ZZ(X,-)$ to get out the short exact sequence
	\begin{equation}
		0\to\op{Hom}_\ZZ(X,\ZZ)\to\op{Hom}_\ZZ(X,\QQ)\to\op{Hom}_\ZZ(X,\QQ/\ZZ)\to0. \label{eq:homdivisibleses}
	\end{equation}
	Now, note that the multiplication-by-$n$ endomorphism on $\op{Hom}_\ZZ(X,\QQ)$ is an isomorphism (namely, $\QQ$ is a divisible abelian group), so the same will be true of $\widehat H^i(G,\op{Hom}_\ZZ(X,\QQ))$ for any $i\in\ZZ$. However, these cohomology groups must be $\#G$-torsion, so in fact $\widehat H^i(G,\op{Hom}_\ZZ(X,\QQ))=0$ for all $i\in\ZZ$.

	Similarly, we note that we can hit \autoref{eq:homdivisibleses} with the functor $-\otimes_\ZZ X$ to get another short exact sequence
	\begin{equation}
		0\to\op{Hom}_\ZZ(X,\ZZ)\otimes_\ZZ X\to\op{Hom}_\ZZ(X,\QQ)\otimes_\ZZ X\to\op{Hom}_\ZZ(X,\QQ/\ZZ)\otimes_\ZZ X\to0. \label{eq:tensorhomdivisibleses}
	\end{equation}
	Notably, this is exact because $X$ is a finitely generated, torsion-free $\ZZ$-module and hence flat as a $\ZZ$-module. Now, $\op{Hom}_\ZZ(X,\QQ)\otimes_\ZZ X$ is still a divisible abelian group, so again $\widehat H^i(G,\op{Hom}_\ZZ(X,\QQ))=0$ for all $i\in\ZZ$.

	The rest of the proof is tracking boundary morphisms around. Fix some $i\in\ZZ$. Note \autoref{eq:divisibleses} and \autoref{eq:homdivisibleses} and \autoref{eq:tensorhomdivisibleses} induce boundary isomorphisms
	\[\arraycolsep=1.4pt\begin{array}{rlcl}
		\delta \colon& \widehat H^{-1}(G,\QQ/\ZZ) &\to& \widehat H^0(G,\ZZ) \\
		\delta_h \colon& \widehat H^{i-1}(G,\op{Hom}_\ZZ(X,\QQ/\ZZ))&\to&\widehat H^i(G,\op{Hom}_\ZZ(X,\ZZ)) \\
		\delta_t \colon& \widehat H^{-1}(G,\op{Hom}_\ZZ(\QQ/\ZZ)\otimes_\ZZ X)&\to&\widehat H^0(G,\op{Hom}_\ZZ(X,\ZZ)\otimes_\ZZ X).
	\end{array}\]
	We also note that we have a morphism of short exact sequences
	% https://q.uiver.app/?q=WzAsMTAsWzAsMCwiMCJdLFsxLDAsIlxcb3B7SG9tfV9cXFpaKFgsXFxaWilcXG90aW1lc19cXFpaIFgiXSxbMiwwLCJcXG9we0hvbX1fXFxaWihYLFxcUVEpXFxvdGltZXNfXFxaWiBYIl0sWzMsMCwiXFxvcHtIb219X1xcWlooWCxcXFFRL1xcWlopXFxvdGltZXNfXFxaWiBYIl0sWzAsMSwiMCJdLFs0LDAsIjAiXSxbNCwxLCIwIl0sWzEsMSwiXFxaWiJdLFsyLDEsIlxcUVEiXSxbMywxLCJcXFFRL1xcWloiXSxbMCwxXSxbMSwyXSxbMiwzXSxbMyw1XSxbNCw3XSxbNyw4XSxbOCw5XSxbOSw2XSxbMSw3LCJcXGV0YV9cXFpaIiwyXSxbMiw4LCJcXGV0YV9cXFFRIiwyXSxbMyw5LCJcXGV0YV97XFxRUS9cXFpafSIsMl1d&macro_url=https%3A%2F%2Fraw.githubusercontent.com%2FdFoiler%2Fnotes%2Fmaster%2Fnir.tex
	\[\begin{tikzcd}
		0 & {\op{Hom}_\ZZ(X,\ZZ)\otimes_\ZZ X} & {\op{Hom}_\ZZ(X,\QQ)\otimes_\ZZ X} & {\op{Hom}_\ZZ(X,\QQ/\ZZ)\otimes_\ZZ X} & 0 \\
		0 & \ZZ & \QQ & {\QQ/\ZZ} & 0
		\arrow[from=1-1, to=1-2]
		\arrow[from=1-2, to=1-3]
		\arrow[from=1-3, to=1-4]
		\arrow[from=1-4, to=1-5]
		\arrow[from=2-1, to=2-2]
		\arrow[from=2-2, to=2-3]
		\arrow[from=2-3, to=2-4]
		\arrow[from=2-4, to=2-5]
		\arrow["{\eta_\ZZ}"', from=1-2, to=2-2]
		\arrow["{\eta_\QQ}"', from=1-3, to=2-3]
		\arrow["{\eta_{\QQ/\ZZ}}"', from=1-4, to=2-4]
	\end{tikzcd}\]
	where the $\eta_\bullet$ are evaluation maps.
	% For peace of mind, we can check that the squares commute by the following lemma.
	% \begin{lemma} \label{lem:evcommutes}
	% 	Let $G$ be a group and $A,B,C$ be $G$-modules with a $G$-module homomorphism $\varphi\colon B\to C$. Then the diagram
	% 	% https://q.uiver.app/?q=WzAsNCxbMCwwLCJBXFxvdGltZXNfXFxaWlxcb3B7SG9tfShBLEIpIl0sWzEsMCwiQVxcb3RpbWVzX1xcWlooQSxDKSJdLFswLDEsIkIiXSxbMSwxLCJDIl0sWzAsMSwiXFx2YXJwaGkiXSxbMiwzLCJcXHZhcnBoaSJdLFswLDJdLFsxLDNdXQ==&macro_url=https%3A%2F%2Fraw.githubusercontent.com%2FdFoiler%2Fnotes%2Fmaster%2Fnir.tex
	% 	\[\begin{tikzcd}
	% 		{A\otimes_\ZZ\op{Hom}_\ZZ(A,B)} & {A\otimes_\ZZ\op{Hom}_\ZZ(A,C)} \\
	% 		B & C
	% 		\arrow["\varphi", from=1-1, to=1-2]
	% 		\arrow["\varphi", from=2-1, to=2-2]
	% 		\arrow[from=1-1, to=2-1]
	% 		\arrow[from=1-2, to=2-2]
	% 	\end{tikzcd}\]
	% 	commutes, where the vertical homomorphisms are evaluation.
	% \end{lemma}
	% \begin{proof}
	% 	We simply pick up some $a\otimes f\in A\otimes_\ZZ\op{Hom}_\ZZ(A,B)$ and track through
	% 	% https://q.uiver.app/?q=WzAsNCxbMCwwLCJhXFxvdGltZXMgZiJdLFsxLDAsImFcXG90aW1lc1xcdmFycGhpXFxjaXJjIGYiXSxbMCwxLCJmKGEpIl0sWzEsMSwiXFx2YXJwaGkoZihhKSkiXSxbMCwxLCJcXHZhcnBoaSIsMCx7InN0eWxlIjp7InRhaWwiOnsibmFtZSI6Im1hcHMgdG8ifX19XSxbMiwzLCJcXHZhcnBoaSIsMCx7InN0eWxlIjp7InRhaWwiOnsibmFtZSI6Im1hcHMgdG8ifX19XSxbMCwyLCIiLDEseyJzdHlsZSI6eyJ0YWlsIjp7Im5hbWUiOiJtYXBzIHRvIn19fV0sWzEsMywiIiwxLHsic3R5bGUiOnsidGFpbCI6eyJuYW1lIjoibWFwcyB0byJ9fX1dXQ==&macro_url=https%3A%2F%2Fraw.githubusercontent.com%2FdFoiler%2Fnotes%2Fmaster%2Fnir.tex
	% 	\[\begin{tikzcd}
	% 		{a\otimes f} & {a\otimes\varphi\circ f} \\
	% 		{f(a)} & {\varphi(f(a))}
	% 		\arrow["\varphi", maps to, from=1-1, to=1-2]
	% 		\arrow["\varphi", maps to, from=2-1, to=2-2]
	% 		\arrow[maps to, from=1-1, to=2-1]
	% 		\arrow[maps to, from=1-2, to=2-2]
	% 	\end{tikzcd}\]
	% 	which finishes the proof.
	% \end{proof}
	Now, \autoref{prop:ceduality} tells us that
	\[\arraycolsep=1.4pt\begin{array}{ccc}
		\widehat H^{i-1}(G,\op{Hom}_\ZZ(X,\QQ/\ZZ)) &\to& \op{Hom}_\ZZ\left(\widehat H^{-i}(G,X),\widehat H^{-1}(G,\QQ/\ZZ)\right) \\
		a &\mapsto& (b\mapsto\eta_{\QQ/\ZZ}(a\cup b))
	\end{array}\]
	is an isomorphism. Composing this with various other isomorphisms, we can build the isomorphism
	\[\arraycolsep=1.4pt\begin{array}{ccccccc}
		\widehat H^i(G,X_*) &\to& \widehat H^{i-1}(G,X^*) &\to& \op{Hom}\left(\widehat H^{-i}(G,X),\widehat H^{-1}(G,\QQ/\ZZ)\right) &\to& \op{Hom}\left(\widehat H^{-i}(G,X),\widehat H^0(G,\QQ/\ZZ)\right)  \\
		a &\mapsto& \delta_h^{-1}a &\mapsto& \left(b\mapsto\eta_{\QQ/\ZZ}(\delta_h^{-1}a\cup b)\right) &\mapsto& \left(b\mapsto\delta\eta_{\QQ/\ZZ}(\delta_h^{-1}a\cup b)\right)
	\end{array}\]
	where $X_*\coloneqq\op{Hom}_\ZZ(X,\ZZ)$ and $X^*\coloneqq\op{Hom}_\ZZ(X,\QQ/\ZZ)$, for brevity. This gives an isomorphism between the desired objects, but to prove the result we need to show that the above map is $a\mapsto(b\mapsto\eta_\ZZ(a\cup b))$. Well, given $a\in\widehat H^i(G,\op{Hom}_\ZZ(X,\ZZ))$ and $b\in\widehat H^{-i}(G,X)$, properties of the boundary morphisms tells us
	\begin{align*}
		\delta\eta_{\QQ/\ZZ}\left(\delta_h^{-1}a\cup b\right) &= \eta_\ZZ\delta_t\left(\delta_h^{-1}a\cup b\right) \\
		&= \eta_\ZZ\left(\delta_h\delta_h^{-1}a\cup b\right) \\
		&= \eta_\ZZ(a\cup b),
	\end{align*}
	which is what we wanted.
\end{proof}
\begin{remark}
	The hypothesis that $X$ be $\ZZ$-free is necessary: the statement is false for $X=\ZZ/\#G\ZZ$ and $i=0$, for example.
\end{remark}
And here is our result.
\begin{prop} \label{prop:finitecohomcheck}
	Let $G$ be a finite group, and let $X$ be a finitely generated $\ZZ$-free $G$-module. The following are equivalent.
	\begin{listalph}
		\item $X$ is a $p$-encoding module.
		\item $\widehat H^p(G,X)\cong\ZZ/\#G\ZZ$ and $\widehat H^0(G,\op{Hom}_\ZZ(X,X))$ is cyclic.
	\end{listalph}
\end{prop}
\begin{proof}
	For brevity, set $n\coloneqq\#G$. That (a) implies (b) is not hard: \autoref{cor:h2xcomputation} tells us that $\widehat H^p(G,X)\cong\ZZ/n\ZZ$, and then being a $p$-encoding module promises an isomorphism
	\[\widehat H^0(G,\op{Hom}_\ZZ(X,X))\simeq\widehat H^p(G,X)\cong\ZZ/n\ZZ.\]
	Thus, the interesting direction is showing that (b) implies (a).
	
	For this, we use \autoref{prop:abstractintegralduality} and \autoref{prop:intdualelement}. We are given $x\in\widehat H^p(G,X)$ of order $\#G$, and we note that there is a morphism
	\[\widehat H^p(G,X)\simeq\ZZ/n\ZZ=\widehat H^0(G,\ZZ)\]
	sending $x$ to $[1]$. Thus, \autoref{prop:abstractintegralduality} grants $x^\lor\in\widehat H^{-p}(G,\op{Hom}_\ZZ(X,\ZZ))$ such that
	\[x\cup x^\lor=[1]\in\widehat H^0(G,\ZZ).\]
	It remains to check that $x^\lor\cup x=[{\id_X}]\in\widehat H^0(G,\op{Hom}_\ZZ(X,X))$. This is more difficult.

	For this, we let $A$ be a $G$-module (which we will set to be $X$ shortly), and we claim that the composite
	\[\widehat H^p(G,A)\stackrel{x^\lor\cup-}\to\widehat H^0(G,\op{Hom}_\ZZ(X,A))\stackrel{x\cup-}\to\widehat H^p(G,A)\]
	is the identity. Indeed, the commutativity of the diagram
	% https://q.uiver.app/?q=WzAsOCxbMCwwLCJYXFxvdGltZXNfXFxaWlxcb3B7SG9tfV9cXFpaKFgsXFxaWilcXG90aW1lc19cXFpaIEEiXSxbMSwwLCJYXFxvdGltZXNfXFxaWlxcb3B7SG9tfV9cXFpaKFgsQSkiXSxbMCwxLCJcXFpaXFxvdGltZXNfXFxaWiBBIl0sWzEsMSwiQSJdLFsyLDAsInhfMFxcb3RpbWVzIGZcXG90aW1lcyBhXzAiXSxbMywwLCJ4XzBcXG90aW1lc1xcYmlnKHlcXG1hcHN0byBmKHkpYV8wXFxiaWcpIl0sWzIsMSwiZih4XzApXFxvdGltZXMgYV8wIl0sWzMsMSwiZih4XzApYV8wIl0sWzAsMl0sWzIsM10sWzAsMV0sWzEsM10sWzQsNiwiIiwyLHsic3R5bGUiOnsidGFpbCI6eyJuYW1lIjoibWFwcyB0byJ9fX1dLFs2LDcsIiIsMix7InN0eWxlIjp7InRhaWwiOnsibmFtZSI6Im1hcHMgdG8ifX19XSxbNCw1LCIiLDAseyJzdHlsZSI6eyJ0YWlsIjp7Im5hbWUiOiJtYXBzIHRvIn19fV0sWzUsNywiIiwwLHsic3R5bGUiOnsidGFpbCI6eyJuYW1lIjoibWFwcyB0byJ9fX1dXQ==&macro_url=https%3A%2F%2Fraw.githubusercontent.com%2FdFoiler%2Fnotes%2Fmaster%2Fnir.tex
	\[\begin{tikzcd}
		{X\otimes_\ZZ\op{Hom}_\ZZ(X,\ZZ)\otimes_\ZZ A} & {X\otimes_\ZZ\op{Hom}_\ZZ(X,A)} & {x_0\otimes f\otimes a_0} & {x_0\otimes\big(y\mapsto f(y)a_0\big)} \\
		{\ZZ\otimes_\ZZ A} & A & {f(x_0)\otimes a_0} & {f(x_0)a_0}
		\arrow[from=1-1, to=2-1]
		\arrow[from=2-1, to=2-2]
		\arrow[from=1-1, to=1-2]
		\arrow[from=1-2, to=2-2]
		\arrow[maps to, from=1-3, to=2-3]
		\arrow[maps to, from=2-3, to=2-4]
		\arrow[maps to, from=1-3, to=1-4]
		\arrow[maps to, from=1-4, to=2-4]
	\end{tikzcd}\]
	allows us to compute, for any $a\in\widehat H^p(G,A)$,
	\begin{equation}
		x\cup x^\lor\cup a=[1]\cup a=a, \label{eq:oneside}
	\end{equation}
	as desired.
	
	Now, taking $A=X$, we note
	\begin{equation}
		x\cup[{\id_X}]=x\in\widehat H^p(G,X). \label{eq:twoside}
	\end{equation}
	Thus, $[{\id_X}]\in\widehat H^0(G,\op{Hom}_\ZZ(X,X))$ has order $n$: if $k[{\id_X}]=0$, then $0=k(x\cup[{\id_X}])=kx$, so $n\mid k$. Because $\widehat H^0(G,\op{Hom}_\ZZ(X,X))$ is cyclic and $n$-torsion, we conclude that in fact $\widehat H^0(G,\op{Hom}_\ZZ(X,X))$ is cyclic of order $n$ generated by $[{\id_X}]$. Thus, we note that there is a unique isomorphism
	\[\widehat H^0(G,\op{Hom}_\ZZ(X,X))\cong\ZZ/n\ZZ\cong\widehat H^p(G,X)\]
	sending $[{\id_X}]$ to $1$ to $x$, so this isomorphism must be $x\cup-$ by \autoref{eq:twoside}. In particular, $x\cup-$ is injective. However, by \autoref{eq:oneside}, we know that $x^\lor\cup x\in\widehat H^0(G,\op{Hom}_\ZZ(X,X))$ has
	\[x\cup(x^\lor\cup x)=x=x\cup[{\id_X}]\]
	as discussed previously, so we conclude $x^\lor\cup x=[{\id_X}]$. This finishes.
\end{proof}
\begin{remark}
	As discussed in \autoref{rem:almosttorsionfree}, the requirement that $X$ be $\ZZ$-free is not too serious.
\end{remark}
\begin{example}
	To see that $\widehat H^0(G,\op{Hom}_\ZZ(X,X))$ being cyclic is necessary, let $G=\langle\sigma\rangle\simeq\ZZ/2\ZZ$ act on $X\coloneqq\ZZ[i]=\ZZ\oplus\ZZ i$ by conjugation. Then
	\[\widehat H^0(G,X)\simeq\widehat H^0(G,\ZZ)\oplus\widehat H^0(G,\ZZ i)\simeq\ZZ/2\ZZ,\]
	but
	\begin{align*}
		\widehat H^0(G,\op{Hom}_\ZZ(X,X)) &\simeq \widehat H^0(G,\op{Hom}_\ZZ(\ZZ,\ZZ))\oplus\widehat H^0(G,\op{Hom}_\ZZ(\ZZ,\ZZ i)) \\
		&\qquad\oplus\widehat H^0(G,\op{Hom}_\ZZ(\ZZ i,\ZZ))\oplus\widehat H^0(G,\op{Hom}_\ZZ(\ZZ i,\ZZ i))
	\end{align*}
	comes out to $\ZZ/2\ZZ\oplus0\oplus0\oplus\ZZ/2\ZZ$. Thus,
	\[\widehat H^0(G,\op{Hom}_\ZZ(X,X))\not\cong\widehat H^0(G,X),\]
	so $X$ is not a $0$-encoding module even though $X$ is $\ZZ$-free and $\widehat H^0(G,X)\cong\ZZ/\#G\ZZ$.
\end{example}
In some sense, the issue with the above example is that we could decompose our $G$-module into $A\oplus B$ when in fact there is no reason to talk about these sorts of $G$-modules as encoding modules.
\begin{cor} \label{cor:indecomposable}
	Let $G$ be a finite $q$-group. If $A\oplus B$ is a (finitely generated) $\ZZ$-free $p$-encoding module, then one of $A$ or $B$ is a $p$-encoding module.
\end{cor}
\begin{proof}
	This follows quickly from the check in \autoref{prop:finitecohomcheck}. On one hand,
	\begin{align*}
		\widehat H^0(G,\op{Hom}_\ZZ(A\oplus B,A\oplus B)) &\simeq \widehat H^0(G,\op{Hom}_\ZZ(A,A))\oplus\widehat H^0(G,\op{Hom}_\ZZ(A,B)) \\
		&\qquad\oplus\widehat H^0(G,\op{Hom}_\ZZ(B,A))\oplus\widehat H^0(G,\op{Hom}_\ZZ(B,B))
	\end{align*}
	tells us that both $\widehat H^0(G,\op{Hom}_\ZZ(A,A))$ and $\widehat H^0(G,\op{Hom}_\ZZ(B,B))$ are both cyclic because $\widehat H^0(G,\op{Hom}_\ZZ(A\oplus B,A\oplus B))$ is.

	On the other hand, we note
	\[\widehat H^p(G,A)\oplus\widehat H^p(G,B)\simeq\widehat H^p(G,A\oplus B)\cong\ZZ/\#G\ZZ,\]
	so we are forced to have $\widehat H^p(G,A)\cong\ZZ/\#G\ZZ$ or $\widehat H^p(G,B)\cong\ZZ/\#G\ZZ$ because $G$ is a finite $q$-group. This finishes.
\end{proof}
\begin{remark}
	It is conceivable that \autoref{cor:indecomposable} is true without requiring $A\oplus B$ to be $\ZZ$-free nor $G$ to be a $q$-group.
\end{remark}

\subsection{New Encoding Modules From Old}
The goal of this section is to build encoding modules up from smaller ones.
\begin{proposition}
	Let $G$ be a finite group. Given a $p$-encoding module $A$ and a $q$-encoding module $B$, the $G$-module $A\otimes_\ZZ B$ is a $(p+q)$-encoding module.
\end{proposition}
\begin{proof}
	By \autoref{cor:betterencodingdef}, we have natural isomorphisms
	\[\widehat H^0(G,\op{Hom}_\ZZ(A,-))\simeq\widehat H^p(G,-)\qquad\text{and}\qquad\widehat H^p(G,\op{Hom}_\ZZ(B,-))\simeq\widehat H^{p+q}(G,-).\]
	As such, we have a natural isomorphism
	\begin{align*}
		\widehat H^0(G,\op{Hom}_\ZZ(A\otimes_\ZZ B,-)) &\simeq \widehat H^0(G,\op{Hom}_\ZZ(A,\op{Hom}_\ZZ(B,-))) \\
		&\simeq \widehat H^p(G,\op{Hom}_\ZZ(B,-)) \\
		&\simeq \widehat H^{p+q}(G,-),
	\end{align*}
	which is what we wanted.
\end{proof}
\begin{proposition} \label{prop:dualmodule}
	Let $G$ be a finite group, and let $X$ be a $p$-encoding module. Then $\op{Hom}_\ZZ(X,\ZZ)$ is a $(-p)$-encoding module.
\end{proposition}
\begin{proof}
	We use \autoref{prop:intdualelement}. For brevity, we set $X^*\coloneqq\op{Hom}_\ZZ(X,\ZZ)$ and $X^{**}\coloneqq\op{Hom}_\ZZ(X^*,\ZZ)$. Observe that there is a (canonical) map $\varphi\colon X\to X^{**}$ by
	\[\varphi\colon f\mapsto f\circ\varphi.\]
	By \autoref{prop:intdualelement}, we may find $x\in\widehat H^p(G,X)$ and $x^\lor\in\widehat H^{-p}(G,X^*)$ such that
	\[x\cup x^\lor=[1]\in\widehat H^0(G,\ZZ)\qquad\text{and}\qquad x^\lor\cup x=[{\id_X}]\in\widehat H^0(G,\op{Hom}_\ZZ(X,X)).\]
	As such, we set $y\coloneqq x^\lor$ and $y^\lor\coloneqq(-1)^p\varphi(x)$. The commutative diagram
	% https://q.uiver.app/?q=WzAsOCxbMCwwLCJYXFxvdGltZXNfXFxaWiBYXioiXSxbMSwwLCJcXFpaIl0sWzAsMSwiWF57Kip9XFxvdGltZXNfXFxaWiBYXioiXSxbMSwxLCJcXFpaIl0sWzIsMCwieFxcb3RpbWVzIGYiXSxbMiwxLCIoZ1xcbWFwc3RvIGcoeCkpXFxvdGltZXMgZiJdLFszLDAsImYoeCkiXSxbMywxLCJmKHgpIl0sWzEsMywiIiwwLHsibGV2ZWwiOjIsInN0eWxlIjp7ImhlYWQiOnsibmFtZSI6Im5vbmUifX19XSxbMCwxXSxbMiwzXSxbMCwyLCJcXHZhcnBoaVxcb3RpbWVze1xcaWR9IiwyXSxbNCw2LCIiLDIseyJzdHlsZSI6eyJ0YWlsIjp7Im5hbWUiOiJtYXBzIHRvIn19fV0sWzQsNSwiIiwwLHsic3R5bGUiOnsidGFpbCI6eyJuYW1lIjoibWFwcyB0byJ9fX1dLFs1LDcsIiIsMCx7InN0eWxlIjp7InRhaWwiOnsibmFtZSI6Im1hcHMgdG8ifX19XSxbNiw3LCIiLDIseyJsZXZlbCI6Miwic3R5bGUiOnsiaGVhZCI6eyJuYW1lIjoibm9uZSJ9fX1dXQ==&macro_url=https%3A%2F%2Fraw.githubusercontent.com%2FdFoiler%2Fnotes%2Fmaster%2Fnir.tex
	\[\begin{tikzcd}
		{X\otimes_\ZZ X^*} & \ZZ & {x\otimes f} & {f(x)} \\
		{X^{**}\otimes_\ZZ X^*} & \ZZ & {(g\mapsto g(x))\otimes f} & {f(x)}
		\arrow[Rightarrow, no head, from=1-2, to=2-2]
		\arrow[from=1-1, to=1-2]
		\arrow[from=2-1, to=2-2]
		\arrow["{\varphi\otimes{\id}}"', from=1-1, to=2-1]
		\arrow[maps to, from=1-3, to=1-4]
		\arrow[maps to, from=1-3, to=2-3]
		\arrow[maps to, from=2-3, to=2-4]
		\arrow[Rightarrow, no head, from=1-4, to=2-4]
	\end{tikzcd}\]
	tells us that we may evaluate
	\[\varphi(x)\cup x^\lor=x\cup x^\lor=[1]\in\widehat H^0(G,\ZZ),\]
	so $y\cup y^\lor=[1]\in\widehat H^0(G,\ZZ)$ after being careful with signs.

	On the other hand, we set $A=B=X$ for clarity and note that the map $\psi\colon\op{Hom}_\ZZ(A,B)\to\op{Hom}_\ZZ(B^*,A^*)$ by
	\[\psi(f)\colon g\mapsto(g\circ f)\]
	gives the commutative diagram
	% https://q.uiver.app/?q=WzAsOCxbMCwwLCJBXipcXG90aW1lc19cXFpaIEIiXSxbMSwwLCJcXG9we0hvbX1fXFxaWihBLEIpIl0sWzAsMSwiQV4qXFxvdGltZXNfXFxaWlxcb3B7SG9tfV9cXFpaKEJeKixcXFpaKSJdLFsxLDEsIlxcb3B7SG9tfV9cXFpaKEJeKixBXiopIl0sWzIsMCwiZlxcb3RpbWVzIGIiXSxbMywwLCJcXGJpZyhhXFxtYXBzdG8gZihhKWJcXGJpZykiXSxbMiwxLCJmXFxvdGltZXNcXGJpZyhnXFxtYXBzdG8gZyhiKVxcYmlnKSJdLFszLDEsIlxcYmlnKGdcXG1hcHN0byBnKGIpZlxcYmlnKSJdLFsxLDMsIlxccHNpIl0sWzAsMV0sWzAsMiwie1xcaWR9XFxvdGltZXNcXHZhcnBoaSIsMl0sWzIsM10sWzQsNiwiIiwwLHsic3R5bGUiOnsidGFpbCI6eyJuYW1lIjoibWFwcyB0byJ9fX1dLFs2LDcsIiIsMCx7InN0eWxlIjp7InRhaWwiOnsibmFtZSI6Im1hcHMgdG8ifX19XSxbNSw3LCIiLDIseyJzdHlsZSI6eyJ0YWlsIjp7Im5hbWUiOiJtYXBzIHRvIn19fV0sWzQsNSwiIiwyLHsic3R5bGUiOnsidGFpbCI6eyJuYW1lIjoibWFwcyB0byJ9fX1dXQ==&macro_url=https%3A%2F%2Fraw.githubusercontent.com%2FdFoiler%2Fnotes%2Fmaster%2Fnir.tex
	\[\begin{tikzcd}
		{A^*\otimes_\ZZ B} & {\op{Hom}_\ZZ(A,B)} & {f\otimes b} & {\big(a\mapsto f(a)b\big)} \\
		{A^*\otimes_\ZZ\op{Hom}_\ZZ(B^*,\ZZ)} & {\op{Hom}_\ZZ(B^*,A^*)} & {f\otimes\big(g\mapsto g(b)\big)} & {\big(g\mapsto g(b)f\big)}
		\arrow["\psi", from=1-2, to=2-2]
		\arrow[from=1-1, to=1-2]
		\arrow["{{\id}\otimes\varphi}"', from=1-1, to=2-1]
		\arrow[from=2-1, to=2-2]
		\arrow[maps to, from=1-3, to=2-3]
		\arrow[maps to, from=2-3, to=2-4]
		\arrow[maps to, from=1-4, to=2-4]
		\arrow[maps to, from=1-3, to=1-4]
	\end{tikzcd}\]
	which tells us that we may evaluate
	\[x^\lor\cup\varphi(x)=\psi(x^\lor\cup x)=\psi([\id_X])=[\id_{X^*}]\in\widehat H^0(G,\op{Hom}_\ZZ(X^*,X^*)),\]
	so $y^\lor\cup y=[{\id_{X^*}}]$ after being careful with signs. This completes the proof.
\end{proof}
\begin{example}
	\autoref{ex:igisencoding} established that $I_G^{\otimes p}$ is a $p$-encoding module for $p\ge0$. As such, $\op{Hom}_\ZZ\left(I_G^{\otimes p},\ZZ\right)$ is a $(-p)$-encoding module for $-p\le0$. Thus, we have established existence for $p$-encoding modules for all $p\in\ZZ$.
\end{example}
\begin{cor} \label{cor:getfreeencoder}
	Let $G$ be a finite group, and let $X$ be a finitely generated $p$-encoding module. Letting $X_t$ denote the $\ZZ$-torsion subgroup of $X$, we have that $X_t$ is a $G$-submodule of $X$, and $X/X_t$ is a $p$-encoding module.
\end{cor}
\begin{proof}
	To see that $X_t$ is a $G$-submodule, we note that any $x\in X_t$ has some $k\in\ZZ$ such that $kx=0$, so any $g\in G$ will have
	\[k\cdot gx=g(kx)=g\cdot0=0.\]
	Thus, $X_t\subseteq X$ is preserved by $G$.

	It remains to show that $X_f\coloneqq X/X_t$ is a $p$-encoding module. Well, we claim that
	\begin{equation}
		\op{Hom}_\ZZ(\op{Hom}_\ZZ(X,\ZZ),\ZZ)\cong\op{Hom}_\ZZ(\op{Hom}_\ZZ(X_f,\ZZ),\ZZ) \label{eq:isodoubledual}
	\end{equation}
	as $G$-modules; by \autoref{prop:dualmodule}, this will imply that $\op{Hom}_\ZZ(\op{Hom}_\ZZ(X_f,\ZZ),\ZZ)$ is a $p$-encoding module. To see this, we note that the short exact sequence
	\[0\to X_t\to X\to X_f\to 0\]
	becomes the left exact sequence
	\[0\to\op{Hom}_\ZZ(X_f,\ZZ)\to\op{Hom}_\ZZ(X,\ZZ)\to\op{Hom}_\ZZ(X_t,\ZZ).\]
	However, $\op{Hom}_\ZZ(X_t,\ZZ)=0$ because $X_t$ is $\ZZ$-torsion, so the above left exact sequence witnesses the isomorphism $\op{Hom}_\ZZ(X_f,\ZZ)\cong\op{Hom}_\ZZ(X,\ZZ)$. Applying $\op{Hom}_\ZZ(-,\ZZ)$ again yields \autoref{eq:isodoubledual}.

	To finish, we note that
	\[\varphi\colon X_f\to\op{Hom}_\ZZ(\op{Hom}_\ZZ(X_f,\ZZ),\ZZ)\]
	by $x\mapsto(f\mapsto f(x))$ is a $G$-module morphism and isomorphism of abelian groups because $X_f$ is torsion-free and finitely generated and hence $\ZZ$-free. Thus, $\varphi$ is an isomorphism of $G$-modules, implying that $X_f$ is a $p$-encoding module.
\end{proof}
\begin{remark} \label{rem:almosttorsionfree}
	Even though \autoref{ex:notalltorsionfree} asserts that not all $p$-encoding modules $X$ are $\ZZ$-torsion-free, \autoref{cor:getfreeencoder} explains that we can canonically obtain a $\ZZ$-torsion-free $p$-encoding module from $X$ in the form of $\op{Hom}_\ZZ(\op{Hom}_\ZZ(X,\ZZ),\ZZ)\cong X/X_t$.
\end{remark}
\begin{proposition} \label{prop:encodingses}
	Let $G$ be a finite group, and let
	\[0\to X'\to M\to X\to 0\]
	be a $\ZZ$-split short exact sequence such that $M$ is an induced $G$-module. Then $X$ is a $p$-encoding module if and only if $X'$ is a $(p+1)$-encoding module.
\end{proposition}
\begin{proof}
	Given a $G$-module $A$, we recall that $\op{Hom}_\ZZ(-,A)$ is a shiftable functor by \autoref{lem:contravariantshiftable}, so $\op{Hom}_\ZZ(M,A)$ is induced. Now, because the short exact sequence is $\ZZ$-split, we have the short exact sequence
	\[0\to\op{Hom}_\ZZ(X,A)\to\op{Hom}_\ZZ(M,A)\to\op{Hom}_\ZZ(X',A)\to0\]
	which gives the isomorphism
	\[\delta_A\colon\widehat H^0(G,\op{Hom}_\ZZ(X',A))\to\widehat H^1(G,\op{Hom}_\ZZ(X,A))\]
	because $\op{Hom}_\ZZ(M,A)$ is induced. In fact, the $\delta_A$ make a natural isomorphism $\delta_\bullet\colon\widehat H^0(G,\op{Hom}_\ZZ(X',-))\Rightarrow\widehat H^1(G,\op{Hom}_\ZZ(X,-))$: given a $G$-module morphism $f\colon A\to B$, the morphism of short exact sequences
	% https://q.uiver.app/?q=WzAsMTAsWzAsMCwiMCJdLFsxLDAsIlxcb3B7SG9tfV9cXFpaKFgsQSkiXSxbMiwwLCJcXG9we0hvbX1fXFxaWihNLEEpIl0sWzMsMCwiXFxvcHtIb219X1xcWlooWCcsQSkiXSxbNCwwLCIwIl0sWzEsMSwiXFxvcHtIb219X1xcWlooWCxCKSJdLFsyLDEsIlxcb3B7SG9tfV9cXFpaKE0sQikiXSxbMywxLCJcXG9we0hvbX1fXFxaWihYJyxCKSJdLFs0LDEsIjAiXSxbMCwxLCIwIl0sWzAsMV0sWzEsMl0sWzIsM10sWzMsNF0sWzksNV0sWzUsNl0sWzYsN10sWzcsOF0sWzEsNSwiZiIsMl0sWzIsNiwiZiIsMl0sWzMsNywiZiIsMl1d&macro_url=https%3A%2F%2Fraw.githubusercontent.com%2FdFoiler%2Fnotes%2Fmaster%2Fnir.tex
	\[\begin{tikzcd}
		0 & {\op{Hom}_\ZZ(X,A)} & {\op{Hom}_\ZZ(M,A)} & {\op{Hom}_\ZZ(X',A)} & 0 \\
		0 & {\op{Hom}_\ZZ(X,B)} & {\op{Hom}_\ZZ(M,B)} & {\op{Hom}_\ZZ(X',B)} & 0
		\arrow[from=1-1, to=1-2]
		\arrow[from=1-2, to=1-3]
		\arrow[from=1-3, to=1-4]
		\arrow[from=1-4, to=1-5]
		\arrow[from=2-1, to=2-2]
		\arrow[from=2-2, to=2-3]
		\arrow[from=2-3, to=2-4]
		\arrow[from=2-4, to=2-5]
		\arrow["f"', from=1-2, to=2-2]
		\arrow["f"', from=1-3, to=2-3]
		\arrow["f"', from=1-4, to=2-4]
	\end{tikzcd}\]
	induces the desired commuting square, as follows.
	% https://q.uiver.app/?q=WzAsNCxbMCwwLCJcXHdpZGVoYXQgSF4wKEcsXFxvcHtIb219X1xcWlooWCcsQSkpIl0sWzEsMCwiXFx3aWRlaGF0IEheMShHLFxcb3B7SG9tfV9cXFpaKFgsQSkpIl0sWzAsMSwiXFx3aWRlaGF0IEheMChHLFxcb3B7SG9tfV9cXFpaKFgnLEIpKSJdLFsxLDEsIlxcd2lkZWhhdCBIXjEoRyxcXG9we0hvbX1fXFxaWihYLEEpKSJdLFswLDEsIlxcZGVsdGFfQSJdLFsyLDMsIlxcZGVsdGFfQiJdLFswLDIsImYiLDJdLFsxLDMsImYiLDJdXQ==&macro_url=https%3A%2F%2Fraw.githubusercontent.com%2FdFoiler%2Fnotes%2Fmaster%2Fnir.tex
	\[\begin{tikzcd}
		{\widehat H^0(G,\op{Hom}_\ZZ(X',A))} & {\widehat H^1(G,\op{Hom}_\ZZ(X,A))} \\
		{\widehat H^0(G,\op{Hom}_\ZZ(X',B))} & {\widehat H^1(G,\op{Hom}_\ZZ(X,B))}
		\arrow["{\delta_A}", from=1-1, to=1-2]
		\arrow["{\delta_B}", from=2-1, to=2-2]
		\arrow["f"', from=1-1, to=2-1]
		\arrow["f"', from=1-2, to=2-2]
	\end{tikzcd}\]
	We now proceed with the proof. In one direction, if $X$ is a $p$-encoding module, then \autoref{cor:betterencodingdef} promises us a natural isomorphism
	\[\Phi_\bullet\colon\widehat H^1(G,\op{Hom}_\ZZ(X,-))\Rightarrow\widehat H^{p+1}(G,-),\]
	so the composite
	\[\widehat H^0(G,\op{Hom}_\ZZ(X',-))\stackrel{\delta_\bullet}\Rightarrow\widehat H^1(G,\op{Hom}_\ZZ(X,-))\stackrel{\Phi_\bullet}\Rightarrow\widehat H^{p+1}(G,-)\]
	shows that $X'$ is a $(p+1)$-encoding module. The other direction is analogous, concatenating with $\delta_\bullet^{-1}$.
\end{proof}
\begin{example}
	Fix a finite group $G$ generated by $S\coloneqq\langle\sigma_1,\ldots,\sigma_n\rangle$, and let $M\coloneqq\ZZ[G]^{\#S}$ have basis $\{e_i\}_{i=1}^m$. Then there is a projection $\pi\colon\ZZ[G]^{\#G}\onto I_G$ by sending $e_i\mapsto(\sigma_i-1)$, giving the short exact sequence
	\[0\to\ker\pi\to\ZZ[G]^{\#S}\to I_G\to0.\]
	This short exact sequence is $\ZZ$-split because $I_G$ is $\ZZ$-free. Because $\ZZ[G]^{\#S}\cong\ZZ[G]\otimes_\ZZ\ZZ^{\#S}$ is induced and $I_G$ is a $1$-encoding module, we conclude that $\ker\pi$ is a $2$-encoding module by \autoref{prop:encodingses}.
\end{example}
By this point, we have a wide array of ways of making $p$-encoding modules, so we call it quits here.

\subsection{A Perfect Pairing}
We close this section with a hint of Artin reciprocity. The main goal of this subsection is to prove the following result.
\begin{theorem} \label{thm:abstractperfectpairing}
	Let $G$ be a finite group, and let $X$ and $A$ be $G$-modules. Then, if there exists an element $c\in H^p(G,X)$ such that the cup-product maps
	\begin{align*}
		c\cup-&\colon\widehat H^{-p}(G,\op{Hom}_\ZZ(X,\ZZ))\to\widehat H^0(G,\ZZ) \\
		c\cup-&\colon\widehat H^0(G,\op{Hom}_\ZZ(X,A))\to\widehat H^{p}(G,A)
	\end{align*}
	are isomorphisms, then the cup-product pairing induces an isomorphism
	\[\widehat H^p(G,A)\to\op{Hom}_\ZZ\left(\widehat H^{-p}(G,\op{Hom}_\ZZ(X,\ZZ)),\widehat H^0(G,\op{Hom}_\ZZ(X,A))\right).\]
\end{theorem}
The main step in the proof is the following lemma.
\begin{lemma}
	Let $G$ be a finite group, and let $X$ and $A$ be $G$-modules. Pick up another $G$-module $A$. Then, given any $i\in\ZZ$ and $c\in\widehat H^p(G,X)$ and $u\in\widehat H^p(G,A)$, the following diagram commutes, where all arrows are cup-product maps.
	% https://q.uiver.app/?q=WzAsNCxbMCwwLCJcXHdpZGVoYXQgSF57aS0yfShHLFxcb3B7SG9tfV9cXFpaKFgsXFxaWikpIl0sWzEsMCwiXFx3aWRlaGF0IEheaShHLFxcb3B7SG9tfV9cXFpaKFgsQSkpIl0sWzAsMSwiXFx3aWRlaGF0IEheaShHLFxcWlopIl0sWzEsMSwiXFx3aWRlaGF0IEhee2krMn0oRyxBKSJdLFswLDEsIi1cXGN1cCB1Il0sWzAsMiwiY1xcY3VwLSIsMl0sWzIsMywiLVxcY3VwIHUiLDJdLFsxLDMsImNcXGN1cC0iXV0=&macro_url=https%3A%2F%2Fraw.githubusercontent.com%2FdFoiler%2Fnotes%2Fmaster%2Fnir.tex
	\[\begin{tikzcd}
		{\widehat H^{i-p}(G,\op{Hom}_\ZZ(X,\ZZ))} & {\widehat H^i(G,\op{Hom}_\ZZ(X,A))} \\
		{\widehat H^i(G,\ZZ)} & {\widehat H^{i+p}(G,A)}
		\arrow["{-\cup u}", from=1-1, to=1-2]
		\arrow["{c\cup-}"', from=1-1, to=2-1]
		\arrow["{-\cup u}"', from=2-1, to=2-2]
		\arrow["{c\cup-}", from=1-2, to=2-2]
	\end{tikzcd}\]
\end{lemma}
\begin{proof}
	Formally, our cup-product maps are induced by the following ``evaluation morphisms.''
	\begin{itemize}
		\item For the left arrow, we have $\eta_L\colon X\otimes_\ZZ\op{Hom}_\ZZ(X,\ZZ)\to\ZZ$ by evaluation.
		\item For the top arrow, we have $\eta_T\colon\op{Hom}_\ZZ(X,\ZZ)\otimes_\ZZ A\to\op{Hom}_\ZZ(X,A)$ by $f\otimes a\mapsto(x\mapsto f(x)a)$.
		\item For the bottom arrow, we have $\eta_B\colon\ZZ\otimes_\ZZ A\to A$ by $k\otimes a\mapsto ka$.
		\item For the right arrow, we have $\eta_R\colon X\otimes_\ZZ\op{Hom}_\ZZ(X,A)\to A$ by evaluation.
	\end{itemize}
	In particular, these maps are defined so that the following diagram commutes.
	% https://q.uiver.app/?q=WzAsNCxbMCwwLCJYXFxvdGltZXNfXFxaWlxcb3B7SG9tfV9cXFpaKFgsXFxaWilcXG90aW1lc19cXFpaIEEiXSxbMSwwLCJYXFxvdGltZXNfXFxaWlxcb3B7SG9tfV9cXFpaKFgsQSkiXSxbMCwxLCJcXFpaXFxvdGltZXNfXFxaWiBBIl0sWzEsMSwiQSJdLFswLDEsIlxcZXRhX1QiXSxbMCwyLCJcXGV0YV9MIiwyXSxbMSwzLCJcXGV0YV9SIl0sWzIsMywiXFxldGFfQiIsMl1d&macro_url=https%3A%2F%2Fraw.githubusercontent.com%2FdFoiler%2Fnotes%2Fmaster%2Fnir.tex
	\begin{equation}
		\begin{tikzcd}
			{X\otimes_\ZZ\op{Hom}_\ZZ(X,\ZZ)\otimes_\ZZ A} & {X\otimes_\ZZ\op{Hom}_\ZZ(X,A)} \\
			{\ZZ\otimes_\ZZ A} & A
			\arrow["{\eta_T}", from=1-1, to=1-2]
			\arrow["{\eta_L}"', from=1-1, to=2-1]
			\arrow["{\eta_R}", from=1-2, to=2-2]
			\arrow["{\eta_B}"', from=2-1, to=2-2]
		\end{tikzcd} \label{eq:innermorphismcoherence}
	\end{equation}
	Indeed, we can just compute along the following diagram.
	% https://q.uiver.app/?q=WzAsNCxbMCwwLCJ4XFxvdGltZXMgZlxcb3RpbWVzIGEiXSxbMSwwLCJ4XFxvdGltZXMoeCdcXG1hcHN0byBmKHgnKWEpIl0sWzAsMSwiZih4KVxcb3RpbWVzIGEiXSxbMSwxLCJmKHgpYSJdLFswLDEsIlxcZXRhX1QiLDAseyJzdHlsZSI6eyJ0YWlsIjp7Im5hbWUiOiJtYXBzIHRvIn19fV0sWzAsMiwiXFxldGFfTCIsMix7InN0eWxlIjp7InRhaWwiOnsibmFtZSI6Im1hcHMgdG8ifX19XSxbMSwzLCJcXGV0YV9SIiwwLHsic3R5bGUiOnsidGFpbCI6eyJuYW1lIjoibWFwcyB0byJ9fX1dLFsyLDMsIlxcZXRhX0IiLDIseyJzdHlsZSI6eyJ0YWlsIjp7Im5hbWUiOiJtYXBzIHRvIn19fV1d&macro_url=https%3A%2F%2Fraw.githubusercontent.com%2FdFoiler%2Fnotes%2Fmaster%2Fnir.tex
	\[\begin{tikzcd}
		{x\otimes f\otimes a} & {x\otimes(x'\mapsto f(x')a)} \\
		{f(x)\otimes a} & {f(x)a}
		\arrow["{\eta_T}", maps to, from=1-1, to=1-2]
		\arrow["{\eta_L}"', maps to, from=1-1, to=2-1]
		\arrow["{\eta_R}", maps to, from=1-2, to=2-2]
		\arrow["{\eta_B}"', maps to, from=2-1, to=2-2]
	\end{tikzcd}\]
	Now, the core of the proof is in drawing the following very large diagram.
	% https://q.uiver.app/?q=WzAsOSxbMCwwLCJcXHdpZGVoYXQgSF57aS0yfShHLFxcb3B7SG9tfV9cXFpaKFgsXFxaWikpIl0sWzEsMCwiXFx3aWRlaGF0IEheaShHLFxcb3B7SG9tfV9cXFpaKFgsXFxaWilcXG90aW1lc19cXFpaIEEpIl0sWzIsMCwiXFx3aWRlaGF0IEheaShHLFxcb3B7SG9tfV9cXFpaKFgsQSkpIl0sWzAsMSwiXFx3aWRlaGF0IEheaShHLFhcXG90aW1lc19cXFpaXFxvcHtIb219X1xcWlooWCxcXFpaKSkiXSxbMSwxLCJcXHdpZGVoYXQgSF57aSsyfShHLFhcXG90aW1lc19cXFpaXFxvcHtIb219X1xcWlooWCxcXFpaKVxcb3RpbWVzX1xcWlogQSkiXSxbMiwxLCJcXHdpZGVoYXQgSF57aSsyfShHLFhcXG90aW1lc19cXFpaXFxvcHtIb219X1xcWlooWCxBKSkiXSxbMiwyLCJcXHdpZGVoYXQgSF57aSsyfShHLEEpIl0sWzAsMiwiXFx3aWRlaGF0IEheaShHLFxcWlopIl0sWzEsMiwiXFx3aWRlaGF0IEheMihHLFhcXG90aW1lc19cXFpaIEEpIl0sWzAsMSwiLVxcY3VwIHUiXSxbMyw0LCItXFxjdXAgdSJdLFs3LDgsIi1cXGN1cCB1Il0sWzAsMywiY1xcY3VwIC0iLDJdLFsxLDQsImNcXGN1cCAtIiwyXSxbMiw1LCJjXFxjdXAgLSIsMl0sWzEsMiwiXFxldGFfVCJdLFs0LDUsIlxcZXRhX1QiXSxbOCw2LCJcXGV0YV9CIl0sWzMsNywiXFxldGFfTCIsMl0sWzQsOCwiXFxldGFfTCIsMl0sWzUsNiwiXFxldGFfUiIsMl0sWzEyLDEzLCIoMSkiLDMseyJzaG9ydGVuIjp7InNvdXJjZSI6MjAsInRhcmdldCI6MjB9LCJzdHlsZSI6eyJib2R5Ijp7Im5hbWUiOiJub25lIn0sImhlYWQiOnsibmFtZSI6Im5vbmUifX19XSxbMTMsMTQsIigyKSIsMyx7InNob3J0ZW4iOnsic291cmNlIjoyMCwidGFyZ2V0IjoyMH0sInN0eWxlIjp7ImJvZHkiOnsibmFtZSI6Im5vbmUifSwiaGVhZCI6eyJuYW1lIjoibm9uZSJ9fX1dLFsxOCwxOSwiKDMpIiwzLHsic2hvcnRlbiI6eyJzb3VyY2UiOjIwLCJ0YXJnZXQiOjIwfSwic3R5bGUiOnsiYm9keSI6eyJuYW1lIjoibm9uZSJ9LCJoZWFkIjp7Im5hbWUiOiJub25lIn19fV0sWzE5LDIwLCIoNCkiLDMseyJzaG9ydGVuIjp7InNvdXJjZSI6MjAsInRhcmdldCI6MjB9LCJzdHlsZSI6eyJib2R5Ijp7Im5hbWUiOiJub25lIn0sImhlYWQiOnsibmFtZSI6Im5vbmUifX19XV0=&macro_url=https%3A%2F%2Fraw.githubusercontent.com%2FdFoiler%2Fnotes%2Fmaster%2Fnir.tex
	\[\begin{tikzcd}
		{\widehat H^{i-p}(G,\op{Hom}_\ZZ(X,\ZZ))} & {\widehat H^i(G,\op{Hom}_\ZZ(X,\ZZ)\otimes_\ZZ A)} & {\widehat H^i(G,\op{Hom}_\ZZ(X,A))} \\
		{\widehat H^i(G,X\otimes_\ZZ\op{Hom}_\ZZ(X,\ZZ))} & {\widehat H^{i+p}(G,X\otimes_\ZZ\op{Hom}_\ZZ(X,\ZZ)\otimes_\ZZ A)} & {\widehat H^{i+p}(G,X\otimes_\ZZ\op{Hom}_\ZZ(X,A))} \\
		{\widehat H^i(G,\ZZ)} & {\widehat H^{i+p}(G,X\otimes_\ZZ A)} & {\widehat H^{i+p}(G,A)}
		\arrow["{-\cup u}", from=1-1, to=1-2]
		\arrow["{-\cup u}", from=2-1, to=2-2]
		\arrow["{-\cup u}", from=3-1, to=3-2]
		\arrow[""{name=0, anchor=center, inner sep=0}, "{c\cup -}"', from=1-1, to=2-1]
		\arrow[""{name=1, anchor=center, inner sep=0}, "{c\cup -}"', from=1-2, to=2-2]
		\arrow[""{name=2, anchor=center, inner sep=0}, "{c\cup -}"', from=1-3, to=2-3]
		\arrow["{\eta_T}", from=1-2, to=1-3]
		\arrow["{\eta_T}", from=2-2, to=2-3]
		\arrow["{\eta_B}", from=3-2, to=3-3]
		\arrow[""{name=3, anchor=center, inner sep=0}, "{\eta_L}"', from=2-1, to=3-1]
		\arrow[""{name=4, anchor=center, inner sep=0}, "{\eta_L}"', from=2-2, to=3-2]
		\arrow[""{name=5, anchor=center, inner sep=0}, "{\eta_R}"', from=2-3, to=3-3]
		\arrow["{(1)}"{marking}, Rightarrow, draw=none, from=0, to=1]
		\arrow["{(2)}"{marking}, Rightarrow, draw=none, from=1, to=2]
		\arrow["{(3)}"{marking}, Rightarrow, draw=none, from=3, to=4]
		\arrow["{(4)}"{marking}, Rightarrow, draw=none, from=4, to=5]
	\end{tikzcd}\]
	We are being asked to show that the outer square commutes; we will show that each inner square commutes, which will be enough.
	\begin{enumerate}[label=(\arabic*)]
		\item This square commutes by the associativity of the cup product.
		\item This square commutes by functoriality of cup products.
		\item This square commutes by functoriality of cup products.
		\item This square commutes by functoriality of $\widehat H^{i+p}(G,-)$ applied to \autoref{eq:innermorphismcoherence}.
	\end{enumerate}
	The above checks complete the proof.
\end{proof}
We may now proceed directly with \autoref{thm:abstractperfectpairing}.
\begin{proof}[Proof of \autoref{thm:abstractperfectpairing}]
	We use the lemma to assert that, for any $u\in H^p(G,A)$, the diagram
	\[\begin{tikzcd}
		{\widehat H^{-p}(G,\op{Hom}_\ZZ(X,\ZZ))} & {\widehat H^0(G,\op{Hom}_\ZZ(X,A))} \\
		{\widehat H^0(G,\ZZ)} & {\widehat H^{p}(G,A)}
		\arrow["{-\cup u}", from=1-1, to=1-2]
		\arrow["{c\cup-}"', from=1-1, to=2-1]
		\arrow["{-\cup u}"', from=2-1, to=2-2]
		\arrow["{c\cup-}", from=1-2, to=2-2]
	\end{tikzcd}\]
	commutes. By hypothesis, the left and right arrows are isomorphisms, so the commutativity means that showing
	\[\arraycolsep=1.4pt\begin{array}{ccc}
		\widehat H^p(G,A) &\to& \op{Hom}_\ZZ\left(\widehat H^{-p}(G,\op{Hom}_\ZZ(X,\ZZ)),\widehat H^0(G,\op{Hom}_\ZZ(X,A))\right) \\
		u &\mapsto& (a\mapsto (a\cup u))
	\end{array}\]
	is an isomorphism is the same as showing that
	\[\arraycolsep=1.4pt\begin{array}{ccc}
		\widehat H^p(G,A) &\to& \op{Hom}_\ZZ\left(\widehat H^0(G,\ZZ),\widehat H^p(G,A)\right) \\
		u &\mapsto& (k\mapsto (k\cup u))
	\end{array}\]
	is an isomorphism. Setting $n\coloneqq\#G$, we see $\widehat H^0(G,\ZZ)=\ZZ/n\ZZ$, and the cup product we are looking at sends $k\in\ZZ/n\ZZ$ and $u\in\widehat H^2(G,A)$ to $k\cup u=ku$ by how the ``evaluation'' map $\ZZ\otimes_\ZZ A\simeq A$ behaves. Thus, we are showing that
	\[\arraycolsep=1.4pt\begin{array}{ccc}
		\widehat H^p(G,A) &\to& \op{Hom}_\ZZ\left(\ZZ/n\ZZ,\widehat H^p(G,A)\right) \\
		u &\mapsto& (k\mapsto ku)
	\end{array}\]
	is an isomorphism.
	
	However, $\widehat H^p(G,A)$ is $n$-torsion, so in fact maps $\ZZ\to\widehat H^p(G,A)$ automatically have $n\ZZ$ in their kernel and hence reduce to maps $\ZZ/n\ZZ\to\widehat H^p(G,A)$. Conversely, any map $\ZZ/n\ZZ\to\widehat H^p(G,A)$ can be extended by $\ZZ\onto\ZZ/n\ZZ$ to a map $\ZZ\to\widehat H^p(G,A)$, so we have a natural isomorphism
	\[\arraycolsep=1.4pt\begin{array}{ccc}
		\op{Hom}_\ZZ\left(\ZZ/n\ZZ,\widehat H^p(G,A)\right) &\simeq& \op{Hom}_\ZZ\left(\ZZ,\widehat H^p(G,A)\right) \\
		f &\mapsto& (k\mapsto f([k])) \\
		([k]\mapsto f(k)) &\mapsfrom& f.
	\end{array}\]
	In particular, it suffices to show that
	\[\arraycolsep=1.4pt\begin{array}{ccc}
		\widehat H^p(G,A) &\to& \op{Hom}_\ZZ\left(\ZZ,\widehat H^p(G,A)\right) \\
		u &\mapsto& (k\mapsto ku)
	\end{array}\]
	is an isomorphism. But this is a standard fact about the functor $\op{Hom}_\ZZ\colon\mathrm{AbGrp}\to\mathrm{AbGrp}$, so we are done.
\end{proof}
We now synthesize this with the theory we have been building.
\begin{cor}
	Let $G$ be a finite group, and let $X$ be a $p$-encoding module. Then, given a $G$-module $A$, the cup-product pairing induces an isomorphism
	\[\widehat H^p(G,A)\to\op{Hom}_\ZZ\left(\widehat H^{-p}(G,\op{Hom}_\ZZ(X,\ZZ)),\widehat H^0(G,\op{Hom}_\ZZ(X,A))\right).\]
\end{cor}
\begin{proof}
	We apply \autoref{thm:abstractperfectpairing} to our case; we take $c$ to be the $x$ of \autoref{cor:encodingsarecups}. The cup-product maps in question are isomorphisms by \autoref{cor:betterencodingdef}. Thus, \autoref{thm:abstractperfectpairing} kicks in, completing the proof.
\end{proof}
\begin{remark}
	The other side of the pairing
	\[\widehat H^{-2}(G,\op{Hom}_\ZZ(X,\ZZ))\to\op{Hom}_\ZZ\left(\widehat H^2(G,A),\widehat H^0(G,\op{Hom}_\ZZ(X,A))\right)\]
	need not be an isomorphism; for example, take $A=0$.
\end{remark}
\begin{remark} \label{rem:artinreciptaste}
	When $X$ is $\ZZ$-free, we can think about $\op{Hom}_\ZZ(X,-)$ as a torus $T$. For example, if $L/K$ is an extension of local fields, and the torus $T$ splits over $L$, then the above statement says that the Artin reciprocity map
	\[\widehat H^{-2}(L/K,X_*(T))\to\widehat H^0(L/K,TL)\]
	uniquely determines $u_{L/K}\in\widehat H^2(L/K,L^\times)$. In theory, a concrete description of this reciprocity map might then be able to describe $u_{L/K}$.
\end{remark}

\section{General Group Extensions} \label{sec:general}
% !TEX root = ../abeliangerbs.tex

Having established some background of what we expect from our encoding modules, we will spend the next few sections building a particularly nice example of a $2$-encoding module with ties to classifying group extensions.

Much of the theory in this section will be similar to that built in \cite{abelian-crossed} and \cite{cohom-abelian-crossed}. In particular, providing a group law for the extensions built from our $G$-module $A$ is essentially the same problem as being able to write down a group law for abelian crossed products. Regardless, we will build the theory from the ground.

\subsection{Motivating Results} \label{sec:singlevar}
Throughout this section, $ G$ will be a finite group and $A$ will be a $ G$-module; we will write the group operation of $A$ and the group action of $ G$ on $A$ multiplicatively. To sketch the idea here, begin with an extension
\[1\to A\to\mc E\stackrel\pi\to G\to1.\]
We know that we can abstractly represent $\mc E$ as the set $A\times G$ with some group law dictated by a $2$-cocycle in $Z^2(G,A)$, so we expect that $\mc E$ can be presented by $A$ and a choice of lifts from $ G$, with some specially chosen relations.

Here are some basic observations realizing this idea. We start by lifting a single element of $ G$.
\begin{lemma} \label{lem:constructalpha}
	Let $A$ be a $ G$-module, and let 
	\[1\to A\to\mc E\stackrel\pi\to G\to1\]
	denote a group extension. Further, fix some $\sigma\in G$ of order $n_\sigma$, and find $F\in\mc E$ such that $\sigma\coloneqq\pi(F)$. Then
	\[\alpha\coloneqq F^{n_\sigma}\]
	has $\alpha\in A^{\langle\sigma\rangle}$.
\end{lemma}
\begin{proof}
	A priori, we only know that $\alpha\in\mc E$, so we compute
	\[\pi(\alpha)=\pi\left(F^{n_\sigma}\right)=\sigma^{n_\sigma}=1,\]
	so $\alpha\in\ker\pi=A$. Thus, we may say that
	\[\sigma(\alpha)=F\alpha F^{-1}=F^{n_\sigma}=\alpha,\]
	so $\alpha\in A^{\langle\sigma\rangle}$, as desired.
\end{proof}
We can make the above proof more explicit by specifying the group law of $\mc E$.
\begin{lemma} \label{lem:explicitalpha}
	Let $A$ be a $ G$-module. Picking up some $2$-cocycle $c\in Z^2( G,A)$, let
	\[1\to A\to\mc E_c\stackrel\pi\to G\to1\]
	be the corresponding extension. Fixing $\sigma\in G$ of order $n_\sigma$, let $F\coloneqq(m,\sigma)\in\mc E_c$ be a lift. Supposing $c(1,\sigma)=1$, then
	\[\alpha\coloneqq F^{n_\sigma}=N_\sigma(m)\prod_{i=0}^{n_\sigma-1}c\left(\sigma^i,\sigma\right),\]
	where $N_\sigma\coloneqq\sum_{i=0}^{n_\sigma-1}\sigma^i$.
\end{lemma}
\begin{proof}
	This is a direct computation. By induction, we can show that
	\[F^k=\left(\prod_{i=0}^{k-1}\sigma^i(m)c\left(\sigma^i,\sigma\right),\sigma^k\right)\]
	for $k\in\NN$. Indeed, there is nothing to say for $k=0$, and the inductive step merely expands out $F^k\cdot F$.

	It follows that
	\[\alpha=F^{n_\sigma}=\left(\prod_{i=0}^{n_\sigma-1}\sigma^i(m)\cdot\prod_{i=0}^{n_\sigma-1}c\left(\sigma^i,\sigma\right),1\right),\]
	which is what we wanted.
\end{proof}
Having this explicit formula lets us say how $\alpha$ changes as we vary the lift.
\begin{prop} \label{prop:findallalpha}
	Let $A$ be a $ G$-module. Fixing a cohomology class $u\in H^2( G,A)$, let 
	\[1\to A\to\mc E\stackrel\pi\to G\to1\]
	be a group extension whose isomorphism class corresponds to $u$. Further, fix some $\sigma\in G$ of order $n_\sigma$, and let $A_\sigma\coloneqq A^{\langle\sigma\rangle}$ be the fixed submodule. Then the set
	\[S_{\mc E,\sigma}\coloneqq\left\{F^{n_\sigma}:\pi(F)=\sigma\right\}\]
	is an equivalence class in $A_\sigma/N_\sigma(A)$, independent of the choice of $\mc E$, where $N_\sigma\coloneqq\sum_{i=1}^{n_\sigma-1}\sigma^i$.
\end{prop}
\begin{proof}
	Note that $S_{\mc E,\sigma}\subseteq A_\sigma$ already from \autoref{lem:constructalpha}.
	
	The point is to use \autoref{lem:explicitalpha}. Note the extension $\mc E$ corresponds to the equivalence class $u\in H^2( G,A)$, so let $c\in Z^2( G,A)$ be a representative. Letting $\mc E_c$ be the extension constructed from $c$, we are promised an isomorphism $\varphi\colon\mc E\cong\mc E_c$ making the following diagram commute.
	% https://q.uiver.app/?q=WzAsMTAsWzAsMCwiMSJdLFsxLDAsIkxeXFx0aW1lcyJdLFsyLDAsIlxcbWMgRSJdLFszLDAsIlxcR2FtbWEiXSxbNCwwLCIxIl0sWzAsMSwiMSJdLFsxLDEsIkxeXFx0aW1lcyJdLFsyLDEsIlxcbWMgRV9jIl0sWzMsMSwiXFxHYW1tYSJdLFs0LDEsIjEiXSxbMCwxXSxbMSwyXSxbMiwzLCJcXHBpIl0sWzMsNF0sWzUsNl0sWzYsN10sWzcsOCwiXFxwaV9jIl0sWzgsOV0sWzIsNywiXFx2YXJwaGkiXSxbMSw2LCIiLDEseyJsZXZlbCI6Miwic3R5bGUiOnsiaGVhZCI6eyJuYW1lIjoibm9uZSJ9fX1dLFszLDgsIiIsMSx7ImxldmVsIjoyLCJzdHlsZSI6eyJoZWFkIjp7Im5hbWUiOiJub25lIn19fV1d&macro_url=https%3A%2F%2Fraw.githubusercontent.com%2FdFoiler%2Fnotes%2Fmaster%2Fnir.tex
	\[\begin{tikzcd}
		1 & {A} & {\mc E} &  G & 1 \\
		1 & {A} & {\mc E_c} &  G & 1
		\arrow[from=1-1, to=1-2]
		\arrow[from=1-2, to=1-3]
		\arrow["\pi", from=1-3, to=1-4]
		\arrow[from=1-4, to=1-5]
		\arrow[from=2-1, to=2-2]
		\arrow[from=2-2, to=2-3]
		\arrow["{\pi_c}", from=2-3, to=2-4]
		\arrow[from=2-4, to=2-5]
		\arrow["\varphi", from=1-3, to=2-3]
		\arrow[Rightarrow, no head, from=1-2, to=2-2]
		\arrow[Rightarrow, no head, from=1-4, to=2-4]
	\end{tikzcd}\]
	We start by claiming that $S_{\mc E,\sigma}=S_{\mc E_c,\sigma}$, which will show that $S_{\mc E,\sigma}$ is independent of the choice of representative $\mc E$. To show $S_{\mc E,\sigma}\subseteq S_{\mc E_c,\sigma}$, note that $\alpha\in S_{\mc E,\sigma}$ has $F\in\mc E$ with $\pi(F)=\sigma$ and $\alpha=F^{n_\sigma}$. Pushing this through $\varphi$, we see $\varphi(F)\in\mc E_c$ has
	\[\pi_c(\varphi(F))=\varphi(\pi(F))=\sigma\qquad\text{and}\qquad\varphi(F)^{n_\sigma}=\varphi(F^{n_\sigma})=\alpha,\]
	so $\alpha\in S_{\mc E_c,\sigma}$ follows. An analogous argument with $\varphi^{-1}$ shows the other needed inclusion.

	It thus suffices to show that $S_{\mc E_c,\sigma}$ is an equivalence class in $A_\sigma/N_\sigma(A)$. However, this is exactly what \autoref{lem:explicitalpha} says as we let the possible lifts $F=(m,\sigma)\in\mc E_c$ of $\sigma$ vary over $m\in A$.
\end{proof}
The fact that we are taking elements of $ G$ to equivalence classes in $A_\sigma/N_\sigma\left(A\right)$ is reminiscent of the (inverse) Artin reciprocity map, and indeed that is exactly what is going on.
\begin{cor} \label{cor:alphaiscupproduct}
	Work in the context of \autoref{prop:findallalpha}. Then
	\[S_\sigma\coloneqq S_{\mc E,\sigma}=[\sigma]\cup[\op{Res}c],\]
	where $\cup\colon\widehat H^{-2}(\langle\sigma\rangle,\ZZ)\times\widehat H^2(\langle\sigma\rangle,A)\to\widehat H^0(\langle\sigma\rangle,A)$ is the cup product in Tate cohomology.
\end{cor}
\begin{proof}
	Note that $S_\sigma\in A_\sigma/N_\sigma(A)=\widehat H^0(\langle\sigma\rangle,A)$, so the conclusion at least makes sense.
	
	Now, using notation as in the proof of \autoref{prop:findallalpha}, we recall that $S_\sigma=S_{\mc E_c,\sigma}$, so it suffices to prove the result for $\mc E_c$. Well, by \autoref{lem:explicitalpha}, $S_\sigma$ is represented by
	\[\prod_{i=0}^{n_\sigma-1}c\left(\sigma^i,\sigma\right),\]
	which is exactly the cup product $[\sigma]\cup[c]$.
\end{proof}
\begin{cor}
	Let $L/K$ be a finite Galois extension of local fields with Galois group $ G\coloneqq\op{Gal}(L/K)$. Further, let
	\[1\to L^\times\to\mc E\stackrel\pi\to G\to1\]
	be an $L/K$-gerb bound by $\mathbb G_m$ whose isomorphism class corresponds to the fundamental class $u_{L/K}\in H^2( G,L^\times)$. Further, fix some $\sigma\in G$ of order $n_\sigma$, and let $L_\sigma\coloneqq L^{\langle\sigma\rangle}$ be the fixed field. Then
	\[\theta_{L/L_\sigma}^{-1}(\sigma)=\left\{F^{n_\sigma}:\pi(F)=\sigma\right\}.\]
\end{cor}
\begin{proof}
	Recalling $\theta_{L/L_\sigma}^{-1}$ is a cup product map, note that $\theta_{L/L_\sigma}^{-1}(\sigma)$ is given by $[\sigma]\cup u_{L/K}$. So we are done by \autoref{cor:alphaiscupproduct}.
\end{proof}
The above results are all interested in lifting single elements of $ G$ and studying how they behave on their own. In the discussion that follows, we will need to study how the lifts interact with each other, but for now, we will justify why lifts are adequate to study as follows.
\begin{proposition} \label{prop:liftsgenerate}
	Let $A$ be a $ G$-module. Further, let
	\[1\to A\to\mc E\stackrel\pi\to G\to1\]
	be a group extension. Given elements $\Sigma$ which generate $ G$, then $\mc E$ is generated by $A$ and a set of lifts $\{F_\sigma\}_{\sigma\in\Sigma}$ with $\pi(F_\sigma)=\sigma$ for each $\sigma\in\Sigma$.
\end{proposition}
\begin{proof}
	Fix some element $w\in\mc E$, which we need to exhibit as a product of elements in $A$ and $F_\sigma$s. Well, because the $\sigma\in\Sigma$ generate $ G$, we know that $\pi(w)\in G$ can be written as
	\[\pi(w)=\prod_{\sigma\in\Sigma}^m\sigma^{a_\sigma}\]
	for some sequence of integers $\{a_\sigma\}_{\sigma\in\Sigma}\in\NN^{\oplus\Sigma}$. It follows that
	\[\pi\left(\frac w{\prod_{\sigma\in\Sigma}F_\sigma^{a_\sigma}}\right)=1,\]
	so $w/\prod_{\sigma\in\Sigma}F_\sigma^{a_\sigma}=\ker\pi=A$. Thus, we can find some $a\in A$ such that
	\[w=a\cdot\prod_{\sigma\in\Sigma}F_\sigma^{a_\sigma},\]
	which is what we wanted.
\end{proof}

\section{Abelian Group Extensions} \label{sec:abelian}
% !TEX root = ../abeliangerbs.tex

\subsection{Extensions to Tuples}
The above proofs technically don't even require that the group $ G$ is abelian. If we want to keep track of the fact our group is abelian, we should extract the elements of $A$ which can do so.
\begin{lemma} \label{lem:constructalphabeta}
	Let $A$ be a $ G$-module, and let 
	\[1\to A\to\mc E\stackrel\pi\to G\to1\]
	be a group extension. Further, fix some $F_1,F_2\in\mc E$ and define $\sigma_i\coloneqq\pi(F_i)$ for $i\in\{1,2\}$, and let $\sigma_i\in G$ have order $n_i$. Then, setting
	\[\alpha_i\coloneqq F_i^{n_i}\qquad\text{and}\qquad\beta\coloneqq F_1F_2F_1^{-1}F_2^{-1},\]
	we have the following.
	\begin{listalph}
		\item $\alpha_i\in A^{\langle\sigma_i\rangle}$ for $i\in\{1,2\}$ and $\beta\in A$.
		\item $N_1(\beta)=\alpha_1/\sigma_2(\alpha_1)$ and $N_2(\beta^{-1})=\alpha_2/\sigma_1(\alpha_2)$, where $N_i\coloneqq\sum_{p=0}^{n_i-1}\sigma_i^p$.
	\end{listalph}
\end{lemma}
\begin{proof}
	These checks are a matter of force. For brevity, we set $A_i\coloneqq A^{\langle\sigma_i\rangle}$ for $i\in\{1,2\}$.
	\begin{listalph}
		\item That $\alpha_i\in A_i$ follows from \autoref{lem:constructalpha}. Lastly, $\beta\in A$ follows from noting
		\[\pi(\beta)=\pi(F_1)\pi(F_2)\pi(F_1)^{-1}\pi(F_2)^{-1}=1,\]
		so $\beta\in\ker\pi=A$.
		\item We will check that $\op N_{L/L_1}(\beta)=\alpha_1/\sigma_2(\alpha_1)$; the other equality follows symmetrically after switching $1$s and $2$s because $\beta^{-1}=F_2F_1F_2^{-1}F_1^{-1}$. Well, we compute
		\begin{align*}
			N_1(\beta) &= \sigma_1^{-1}(\beta)\cdot\sigma_1^{-2}(\beta)\cdot\sigma^{-3}\cdot\ldots\cdot\sigma^{-n_1}(\beta) \\
			&= F_1^{-1}\left(F_1F_2F_1^{-1}F_2^{-1}\right)F_1 \\
			&\phantom{{}={}}\cdot F_1^{-2}\left(F_1F_2F_1^{-1}F_2^{-1}\right)F_1^2 \\
			&\phantom{{}={}}\cdot F_1^{-3}\left(F_1F_2F_1^{-1}F_2^{-1}\right)F_1^3\cdot\ldots \\
			&\phantom{{}={}}\cdot F_1^{-n_1}(F_1F_2F_1^{-1}F_2^{-1})F_1^{n_1} \\
			% &= F_2F_1^{-1}F_2^{-1} \\
			% &\phantom{{}={}}\cdot F_2F_1^{-1}F_2^{-1} \\
			% &\phantom{{}={}}\cdot F_2F_1^{-1}F_2^{-1}\cdot\ldots \\
			% &\phantom{{}={}}\cdot F_2F_1^{-1}F_2^{-1}F_1^{n_1} \\
			&= F_2F_1^{-1} \\
			&\phantom{{}={}}\cdot F_1^{-1} \\
			&\phantom{{}={}}\cdot F_1^{-1}\cdot\ldots \\
			&\phantom{{}={}}\cdot F_1^{-1}F_2^{-1}F_1^{n_1} \\
			&= F_2F_1^{-n_1}F_2^{-1}F_1^{n_1} \\
			&= \alpha_1/\sigma_2(\alpha_1).
		\end{align*}
	\end{listalph}
	The above computations finish the proof.
\end{proof}
The proof of (b) above might appear magical, but in fact it comes from a more general idea.
\begin{lemma} \label{lem:switchtwo}
	Fix everything as in \autoref{lem:constructalphabeta}. Then, for $x,y\ge0$, we have
	\[F_1^xF_2^y=\prod_{k=0}^{x-1}\prod_{\ell=0}^{y-1}\sigma_1^k\sigma_2^\ell(\beta)F_2^yF_1^x.\]
\end{lemma}
\begin{proof}
	We induct. We take a moment to write out the case of $x=1$, for which we induct on $y$. To be explicit, we will prove
	\[F_1F_2^y=\prod_{\ell=0}^{y-1}\sigma_2^\ell(\beta)F_2^yF_1.\]
	For $y=0$, there is nothing to say. So suppose the statement for $y$ (and $x=1$), and we show $y+1$ (and $x=1$). Well, we compute
	\begin{align*}
		F_1F_2^{y+1} &= F_1F_2^y\cdot F_2 \\
		&= \prod_{\ell=0}^{y-1}\sigma_2^\ell(\beta)F_2^yF_1\cdot F_2 \\
		&= \prod_{\ell=0}^{y-1}\sigma_2^\ell(\beta)F_2^y\beta F_2F_1 \\
		&= \prod_{\ell=0}^{y-1}\sigma_2^\ell(\beta)\cdot \sigma_2^y(\beta)F_2^y\cdot F_2F_1 \\
		&= \prod_{\ell=0}^{(y+1)-1}\sigma_2^\ell(\beta)\cdot F_2^{y+1}F_1,
	\end{align*}
	which is what we wanted.
	
	We now move on to the general case. We will induct on $y$. Note that $y=0$ makes the product empty, leaving us with $F_1^x=F_1^x$, for any $x$. So suppose that the statement is true for some $y\ge0$, and we will show $y+1$. For this, we now turn to inducting on $x$. For $x=0$, we note that the product is once again empty, so we are left with showing $F_2^{y+1}=F_2^{y+1}$, which is true.
	
	To finish, we suppose the statement for $x$ and show the statement for $x+1$. Well, we compute
	\begin{align*}
		F_1^{x+1}F_2^{y+1} &= F_1\cdot F_1^xF_2^{y+1} \\
		&= F_1\cdot \prod_{k=0}^{x-1}\prod_{\ell=0}^{(y+1)-1}\sigma_1^k\sigma_2^\ell(\beta)\cdot F_2^{y+1}F_1^x \\
		&= \sigma_1\left(\prod_{k=0}^{x-1}\prod_{\ell=0}^{(y+1)-1}\sigma_1^k\sigma_2^\ell(\beta)\right)\cdot F_1F_2^{y+1}F_1^x \\
		&= \prod_{k=1}^{(x+1)-1}\prod_{\ell=0}^{(y+1)-1}\sigma_1^k\sigma_2^\ell(\beta)\cdot F_1F_2^{y+1}F_1^x \\
		&= \prod_{k=1}^{(x+1)-1}\prod_{\ell=0}^{(y+1)-1}\sigma_1^k\sigma_2^\ell(\beta)\cdot \prod_{\ell=0}^{(y+1)-1}\sigma_2^\ell(\beta)\cdot \sigma_2^y(\beta)\cdot F_2^{y+1}F_1\cdot F_1^x \\
		&= \prod_{k=0}^{(x+1)-1}\prod_{\ell=0}^{(y+1)-1}\sigma_1^k\sigma_2^\ell(\beta)F_2^{y+1}F_1^{x+1},
	\end{align*}
	which is what we wanted.
\end{proof}
\begin{remark} \label{rem:alphabetarelation}
	Setting $x=n_1$ and $y=1$ recovers $\op N_{L/L^{\langle\sigma_1\rangle}}(\beta)=\alpha_1/\sigma_2(\alpha_1)$.
\end{remark}
In particular, \autoref{rem:alphabetarelation} tells us that coherence of the group law in $\mc E$ should give rise to relations between our elements of $A$. Here is a more complex example.
\begin{lemma} \label{lem:betarelations}
	Let $A$ be a $ G$-module, and let 
	\[1\to A\to\mc E\stackrel\pi\to G\to1\]
	be a group extension. Further, fix some $F_1,F_2,F_3\in\mc E$ and define $\sigma_i\coloneqq\pi(F_i)$ for $i\in\{1,2,3\}$, and let $\sigma_i\in G$ have order $n_i$. Then, setting
	\[\beta_{ij}\coloneqq F_iF_jF_i^{-1}F_j^{-1}\]
	for each pair of indices $(i,j)$ with $i>j$. Then
	\[\frac{\sigma_2(\beta_{31})}{\beta_{31}}=\frac{\sigma_1(\beta_{32})}{\beta_{32}}\cdot\frac{\sigma_3(\beta_{21})}{\beta_{21}}.\]
\end{lemma}
\begin{proof}
	The point is to turn $F_3F_2F_1$ into $F_1F_2F_3$ in two different ways. On one hand,
	\begin{align*}
		(F_3F_2)F_1 &= \beta_{32}F_2F_3F_1 \\
		&= \beta_{32}F_2\beta_{31}F_1F_3 \\
		&= \beta_{32}\sigma_2(\beta_{31})(F_2F_1)F_3 \\
		&= \beta_{32}\sigma_2(\beta_{31})\beta_{21}F_1F_2F_3.
	\end{align*}
	On the other hand,
	\begin{align*}
		F_3(F_2F_1) &= F_3\beta_{21}F_1F_2 \\
		&= \sigma_3(\beta_{21})(F_3F_1)F_2 \\
		&= \sigma_3(\beta_{21})\beta_{31}F_1(F_3F_2) \\
		&= \sigma_3(\beta_{21})\beta_{31}F_1\beta_{32}F_2F_3 \\
		&= \sigma_3(\beta_{21})\beta_{31}\sigma_1(\beta_{32})F_1F_2F_3.
	\end{align*}
	Thus,
	\[\beta_{32}\sigma_2(\beta_{31})\beta_{21}=\sigma_3(\beta_{21})\beta_{31}\sigma_1(\beta_{32}),\]
	which rearranges into the desired equation.
\end{proof}
\begin{remark}
	The relation from \autoref{lem:betarelations} may look asymmetric in the $\beta_{ij}$, but this is because the definitions of the $\beta_{ij}$s themselves are asymmetric in $F_i$.
\end{remark}

\subsection{Tuples to Cocycles}
\subsubsection{The Set-Up}
The proceeding lemma is intended to give intuition that the element $\beta$ is helping to specify the group law on $\mc E$.

More concretely, we will take the following set-up for the following results: fix a $ G$-module $A$, and let
\[1\to A\to\mc E\to G\to1\]
be a group extension. Once we choose elements $\{\sigma_i\}_{i=1}^m$ generating $ G$, we know by \autoref{prop:liftsgenerate} that we can generate $\mc E$ by $A$ and some arbitrarily chosen lifts $\{F_i\}_{i=1}^m$ of the $\{\sigma_i\}_{i=1}^m$. Then, letting $n_i$ be the order of $\sigma_i$, we set
\[\alpha_i\coloneqq F_i^{n_i}\]
for each index $i$ and
\[\beta_{ij}\coloneqq F_iF_jF_i^{-1}F_j^{-1}\]
for each index $1\le j<i\le m$. Notably, we will not need more $\beta$s: indeed, $\beta_{ii}=1$ and $\beta_{ij}=\beta_{ji}^{-1}$ for any $i$ and $j$. Setting $A_i\coloneqq A^{\langle\sigma_i\rangle}$ and $N_i\coloneqq\sum_{p=0}^{n_i-1}\sigma_i^p$, the story so far is that
\begin{equation}
	\alpha_i\in A_i\text{ for each }i\qquad\text{and}\qquad\beta_{ij}\in A\text{ for each }i>j \label{eq:tuplefields}
\end{equation}
and
\begin{equation}
	N_i(\beta_{ij})=\alpha_i/\sigma_j(\alpha_i)\qquad\text{and}\qquad N_j(\beta_{ij}^{-1})=\alpha_j/\sigma_i(\alpha_j)\qquad\text{ for each }i>j \label{eq:tuplerelations}
\end{equation}
by \autoref{lem:constructalphabeta}, and
\begin{equation}
	\frac{\sigma_j(\beta_{ik})}{\beta_{ik}}=\frac{\sigma_k(\beta_{ij})}{\beta_{ij}}\cdot\frac{\sigma_i(\beta_{jk})}{\beta_{jk}}\qquad\text{ for each }i>j>k \label{eq:betarelations}
\end{equation}
by \autoref{lem:betarelations}. This data is so important that we will give it a name.
\begin{definition}
	In the above set-up, the data of $(\{\alpha_i\},\{\beta_{ij}\})$ satisfying \autoref{eq:tuplefields} and \autoref{eq:tuplerelations} and \autoref{eq:betarelations} will be called a \textit{$\{\sigma_i\}_{i=1}^m$-tuple}. When understood, the $\{\sigma_i\}_{i=1}^m$ will be abbreviated. Once $G$ and $A$ are fixed, we will denote the set of $\{\sigma_i\}_{i=1}^m$-tuples by $\mathcal T(G,A)$.
\end{definition}
Note that this definition is independent of $\mc E$, but a choice of extension $\mc E$ and lifts $F_i$ give a $\{\sigma_i\}_{i=1}^m$-tuple as described above.
\begin{remark}
	The $\mathcal T(G,A)$ form a group under multiplication in $A$. Indeed, the conditions \autoref{eq:tuplefields} and \autoref{eq:tuplerelations} and \autoref{eq:betarelations} are closed under multiplication and inversion.
\end{remark}
We also know from \autoref{lem:switchtwo} that
\[F_i^xF_j^y=\prod_{k=0}^{x-1}\prod_{\ell=0}^{y-1}\sigma_i^k\sigma_j^\ell(\beta_{ij})F_j^yF_i^x\]
for $i>j$ and $x,y\ge0$. It will be helpful to have some notation for the residue term in $A$, so we define
\[\beta_{ij}^{(xy)}\coloneqq\prod_{k=0}^{x-1}\prod_{\ell=0}^{y-1}\sigma_i^k\sigma_j^\ell(\beta_{ij}).\]
Now, combined with the fact that $F_ix=\sigma_i(x)F_i$ for each $F_i$ and $x\in A$, we have been approximately told how the group operation works in $\mc E$. Namely, we could conceivably write any element of $\mc E$ in the form
\[xF_1^{a_1}\cdots F_m^{a_m}\]
for $x\in A$ and $a_i\in\ZZ/n_i\ZZ$ because we know how to make these elements commute and generate $\mc E$. Further, we can multiply out two terms of the form
\[xF_1^{a_1}\cdots F_m^{a_m}\cdot yF_1^{b_1}\cdots F_m^{b_m}\]
into a term of the form $zF_1^{c_1}\cdots F_m^{c_m}$. In fact, it will be helpful for us to see how to do this.
\begin{proposition} \label{prop:multiplytwoelements}
	Fix everything as in the set-up, except drop the assumption that $\{\sigma_i\}_{i=1}^m$ generate $ G$. Then, choosing $a_i,b_i\in\NN$ for each $i$, we have
	\[\left(\prod_{i=1}^mF_i^{a_i}\right)\left(\prod_{i=1}^mF_i^{b_i}\right)=\left[\prod_{1\le j<i\le m}\Bigg(\prod_{1\le k<j}\sigma_k^{a_k+b_k}\Bigg)\Bigg(\prod_{j\le k<i}\sigma_k^{a_k}\Bigg)\beta_{ij}^{(a_ib_j)}\right]\left(\prod_{i=1}^mF_i^{a_i+b_i}\right).\]
\end{proposition}
\begin{proof}
	The reason that we dropped the assumption on $\{\sigma_i\}_{i=1}^m$ is so that we may induct directly on $m$. We start by showing that
	\[\left(\prod_{i=1}^mF_i^{a_i}\right)F_1^{b_1}=\left[\prod_{1<i\le m}\left(\prod_{1\le k<i}\sigma_k^{a_k}\right)\beta_{i1}^{(a_ib_1)}\right]F_1^{a_1+b_1}\prod_{i=2}^mF_i^{a_i}.\]
	We do this by induction on $m$. When $m=0$ and even for $m=1$, there is nothing to say. For the inductive step, we assume
	\[\left(\prod_{i=1}^mF_i^{a_i}\right)F_1^{b_1}=\left[\prod_{1<i\le m}\left(\prod_{1\le k<i}\sigma_k^{a_k}\right)\beta_{i1}^{(a_ib_1)}\right]F_1^{a_1+b_1}\prod_{i=2}^mF_i^{a_i}\]
	and compute
	\begin{align*}
		\left(\prod_{i=1}^{m+1}F_i^{a_i}\right)F_1^{b_1} &= \left(\prod_{i=1}^{m}F_i^{a_i}\right)F_{m+1}^{a_{m+1}}F_1^{b_1} \\
		&= \left(\prod_{i=1}^{m}F_i^{a_i}\right)\beta_{m+1,1}^{(a_{m+1}b_1)}F_1^{b_1}F_{m+1}^{a_{m+1}} \\
		&= \left[\left(\prod_{k=1}^m\sigma_k^{a_k}\right)\beta_{m+1,1}^{(a_{m+1}b_1)}\right]\left[\prod_{1<i\le m}\left(\prod_{1\le k<i}\sigma_k^{a_k}\right)\beta_{i1}^{(a_ib_1)}\right]F_1^{a_1+b_1}\left(\prod_{i=2}^mF_i^{a_i}\right)F_{m+1}^{a_{m+1}} \\
		&= \left[\prod_{1<i\le m+1}\left(\prod_{1\le k<i}\sigma_k^{a_k}\right)\beta_{i1}^{(a_ib_1)}\right]F_1^{a_1+b_1}\left(\prod_{i=2}^{m+1}F_i^{a_i}\right),
	\end{align*}
	which completes our inductive step.

	We now attack the statement of the proposition directly, again inducting on $m$. For $m=0$ and even for $m=1$, there is again nothing to say. For the inductive step, take $m>1$, and we get to assume that
	\[\left(\prod_{i=2}^mF_i^{a_i}\right)\left(\prod_{i=2}^mF_i^{b_i}\right)=\left[\prod_{2\le j<i\le m}\Bigg(\prod_{2\le k<j}\sigma_k^{a_k+b_k}\Bigg)\Bigg(\prod_{j\le k<i}\sigma_k^{a_k}\Bigg)\beta_{ij}^{(a_ib_j)}\right]\left(\prod_{i=2}^mF_i^{a_i+b_i}\right).\]
	From here, we can compute
	\begin{align*}
		\left(\prod_{i=1}^mF_i^{a_i}\right)\left(\prod_{i=1}^mF_i^{b_i}\right) &= \left(\prod_{i=1}^mF_i^{a_i}\right)F_1^{b_1}\left(\prod_{i=2}^mF_i^{b_i}\right) \\
		&= \left[\prod_{1<i\le m}\Bigg(\prod_{1\le k<i}\sigma_k^{a_k}\Bigg)\beta_{i1}^{(a_ib_1)}\right]F_1^{a_1+b_1}\left(\prod_{i=2}^mF_i^{a_i}\right)\left(\prod_{i=2}^mF_i^{b_i}\right) \\
		&= \left[\prod_{1<i\le m}\Bigg(\prod_{1\le k<i}\sigma_k^{a_k}\Bigg)\beta_{i1}^{(a_ib_1)}\right]F_1^{a_1+b_1}\cdot \\
		&\qquad\qquad\left[\prod_{2\le j<i\le m}\Bigg(\prod_{2\le k<j}\sigma_k^{a_k+b_k}\Bigg)\Bigg(\prod_{j\le k<i}\sigma_k^{a_k}\Bigg)\beta_{ij}^{(a_ib_j)}\right]\left(\prod_{i=2}^mF_i^{a_i+b_i}\right) \\
		&= \left[\prod_{1<i\le m}\Bigg(\prod_{1\le k<i}\sigma_k^{a_k}\Bigg)\beta_{i1}^{(a_ib_1)}\right]\cdot \\
		&\qquad\qquad \sigma_1^{a_1+b_1}\left[\prod_{2\le j<i\le m}\Bigg(\prod_{2\le k<j}\sigma_k^{a_k+b_k}\Bigg)\Bigg(\prod_{j\le k<i}\sigma_k^{a_k}\Bigg)\beta_{ij}^{(a_ib_j)}\right]\left(\prod_{i=2}^mF_i^{a_i+b_i}\right).
	\end{align*}
	From here, a little rearrangement finishes the inductive step.
\end{proof}
The reason we exerted this pain upon ourselves is for the following result.
\begin{prop} \label{prop:writedowncocycle}
	Fix everything as in the set-up. Then, if well-defined, we can represent the cohomology class corresponding to $\mc E$ by the cocycle
	\[c(g,h)\coloneqq\left[\prod_{1\le j<i\le m}\Bigg(\prod_{1\le k<j}\sigma_k^{a_k+b_k}\Bigg)\Bigg(\prod_{j\le k<i}\sigma_k^{a_k}\Bigg)\beta_{ij}^{(a_ib_j)}\right]\left[\prod_{i=1}^m\Bigg(\prod_{1\le k<i}\sigma_k^{a_k+b_k}\Bigg)\alpha_i^{\floor{\frac{a_i+b_i}{n_i}}}\right],\]
	where $g=\prod_i\sigma_i^{a_i}$ and $h=\prod_i\sigma_i^{b_i}$.
\end{prop}
Observe that \autoref{prop:writedowncocycle} has a fairly strong hypothesis that $c$ is well-defined; we will return to this later.
\begin{proof}
	Very quickly, we use the division algorithm to define
	\[a_i+b_i=n_iq_i+r_i\]
	where $q_\in\{0,1\}$ and $0\le r_i<n_i$. In particular,
	\[gh=\prod_{i=1}^mF_i^{r_i}.\]
	Now, because the elements $\sigma_i$ generate $ G$, we see that the lifts $\sigma_i\mapsto F_i$ defines a section $s\colon G\to\mc E$. As such, we can compute a representing cocycle for our cohomology class as
	\begin{align*}
		c(g,h) &= s(g)s(h)s(gh)^{-1} \\
		&= \Bigg(\prod_{i=1}^mF_i^{a_i}\Bigg)\Bigg(\prod_{i=1}^mF_i^{b_i}\Bigg)\Bigg(\prod_{i=1}^mF_i^{r_i}\Bigg)^{-1} \\
		&= \left[\prod_{1\le j<i\le m}\Bigg(\prod_{1\le k<j}\sigma_k^{a_k+b_k}\Bigg)\Bigg(\prod_{j\le k<i}\sigma_k^{a_k}\Bigg)\beta_{ij}^{(a_ib_j)}\right]\left(\prod_{i=1}^mF_i^{a_i+b_i}\right)\Bigg(\prod_{i=1}^mF_{m-i+1}^{-r_{m-i+1}}\Bigg).
	\end{align*}
	It remains to deal with the last products; we claim that it is equal to
	\[\left(\prod_{i=1}^mF_i^{a_i+b_i}\right)\Bigg(\prod_{i=1}^mF_{m-i+1}^{-r_{m-i+1}}\Bigg)=\prod_{i=1}^m\Bigg(\prod_{1\le k<i}\sigma_k^{a_k+b_k}\Bigg)\alpha_i^{q_i},\]
	which will finish the proof. We induct on $m$; for $m=0$ and $m=1$, there is nothing to say. For the inductive step, we assume that
	\[\left(\prod_{i=2}^mF_i^{a_i+b_i}\right)\Bigg(\prod_{i=1}^{m-1}F_{m-i+1}^{-r_{m-i+1}}\Bigg)=\prod_{i=2}^m\Bigg(\prod_{2\le k<i}\sigma_k^{a_k+b_k}\Bigg)\alpha_i^{q_i}\]
	and compute
	\begin{align*}
		\left(\prod_{i=1}^mF_i^{a_i+b_i}\right)\Bigg(\prod_{i=1}^mF_{m-i+1}^{-r_{m-i+1}}\Bigg) &= F_1^{a_1+b_1}\left(\prod_{i=2}^mF_i^{a_i+b_i}\right)\Bigg(\prod_{i=1}^{m-1}F_{m-i+1}^{-r_{m-i+1}}\Bigg)F_1^{-a_1-b_1}F_1^{a_1+b_1-r_1} \\
		&= F_1^{a_1+b_1}\left(\prod_{i=2}^m\Bigg(\prod_{2\le k<i}\sigma_k^{a_k+b_k}\Bigg)\alpha_i^{q_i}\right)F_1^{-a_1-b_1}\alpha_1^{q_1} \\
		&= \left(\prod_{i=2}^m\Bigg(\prod_{1\le k<i}\sigma_k^{a_k+b_k}\Bigg)\alpha_i^{q_i}\right)\alpha_1^{q_1} \\
		&= \prod_{i=1}^m\Bigg(\prod_{1\le k<i}\sigma_k^{a_k+b_k}\Bigg)\alpha_i^{q_i},
	\end{align*}
	finishing.
\end{proof}

\subsubsection{The Modified Set-Up}
A priori we have no reason to expect that the $c$ constructed in \autoref{prop:writedowncocycle} is actually a cocycle, especially if the $\sigma_i$ have nontrivial relations.

To account for this, we modify our set-up slightly. By the classification of finitely generated abelian groups, we may write
\[ G\simeq\bigoplus_{k=1}^m G_k,\]
where $ G_k\subseteq G$ with $ G_k\cong\ZZ/n_k\ZZ$ and $n_k>1$ for each $n_k$. As such, we let $\sigma_k$ be a generating element of $ G_k$ so that we still know that the $\sigma_k$ generate $ G$. In this case, we have the following result.
\begin{theorem} \label{thm:getcocycle}
	Fix everything as in the modified set-up, forgetting about the extension $\mc E$. Then a $\{\sigma_i\}_{i=1}^m$-tuple of $\{\alpha_i\}_{i=1}^m$ and $\{\beta_{ij}\}_{i>j}$ makes
	\[c(g,h)\coloneqq\left[\prod_{1\le j<i\le m}\Bigg(\prod_{1\le k<j}\sigma_k^{a_k+b_k}\Bigg)\Bigg(\prod_{j\le k<i}\sigma_k^{a_k}\Bigg)\beta_{ij}^{(a_ib_j)}\right]\left[\prod_{i=1}^m\Bigg(\prod_{1\le k<i}\sigma_k^{a_k+b_k}\Bigg)\alpha_i^{\floor{\frac{a_i+b_i}{n_i}}}\right],\]
	where $g\coloneqq\prod_i\sigma_i^{a_i}$ with $h\coloneqq\prod_i\sigma_j^{a_j}$ and $0\le a_i,b_i<n_i$, into a cocycle in $Z^2( G,A)$.
\end{theorem}
\begin{proof}
	Note that $c$ is now surely well-defined because the elements $g$ and $h$ have unique representations as described. Anyway, we relegate the direct cocycle check to \autoref{sec:verifycocycle} because it is long, annoying, and unenlightening. We will also present an alternative proof in \autoref{sec:tuplestudy}, using more abstract theory.
\end{proof}
Observe that the above construction has now completely forgotten about $\mc E$! Namely, we have managed to go from tuples straight to cocycles; this is theoretically good because it will allow us to go fully in reverse: we will be able to start with a tuple, build the corresponding cocycle, from which the extension arises. However, equivalence classes of cocycles give the ``same'' extension, so we will also need to give equivalence classes for tuples as well.

\subsection{Building Tuples}
We continue in the modified set-up of the previous section. There is already an established way to get from a cocycle to an extension, which means that it should be possible to go straight from the cocycle to a $\{\sigma_i\}_{i=1}^m$-tuple. Again, it will be beneficial to write this out.
\begin{lemma} \label{lem:explicitalphabeta}
	Fix everything as in the modified set-up, but suppose that $\mc E=\mc E_c$ is the extension generated from a cocycle $c\in Z^2( G,A)$. Then, if $F_i=(x_i,\sigma_i)$ are our lifts, we have
	\[\alpha_i=N_i(x_i)\cdot\prod_{k=0}^{n_i-1}c\left(\sigma_i^k,\sigma_i\right)\qquad\text{and}\qquad\beta_{ij}=\frac{x_i}{\sigma_j(x_i)}\cdot\frac{\sigma_i(x_j)}{x_j}\cdot\frac{c(\sigma_i,\sigma_j)}{c(\sigma_j,\sigma_i)}\]
	for each $\alpha_i$ and $\beta_{ij}$.
\end{lemma}
\begin{proof}
	The equality for the $\alpha_i$ follow from \autoref{lem:explicitalpha}. For the equality about $\beta_{ij}$, we simply compute
	by brute force, writing
    \begin{align*}
        F_iF_j &= (x_i\cdot\sigma_ix_j\cdot c(\sigma_i,\sigma_j),\sigma_i\sigma_j) \\
        F_jF_i &= (x_j\cdot\sigma_jx_i\cdot c(\sigma_j,\sigma_i),\sigma_j\sigma_i) \\
        (F_jF_i)^{-1} &= \left((\sigma_j\sigma_i)^{-1}(x_j\cdot\sigma_jx_i\cdot c(\sigma_j,\sigma_i))^{-1},\sigma_i^{-1}\sigma_j^{-1}\right),
    \end{align*}
    which gives
    \begin{align*}
        \beta_{ij} &= (F_iF_j)(F_jF_i)^{-1} \\
        &= \left(\frac{x_i}{\sigma_jx_i}\cdot\frac{\sigma_ix_j}{x_j}\cdot\frac{c(\sigma_i,\sigma_j)}{c(\sigma_j,\sigma_i)},1\right),
    \end{align*}
	finishing.
\end{proof}
Here is a nice sanity check that we are doing things in the right setting: not only can we build tuples from extensions, but we can find an extension corresponding to any tuple.
\begin{cor} \label{cor:alltuplesfromextens}
	Fix everything as in the modified set-up, forgetting about the extension $\mc E$. For any $\{\sigma_i\}_{i=1}^m$-tuple of $\{\alpha_i\}_{i=1}^m$ and $\{\beta_{ij}\}_{i>j}$, there exists an extension $\mc E$ and lifts $F_i$ of the $\sigma_i$ so that
	\[\alpha_i=F_i^{n_i}\qquad\text{and}\qquad\beta_{ij}=F_iF_jF_i^{-1}F_j^{-1}.\]
\end{cor}
\begin{proof}
	From \autoref{thm:getcocycle}, we may build the cocycle $c\in Z^2( G,A)$ defined by
	\begin{equation}
		c(g,h)\coloneqq\left[\prod_{1\le j<i\le m}\Bigg(\prod_{1\le k<j}\sigma_k^{a_k+b_k}\Bigg)\Bigg(\prod_{j\le k<i}\sigma_k^{a_k}\Bigg)\beta_{ij}^{(a_ib_j)}\right]\left[\prod_{i=1}^m\Bigg(\prod_{1\le k<i}\sigma_k^{a_k+b_k}\Bigg)\alpha_i^{\floor{\frac{a_i+b_i}{n_i}}}\right], \label{eq:uglycocycle}
	\end{equation}
	where $g\coloneqq\prod_iF_i^{a_i}$ and $h\coloneqq\prod_iF_j^{a_j}$ and $0\le a_i,b_i<n_i$. As such, we use $\mc E\coloneqq\mc E_c$ to be the corresponding extension and $F_i\coloneqq(1,\sigma_i)$ as our lifts. We have the following checks.
	\begin{itemize}
		\item To show $\alpha_i=F_i^{n_i}$, we use \autoref{lem:explicitalphabeta} to compute $F_i^{n_i}$, which means we want to compute
		\[\prod_{k=0}^{n_i-1}c\left(\sigma_i^k,\sigma_i\right).\]
		Well, plugging $c\left(\sigma_i^k,\sigma_i\right)$ into \autoref{eq:uglycocycle}, we note that all $\beta_{k\ell}^{(a_kb_\ell)}$ terms vanish (either $a_k=0$ or $b_\ell=0$ for each $k\ne\ell$), so the big left product completely vanishes.
		
		As for the right product, the only term we have to worry about is
		\[\Bigg(\prod_{1\le k<i}\sigma_k^{0+0}\Bigg)\alpha_i^{\floor{\frac{k+1}{n_i}}},\]
		which is equal to $1$ when $k\le n_i-1$ and $\alpha_i$ when $k=n_i-1$. As such, we do indeed have $\alpha_i=F_i^{n_i}$.
		\item To show $\beta_{ij}=F_iF_jF_i^{-1}F_j^{-1}$ for $i>j$, we again use \autoref{lem:explicitalphabeta} to compute $F_iF_jF_i^{-1}F_j^{-1}$, which means we want to compute
		\[\frac{c(\sigma_i,\sigma_j)}{c(\sigma_j,\sigma_i)}.\]
		Plugging into \autoref{eq:uglycocycle} once more, there is no way to make $\floor{(a_k+b_k)/n_k}$ nonzero (recall we set $n_k>1$ for each $k$) in either $c(\sigma_i,\sigma_j)$ or $c(\sigma_j,\sigma_i)$. As such, the right-hand product term disappears.

		As for the left product, we note that it still vanishes for $c(\sigma_j,\sigma_i)$ because $i>j$ implies that either $a_k=0$ or $b_\ell=0$ for each $k>\ell$. However, for $c(\sigma_i,\sigma_j)$, we do have $a_i=1$ and $b_j=1$ only, so we have to deal with exactly the term
		\[\Bigg(\prod_{1\le k<j}\sigma_k^{a_k+b_k}\Bigg)\Bigg(\prod_{j\le k<i}\sigma_k^{a_k}\Bigg)\beta_{ij}.\]
		With $i>j$ and $a_k=b_k=0$ for $k\notin\{i,j\}$, we see that the product of all the $\sigma_k$s will disappear, indeed only leaving us with $\beta_{ij}$.
	\end{itemize}
	The above computations complete the proof.
\end{proof}
And here is our first taste of (partial) classification.
\begin{cor} \label{cor:cocycletuplesection}
	Fix everything as in the modified set-up, forgetting about the extension $\mc E$. Then the formula of \autoref{thm:getcocycle} and the formulae of \autoref{lem:explicitalphabeta} (setting $x_i=1$ for each $i$) are homomorphisms of abelian groups between tuples in $\mathcal T(G,A)$ and cocycles in $Z^2( G,A)$. In fact, the formula of \autoref{thm:getcocycle} is a section of the formulae of \autoref{lem:explicitalphabeta}.
\end{cor}
\begin{proof}
	The formulae in \autoref{thm:getcocycle} and \autoref{lem:explicitalphabeta} are both large products in their inputs, so they are multiplicative (i.e., homomorphisms). It remains to check that we have a section. Well, starting with a $\{\sigma_i\}_{i=1}^m$-tuple and building the corresponding cocycle $c$ by \autoref{thm:getcocycle}, the proof of \autoref{cor:alltuplesfromextens} shows that the formulae of \autoref{lem:explicitalphabeta} recovers the correct $\{\sigma_i\}_{i=1}^m$-tuple.
\end{proof}

\subsection{Equivalence Classes of Tuples}
We continue in the modified set-up. We would like to make \autoref{cor:cocycletuplesection} into a proper isomorphism of abelian groups, but this is not feasible; for example, the cocycle $c$ generated by \autoref{thm:getcocycle} will always have $c(\sigma_j,\sigma_i)=1$ for $i>j$, which is not true of all cocycles in $Z^2( G,A)$.

However, we did have a notion that the data of a $\{\sigma_i\}_{i=1}^m$ should be enough to specify the group law of the extension that the tuple comes from, so we do expect to be able to define all extensions---and hence achieve all cohomology classes---from a specially chosen $\{\sigma_i\}_{i=1}^m$-tuple.

To make this precise, we want to define an equivalence relation on tuples which go to the same cohomology class and then show that the map \autoref{thm:getcocycle} is surjective on these equivalence classes. The correct equivalence relation is taken from \autoref{lem:explicitalphabeta}.
\begin{definition}
	Fix everything as in the modified set-up. We say that two $\{\sigma_i\}_{i=1}^m$-tuples $(\{\alpha_i\},\{\beta_{ij}\})$ and $(\{\alpha_i'\},\{\beta_{ij}'\})$ are \textit{equivalent} if and only if there exist elements $x_1,\ldots,x_m\in A$ such that
	\[\alpha_i=N_i(x_i)\cdot\alpha_i'\qquad\text{and}\qquad\beta_{ij}=\frac{x_i}{\sigma_j(x_i)}\cdot\frac{\sigma_i(x_j)}{x_j}\cdot\beta_{ij}'\]
	for each $\alpha_i$ and $\beta_{ij}$. We may notate this by $(\{\alpha_i\},\{\beta_{ij}\})\sim(\{\alpha_i'\},\{\beta_{ij}'\})$.
\end{definition}
\begin{remark}
	It is not too hard to see directly from the definition that this is in fact an equivalence relation. In fact, the set of tuples equivalent to the ``trivial'' tuple of all $1$s is closed under multiplication (and inversion) and hence forms a subgroup of $\mathcal T(G,A)$. As such, the set of equivalence classes forms a quotient group of $\mathcal T(G,A)$. We will denote this quotient group by $\overline{\mathcal T}(G,A)$.
\end{remark}
This notion of equivalence can be seen to be the correct one in the sense that it correctly generalizes \autoref{prop:findallalpha}.
\begin{proposition} \label{prop:extenmakesaclass}
	Fix everything as in the modified set-up with an extension $\mc E$. As the lifts $F_i$ change, the corresponding values of
	\[\alpha_i\coloneqq F_i^{n_i}\qquad\text{and}\qquad\beta_{ij}\coloneqq F_iF_jF_i^{-1}F_j^{-1}\]
	go through a full equivalence class of $\{\sigma_i\}_{i=1}^m$-tuples.
\end{proposition}
\begin{proof}
	We proceed as in \autoref{prop:findallalpha}. Given an extension $\mc E'$, let $S_{\mc E'}$ be the set of $\{\sigma_i\}_{i=1}^m$-tuples generated as the lifts $F_i$ change. We start by showing that an isomorphism $\varphi\colon\mc E\simeq\mc E'$ of extensions implies that $S_{\mc E}=S_{\mc E'}$; by symmetry, it will be enough for $S_{\mc E}\subseteq S_{\mc E'}$. The isomorphism induces the following diagram.
	% https://q.uiver.app/?q=WzAsMTAsWzAsMCwiMSJdLFsxLDAsIkxeXFx0aW1lcyJdLFsyLDAsIlxcbWMgRSJdLFszLDAsIlxcR2FtbWEiXSxbNCwwLCIxIl0sWzAsMSwiMSJdLFsxLDEsIkxeXFx0aW1lcyJdLFsyLDEsIlxcbWMgRSciXSxbMywxLCJcXEdhbW1hIl0sWzQsMSwiMSJdLFswLDFdLFsxLDJdLFsyLDMsIlxccGkiXSxbMyw0XSxbNSw2XSxbNiw3XSxbNyw4LCJcXHBpJyJdLFs4LDldLFsyLDcsIlxcdmFycGhpIl0sWzEsNiwiIiwxLHsibGV2ZWwiOjIsInN0eWxlIjp7ImhlYWQiOnsibmFtZSI6Im5vbmUifX19XSxbMyw4LCIiLDEseyJsZXZlbCI6Miwic3R5bGUiOnsiaGVhZCI6eyJuYW1lIjoibm9uZSJ9fX1dXQ==&macro_url=https%3A%2F%2Fraw.githubusercontent.com%2FdFoiler%2Fnotes%2Fmaster%2Fnir.tex
	\[\begin{tikzcd}
		1 & {A} & {\mc E} &  G & 1 \\
		1 & {A} & {\mc E'} &  G & 1
		\arrow[from=1-1, to=1-2]
		\arrow[from=1-2, to=1-3]
		\arrow["\pi", from=1-3, to=1-4]
		\arrow[from=1-4, to=1-5]
		\arrow[from=2-1, to=2-2]
		\arrow[from=2-2, to=2-3]
		\arrow["{\pi'}", from=2-3, to=2-4]
		\arrow[from=2-4, to=2-5]
		\arrow["\varphi", from=1-3, to=2-3]
		\arrow[Rightarrow, no head, from=1-2, to=2-2]
		\arrow[Rightarrow, no head, from=1-4, to=2-4]
	\end{tikzcd}\]
	To show that $S_{\mc E}\subseteq S_{\mc E'}$, pick up some $\{\sigma_i\}_{i=1}^m$-tuple $(\{\alpha_i\},\{\beta_{ij}\})$ generated from lifts $F_i\in\mc E$ (i.e., $\pi(F_i)=\sigma_i$), where
	\[\alpha_i\coloneqq F_i^{n_i}\qquad\text{and}\qquad\beta_{ij}\coloneqq F_iF_jF_i^{-1}F_j^{-1}.\]
	Now, we note that $F_i'\coloneqq\varphi(F_i)$ will have
	\[\pi(F_i')=\pi(\varphi(F_i))=\varphi(\pi(F_i))=\sigma_i\]
	by the commutativity of the diagram, so the $F_i'$ are lifts of the $\sigma_i$. Further, we see that
	\[(F_i')^{n_i}=\varphi(F_i)^{n_i}=\varphi\left(F_i^{n_i}\right)=\varphi(\alpha_i)=\alpha_i\]
	for each $i$, and
	\[F_i'F_j'(F_i')^{-1}(F_j')^{-1}=\varphi\left(F_iF_jF_i^{-1}F_j^{-1}\right)=\varphi(\beta_{ij})=\beta_{ij}\]
	for each $i>j$. Thus, $(\{\alpha_i\},\{\beta_{ij}\})$ is a $\{\sigma_i\}_{i=1}^m$-tuple generated by lifts from $\mc E'$, implying that $(\{\alpha_i\},\{\beta_{ij}\})\in S_{\mc E'}$.

	It now suffices to show the statement in the proposition for a specific extension isomorphic to $\mc E$. Well, the isomorphism class of $\mc E$ corresponds to some cohomology class in $H^2( G,A)$, for which we let $c$ be a representative; then $\mc E\simeq\mc E_c$, so we may show the statement for $\mc E\coloneqq\mc E_c$. Indeed, as the lifts $F_i=(x_i,\sigma_i)$ change, we know by \autoref{lem:explicitalphabeta} that
	\[\alpha_i=N_i(x_i)\cdot\prod_{k=0}^{n_i-1}c\left(\sigma_i^k,\sigma_i\right)\qquad\text{and}\qquad\beta_{ij}=\frac{x_i}{\sigma_j(x_i)}\cdot\frac{\sigma_i(x_j)}{x_j}\cdot\frac{c(\sigma_i,\sigma_j)}{c(\sigma_j,\sigma_i)}\]
	for each $\alpha_i$ and $\beta_{ij}$. All of these live in the same equivalence class by definition of the equivalence, and as the $x_i$ are allowed to vary over all of $A$, they will fill up that equivalence class fully. This finishes.
\end{proof}
We are now ready to upgrade our section.
\begin{cor} \label{cor:cohomologymakesaclass}
	Fix everything as in the modified set-up, forgetting about the extension $\mc E$. Fixing a cohomology class $[c]\in H^2( G,A)$, the set of $\{\sigma_i\}_{i=1}^m$-tuples which correspond to $[c]$ (via \autoref{thm:getcocycle}) forms exactly one equivalence class.
\end{cor}
\begin{proof}
	We show that two tuples are equivalent if and only if their corresponding cocycles (via \autoref{thm:getcocycle}) to the same cohomology class, which will be enough.
	
	In one direction, suppose $(\{\alpha_i\},\{\beta_{ij}\})\sim(\{\alpha_i'\},\{\beta_{ij}'\})$. By \autoref{cor:alltuplesfromextens}, we can find an extension $\mc E$ which gives $(\{\alpha_i\},\{\beta_{ij}\})$ by choosing an appropriate set of lifts. By \autoref{prop:extenmakesaclass}, we see that $(\{\alpha_i'\},\{\beta_{ij}'\})$ must also come from choosing an appropriate set of lifts in $\mc E$. However, the cocycles in $Z^2( G,A)$ generated by \autoref{thm:getcocycle} from our two tuples now both represent the isomorphism class of $\mc E$ by \autoref{prop:writedowncocycle}, so these cocycles belong to the same cohomology class.

	In the other direction, name the cocycles corresponding to $(\{\alpha_i\},\{\beta_{ij}\})$ and $(\{\alpha_i'\},\{\beta_{ij}'\})$ by $c$ and $c'$ respectively, and suppose $[c]=[c']$. Then $\mc E_c\simeq\mc E_{c'}$ as extensions, but we know by the proof of \autoref{cor:alltuplesfromextens} that $(\{\alpha_i\},\{\beta_{ij}\})$ comes from choosing lifts of $\mc E_c$ and similar for $(\{\alpha_i'\},\{\beta_{ij}'\})$. In particular, because $\mc E_c\simeq\mc E_{c'}$, we know that $(\{\alpha_i'\},\{\beta_{ij}'\})$ will also come from choosing some lifts in $\mc E_c$ (recall the proof of \autoref{prop:extenmakesaclass}), so $(\{\alpha_i\},\{\beta_{ij}\})\sim(\{\alpha_i'\},\{\beta_{ij}'\})$ follows.
\end{proof}
\begin{theorem} \label{thm:classisomorphism}
	The maps described in \autoref{cor:cocycletuplesection} descend to an isomorphism of abelian groups between the equivalence classes in $\overline{\mathcal T}(G,A)$ and cohomology classes in $H^2( G,A)$.
\end{theorem}
\begin{proof}
	The fact that the maps are well-defined (in both directions) and hence injective is \autoref{cor:cohomologymakesaclass}. The fact that we had a section from tuples to cocycles implies that the map from cocycles to tuples was also surjective. Thus, we have a bona fide isomorphism.
\end{proof}

\subsection{Classification of Extensions}
We remark that we are now able to classify all extensions up to isomorphism, in some sense. At a high level, an isomorphism class of extensions corresponds to a particular cohomology class in $H^2( G,A)$, so choosing a $\{\sigma_i\}_{i=1}^m$-tuple $(\{\alpha_i\},\{\beta_{ij}\})$ corresponding to this class, we can write out a representative of this cocycle by \autoref{thm:getcocycle}, properly corresponding to the original extension by \autoref{prop:writedowncocycle}.

In fact, the cocycle in \autoref{prop:writedowncocycle} is generated by the description of the group law in \autoref{prop:multiplytwoelements}, and the entire computation only needed to use the following relations, for the appropriate choice of lifts $F_i$.
\begin{listalph}
	\item $F_ix=\sigma_i(x)F_i$ for each $i$ and $x\in A$.
	\item $F_i^{n_i}=\alpha_i$ for each $i$.
	\item $F_iF_jF_i^{-1}F_j^{-1}=\beta_{ij}$ for each $i>j$; i.e., $F_iF_j=\beta_{ij}F_jF_i$.
\end{listalph}
As such, the above relations fully describe the extension because they also specify the cocycle, and we know that this cocycle is well-defined. We summarize this discussion into the following theorem.
\begin{theorem}
	Fix everything as in the modified set-up, forgetting about the extension $\mc E$. Given a $\{\sigma_i\}_{i=1}^m$-tuple $(\{\alpha_i\},\{\beta_{ij}\})$, define the group $\mc E(\{\alpha_i\},\{\beta_{ij}\})$ as being generated by $A$ and elements $\{F_i\}_{i=1}^n$ having the following relations.
	\begin{listalph}
		\item $F_ix=\sigma_i(x)F_i$ for each $i$ and $x\in A$.
		\item $F_i^{n_i}=\alpha_i$ for each $i$.
		\item $F_iF_j=\beta_{ij}F_jF_i$ for each $i>j$.
	\end{listalph}
	Then the natural embedding $A\into\mc E(\{\alpha_i\},\{\beta_{ij}\})$ and projection $\pi\colon\mc E(\{\alpha_i\},\{\beta_{ij}\})\onto G$ by $F_i\mapsto\sigma_i$ makes $\mc E(\{\alpha_i\},\{\beta_{ij}\})$ into an extension. In fact, all extensions are isomorphic to some $\mc E(\{\alpha_i\},\{\beta_{ij}\})$.
\end{theorem}
\begin{proof}
	This follows from the preceding discussion, though we will provide a few more words in this proof. The exactness of
	\[1\to A\to\mc E(\{\alpha_i\},\{\beta_{ij}\})\stackrel\pi\to G\to1\]
	follows quickly. Further, the action of conjugation of $\mc E$ on $A$ corresponds correctly to the $ G$-action by (a). So we do indeed have an extension.

	It remains to show that all extensions are isomorphic to one of this type. Well, note that \autoref{prop:multiplytwoelements} and \autoref{prop:writedowncocycle} use only the above relations to write down a cocycle representing the isomorphism class of $\mc E(\{\alpha_i\},\{\beta_{ij}\})$, and it is the cocycle corresponding to the $\{\sigma_i\}_{i=1}^m$-tuple $(\{\alpha_i\},\{\beta_{ij}\})$ itself as described in \autoref{thm:getcocycle}.

	However, we know that as the equivalence class of $(\{\alpha_i\},\{\beta_{ij}\})$ changes, we will hit all cohomology classes in $H^2( G,A)$ by \autoref{thm:classisomorphism}. Thus, because every extension is represented by some cohomology class, every extension will be isomorphic to some $\mc E(\{\alpha_i\},\{\beta_{ij}\})$. This completes the proof.
\end{proof}

\subsection{Change of Group}
We continue in the modified set-up, but we will no longer need access to an extension $\mc E$. In this subsection, we are interested in what happens to tuples when the cocycle operations of $\op{Inf}\colon H^2\left(G/H,A^H\right)\to H^2(G,A)$ and $\op{Res}\colon H^2(G,A)\to H^2(H,A)$ are applied, where $H\subseteq G$ is some subgroup.

In general, this is difficult because the structure of a subgroup $H\subseteq G$ might not be particularly amenable to forming a tuple from a tuple in $G$. More concretely, $H$ might have generators which look very different from those of $G$. However, it will be enough for our purposes to restrict our attention to the subgroups of the form
\[H=\langle\sigma_1^{t_1},\ldots,\sigma_m^{t_m}\rangle,\]
where the $\{t_i\}_{i=1}^m$ are some positive integers. With that said, here are our computations. We begin with inflation.
\begin{lemma} \label{lem:tupleinflation}
	Fix everything as in the modified set-up, forgetting about the extension $\mc E$. Further, let $H\coloneqq\langle\sigma_1^{t_1},\ldots,\sigma_m^{t_m}\rangle$ be a subgroup, and let $\overline\sigma_i$ be the image of $\sigma_i$ in $G/H$. Consider the inflation map $\op{Inf}\colon H^2\left(G/H,A^H\right)\to H^2(G,A)$.
	
	If the cocycle $\overline c\in Z^2\left(G/H,A^H\right)$ gives the $\left\{\overline\sigma_i\right\}_{i=1}^m$-tuple $(\{\overline\alpha_i\},\{\overline\beta_{ij}\})$ (by \autoref{cor:cocycletuplesection}), then the cocycle $\op{Inf}\overline c\in Z^2(G,A)$ gives the $\{\sigma_i\}_{i=1}^m$-tuple
	\[\op{Inf}(\{\overline\alpha_i\},\{\overline\beta_{ij}\})\coloneqq(\{\alpha_i\},\{\beta_{ij}\})=\left(\left\{\overline\alpha_i^{n_i/\gcd(t_i,n_i)}\right\},\{\overline\beta_{ij}\}\right).\]
	Notably, $\gcd(t_i,n_i)$ is the order of $\overline\sigma_i\in G/H$.
\end{lemma}
\begin{proof}
	The point is to use the explicit formulae for the $\alpha_i$ and $\beta_{ij}$ of \autoref{lem:explicitalphabeta}.
	
	More explicitly, the map of \autoref{cor:cocycletuplesection} tells us that we can compute the tuple for $\op{Inf}\overline c$ by using our explicit formulae for $\alpha_i$ and $\beta_{ij}$ on the $2$-cocycle $\op{Inf}\overline c\in Z^2(G,A)$. For some $\alpha_i$, the computation is
	\begin{align*}
		\alpha_i &= \prod_{k=0}^{n_i-1}(\op{Inf}c)\left(\sigma_i^k,\sigma_i\right) \\
		&= \prod_{k=0}^{n_i-1}\overline c\left(\overline\sigma_i^k,\overline\sigma_i\right) \\
		&= \Bigg(\prod_{k=0}^{\gcd(n_i,t_i)-1}\overline c\left(\overline\sigma_i^k,\overline\sigma_i\right)\Bigg)^{n_i/\gcd(n_i,t_i)}
	\end{align*}
	where the last equality is because $\overline\sigma_i^{\gcd(n_i,t_i)}=1$ in $G/H$. In fact, $\gcd(n_i,t_i)$ is the order of $\overline\sigma_i$, so the product is just $\overline\alpha_i$ by \autoref{lem:explicitalphabeta} and how we defined $\overline\alpha_i$. It follows
	\[\alpha_i=\overline\alpha_i^{n_i/\gcd(n_i,t_i)}.\]
	Continuing, for some $\beta_{ij}$, we have
	\begin{align*}
		\beta_{ij} &= \frac{(\op{Inf}\overline c)(\sigma_i,\sigma_j)}{(\op{Inf}\overline c)(\sigma_j,\sigma_i)} \\
		&= \frac{\overline c(\overline\sigma_i,\overline\sigma_j)}{\overline c(\overline\sigma_j,\overline\sigma_i)} \\
		&= \overline\beta_{ij},
	\end{align*}
	where the last equality is by how we defined $\overline\beta_{ij}$. These computations complete the proof.
\end{proof}
\begin{remark} \label{rem:tupleinflationcommutativediagram}
	We can also the statement of \autoref{lem:tupleinflation} as asserting that the diagram
	% https://q.uiver.app/?q=WzAsNCxbMCwwLCJcXG1hdGhjYWwgVFxcbGVmdChHL0gsQV5IXFxyaWdodCkiXSxbMSwwLCJcXG1hdGhjYWwgVChHLEEpIl0sWzAsMSwiWl4yXFxsZWZ0KEcvSCxBXkhcXHJpZ2h0KSJdLFsxLDEsIlpeMihHLEEpIl0sWzAsMSwiXFxvcHtJbmZ9Il0sWzAsMiwiIiwyLHsic3R5bGUiOnsiaGVhZCI6eyJuYW1lIjoiZXBpIn19fV0sWzEsMywiIiwwLHsic3R5bGUiOnsiaGVhZCI6eyJuYW1lIjoiZXBpIn19fV0sWzIsMywiXFxvcHtJbmZ9IiwyXV0=&macro_url=https%3A%2F%2Fraw.githubusercontent.com%2FdFoiler%2Fnotes%2Fmaster%2Fnir.tex
	\[\begin{tikzcd}
		{Z^2\left(G/H,A^H\right)} & {Z^2(G,A)} \\
		{\mathcal T\left(G/H,A^H\right)} & {\mathcal T(G,A)}
		\arrow["{\op{Inf}}", from=1-1, to=1-2]
		\arrow[two heads, from=1-1, to=2-1]
		\arrow[two heads, from=1-2, to=2-2]
		\arrow["{\op{Inf}}", from=2-1, to=2-2]
	\end{tikzcd}\]
	commutes, where the vertical morphisms are from \autoref{cor:cocycletuplesection}.
\end{remark}
\begin{remark} \label{rem:inflationclasses}
	In light of the fact that the cohomology class of some $\op{Inf}\overline c\in Z^2(G,A)$ is only defined up to the cohomology class of $\overline c\in Z^2\left(G/H,A^H\right)$, changing an input tuple $(\{\overline\alpha_i\},\{\overline\beta_{ij}\})\in\mathcal T\left(G/H,A^H\right)$ up to equivalence will not change the cohomology class of the associated cocycle in $\overline c\in Z^2\left(G/H,A^H\right)$ and hence will not change the cohomology class of $\op{Inf}\overline c$ nor the equivalence class of $\op{Inf}(\{\overline\alpha_i\},\{\overline\beta_{ij}\})\in\mathcal T\left(G,A\right)$. All this is to say that we have a well-defined map
	\[\op{Inf}\colon\overline{\mathcal T}\left(G/H,A^H\right)\to\overline{\mathcal T}(G,A)\]
	and commutative diagram
	% https://q.uiver.app/?q=WzAsNCxbMCwwLCJcXG92ZXJsaW5le1xcbWF0aGNhbCBUfVxcbGVmdChHL0gsQV5IXFxyaWdodCkiXSxbMSwwLCJcXG92ZXJsaW5le1xcbWF0aGNhbCBUfShHLEEpIl0sWzAsMSwiSF4yXFxsZWZ0KEcvSCxBXkhcXHJpZ2h0KSJdLFsxLDEsIkheMihHLEEpIl0sWzAsMSwiXFxvcHtJbmZ9Il0sWzAsMiwiIiwyLHsic3R5bGUiOnsidGFpbCI6eyJuYW1lIjoiaG9vayIsInNpZGUiOiJ0b3AifSwiaGVhZCI6eyJuYW1lIjoiZXBpIn19fV0sWzEsMywiIiwwLHsic3R5bGUiOnsidGFpbCI6eyJuYW1lIjoiaG9vayIsInNpZGUiOiJ0b3AifSwiaGVhZCI6eyJuYW1lIjoiZXBpIn19fV0sWzIsMywiXFxvcHtJbmZ9IiwyXV0=&macro_url=https%3A%2F%2Fraw.githubusercontent.com%2FdFoiler%2Fnotes%2Fmaster%2Fnir.tex
	\[\begin{tikzcd}
		{\overline{\mathcal T}\left(G/H,A^H\right)} & {\overline{\mathcal T}(G,A)} \\
		{H^2\left(G/H,A^H\right)} & {H^2(G,A)}
		\arrow["{\op{Inf}}", from=1-1, to=1-2]
		\arrow[hook, two heads, from=1-1, to=2-1]
		\arrow[hook, two heads, from=1-2, to=2-2]
		\arrow["{\op{Inf}}", from=2-1, to=2-2]
	\end{tikzcd}\]
	induced by modding out from \autoref{rem:tupleinflationcommutativediagram}.
\end{remark}
Restriction is similar.
\begin{lemma}
	Fix everything as in the modified set-up, forgetting about the extension $\mc E$. Further, let $H\coloneqq\langle\sigma_1^{t_1},\ldots,\sigma_m^{t_m}\rangle$ be a subgroup. Consider the inflation map $\op{Res}\colon H^2\left(G,A\right)\to H^2(H,A)$.
	
	If the cohomology class $[c]\in H^2\left(G,A\right)$ is represented by the $\left\{\sigma_i\right\}_{i=1}^m$-tuple $(\{\alpha_i\},\{\beta_{ij}\})$, then the cohomology class $[\op{Res}\overline c]$ is represented by the $\{\sigma_i\}_{i=1}^m$-tuple
	\[(\{\overline\alpha_i\},\{\overline\beta_{ij}\})=\left(\left\{\alpha_i^{1_{n_i\mid t_i}}\right\},\left\{\beta_{ij}^{(\gcd(t_i,n_i)1_{n_i\mid t_i},\,\gcd(t_j,n_j)1_{n_i\mid t_i})}\right\}\right).\]
\end{lemma}
\begin{proof}
	By replacing $t_i$ with $\gcd(t_i,n_i)$ (which does not affect $\langle\sigma_i^{t_i}\rangle$ and hence does not affect $H$), we may assume that $t_i=\gcd(t_i,n_i)$. As in the previous proof, we will simply define $c$ by \autoref{thm:getcocycle}, and we will use the formulae of \autoref{lem:explicitalphabeta} to retrieve the $\left\{\sigma_i^{t_i}\right\}$-tuple for $\op{Res}c$. Indeed, we compute
	\begin{align*}
		\overline\alpha_i &= \prod_{k=0}^{n_i/t_i-1}(\op{Res}c)\left(\sigma_i^{t_ik},\sigma_i^{t_i}\right) \\
		&= \prod_{k=0}^{n_i/t_i-1}c\left(\sigma_i^{t_ik},\sigma_i^{t_i}\right) \\
		&= \prod_{k=0}^{n_i/t_i-1}\alpha_i^{\floor{t_i(k+1)/n_i}},
	\end{align*}
	where in the last equality we have used the construction of $c$. Now, if $n_i\mid t_i$, then $n_i=t_i$, and the product is empty, and we get $1$; otherwise, the last term of the product $k=n_i/t_i-1$ is the only term which does not return $1$, and it returns $\alpha_i$. So this matches the claimed $\alpha_i^{1_{n_i\mid t_i}}$.

	Continuing, we compute
	\begin{align*}
		\overline\beta_{ij} &= \frac{(\op{Res}c)\left(\sigma_i^{t_i},\sigma_j^{t_j}\right)}{(\op{Res}c)\left(\sigma_j^{t_j},\sigma_i^{t_i}\right)} \\
		&= \frac{c\left(\sigma_i^{t_i},\sigma_j^{t_j}\right)}{c\left(\sigma_j^{t_j},\sigma_i^{t_i}\right)} \\
		&= c\left(\sigma_i^{t_i},\sigma_j^{t_j}\right),
	\end{align*}
	where in the last step we have used the construction of $c$. Now, if $n_i\mid t_i$ or $n_i\mid t_j$, then we are computing $c\left(1,\sigma_j^{t_j}\right)$ or $c\left(\sigma_i^{t_i},1\right)$, which are both $1$, as needed. Otherwise, $t_i<n_i$ and $t_j<n_j$, so
	\[\overline\beta_{ij}=\beta_{ij}^{(t_it_j)},\]
	which again is as claimed.
\end{proof}
Thankfully, we will really only care about inflation in the following discussion, but we will say that there are analogues of \autoref{rem:tupleinflationcommutativediagram} and \autoref{rem:inflationclasses}.

\subsection{Profinite Groups}
In this subsection, we will use our results on change of group to extend our results a little to allow profinite groups. As such, we will want to slightly modify our set-up; we will call the following set-up the ``profinite set-up.''

Let $\mathcal I$ be a poset category such that any pair of elements has an upper bound (i.e., a directed set), and let the functor $G_\bullet\colon\mathcal I\opp\to\op{FinAbGrp}$ be an inverse system of finite abelian groups. These will create a profinite group
\[G\coloneqq\limit_{i\in\mathcal I}G_i.\]
In order to be able to apply our theory, we will assume that $G$ is a finite direct sum of procyclic groups as
\[G\simeq\bigoplus_{k=1}^m\overline{\langle\sigma_k\rangle}\]
for some elements $\{\sigma_k\}_{k=1}^m\subseteq G$. Further, we will require that the kernel $N_i$ of the map $G\onto G_i$ to take the form
\[N_i\coloneqq\overline{\left\langle\sigma_1^{t_{i,1}},\ldots,\sigma_m^{t_{i,m}}\right\rangle}.\]
In short, our restriction on the $N_i$ will allow our inflation maps to be computable in the sense of \autoref{lem:tupleinflation}. We quickly remark that, because the topology on $G$ is the coarsest one making the projections $G\onto G_i$ continuous, the subsets $\{N_i\}_{i\in\mathcal I}$ give a fundamental system of open neighborhoods around the identity.
\begin{example}
	To show that we are still allowing interesting groups, we can set 
	\[G_{m,\nu}\coloneqq\op{Gal}\left(\QQ_p(\zeta_{p^m-1})\QQ_p(\zeta_{p^\nu})/\QQ_p\right)\simeq\op{Gal}\left(\QQ_p(\zeta_{p^m-1})/\QQ_p\right)\oplus\op{Gal}\left(\QQ_p(\zeta_{p^\nu})/\QQ_p\right),\]
	which becomes $G=\op{Gal}\left(\QQ_p^{\op{ab}}/\QQ_p\right)\simeq\widehat\ZZ\oplus\ZZ_p^\times$ upon taking the inverse limit. It is not very hard to check that the kernels are generated correctly; for example, when $p$ is odd, we have $\ZZ_p^\times\cong\ZZ/(p-1)\ZZ\oplus\ZZ_p$, and under our isomorphisms, we will have
	\[\op{Gal}\left(\QQ(\zeta_{p^\nu})/\QQ_p\right)\simeq\ZZ/(p-1)\ZZ\oplus\ZZ_p/p^{\nu-1}\ZZ_p,\]
	so the kernel of $G\onto G_{m,\nu}$ is $m\widehat\ZZ\oplus(\ZZ/(p-1)\ZZ)^{1_{\nu=0}}\oplus p^{\nu-1}\ZZ_p$.
	% let $K$ be a local field with $\op{char}K=0$, and set $G_{\pi,n,m}\coloneqq\op{Gal}(K_{\pi,n}K_m/K)$, where $\{K_{\pi,n}\}$ is an ascending chain of Lubin--Tate extension and $K_m$ is the unramified extension of degree $m$. Then
	% \begin{align*}
	% 	\colimit_{n,m}\op{Gal}(K_{\pi,n}K_m/K) &\cong \op{Gal}(K^{\mathrm{ab}}/K) \\
	% 	&\cong \op{Gal}(K^{\op{unr}}/K)\oplus\op{Gal}(K_\pi) \\
	% 	&\cong \overline{\langle\op{Frob}_K\rangle}\oplus\mathcal O_K^\times \\
	% 	&\cong \widehat\ZZ\oplus\FF_\mf p^\times\oplus(1+\mf p) \\
	% 	&\cong \widehat\ZZ\oplus\FF_\mf p^\times\oplus\ZZ/p^a\ZZ\oplus\ZZ_p^{[K:\QQ_p]}
	% \end{align*}
	% for some sufficiently large $a\in\NN$; here the last isomorphism is by the logarithm map, which exists because $\op{char}K=0$. (For details, see \cite{neukirch-alg-nt}, Proposition II.5.7.)
\end{example}
\begin{remark}
	I'm not sure if such an explicit construction can be extended to other local fields $K$ (say, via Lubin--Tate theory). Because $K^\times$ is not topologically finitely generated when $K$ is in positive characteristic (see for example \cite{neukirch-alg-nt}, Proposition~II.5.7) such a construction must do something subtle.
\end{remark}
Let $A$ be a discrete $G$-module. The main goal of this subsection is to be able to provide a notion of a ``compatible system'' of tuples from each individual $H^2(G_i,A)$ to be able to exactly describe an element of $H^2(G,A)$. To effect this, we have the following somewhat annoying checks.
% \begin{lemma}
% 	Fix everything as in the profinite set-up, and let $N\subseteq G$ be an open normal subgroup. If $\sigma\in G$ is such that $[\sigma]_N\in G/N$ has finite order $n_\sigma$, then there exists some $i\in\mathcal I$ such that the order of $[\sigma]_{N_i}\in G/N_i=G_i$ has order divisible by $n_\sigma$.
% \end{lemma}
% \begin{proof}
% 	We proceed in steps. Let $1$ denote the identity of $G$.
% 	\begin{enumerate}
% 		\item Suppose that $p\coloneqq n_\sigma$ is prime. We proceed by contraposition. Namely, suppose there is no $N_i$ such that $\sigma^p\in N_i$, and we will show that $[\sigma]_N\in G/N$ cannot have order $p$. We may assume that $\sigma^p\in N$, which means that we are actually interested in showing $\sigma\in N$.

% 		Well, we claim that $\sigma$ is a limit point of $\langle\sigma^p\rangle$. To see this, we need to show that any open neighborhood $U$ around $\sigma$ has nontrivial intersection with $\langle\sigma^p\rangle$. Indeed, $\sigma^{-1}U$ is an open set containing the identity, but because the $\{N_i\}_{i\in\mathcal I}$ form a fundamental system of open neighborhoods around the identity, we have $N_i\subseteq\sigma^{-1}U$ for some $i\in\mathcal I$. Thus, it suffices to show that 
% 		\[\sigma N_i\cap\langle\sigma^p\rangle\ne\emp.\]
% 		Now, the order of $\sigma N_i$ is not divisible by $p$, so $\langle\sigma^pN_i\rangle=\langle\sigma N_i\rangle$ and in particular $\langle\sigma^pN_i\rangle$ contains $\sigma N_i$. Concretely, let's say $\sigma^{pk}N_i=\sigma N_i$; then $\sigma^{pk}\in\sigma N_i\cap\langle\sigma^p\rangle$, finishing.

% 		In total, because $N$ is an open subgroup and hence closed, we see that $\sigma^p\in N$ must also contain $\langle\sigma^p\rangle$ and hence must contain the limit point $\sigma$. This finishes.

% 		\item Next suppose that $n_\sigma=p^\nu$ is a power of a prime. We proceed by induction on $\nu$. When $\nu=0$, there is nothing to say because the order of a group element is always divisible by $1$; when $\nu=1$, this is the previous step. Otherwise, when $\nu>1$, we note that $\left[\sigma^p\right]_N=[\sigma]_N^p$ has order $p^{\nu-1}$: certainly $[\sigma]_N^{p\cdot p^{\nu-1}}$ vanishes, so the order divides $p^{\nu-1}$, but no smaller of $p$ will do because this would make the order of $[\sigma]_N$ too small.

% 		Thus, by the inductive hypothesis, there exists some $i\in\mathcal I$ such that the order of $[\sigma^p]_{N_i}$ has order divisible by $p^{\nu-1}$. We claim $[\sigma]_{N_i}$ has order divisible by $p^\nu$. Indeed, if not, then there exists some $k$ with $p\nmid k$ such that
% 		\[[\sigma]_{N_i}^{p^{\nu-1}k}=[1]_{N_i},\]
% 		from which we conclude that the order of $[\sigma^p]$ divides $p^{\nu-2}k$, which is not divisible by $p^{\nu-1}$.

% 		\item To finish, we show a version of multiplicativity: if the order of $[\sigma]_{N_p}$ is divisible by $n_p$, and the order of $[\sigma]_{N_q}$ is divisible by $n_q$ with $\gcd(n_p,n_q)=1$, then there exists $r\in\mathcal I$ such that the order of $[\sigma]_{N_r}$ is divisible by $n_pn_q$.

% 		Indeed, because $\mathcal I$ is a directed set, there exists some $r\in\mathcal I$ with morphisms $i\to r$ and $j\to r$. These correspond to having morphisms $G_r\to G_i$ and $G_r\to G_j$, and the fact that these morphisms are well-defined requires $N_r\subseteq N_i,N_j$.

% 		Now, let's say that the order of $[\sigma]_{N_r}$ is $n_r$; we want to show $n_pn_q\mid n_r$. Because $\gcd(n_p,n_q)=1$, it suffices (by symmetry) to show $n_p\mid n_r$. Well, $\sigma^{n_r}\in N_r\subseteq N_p$, so
% 		\[[\sigma]_{N_p}^{n_r}=[1]_{N_p},\]
% 		so the order of $[\sigma]_{N_p}$ divides $n_r$. In particular, $n_p\mid n_r$.
% 	\end{enumerate}
% 	We now note that, for the general case of $n\in\NN$, we can prime-factor $n$ into coprime factors, use step 2 to create a list of $N_i$, one for each prime factor, and then use step 3 to glue them all together. This completes the proof.
% \end{proof}
% \begin{lemma}
% 	Fix everything as in the profinite set-up. Then, for any open normal subgroup $N\subseteq G$, there exists $i\in\mathcal I$ so that $N$ contains $N_i$.
% \end{lemma}
% \begin{proof}
% 	This follows directly from the fact that the collection $\{N_i\}_{i\in\mathcal I}$ is a fundamental system of open neighborhoods around the identity of $G$. In particular, $N$ contains the identity and is open.
% 	% Because $G$ is compact (it's profinite), we have $[G:N]<\infty$. In particular, for any $\sigma\in G$, we must have $\sigma^{[G:N]}\in N$.
% 	% Thus, it will roughly speaking be enough to show that, for any $\sigma_k$ and $t\in\NN$, we have $\sigma_k^t\in N_i$ for some $i\in\mathcal I$. We have two cases.
% 	% \begin{itemize}
% 	% 	\item Suppose that $\sigma_k\in G$ has infinite order. 
% 	% \end{itemize}
% \end{proof}
\begin{lemma} \label{lem:colimitfiltered}
	Suppose that $\mathcal P$ is a directed set, and let $\mathcal P'\subseteq\mathcal P$ be a subcategory such that any $x\in\mathcal P$ has some $x'\in\mathcal P'$ such that $x\le x'$. Then, given a functor $F\colon\mathcal P\to\mathcal C$, we have
	\[\colimit_\mathcal PF\simeq\colimit_{\mathcal P'}F,\]
	provided that both colimits exist.
\end{lemma}
\begin{proof}
	For concreteness, if $x\le y$ in $\mathcal P$, we will let $f_{yx}\colon x\to y$ be the corresponding morphism; in particular, $x\le y\le z$ has $f_{zx}=f_{zy}f_{yx}$. Now, for brevity, set
	\[X\coloneqq\colimit_\mathcal PF\qquad\text{and}\qquad X'\coloneqq\colimit_{\mathcal P'}F.\]
	By the Yoneda lemma, it suffices to fix some object $Y\in\mathcal C$ and show that $\op{Mor}_\mathcal C(X,Y)\simeq\op{Mor}_\mathcal C(X',Y)$. Well, morphsims $X\to Y$ are in (natural) bijection with cones under $F$ with nadir $Y$, and morphisms $X'\to Y$ are in (natural) bijection with cones under $F'\coloneqq F|_{\mathcal P'}$ with nadir $Y$.

	Thus, it suffices to give a natural bijection between cones under $F$ with nadir $Y$ and cones under $F'$ with nadir $Y$. Well, given a cone under $F$ with nadir $Y$, we can simply restrict it to $\mathcal P'$ to get a cone under $F'$. In the other direction, given a cone under $F'$ with nadir $Y$, we can build a cone under $F$ with nadir $Y$ as follows; let $\varphi_{x'}\colon F(x')\to Y$ for $x'\in\mathcal P'$ be the corresponding morphisms in our cone.
	
	For any $x\in\mathcal P$, find $x'\in\mathcal P'$ such that $x\le x'$. Then set
	\[\varphi_x\coloneqq\varphi_{x'}\circ f_{x'x}\]
	Note that $\varphi_x$ is in fact independent of our choice of $x'$: if $x\le x_1'$ and $x\le x_2'$, then because $\mathcal P$ is a directed set, we can find $y\in\mathcal P$ such that $x_1',x_2'\le y$ and then $y'\in\mathcal P'$ with $y\le y'$. Then
	\begin{align*}
		\varphi_{x_\bullet'}\circ f_{x_\bullet'x} &= \varphi_{y'}\circ f_{y'x_\bullet'}\circ f_{x_\bullet'x} \\
		&= \varphi_{y'}\circ f_{y'x}
	\end{align*}
	for $x_\bullet'\in\{x_1',x_2'\}$. Anyway, we can check that the morphisms $\varphi$ do assemble to a cone under $F'$: if $x\le y$ in $\mathcal P$, then find $y'\in\mathcal P$ with $x\le y\le y'$, and we compute
	\begin{align*}
		\varphi_y\circ f_{yx} &= \varphi_{y'}\circ f_{y'y}\circ f_{yx} \\
		&= \varphi_{y'}\circ f_{y'x} \\
		&= \varphi_x.
	\end{align*}
	Thus, we do have a natural, well-defined map sending cones under $F'$ with nadir $Y$ to cones under $F$ with nadir $Y$. It is not too hard to see that these maps are inverse to each other (for example, the cone under $F'$, extended to $F$, does indeed restrict back to $F'$ properly), which completes the proof.
\end{proof}
\begin{remark}
	One can remove the hypothesis that the colimits exist and use essentially the same proof.
\end{remark}
\begin{proposition} \label{prop:bettercohomlimit}
	Fix everything as in the profinite set-up. Then, given a discrete $G$-module $A$,
	\[H^2(G,A)\simeq\colimit_{i\in\mathcal I}H^2\left(G_i,A^{N_i}\right).\]
	Here, the morphisms between the collection of $H^2\left(G_i,A^{N_i}\right)$ are induced by inflation: if $i\to j$ in $\mathcal I$, then $G_j\to G_i$ in $\mathrm{FinAbGrp}$, giving an inflation map $\op{Inf}\colon H^2\left(G_i,A^{N_i}\right)\to H^2\left(G_j,A^{N_j}\right)$.
\end{proposition}
\begin{proof}
	Let $\mathcal N$ be the poset category of open normal subgroups of $G$, reverse ordered under inclusion; i.e., $N_1\subseteq N_2$ in $G$ induces a map $N_2\to N_1$. Then it is already known that
	\[H^2(G,A)\simeq\colimit_{N\in\mathcal N}H^2\left(G/N,A^N\right).\]
	On the other hand, observe that $i\le j$ in $\mathcal I$ induces $G_j\to G_i$, so $N_j\subseteq N_i$. In other words, $i\mapsto N_i$ will define a functor $\mathcal I\to\mathcal N$; functoriality follows because $\mathcal I$ and $\mathcal N$ are poset categories. Letting $\mathcal N'$ denote the image of $\mathcal I$ in $\mathcal N$, we see
	\[\colimit_{i\in\mathcal I}H^2\left(G_i,A^{N_i}\right)\simeq\colimit_{N\in\mathcal N'}H^2\left(G/N,A^N\right).\]
	Notably, the inflation maps $\op{Inf}\colon H^2\left(G_i,A^{N_i}\right)\to H^2\left(G_j,A^{N_j}\right)$ when $i\le j$ become the inflation maps $\op{Inf}\colon H^2\left(G/N,A^N\right)\to H^2\left(G/N',A^{N'}\right)$ when $N'\subseteq N$. So if we let $F\colon\mathcal N\to\op{AbGrp}$ be the functor taking $N$ to $H^2\left(G/N,A^N\right)$ (and $N\subseteq N'$ to the inflation map), we are trying to show
	\[\colimit_{\mathcal N}F=\colimit_{\mathcal N'}F.\]
	For this, we use \autoref{lem:colimitfiltered}. Indeed, for a given open normal subgroup $N\in\mathcal N$, we need to find some $N'\in\mathcal N'$ such that $N\le N'$, which means $N'\subseteq N$.
	
	However, the elements of $\mathcal N'$ are the collection $\{N_i\}_{i\in\mathcal I}$, which form a fundamental system of open neighborhoods around the identity. Thus, the fact that $N$ is an open set containing the identity implies there is some $N_i\in\mathcal N'$ such that $N_i\subseteq N$. This finishes the proof.
\end{proof}
Observe that the above proofs did not use the extra hypotheses on $G$ nor $N_i$ to be products of procyclic groups. We use these hypotheses now.
% \begin{definition}
% 	Fix everything as in the profinite set-up, and let $A$ be a discrete $G$-module. Then a \textit{compatible system of $\{\sigma_p\}_{p=1}^m$-tuples} is an indexed set
% 	\[\left(\{\alpha_{i,p}\},\{\beta_{i,pq}\}\right)_{i\in\mathcal I}\]
% 	such that $\left(\{\alpha_{i,p}\},\{\beta_{i,pq}\}\right)$ is a $\{\sigma_pN_i\}_{p=1}^m$-tuple (corresponding to a class in $H^2\left(G_i,A^{N_i}\right)$) and
% 	\[\op{Inf}\left(\{\alpha_{i,p}\},\{\beta_{i,pq}\}\right)\sim\left(\{\alpha_{j,p}\},\{\beta_{j,pq}\}\right)\]
% 	as tuples corresponding to $H^2\left(G_j,A^{N_j}\right)$, whenever $i\le j$ in $\mathcal I$. We also define the relation $\sim$ of equivalence between compatible systems if and only if they are pointwise equivalent.
% \end{definition}
% The precise definition above is one of technical convenience, as we will shortly see.
To work more concretely, we note that any $i\in\mathcal I$ has
\[G_i\simeq\frac G{N_i}\simeq\bigoplus_{p=1}^m\overline{\langle\sigma_p\rangle}/\overline{\langle\sigma_p^{t_{i,p}}\rangle}\simeq\bigoplus_{p=1}^m\langle\sigma_p\rangle/\langle\sigma_p^{t_{i,p}}\rangle\subseteq\bigoplus_{p=1}^m\ZZ/t_{i,p}\ZZ\]
is a finite abelian group generated by the elements $\sigma_pN_i$. As a warning, the order of $\sigma_pN_i$ might not be $t_{i,p}$, for example if $\sigma_p$ itself has some small finite order which $t_{i,p}$ is not properly capitalizing on. More concretely, $\ZZ_5/3\ZZ_5=0$.

Regardless, the main point is that, given a discrete $G$-module $A$, we can consider the $\{\sigma_pN_i\}_{p=1}^m$-tuples $\mathcal T\left(G_i,A^{N_i}\right)$. Now, as discussed above, $i\le j$ in $\mathcal I$ induces a quotient map $G_j\simeq G/N_j\onto G_i/N_i$. From this, we have the following coherence check.
\begin{lemma} \label{lem:tupleinflationcommutes}
	Fix everything as in the profinite set-up, and let $A$ be a discrete $G$-module. Then, given $i\le j\le k$ in $\mathcal I$, the diagram
	% https://q.uiver.app/?q=WzAsMyxbMCwwLCJcXG1hdGhjYWwgVFxcbGVmdChHX2ksQV9pXntOX2l9XFxyaWdodCkiXSxbMSwwLCJcXG1hdGhjYWwgVFxcbGVmdChHX2osQV9qXntOX2p9XFxyaWdodCkiXSxbMSwxLCJcXG1hdGhjYWwgVFxcbGVmdChHX2ssQV9rXntOX2t9XFxyaWdodCkiXSxbMCwxLCJcXG9we0luZn0iXSxbMSwyLCJcXG9we0luZn0iXSxbMCwyLCJcXG9we0luZn0iLDJdXQ==&macro_url=https%3A%2F%2Fraw.githubusercontent.com%2FdFoiler%2Fnotes%2Fmaster%2Fnir.tex
	\[\begin{tikzcd}
		{\mathcal T\left(G_i,A^{N_i}\right)} & {\mathcal T\left(G_j,A^{N_j}\right)} \\
		& {\mathcal T\left(G_k,A^{N_k}\right)}
		\arrow["{\op{Inf}}", from=1-1, to=1-2]
		\arrow["{\op{Inf}}", from=1-2, to=2-2]
		\arrow["{\op{Inf}}"', from=1-1, to=2-2]
	\end{tikzcd}\]
	commutes. Here, the $\op{Inf}$ maps are defined as in \autoref{lem:tupleinflation}.
\end{lemma}
\begin{proof}
	For each $i\in\mathcal I$, we let $n_{i,p}$ denote the order of $\sigma_pN_i\in G_i$. Using the definition of $\op{Inf}$ from \autoref{lem:tupleinflation}, we just pick up some $\{\sigma_pN_p\}_{p=1}^m$-tuple $(\{\alpha_p\},\{\beta_{pq}\})$-tuple in $\mathcal T\left(G_i,A^{N_i}\right)$ and track through the diagram as follows.
	% https://q.uiver.app/?q=WzAsNCxbMCwwLCIoXFx7XFxhbHBoYV9wXFx9LFxce1xcYmV0YV97cHF9XFx9KSJdLFsxLDAsIlxcbGVmdChcXGJpZ1xce1xcYWxwaGFfcF57bl97aixwfS9uX3tpLHB9fVxcYmlnXFx9LFxce1xcYmV0YV97cHF9XFx9XFxyaWdodCkiXSxbMSwxLCJcXGxlZnQoXFxiaWdcXHtcXGFscGhhX3BeeyhuX3tqLHB9L25fe2kscH0pKG5fe2sscH0vbl97aixwfSl9XFxiaWdcXH0sXFx7XFxiZXRhX3twcX1cXH1cXHJpZ2h0KSJdLFswLDEsIlxcbGVmdChcXGJpZ1xce1xcYWxwaGFfcF57bl97ayxwfS9uX3tpLHB9fVxcYmlnXFx9LFxce1xcYmV0YV97cHF9XFx9XFxyaWdodCkiXSxbMCwxLCJcXG9we0luZn0iXSxbMSwyLCJcXG9we0luZn0iXSxbMCwzLCJcXG9we0luZn0iLDJdLFszLDIsIiIsMix7ImxldmVsIjoyLCJzdHlsZSI6eyJoZWFkIjp7Im5hbWUiOiJub25lIn19fV1d&macro_url=https%3A%2F%2Fraw.githubusercontent.com%2FdFoiler%2Fnotes%2Fmaster%2Fnir.tex
	\[\begin{tikzcd}
		{(\{\alpha_p\},\{\beta_{pq}\})} & {\left(\big\{\alpha_p^{n_{j,p}/n_{i,p}}\big\},\{\beta_{pq}\}\right)} \\
		{\left(\big\{\alpha_p^{n_{k,p}/n_{i,p}}\big\},\{\beta_{pq}\}\right)} & {\left(\big\{\alpha_p^{(n_{j,p}/n_{i,p})(n_{k,p}/n_{j,p})}\big\},\{\beta_{pq}\}\right)}
		\arrow["{\op{Inf}}", from=1-1, to=1-2]
		\arrow["{\op{Inf}}", from=1-2, to=2-2]
		\arrow["{\op{Inf}}"', from=1-1, to=2-1]
		\arrow[Rightarrow, no head, from=2-1, to=2-2]
	\end{tikzcd}\]
	This completes the proof.
\end{proof}
% Now, if we let $n_{i,p}$ denote the actual order of $\sigma_{i,p}N_i\in G_i$, then we may compute the inflation map $\op{Inf}\colon H^2\left(G_i,A^{N_i}\right)\to H^2\left(G_j,A^{N_j}\right)$ by
% \[\op{Inf}\left(\{\alpha_{i,p}\},\{\beta_{i,pq}\}\right)=\left(\{\alpha_{i,p}^{n_{j,p}}\},\{\beta_{i,pq}\}\right),\]
% so we are asking for
% \[\left(\{\alpha_{i,p}^{n_{j,p}}\},\{\beta_{i,pq}\}\right)\sim\left(\{\alpha_{j,p}\},\{\beta_{j,pq}\}\right)\]
% in the coherence condition for a compatible tuple.
% \begin{remark}
% 	From the above description, we can see why we ``have'' to allow the equivalence relation into our notion of compatibility. For example, if one of the $G_i$ is the trivial group, and $A^G$ is trivial, then we would be requiring all the $\beta_{i,pq}$ elements to be trivial for all $i\in\mathcal I$. This is not good.
% \end{remark}
And here is the result.
\begin{theorem}
	Fix everything as in the profinite set-up, and let $A$ be a discrete $G$-module. Then the isomorphisms of \autoref{thm:classisomorphism} upgrade into an isomorphism
	\[H^2(G,A)\simeq\colimit_{i\in\mathcal I}\overline{\mathcal T}\left(G_i,A^{N_i}\right).\]
	Here the morphisms between the $\overline{\mathcal T}\left(G_i,A^{N_i}\right)$ are inflation maps of \autoref{lem:tupleinflation}.
\end{theorem}
\begin{proof}
	Note that the objects $\overline{\mathcal T}\left(G_i,A^{N_i}\right)$ do make a directed system over $\mathcal I$ because of the commutativity of \autoref{lem:tupleinflationcommutes}. Namely, the lemma checks that $\mathcal I\to\op{AbGrp}$ by $i\mapsto\overline{\mathcal T}\left(G_i,A^{N_i}\right)$ is actually functorial; technically we must also check that the maps $\overline{\mathcal T}\left(G_i,A^{N_i}\right)\to\overline{\mathcal T}\left(G_i,A^{N_i}\right)$ are the identity, but this follows from the definition.

	Now, by \autoref{prop:bettercohomlimit}, we have
	\[H^2(G,A)\simeq\colimit_{i\in\mathcal I}H^2\left(G_i,A^{N_i}\right),\]
	but now the natural isomorphism induced by \autoref{rem:inflationclasses} induces an isomorphism of direct limits
	\[\colimit_{i\in\mathcal I}H^2\left(G_i,A^{N_i}\right)\simeq\colimit_{i\in\mathcal I}\overline{\mathcal T}\left(G_i,A^{N_i}\right)\]
	given by the isomorphism of \autoref{thm:classisomorphism} acting pointwise. This completes the proof.
\end{proof}
Because there are reasonably explicit descriptions of direct limits of abelian groups, and we already have an explicit description of each $\overline{\mathcal T}\left(G_i,A^{N_i}\right)$ term in addition to a description of the inflation maps between them, we will be content with our sufficiently explicit description of $H^2(G,A)$. So we call it done here.

\section{Studying Tuples} \label{sec:tuplestudy}
% !TEX root = ../abeliangerbs.tex

% The story so far has been able to generalize the one-variable results from \autoref{sec:general} to results using all generators of an abelian group in \autoref{sec:abelian}. It remains to prove \autoref{thm:getcocycle}, which is the main goal of this section.

The story from \autoref{sec:general} was able to encode a cohomology class in $H^2(G,A)$ into a (somewhat complex) tuple of elements in $A$. This mirrors the introductory comments from \autoref{sec:crackpot}, so we will spend this section connecting the two stories.

\subsection{Set-Up and Overview} \label{sec:overview}
The approach here will be to attempt to abstract our data away from the $ G$-module $A$ as much as possible. To set up our discussion, we continue with
\[G\simeq\bigoplus_{i=1}^mG_i,\]
where $G_i=\langle\sigma_i\rangle\subseteq G$ and $\sigma_i$ has order $n_k$. These variables allow us to define
\[T_i\coloneqq(\sigma_i-1)\qquad\text{and}\qquad N_i\coloneqq\sum_{p=0}^{n_i-1}\sigma_i^p\]
for each index $i$. In fact, it will be helpful to also have notation
\[\sigma^{(a)}\coloneqq\sum_{p=0}^{a-1}\sigma^p\]
for any $\sigma\in G$ and nonnegative integer $a\ge0$; in particular, $\sigma^{(0)}=0$ and $\sigma_i^{(n_i)}=N_i$. The main benefits to this notation will be the facts that
\[\sigma^{(a+b)}=\sigma^{(a)}+\sigma^a\sigma^{(b)}\qquad\text{and}\qquad\sigma_i^a=T_i\sigma_i^{(a)}+1,\]
which can be seen by direct expansion. Given $g\in\prod_{p=1}^n\sigma_p^{a_p}$, we will also define the notation
\[g_i\coloneqq\prod_{p=1}^{i-1}\sigma_p^{a_p}\]
for $i\ge0$. In particular $g_0=g_1=1$ and $g_{n+1}=g$.

Now, our tool in the proof of \autoref{thm:getcocycle} will be the magical map $\mathcal F\colon\ZZ[G]^m\times\ZZ[G]^{\binom m2}\to\ZZ[G]^m$ defined by
\[\mathcal F\colon\big((x_i)_{i=1}^m,(y_{ij})_{i>j}\big)\mapsto\Bigg(x_iN_i-\sum_{j=1}^{i-1}y_{ij}T_j+\sum_{j=i+1}^my_{ji}T_j\Bigg)_{i=1}^m.\]
This is of course a $G$-module homomorphism. We will go ahead and state the main results we will prove. Roughly speaking, $\mathcal F$ is manufactured to make the following result true.
\begin{prop} \label{prop:manufacturedcocycle}
	Fix everything as in the set-up. Then the function
	\[\overline c(g)\coloneqq\left(g_i\sigma_i^{(a_i)}\right)_{i=1}^m,\]
	where $g\coloneqq\prod_{i=1}^m\sigma_i^{a_i}$, is a $1$-cocycle in $Z^1(G,\coker\mathcal F)$.
\end{prop}
The reason we care about this cocycle is that we can pass it through a boundary morphism induced by the short exact sequence
\[0\to\underbrace{\frac{\ZZ[G]^m\times\ZZ[G]^{\binom m2}}{\ker\mathcal F}}_{X\coloneqq}\stackrel{\mathcal F}\to\ZZ[G]^m\to\coker\mathcal F\to0,\]
so we have a $2$-cocycle $\delta(\overline c)\in Z^2(G,X)$; in fact, we will be able to explicitly compute $\delta(\overline c)$ as a result of the proof of \autoref{prop:manufacturedcocycle}.

Only now will we bring in tuples. The first result provides an alternate description of tuples.
\begin{restatable}{prop}{propalternativetuple} \label{prop:alternativetuple}
	Fix everything as in the set-up, and now let $A$ be a $G$-module. Then $\{\sigma_i\}_{i=1}^m$-tuples are canonically isomorphic to $\op{Hom}_{\ZZ[G]}(X,A)=H^0(G,\op{Hom}_\ZZ(X,A))$.
\end{restatable}
\noindent The second result brings in the last ingredient, the cup product.
\begin{restatable}{theorem}{thmyesitisacocycle} \label{thm:yesitisacocycle}
	Fix everything as in the set-up. Also, fix a $G$-module $A$ and a $\{\sigma_i\}_{i=1}^m$-tuple $\left(\{\alpha_i\},\{\beta_{ij}\}\right)$. Then observe there is a natural cup product map
	\[\cup\colon H^2(G,X)\times H^0(G,\op{Hom}_\ZZ(X,A))\to H^2(G,A)\]
	by using the evaluation map $X\otimes_\ZZ\op{Hom}_\ZZ(X,A)\to A$. Then, using the isomorphism of \autoref{prop:alternativetuple}, the cocycle defined in \autoref{thm:getcocycle} is simply the output of $\delta(\overline c)\cup\left(\{\alpha_i\},\{\beta_{ij}\}\right)$ on cocycles.
\end{restatable}
\noindent Because we know that the cup product sends cocycles to cocycles, this will show that the cocycle of \autoref{thm:getcocycle} is in fact well-defined. More importantly, $X$ will be a $2$-encoding module.

% it might be worth stating the main results we are going to prove here, but they are somewhat notation-heavy

\subsection{Preliminary Work}
We continue in the set-up of the previous subsection.
% The goal of this subsection is to prove \autoref{prop:manufacturedcocycle}. In fact, we will show the following stronger result.
% \begin{proposition} \label{prop:allmanufacturedcocycles}
% 	Fix everything as in the set-up. Then $H^1(G,\coker\mathcal F)$ is cyclic generated by the class $[\overline c]$ represented by $\overline c$, where
% 	\[\overline c(g)\coloneqq\left(g_i\sigma_i^{(a_i)}\right)_{i=1}^m,\]
% 	with $g\coloneqq\prod_{i=1}^m\sigma_i^{a_i}$
% \end{proposition}
Before jumping into any hard logic, we define some (more) notation which will be useful later on as well. First, in $\ZZ[G]^m\times\ZZ[G]^{\binom m2}$, we define
\[\kappa_p\coloneqq\big((1_{i=p})_i,(0)_{i>j}\big)\in X\qquad\text{and}\qquad\lambda_{pq}\coloneqq\big((0)_i,(1_{(i,j)=(p,q)})_{i>j}\big)\]
for all relevant indices $p$ and $q$ so that the $\kappa_p$ and $\lambda_{pq}$ are a basis for $\ZZ[G]^m\times\ZZ[G]^{\binom m2}$ as a $\ZZ[G]$-module. Secondly, we define
\[\varepsilon_p\coloneqq(1_{i=p})_{i=1}^m\]
for all indices $p$, again giving a basis for $\ZZ[G]^m$ as a $\ZZ[G]$-module. For example, this notation lets us write
\begin{equation}
	\mathcal F\left(\sum_{i=1}^mx_i\kappa_i+\sum_{i>j}y_{ij}\lambda_{ij}\right)=\sum_{i=1}^mx_iN_i\varepsilon_i+\sum_{i>j}y_{ij}(T_i\varepsilon_j-T_j\varepsilon_i), \label{eq:betterf}
\end{equation}
and
\[\overline c(g)=\sum_{i=1}^mg_i\sigma_i^{(a_i)}\varepsilon_i\]
where $g\coloneqq\prod_{i=1}^m\sigma_i^{a_i}$.

Additionally, so that we do not need to interrupt our discussion later, we establish a few lemmas which will aide our proof of \autoref{prop:manufacturedcocycle}.
\begin{lemma} \label{lem:separatenijs}
	Fix everything as in the set-up. Then, for any set of distinct indices $(i_1,\ldots,i_k)$, we have
	\[\bigcap_{p=1}^k\im N_{i_p}=\im\prod_{p=1}^kN_{i_p},\]
	where we are identifying $x\in\ZZ[G]$ with its associated multiplication map $x\colon\ZZ[G]\to\ZZ[G]$.
\end{lemma}
\begin{proof}
	The point is that the elements of $\bigcap_{p=1}^k\im N_{i_p}$ and $\im\prod_{p=1}^kN_{i_p}$ are both simply the elements whose expansion in the form $\sum_gc_gg\in\ZZ[G]$ have $c_j$ ``constant in $\sigma_p$ and $\sigma_q$.'' More explicitly, of course, $\prod_{p=1}^kN_{i_p}\in\bigcap_{p=1}^k\im N_{i_p}$, so
	\[\im\prod_{p=1}^kN_{i_p}\subseteq\bigcap_{p=1}^k\im N_{i_p}.\]
	In the other direction, suppose that we have some element
	\[z\coloneqq\sum_{(a_i)_i}c_{(a_i)_i}\sigma_1^{a_1}\cdots\sigma_m^{a_m}\in\bigcap_{p=1}^k\im N_{i_p},\]
	the sum is over sequences $(a_i)_{i=1}^m$ such that $0\le a_i<n_i$ for each index $i$. We will show $z\in\im\prod_{p=1}^kN_{i_p}$.
	
	Now, $z\in\im N_r$ for $r\in\{p,q\}$ is equivalent to $z\in\ker T_r$, but upon multiplying by $(\sigma_r-1)$ we see that we are asking for
	\[\sum_{(a_i)_i}c_{(a_i)_i}\sigma_1^{a_1}\cdots\sigma_{r-1}^{a_{r-1}}\sigma_r^{a_r}\sigma_{r+1}^{a_{r+1}}\cdots\sigma_n^{a_n}=\sum_{(a_i)_i}c_{(a_i)_i}\sigma_1^{a_1}\cdots\sigma_{r-1}^{a_{r-1}}\sigma_r^{a_r+1}\sigma_{r+1}^{a_{r+1}}\cdots\sigma_n^{a_n}.\]
	In other words, this is asking for $c_{(a_i)_i}=c_{(a_i)_i+(1_{i=r})_i}$, or more succinctly just that $c$ is constant in the $i=r$ coordinate.

	Thus, $c$ is constant in all the $i=i_p$ coordinates for each index $i_p$. Thus, we let $d_{(a_i)_{i\notin\{i_p\}}}$ be the restricted function equal to $c_{(a_i)_i}$ but forgetting the information input from any of the $a_{i_p}$. This allows us to write
	\begin{align*}
		z &= \sum_{(a_i)_i}c_{(a_i)_i}\sigma_1^{a_1}\cdots\sigma_m^{a_m} \\
		&= \sum_{(a_i)_{i\notin\{i_p\}}}\sum_{a_{i_1}=0}^{n_{i_1}-1}\cdots\sum_{a_{i_k}=0}^{n_{i_k}-1}d_{(a_i)_{i\notin\{i_p\}}}\sigma_1^{a_1}\cdots\sigma_m^{a_m} \\
		&= \Bigg(\sum_{(a_i)_{i\notin\{i_p\}}}d_{(a_i)_{i\notin\{i_p\}}}\prod_{\substack{i=0\\i\notin\{i_p\}}}^m\sigma_i^{a_i}\Bigg)\Bigg(\sum_{a_{i_1}=0}^{n_{i_1}-1}\sigma_{i_1}^{a_{i_1}}\Bigg)\cdots\Bigg(\sum_{a_{i_k}=0}^{n_{i_k}-1}\sigma_{i_k}^{a_{i_k}}\Bigg),
	\end{align*}
	which is now manifestly in $\im\prod_{p=1}^kN_{i_p}$.
\end{proof}
\begin{lemma} \label{lem:expandgi}
	Fix everything as in the set-up. Then, given $g\coloneqq\prod_{i=1}^m\sigma_i^{a_i}$, we have
	\[g_i=1+\sum_{p=1}^{i-1}g_p\sigma_p^{(a_p)}T_p\]
	for $i\ge1$.
\end{lemma}
\begin{proof}
	This is by induction. For $i=1$, there is nothing to say. For the inductive step, we take $i>1$ where we may assume the statement for $i-1$. Via some relabeling, we may make our inductive hypothesis assert
	\[\prod_{p=2}^{i-1}\sigma_p^{a_p}=1+\sum_{p=2}^{i-1}\Bigg(\prod_{q=2}^{p-1}\sigma_q^{a_q}\Bigg)\sigma_p^{(a_p)}T_p.\]
	In particular, multiplying through by $\sigma_1^{a_1}$ yields
	\begin{align*}
		g_i &= \sigma_1^{a_1}\cdot\prod_{p=2}^{i-1}\sigma_p^{a_p} \\
		&= \sigma_1^{a_1}+\sigma_1^{a_1}\sum_{p=2}^{i-1}\Bigg(\prod_{q=2}^{p-1}\sigma_q^{a_q}\Bigg)\sigma_p^{(a_p)}T_p \\
		&= \sigma_1^{a_1}+\sum_{p=2}^{i-1}g_p\sigma_p^{(a_p)}T_p \\
		&= 1+\sigma_1^{(a_1)}T_1+\sum_{p=2}^{i-1}g_p\sigma_p^{(a_p)}T_p,
	\end{align*}
	which is exactly what we wanted, after a little more rearrangement.
\end{proof}
And mostly because we can, we show that our main short exact sequence splits.
\begin{lemma} \label{lem:sessplits}
	Fix everything as in the set-up. Then consider $\ZZ$-module map $\rho\colon\ZZ[G]^m\to\ZZ[G]^m$ defined by
	\[\rho(g\varepsilon_i)\coloneqq g_i\big(\sigma_i^{a_i}-N_i1_{a_i=n_i-1}\big)\varepsilon_i+\sum_{j=i+1}^mg_j\sigma_j^{(a_j)}T_i\varepsilon_j,\]
	where $g\coloneqq\prod_{i=1}^m\sigma_i^{a_i}$ with $0\le a_i<n_i$. Then $\rho$ descends to a map $\overline\rho\colon\coker\mathcal F\to\ZZ[G]^m$ witnessing the splitting of the short exact sequence
	\[0\to X\to\ZZ[G]^m\to\coker\mathcal F\to0\]
	over $\ZZ$.
\end{lemma}
\begin{proof}
	Observe that we have a well-defined map $\rho\colon\ZZ[G]^m\to\ZZ[G]^m$ because $\ZZ[G]^m$ is a free abelian group generated by $g\varepsilon_i$ for $g\in G$ and indices $i$. It remains to show that $\im\mathcal F\subseteq\ker\rho$ to get a map $\overline\rho\colon\coker\mathcal F\to\ZZ[G]^m$ and then to show that $\rho(z)\equiv z\pmod{\im\mathcal F}$ to get the splitting. We show these individually.

	To show that $\im\mathcal F\subseteq\ker\rho$, we note from \autoref{eq:betterf} that $\im\mathcal F$ is generated over $\ZZ[G]$ by the elements $N_i\varepsilon_i$ and $T_i\varepsilon_j-T_j\varepsilon_i$ for relevant indices $i$ and $j$. Thus, $\im\mathcal F$ is generated over $\ZZ$ by the elements $gN_i\varepsilon_i$ and $gT_i\varepsilon_j-gT_j\varepsilon_i$ for relevant indices $i$ and $j$. Thus, we fix any $g\coloneqq\prod_{i=1}^n\sigma_i^{a_i}$ and show that $gN_i\varepsilon_i\in\ker\rho$ and $gT_i\varepsilon_j-gT_j\varepsilon_i\in\ker\rho$ for relevant indices $i$ and $j$.
	\begin{itemize}
		\item We show $gN_i\varepsilon_i\in\ker\rho$ for any $i$. Because $gN_i=g\sigma_iN_i$, we may as well as assume that $a_i=0$. Then
		\[\rho\left(g\sigma_i^a\varepsilon_i\right)=g_i\big(\sigma_i^{a}-N_i1_{a=n_i-1}\big)\varepsilon_i+\sum_{j=i+1}^mg_j\sigma_i^a\sigma_j^{(a_j)}T_i\varepsilon_j.\]
		As $a$ varies from $0$ to $n_i-1$, we note that the term $g_i\big(\sigma_i^{a}-N_i1_{a=n_i-1}\big)\varepsilon_i$ will only get the $-N_i$ contribution exactly once at $a=n_i-1$. Summing, we thus see that
		\[\rho(gN_i\varepsilon_i)=g_i\Bigg(-N_i+\sum_{a=0}^{n_i-1}\sigma_i^{a}\Bigg)\varepsilon_i+\sum_{a=0}^{n_i-1}\sum_{j=i+1}^mg_j\sigma_i^a\sigma_j^{(a_j)}T_i\varepsilon_j.\]
		The left term vanishes because $N_i=\sum_{a=0}^{n_i-1}\sigma_i^a$. Additionally, the right term vanishes because we can factor $T_i\sum_{a=0}^{n_i-1}\sigma_i^a=T_iN_i=0$. So $gN_i\varepsilon_i\in\ker\rho$.
		\item We show $gT_p\varepsilon_q-gT_q\varepsilon_p\in\ker\rho$ for any $p>q$. Equivalently, we will show that $\rho(g\sigma_p\varepsilon_q)-\rho(g\varepsilon_q)=\rho(g\sigma_q\varepsilon_p)-\rho(g\varepsilon_p)$. On one hand, note
		\begin{align*}
			\rho(g\sigma_p\varepsilon_q) &= g_q\big(\sigma_q^{a_q}-N_i1_{a_q=n_q-1}\big)\varepsilon_q \\
			&\qquad\qquad+\sum_{j=q+1}^{p-1}g_j\sigma_j^{(a_j)}T_q\varepsilon_j \\
			&\qquad\qquad+g_p\left(\sigma_p^{(a_p+1)}-N_p1_{a_p=n_p-1}\right)T_q\varepsilon_p \\
			&\qquad\qquad+\sum_{j=p+1}^m\sigma_pg_j\sigma_j^{(a_j)}T_q\varepsilon_j
		\end{align*}
		because $g_j$ doesn't ``see'' the extra $\sigma_p$ term until $j>p$. (For the $j=p$ term, we would like to write $\sigma_p^{(a_p+1)}$ above, but when $a_p=n_p-1$, we actually end up with $\sigma_p^{(0)}=0$ and hence have to subtract out $\sigma_p^{(n_p)}=N_p$.) Thus,
		\[\rho(g\sigma_p\varepsilon_q)-\rho(g\varepsilon_q) = g_p\left(\sigma_p^{a_p}-N_p1_{a_p=n_p-1}\right)T_q\varepsilon_p+\sum_{j=p+1}^mg_j\sigma_j^{(a_j)}T_pT_q\varepsilon_j.\]
		On the other hand, we have
		\[\rho(g\sigma_q\varepsilon_p) = \sigma_qg_p\big(\sigma_p^{a_p}-N_p1_{a_p=n_p-1}\big)\varepsilon_p+\sum_{j=p+1}^m\sigma_qg_j\sigma_j^{(a_j)}T_p\varepsilon_j\]
		where this time all $j>p$ also have $j>q$ and so $(\sigma_qg)_j=\sigma_qg_j$. Thus,
		\[\rho(g\sigma_q\varepsilon_p)-\rho(g\varepsilon_p) = g_p\left(\sigma_p^{a_p}-N_p1_{a_p=n_p-1}\right)T_q\varepsilon_p+\sum_{j=p+1}^mg_j\sigma_j^{(a_j)}T_pT_q\varepsilon_j,\]
		as desired.
	\end{itemize}
	We now check the splitting. For this, we simply need to check that $\rho(g\varepsilon_i)\equiv g\varepsilon_i\pmod{\im\mathcal F}$, and we will get the result for all elements of $\ZZ[G]^m$ by additivity of $\rho$. Well, using \autoref{lem:expandgi}, we write
	\begin{align*}
		g\varepsilon_i &= g_i\sigma_i^{a_i}\Bigg(\prod_{j=i+1}^m\sigma_j^{a_j}\Bigg)\varepsilon_i \\
		&= g_i\sigma_i^{a_i}\Bigg(1+\sum_{j=i+1}^m\Bigg(\prod_{q=i+1}^{j-1}\sigma_q^{a_q}\Bigg)\sigma_j^{(a_j)}T_j\Bigg)\varepsilon_i \\
		&= g_i\sigma_i^{a_i}\varepsilon_i+\sum_{j=i+1}^mg_i\sigma_i^{a_i}\Bigg(\prod_{q=i+1}^{j-1}\sigma_q^{a_q}\Bigg)\sigma_j^{(a_j)}T_j\varepsilon_i \\
		&\equiv g_i\sigma_i^{a_i}\varepsilon_i+\sum_{j=i+1}^mg_j\sigma_j^{(a_j)}T_i\varepsilon_j,
	\end{align*}
	where in the last step we have used the fact that $T_j\varepsilon_i\equiv T_j\varepsilon_i\pmod{\im\mathcal F}$. Lastly, we note that $hN_i\varepsilon_i\equiv h\varepsilon_i\pmod{\im\mathcal F}$ for any $h\in G$, so in fact
	\[g\varepsilon_i\equiv g_i\left(\sigma_i^{a_i}-N_i1_{a_i=n_i-1}\right)\varepsilon_i+\sum_{j=i+1}^mg_j\sigma_j^{(a_j)}T_i\varepsilon_j,\]
	and now the right-hand side is $\rho(g\varepsilon_i)$.
\end{proof}
% \begin{remark}
% 	The purpose of \autoref{lem:sessplits} is to give an injective map from $\coker\mathcal F$ to a more controlled setting. In particular, it is somewhat annoying to check if an element $z\in\ZZ[G]^m$ lives in $\im\mathcal F$, but it is easier to check the equivalent condition $\overline\rho(z)=0$.
% \end{remark}
% We are now ready to more directly attack the proof of \autoref{prop:allmanufacturedcocycles}. We begin by reducing the amount of data we have to carry around in a cocycle.
% \begin{lemma} \label{lem:compresscocycle}
% 	Fix everything as in the set-up, and let $A$ be a $G$-module. Then, if $f\in Z^1(G,A)$ is a cocycle, then
% 	\[f(g)=\sum_{i=1}^mg_i\sigma_i^{(a_i)}f(\sigma_i),\]
% 	where $g\coloneqq\prod_{i=1}^m\sigma_i^{a_i}$ with $a_i\ge0$.
% \end{lemma}
% \begin{proof}
% 	Unsurprisingly, this is by induction. To begin, we claim that
% 	\[f\left(\sigma^a\right)=\sigma^{(a)}f(\sigma)\]
% 	by induction on $a$. When $a=0$, we are showing that $f(1)=0$, for which we note that the $1$-cocycle condition implies $f(1)=f(1)+f(1)$ and so $f(1)=0$. Then for the inductive step, we assume $f(\sigma^a)=\sigma^{(a)}f(\sigma)$ and note
% 	\[f\left(\sigma^{a+1}\right)=\sigma f\left(\sigma^a\right)+f(\sigma)=\left(1+\sigma\sigma^{(a)}\right)f(\sigma)=\sigma^{(a+1)}f(\sigma),\]
% 	finishing.

% 	We now show the original statement by an induction on $m$. For $m=0$, this is asserting $f(1)=0$, which is true. Then for the inductive step, we assume for $m-1$ and note that $m>1$ has
% 	\[f\left(g_m\sigma_m^{a_m}\right)=f(g_m)+g_mf\left(\sigma_m^{a_m}\right)=\sum_{i=1}^{m-1}g_i\sigma_i^{(a_i)}f(\sigma_i)+g_m\sigma_m^{(a_m)}f(\sigma_m),\]
% 	which is what we wanted.
% \end{proof}
% Thus, to build a $1$-cocycle, we only have to specify $f(\sigma_i)$ for indices $i$ and then check the $1$-cocycle condition to make sure we are okay.

% As such, we now run through what the $1$-cocycle check requires.
% \begin{lemma} \label{lem:cocycleforcecoord}
% 	Fix everything as in the set-up. Further, fix some $z\in\ZZ[G]^m$. Then $N_iz\in\im\mathcal F$ if and only if $[z]\in\coker\mathcal F$ has a representative of the form $a_i\varepsilon_i\in\ZZ[G]^m$ where $a_i\in\ZZ[G]$.
% \end{lemma}
% \begin{proof}
% 	In one direction, if $z\equiv a_i\varepsilon_i\pmod{\im\mathcal F}$, then
% 	\[N_iz\equiv a_i\cdot N_i\varepsilon_i\equiv a_i\cdot0\equiv0\pmod{\im\mathcal F}\]
% 	because $N_i\varepsilon_i\in\im\mathcal F$.

% 	In the other direction, we pass through $\overline\rho$ of \autoref{lem:sessplits}. By possibly rearranging our $\sigma$s, we may set $i=1$. As such, suppose $N_1z\in\im\mathcal F$, and write
% 	\[z\coloneqq\sum_{i=1}^mz_i\varepsilon_i\]
% 	where $z_i\in\ZZ[G]$. By using the fact that $T_i\varepsilon_1\equiv T_1\varepsilon_i\pmod{\im\mathcal F}$ for any index $i$, we can find a representative for $z$ in $\ZZ[G]^m$ such that $z_i$ has no $\sigma_1$ powers for each $i>1$; without loss of generality, replace $z$ with this representative.
	
% 	We thus claim that $w\coloneqq z-z_1\varepsilon_1\in\im\mathcal F$, which means that $z$ is represented by $z_1\varepsilon_1$; to show this, we already know that $N_1w=N_1(z-z_1\varepsilon_1)\in\im\mathcal F$, so we pass through $\overline\rho$. In other words, it suffices to show that $\rho(w)=0$ from $\rho(N_1w)=0$ and the fact that $w$ features no $\sigma_1$ nor $\varepsilon_1$ terms.
	
% 	Well, because $w$ features no $\sigma_1$ nor $\varepsilon_1$ terms, the only terms we care about have $g\varepsilon_i$ where $g$ has no $\sigma_1$ and $i>1$; in this case,
% 	\[\rho\left(g\sigma_1^a\varepsilon_i\right)\coloneqq\sigma_1^ag_i\big(\sigma_i^{a_i}-N_i1_{a_i=n_i-1}\big)\varepsilon_i+\sum_{j=i+1}^m\sigma_1^ag_j\sigma_j^{(a_j)}T_i\varepsilon_j=\sigma_1^a\rho(g\varepsilon_i),\]
% 	where $g\coloneqq\prod_{i=2}^m\sigma_i^{a_i}$ with $0\le a_i<n_i$. Looping over all possible $g$ and $\varepsilon_i$, we see $\rho(\sigma_1^aw)=\sigma_1^a\rho(w)$, so
% 	\[N_1\rho(w)=\rho(N_1w)=0.\]
% 	Thus, $\rho(w)\in\im T_1$, so say $\rho(w)=(\sigma_1-1)w'$. However, because $w$ has no $\varepsilon_1$ terms nor any term with a $\sigma_1$, we can see from the expansion of $\rho(w)$ that $\rho(w)$ will have no $\sigma_1$ terms. It follows that $\rho(w)\in\ZZ[G]^m$ is preserved upon applying $\sigma_1\mapsto1$, but then $(\sigma_1-1)w'$ gets sent to $0$, so it follows $\rho(w)=0$. This finishes.
% \end{proof}
% \begin{lemma} \label{lem:cocycleforcecohere}
% 	Fix everything as in the set-up. Suppose we have $\{z_i\}_{i=1}^m\subseteq\ZZ[G]$ such that
% 	\[T_iz_j\varepsilon_j=T_jz_i\varepsilon_i\]
% 	in $\coker\mathcal F$, for any pair of indices $(i,j)$. Then there exists $z\in\ZZ[G]$ such that $z\varepsilon_i=z\varepsilon_i$ (in $\coker\mathcal F$) for each index $i$.
% \end{lemma}
% \begin{proof}
% 	We proceed by induction on $m$. For $m=1$, we simply set $z\coloneqq z_1$. For the inductive step, take $m>1$, and we are given elements $\{z_i\}_{i=1}^m\subseteq\ZZ[G]$ such that
% 	\[T_iz_j\varepsilon_j=T_jz_i\varepsilon_i\]
% 	for any pair of indices $(i,j)$. By the inductive hypothesis, we may use the equations with indices less than $m$ to conjure some $z\in\ZZ[G]$ such that
% 	\[z\varepsilon_i\equiv z_i\varepsilon_i\pmod{\im\mathcal F}\]
% 	for each $i<m$. It remains to deal with the equations which have $m$ as an index; namely, for each $i<m$, we have an equation
% 	\[T_iz_m\varepsilon_m\equiv T_mz_i\varepsilon_i\equiv T_mz\varepsilon_i\pmod{\im\mathcal F}.\]
% 	Now, $T_m\varepsilon_i\equiv T_i\varepsilon_m\pmod{\im\mathcal F}$, so this is equivalent to asserting
% 	\[T_i(z_m-z)\varepsilon_m\equiv0\pmod{\im\mathcal F}\]
% 	for each index $i<m$. Thus, $T_i(z_m-z)\varepsilon_m\in\im\mathcal F$ for each $i$, which we will use by passing through the $\rho$ of \autoref{lem:sessplits}: this is equivalent to $\rho(T_i(z_m-z)\varepsilon_m)=0$ for each $i<m$. Now, we note that any $g=\prod_{j=1}^m\sigma_j^{a_j}\sigma\in G$ and $i<m$ will have
% 	\[\rho(\sigma_ig\varepsilon_m)=\sigma_ig_m\big(\sigma_m^{a_m}-N_i1_{a_m=n_m-1}\big)\varepsilon_m=\sigma_i\rho(g\varepsilon_m),\]
% 	where in particular the sum in $\rho$ vanished because $m$ is the largest index. (Also, we note $(\sigma_ig)_m=\sigma_ig_m$ because $i<m$.) Extending this linearly over all $g\in G$, we see that
% 	\[0=\rho(T_i(z_m-z)\varepsilon_m)=T_i\rho((z_m-z)\varepsilon_m)\]
% 	for each $i<m$. In particular, letting $\rho((z_m-z)\varepsilon_m)=r\varepsilon_m$, we see$r\in\im N_i$ for each $i<m$, so it follows from \autoref{lem:separatenijs} that $r\in\im N_1\cdots N_{m-1}$, so we can find $w\in\ZZ[G]$ such that
% 	\[\rho((z_m-z)\varepsilon_m)=N_1\cdots N_{m-1}w\varepsilon_m.\]
% 	Now, for technical reasons we note that any $g=\prod_{j=1}^m\sigma_j^{a_j}$ gives
% 	\[\rho(g\varepsilon_m)=g_m\big(\sigma_m^{a_m}-N_i1_{a_m=n_m-1}\big)\varepsilon_m,\]
% 	which can have no $\sigma_m^{n_m-1}$ term in it because this would have to come from $\big(\sigma_m^{a_m}-N_i1_{a_m=n_m-1}\big)$, which manually kills all such terms. As such, $N_1\cdots N_{m-1}w$ should have no $\sigma_m^{n_m-1}$ terms, which means $w$ itself should have no such terms.

% 	With this in mind, we set $z'\coloneqq z+N_1\cdots N_{m-1}w$. To check that we haven't broken anything, we note that any $i<m$ has
% 	\[z'\varepsilon_i=z\varepsilon_i+N_1\cdots N_{m-1}w\varepsilon_i\equiv z\varepsilon_i\equiv z_i\varepsilon_i\pmod{\im\mathcal F}\]
% 	where we note that $N_i\varepsilon_i\equiv0\pmod{\im\mathcal F}$. It remains to deal with $i=m$. Because $w$ features no $\sigma_m^{a_m-1}$ terms, we can check that any $g=\prod_{j=1}^m\sigma_j^{a_j}$ with $a_m<n_m-1$ has
% 	\[\rho(g\varepsilon_m)=g_m\big(\sigma_m^{a_m}-N_i1_{a_m=n_m-1}\big)\varepsilon_m=g_m\sigma_m^{a_m}\varepsilon_m=g\varepsilon_m,\]
% 	so $\rho$ will just act as the identity on $w$! Extending this linearly, we see that
% 	\begin{align*}
% 		\rho((z_m-z')\varepsilon_m) &= \rho((z_m-z)\varepsilon_m)-\rho(N_1\cdots N_{m-1}w\varepsilon_m) \\
% 		&= N_1\cdots N_{m-1}w\varepsilon_m-N_1\cdots N_{m-1}w\varepsilon_m \\
% 		&= 0.
% 	\end{align*}
% 	Thus, $(z_m-z')\varepsilon_m\in\im\mathcal F$, so $z_m\varepsilon_m\equiv z\varepsilon_m\pmod{\im\mathcal F}$ as well.
% \end{proof}
% We are now ready to classify our $1$-cocycles.
% \begin{proposition} \label{prop:cocycleclassify}
% 	Fix everything as in the set-up. If $f\in Z^1(G,\coker\mathcal F)$ is a $1$-cocycle, then there exists $z\in\ZZ[G]$ such that $f(\sigma_i)=z\varepsilon_i$ for each index $i$. Combined with the formula in \autoref{lem:compresscocycle}, this fully determines $f$.
% \end{proposition}
% \begin{proof}
% 	We start by noting that each index $i$ has
% 	\[0=f(1)=f\left(\sigma_i^{n_i}\right)=\sigma_i^{(n_i)}f(\sigma_i)=N_i(f(\sigma_i))\]
% 	by plugging in $\sigma_i^{n_i}$ into \autoref{lem:compresscocycle}. Thus, \autoref{lem:cocycleforcecoord} grants us some $z_i\in\ZZ[G]$ such that $f(\sigma_i)=z_i\varepsilon_i$ for each index $i$.

% 	Continuing, we note that each pair of indices $(i,j)$ has
% 	\[\sigma_if(\sigma_j)+f(\sigma_i)=f(\sigma_i\sigma_j)=f(\sigma_j\sigma_i)=\sigma_jf(\sigma_i)+f(\sigma_j),\]
% 	so
% 	\[T_iz_j\varepsilon_j=T_if(\sigma_j)=T_jf(\sigma_i)=T_jz_i\varepsilon_i.\]
% 	Thus, we know from \autoref{lem:cocycleforcecohere} that there exists $z\in\ZZ[G]$ such that $f(\sigma_i)=z_i\varepsilon_i=z\varepsilon_i$ for each index $i$. This completes the proof.
% \end{proof}
% Note that \autoref{prop:cocycleclassify} does not say that all the conjured $1$-cocycles are actually $1$-cocycles. It will be beneficial for us to show this by hand, so we postpone it to the next subsection.

\subsection{Verification of 1-Cocycles}
Here we prove \autoref{prop:manufacturedcocycle}.
% verify that all the $1$-cocycles of \autoref{prop:cocycleclassify} are indeed $1$-cocycles.
Namely, we show that the $1$-cochain $\overline c\in C^1(G,\coker\mathcal F)$ defined by
\[\overline c(g)=\sum_{i=1}^mg_i\sigma_i^{(a_i)}\varepsilon_i\]
where $g\coloneqq\prod_{i=1}^m\sigma_i^{a_i}$ is actually a $1$-cocycle. It will be beneficial for us to do this by hand, which is a matter of brute force. Set $c\in C^1\left(G,\ZZ[G]^m\right)$ defined by
\[c(g)\coloneqq\left(g_i\sigma_i^{(a_i)}\right)^m_{i=1},\]
where $g\coloneqq\prod_{i=1}^m\sigma_i^{a_i}$. We will show that $\im dc\subseteq\im\mathcal F$, which we will mean that $\im\overline{dc}=\im d\overline c=0$, where $f\mapsto\overline f$ is the map $C^\bullet\left(G,\ZZ[G]^m\right)\onto C^\bullet\left(G,\coker\mathcal F\right)$ induced by modding out.

As such, we set $g\coloneqq\prod_{i=1}^m\sigma_i^{a_i}$ and $h\coloneqq\prod_{i=1}^m\sigma_i^{b_i}$ with $0\le a_i,b_i<n_i$ for each $i$. Then, using the division algorithm, write
\[a_i+b_i=n_iq_i+r_i\]
where $q_i\in\{0,1\}$ and $0\le r_i<n_i$ for each $i$. Now, we want to show $dc(g,h)\in\im\mathcal F$, so we begin by writing
\begin{align}
	dc(g,h) &= gc(h)-c(gh)+c(g) \notag \\
	&= g\left(h_i\sigma_i^{(b_i)}\right)_{i=1}^m-\Bigg(\prod_{p=0}^{i-1}\sigma_p^{r_p}\cdot\sigma_i^{(r_i)}\Bigg)_{i=1}^m+\left(g_i\sigma_i^{(a_i)}\right)_{i=1}^m \notag \\
	&= \left(gh_i\sigma_i^{(b_i)}\right)_{i=1}^m-\left(g_ih_i\sigma_i^{(r_i)}\right)_{i=1}^m+\left(g_i\sigma_i^{(a_i)}\right)_{i=1}^m. \label{eq:expandedcocycle}
\end{align}
We now go term-by-term in \autoref{eq:expandedcocycle}. The easiest is the middle term of \autoref{eq:expandedcocycle}, for which we write
\begin{align*}
	g_ih_i\sigma_i^{(r_i)} &= g_ih_i\sigma_i^{(a_i+b_i)}-g_ih_i\sigma_i^{r_i}\sigma_i^{(n_iq_i)} \\
	&= g_ih_i\sigma_i^{(a_i+b_i)}-g_ih_i\sigma_i^{a_i+b_i}\cdot q_iN_i \\
	&= g_ih_i\sigma_i^{(a_i+b_i)}-g_ih_i\cdot q_iN_i,
\end{align*}
where the last equality is because $\sigma_iN_i=N_i$. Thus,
\begin{align*}
	-\left(g_ih_i\sigma_i^{(r_i)}\right)_{i=1}^m &= -\left(g_ih_i\sigma_i^{(a_i+b_i)}\right)_{i=1}^m+\left(g_ih_i\cdot q_iN_i\right)_{i=1}^m \\
	&= -\left(g_ih_i\sigma_i^{(a_i+b_i)}\right)_{i=1}^m+\mathcal F\big((g_ih_iq_i)_i,(0)_{i>j}\big).
\end{align*}
Now, using \autoref{lem:expandgi}, the $i$th coordinate of the left term of \autoref{eq:expandedcocycle} is
\begin{align*}
	gh_i\sigma_i^{(b_i)} &= g_i\sigma_i^{a_i}\Bigg(\prod_{ j=i+1}^{m}\sigma_j^{a_j}\Bigg)h_i\sigma_i^{(b_i)} \\
	&= g_i\Bigg(1+\sum_{j=i+1}^{m}\Bigg(\prod_{q=i+1}^{j-1}\sigma_q^{a_q}\Bigg)\sigma_j^{(a_j)}T_j\Bigg)h_i\sigma_i^{a_i}\sigma_i^{(b_i)} \\
	&= g_ih_i\sigma_i^{a_i}\sigma_i^{(b_i)}+\sum_{j=i+1}^{m}\Bigg(g_i\sigma_i^{a_i}\prod_{q=i+1}^{j-1}\sigma_q^{a_q}\Bigg)h_i\sigma_j^{(a_j)}\sigma_i^{(b_i)}T_j \\
	&= g_ih_i\sigma_i^{a_i}\sigma_i^{(b_i)}+\sum_{j=i+1}^{m}g_jh_i\sigma_j^{(a_j)}\sigma_i^{(b_i)}T_j.
\end{align*}
And lastly, for the right term of \autoref{eq:expandedcocycle}, the $i$th coordinate is
\begin{align*}
	g_i\sigma_i^{(a_i)} &= g_i\Bigg(h_i-\sum_{j=1}^{i-1}h_j\sigma_j^{(b_j)}T_j\Bigg)\sigma_i^{(a_i)} \\
	&= g_ih_i\sigma_i^{(a_i)}-\sum_{j=1}^{i-1}g_ih_j\sigma_i^{(a_i)}\sigma_j^{(b_j)}T_j.
\end{align*}
So to finish, we continue from \autoref{eq:expandedcocycle}, which gives
\begin{align*}
	dc(g,h)-\mathcal F\big((g_ih_iq_i)_i,(0)_{i>j}\big) &= \left(g_ih_i\sigma_i^{a_i}\sigma_i^{(b_i)}\right)_{i=1}^m-\left(g_ih_i\sigma_i^{(a_i+b_i)}\right)_{i=1}^m+\left(g_ih_i\sigma_i^{(a_i)}\right)_{i=1}^m \\
	&\qquad\qquad+\Bigg(\sum_{j=i+1}^{m}g_jh_i\sigma_j^{(a_j)}\sigma_i^{(b_i)}T_j-\sum_{j=1}^{i-1}g_ih_j\sigma_i^{(a_i)}\sigma_j^{(b_j)}T_j\Bigg)_{i=1}^m \\
	&= \Bigg(-\sum_{j=1}^{i-1}g_ih_j\sigma_i^{(a_i)}\sigma_j^{(b_j)}T_j+\sum_{j=i+1}^{m}g_jh_i\sigma_j^{(a_j)}\sigma_i^{(b_i)}T_j\Bigg)_{i=1}^m \\
	&= \mathcal F\left((0)_i,\big(g_ih_j\sigma_i^{(a_i)}\sigma_j^{(b_j)}\big)_{i>j}\right).
\end{align*}
Thus,
\begin{equation}
	dc(g,h) = \mathcal F\left((g_ih_iq_i)_i,\big(g_ih_j\sigma_i^{(a_i)}\sigma_j^{(b_j)}\big)_{i>j}\right)\in\im\mathcal F. \label{eq:computedelta}
\end{equation}
This completes the proof of \autoref{prop:manufacturedcocycle}.

In fact, the above proof has found an explicit element $z$ so that $\mathcal F(z)=dc(g,h)$ for each $g,h\in G$. As such, we recall that we set
\[X\coloneqq\frac{\ZZ[G]^m\times\ZZ[G]^{\binom m2}}{\ker\mathcal F}\]
to give the short exact sequence
\[0\to X\stackrel{\mathcal F}\to\ZZ[G]^m\to\coker\mathcal F\to0.\]
In particular, we can track $\overline c\in Z^1(G,\coker\mathcal F)$ through a boundary morphism: we already have a chosen lift $c\in Z^1(G,\ZZ[G]^m)$ for $\overline c$, and we have also computed $\mathcal F^{-1}\circ dc$ from the above work. This gives the following result.
\begin{cor} \label{cor:deltaccomputation}
	Fix everything as in the set-up. Then the $\overline c$ of \autoref{prop:manufacturedcocycle} has
	\[\delta(c)(g,h)\coloneqq\left((g_ih_iq_i)_i,\big(g_ih_j\sigma_i^{(a_i)}\sigma_j^{(b_j)}\big)_{i>j}\right)\in Z^2(G,X)\]
	where $\delta$ is induced by
	\[0\to X\stackrel{\mathcal F}\to\ZZ[G]^m\to\coker\mathcal F\to0.\]
\end{cor}
\begin{proof}
	This follows from tracking how $\delta$ behaves, using \autoref{eq:computedelta}.
\end{proof}
\begin{remark}
	In some sense, this $\delta(c)$ is exactly the cocycle of \autoref{thm:getcocycle}, where we have abstracted away everything about $A$. We will rigorize this notion in our proof of \autoref{thm:yesitisacocycle}.
\end{remark}
% We are now ready to complete the proof of \autoref{prop:allmanufacturedcocycles}. In fact, we show the following stronger result.
% \begin{proposition} \label{prop:computeh1cokerF}
% 	Fix everything as in the set-up. Further, let $\varepsilon\colon\ZZ[G]\to\ZZ$ be the augmentation map sending $\sigma_i\mapsto1$ for each $i$. Then the following are true.
% 	\begin{listalph}
% 		\item Given any $z\in\ZZ[G]$, the formula
% 		\[f(g)=\sum_{i=1}^mg_i\sigma_i^{(a_i)}\cdot z\varepsilon_i=(z\cdot\overline c)(g)\]
% 		for $g\coloneqq\prod_{i=1}^m\sigma_i^{a_i}$ defines a $1$-cocycle in $Z^1(G,\coker\mathcal F)$. These are all the $1$-cocycles.
% 		\item If $f\in Z^1(G,\coker\mathcal F)$ is a $1$-cocycle, then $[f]=[\varepsilon(z)\cdot\overline c]$ in $H^1(G,\coker\mathcal F)$, for the $z\in\ZZ[G]$ of \autoref{prop:cocycleclassify}. In particular, $H^1(G,\coker\mathcal F)$ is a cyclic abelian group generated by $[\overline c]$.
% 	\end{listalph}
% \end{proposition}
% \begin{proof}
% 	We proceed one at a time.
% 	\begin{listalph}
% 		\item Given $z\in\ZZ[G]$, to see that $f$ is a $1$-cocycle, note that $f=z\cdot\overline c$. Thus, for the $1$-cocycle check, we just note that any $g,h\in G$ have
% 		\begin{align*}
% 			f(gh) &= z\cdot\overline c(gh) \\
% 			&= z\cdot(g\overline c(h)+\overline c(g)) \\
% 			&= gf(h)+f(g)
% 		\end{align*}
% 		because we already know that $\overline c\in Z^1(G,\coker\mathcal F)$.

% 		To see that these are all the $1$-cocycles, let $f\in Z^1(G,\coker\mathcal F)$ be any $1$-cocycle. Then \autoref{prop:cocycleclassify} promises $z\in\ZZ[G]$ such that $f(\sigma_i)=z\varepsilon_i$ for each index $i$, for which \autoref{lem:compresscocycle} tells us that
% 		\[f(g)=\sum_{i=1}^mg_i\sigma_i^{(a_i)}f(\sigma_i)=\sum_{i=1}^mg_i\sigma_i^{(a_i)}\cdot z\varepsilon_i\]
% 		for $g\coloneqq\prod_{i=1}^m\sigma_i^{a_i}$. So $f$ does have the desired form.

% 		\item Fix $f\in Z^1(G,\coker\mathcal F)$, and conjure the corresponding $z\in\ZZ[G]$ of \autoref{prop:cocycleclassify}. We note in part (a) that $f=z\cdot\overline c$, so it remains to show that $[z\cdot\overline c]=[\varepsilon(z)\cdot\overline c]$ in $H^1(G,\coker\mathcal F)$.
		
% 		By linearity of $\ZZ[G]$, it suffices to show that $[g\cdot\overline c]=[\overline c]$ for each $g\in G$. By induction on the number of generators $\sigma_i$ appearing in $g\in G$, it suffices to show that $[\sigma_i\cdot\overline c]=[\overline c]$ for each index $i$. Lastly, by rearranging the $\sigma_i$, it suffices to show that $[\sigma_1\cdot\overline c]=[\overline c]$.

% 		Well, for any $\sigma_i$, we note that
% 		\[(\sigma_1\overline c-\overline c)(\sigma_i)=\sigma_1\varepsilon_i-\varepsilon_i=T_1\varepsilon_i=T_i\varepsilon_1,\]
% 		where in the last equality we have used that we're living in $\coker\mathcal F$. Letting $d\colon C^0(G,\coker\mathcal F)\to B^1(G,\coker\mathcal F)$ denote the corresponding differential, we see
% 		\[(\sigma_1\overline c-\overline c-d\varepsilon_1)(\sigma_i)=T_i\varepsilon_1-(\sigma_i-1)\varepsilon_1=0\]
% 		for each index $i$. Thus, $\sigma_1\overline c-\overline c-d\varepsilon_1)\in Z^1(G,\coker\mathcal F)$ vanishes on all $\sigma_i$, so \autoref{lem:compresscocycle} tells us that it vanishes on all $g\in G$. It follows $[\sigma_1\overline c-\overline c]=[0]$, which finishes.
% 	\end{listalph}
% 	The above parts complete the proof.
% \end{proof}
% \begin{cor} \label{cor:computeh2x}
% 	Fix everything as in the set-up. Then $H^2(G,X)$ is a cyclic abelian group generated by $[\delta(\overline c)]$, where $\delta$ is induced by
% 	\[0\to X\stackrel{\mathcal F}\to\ZZ[G]^m\to\coker\mathcal F\to0.\]
% \end{cor}
% \begin{proof}
% 	From the long exact sequence of cohomology, we see that
% 	\[\delta\colon H^1(G,\coker\mathcal F)\to H^2(G,X)\]
% 	is an isomorphism because $\ZZ[G]^m$ is projective and hence acyclic. Thus, this follows from (b) of \autoref{prop:computeh1cokerF}.
% \end{proof}

\subsection{Tuples via Cohomology}
We continue in the set-up of the previous subsection. The goal of this subsection is to prove \autoref{prop:alternativetuple}. The main idea is that we will be able to finitely generate $\ker\mathcal F$ essentially using the relations of a $\{\sigma_i\}_{i=1}^m$-tuple.

We start with the following basic result.
\begin{lemma} \label{lem:getgens}
	Fix everything as in the set-up. Then $\ker\mathcal F$ contains the following elements.
	\begin{listalph}
		\item $T_p\kappa_p$ for any index $p$.
		\item $N_pN_q\lambda_{pq}$ for any pair of indices $(p,q)$ with $p>q$.
		\item $T_q\kappa_p+N_p\lambda_{pq}$ for any pair of indices $(p,q)$ with $p>q$.
		\item $T_p\kappa_q-N_q\lambda_{pq}$ for any pair of indices $(p,q)$ with $p>q$.
		\item $T_q\lambda_{pr}-T_r\lambda_{pq}-T_p\lambda_{qr}$ for any triplet of indices $(p,q,r)$ with $p>q>r$.
	\end{listalph}
\end{lemma}
\begin{proof}
	We start by showing that all the listed elements are in fact in $\ker\mathcal F$.
	\begin{listalph}
		\item Note that $\mathcal F$ only ever takes the $x_i$ term to $x_iN_i$, so if $x_i=T_i$, then the effect of $x_i$ vanishes.
		\item Similarly, note that $\mathcal F$ only ever takes the $y_{ij}$ term to $y_{ij}T_i$ or $y_{ij}T_j$. As such, if $y_{ij}=N_iN_j$, then the effect of $y_{ij}$ vanishes again.
		\item The only relevant terms are at indices $p$ and $q$. Here, $i=p$ has $\mathcal F$ output
		\[T_qN_p-N_pT_q+0=0.\]
		For $i=q$, we have no $x_q$ term, so we are left with $N_pT_p=0$.
		\item Again, the only relevant terms are at indices $p$ and $q$. This time the interesting term is at $i=q$, where we have
		\[T_pN_q-0+(-N_q)T_p=0.\]
		Then at $i=p$, we simply have $0N_p-(-N_q)T_q+0=0$.
		\item The relevant terms, as usual, are for $i\in\{p,q,r\}$.
		\begin{itemize}
			\item At $i=p$, we have $0-(T_qT_r+(-T_r)T_q)+0=0.$
			\item At $i=q$, we have $0-(-(T_p)T_r)+((-T_r)T_p)=0$.
			\item At $i=r$, we have $0-0+(T_qT_p+(-T_p)T_q)=0$.
		\end{itemize}
	\end{listalph}
	The above checks complete this part of the proof.
\end{proof}
\begin{remark}
	The above elements are intended to encode the relations to be a $\{\sigma_i\}_{i=1}^n$-tuple. We will see this made rigorous in the proof of \autoref{prop:alternativetuple}.
\end{remark}
In fact, the following is true.
\begin{lemma} \label{lem:havegens}
	Fix everything as in the set-up. Then the elements (a)--(e) of \autoref{lem:getgens}, with (b) removed, generate $\ker\mathcal F$.
\end{lemma}
\begin{proof}
	We remark that we callously removed (b) because it is implied (c): $T_q\kappa_p+N_p\lambda_{pq}\in\ker\mathcal F$ implies that
	\[N_q\cdot(T_q\kappa_p+N_p\lambda_{pq})=N_pN_q\lambda_{pq}\]
	is also in $\ker\mathcal F$. Anyway, this proof is long and annoying and hence relegated to \autoref{sec:havegensproof}.
\end{proof}
Here is the payoff for the hard work in \autoref{lem:havegens}.
\propalternativetuple*
\begin{proof}
	Let $\mathcal T$ denote the set of $\{\sigma_i\}_{i=1}^m$-tuples. We now define the map $\varphi\colon\op{Hom}_{\ZZ[G]}(X,A)\to\mathcal T$ by
	\[\varphi\colon f\mapsto\Big(\big(f(\kappa_i)\big)_i,\big(f(\lambda_{ij})\big)_{i>j}\Big).\]
	In other words, we simply read off the values of $f$ from indicators on the coordinates of $X$. It's not hard to see that $\varphi$ is in fact a $G$-module homomorphism, but we will have to check that $\varphi$ is well-defined, for which we have to check the conditions on being a $\{\sigma_i\}_{i=1}^m$-tuple.
	\begin{lemma} \label{lem:kernelisrelations}
		Fix everything as in the set-up, and let $A$ be a $G$-module. Then, given $f\colon\ZZ[G]^m\times\ZZ[G]^{\binom m2}$, we have $\ker\mathcal F\subseteq\ker f$ if and only if
		\[\Big(\big(f(\kappa_i)\big)_i,\big(f(\lambda_{ij})\big)_{i>j}\Big)\]
		is a $\{\sigma_i\}_{i=1}^m$-tuple.
	\end{lemma}
	\begin{proof}
		By \autoref{lem:havegens}, we see $\ker\mathcal F\subseteq\ker f$ if and only if $f$ vanishes on the elements given in \autoref{lem:getgens}. As such, we now run the following checks.
		\begin{enumerate}
			\item We discuss \autoref{eq:tuplefields}. For one, note that $f(\lambda_{ij})\in A$ essentially for free. Now, we note
			\begin{align*}
				f(\kappa_i)\in A^{\langle\sigma_i\rangle} &\iff T_if(\kappa_i)=0 \\
				&\iff f(T_i\kappa_i)=0 \\
				&\iff T_i\kappa_i\in\ker f.
			\end{align*}
			\item We discuss \autoref{eq:tuplerelations}. On one hand, note that $i>j$ has
			\begin{align*}
				N_if(\lambda_{ij})=-T_jf(\lambda_i) &\iff f(N_i\lambda_{ij}+T_j\lambda_i) \\
				&\iff N_i\lambda_{ij}+T_j\lambda_i\in\ker f.
			\end{align*}
			On the other hand,
			\begin{align*}
				-N_jf(\lambda_{ij})=-T_if(\lambda_j) &\iff f(N_j\lambda_{ij}+T_i\lambda_j)=0 \\
				&\iff N_j\lambda_{ij}+T_i\lambda_j\in\ker f.
			\end{align*}
			\item We discuss \autoref{eq:betarelations}. Simply note indices $i>j>k$ have
			\begin{align*}
				T_jf(\lambda_{ik})=T_kf(\lambda_{ij})+T_if(\lambda_{jk}) &\iff f(T_j\lambda_{ik}-T_k\lambda_{ij}-T_i\lambda_{jk})=0 \\
				&\iff T_j\lambda_{ik}-T_k\lambda_{ij}-T_i\lambda_{jk}\in\ker f.
			\end{align*}
		\end{enumerate}
		In total, we see that satisfying the relations to be a $\{\sigma_i\}_{i=1}^m$-tuple exactly encodes the data of having the generators of $\ker\mathcal F$ live in $\ker f$.
	\end{proof}
	So indeed, given $f\colon X\to A$, the above lemma applied to the composite
	\[\ZZ[G]^m\times\ZZ[G]^{\binom m2}\onto X\stackrel{f}\to A\]
	shows that $\varphi(f)\in\mathcal T$.

	To show that $\varphi$ is an isomorphism, we exhibit its inverse; fix some $(\{\alpha_i\},\{\beta_{ij}\}_{i>j})\in\mathcal T$. Well, $\ZZ[G]\times\ZZ[G]^{\binom m2}$ has as a basis the $\kappa_i$ and $\lambda_{ij}$, so we can uniquely define a $G$-module homomorphism $f\colon X\to A$ by
	\[f(\kappa_i)\coloneqq\alpha_i\qquad\text{and}\qquad f(\lambda_{ij})\coloneqq\beta_{ij}\]
	for all relevant indices $i,j$, and in fact the map $\mathcal T\to\op{Hom}_\ZZ\left(\ZZ[G]^m\times\ZZ[G]^{\binom m2},A\right)$ we can see to be a $G$-module homomorphism. However, because these outputs are a $\{\sigma_i\}_{i=1}^m$-tuple, we can read \autoref{lem:kernelisrelations} backward to say that $f$ has kernel containing $\ker\mathcal F$, so in fact we induce a map $\overline f\colon X\to A$.
	
	So in total, we get a $G$-module homomorphism $\psi\colon\mathcal T\to\op{Hom}_{\ZZ[G]}(X,A)$ by
	\[\psi\colon(\{\alpha_i\},\{\beta_{ij}\}_{i>j})\mapsto\overline f,\]
	where $\overline f$ is defined on the basis elements above. Further, $\psi$ is the inverse of $\varphi$ essentially because the $\{\kappa_i\}_i\cup\{\lambda_{ij}\}_{i>j}$ form a basis of $\ZZ[G]^m\times\ZZ[G]^{\binom m2}$. This completes the proof.
\end{proof}
And now because it is so easy, we might as well prove \autoref{thm:yesitisacocycle}.
\thmyesitisacocycle*
\begin{proof}
	The main point is that we have a computation of $\delta(\overline c)$ from \autoref{cor:deltaccomputation}, which we merely need to track through. In particular, fix a $\{\sigma_i\}_{i=1}^m$-tuple $(\{\alpha_i\}_i,\{\beta_{ij}\}_{i>j})$, and let $f\in H^0(G,\op{Hom}_\ZZ(X,A))$ be the corresponding morphism. As such, we may compute
	\[\delta(\overline c)\cup f\colon(g,h)\mapsto\delta(\overline c)(g,h)\otimes_\ZZ gh\cdot f=\delta(\overline c)(g,h)\otimes_\ZZ f.\]
	To pass through evaluation, we set $g\coloneqq\prod_i\sigma_i^{a_i}$ and $h\coloneqq\prod_i\sigma_i^{b_i}$, from which we get
	\begin{align*}
		f(\delta(\overline c)(g,h)) &= f\left((g_ih_iq_i)_i,\big(g_ih_j\sigma_i^{(a_i)}\sigma_j^{(b_j)}\big)_{i>j}\right) \\
		&= \sum_{i=1}^mg_ih_i\floor{\frac{a_i+b_i}{n_i}}\cdot\alpha_i+\sum_{\substack{i,j=1\\i>j}}^mg_ih_j\sigma_i^{(a_i)}\sigma_j^{(b_j)}\cdot\beta_{ij} \\
		&= \sum_{\substack{i,j=1\\i>j}}^m\Bigg(\prod_{p<i}\sigma_p^{a_p}\Bigg)\Bigg(\prod_{q<j}\sigma_q^{b_q}\Bigg)\sigma_i^{(a_i)}\sigma_j^{(b_j)}\beta_{ij}+\sum_{i=1}^mg_ih_i\alpha_i^{\floor{\frac{a_i+b_i}{n_i}}}.
	\end{align*}
	Doing a little more rearrangement and writing this multiplicatively exactly recovers the cocycle of \autoref{thm:getcocycle}. This completes the proof.
\end{proof}

% \subsection{Some Loose Ends}
% We continue in the set-up and notation of the previous subsection.
Though we have proven everything we set out to do in \autoref{sec:overview}, there is more to discuss with our alternate description of tuples. As a taste, we prove the following extension of \autoref{prop:alternativetuple}.
\begin{proposition} \label{prop:alternativetupleclass}
	Fix everything as in the set-up, and let $A$ be a $G$-module. Then the isomorphism of \autoref{prop:alternativetuple} descends to an isomorphism between equivalence classes of $\{\sigma_i\}_{i=1}^m$-tuples are canonically isomorphic to $\widehat H^0(G,\op{Hom}_\ZZ(X,A))$.
\end{proposition}
\begin{proof}
	Recall that the short exact sequence
	\[0\to X\stackrel{\mathcal F}\to\ZZ[G]^m\to\coker\mathcal F\to 0\]
	of $G$-modules splits as $\ZZ$-modules by \autoref{lem:sessplits}, so we have a short exact sequence
	\[0\to\op{Hom}_\ZZ(\coker\mathcal F,A)\to\op{Hom}_\ZZ(\ZZ[G]^m,A)\stackrel{-\circ\mathcal F}\to\op{Hom}_\ZZ(X,A)\to 0.\]
	Now, the key trick will be to compare regular group cohomology with Tate cohomology. To begin, we note that our cohomology theories give the following commutative diagram with exact rows.
	% https://q.uiver.app/?q=WzAsNixbMCwwLCJIXjAoRyxcXG9we0hvbX1fXFxaWihcXFpaW0ddXm0sQSkpIl0sWzEsMCwiSF4wKEcsXFxvcHtIb219X1xcWlooWCxBKSkiXSxbMSwxLCJcXHdpZGVoYXQgSF4wKEcsXFxvcHtIb219X1xcWlooWCxBKSkiXSxbMiwwLCJIXjEoRyxcXG9we0hvbX1fXFxaWihcXGNva2VyXFxtYXRoY2FsIEYsQSkpIl0sWzIsMSwiXFx3aWRlaGF0IEheMShHLFxcb3B7SG9tfV9cXFpaKFxcY29rZXJcXG1hdGhjYWwgRixBKSkiXSxbMCwxLCIwIl0sWzEsMiwiIiwwLHsic3R5bGUiOnsiaGVhZCI6eyJuYW1lIjoiZXBpIn19fV0sWzMsNCwiIiwwLHsibGV2ZWwiOjIsInN0eWxlIjp7ImhlYWQiOnsibmFtZSI6Im5vbmUifX19XSxbMSwzXSxbMiw0XSxbNSwyXSxbMCwxLCItXFxjaXJjXFxtYXRoY2FsIEYiXV0=&macro_url=https%3A%2F%2Fraw.githubusercontent.com%2FdFoiler%2Fnotes%2Fmaster%2Fnir.tex
	\begin{equation}
		\begin{tikzcd}
			{H^0(G,\op{Hom}_\ZZ(\ZZ[G]^m,A))} & {H^0(G,\op{Hom}_\ZZ(X,A))} & {H^1(G,\op{Hom}_\ZZ(\coker\mathcal F,A))} \\
			0 & {\widehat H^0(G,\op{Hom}_\ZZ(X,A))} & {\widehat H^1(G,\op{Hom}_\ZZ(\coker\mathcal F,A))}
			\arrow[two heads, from=1-2, to=2-2]
			\arrow[Rightarrow, no head, from=1-3, to=2-3]
			\arrow[from=1-2, to=1-3]
			\arrow[from=2-2, to=2-3]
			\arrow[from=2-1, to=2-2]
			\arrow["{-\circ\mathcal F}", from=1-1, to=1-2]
		\end{tikzcd} \label{eq:crazycohomology}
	\end{equation}
	Here, the middle vertical map is reduction modulo $\im N_G$. The rows are exact from the long exact sequences, and the square commutes by construction of Tate cohomology. Now, the point is that the diagram induces the isomorphism
	\begin{equation}
		\frac{H^0(G,\op{Hom}_\ZZ(X,A))}{\im(-\circ\mathcal F)}\simeq\widehat H^0(G,\op{Hom}_\ZZ(X,A)), \label{eq:crazyinducediso}
	\end{equation}
	which simply sends $[f]\mapsto[f]$.

	Thus, the main content here will be to track through the image of $-\circ\mathcal F$ in \autoref{eq:crazycohomology}. Let $\mathcal T$ denote the set of $\{\sigma_i\}_{i=1}^m$-triples of $A$, and let $\mathcal T_0$ denote the set (in fact, equivalence class) of triples corresponding to $[0]\in H^2(G,A)$. Letting $\varphi\colon H^0(G,\op{Hom}_\ZZ(X,A))\to\mathcal T$ be defined by
	\[\varphi\colon f\mapsto\Big(\big(f(\kappa_i)\big)_i,\big(f(\lambda_{ij})\big)_{i>j}\Big)\]
	be the isomorphism of \autoref{prop:alternativetuple}, we claim that the image of $-\circ\mathcal F$ in $H^0(G,\op{Hom}_\ZZ(X,A))$ corresponds under $\varphi$ to exactly $\mathcal T_0$.

	Indeed, we take a $G$-module homomorphism $f\colon\ZZ[G]^m\to A$ to the $G$-module homomorphism $(f\circ\mathcal F)\colon X\to A$. Then we compute
	\begin{align*}
		(f\circ\mathcal F)(\kappa_i) &= f(N_i\varepsilon_i) \\
		&= N_if(\varepsilon_i) \\
		(f\circ\mathcal F)(\lambda_{ij}) &= f(T_i\varepsilon_j-T_j\varepsilon_i) \\
		&= T_if(\varepsilon_j)-T_jf(\varepsilon_i)
	\end{align*}
	for all relevant indices $i$ and $j$. Thus,
	\[\varphi(f\circ\mathcal F)=\left(\big(N_if(\varepsilon_i)\big)_{i},\big(T_if(\varepsilon_j)-T_jf(\varepsilon_i)\big)_{i>j}\right),\]
	which we can see lives in $\mathcal T_0$ by definition of our equivalence relation (upon using multiplicative notation). In fact, as $f$ varies, we see that the values of $f(\varepsilon_i)$ may vary over all $A$, so the image of $f\mapsto\varphi(f\circ\mathcal F)$ is exactly all of $\mathcal T_0$. Thus, $\varphi$ induces an isomorphism
	\[\overline\varphi\colon\frac{H^0(G,\op{Hom}_\ZZ(X,A))}{\im(-\circ\mathcal F)}\simeq\frac{\mathcal T}{\mathcal T_0}.\]
	Composing this with the ``identity'' map \autoref{eq:crazyinducediso} finishes the proof.
	% The main point is that the cup product with $\delta(\overline c)$ will induce an isomorphism
	% \[\widehat H^0(G,\op{Hom}_\ZZ(X,L^\times))\to H^2(G,L^\times).\]
	% Indeed, note that $\mathcal F\colon X\to\ZZ[G]^m$ is an embedding of $\ZZ$-modules, so $X$ is a free abelian group because $\ZZ[G]^m$ is. It follows that $X$ is the character group of an algebraic torus $\mathcal T=\op{Hom}_\ZZ(X,\mathbb G_m)$, so we write $X=X^*(\mathcal T)$. Now, the main point is that we can realize the cup-product map of \autoref{thm:yesitisacocycle} in Tate cohomology as
	% \[\cup\colon\widehat H^0(G,\mathcal T(L))\times\widehat H^2(G,X^*(\mathcal T))\to H^2(G,L^\times).\]
	% However, by Tate--Nakayama duality, we know that this pairing is non-degenerate. In particular, because $\delta(\overline c)$ generates $H^2(G,X^*(\mathcal T))=H^2(G,X)$ by \autoref{cor:computeh2x}, we know that the map
	% \[\delta(\overline c)\cup-\colon\widehat H^0(G,\op{Hom}_\ZZ(X,L^\times))\to H^2(G,L^\times)\]
	% must be injective. On the other hand, by taking a cohomology class $[c]\in H^2(G,L^\times)$, lifting to a representative $\{\sigma_i\}_{i=1}^m$-tuple (as in \autoref{thm:classisomorphism}) gives an input to the above cup product map hitting $[c]$. Thus, the above cup product map we already know to be surjective, so it is an isomorphism.
	% We now attack the statement directly. Let $\mathcal T$ denote the set of $\{\sigma_i\}_{i=1}^m$-tuples and $\mathcal T_0$ denote the set (in fact, equivalence class) of tuples corresponding to the trivial cohomology class in $H^2(G,L^\times)$. Then we draw the following diagram, which we claim commutes and is made of isomorphisms.
	% % https://q.uiver.app/?q=WzAsMyxbMCwwLCJcXG1hdGhjYWwgVC9cXG1hdGhjYWwgVF8wIl0sWzAsMSwiXFx3aWRlaGF0IEheMChHLFxcb3B7SG9tfV9cXFpaKFgsTF5cXHRpbWVzKSkiXSxbMSwxLCJcXHdpZGVoYXQgSF4yKEcsTF5cXHRpbWVzKSJdLFswLDJdLFswLDFdLFsxLDJdXQ==&macro_url=https%3A%2F%2Fraw.githubusercontent.com%2FdFoiler%2Fnotes%2Fmaster%2Fnir.tex
	% \[\begin{tikzcd}
	% 	{\mathcal T/\mathcal T_0} \\
	% 	{\widehat H^0(G,\op{Hom}_\ZZ(X,L^\times))} & {\widehat H^2(G,L^\times)}
	% 	\arrow[from=1-1, to=2-2]
	% 	\arrow[from=1-1, to=2-1, dashed]
	% 	\arrow[from=2-1, to=2-2]
	% \end{tikzcd}\]
	% Namely, the map $\mathcal T/\mathcal T_0\to\widehat H^2(G,L^\times)$ sends an equivalence class of tuples to its cocycle, and it is an isomorphism by \autoref{thm:classisomorphism}. Further, the map $\widehat H^0(G,\op{Hom}_\ZZ(X,L^\times))\to\widehat H^2(G,L^\times)$ is the cup product with $\delta(\overline c)$, and it is an isomorphism as described above.
	% Lastly, $\mathcal T/\mathcal T_0\to\widehat H^0(G,\op{Hom}_\ZZ(X,L^\times))$ is descended from the morphism of \autoref{prop:alternativetuple}, so the diagram does indeed commute by \autoref{thm:yesitisacocycle}. In particular, this vertical map is well-defined and in fact an isomorphism by the commutativity of the diagram. This completes the proof.
\end{proof}
\begin{remark}
	This proof feels more motivated coming from the perspective that $X$ ``should'' be a $2$-encoding module (for example, $\coker\mathcal F$ ``should'' be a $1$-encoding module, allowing us to use \autoref{prop:encodingses}), so actually the equivalence relation on the tuples from \autoref{defi:tupleequiv} can be seen as falling out of the quotient
	\[H^0(G,\op{Hom}_\ZZ(X,-))\Rightarrow\widehat H^0(G,\op{Hom}_\ZZ(X,-)).\]
	Indeed, the equivalence relations had better match up anyway.
\end{remark}
% Another loose end we have to tie up is that we showed $H^2(G,X)$ is cyclic generated by $[\delta(\overline c)]$, but we do not actually know the order. Tracking through Tate--Nakayama duality in the proof will tell us that the order is $\#G$, but this requires $G$ to be a Galois group. Thankfully, we are able to work this out for general $G$ using the rest of the theory that we have built.
% \begin{lemma} \label{lem:zivanish}
% 	Fix everything as in the set-up. If $z\in\ZZ[G]$ has $z\varepsilon_i=0$ in $\coker\mathcal F$, then $z\in\im N_i$.
% \end{lemma}
% \begin{proof}
% 	The point is to pass through $\rho$ of \autoref{lem:sessplits}. By possibly rearranging the $\sigma_i$, we may assume that $i=m$. Then, for any $g\coloneqq\prod_{i=1}^m\sigma_i^{a_i}$, we see
% 	\[\rho(g\varepsilon_m)=g_m\big(\sigma_m^{a_m}-N_m1_{a_m=n_m-1}\big)\varepsilon_m=g\varepsilon_m-g_m1_{a_m=n_m-1}\cdot N_m\varepsilon_m.\]
% 	Namely, $\rho(g\varepsilon_m)-g\varepsilon_m=N_mz_g\varepsilon_m$ for some $z_g\in\ZZ[G]$.
	
% 	Extending this linearly, we see that
% 	\[\rho(z\varepsilon_m)-z\varepsilon_m=w\cdot N_m\varepsilon_m\]
% 	for some $w\in\ZZ[G]$, but $z\varepsilon_m=0$ in $\coker\mathcal F$ makes this say $z\varepsilon_m=-w\cdot N_m\varepsilon_m$. Because this is now an equality in $\ZZ[G]^m$, we conclude $z=-w\cdot N_m\in N_m$.
% \end{proof}
% \begin{lemma} \label{lem:computeordc}
% 	Fix everything as in the set-up. Then $z\cdot\overline c=0$ in $Z^1(G,\coker\mathcal F)$ if and only if $z\in\im N_G$, where $N_G=\sum_{g\in G}g$.
% \end{lemma}
% \begin{proof}
% 	In one direction, if $z=N_Gw$, then
% 	\[z\varepsilon_i=N_Gw\varepsilon_i\equiv0\pmod{\im\mathcal F}\]
% 	for each index $i$, so it follows that $(z\cdot\overline c)(\sigma_i)=z\varepsilon_i=0$ for each $\sigma_i$. Thus, using \autoref{lem:compresscocycle}, we conclude that $z\cdot\overline c=0$.

% 	The other direction is more difficult. Suppose that $z\cdot\overline c=0$. In particular, it follows that $(z\cdot\overline c)(\sigma_i)=z\varepsilon_i$ must equal $0$ for each index $i$. In particular, by \autoref{lem:zivanish}, we conclude that $z\in\im N_i$ for each index $i$, which by \autoref{lem:separatenijs} tells us that
% 	\[z\in\im N_1\cdots N_m=\im N_G.\]
% 	This completes the proof.
% \end{proof}
% \begin{prop} \label{prop:finishh1cokerFcomputation}
% 	Fix everything as in the set-up. Then $H^1(G,\coker\mathcal F)$ is cyclic of order $\#G$, generated by $[\overline c]$.
% \end{prop}
% \begin{proof}
% 	To help us use \autoref{prop:computeh1cokerF}, let $\varepsilon\colon\ZZ[G]\to\ZZ$ denote the augmentation map.
	
% 	Note that we already know $H^1(G,\coker\mathcal F)$ is cyclic generated by $[\overline c]$ by \autoref{prop:computeh1cokerF}, so it only remains to compute the order of $[\overline c]$. On one hand, we have an upper bound on the order of $[\overline c]$ because $H^1(G,\coker\mathcal F)$ is $\#G$-torsion, but we can also see this directly: note that \autoref{lem:computeordc} tells us that
% 	\[[0]=[N_G\cdot\overline c].\]
% 	However, $[N_G\cdot\overline c]=[\varepsilon(N_G)\cdot\overline c]=[\#G\cdot\overline c]$ by \autoref{prop:computeh1cokerF}, so we do see that $\#G\cdot\overline c=0$.
	
% 	It remains to show that $[\overline c]$ has order at least $\#G$. As such, it suffices to show that if $n$ has $[n\cdot\overline c]=[0]$, then $\#G\mid n$. In particular, $n\cdot\overline c$ is a coboundary, so letting $d\colon C^0(G,\coker\mathcal F)\to B^1(G,\coker\mathcal F)$ denote the corresponding differential, we have
% 	\[n\cdot\overline c=d\left(\sum_{i=1}^mb_i\varepsilon_i\right)=\sum_{i=1}^mb_i(d\varepsilon_i)\]
% 	for some $\{b_i\}_{i=1}^m\subseteq\ZZ[G]$. Now, $(d\varepsilon_i)(\sigma_j)=T_j\varepsilon_i=T_i\varepsilon_j$ for any pair of indices $(i,j)$, so by the uniqueness of the extension in \autoref{lem:compresscocycle}, we conclude $d\varepsilon_i=T_i\overline c$. Thus, we set
% 	\[z\coloneqq n-\sum_{i=1}^mb_iT_i\]
% 	so that $\varepsilon(z)=n$ and $z\cdot\overline c=0$.

% 	To finish, we note \autoref{lem:computeordc} now tells us that $z\in\im N_G$, so letting $z=N_Gw$, we see that
% 	\[n=\varepsilon(z)=\varepsilon(N_G)\varepsilon(w)=\#G\cdot\varepsilon(w),\]
% 	so $\#G\mid n$. This completes the proof.
% \end{proof}
% \begin{cor}
% 	Fix everything as in the set-up. Then $H^1(G,\coker\mathcal F)$ is cyclic of order $\#G$, generated by $[\delta(\overline c)]$, where $\delta$ is induced by
% 	\[0\to X\stackrel{\mathcal F}\to\ZZ[G]^m\to\coker\mathcal F\to0.\]
% \end{cor}
% \begin{proof}
% 	As in the proof of \autoref{cor:computeh2x}, we note $\delta\colon H^1(G,\coker\mathcal F)\to H^2(G,X)$ is an isomorphism, so this follows from \autoref{prop:finishh1cokerFcomputation}.
% \end{proof}

% \subsection{Some Cup Product Computations}
% We take a brief intermission to establish a little theory on cup products. In this section, we let $G$ denote a generic finite group (not necessarily assumed to be abelian) and $A$ a $G$-module.
% \begin{lemma}[\cite{bonn-lectures}, Proposition~I.5.3] \label{lem:cupproductmorphism}
% 	Let $G$ be a finite group. Given any $G$-modules $A,B,C$ with a $G$-module homomorphism $\varphi\colon B\to C$, the following diagram commutes for any $p,q\in\ZZ$ and $[a]\in\widehat H^p(G,A)$.
% 	% https://q.uiver.app/?q=WzAsNCxbMCwwLCJcXHdpZGVoYXQgSF5wKEcsQikiXSxbMSwwLCJcXHdpZGVoYXQgSF5wKEcsQykiXSxbMCwxLCJcXHdpZGVoYXQgSF57cCtxfShHLEFcXG90aW1lc19cXFpaIEIpIl0sWzEsMSwiXFx3aWRlaGF0IEhee3ArcX0oRyxBXFxvdGltZXNfXFxaWiBDKSJdLFswLDEsIlxcdmFycGhpIl0sWzIsMywiXFx2YXJwaGkiXSxbMCwyLCJhXFxjdXAgLSIsMl0sWzEsMywiYVxcY3VwIC0iXV0=&macro_url=https%3A%2F%2Fraw.githubusercontent.com%2FdFoiler%2Fnotes%2Fmaster%2Fnir.tex
% 	\[\begin{tikzcd}
% 		{\widehat H^q(G,B)} & {\widehat H^q(G,C)} \\
% 		{\widehat H^{p+q}(G,A\otimes_\ZZ B)} & {\widehat H^{p+q}(G,A\otimes_\ZZ C)}
% 		\arrow["\varphi", from=1-1, to=1-2]
% 		\arrow["\id_A\otimes\varphi", from=2-1, to=2-2]
% 		\arrow["{[a]\cup -}"', from=1-1, to=2-1]
% 		\arrow["{[a]\cup -}", from=1-2, to=2-2]
% 	\end{tikzcd}\]
% \end{lemma}
% \begin{proof}
% 	When $p,q\ge0$, we can argue directly. Indeed, we claim that the diagram commutes on the level of homogeneous cochains: let $[a]\in\widehat H^p(G,A)$ and $[b]\in\widehat H^q(G,B)$ be cohomology classes represented by the homogeneous cochains $a\in[a]$ and $b\in[b]$. Tracking along the top of the diagram, we see
% 	\begin{align*}
% 		(a\cup\varphi(b))(g_0,\ldots,g_{p+q}) &= a(g_0,\ldots,g_p)\otimes\varphi(b)(g_p,\ldots,g_{p+1}) \\
% 		&= a(g_0,\ldots,g_p)\otimes\varphi(b(g_p,\ldots,g_{p+1})).
% 	\end{align*}
% 	Tracking along the bottom of the diagram, we see
% 	\begin{align*}
% 		(\id_A\otimes\varphi)(a\cup b)(g_0,\ldots,g_{p+q}) &= (\id_A\otimes\varphi)(a(g_0,\ldots,g_p)\otimes b(g_p,\ldots,g_{p+q})) \\
% 		&= a(g_0,\ldots,g_p)\otimes\varphi(b(g_p,\ldots,g_{p+q})),
% 	\end{align*}
% 	which is equal. This completes the proof in the case of $p,q\ge0$.

% 	We will only need the case of $p,q\ge0$ in the application, but we will go ahead and do the general case now; we dimension-shift $p$ and $q$ downwards. For example, to shift $p$ downwards, we note that the (split) short exact sequence
% 	\begin{equation}
% 		0\to A\otimes_\ZZ I_G\to A\otimes_\ZZ\ZZ[G]\to A\to0 \label{eq:standardashift}
% 	\end{equation}
% 	induces the isomorphism $\delta\colon\widehat H^{p-1}(G,A)\to\widehat H^p(G,I_G\otimes_\ZZ A)$. As such, given $a\in\widehat H^{p-1}(G,A)$, the inductive hypothesis reassures that the following diagram commutes.
% 	% https://q.uiver.app/?q=WzAsNCxbMCwwLCJcXHdpZGVoYXQgSF5wKEcsQikiXSxbMSwwLCJcXHdpZGVoYXQgSF5wKEcsQykiXSxbMCwxLCJcXHdpZGVoYXQgSF57cCtxfShHLElfR1xcb3RpbWVzX1xcWlogQVxcb3RpbWVzX1xcWlogQikiXSxbMSwxLCJcXHdpZGVoYXQgSF57cCtxfShHLElfR1xcb3RpbWVzX1xcWlogQVxcb3RpbWVzX1xcWlogQykiXSxbMCwxLCJcXHZhcnBoaSJdLFsyLDMsIlxcdmFycGhpIl0sWzAsMiwiXFxkZWx0YShhKVxcY3VwIC0iLDJdLFsxLDMsIlxcZGVsdGEoYSlcXGN1cCAtIl1d&macro_url=https%3A%2F%2Fraw.githubusercontent.com%2FdFoiler%2Fnotes%2Fmaster%2Fnir.tex
% 	\[\begin{tikzcd}
% 		{\widehat H^q(G,B)} & {\widehat H^q(G,C)} \\
% 		{\widehat H^{p+q}(G,I_G\otimes_\ZZ A\otimes_\ZZ B)} & {\widehat H^{p+q}(G,I_G\otimes_\ZZ A\otimes_\ZZ C)}
% 		\arrow["\varphi", from=1-1, to=1-2]
% 		\arrow["\varphi", from=2-1, to=2-2]
% 		\arrow["{\delta(a)\cup -}"', from=1-1, to=2-1]
% 		\arrow["{\delta(a)\cup -}", from=1-2, to=2-2]
% 	\end{tikzcd}\]
% 	In other words, all $b\in\widehat H^q(G,B)$ have $\varphi(\delta(a)\cup b)=\delta(a)\cup\varphi(b)$.
	
% 	Now, because \autoref{eq:standardashift} is split, we can hit it with $-\otimes_\ZZ B$ and $-\otimes_\ZZ C$ to induce the following commutative diagram with exact rows.
% 	% https://q.uiver.app/?q=WzAsMTAsWzAsMCwiMCJdLFsxLDAsIkFcXG90aW1lc19cXFpaIElfR1xcb3RpbWVzX1xcWlogQiJdLFsyLDAsIkFcXG90aW1lc19cXFpaXFxaWltHXVxcb3RpbWVzX1xcWlogQiJdLFszLDAsIkFcXG90aW1lc19cXFpaIEIiXSxbNCwwLCIwIl0sWzEsMSwiQVxcb3RpbWVzX1xcWlogSV9HXFxvdGltZXNfXFxaWiBDIl0sWzIsMSwiQVxcb3RpbWVzX1xcWlpcXFpaW0ddXFxvdGltZXNfXFxaWiBDIl0sWzMsMSwiQVxcb3RpbWVzX1xcWlogQyJdLFswLDEsIjAiXSxbNCwxLCIwIl0sWzAsMV0sWzEsMl0sWzIsM10sWzMsNF0sWzgsNV0sWzUsNl0sWzYsN10sWzcsOV0sWzEsNSwiXFx2YXJwaGkiLDJdLFsyLDYsIlxcdmFycGhpIiwyXSxbMyw3LCJcXHZhcnBoaSIsMl1d&macro_url=https%3A%2F%2Fraw.githubusercontent.com%2FdFoiler%2Fnotes%2Fmaster%2Fnir.tex
% 	\[\begin{tikzcd}
% 		0 & {A\otimes_\ZZ I_G\otimes_\ZZ B} & {A\otimes_\ZZ\ZZ[G]\otimes_\ZZ B} & {A\otimes_\ZZ B} & 0 \\
% 		0 & {A\otimes_\ZZ I_G\otimes_\ZZ C} & {A\otimes_\ZZ\ZZ[G]\otimes_\ZZ C} & {A\otimes_\ZZ C} & 0
% 		\arrow[from=1-1, to=1-2]
% 		\arrow[from=1-2, to=1-3]
% 		\arrow[from=1-3, to=1-4]
% 		\arrow[from=1-4, to=1-5]
% 		\arrow[from=2-1, to=2-2]
% 		\arrow[from=2-2, to=2-3]
% 		\arrow[from=2-3, to=2-4]
% 		\arrow[from=2-4, to=2-5]
% 		\arrow["\varphi"', from=1-2, to=2-2]
% 		\arrow["\varphi"', from=1-3, to=2-3]
% 		\arrow["\varphi"', from=1-4, to=2-4]
% 	\end{tikzcd}\]
% 	Letting $\delta_B\colon\widehat H^{p-1}(A\otimes_\ZZ B)\to\widehat H^p(I_G\otimes_\ZZ A\otimes_\ZZ B)$ and $\delta_C\colon\widehat H^{p-1}(A\otimes_\ZZ B)\to\widehat H^p(I_G\otimes_\ZZ A\otimes_\ZZ B)$ denote the corresponding isomorphisms (note that the middle terms are induced and hence acyclic), we note that the functoriality of boundary morphisms tells us that $\varphi\delta_B=\delta_C\varphi$. In total, it follows that $b\in\widehat H^q(G,B)$ will have
% 	\[\delta_C(\varphi(a\cup b))=\varphi(\delta_B(a\cup b))=\varphi(\delta(a)\cup b)\stackrel*=\delta(a)\cup\varphi(b)=\delta_C(a\cup\varphi(b)),\]
% 	where we have used the inductive hypothesis at $\stackrel*=$. Because $\delta_C$ is an isomorphism, this completes the step to shift $p$ downwards to $p-1$.

% 	Shifting $q$ downwards is similar. This time we start with the following commutative diagram whose rows are (split) short exact sequences.
% 	% https://q.uiver.app/?q=WzAsMTAsWzAsMCwiMCJdLFsxLDAsIklfR1xcb3RpbWVzX1xcWlogQiJdLFsyLDAsIlxcWlpbR11cXG90aW1lc19cXFpaIEIiXSxbMywwLCJCIl0sWzQsMCwiMCJdLFsxLDEsIklfR1xcb3RpbWVzX1xcWlogQyJdLFsyLDEsIlxcWlpbR11cXG90aW1lc19cXFpaIEMiXSxbMywxLCJDIl0sWzAsMSwiMCJdLFs0LDEsIjAiXSxbMCwxXSxbMSwyXSxbMiwzXSxbMyw0XSxbOCw1XSxbNSw2XSxbNiw3XSxbNyw5XSxbMSw1LCJcXHZhcnBoaSIsMl0sWzIsNiwiXFx2YXJwaGkiLDJdLFszLDcsIlxcdmFycGhpIiwyXV0=&macro_url=https%3A%2F%2Fraw.githubusercontent.com%2FdFoiler%2Fnotes%2Fmaster%2Fnir.tex
% 	\[\begin{tikzcd}
% 		0 & {I_G\otimes_\ZZ B} & {\ZZ[G]\otimes_\ZZ B} & B & 0 \\
% 		0 & {I_G\otimes_\ZZ C} & {\ZZ[G]\otimes_\ZZ C} & C & 0
% 		\arrow[from=1-1, to=1-2]
% 		\arrow[from=1-2, to=1-3]
% 		\arrow[from=1-3, to=1-4]
% 		\arrow[from=1-4, to=1-5]
% 		\arrow[from=2-1, to=2-2]
% 		\arrow[from=2-2, to=2-3]
% 		\arrow[from=2-3, to=2-4]
% 		\arrow[from=2-4, to=2-5]
% 		\arrow["\varphi"', from=1-2, to=2-2]
% 		\arrow["\varphi"', from=1-3, to=2-3]
% 		\arrow["\varphi"', from=1-4, to=2-4]
% 	\end{tikzcd}\]
% 	In particular, we let $\delta_B'\colon\widehat H^{q-1}(G,B)\to\widehat H^q(G,I_G\otimes_\ZZ B)$ and $\delta_C'\colon\widehat H^{q-1}(G,B)\to\widehat H^q(G,I_G\otimes_\ZZ C)$ denote the induced isomorphisms, and again functoriality of the boundary morphisms tells us that $\varphi\delta_B=\delta_C\varphi$. Now, the inductive hypothesis tells us that the following diagram commutes for any $a\in\widehat H^p(G,A)$.
% 	% https://q.uiver.app/?q=WzAsNCxbMCwwLCJcXHdpZGVoYXQgSF5wKEcsSV9HXFxvdGltZXNfXFxaWiBCKSJdLFsxLDAsIlxcd2lkZWhhdCBIXnAoRyxJX0dcXG90aW1lc19cXFpaIEMpIl0sWzAsMSwiXFx3aWRlaGF0IEhee3ArcX0oRyxBXFxvdGltZXNfXFxaWiBJX0dcXG90aW1lc19cXFpaIEIpIl0sWzEsMSwiXFx3aWRlaGF0IEhee3ArcX0oRyxBXFxvdGltZXNfXFxaWiBJX0dcXG90aW1lc19cXFpaIEMpIl0sWzAsMSwiXFx2YXJwaGkiXSxbMiwzLCJcXHZhcnBoaSJdLFswLDIsImFcXGN1cCAtIiwyXSxbMSwzLCJhXFxjdXAgLSJdXQ==&macro_url=https%3A%2F%2Fraw.githubusercontent.com%2FdFoiler%2Fnotes%2Fmaster%2Fnir.tex
% 	\[\begin{tikzcd}
% 		{\widehat H^q(G,I_G\otimes_\ZZ B)} & {\widehat H^q(G,I_G\otimes_\ZZ C)} \\
% 		{\widehat H^{p+q}(G,A\otimes_\ZZ I_G\otimes_\ZZ B)} & {\widehat H^{p+q}(G,A\otimes_\ZZ I_G\otimes_\ZZ C)}
% 		\arrow["\varphi", from=1-1, to=1-2]
% 		\arrow["\varphi", from=2-1, to=2-2]
% 		\arrow["{a\cup -}"', from=1-1, to=2-1]
% 		\arrow["{a\cup -}", from=1-2, to=2-2]
% 	\end{tikzcd}\]
% 	Namely, any $b\in\widehat H^{p-1}(G,B)$ has
% 	\begin{align*}
% 		\delta_C'(a\cup\varphi(b)) &= (-1)^p\big(a\cup\delta_C'(\varphi(b))\big) \\
% 		&= (-1)^p\big(a\cup\varphi(\delta_B'(b))\big) \\
% 		&\stackrel*= (-1)^p\varphi(a\cup\delta_B'(b)) \\
% 		&= (-1)^p\cdot(-1)^p\varphi(\delta_B'(a\cup b)) \\
% 		&= \delta_C'(\varphi(a\cup b)),
% 	\end{align*}
% 	where we've applied the inductive hypothesis at $\stackrel*=$. Because $\delta_C'$ is an isomorphism, this completes shifting $q$ downwards to $q-1$.
% \end{proof}
% \begin{remark}
% 	An analogous argument shows that a $G$-module homomorphism $\psi\colon A\to B$ induces the following commutative diagram, for any $p,q\in\ZZ$ and $c\in\widehat H^q(G,C)$.
% 	% https://q.uiver.app/?q=WzAsNCxbMCwwLCJcXHdpZGVoYXQgSF5wKEcsQSkiXSxbMSwwLCJcXHdpZGVoYXQgSF5wKEcsQikiXSxbMCwxLCJcXHdpZGVoYXQgSF57cCtxfShHLEFcXG90aW1lc19cXFpaIEMpIl0sWzEsMSwiXFx3aWRlaGF0IEhee3ArcX0oRyxCXFxvdGltZXNfXFxaWiBDKSJdLFswLDEsIlxccHNpIl0sWzIsMywiXFxwc2kiXSxbMCwyLCItXFxjdXAgYyIsMl0sWzEsMywiLVxcY3VwIGMiLDJdXQ==&macro_url=https%3A%2F%2Fraw.githubusercontent.com%2FdFoiler%2Fnotes%2Fmaster%2Fnir.tex
% 	\[\begin{tikzcd}
% 		{\widehat H^p(G,A)} & {\widehat H^p(G,B)} \\
% 		{\widehat H^{p+q}(G,A\otimes_\ZZ C)} & {\widehat H^{p+q}(G,B\otimes_\ZZ C)}
% 		\arrow["\psi", from=1-1, to=1-2]
% 		\arrow["\psi", from=2-1, to=2-2]
% 		\arrow["{-\cup c}"', from=1-1, to=2-1]
% 		\arrow["{-\cup c}"', from=1-2, to=2-2]
% 	\end{tikzcd}\]
% \end{remark}
% And here are some corollaries, tying back into our theory.
% \begin{cor} \label{cor:xhasallcupisos}
% 	Fix notation as in \autoref{sec:overview}. Then, for any $G$-module $A$ and index $i\in\ZZ$, the cup-product map
% 	\[[\delta(\overline c)]\cup-\colon\widehat H^i(G,\op{Hom}_\ZZ(X,A))\to\widehat H^{i+2}(G,A)\]
% 	is an isomorphism.
% \end{cor}
% \begin{proof}
% 	Set $p=2$ and $q=0$ and $c$ to $[\delta(\overline c)]$ in \autoref{prop:dimshiftcupisos}; the hypothesis is satisfied by combining the cup-product map of \autoref{thm:yesitisacocycle} with \autoref{prop:alternativetupleclass}. (Namely, the cup-product map is sending an equivalence class of tuples to the corresponding cohomology class, which is an isomorphism by \autoref{thm:classisomorphism}.) Anyway, \autoref{prop:dimshiftcupisos} does indeed give the result.
% \end{proof}

% \section{Torus Reciprocity}
% In this section we will apply the theory we have built to the specific case where $G$ is a Galois group of an extension of local fields $L/K$.

% In particular, we keep the notation as in \autoref{sec:tuplestudy} while asserting that $G=\op{Gal}(L/K)$ for some finite abelian extension of local fields $L/K$. Now, we note that the embedding
% \[X\stackrel{\mathcal F}\into\ZZ[G]^m\]
% tells us that $X$ embeds into a free abelian group and hence must be a free abelian group. In particular, because $X$ has a $G$-action---which extends to a $\op{Gal}(K^{\op{sep}}/K)$-action by taking quotients---we are promised a $K$-torus $T$ such that
% \[X^*(T)=X.\]
% By dualizing again, we see that $T\simeq\op{Hom}_\ZZ(X,\mathbb G_m-)$. As such, we just set $T\coloneqq\op{Hom}_\ZZ(X,\mathbb G_m-)$, which gives the following result.
% \begin{cor} \label{cor:torustupledescription}
% 	Fix notation as above. Then $H^0(L/K,T(L))$ is in natural bijection with $\{\sigma_i\}_{i=1}^m$-tuples, and $\widehat H^0(L/K,T(L))$ is in natural bijection with equivalence classes of $\{\sigma_i\}_{i=1}^m$-tuples.
% \end{cor}
% \begin{proof}
% 	Because $T(L)=\op{Hom}_\ZZ(X^*(T),L^\times)$, we may plug into \autoref{prop:alternativetuple} and \autoref{prop:alternativetupleclass}.
% \end{proof}
% To continue our discussion, we recall the following generalization of Artin reciprocity.
% \begin{theorem} \label{thm:torusreciprocity}
% 	Let $K$ be a local field and $T$ a $K$-torus. Suppose that $L/K$ is an extension of fields such that the base change $T_L$ is a split torus. Then cup product with the fundamental class $u_{L/K}\in H^2(L/K,L^\times)$ induces an isomorphism
% 	\[-\cup u_{L/K}\colon\widehat H^n(L/K,X_*(T))\to\widehat H^{n+2}(L/K,T(L))\]
% 	for all integers $n$.
% \end{theorem}
% \begin{proof}
% 	Omitted.
% \end{proof}
% Importantly, our torus $T$ is split over $L$ because (say) all characters in $X^*(T)$ are defined over $L$.\todo{I'm not sure if this makes sense.}

% In light of \autoref{cor:torustupledescription}, we are particularly interested in the case of $n=-2$ in \autoref{thm:torusreciprocity}, giving an isomorphism
% \[-\cup u_{L/K}\colon\widehat H^{-2}(L/K,X_*(T))\to\widehat H^0(L/K,T(L)).\]
% For example, we may use \autoref{thm:yesitisacocycle} to create the following diagram.
% % https://q.uiver.app/?q=WzAsMyxbMCwwLCJcXHdpZGVoYXQgSF57LTJ9KEwvSyxYXyooVCkpIl0sWzEsMCwiXFx3aWRlaGF0IEheMChML0ssVChMKSkiXSxbMSwxLCJcXHdpZGVoYXQgSF4yKEwvSyxMXlxcdGltZXMpIl0sWzAsMSwiLVxcY3VwIHVfe0wvS30iXSxbMSwyLCJcXGRlbHRhKFxcb3ZlcmxpbmUgYylcXGN1cC0iXV0=&macro_url=https%3A%2F%2Fraw.githubusercontent.com%2FdFoiler%2Fnotes%2Fmaster%2Fnir.tex
% \[\begin{tikzcd}
% 	{\widehat H^{-2}(L/K,X_*(T))} & {\widehat H^0(L/K,T(L))} \\
% 	& {\widehat H^2(L/K,L^\times)}
% 	\arrow["{-\cup u_{L/K}}", from=1-1, to=1-2]
% 	\arrow["{[\delta(\overline c)]\cup-}", from=1-2, to=2-2]
% \end{tikzcd}\]
% In particular, the vertical arrow is now an isomorphism because we know that equivalence classes of tuples uniquely correspond to cocycles from \autoref{thm:classisomorphism}.

% To complete the above suggestive diagram, we have the following lemma.
% \begin{lemma} \label{lem:torusdiagram}
% 	Let $L/K$ be an extension of local fields, and let $T$ be a $K$-torus which splits over $L$. Then, given arbitrary classes $u\in H^2(L/K,L^\times)$ and $c\in H^2(L/K,X^*(T))$, the following diagram commutes.
% 	% https://q.uiver.app/?q=WzAsNCxbMCwwLCJcXHdpZGVoYXQgSF57LTJ9KEwvSyxYXyooVCkpIl0sWzEsMCwiXFx3aWRlaGF0IEheMChML0ssVChMKSkiXSxbMSwxLCJcXHdpZGVoYXQgSF4yKEwvSyxMXlxcdGltZXMpIl0sWzAsMSwiXFx3aWRlaGF0IEheMChML0ssXFxaWikiXSxbMCwxLCItXFxjdXAgdSJdLFsxLDIsIlxcZGVsdGEoXFxvdmVybGluZSBjKVxcY3VwLSJdLFszLDIsIi1cXGN1cCB1IiwyXSxbMCwzLCJcXGRlbHRhKFxcb3ZlcmxpbmUgYylcXGN1cC0iLDJdXQ==&macro_url=https%3A%2F%2Fraw.githubusercontent.com%2FdFoiler%2Fnotes%2Fmaster%2Fnir.tex
% 	\[\begin{tikzcd}
% 		{\widehat H^{-2}(L/K,X_*(T))} & {\widehat H^0(L/K,T(L))} \\
% 		{\widehat H^0(L/K,\ZZ)} & {\widehat H^2(L/K,L^\times)}
% 		\arrow["{-\cup u}", from=1-1, to=1-2]
% 		\arrow["{c\cup-}", from=1-2, to=2-2]
% 		\arrow["{-\cup u}"', from=2-1, to=2-2]
% 		\arrow["{c\cup-}"', from=1-1, to=2-1]
% 	\end{tikzcd}\]
% \end{lemma}
% \begin{proof}
% 	At a high level, this should follow from the associativity of the cup product, but some care is required because the cup product maps are augmented in various ways throughout the diagram. For peace of mind, we will actually track through these maps.
	
% 	We use the standard resolution by homogeneous cochains; let $G\coloneqq\op{Gal}(L/K)$. We can represent a class $[x]\in\widehat H^{-2}(L/K,X_*(T))$ by an element $x\in\op{Hom}_G(P_{-2},X_*(T))$, where $P_{-2}\coloneqq\op{Hom}_\ZZ(P_1,\ZZ)=\op{Hom}_\ZZ(\ZZ[G]^2,\ZZ)$. To compute our cup products, we also need to define the notation
% 	\[(g_1^*,\ldots,g_q^*)\in P_{-q}\coloneqq\op{Hom}_\ZZ(P_{q-1},\ZZ)=\op{Hom}_\ZZ(\ZZ[G]^q,\ZZ)\]
% 	which behaves as the indicator function for $(g_1,\ldots,g_q)$ on $G^q$. We are now ready to track through our diagram
% 	\begin{itemize}
% 		\item Along the bottom, we start with $c\cup x\in\widehat H^0(L/K,X^*(T)\otimes_\ZZ X_*(T))$, which is
% 		\[(c\cup x)(g)=\sum_{s_1,s_2\in G}c(g,s_1,s_2)\otimes x(s_2^*,s_1^*).\]
% 		Now, passing through $X^*(T)\otimes_\ZZ X_*(T)\to\op{Hom}(\mathbb G_m,\mathbb G_m)$ by $(f\otimes g)\mapsto(f\circ g)$, we get the map
% 		\[(c\cup x)(g)=\prod_{s_1,s_2\in G}c(g,s_1,s_2)\circ x(s_2^*,s_1^*).\]
% 		In particular, the right-hand side is an algebraic map $L^\times\to L^\times$, which we know must take the form $z\mapsto z^{r_g}$ for some $r_g\in\ZZ$. The mapping $g\mapsto r_g$ is the $0$-cocycle in $\widehat H^0(L/K,\ZZ)$.

% 		Next we must compute $(c\cup x)\cup u\in\widehat H^2(L/K,\ZZ\otimes_\ZZ L^\times)$, which is
% 		\[((c\cup x)\cup u)(g_0,g_1,g_2)=r_{g_0}\otimes u(g_0,g_1,g_2).\]
% 		Passing through $\ZZ\otimes_\ZZ L^\times\to L^\times$ by $r\otimes z\mapsto z^r$, we see that we get
% 		\begin{align}
% 			((c\cup x)\cup u)(g_0,g_1,g_2) &= u(g_0,g_1,g_2)^{r_{g_0}} \notag \\
% 			&= \prod_{s_1,s_2\in G}\big(c(g_0,s_1,s_2)\circ x(s_2^*,s_1^*)\big)\big(u(g_0,g_1,g_2)\big), \label{eq:cupsleft}
% 		\end{align}
% 		where in the last equality we applied the definition of $r_{g_0}$.

% 		\item We now track along the top. Starting with $x\cup u\in\widehat H^0(L/K,X_*(T)\otimes_\ZZ L^\times)$, we have
% 		\[(x\cup u)(g)=\sum_{s_1,s_2\in G}x(s_1^*,s_2^*)\otimes u(s_2,s_1,g).\]
% 		Passing through $X_*(T)\otimes_\ZZ L^\times\to T(L)$ by $(f\otimes z)\mapsto f(z)$, we get
% 		\[(x\cup u)(g)=\prod_{s_1,s_2\in G}x(s_1^*,s_2^*)\big(u(s_2,s_1,g)\big).\]
% 		Continuing, we compute $(c\cup(x\cup u))\in\widehat H^2(L/K,X*(T)\otimes_\ZZ T(L))$ as
% 		\[(c\cup(x\cup u))(g_0,g_1,g_2)=c(g_0,g_1,g_2)\otimes\prod_{s_1,s_2\in G}x(s_1^*,s_2^*)\big(u(s_2,s_1,g_2)\big).\]
% 		And to finish, we pass through $X^*(T)\otimes_\ZZ T(L)\to L^\times$ by $(f\otimes z)\mapsto f(z)$, which gives
% 		\begin{equation}
% 			(c\cup(x\cup u))(g_0,g_1,g_2)=\prod_{s_1,s_2\in G}\big(c(g_0,g_1,g_2)\circ x(s_1^*,s_2^*)\big)\big(u(s_2,s_1,g_2)\big). \label{eq:cupsright}
% 		\end{equation}
% 	\end{itemize}
% 	At this point, it might look like we're in trouble because \autoref{eq:cupsleft} and \autoref{eq:cupsright} look different from each other. However, this is just an outcome of how the cup product is defined. Indeed, we know abstractly that we must have $(c\cup x)\cup u=c\cup(x\cup u)$ in $\widehat H^2(L/K,X^*(T)\otimes_\ZZ X_*(T)\otimes_\ZZ L^\times)$, but these are
% 	\begin{align*}
% 		((c\cup x)\cup u)(g_0,g_1,g_2) &= (c\cup x)(g_0)\otimes u(g_0,g_1,g_2) \\
% 		&= \sum_{s_1,s_2\in G}c(g_0,s_1,s_2)\otimes x(s_2^*,s_1^*)\otimes u(g_0,g_1,g_2),
% 	\end{align*}
% 	and
% 	\begin{align*}
% 		(c\cup(x\cup u))(g_0,g_1,g_2) &= c(g_0,g_1,g_2)\otimes(x\cup u)(g_0) \\
% 		&= \sum_{s_1,s_2\in G}c(g_0,g_1,g_2)\otimes x(s_1^*,s_2^*)\otimes u(s_2,s_1,g_0).
% 	\end{align*}
% 	The above two cocycles need to be equal up to a coboundary in $B^2(G,X^*(T)\otimes_\ZZ X_*(T)\otimes_\ZZ L^\times)$, so we have that the map sending $(g_0,g_1,g_2)$ to
% 	\[\sum_{s_1,s_2\in G}c(g_0,s_1,s_2)\otimes x(s_2^*,s_1^*)\otimes u(g_0,g_1,g_2)-\sum_{s_1,s_2\in G}c(g_0,g_1,g_2)\otimes x(s_1^*,s_2^*)\otimes u(s_2,s_1,g_0)\]
% 	lives in $B^2(G,X^*(T)\otimes_\ZZ X_*(T)\otimes_\ZZ L^\times)$. Passing this all through the evaluation map $X^*(T)\otimes_\ZZ X_*(T)\otimes_\ZZ L^\times\to L^\times$ by $(f\otimes g\otimes z)\mapsto(f\circ g)(z)$, we see that the map sending $(g_0,g_1,g_2)$ to
% 	\[\prod_{s_1,s_2\in G}\big(c(g_0,s_1,s_2)\circ x(s_2^*,s_1^*)\big)\big(u(g_0,g_1,g_2)\big)\bigg/\prod_{s_1,s_2\in G}\big(c(g_0,g_1,g_2)\circ x(s_1^*,s_2^*)\big)\big(u(s_2,s_1,g_2)\big)\]
% 	is a coboundary in $B^2(G,L^\times)$. (Namely, homomorphisms $M\to M'$ induce homomorphisms $B^\bullet(G,M)\to B^\bullet(G,M)$.) Thus, \autoref{eq:cupsleft} and \autoref{eq:cupsright} are indeed in the same cocycle class.
% \end{proof}
% The point of \autoref{lem:torusdiagram} is the following result.
% \begin{theorem} \label{thm:goodcuppairing}
% 	Let $L/K$ be an extension of local fields, and let $T$ be a $K$-torus which splits over $L$. Further, suppose there is some $c\in H^2(L/K,X^*(T))$ such that
% 	\[c\cup-\colon\widehat H^0(L/K,T(L))\to\widehat H^2(L/K,L^\times)\]
% 	is an isomorphism. Now, if $u,u'\in H^2(L/K,L^\times)$ has
% 	\[x\cup u=x\cup u'\in\widehat H^0(L/K,T(L))\]
% 	for all $x\in\widehat H^{-2}(L/K,X_*(T))$, then $u=u'$.
% \end{theorem}
% \begin{proof}
% 	In the language of \autoref{lem:torusdiagram}, we are being told that $v=u$ and $v=u'$ are inducing the same top row of the following commutative diagram.
% 	\begin{equation}
% 		\begin{tikzcd}
% 			{\widehat H^{-2}(L/K,X_*(T))} & {\widehat H^0(L/K,T(L))} \\
% 			{\widehat H^0(L/K,\ZZ)} & {\widehat H^2(L/K,L^\times)}
% 			\arrow["{-\cup v}", from=1-1, to=1-2]
% 			\arrow["{c\cup-}", from=1-2, to=2-2]
% 			\arrow["{-\cup v}"', from=2-1, to=2-2]
% 			\arrow["{c\cup-}"', from=1-1, to=2-1]
% 		\end{tikzcd} \label{eq:torusdiagram}
% 	\end{equation}
% 	Quickly, note that $v=u_{L/K}$ makes the top row an isomorphism by \autoref{thm:torusreciprocity} as well as the bottom row an isomorphism (e.g., the generator $[1]\in\widehat H^0(L/K,\ZZ)$ maps to the generator $[1]\cup u_{L/K}=u^1_{L/K}=u_{L/K}$). Additionally, the right arrow is an isomorphism by hypothesis on $c$. Thus, the commutativity of the diagram tells us that the left arrow is also an isomorphism.

% 	Now, because $-\cup u,-\cup u'\colon H^{-2}\colon\widehat H^{-2}(L/K,X_*(T))\to\widehat H^0(L/K,T(L))$ induce the same map along the top row of \autoref{eq:torusdiagram}, and the vertical arrows of \autoref{eq:torusdiagram} are isomorphisms, we see that
% 	\[-\cup u,-\cup u'\colon\widehat H^0(L/K,\ZZ)\to\widehat H^2(L/K,L^\times)\]
% 	must also be the same map (along the bottom row). However, for any $v\in H^2(L/K,L^\times)$, we see that
% 	\[[1]\cup v=v^1=v,\]
% 	so it follows $u=[1]\cup u=[1]\cup u'=u'$. This finishes.
% \end{proof}
% \begin{cor}
% 	Fix notation and in particular the torus $T$ as above. If $u\in H^2(L/K,L^\times)$ induces the same isomorphism of \autoref{thm:torusreciprocity} via the cup-product map $-\cup u$, then $u=u_{L/K}$.
% \end{cor}
% \begin{proof}
% 	We apply \autoref{thm:goodcuppairing} with $c$ set to be $[\delta(\overline c)]$; notably,
% 	\[[\delta(\overline c)]\cup-\colon\widehat H^0(L/K,T(L))\to\widehat H^2(L/K,L^\times)\]
% 	is an isomorphism by combining \autoref{prop:alternativetupleclass} with \autoref{thm:yesitisacocycle}. Thus, \autoref{thm:goodcuppairing} applies and achieves the result upon setting $u'\coloneqq u_{L/K}$.
% \end{proof}
% \begin{remark}
% 	The above corollary is false when $T=\mathbb G_m$ and $G$ is non-cyclic. What makes our $T$ special is that we have some $[\delta(\overline c)]\in H^2(L/K,X^*(T))$ such that
% 	\[[\delta(\overline c)]\cup-\colon\widehat H^0(L/K,T(L))\to\widehat H^2(L/K,L^\times)\]
% 	is an isomorphism. No such element exists when $T=\mathbb G_m$ and $G$ is non-cyclic because the above groups need not even be isomorphic!
% \end{remark}

\subsection{Algebraic Corollaries}
We continue in the set-up of the previous subsection. Observe that \autoref{prop:alternativetupleclass} combined with \autoref{thm:classisomorphism} tells us that we have isomorphisms
\[\delta(\overline c)\cup-\colon\widehat H^0(G,\op{Hom}_\ZZ(X,A))\to\widehat H^2(G,A).\]
In fact, \autoref{lem:cuppingisnatural} tells us that these isomorphisms assemble into a natural isomorphism, so we have the following result.
\begin{theorem}
	Fix everything as in the set-up. Then $X$ is a $2$-encoding module.
\end{theorem}
\begin{proof}
	This follows from the above discussion.
\end{proof}
\begin{remark}
	It is perhaps useful to note that we can show that $X$ is a $2$-encoding module, without the need to digress to tuples as done in \autoref{prop:alternativetupleclass}. Indeed, we recall that
	\[0\to X\to\ZZ[G]^m\to\coker\mathcal F\to0\]
	splits by \autoref{lem:sessplits}, so because $\ZZ[G]^m\cong\ZZ[G]\otimes_\ZZ\ZZ^m$ is induced, it suffices to show that $\coker\mathcal F$ is a $1$-encoding module by \autoref{prop:encodingses}.
	
	For this, we can use \autoref{prop:intdualelement} and manually give $x$ and $x^\lor$; here, $x=[\overline c]$ will work, and one can solve for $x^\lor$. Alternatively, one could check the cohomology groups from \autoref{prop:finitecohomcheck}. One could even solve for $[\delta(\overline c)]^\lor$ explicitly, though this is harder.
\end{remark}
Now that we have a $2$-encoding module, we can apply all the theory we built in \autoref{sec:crackpot}. For example, it might have felt like magic that the isomorphism sending a tuple to its cohomology class was induced by a cup product, but in fact this must have been true all along by \autoref{prop:encodingsarecups}.

Here are some other results.
\begin{cor}
	Fix everything as in the set-up. Then $X$ is cohomologically equivalent to $I_G\otimes_\ZZ I_G$.
\end{cor}
\begin{proof}
	We know that $I_G\otimes_\ZZ I_G$ is a $2$-encoding module by \autoref{ex:igisencoding}, from which \autoref{prop:encodingmodules} finishes.
\end{proof}
\begin{cor}
	Fix everything as in the set-up. Then, for any $i\in\ZZ$ and subgroup $H\subseteq G$, we have natural isomorphisms
	\[\op{Res}[\delta(\overline c)]\cup-\colon\widehat H^i(H,\op{Hom}_\ZZ(X,A))\to\widehat H^{i+2}(H,A).\]
\end{cor}
\begin{proof}
	Follow the proof of \autoref{cor:betterencodingdef} to see that we can set $x=[\delta(\overline c)]$ there. This gives the result for $H=G$, and we get general subgroups by appealing to \autoref{cor:encodingsubgroups}.
\end{proof}
\begin{remark}
	Even though we have some notion of restriction, writing a ``tuple'' in $\widehat H^0(H,\op{Hom}_\ZZ(X,A))$ seems somewhat difficult in general. For example, it is not clear how to (in general) write $X$ as $\ZZ[H]^m/M$ for an $H$-module $M$. In simple cases, we have worked this out in \autoref{lem:restricttuple}.
\end{remark}
\begin{cor}
	Fix everything as in the set-up. Then $\widehat H^2(G,X)$ is cyclic of order $\#G$ generated by $[\delta(\overline c)]$.
\end{cor}
\begin{proof}
	This follows from \autoref{cor:hpxcomputation}.
\end{proof}
\begin{remark}
	Fix notation as in \autoref{sec:overview}, and take $m=2$. Then there are natural transformations
	\[\widehat H^2(G,-)\stackrel{[\delta(\overline c)]^\lor\cup-}\Rightarrow\widehat H^0(G,\op{Hom}_\ZZ(X,-))\Rightarrow\widehat H^{-1}(G,-)\]
	sending a $2$-cocycle to its $\{\sigma_i\}_{i=1}^m$-tuple and then to the (class of) $\beta_{10}$. (It turns out that, because $G$ is bicyclic, the equivalence relation on $\beta_{10}$ is exactly what we need to form a class of $\widehat H^{-1}$.) Now, applying \autoref{prop:encodingsarecups}, we see that the right natural transformation must be a cup-product map, so by associativity of the cup product, the entire natural transformation is a cup-product map.

	Thus, analogously to what \autoref{cor:alphaiscupproduct} says for $\alpha$s, we can describe the projection from $2$-cocycles to $\beta$s purely via (restricted) cup products.
\end{remark}
\begin{remark}
	Noting that $\mathcal F\colon X\into\ZZ[G]^m$ implies that $X$ is $\ZZ$-free, there is a torus $T\coloneqq\op{Hom}_\ZZ(X,\mathbb G_m)$. It is conceivable that one could realize the approach of \autoref{rem:artinreciptaste} for our torus $T$.
\end{remark}
% \begin{remark}
% 	One might be surprised that the equivalence class for tuples appeared when we moved from $H^0(G,\op{Hom}_\ZZ(X,-))$ to $\widehat H^0(G,\op{Hom}_\ZZ(X,-))$. In fact, one could see this without magic: with some effort, one can show that $\widehat H^1(G,\coker\mathcal F)=\ZZ/\#G\ZZ$ by direct classification of $1$-cocycles, yielding
% 	\[\widehat H^2(G,X)=\ZZ/\#G\ZZ.\]
% 	From this it follows that a generator $x\in\widehat H^2(G,X)$ must give rise to an isomorphism
% 	\[x\cup-\colon\widehat H^0(G,\op{Hom}_\ZZ(X,-))\Rightarrow\widehat H^2(G,-).\]
% \end{remark}

\printbibliography[title={References}]

\newpage
\appendix
\section{Verification of the Cocycle} \label{sec:verifycocycle}
% !TEX root = ../abeliangerbs.tex

In this section, we verify \autoref{thm:getcocycle}. As such, in this section, we will work under the modified set-up, forgetting about the extension $\mc E$ but letting $(\{\alpha_i\},\{\beta_{ij}\})$ be some $\{\sigma_i\}_{i=1}^m$-tuple.

Here the formula looks like
\[c(g,g')\coloneqq\left[\prod_{1\le j<i\le m}\Bigg(\prod_{1\le k<j}\sigma_k^{a_k+b_k}\Bigg)\Bigg(\prod_{j\le k<i}\sigma_k^{a_k}\Bigg)\beta_{ij}^{(a_ib_j)}\right]\left[\prod_{i=1}^m\Bigg(\prod_{1\le k<i}\sigma_k^{a_k+b_k}\Bigg)\alpha_i^{\floor{\frac{a_i+b_i}{n_i}}}\right],\]
where $g=\prod_i\sigma_i^{a_i}$ and $g'=\prod_i\sigma_i^{b_i}$ with $0\le a_i,b_i<n_i$ and $q_i\coloneqq\floor{(a_i+b_i)/n_i}$. To make this more digestible, we define
\[g_i\coloneqq\prod_{1\le k<i}\sigma_k^{a_k}\]
for any $g=\prod_i\sigma_i^{a_i}\in G$, so we can write down our formula as
\[c(g,g')\coloneqq\left[\prod_{1\le j<i\le m}g_ig'_j\beta_{ij}^{(a_ib_j)}\right]\left[\prod_{i=1}^mg_ig'_i\alpha_i^{\floor{\frac{a_i+b_i}{n_i}}}\right].\]
Now, given $g,g',g''\in G$, we would like to check
\[gc(g',g'')\cdot c(g,g'g'')\stackrel?=c(gg',g'')\cdot c(g,g'),\]
where $g=\prod_i\sigma_i^{a_i}$ and $g'=\prod_i\sigma_i^{b_i}$ and $g''=\prod_i\sigma_i^{c_i}$ with $0\le a_i,b_i,c_i<n_i$.

\subsection{Carries}
We will begin our verification by dealing with carries; we start with the following lemma, intended to beef up our relation \autoref{eq:tuplerelations}.
\begin{lemma}
	Given indices $i>j$ with $a_i,a_j,q_i,q_j\ge0$, we have
	\[\beta_{ij}^{(a_ia_j)}=\beta_{ij}^{(a_i+q_in_i,a_j)}\left(\frac{\sigma_j^{a_j}(\alpha_i)}{\alpha_i}\right)^{q_i}\qquad\text{and}\qquad\beta_{ij}^{(a_ia_j)}=\beta_{ij}^{(a_i,a_j+q_jn_j)}\left(\frac{\alpha_j}{\sigma_i^{a_i}(\alpha_j)}\right)^{q_j}.\]
\end{lemma}
\begin{proof}
	This is a matter of force. For one, we compute
	\begin{align*}
		\beta_{ij}^{(a_i+n_iq_i,a_j)} &= \prod_{p=0}^{a_i+n_iq_i-1}\prod_{q=0}^{a_j-1}\sigma_i^p\sigma_j^q\beta_{ij} \\
		&= \left(\prod_{p=0}^{a_i-1}\prod_{q=0}^{a_j-1}\sigma_i^p\sigma_j^q\beta_{ij}\right)\left(\prod_{q=0}^{a_j-1}\prod_{p=a_i}^{a_i+n_iq_i-1}\sigma_i^p\sigma_j^q\beta_{ij}\right) \\
		&= \beta_{ij}^{(a_ia_j)}\left(\prod_{q=0}^{a_j-1}\sigma_j^q\op N_{L/L_i}(\beta_{ij})\right)^{q_i}.
	\end{align*}
	Now, using the relation $\op N_{L/L_i}(\beta_{ij})=\alpha_i/\sigma_j(\alpha_i)$ from \autoref{eq:tuplerelations}, this becomes
	\begin{align*}
		\beta_{ij}^{(a_i+n_iq_i,a_j)} &= \beta_{ij}^{(a_ia_j)}\left(\prod_{q=0}^{a_j-1}\frac{\sigma_j^q\alpha_i}{\sigma^{j+1}\alpha_i}\right)^{q_i} \\
		&= \beta_{ij}^{(a_ia_j)}\left(\frac{\alpha_i}{\sigma^{a_j}\alpha_i}\right)^{q_i},
	\end{align*}
	which rearranges into what we wanted.

	For the other, we again just compute
	\begin{align*}
		\beta_{ij}^{(a_i,a_j+n_jq_j)} &= \prod_{p=0}^{a_i-1}\prod_{q=0}^{a_j+n_jq_j-1}\sigma_i^p\sigma_j^q\beta_{ij} \\
		&= \left(\prod_{p=0}^{a_i-1}\prod_{q=0}^{a_j-1}\sigma_i^p\sigma_j^q\beta_{ij}\right)\left(\prod_{p=0}^{a_i-1}\prod_{q=q_j}^{a_j+n_jq_j-1}\sigma_i^p\sigma_j^q\beta_{ij}\right) \\
		&= \beta_{ij}^{(a_ia_j)}\left(\prod_{p=0}^{a_i-1}\sigma_i^p\op N_{L/L_q}(\beta_{ij})\right)^{q_i}.
	\end{align*}
	This time, we use the relation $\op N_{L/L_j}(\beta_{ij})=\sigma_i(\alpha_j)/\alpha_j$, which gives
	\begin{align*}
		\beta_{ij}^{(a_i,a_j+n_jq_j)} &= \beta_{ij}^{(a_ia_j)}\left(\prod_{p=0}^{a_i-1}\frac{\sigma_i^{p+1}(\alpha_j)}{\sigma_i^p(\alpha_j)}\right)^{q_i} \\
		&= \beta_{ij}^{(a_ia_j)}\left(\frac{\sigma_i^{a_j}(\alpha_j)}{\alpha_j}\right)^{q_i},
	\end{align*}
	which again rearranges into the desired.
\end{proof}
We are now ready to begin the computation, dealing with carries to start. Use the division algorithm to write
\[a_i+b_i=n_iu_i+x_i\qquad\text{and}\qquad b_i+c_i=n_iv_i+y_i,\]
where $u_i,v_i\in\{0,1\}$ and $0\le x_i,y_i<n_i$ for each $i$. We start by collecting remainder terms on the side of $gc(g',g'')\cdot c(g,g'g'')$.
\begin{enumerate}
	\item Note
	\begin{align*}
		gc(g',g'') &= g\left[\prod_{1\le j<i\le m}g_i'g''_j\beta_{ij}^{(b_ic_j)}\right]\cdot g\left[\prod_{i=1}^mg'_ig''_i\alpha_i^{v_i}\right],
	\end{align*}
	so we set
	\[R_1\coloneqq\prod_{i=1}^mgg'_ig''_i\alpha_i^{v_i}\]
	to be our remainder term.
	\item Note
	\begin{align*}
		c(g,g'g'') &= \left[\prod_{1\le j<i\le m}g_ig'_jg''_j\beta_{ij}^{(a_iy_j)}\right]\left[\prod_{i=1}^mg_ig'_ig''_i\alpha_i^{\floor{\frac{a_i+y_i}{n_i}}}\right] \\
		&= \left[\prod_{1\le j<i\le m}g_ig'_jg''_j\beta_{ij}^{(a_i,b_j+c_j)}\cdot g_ig'_jg''_j\left(\frac{\alpha_j}{\sigma_i^{a_i}\alpha_j}\right)^{v_i}\right]\left[\prod_{i=1}^mg_ig'_ig''_i\alpha_i^{\floor{\frac{a_i+y_i}{n_i}}}\right] \\
		&= \left[\prod_{1\le j<i\le m}g_ig'_jg''_j\beta_{ij}^{(a_i,b_j+c_j)}\right]\left[\prod_{1\le j<i\le m}g_ig'_jg''_j\left(\frac{\alpha_j}{\sigma_i^{a_i}\alpha_j}\right)^{v_i}\right]\left[\prod_{i=1}^mg_ig'_ig''_i\alpha_i^{\floor{\frac{a_i+y_i}{n_i}}}\right],
	\end{align*}
	so we set
	\[R_2\coloneqq\left[\prod_{1\le j<i\le m}g_ig'_jg''_j\left(\frac{\alpha_j}{\sigma_i^{a_i}\alpha_j}\right)^{v_i}\right]\left[\prod_{i=1}^mg_ig'_ig''_i\alpha_i^{\floor{\frac{a_i+y_i}{n_i}}}\right]\]
	to be our remainder term.
	\item Lastly, we collect our remainders. Observe
	\begin{align*}
		R_2 &= \left[\prod_{j=1}^mg'_jg''_j\Bigg(\prod_{i=j+1}^mg_i\cdot\frac{\alpha_j}{\sigma_i^{a_i}\alpha_j}\Bigg)^{v_i}\right]\left[\prod_{i=1}^mg_ig'_ig''_i\alpha_i^{\floor{\frac{a_i+y_i}{n_i}}}\right] \\
		&= \left[\prod_{j=1}^mg'_jg''_j\Bigg(\prod_{i=j+1}^m\frac{(\sigma_1^{a_1}\cdots\sigma_{i-1}^{a_{i-1}})\alpha_j}{(\sigma_1^{a_1}\cdots\sigma_{i-1}^{a_{i-1}})\sigma_i^{a_i}\alpha_j}\Bigg)^{v_i}\right]\left[\prod_{i=1}^mg_ig'_ig''_i\alpha_i^{\floor{\frac{a_i+y_i}{n_i}}}\right] \\
		&= \left[\prod_{j=1}^mg'_jg''_j\Bigg(\prod_{i=j+1}^m\frac{g_i\alpha_j}{g_{i+1}\alpha_j}\Bigg)^{v_i}\right]\left[\prod_{i=1}^mg_ig'_ig''_i\alpha_i^{\floor{\frac{a_i+y_i}{n_i}}}\right] \\
		&= \left[\prod_{j=1}^mg'_jg''_j\cdot\frac{g_{j+1}\alpha_j^{v_j}}{g\alpha_j^{v_j}}\right]\left[\prod_{i=1}^mg_ig'_ig''_i\alpha_i^{\floor{\frac{a_i+y_i}{n_i}}}\right].
	\end{align*}
	We now note that $g_{j+1}\alpha_j=g_j\alpha_j$ because $\alpha_j$ is fixed by $\sigma_j$. As such,
	\begin{align*}
		R_1R_2 &= \left[\prod_{i=1}^mgg'_ig''_i\alpha_i^{v_i}\right]\left[\prod_{i=1}^mg'_ig''_i\cdot\frac{g_i\alpha_i^{v_i}}{g\alpha_i^{v_i}}\right]\left[\prod_{i=1}^mg_ig'_ig''_i\alpha_i^{\floor{\frac{a_i+y_i}{n_i}}}\right] \\
		&= \prod_{i=1}^mg_ig'_ig''_i\alpha_i^{v_i+\floor{\frac{a_i+y_i}{n_i}}},
	\end{align*}
	which is nice enough for us now.
\end{enumerate}
Now, we collect remainder terms from $c(gg',g'')\cdot c(g,g')$.
\begin{enumerate}
	\item Note
	\begin{align*}
		c(gg',g'') &= \left[\prod_{1\le j<i\le m}g_ig'_ig''_j\beta_{ij}^{(x_ic_j)}\right]\left[\prod_{i=1}^mg_ig'_ig''_i\alpha_i^{\floor{\frac{x_i+c_i}{n_i}}}\right] \\
		&= \left[\prod_{1\le j<i\le m}g_ig'_ig''_j\beta_{ij}^{(a_i+b_i,c_j)}\cdot g_ig'_ig''_j\left(\frac{\sigma_j^{c_j}\alpha_i}{\alpha_i}\right)^{u_i}\right]\left[\prod_{i=1}^mg_ig'_ig''_i\alpha_i^{\floor{\frac{x_i+c_i}{n_i}}}\right] \\
		&= \left[\prod_{1\le j<i\le m}g_ig'_ig''_j\beta_{ij}^{(a_i+b_i,c_j)}\right]\left[\prod_{1\le j<i\le m}g_ig'_ig''_j\left(\frac{\sigma_j^{c_j}\alpha_i}{\alpha_i}\right)^{u_i}\right]\left[\prod_{i=1}^mg_ig'_ig''_i\alpha_i^{\floor{\frac{x_i+c_i}{n_i}}}\right],
	\end{align*}
	so we set
	\[R_3\coloneqq\left[\prod_{1\le j<i\le m}g_ig'_ig''_j\left(\frac{\sigma_j^{c_j}\alpha_i}{\alpha_i}\right)^{u_i}\right]\left[\prod_{i=1}^mg_ig'_ig''_i\alpha_i^{\floor{\frac{x_i+c_i}{n_i}}}\right].\]
	\item Note
	\[c(g,g') = \left[\prod_{1\le j<i\le m}g_ig'_j\beta_{ij}^{(a_ib_j)}\right]\left[\prod_{i=1}^mg_ig'_i\alpha_i^{u_i}\right],\]
	so we set
	\[R_4\coloneqq\left[\prod_{i=1}^mg_ig'_i\alpha_i^{u_i}\right].\]
	\item Lastly, we collect our remainder terms. Observe
	\begin{align*}
		R_3 &= \left[\prod_{i=1}^mg_ig'_i\Bigg(\prod_{j=1}^{i-1}g''_j\cdot\frac{\sigma_j^{c_j}\alpha_i}{\alpha_i}\Bigg)^{u_i}\right]\left[\prod_{i=1}^mg_ig'_ig''_i\alpha_i^{\floor{\frac{x_i+c_i}{n_i}}}\right] \\
		&= \left[\prod_{i=1}^mg_ig'_i\Bigg(\prod_{j=1}^{i-1}\frac{(\sigma_1^{c_1}\cdots\sigma_{j-1}^{c_{j-1}})\sigma_j^{c_j}\alpha_i}{(\sigma_1^{c_1}\cdots\sigma_{j-1}^{c_{j-1}})\alpha_i}\Bigg)^{u_i}\right]\left[\prod_{i=1}^mg_ig'_ig''_i\alpha_i^{\floor{\frac{x_i+c_i}{n_i}}}\right] \\
		&= \left[\prod_{i=1}^mg_ig'_i\Bigg(\prod_{j=1}^{i-1}\frac{g''_{j+1}\alpha_i}{g''_j\alpha_i}\Bigg)^{u_i}\right]\left[\prod_{i=1}^mg_ig'_ig''_i\alpha_i^{\floor{\frac{x_i+c_i}{n_i}}}\right] \\
		&= \left[\prod_{i=1}^mg_ig'_i\cdot\frac{g''_i\alpha_i^{u_i}}{\alpha_i^{u_i}}\right]\left[\prod_{i=1}^mg_ig'_ig''_i\alpha_i^{\floor{\frac{x_i+c_i}{n_i}}}\right].
	\end{align*}
	Thus,
	\begin{align*}
		R_3R_4 &= \left[\prod_{i=1}^mg_ig'_i\cdot\frac{g''_i\alpha_i^{u_i}}{\alpha_i^{u_i}}\right]\left[\prod_{i=1}^mg_ig'_ig''_i\alpha_i^{\floor{\frac{x_i+c_i}{n_i}}}\right]\left[\prod_{i=1}^mg_ig'_i\alpha_i^{u_i}\right] \\
		&= \prod_{i=1}^mg_ig'_ig''_i\alpha_i^{u_i+\floor{\frac{x_i+c_i}{n_i}}},
	\end{align*}
	which is again simple enough for our purposes.
\end{enumerate}
We now note that, for each $i$,
\[u_i+\floor{\frac{x_i+c_i}{n_i}}=\floor{\frac{a_i+b_i+c_i}{n_i}}=v_i+\floor{\frac{a_i+y_i}{n_i}}\]
by how carried addition behaves. It follows that
\[R_1R_2=\prod_{i=1}^mg_ig'_ig''_i\alpha_i^{v_i+\floor{\frac{a_i+y_i}{n_i}}}=\prod_{i=1}^mg_ig'_ig''_i\alpha_i^{u_i+\floor{\frac{x_i+c_i}{n_i}}}=R_3R_4.\]
Thus, it suffices to show that
\[\frac{gc(g',g'')}{R_1}\cdot\frac{c(g,g'')}{R_2}\stackrel?=\frac{c(gg',g'')}{R_3}\cdot\frac{c(g,g')}{R_4},\]
which is equivalent to
\[g\left[\prod_{1\le j<i\le m}g_i'g''_j\beta_{ij}^{(b_ic_j)}\right]\cdot\left[\prod_{1\le j<i\le m}g_ig'_jg''_j\beta_{ij}^{(a_i,b_j+c_j)}\right]\stackrel?=\left[\prod_{1\le j<i\le m}g_ig'_ig''_j\beta_{ij}^{(a_i+b_i,c_j)}\right]\cdot\left[\prod_{1\le j<i\le m}g_ig'_j\beta_{ij}^{(a_ib_j)}\right]\]
by the work above.

\subsection{Finishing}
We need to verify that
\[g\left[\prod_{1\le j<i\le m}g_i'g''_j\beta_{ij}^{(b_ic_j)}\right]\cdot\left[\prod_{1\le j<i\le m}g_ig'_jg''_j\beta_{ij}^{(a_i,b_j+c_j)}\right]\stackrel?=\left[\prod_{1\le j<i\le m}g_ig'_ig''_j\beta_{ij}^{(a_i+b_i,c_j)}\right]\cdot\left[\prod_{1\le j<i\le m}g_ig'_j\beta_{ij}^{(a_ib_j)}\right]\]
as discussed in the previous subsection.

Before beginning the check, we recall the relations on the $\beta$s from \autoref{eq:betarelations} can be written as
\[\frac{\sigma_2(\beta_{31})}{\beta_{31}}=\frac{\sigma_1(\beta_{32})}{\beta_{32}}\cdot\frac{\sigma_3(\beta_{21})}{\beta_{21}},\]
because we only have one triple $(i,j,k)$ of indices with $i>j>k$. This is somewhat difficult to deal with directly, so we quickly show a more general version.
\begin{lemma} \label{lem:betterbetarelation}
	Fix indices with $i>j>k$, and let $a_i,a_j,a_k\ge0$. Then
	\[\frac{\sigma_j^{a_j}\beta_{ik}^{(a_ia_k)}}{\beta_{ik}^{(a_ia_k)}}=\frac{\sigma_k^{a_k}\beta_{ij}^{(a_ia_j)}}{\beta_{ij}^{(a_ia_j)}}\cdot\frac{\sigma_i^{a_i}\beta_{jk}^{(a_ja_k)}}{\beta_{jk}^{(a_ja_k)}}.\]
\end{lemma}
\begin{proof}
	We simply compute
	\begin{align*}
		\frac{\sigma_i^{a_i}\beta_{jk}^{(a_ja_k)}}{\beta_{jk}^{(a_ja_k)}}\cdot\frac{\sigma_k^{a_k}\beta_{ij}^{(a_ia_j)}}{\beta_{ij}^{(a_ia_j)}} &= \prod_{r=0}^{a_i-1}\frac{\sigma_i^{r+1}\beta_{jk}^{(a_ja_k)}}{\sigma_i^r\beta_{jk}^{(a_ja_k)}}\cdot\prod_{p=0}^{a_k-1}\frac{\sigma_k^{p+1}\beta_{ij}^{(a_ia_j)}}{\sigma_k^p\beta_{ij}^{(a_ia_j)}} \\
		&= \prod_{p=0}^{a_k-1}\prod_{q=0}^{a_j-1}\prod_{r=0}^{a_i-1}\left(\frac{\sigma_k^p\sigma_j^q\sigma_i^{r+1}\beta_{jk}}{\sigma_k^p\sigma_j^q\sigma_i^r\beta_{jk}}\cdot\frac{\sigma_k^{p+1}\sigma_j^q\sigma_i^r\beta_{ij}}{\sigma_k^p\sigma_j^q\sigma_i^r\beta_{ij}}\right) \\
		&= \prod_{p=0}^{a_k-1}\prod_{q=0}^{a_j-1}\prod_{r=0}^{a_i-1}\sigma_k^p\sigma_j^q\sigma_i^r\left(\frac{\sigma_i\beta_{jk}}{\beta_{jk}}\cdot\frac{\sigma_k\beta_{ij}}{\beta_{ij}}\right) \\
		&= \prod_{p=0}^{a_k-1}\prod_{q=0}^{a_j-1}\prod_{r=0}^{a_i-1}\sigma_k^p\sigma_j^q\sigma_i^r\left(\frac{\sigma_j\beta_{ik}}{\beta_{ik}}\right),
	\end{align*}
	where in the last equality we have use the relation on the $\beta$s. Continuing,
	\begin{align*}
		\frac{\sigma_i^{a_i}\beta_{jk}^{(a_ja_k)}}{\beta_{jk}^{(a_ja_k)}}\cdot\frac{\sigma_k^{a_k}\beta_{ij}^{(a_ia_j)}}{\beta_{ij}^{(a_ia_j)}} &= \prod_{q=0}^{a_j-1}\left(\prod_{p=0}^{a_k-1}\prod_{r=0}^{a_i-1}\frac{\sigma_j^{q+1}\sigma_k^p\sigma_i^r\beta_{ik}}{\sigma_j^q\sigma_k^p\sigma_i^r\beta_{ik}}\right) \\
		&= \prod_{q=0}^{a_j-1}\frac{\sigma_j^{q+1}\beta_{ik}^{(a_ia_k)}}{\sigma_j^q\beta_{ik}^{(a_ia_k)}} \\
		&= \frac{\sigma_j^{a_j}\beta_{ik}^{(a_ia_k)}}{\beta_{ik}^{(a_ia_k)}},
	\end{align*}
	which is what we wanted.
\end{proof}
We now proceed with the check, by induction. More precisely, we claim that any $m'\le m$ gives
\[g_{m'+1}\left[\prod_{j<i\le m'}g_i'g''_j\beta_{ij}^{(b_ic_j)}\right]\left[\prod_{j<i\le m'}g_ig'_jg''_j\beta_{ij}^{(a_i,b_j+c_j)}\right]\stackrel?=\left[\prod_{j<i\le m'}g_ig'_ig''_j\beta_{ij}^{(a_i+b_i,c_j)}\right]\left[\prod_{j<i\le m'}g_ig'_j\beta_{ij}^{(a_ib_j)}\right]\]
which we will show by induction on $m'$. For $m'=1$, there is nothing to say because there are no indices $i>j$.

So now suppose we have equality for $m'<m$, and we give equality for $m''\coloneqq m'+1$. That is, we want to show that
\[g_{m'+2}\prod_{j<i\le m'+1}g_i'g''_j\beta_{ij}^{(b_ic_j)}\cdot\prod_{j<i\le m'+1}g_ig'_jg''_j\beta_{ij}^{(a_i,b_j+c_j)}\stackrel?=\prod_{j<i\le m'+1}g_ig'_ig''_j\beta_{ij}^{(a_i+b_i,c_j)}\cdot\prod_{j<i\le m'+1}g_ig'_j\beta_{ij}^{(a_ib_j)}\]
but by the inductive hypothesis it suffices for
\[\frac{\displaystyle g_{m''+1}\prod_{j<i\le m'+1}g_i'g''_j\beta_{ij}^{(b_ic_j)}}{\displaystyle g_{m'+1}\prod_{j<i\le m'}g_i'g''_j\beta_{ij}^{(b_ic_j)}}\cdot
\frac{\displaystyle\prod_{j<i\le m'+1}g_ig'_jg''_j\beta_{ij}^{(a_i,b_j+c_j)}}{\displaystyle\prod_{j<i\le m'}g_ig'_jg''_j\beta_{ij}^{(a_i,b_j+c_j)}}
\stackrel?=
\frac{\displaystyle\prod_{j<i\le m'+1}g_ig'_ig''_j\beta_{ij}^{(a_i+b_i,c_j)}}{\displaystyle\prod_{j<i\le m'}g_ig'_ig''_j\beta_{ij}^{(a_i+b_i,c_j)}}\cdot
\frac{\displaystyle\prod_{j<i\le m'+1}g_ig'_j\beta_{ij}^{(a_ib_j)}}{\displaystyle\prod_{j<i\le m'}g_ig'_j\beta_{ij}^{(a_ib_j)}}\]
which is collapses to
\[\frac{\displaystyle g_{m''+1}\prod_{j<i\le m'+1}g_i'g''_j\beta_{ij}^{(b_ic_j)}}{\displaystyle g_{m'+1}\prod_{j<i\le m'}g_i'g''_j\beta_{ij}^{(b_ic_j)}}\cdot
\prod_{j\le m'}g_{m''}g'_jg''_j\beta_{m''j}^{(a_{m''},b_j+c_j)}
\stackrel?=
\prod_{j\le m'}g_{m''}g'_{m''}g''_j\beta_{m''j}^{(a_{m''}+b_{m''},c_j)}\cdot
\displaystyle\prod_{j\le m'}g_{m''}g'_j\beta_{ij}^{(a_{m''}b_j)}\]
because the terms with $i<m''=m'+1$ got cancelled in the rightmost three products. Rearranging, this is the same as
\[\frac{\displaystyle g_{m''+1}\prod_{j<i\le m'+1}g_i'g''_j\beta_{ij}^{(b_ic_j)}}{\displaystyle g_{m'+1}\prod_{j<i\le m'}g_i'g''_j\beta_{ij}^{(b_ic_j)}}
\stackrel?=
\frac{\displaystyle\prod_{j<m''}g_{m''}g'_{m''}g''_j\beta_{m''j}^{(a_{m''}+b_{m''},c_j)}\cdot
\displaystyle\prod_{j<m''}g_{m''}g'_j\beta_{m''j}^{(a_{m''}b_j)}}
{\displaystyle\prod_{j<m''}g_{m''}g'_jg''_j\beta_{m''j}^{(a_{m''},b_j+c_j)}}.\]
Peeling off the $i=m''=m'+1$ terms from the left-hand side numerator, we're showing
\[\frac{\displaystyle g_{m''+1}\prod_{j<i\le m'}g_i'g''_j\beta_{ij}^{(b_ic_j)}}{\displaystyle g_{m'+1}\prod_{j<i\le m'}g_i'g''_j\beta_{ij}^{(b_ic_j)}}
\stackrel?=
\frac{\displaystyle\prod_{j<m''}g_{m''}g'_{m''}g''_j\beta_{m''j}^{(a_{m''}+b_{m''},c_j)}\cdot
\displaystyle\prod_{j<m''}g_{m''}g'_j\beta_{m''j}^{(a_{m''}b_j)}}
{\displaystyle\prod_{j<m''}g_{m''+1}g_{m''}'g''_j\beta_{m''j}^{(b_{m''}c_j)}\cdot
\prod_{j<m''}g_{m''}g'_jg''_j\beta_{m''j}^{(a_{m''},b_j+c_j)}}.\]
We take a moment to simplify the left-hand side with \autoref{lem:betterbetarelation} by writing
\begin{align*}
	g_{m'+1}\prod_{j<i\le m'}g_i'g''_j\left(\frac{\sigma_{m''}^{a_{m''}}\beta_{ij}^{(b_ic_j)}}{\beta_{ij}^{(b_ic_j)}}\right) &= g_{m''}\prod_{j<i\le m'}g_i'g''_j\left(\frac{\sigma_i^{b_i}\beta_{m''j}^{(a_{m''}c_j)}}{\beta_{m''j}^{(a_{m''}c_j)}}\cdot\frac{\beta_{m''i}^{(a_{m''}b_i)}}{\sigma_j^{c_j}\beta_{m''i}^{(a_{m''}b_i)}}\right) \\
	&= g_{m''}\left[\prod_{j=1}^{m'}g''_j\prod_{i=j+1}^{m'}g_i'\left(\frac{\sigma_i^{b_i}\beta_{m''j}^{(a_{m''}c_j)}}{\beta_{m''j}^{(a_{m''}c_j)}}\right)\cdot
	\prod_{i=1}^{m'}g_i'\prod_{j=1}^{i-1}g_j''\left(\frac{\beta_{m''i}^{(a_{m''}b_i)}}{\sigma_j^{c_j}\beta_{m''i}^{(a_{m''}b_i)}}\right)\right] \\
	&= g_{m''}\left[\prod_{j=1}^{m'}\frac{g'_{m'+1}g''_j\beta_{m''j}^{(a_{m''}c_j)}}{g'_{j+1}g''_j\beta_{m''j}^{(a_{m''}c_j)}}\cdot
	\prod_{i=1}^{m'}\frac{g_i'\beta_{m''i}^{(a_{m''}b_i)}}{g_i'g_i''\beta_{m''i}^{(a_{m''}b_i)}}\right] \\
	&= g_{m''}\left[\prod_{j<m''}\frac{g'_{m''}g''_j\beta_{m''j}^{(a_{m''}c_j)}}{g'_{j+1}g''_j\beta_{m''j}^{(a_{m''}c_j)}}\cdot
	\prod_{j<m''}\frac{g_j'\beta_{m''j}^{(a_{m''}b_j)}}{g_j'g_j''\beta_{m''j}^{(a_{m''}b_j)}}\right]
\end{align*}
after doing a lot of telescoping. Now, we can remove $g_{m''}$ everywhere to give
\[\prod_{j<m''}\frac{g'_{m''}g''_j\beta_{m''j}^{(a_{m''}c_j)}}{g'_{j+1}g''_j\beta_{m''j}^{(a_{m''}c_j)}}\cdot
\prod_{j<m''}\frac{g_j'\beta_{m''j}^{(a_{m''}b_j)}}{g_j'g_j''\beta_{m''j}^{(a_{m''}b_j)}}
\stackrel?=
\frac{\displaystyle\prod_{j<m''}g'_{m''}g''_j\beta_{m''j}^{(a_{m''}+b_{m''},c_j)}\cdot
\displaystyle\prod_{j<m''}g'_j\beta_{m''j}^{(a_{m''}b_j)}}
{\displaystyle\prod_{j<m''}g'_{m''+1}g''_j\beta_{m''j}^{(b_{m''}c_j)}\cdot
\prod_{j<m''}g'_jg''_j\beta_{m''j}^{(a_{m''},b_j+c_j)}},\]
or
\[\prod_{j<m''}\frac{g'_{m''}g''_j\beta_{m''j}^{(a_{m''}c_j)}}{g'_{j+1}g''_j\beta_{m''j}^{(a_{m''}c_j)}}
\stackrel?=
\frac{\displaystyle\prod_{j<m''}g'_{m''}g''_j\beta_{m''j}^{(a_{m''}+b_{m''},c_j)}\cdot
\displaystyle\prod_{j<m''}g'_jg''_j\beta_{m''j}^{(a_{m''}b_j)}}
{\displaystyle\prod_{j<m''}g'_{m''+1}g''_j\beta_{m''j}^{(b_{m''}c_j)}\cdot
\prod_{j<m''}g'_jg''_j\beta_{m''j}^{(a_{m''},b_j+c_j)}}.\]
Rearranging, we want
\[\prod_{j<m''}
\frac{g'_jg''_j\beta_{m''j}^{(a_{m''},b_j+c_j)}}
{g'_jg''_j\beta_{m''j}^{(a_{m''}b_j)}\cdot
g'_{j+1}g''_j\beta_{m''j}^{(a_{m''}c_j)}}
\stackrel?=\prod_{j<m''}
\frac{g'_{m''}g''_j\beta_{m''j}^{(a_{m''}+b_{m''},c_j)}}
{g'_{m''}g''_j\beta_{m''j}^{(a_{m''}c_j)}\cdot
g'_{m''+1}g''_j\beta_{m''j}^{(b_{m''}c_j)}},\]
which is
\[\prod_{j<m''}
g'_jg''_j\left(\frac{\beta_{m''j}^{(a_{m''},b_j+c_j)}}
{\beta_{m''j}^{(a_{m''}b_j)}\cdot
\sigma_j^{b_j}\beta_{m''j}^{(a_{m''}c_j)}}\right)
\stackrel?=\prod_{j<m''}
g'_{m''}g''_j\left(\frac{\beta_{m''j}^{(a_{m''}+b_{m''},c_j)}}
{\beta_{m''j}^{(a_{m''}c_j)}\cdot
\sigma_{m''}^{a_{m''}}\beta_{m''j}^{(b_{m''}c_j)}}\right).\]
However, by definition of the $\beta_{ij}^{(xy)}$, we see that
\[\frac{\beta_{m''j}^{(a_{m''},b_j+c_j)}}
{\beta_{m''j}^{(a_{m''}b_j)}\cdot
\sigma_j^{b_j}\beta_{m''j}^{(a_{m''}c_j)}}=\frac{\beta_{m''j}^{(a_{m''}+b_{m''},c_j)}}
{\beta_{m''j}^{(a_{m''}c_j)}\cdot
\sigma_{m''}^{a_{m''}}\beta_{m''j}^{(b_{m''}c_j)}}=1,\]
so everything does indeed cancel out properly. This completes the check.

\section{Computation of \texorpdfstring{$\ker\mathcal F$}{ker F}} \label{sec:havegensproof}
% !TEX root = ../abeliangerbs.tex

In this section we give a proof of \autoref{lem:havegens}. As such, we will use all the context from the statement and proceed directly with the proof; as mentioned earlier, we may add (b) back to our list of generators because it is induced by (c). Pick up some $z\coloneqq((x_i)_i,(y_{ij})_{i>j})\in\ker\mathcal F$, which is equivalent to saying
\[x_iN_i-\sum_{j=1}^{i-1}y_{ij}T_j+\sum_{j=i+1}^my_{ji}T_j=0\]
for each index $i$. We want to write $z$ as a $\ZZ[G]$-linear combination of the elements from (a)--(e). The main idea will be to slowly subtract out $\ZZ[G]$-linear combinations of the above elements (which does not affect $z\in\ker\mathcal F$) until we can prove that we have $0$ left over. We start with the $x_i$ terms, which we do in two steps.
\begin{enumerate}
	\item We begin by dealing with the $x_i$ terms. Fix some index $p$, and we will subtract out a suitable $\ZZ[G]$-linear combination of the above generators to set $x_p=0$ while not changing the other $x_i$ terms. Well, using the element
	\[\kappa_pT_p,\tag{a}\]
	we may assume that $x_p$ has no $\sigma_p$ terms because $\sigma_p\equiv1\pmod{T_p}$. Then for each $q<p$, we can subtract out a suitable multiple of
	\[T_q\kappa_p+N_p\lambda_{pq}\tag{c}\]
	to make it so that we may assume $x_p$ has no $\sigma_q$ terms because $\sigma_q\equiv1\pmod{T_q}$. Similarly, for each $q>p$, we can subtract out a suitable multiple of
	\[T_q\kappa_p-N_p\lambda_{pq}\tag{d}\]
	to make it so that we may assume $x_p$ has no $\sigma_q$ terms because $\sigma_q\equiv1\pmod{T_q}$.

	\item Thus, the above process allows us to assume that $x_p\in\ZZ$, and the above linear combinations have not affected any $x_i$ for $i\ne p$. We now use the fact that $z\in\ker\mathcal F$. Indeed, we know that
	\[x_pN_p-\sum_{j=1}^{p-1}y_{pj}T_j+\sum_{j=p+1}^my_{jp}T_j=0.\]
	Applying the augmentation map $\varepsilon\colon\ZZ[G]\to\ZZ$, sending $\varepsilon\colon\sigma_i\mapsto1$ for each index $i$, we see that $x_p\in\ZZ$ implying that $x_p$ remains fixed. On the other hand $\varepsilon\colon T_j\mapsto0$ for each index $j$ and $\varepsilon\colon N_p\mapsto n_p$, so we are left with
	\[n_px_p=0.\]
	Because $n_p\ne0$ (it's the order of $\sigma_p$), we conclude that $x_p=0$. Applying this argument to the other $x_i$ terms, we conclude that we may assume $x_i=0$ for each $i$.
\end{enumerate}
It remains to deal with the $y_{ij}$ terms, which is a little more involved. For reference, we are showing that
\[-\sum_{j=1}^{i-1}y_{ij}T_j+\sum_{j=i+1}^my_{ji}T_j=0\]
for each index $i$ implies that $z=((0)_i,(y_{ij})_{i>j})$ is a $\ZZ[G]$-linear combination of the terms from (b) and (e).

We will now more or less proceed with the $y_{ij}$ by induction on $m$, allowing the group $G$ (in its number of generators $m$) to be changed in the process. For $m=1$, there is nothing to say because there is no $y_{ij}$ term at all. For a taste of how we will use \autoref{lem:separatenijs}, we also work out $m=2$: our equations read
\[\underbrace{-y_{21}T_1=0}_{i=1}\qquad\text{and}\qquad\underbrace{y_{21}T_2=0}_{i=2}.\]
Thus, $y_{21}\in(\ker T_1)\cap(\ker T_2)=(\im N_1)\cap(\im N_2)$, which is $\im N_1N_2$ by \autoref{lem:separatenijs}.

We now proceed with the general case; take $m>2$. Let $G'\coloneqq\langle\sigma_2,\ldots,\sigma_m\rangle$, which has $m-1$ generators. By the inductive hypothesis, we may assume the statement for $G'$. Explicitly, we will assume that, if $(y_{ij}')_{i>j\ge2}\in\ZZ[G']^{\binom{m-1}2}$ are variables satisfying
\[-\sum_{j=2}^{i-1}y_{ij}'T_j+\sum_{j=i+1}^my_{ji}'T_j=0\]
for each index $i\ge2$, then $y_{ij}'$ are a linear combination of terms from the elements from (b) and (e) above, only using indices at least $2$.

We will again proceed in steps, for clarity.
\begin{enumerate}
	\item To apply the inductive hypothesis, we need to force $y_{pq}\in\ZZ[G']$ for each pair of indices $(p,q)$ with $p>q\ge2$. Well, we use the relation (e) so that we can subtract multiples of
	\[T_q\lambda_{p1}-T_1\lambda_{pq}-T_p\lambda_{q1}.\]
	In particular, this element will subtract out $T_1$ from $y_{pq}$ while only introducing chaos to the elements $y_{p1}$ and $y_{q1}$ in the process. Thus, subtracting a suitable multiple allows us to assume that $y_{pq}$ has no $\sigma_1$ terms while not affecting any other $y_{ij}$ with $i>j\ge2$.

	Applying this process to all $y_{ij}$ with $i>j\ge2$, we do indeed get $y_{ij}\in\ZZ[G']$ for each $i>j\ge2$.

	\item We are now ready to apply the inductive hypothesis. For each index $i\ge2$, we have the equation
	\[-y_{i1}T_1-\sum_{j=2}^{i-1}y_{ij}T_j+\sum_{j=i+1}^my_{ji}T_j=0.\]
	Because each $y_{pq}$ term with $p>q\ge2$ features no $\sigma_1$, applying the transformation $\sigma_1\mapsto1$ will affect no term in the sums while causing $y_{i1}T_1$ to vanish. Thus, we have the equations
	\[-\sum_{j=2}^{i-1}y_{ij}T_j+\sum_{j=i+1}^my_{ji}T_j=0\]
	for each index $i\ge2$. Because $y_{ij}\in\ZZ[G']$ for $i>j\ge2$ already, we see that we may apply the inductive hypothesis to assert that the $y_{ij}$ are $\ZZ[G']$-linear combinations of terms from (b) and (e) (only using indices at least $2$).
	
	Subtracting these linear combinations out, we may assume $y_{ij}=0$ for each $i>j\ge2$.

	\item To take stock, our equations for $i\ge2$ now read
	\[-y_{i1}T_1=0,\]
	which simply tells us that $y_{i1}\in\im N_1$ for each $i\ge2$. As such, we pick up $w_i\in\ZZ[G]$ so that $y_{i1}=w_iN_1$ for each $i\ge2$; because $\sigma_1N_1=N_1$, we may assume that $w_i\in\ZZ[G']$ for each $i\ge2$.

	Now the equation for $i=1$ reads
	\[\sum_{j=2}^my_{j1}T_j=0,\]
	or
	\[\sum_{i=2}^mw_iN_1T_i=0.\]
	Sending $\sigma_1\mapsto1$, we see that $w_i$ and $T_i$ are both fixed because they feature no $\sigma_1$s, so we merely have
	\[n_1\sum_{i=2}^mw_iT_i=0.\]
	Dividing out by $n_1$, we are left with
	\[\sum_{i=2}^mw_iT_i=0.\]

	\item At this point, we may appear stuck, but we have one final trick: taking indices $p>q\ge2$, subtracting out multiples of
	\[\big(T_q\lambda_{p1}-T_1\lambda_{pq}-T_p\lambda_{q1}\big)\cdot N_1\]
	will not affect the $y_{pq}$ term because $T_1N_1$. Indeed, subtracting this term out looks like
	\[T_qN_1\lambda_{p1}-T_pN_1\lambda_{q1},\]
	which after factoring out $N_1$ takes $w_p\mapsto w_p-T_q$ and $w_q\mapsto w_q+T_p$.

	In particular, fixing any $q\ge2$ and then applying this trick for all $p>q$, we may assume that $w_q$ does not feature any $\sigma_p$ terms for $p>q$. Thus, looking at our equation
	\[\sum_{i=2}^mw_iT_i=0,\]
	we are now able to show that $w_i\in\ker T_i=\im N_i$ for each $i\ge2$, which will finish because it shows $y_{i1}\in N_iN_1$. Indeed, starting with $i=2$, we see that $w_2$ features no $\sigma_p$ for $p>2$, so we may take $\sigma_p\mapsto1$ for each $p>2$ safely, giving the equation
	\[w_2T_2=0,\]
	finishing for $w_2$. Thus, we are left with the equation
	\[\sum_{i=3}^mw_iT_i=0,\]
	from which we see we can induct downwards (this has fewer variables) to finish.
\end{enumerate}
The above steps complete the proof, as advertised.

\end{document}